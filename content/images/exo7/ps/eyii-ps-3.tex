%%%%%%%%%%%%%%%%%% PREAMBULE %%%%%%%%%%%%%%%%%%

\documentclass[12pt,a4paper]{article}

\usepackage{amsfonts,amsmath,amssymb,amsthm}
\usepackage[francais]{babel}
\usepackage[utf8]{inputenc}
\usepackage[T1]{fontenc}

%----- Ensemles : entiers, reels, complexes -----
\newcommand{\Nn}{\mathbb{N}} \newcommand{\N}{\mathbb{N}}
\newcommand{\Zz}{\mathbb{Z}} \newcommand{\Z}{\mathbb{Z}}
\newcommand{\Qq}{\mathbb{Q}} \newcommand{\Q}{\mathbb{Q}}
\newcommand{\Rr}{\mathbb{R}} \newcommand{\R}{\mathbb{R}}
\newcommand{\Cc}{\mathbb{C}} \newcommand{\C}{\mathbb{C}}

%----- Modifications de symboles -----
\renewcommand {\epsilon}{\varepsilon}
\renewcommand {\Re}{\mathop{\mathrm{Re}}\nolimits}
\renewcommand {\Im}{\mathop{\mathrm{Im}}\nolimits}

%----- Fonctions usuelles -----
\newcommand{\ch}{\mathop{\mathrm{ch}}\nolimits}
\newcommand{\sh}{\mathop{\mathrm{sh}}\nolimits}
\renewcommand{\tanh}{\mathop{\mathrm{th}}\nolimits}
\newcommand{\Arcsin}{\mathop{\mathrm{Arcsin}}\nolimits}
\newcommand{\Arccos}{\mathop{\mathrm{Arccos}}\nolimits}
\newcommand{\Arctan}{\mathop{\mathrm{Arctan}}\nolimits}
\newcommand{\Argsh}{\mathop{\mathrm{Argsh}}\nolimits}
\newcommand{\Argch}{\mathop{\mathrm{Argch}}\nolimits}
\newcommand{\Argth}{\mathop{\mathrm{Argth}}\nolimits}
\newcommand{\pgcd}{\mathop{\mathrm{pgcd}}\nolimits} 

%----- Commandes special dessin a ajouter localement ------
\usepackage{geometry}
\usepackage{pstricks}
\usepackage{pst-plot}
\usepackage{pst-node}
\usepackage{graphics,epsfig}

\pagestyle{empty}

% Que faire avec ce fichier monimage.tex ?
%   1/ latex monimage.tex
%   2/ dvips monimage.dvi
%   3/ ps2eps monimage.ps
%   4/ ps2pdf -dEPSCrop monimage.eps
%   5/ Dans votre fichier d'exos \includegraphics{monimage}

\begin{document}

\begin{center}
\begin{pspicture}(-5,-4)(9,7)
\pscircle(0,0){2.5}
\psdots[linecolor=blue](-1.5,-2)(1.5,-2)(0.5,2.449)(1.38,4.4)(9,-2)(0.8,1.1)
\psline[linestyle=dashed](1.38,4.4)(-1.678,2.855)
\psline[linestyle=dashed](-1.678,2.855)(0.286,1.972)
\psdots(2.25,1.089)(-1.678,2.855)
%\pspolygon[linecolor=blue](-1.5,-2)(1.5,-2)(0.5,2.449)
\psplot[linecolor=blue]{-2.5}{2.5}{x 1.5 add 4.449 mul 2 div 2 sub}
\psplot[linecolor=blue]{-0.5}{2}{x 1.5 sub 4.449 mul neg 2 sub}
\psline[linecolor=blue](-5,-2)(10,-2)
\uput[dr](1.5,-2){\textcolor{blue}{$B$}}
\uput[dl](-1.5,-2){\textcolor{blue}{$A$}}
\uput[u](0.5,2.449){\textcolor{blue}{$C$}}
\uput[ur](2.25,1.089){$F$}
\uput[ur](2.25,-2){\textcolor{red}{$R$}}
\uput[ur](0.875,0.78){\textcolor{red}{$P$}}
\uput[l](0.286,1.972){\textcolor{red}{$Q$}}
\uput[r](1.039,0.491){\textcolor{red}{$S$}}
\psdots[linecolor=red](2.25,-2)(0.875,0.78)(0.286,1.972)(1.039,0.491)
\psline[linecolor=red,linestyle=dashed](2.25,1.089)(2.25,-2)
\psline[linecolor=red,linestyle=dashed](2.25,1.089)(0.875,0.78)
\psline[linecolor=red,linestyle=dashed](2.25,1.089)(0.286,1.972)
\psline[linecolor=red,linestyle=dashed](2.25,1.089)(1.039,0.491)
\psplot[linecolor=red]{-2}{3.2}{0.78 x 0.875 sub 2.78 mul 1.375 div sub}
\psplot[linestyle=dashed]{-3.2}{1.9}{0.107 x 0.172 add 2.78 mul 1.375 div add neg}
\parametricplot[linecolor=red]{-5}{6.5}{0.165 t dup mul mul 0.443 t mul add 1.039 add 
0.081 t dup mul mul 0.897 t mul sub 0.491 add}
\uput[r](2.8,-3){\textcolor{red}{$(T_0)$}}
\end{pspicture}
\end{center}

\end{document}

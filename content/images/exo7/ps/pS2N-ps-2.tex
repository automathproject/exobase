%%%%%%%%%%%%%%%%%% PREAMBULE %%%%%%%%%%%%%%%%%%

\documentclass[12pt,a4paper]{article}

\usepackage{amsfonts,amsmath,amssymb,amsthm}
\usepackage[francais]{babel}
\usepackage[utf8]{inputenc}
\usepackage[T1]{fontenc}

%----- Ensemles : entiers, reels, complexes -----
\newcommand{\Nn}{\mathbb{N}} \newcommand{\N}{\mathbb{N}}
\newcommand{\Zz}{\mathbb{Z}} \newcommand{\Z}{\mathbb{Z}}
\newcommand{\Qq}{\mathbb{Q}} \newcommand{\Q}{\mathbb{Q}}
\newcommand{\Rr}{\mathbb{R}} \newcommand{\R}{\mathbb{R}}
\newcommand{\Cc}{\mathbb{C}} \newcommand{\C}{\mathbb{C}}

%----- Modifications de symboles -----
\renewcommand {\epsilon}{\varepsilon}
\renewcommand {\Re}{\mathop{\mathrm{Re}}\nolimits}
\renewcommand {\Im}{\mathop{\mathrm{Im}}\nolimits}

%----- Fonctions usuelles -----
\newcommand{\ch}{\mathop{\mathrm{ch}}\nolimits}
\newcommand{\sh}{\mathop{\mathrm{sh}}\nolimits}
\renewcommand{\tanh}{\mathop{\mathrm{th}}\nolimits}
\newcommand{\Arcsin}{\mathop{\mathrm{Arcsin}}\nolimits}
\newcommand{\Arccos}{\mathop{\mathrm{Arccos}}\nolimits}
\newcommand{\Arctan}{\mathop{\mathrm{Arctan}}\nolimits}
\newcommand{\Argsh}{\mathop{\mathrm{Argsh}}\nolimits}
\newcommand{\Argch}{\mathop{\mathrm{Argch}}\nolimits}
\newcommand{\Argth}{\mathop{\mathrm{Argth}}\nolimits}
\newcommand{\pgcd}{\mathop{\mathrm{pgcd}}\nolimits} 

%----- Commandes special dessin a ajouter localement ------
\usepackage{geometry}
\usepackage{pstricks}
\usepackage{pst-plot}
\usepackage{pst-node}
\usepackage{graphics,epsfig}

\pagestyle{empty}

% Que faire avec ce fichier monimage.tex ?
%   1/ latex monimage.tex
%   2/ dvips monimage.dvi
%   3/ ps2eps monimage.ps
%   4/ ps2pdf -dEPSCrop monimage.eps
%   5/ Dans votre fichier d'exos \includegraphics{monimage}

\begin{document}

\begin{center}
\begin{tabular}{|c|lcrclcccccr|}
\hline
$\theta$&$-\pi$& & &$-\frac{5\pi}{6}$& &$-\frac{2\pi}{3}$& &$-\frac{\pi}{6}$& &$\frac{2\pi}{3}$&$\pi$
\tabularnewline
\hline
$r'(\theta)$& &$-$& &\begin{pspicture}(0,0)(0,0)
\psline(0,-1.43)(0,0.3)
\psline(0.2,-1.43)(0.2,0.3)\end{pspicture}& &$-$& &\begin{pspicture}(0,0)(0,0)
\psline(0,-1.43)(0,0.3)
\psline(0.2,-1.43)(0.2,0.3)\end{pspicture}& &$-$& 
\tabularnewline
\hline
 &\rnode{a}{$-1$}& & & &\rnode{c}{$+\infty$}& & & &\rnode{e}{$+\infty$}& & 
\tabularnewline
$r$& & & & & &$0$& & & &$0$& 
\tabularnewline
 & & &\rnode{b}{$-\infty$}& & & &\rnode{d}{$-\infty$}& & & &\rnode{f}{$-1$}
\tabularnewline
\hline
\end{tabular}
\ncline{->}{a}{b}
\ncline{->}{c}{d}
\ncline{->}{e}{f}
\end{center}

\end{document}

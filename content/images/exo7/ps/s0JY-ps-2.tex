%%%%%%%%%%%%%%%%%% PREAMBULE %%%%%%%%%%%%%%%%%%

\documentclass[12pt,a4paper]{article}

\usepackage{amsfonts,amsmath,amssymb,amsthm}
\usepackage[francais]{babel}
\usepackage[utf8]{inputenc}
\usepackage[T1]{fontenc}

%----- Ensemles : entiers, reels, complexes -----
\newcommand{\Nn}{\mathbb{N}} \newcommand{\N}{\mathbb{N}}
\newcommand{\Zz}{\mathbb{Z}} \newcommand{\Z}{\mathbb{Z}}
\newcommand{\Qq}{\mathbb{Q}} \newcommand{\Q}{\mathbb{Q}}
\newcommand{\Rr}{\mathbb{R}} \newcommand{\R}{\mathbb{R}}
\newcommand{\Cc}{\mathbb{C}} \newcommand{\C}{\mathbb{C}}

%----- Modifications de symboles -----
\renewcommand {\epsilon}{\varepsilon}
\renewcommand {\Re}{\mathop{\mathrm{Re}}\nolimits}
\renewcommand {\Im}{\mathop{\mathrm{Im}}\nolimits}

%----- Fonctions usuelles -----
\newcommand{\ch}{\mathop{\mathrm{ch}}\nolimits}
\newcommand{\sh}{\mathop{\mathrm{sh}}\nolimits}
\renewcommand{\tanh}{\mathop{\mathrm{th}}\nolimits}
\newcommand{\Arcsin}{\mathop{\mathrm{Arcsin}}\nolimits}
\newcommand{\Arccos}{\mathop{\mathrm{Arccos}}\nolimits}
\newcommand{\Arctan}{\mathop{\mathrm{Arctan}}\nolimits}
\newcommand{\Argsh}{\mathop{\mathrm{Argsh}}\nolimits}
\newcommand{\Argch}{\mathop{\mathrm{Argch}}\nolimits}
\newcommand{\Argth}{\mathop{\mathrm{Argth}}\nolimits}
\newcommand{\pgcd}{\mathop{\mathrm{pgcd}}\nolimits} 

%----- Commandes special dessin a ajouter localement ------
\usepackage{geometry}
\usepackage{pstricks}
\usepackage{pst-plot}
\usepackage{pst-node}
\usepackage{graphics,epsfig}

\pagestyle{empty}

% Que faire avec ce fichier monimage.tex ?
%   1/ latex monimage.tex
%   2/ dvips monimage.dvi
%   3/ ps2eps monimage.ps
%   4/ ps2pdf -dEPSCrop monimage.eps
%   5/ Dans votre fichier d'exos \includegraphics{monimage}

\begin{document}

\begin{center}
\begin{pspicture}(-7.5,-2.5)(7.5,2.5)
\psline{->}(-8,0)(8,0)
\psline{->}(0,-2.5)(0,2.5)
\parametricplot[linecolor=blue,plotpoints=1000]{-450}{500}{t 180 div 3.14 mul t sin sub
1 t cos sub}
\parametricplot[linecolor=red,plotpoints=1000]{-450}{500}{t 180 div 3.14 mul t sin add
t cos 1 sub}
\psline[linecolor=red,linewidth=0.4mm]{->}(1.3,1.54)(4.44,-0.46)
\psdots[linecolor=blue](1.3,1.54)(6.09,0.52)
\psdots[linecolor=red](4.44,-0.46)
\uput[ul](1.3,1.54){\textcolor{blue}{$M(t)$}}
\uput[l](6.09,0.52){\textcolor{blue}{$M(t+\pi)$}}
\uput[dr](4.44,-0.46){\textcolor{red}{$\Omega(t+\pi)$}}
\uput[d](2,1){\textcolor{red}{$\overrightarrow{u}$}}
\psplot[linecolor=red,linewidth=0.2mm]{3.2}{6.7}{0.6 x 4.44 sub mul 0.46 sub}
\pscircle[linestyle=dashed,linecolor=red](4.44,-0.46){1.9}
\end{pspicture}
\end{center}


\end{document}

\uuid{Jvru}
\niveau{PCSI}
\module{Analyse}
\chapitre{Convergence d'une suite}
\sousChapitre{Suites et limites}
\duree{15}
\difficulte{1}
\auteur{Antoine Crouzet}
\datecreate{01/12/2024}
\titre{Exercice bilan I}
\contenu{
\texte{On considère la suite $(u_n)$ définie par $u_0=3$ et pour  tout $n$, $$u_{n+1}=\frac{2}{1+u_n}$$}
\question{Démontrer que pour tout $n\in\N$, on a $$0\leq u_n \leq 3$$}
\reponse{\begin{align*}
Soit $P_n$ la proposition définie pour tout entier $n$ par ``$0\leq u_n \leq 3$''.\begin{itemize}
    \item[$\bullet$] Initialisation : Pour $n=0$, $u_0=3$ et donc $0\leq u_0\leq 3$ : $P_0$ est vraie.
    \item[$\bullet$] Hérédité : supposons que la proposition $P_n$ est vraie pour un certain entier $n$, et montrons que $P_{n+1}$ est vraie :
    \begin{eqnarray*}
        & 0\leq u_n \leq 3 &\\
        \textrm{donc } & 1  \leq u_n +1 \leq 4 &\\
        \textrm{et donc } & \frac{1}{1} \geq \frac{1}{u_n+1} \geq \frac{1}{4} &\textrm{ car la fonction inverse est décroissante sur }\R^*_+\\
        \textrm{ainsi } & 2 \geq u_{n+1} \geq \frac{1}{2} &\textrm{ soit } 3 \geq u_n \geq 0
    \end{eqnarray*}
    La proposition $P_{n+1}$ est donc vraie.
\end{itemize}
    Bilan : d'après le principe de récurrence, la proposition $P_n$ est vraie pour tout entier $n$.
\end{align*}}
\question{On considère la suite $(v_n)$ définie pour tout $n$ par $$v_n=\frac{u_n-1}{u_n+2}$$
	\begin{enumerate}
		\item Expliquer pourquoi la suite $(v_n)$ est bien définie pour tout $n$.
		\item Démontrer que la suite $(v_n)$ est géométrique.
		\item Exprimer $v_n$ en fonction de $n$.
	\end{enumerate}}
\reponse{\begin{align*}
~\begin{enumerate}
            \item Puisque pour tout entier $n$, $0\leq u_n$ alors $u_n+2 \geq 2 \neq 0$. Ainsi $(v_n)$ est bien définie.
            \item Montrons que la suite $(v_n)$ est géométrique :
            $$v_{n+1}=\frac{u_{n+1}-1}{u_{n+1}+2} = \frac{\frac{2}{1+u_n}-1}{\frac{2}{1+u_n}+2}= \frac{\frac{1-u_n}{1+u_n}}{\frac{4+2u_n}{1+u_n}}=\frac{1-u_n}{4+2u_n}=\frac{-1}{2}v_n$$
            La suite $(v_n)$ est donc géométrique, de raison $-\frac{1}{2}$ et de premier terme $v_0=\frac{u_0-1}{u_0+2} = \frac{2}{5}$.
            \item On a donc, pour tout entier $n$, $\displaystyle{v_n=\frac{2}{5}\left(-\frac{1}{2}\right)^n}$.
        \end{enumerate}
\end{align*}}
\question{Exprimer $u_n$ en fonction de $n$. En déduire la limite de $(u_n)$.}
\reponse{\begin{align*}
Puisque pour tout entier $n$, on a $\displaystyle{v_n=\frac{u_n-1}{u_n+2}}$, alors $$v_n=\frac{u_n-1}{u_n+2} \Leftrightarrow v_n(u_n+2)=u_n-1 \Leftrightarrow u_n=\frac{-1-2v_n}{v_n-1}$$
    Ainsi, pour tout entier $n$,
    $$u_n=\frac{-1-\frac{4}{5}\left(-\frac{1}{2}\right)^n}{\frac{2}{5}\left(-\frac{1}{2}\right)^n-1}$$
    Puisque $-1<-\frac{1}{2}<1$, par théorème, $\displaystyle{\lim_{n\rightarrow +\infty} \left(-\frac{1}{2}\right)^n=0}$. Par somme et quotient
    $$\lim_{n\rightarrow +\infty}u_n = \frac{-1}{-1} = 1$$
\end{align*}}
}

\uuid{lI9G}
\niveau{PCSI}
\module{Analyse}
\chapitre{Convergence d'une suite}
\sousChapitre{Suites adjacentes}
\duree{15}
\difficulte{1}
\auteur{Antoine Crouzet}
\datecreate{01/12/2024}
\titre{Suites adjacentes I}
\contenu{
\question{On considère les suites $(u_n)_{n\geq 1}$ et $(v_n)_{n\geq 1}$ définies, pour $n\geq1$, par :
$u_n=\displaystyle\sum_{k=1}^n \frac{1}{k^2}$ et $ v_n=u_n+\dfrac{1}{n}$.
Montrer que $(u_n)$ et $(v_n)$ sont adjacentes. Que peut-on conclure ?}
\reponse{Démontrons que $(u_n)_{n\geq 1}$ est croissante, $(v_n)_{n\geq 1}$ est décroissante et que $v_n-u_n\tendversen{n\to+\infty} 0$.

Soit $n\geq 1$.
\begin{align*}
	u_{n+1}-u_n &= \sum_{k=1}^{n+1} \frac{1}{k^2} - \sum_{k=1}^n \frac{1}{k^2} \\
			    &= \frac{1}{(n+1)^2} > 0
\end{align*}
	La suite $u$ est donc croissante. De même, soit $n\geq 1$.
\begin{align*}
	v_{n+1}-v_n &= u_{n+1}+\frac{1}{n+1} - \left(u_n+\frac1n\right) \\
			    &= (u_{n+1}-u_n) + \frac{1}{n+1}-\frac1n\\
				&= \frac{1}{(n+1)^2}+\frac{1}{n+1}-\frac1n\\
				&= \frac{n}{n(n+1)^2}+\frac{n(n+1)}{n(n+)^2}-\frac{(n+1)^2}{n(n+1)^2}\\
				&= \frac{n+n(n+1)-(n+1)^2}{n(n+1)^2} = \frac{-1}{n(n+1)^2} < 0
\end{align*}
La suite $v$ est donc décroissante. Enfin \[ v_n-u_n=\frac1n\tendversen{n\to+\infty} 0 \]
Les suites $u$ et $v$ sont donc adjacentes. Par théorème, elles sont donc convergente et ont la même limite.
\begin{remarque}
	Leur limite commune est $\frac{\pi^2}{6}$, mais la démonstration n'est pas aisée.
\end{remarque}}
}

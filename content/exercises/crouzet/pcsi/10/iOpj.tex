\uuid{iOpj}
\niveau{PCSI}
\module{Analyse}
\chapitre{Convergence d'une suite}
\sousChapitre{Premières limites}
\duree{10}
\difficulte{1}
\auteur{Antoine Crouzet}
\datecreate{01/12/2024}
\titre{Limites d'un quotient}
\contenu{
\texte{Déterminer la limite (si elle existe)  des suites $(u_n)$ définies par}
\question{$u_n=\dfrac{\left({5}/{6}\right)^n-7}{5n^2}$}
\reponse{\begin{align*}
Puisque $-1<\frac56<1$, $\left(\frac56\right)^n\tendversen{n\to+\infty} 0$. Par somme,  \[ \lim_{n\to+\infty}\left(\frac56\right)^n-7=-7 \]
	De plus, $5n^2\tendversen{n\to +\infty} +\infty$. Par quotient \[ \lim_{n\to +\infty} \dfrac{\left({5}/{6}\right)^n-7}{5n^2} = 0 \]
\end{align*}}
\question{$u_n=\dfrac{-2n-7}{\left({1}/{2}\right)^n}$}
\reponse{\begin{align*}
Puisque $n\tendversen{n\to +\infty} +\infty$, par produit, $-2n\tendversen{n\to +\infty} -\infty$. Par somme \[ \lim_{n\to +\infty} -2n-7=-\infty \]
	De plus, $-1<\frac12<1$ et donc $\left(\frac12\right)^n \tendversen{n\to +\infty} 0^+$ ($0^+$ car pour tout $n$, $\left(\frac12\right)^n\geq 0$). Par quotient \[ \lim_{n\to +\infty} \dfrac{-2n-7}{\left({1}/{2}\right)^n} = -\infty \]
\end{align*}}
\question{$u_n=\dfrac{\left({5}/{6}\right)^n-7}{5n^{-2}+1}$}
\reponse{\begin{align*}
Puisque $-1<\frac56<1$, $\left({5}/{6}\right)^n\tendversen{n\to +\infty} 0$. Par somme \[ \lim_{n\to+\infty}\left({5}/{6}\right)^n-7=-7 \]
	De plus, $n^{-2}=\frac{1}{n^2}\tendversen{n\to +\infty} 0$, donc par somme \[ \lim_{n\to +\infty} 5n^{-2}+1=1 \]
	Par quotient \[ \lim_{n\to +\infty} \dfrac{\left({5}/{6}\right)^n-7}{5n^{-2}+1}=-7 \]
\end{align*}}
}

\uuid{iuc7}
\niveau{PCSI}
\module{Analyse}
\chapitre{Convergence d'une suite}
\sousChapitre{Premières limites}
\duree{10}
\difficulte{1}
\auteur{Antoine Crouzet}
\datecreate{01/12/2024}
\titre{Limites d'un polynôme ou de fraction rationnelle}
\contenu{
\texte{Déterminer la limite (si elle existe) de chacune des suites $(u_n)$ définies par}
\question{$u_n=-2n^5+n^4-7n^3+8n+2$}
\reponse{On utilise la mise en facteur pour lever l'indétermination. 
\begin{enumerate}
	\item On met $n^5$ en facteur : 
	\begin{align*}
		u_n &= n^5\left(-2+\frac{n^4}{n^5}-\frac{7n^3}{n^5}+\frac{8n}{n^5}+\frac{2}{n^5}\right)\\
		    &= n^5\left( -2 +\frac{1}{n}-\frac{7}{n^2}+\frac{8}{n^4}+\frac{2}{n^5}\right)
		\end{align*}
	Par somme, on a \[ \lim_{n\to +\infty} -2 +\frac{1}{n}-\frac{7}{n^2}+\frac{8}{n^4}+\frac{2}{n^5} = -2 \]
	De plus, $\ds{\lim_{n\to+\infty} n^5 = +\infty}$. Par produit, $\ds{\lim_{n\to+\infty} u_n=-\infty}$.}
\question{$u_n=\dfrac{-2n-7}{n^2+2n-6}$}
\reponse{\item De même :
	\begin{align*}
		u_n &= \frac{n\left(-2-\frac{7}{n}\right)}{n^2\left(1+\frac{2}{n}-\frac{6}{n}\right)}\\
		&= \frac{\left(-2-\frac{7}{n}\right)}{n\left(1+\frac{2}{n}-\frac{6}{n}\right)}
	\end{align*}
	Par somme, \[ \lim_{n\to +\infty} -2-\frac{7}{n} = -2 \qeq \lim_{n\to +\infty} 1+\frac{2}{n}-\frac{6}{n}=1 \]
	Puisque $n\tendversen{n\to +\infty} +\infty$, par produit \[ \lim_{n\to +\infty} n\left(1+\frac{2}{n}-\frac{6}{n}\right)=+\infty \]}
\question{$u_n=3n+1+\dfrac{-3}{n^{2}+1}$}
\reponse{\begin{align*}
Par quotient, $\ds{\lim_{n\to +\infty} u_n=0}$.
	\item Enfin, $n^2+1\tendversen{n\to \infty} +\infty$ par somme. Par quotient, \[ \lim_{n\to +\infty} \frac{-3}{n^2+1} = 0 \]
	De plus, $3n+1\tendversen{n\to+\infty} +\infty$ par somme. On conclut donc par somme que $\ds{\lim_{n\to +\infty} u_n=+\infty}$
\end{enumerate}
\end{align*}}
}

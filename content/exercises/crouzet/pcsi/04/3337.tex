\uuid{3337}
\niveau{PCSI}
\module{Analyse}
\chapitre{Systèmes linéaires}
\sousChapitre{Applications}
%!TeX root=../../../encours.nouveau.tex
%%% Début exercice %%%

\duree{10}
\difficulte{1}
\auteur{Antoine Crouzet}
\datecreate{01/12/2024}
\titre{Des polynômes}
\contenu{
\question{Existe-t-il un polynôme $P\in \R[-3]$ vérifiant $P(1)=P(-1)=P'(1)=1$ ?

Existe-t-il un polynôme $P\in \R[-3]$ vérifiant, pour tout entier $k\in \interent{0 3}$, $P(k)=k$ ?}
\reponse{On écrit $P=aX^3+bX^2+cX+d$. Les hypothèses s'écrivent alors :
\[ \systeme{a+b+c+d=1,-a+b-c+d=1, 3a+2b+c=1} \Leftrightarrow \systeme{a+b+c+d=1, 2b+2d=2,-b-2c-3d=-2} \Leftrightarrow \systeme{a+b+c+d=1, b+d=1, -2c-2d=-1}\]
Ce système admet une infinité de solutions : il y a donc une infinité de polynôme vérifiant les hypothèses, par exemple $P=  -\frac12 X^3+X^2+ \frac12 X$.

De même, si $P=aX^3+bX^2+cX+d$, les hypothèses s'écrivent :
\begin{align*}
  \systeme{a3^3+b3^2+c3+d=3, a2^3+b2^2+c2+d=2, a+b+c+d=1, d=0} &\iff \systeme{a+b+c=1, 9a+3b+c=1, 4a+2b+c=1, d=0}
  \iff \systeme{a+b+c=1, -6b-8c=-8, -2b-3c=-3,d=0}\\
  &\iff \systeme{a+b+c=1, -6b-8c=-8, -c=-1,d=0} \iff \systeme*{a=0,b=0,c=1,d=0}
\end{align*}
Il y a ainsi un unique polynôme qui convient, le polynôme $P=X$.}
}

%%% Fin exercice %%%

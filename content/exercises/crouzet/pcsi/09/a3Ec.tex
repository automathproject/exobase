\uuid{a3Ec}
\niveau{PCSI}
\module{Analyse}
\chapitre{Généralités sur les suites}
\sousChapitre{Sens de variation, majoration, minoration}
\duree{10}
\difficulte{1}
\auteur{Antoine Crouzet}
\datecreate{01/12/2024}
\titre{Majoration, minoration}
\contenu{
\question{Déterminer si les suites suivantes sont majorées, minorées, ou bornées.
\begin{itemize}
 \item La suite $u$ définie par $\displaystyle{u_n=\frac{n+2}{2n+1}}$.\\
 \item La suite $v$ définie par $\displaystyle{v_n=\frac{n^2+2}{n}}$.
\end{itemize}

\begin{methode}
On peut calculer les premiers termes pour avoir une idée d'un majorant $M$ ou minorant $m$. On calcule ensuite $u_n-M$ ou $u_n-m$ et on étudie le signe de cette différence.
\end{methode}}
\reponse{\begin{align*}
\begin{itemize}[label=\textbullet]
	\item Pour tout entier $n$, on a $n+2>0$ et $2n+1>0$ donc par quotient, $$\forall~n,~u_n>0$$
		Il semblerait qu'elle soit majorée par $2$. Calculons $u_n-2$ :
		$$\forall~n,~u_n-2=\frac{n+2}{2n+1}-2=\frac{n+2-2(2n+1)}{2n+1}=\frac{-n}{2n+1}$$
		Pour tout entier $n$, $-n\leq 0$ et $2n+1>0$ donc $u_n-2\leq 0$.\\
		\textbf{Bilan} : $\forall~n,~0<u_n\leq 2$.
		\\\textit{Remarque} : on peut aussi montrer que pour tout entier $n$, $u_n\geq \frac{1}{2}$.
	\item Pour tout entier $n>0$, on a $n^2+2>0$ et donc $$\forall~n\geq 1,~v_n>0$$
		Constatons également que pour tout entier $n\geq 1$, $$v_n=\frac{n^2+2}{n}=\frac{n^2}{n}+\frac{2}{n}=n+\frac{2}{n}\geq n$$
		car $\frac{2}{n}>0$.\\
		Or la suite $w$ définie pour tout $n$ par $w_n=n$ n'est pas majorée. Par comparaison, la suite $v$ n'est pas majorée.\\\textbf{Bilan} : $v$ est minorée par $0$ mais n'est pas majorée.
\end{itemize}
\end{align*}}
}

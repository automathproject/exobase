\uuid{lR40}
\niveau{PCSI}
\module{Analyse}
\chapitre{Généralités sur les suites}
\sousChapitre{Suites arithmétiques et géométriques}
\duree{10}
\difficulte{1}
\auteur{Antoine Crouzet}
\datecreate{01/12/2024}
\titre{Justifications}
\contenu{
\texte{Les suites suivantes sont elles arithmétiques ? géométriques ?}
\question{La suite $u$ définie pour tout $n$ par $u_n=3n+5$.\vspace*{2mm}}
\reponse{Remarquons que, pour tout entier $n$, $u_{n+1}-u_n=3$. La suite $(u_n)$ est donc arithmétique, de raison $3$ et de premier terme $u_0=5$.}
\question{La suite $v$ définie pour tout $n$ par $v_n=\frac{n+1}{n^2+1}$.\vspace*{2mm}}
\reponse{\begin{align*}
On a $v_0=1$, $v_1=1$ et $v_2=\frac{3}{5}$. On remarque alors que
		$$v_1-v_0=0\textrm{ et } v_2-v_1=-\frac{2}{5} \neq v_1-v_0$$
		La suite $v$ n'est donc pas arithmétique.\\
		De même, on a
		$$\frac{v_1}{v_0}=1 \textrm{ et } \frac{v_2}{v_1}=\frac{3}{5}\neq \frac{v_1}{v_0}$$
		La suite $v$ n'est pas géométrique.
\end{align*}}
\question{La suite $w$ définie pour tout $n$ par  $w_n=3\times 2^n$.
\begin{methode}
Pour montrer qu'une suite est arithmétique, on calcule $u_{n+1}-u_n$. Pour montrer qu'elle est géométrique, on part de $u_{n+1}$ pour essayer de montrer qu'elle s'écrit $qu_n$.\\Pour montrer qu'une suite n'est ni arithmétique, ni géométrique, on calcule les premiers termes et on les utilise pour le montrer.
\end{methode}}
\reponse{\begin{align*}
Remarquons que, pour tout entier $n$, $w_{n+1}=3\times 2^{n+1}=2\times 3\times 2^n=2\times w_n$. La suite $(w_n)$ est donc géométrique, de raison $2$ et de premier terme $w_0=3$.
\end{align*}}
}

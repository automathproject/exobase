\uuid{zPra}
\niveau{PCSI}
\module{Analyse}
\chapitre{Généralités sur les suites}
\sousChapitre{Sens de variation, majoration, minoration}
\duree{20}
\difficulte{1}
\auteur{Antoine Crouzet}
\datecreate{01/12/2024}
\titre{Variations}
\contenu{
\question{\begin{itemize}
 	\item Soit $u$ la suite définie pour tout $n$ par $u_n=\frac{n-1}{n+2}$. Démontrer que la suite $(u_n)$ est croissante.
 	\item Etudier le sens de variation des suites $\left(\frac{n^2+1}{n}\right)$,$\left(\frac{2^n}{n}\right)$ et $\left(\frac{n}{2n-1}\right)$
 \end{itemize}}
\reponse{\begin{align*}
\begin{itemize}
 	\item Pour tout entier $n$, on a $$u_{n+1}-u_{n}=\frac{n+1-1}{n+1+2}-\frac{n-1}{n+2}=\frac{n}{n+3}-\frac{n-1}{n+2}=\frac{n(n+2)-(n-1)(n+3)}{(n+3)(n+2)}=\frac{3}{(n+3)(n+2)}$$
 	Puisque pour tout entier naturel $n$, $n+2>0$ et $n+3>0$, on en déduit que pour tout $n$, $u_{n+1}-u_{n}>0$.\\\textbf{Bilan} : la suite $(u_n)$ est strictement croissante.
 	\item Notons les suites respectivement $v, w$ et $z$. De la même manière, pour tout entier $n\geq 1$ :
 		$$v_{n+1}-v_n=\frac{(n+1)^2+1}{n+1}-\frac{n^2+1}{n}=\frac{n^2+n-1}{n(n+1)}$$
 		Pour tout $n\geq 1$, on a $n-1\geq 0$ et donc $n-1+n^2\geq 0$ (car $n^2\geq 0)$. Ainsi $$\forall~n\geq 1,~v_{n+1}-v_n\geq 0$$
 		\textbf{Bilan} : la suite $(v_n)_{n\geq 1}$ est croissante.
 	Pour tout $n\geq 1$, on a $$w_{n+1}-w_n=\frac{2^{n+1}}{n+1}-\frac{2^n}{n}=\frac{2^{n+1}n-2^n(n+1)}{n(n+1)}=\frac{2^n(n-1)}{n(n+1)}$$
 		Pour $n\geq 1$, on a $2^n>0$, $n-1\geq 0$ et $n(n+1)>0$. Par quotient, pour tout entier $n\geq 1$, $w_{n+1}-w_n\geq 0$.\\\textbf{Bilan} : la suite $(w_n)_{n\geq 1}$ est croissante.
 	Pour tout entier $n$, on a $$z_{n+1}-z_n=\frac{n+1}{2(n+1)-1}-\frac{n}{2n-1} = \frac{(n+1)(2n-1)-n(2n+1)}{(2n-1)(2n+1)}=\frac{}{(2n-1)(2n+1)}=\frac{-1}{(2n-1)(2n+1)}$$
 		Pour tout entier $n\geq 1$, on a $2n-1>0$ et $2n+1>0$ donc par quotient,
 		$$\forall~n\geq 1,~z_{n+1}-z_n<0$$
 		\textbf{Bilan} : la suite $(z_n)$ est strictement décroissante à partir du rang $1$.
 \end{itemize}
\end{align*}}
}

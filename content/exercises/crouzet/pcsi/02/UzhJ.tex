\uuid{UzhJ}
\niveau{PCSI}
\module{Analyse}
\chapitre{Généralités sur les nombres réels}
\sousChapitre{Ensembles de nombres}
%!TeX root=../../../encours.nouveau.tex
%%% Début exercice %%%

\duree{5}
\difficulte{1}
\auteur{Antoine Crouzet}
\datecreate{01/12/2024}
\titre{Non inclusion}
\contenu{
\question{Montrer que $\Z\not \subset \N$, $\mathbb{D}\not \subset \Z$ et $\Q \not \subset \mathbb{D}$.}
\reponse{\begin{align*}
On donne un contre exemple pour chaque cas. Ainsi :
	\[ -1 \in \Z \text{ mais } -1 \not \in \N  \]
	\[ 0,1 \in \mathbb{D} \text{ mais } 0,1\not \in \Z \]
	\[ \frac{1}{3}\in \Q \text{ mais } \frac{1}{3} \not \in \mathbb{D} \]
Le dernier résultat venant du fait qu'il y a une infinité de chiffres après la virgule pour $\frac{1}{3}$.
\end{align*}}
}

%%% Fin exercice %%%

\uuid{w1xE}
\niveau{PCSI}
\module{Analyse}
\chapitre{Généralités sur les nombres réels}
\sousChapitre{Calculs}
%!TeX root=../../../encours.nouveau.tex
%%% Début exercice %%%
\duree{10}
\difficulte{1}
\auteur{Antoine Crouzet}
\datecreate{01/12/2024}
\titre{Factorisation}
\contenu{
\question{Factoriser chacune des expressions littérales suivantes :
{\setlength{\linewidth}{18cm}
	
		$A=64\,x^{2}-100$\\
		$B=-\left( 9\,x-2\right) \times \left( 4\,x+9\right) +\left( 3\,x-8\right) \times \left( 9\,x-2\right) $\\
		$C=16\,x^{2}-24\,x+9$\\
		$D=\left( x+5\right) ^{2}-81$\\
		$E=\left( -8\,x+2\right) ^{2}+\left( -8\,x+2\right) \times \left( 8\,x+4\right) $\\
		$F=\left( 9\,x+7\right) \times \left( 3\,x+3\right) +9\,x+7$

}}
\reponse{On utilise les identités remarquables, ou les méthodes de factorisation usuelles. Cela donne ainsi :
\begin{align*}
	A &= (8x)^2-10^2=(8x-10)(8x+10)\\
	B &= (9x-2)\left[ -(4x+9)+(3x-8) \right] = (9x-2)(-x-17) \\
  C &= (4x)^2-2\times (4x)\times 3 + 3^2 =(4x-3)^2\\
	D &= (x+5)^2-9^2 = (x+5-9)(x+5+9)=(x-4)(x+14)\\
	E &= (-8x+2)\left [ (-8x+2)+(8x+4) \right ] = 6(-8x+2)\\
	F &= (9x+7)\left [ (3x+3)+1 \right ] = (9x+7)(3x+4)
\end{align*}}
}

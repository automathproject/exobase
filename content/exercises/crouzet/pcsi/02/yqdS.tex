\uuid{yqdS}
\niveau{PCSI}
\module{Analyse}
\chapitre{Généralités sur les nombres réels}
\sousChapitre{\'Equations et inéquations}
%!TeX root=../../../encours.nouveau.tex

\duree{15}
\difficulte{2}
\auteur{Antoine Crouzet}
\datecreate{01/12/2024}
\titre{Inéquation et racines}
\contenu{
\question{Résoudre l'équation suivante, d'inconnue $y$ :
\[ 2y-5 - \sqrt{4y-7} \leq 0 \]}
\reponse{\begin{align*}
On commence par déterminer le domaine de définition : cette inéquation n'a de sens que si $4y-7\geq 0$, c'est-à-dire $y\in \left [ \frac{7}{4} +\infty\right [$.

On raisonne ensuite par disjonction de cas :
\begin{itemize}
  \item Premier cas : $2y-5\leq 0$, c'est-à-dire $y\leq \frac{5}{2}$. Dans ce cas, l'inéquation est vérifiée (puisqu'une racine est toujours positive). Ainsi, l'inéquation est vérifiée si $y\in \left[ \frac{7}{4},\, \frac{5}{2}\right]$.
  \item Deuxième cas : $2y-5\geq 0$. Dans ce cas, l'inéquation est équivalente à $(2y-5)^2 \leq \sqrt{4y-7}^2$, c'est-à-dire \[ (2y-5)^2 \leq 4y-7 \Longleftrightarrow 4y^2-24y+32\leq 0 \Longleftrightarrow y^2-6y+8\leq 0 \]
  Les solutions de cette inéquation est $\interff{2 4}$, mais on a $2y-5\geq 0$, c'est-à-dire $y\geq \frac{5}{2}$. Ainsi, sur cet intervalle, les solutions sont $\left[ \frac{5}{2},\, 4\right]$.
\end{itemize}
On conclut : l'ensemble des solutions de cette inéquation est donc \[ \boxed{\mathcal{S}}=\left[ \frac{7}{4},\,\frac{5}{2}\right] \cup \left[\frac{5}{2},\,4\right] = \boxed{\left[ \frac{7}{4},\, 4\right]}\]
\end{align*}}
}

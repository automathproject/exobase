\uuid{K84O}
\niveau{PCSI}
\module{Analyse}
\chapitre{Généralités sur les nombres réels}
\sousChapitre{Calculs}
%!TeX root=../../../encours.nouveau.tex
%%% Début exercice %%%
\duree{15}
\difficulte{1}
\auteur{Antoine Crouzet}
\datecreate{01/12/2024}
\titre{Racines}
\contenu{
\question{Calculer les expressions suivantes et donner le résultat sous la forme $a\,\sqrt{b}$ avec $a$ et $b$ entiers, $b$ le plus petit possible.
	\noindent
		\[ A = \sqrt{54}-3\,\sqrt{96}-5\,\sqrt{24}\]
		
		\[ B = \sqrt{160}\times\sqrt{40}\times\sqrt{90}\]
	
	\vspace{-0.5cm}}
\reponse{On utilise la règle \(\sqrt{a\times b}=\sqrt{a}\sqrt{b}\) pour $a$ et $b$ deux réels positifs. Ainsi :
	\begin{align*}
		A &= \sqrt{2\times 3^3}-3\sqrt{3\times 2^5}-5\sqrt{2^3\times 3} \\
		  &= 3\sqrt{6}-12\sqrt{6}-10\sqrt{6}=-19\sqrt{6}\\
		B &= \sqrt{5\times 2^5}\times \sqrt{2^3\times 5}\times \sqrt{3^2\times 2\times 5}\\
		  &= \sqrt{5^3 \times 2^9 \times 3^2} = 240\sqrt{10}
	\end{align*}}
\question{Calculer les expressions suivantes et donner le résultat sous la forme $a+b\,\sqrt{c}$ avec $a$, $b$ et $c$ entiers.
	\noindent
		\[ C = \left( 3\,\sqrt{10}-5\,\sqrt{3} \right)^2\]
		
		\[ D = \left( 3\,\sqrt{5}+2\,\sqrt{6} \right)^2\]
	
 	 \vspace{-0.5cm}}
\reponse{On développe par l'identité remarquable et on conclut :
	\begin{align*}
		C &= (3\sqrt{10})^2 - 2 \times 3\sqrt{10}\times 5\sqrt{3}+(5\sqrt{3})^2\\
		  &= 90 - 30\sqrt{30}+75 = 165-30\sqrt{30}\\
		D &= (3\sqrt{5})^2 + 2\times 3\sqrt{5}\times 2\sqrt{6}+(2\sqrt{6})^2\\
		  &= 45 + 12\sqrt{30}+24 = 69 + 12\sqrt{30}
	\end{align*}}
\question{Calculer les expressions suivantes et donner le résultat sous la forme d'un nombre entier.
	\noindent
		\[ E = \left( 3-2\,\sqrt{5} \right)\left( 3+2\,\sqrt{5} \right)\]
		
		\[ F = \frac{24\,\sqrt{45}}{9\,\sqrt{80}}\]
	
	\vspace{-1cm}}
\reponse{On développe et utilise les formules sur les racines.
	\begin{align*}
		E &= 3^2 - (2\sqrt{5})^2 = 9-20=-11\\
		F &= \frac{24}{9}\sqrt{\frac{45}{80}}\\
		  &= \frac{8}{3} \sqrt{\frac{9}{16}} = \frac{8}{3}\frac{3}{4}=2
	\end{align*}}
}

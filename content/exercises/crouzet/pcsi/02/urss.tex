\uuid{urss}
\niveau{PCSI}
\module{Analyse}
\chapitre{Généralités sur les nombres réels}
\sousChapitre{Calculs}
%!TeX root=../../../encours.nouveau.tex
%%% Début exercice %%%
\duree{10}
\difficulte{3}
\auteur{Antoine Crouzet}
\datecreate{01/12/2024}
\contenu{
\question{Simplifier $\sqrt[3]{5\sqrt{2}+7} - \sqrt[3]{5\sqrt{2}-7}$.}
\reponse{On note $a = \sqrt[3]{5\sqrt{2}+7} - \sqrt[3]{5\sqrt{2}-7}$. On élève au cube et on utilise les propriétés des racines (cubiques, ici) :
{\footnotesize
\begin{align*}
  \left(\sqrt[3]{5\sqrt{2}+7} - \sqrt[3]{5\sqrt{2}-7} \right)^3 &= \left(\sqrt[3]{5\sqrt{2}+7}\right)^3 - 3 \left(\sqrt[3]{5\sqrt{2}+7}\right)^2\sqrt[3]{5\sqrt{2}-7}+3\sqrt[3]{5\sqrt{2}+7}\left(\sqrt[3]{5\sqrt{2}-7}\right)^2 - \left(\sqrt[3]{5\sqrt{2}-7}\right)^3\\
  &=\left(5\sqrt{2}+7\right) - 3 \sqrt[3]{\left(5\sqrt{2}+7\right)\left(5\sqrt{2}+7\right)\left(5\sqrt{2}-7\right)}\\&\quad+ 3 \sqrt[3]{\left(5\sqrt{2}+7\right)\left(5\sqrt{2}-7\right)\left(5\sqrt{2}-7\right)} - \left(5\sqrt{2}-7\right) \\
  &= 14 - 3\sqrt[3]{\left(5\sqrt{2}+7\right)\left(\left(5\sqrt{2}\right)^2-7^2\right)} + 3\sqrt[3]{\left(5\sqrt{2}-7\right)\left(\left(5\sqrt{2}\right)^2-7^2\right)} \\
  &= 14 - 3\sqrt[3]{5\sqrt{2}+7} + 3\sqrt[3]{5\sqrt{2}-7} = 14 - 3\left(\sqrt[3]{5\sqrt{2}+7} - \sqrt[3]{5\sqrt{2}-7}\right)
\end{align*}
}
Ainsi, $a^3=14-3a$. Il faut trouver les solutions de cette équation. On constate que $2$ est une solution. En factorisant : 
\[ a^3+3a-14 = (a-2)(a^2+2a+7) \]
et l'équation $a^2+2a+7=0$ n'admet pas de solution ($\Delta=-24$). La seule solution possible est donc $2$ et finalement
\[ \boxed{ \sqrt[3]{5\sqrt{2}+7} - \sqrt[3]{5\sqrt{2}-7} = 2.}\]}
}

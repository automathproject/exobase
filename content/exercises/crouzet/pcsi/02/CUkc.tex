\uuid{CUkc}
\niveau{PCSI}
\module{Analyse}
\chapitre{Généralités sur les nombres réels}
\sousChapitre{Inégalités et parties de $\R$}
%!TeX root=../../../encours.nouveau.tex

\duree{10}
\difficulte{1}
\auteur{Antoine Crouzet}
\datecreate{01/12/2024}
\titre{Des inégalités}
\contenu{
\texte{Soient $x$ et $y$ deux réels.}
\question{Montrer que $xy \leq \dfrac{x^2+y^2}{2}$ avec égalité si et seulement si $x=y$.}
\reponse{Soient $x$ et $y$ deux réels.
\begin{enumerate}
	\item On a
\begin{align*}
		\frac{x^2+y^2}{2}-xy &= \frac{x^2+y^2-2xy}{2} \\
		&= \frac{(x-y)^2}{2} \geq 0
	\end{align*}}
\question{Montrer que $(x+y)^2 \leq 2(x^2+y^2)$ avec égalité si et seulement si $x=y$.}
\reponse{avec égalité si et seulement si $(x-y)^2=0$, c'est-à-dire $x=y$.
	\item On a
\begin{align*}
			2(x^2+y^2)-(x+y)^2 &= 2(x^2+y^2)-(x^2+2xy+y^2) \\
			&= x^2+y^2-2xy = (x-y)^2
	\end{align*}}
\question{On suppose désormais que $x$ et $y$ sont strictement positifs. Montrer que $\dfrac{x}{y}+\dfrac{y}{x}\geq 2$.}
\reponse{et on conclut de la même manière.
	\item Enfin, si $x$ et $y$ sont strictement positifs :
\begin{align*}
		\frac{x}{y}+\frac{y}{x} - 2 &= \frac{x^2+y^2}{xy} - 2 \\
		&= \frac{x^2+y^2-2xy}{xy} = \frac{(x-y)^2}{xy}
	\end{align*}
ce qui nous donne le résultat puisque $xy>0$ car $x$ et $y$ sont strictement positifs.
\end{enumerate}}
}

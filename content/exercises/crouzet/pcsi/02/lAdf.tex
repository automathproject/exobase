\uuid{lAdf}
\niveau{PCSI}
\module{Analyse}
\chapitre{Généralités sur les nombres réels}
\sousChapitre{Parties entières}
\duree{5}
\difficulte{2}
\auteur{Antoine Crouzet}
\datecreate{01/12/2024}
\contenu{
\question{Soit $f:x\mapsto x-[x]$. Que représente $f(x)$ pour un réel $x$ ?}
\reponse{\begin{align*}
Si $x\in \R+$, $f(x)$ représente la partie décimale d'un nombre :
 \[ f(1,4) = 0,4,\quad f(\pi) = 0,141592653\dots.\]
 Si $x\in \R-$, $f(x)$ n'est pas égale à la partir décimale, mais à $1$ moins la partie décimale :
 \[ f(-2,1) = -2,1-(-3)= 0,9. \]
\end{align*}}
}

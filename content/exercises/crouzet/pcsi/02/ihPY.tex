\uuid{ihPY}
\niveau{PCSI}
\module{Analyse}
\chapitre{Généralités sur les nombres réels}
\sousChapitre{Calculs}
%!TeX root=../../../encours.nouveau.tex
%%% Début exercice %%%
\duree{5}
\difficulte{2}
\auteur{Antoine Crouzet}
\datecreate{01/12/2024}
\contenu{
\question{Soient $x$ et $y$ deux réels positifs tels que $y\leq x^2$. Montrer que \[ \sqrt{x+\sqrt{y}} = \sqrt{\frac{x+\sqrt{x^2-y}}{2}} + \sqrt{\frac{x-\sqrt{x^2-y}}{2}}. \]}
\reponse{Les termes étant positifs, on va comparer leurs carrés. On part du membre de droite :
{\footnotesize
\begin{align*}
 \left(\sqrt{\frac{x+\sqrt{x^2-y}}{2}} + \sqrt{\frac{x-\sqrt{x^2-y}}{2}}\right)^2 &= \left(\sqrt{\frac{x+\sqrt{x^2-y}}{2}}\right)^2 +2 \sqrt{\frac{x+\sqrt{x^2-y}}{2}} \times \sqrt{\frac{x-\sqrt{x^2-y}}{2}} + \left(\sqrt{\frac{x-\sqrt{x^2-y}}{2}}\right)^2 \\
 &= \frac{x+\sqrt{x^2-y}}{2} + 2\sqrt{ \frac{x+\sqrt{x^2-y}}{2}\frac{x-\sqrt{x^2-y}}{2}} + \frac{x-\sqrt{x^2-y}}{2}\\
 &= x + 2 \sqrt{ \frac{x^2-(x^2-y)}{4}} = x+ \sqrt{y}
\end{align*}
}
Puisque $x+ \sqrt{y}$ et $\ds{\sqrt{\frac{x+\sqrt{x^2-y}}{2}} + \sqrt{\frac{x-\sqrt{x^2-y}}{2}}}$ sont positifs, on en déduit le résultat par passage à la racine.}
}

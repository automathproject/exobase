\uuid{H3Jn}
\niveau{PCSI}
\module{Analyse}
\chapitre{Généralités sur les nombres réels}
\sousChapitre{\'Equations et inéquations}
%!TeX root=../../../encours.nouveau.tex

\duree{10}
\difficulte{1}
\auteur{Antoine Crouzet}
\datecreate{01/12/2024}
\titre{Une inégalité secondaire}
\contenu{
\question{Soient $a, b, c$ et $x$ des réels tels que $a\neq 0$. A quelle condition a-t-on $ax^2+bx+c > 0$ ?}
\reponse{\begin{align*}
\begin{attention}
  Ici, la condition va porter sur $x$, $a$, $b$ et $c$ !
\end{attention}
  \begin{itemize}
    \item Si $x=0$, alors l'inégalité est vraie si et seulement si $c>0$.
    \item On note $\Delta$ son discriminant : $\Delta=b^2-4ac$.
    \begin{itemize}
      \item Si $b^2<4ac$, l'inéquation est vérifiée pour tout $x\in \R$ si et seulement si $a>0$.
      \item Si $b^2=4ac$, l'inéquation est vérifiée pour tout $x\neq -\frac{b}{2a}$ si et seulement si $a>0$.
      \item Si $b^2>4ac$, l'inéquation est vérifiée pour $x\in\interoo{x_1 x_2}$ si $a<0$, et pour $x\in \interoo{-\infty{} x_1}\cup \interoo{x_2 +\infty}$ si $a>0$.
    \end{itemize}
  \end{itemize}
\end{align*}}
}

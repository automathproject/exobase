\uuid{FL5j}
\niveau{PCSI}
\module{Analyse}
\chapitre{Généralités sur les fonctions}
\sousChapitre{Généralités}
%!TeX root=../../../encours.nouveau.tex
%%% Début exercice %%%

\duree{5}
\difficulte{1}
\auteur{Antoine Crouzet}
\datecreate{01/12/2024}
\titre{Décomposée}
\contenu{
\question{Soit $f:x\donne \frac{1}{x^2+1}$ et $g:x\donne \eu{-x^2}$.
 Déterminer quatre fonctions $u$, $v$, $w$, et $z$ telles que $f=u\circ v$ et $g=w\circ z$.}
\reponse{\begin{align*}
On peut écrire $f$ sous la forme $\frac{1}{v}$ avec $v:x\donne x^2+1$. Ainsi, $f=u\circ v$ avec $u:x\donne \frac{1}{x}$ et $v:x\donne x^2+1$.\\
De même, $g$ s'écrit $\E^z$ avec $z:x\donne -x^2$. Ainsi, $g=w\circ z$ avec $w:x \donne \E^x$ et $z:x\donne -x^2$.
\end{align*}}
}

%%% Fin exercice %%%

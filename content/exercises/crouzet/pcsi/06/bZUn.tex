\uuid{bZUn}
\niveau{PCSI}
\module{Analyse}
\chapitre{Généralités sur les fonctions}
\sousChapitre{Études de fonctions}
%!TeX root=../../../encours.nouveau.tex
%%% Début exercice %%%

\duree{10}
\difficulte{1}
\auteur{Antoine Crouzet}
\datecreate{01/12/2024}
\titre{Des domaines de définitions}
\contenu{
\question{Déterminer l'ensemble de définition des fonctions suivantes :
\[f:x\donne \frac{\sqrt{2x+1}}{x^2+5x+6}\quad \quad \quad g:x\donne \ln\left( 2x^2+2x-12 \right) \]}
\reponse{\begin{align*}
\begin{methode}
Pour déterminer le domaine de définition d'une fonction, on détermine toutes les conditions d'existence (racine, dénominateur, logarithme,...). On conclut ensuite.	
\end{methode}

Pour la fonction $f$, nous avons deux conditions :
\begin{itemize}[label=\textbullet]
  \item $\sqrt{2x+1}$ n'est défini que si $2x+1\geq 0$ soit $x\geq -\frac{1}{2}$.
  \item Le quotient n'a un sens que si $x^2+5x+6\neq 0$. Le discriminant de ce trinôme vaut $\Delta=1$. Ce trinôme possède donc deux racines : $x_1=\frac{-5-1}{2}=-3$ et $x_2=\frac{-5+1}{2}=-2$. Il faut donc également que $x\neq -2$ et $x\neq -3$.
\end{itemize}
\textbf{Bilan} : la fonction $f$ est donc définie sur $\left[-\frac{1}{2};+\infty\right[$.\\~\\
Pour la fonction $g$, celle-ci n'est définie que si $2x^2+2x-12>0$. Le discriminant de ce trinôme vaut $\Delta=100$. Le trinôme possède donc deux racines, $x_1=\frac{-2-10}{4}=-3$ et $x_2=\frac{-2+10}{4}=2$. Ainsi, nous avons le tableau de signes suivant :
\[\begin{array}{|c|ccccccc|}\hline
	x & -\infty & & -3 & & 2 & & \\\hline
	2x^2+2x-12 & & + & 0 & - & 0 & + & \\\hline
\end{array}\]
Ainsi, \[\mathcal{D}_g=]-\infty; -3[ \cup ] 2;+\infty[\]
\end{align*}}
}

%%% Fin exercice %%%

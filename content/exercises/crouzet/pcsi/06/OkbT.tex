\uuid{OkbT}
\niveau{PCSI}
\module{Analyse}
\chapitre{Généralités sur les fonctions}
\sousChapitre{Généralités}
%!TeX root=../../../encours.nouveau.tex
%%% Début exercice %%%

\duree{5}
\difficulte{1}
\auteur{Antoine Crouzet}
\datecreate{01/12/2024}
\titre{Des composées}
\contenu{
\question{On considère les fonctions $f:x\donne x^2, g:x\donne \sqrt{x}$ et $h:x\donne x-2$.
Déterminer l'ensemble de définition et l'expression des fonctions suivantes :
\[f\circ g, f\circ h, g\circ h, h\circ f, h\circ g, g\circ f\]}
\reponse{\begin{align*}
Remarquons tout d'abord que $f$ est définie sur $\R$, $g$ est définie sur $\R[+]$ et $h$ est définie sur $\R$. De plus, $f$ est toujours positive, et $h$ est positive sur $\interfo{2 \plusinf}$. Ainsi :
\begin{itemize}
    \item[$\bullet$] $f\circ g$ est définie sur $\R[+]$, et pour tout $x\pgq 0$, $f\circ g(x) = \left(\sqrt{x}\right)^2 = x$.
    \item[$\bullet$] $f\circ h$ est définie sur $\R$, et pour tout $x\in \R$, $f\circ h(x) = \left(x-2\right)^2$.
    \item[$\bullet$] $g\circ h$ est définie sur $\interfo{2 \plusinf}$, et pour tout $x\pgq 2$, $g\circ h(x) = \sqrt{x-2}$.
    \item[$\bullet$] $h\circ f$ est définie sur $\R$, et pour tout $x\in \R$, $h\circ f(x) = x^2-2$.
    \item[$\bullet$] $h\circ g$ est définie sur $\R[+]$, et pour tout $x\pgq 0$, $h\circ g(x) = \sqrt{x}-2$.
    \item[$\bullet$] $g\circ f$ est définie sur $\R$, et pour tout $x\in \R$, $g\circ f(x) = \sqrt{x^2} = |x|$.    
\end{itemize}
\end{align*}}
}

%%% Fin exercice %%%

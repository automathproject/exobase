\uuid{a3e0}
\niveau{PCSI}
\module{Analyse}
\chapitre{Généralités sur les fonctions}
\sousChapitre{Equations, inéquations}
%!TeX root=../../../encours.nouveau.tex
%%% Début exercice %%%

\duree{15}
\difficulte{1}
\auteur{Antoine Crouzet}
\datecreate{01/12/2024}
\titre{Des équations à prendre au second degré}
\contenu{
\texte{Résoudre les équations suivantes :}
\question{$(E_1): x^4+x^2-2=0$}
\reponse{\begin{align*}
\begin{methode}
Dans le cas d'équation proche d'une équation du second degré, on change de variable inconnue pour se ramener à une équation de ce type.	
\end{methode}
 
\begin{enumerate}
	\item On pose $X=x^2$. $(E_1)$ s'écrit alors $X^2+X-2=0$. \\Son discriminant vaut $\Delta=1^2-4\times 1\times -2 =9$, et le trinôme possède donc deux solutions : $X_1=\frac{-1-3}{2}=-2$ et $X_2=\frac{-1+3}{2}=1$.\\On revient à l'inconnue de départ. Nous avons donc $X=1$ soit $x^2=1$, ce qui nous donne finalement deux solutions : $x=1$ ou $x=-1$ :
\end{align*}}
\question{$(E_2): \exp(2x)+\exp(x)-6=0$}
\reponse{\begin{align*}
\[\boxed{\mathcal{S}=\left \{-1;1\right \}}\]
	\item $(E_2)$ s'écrit également $\left(\eu{x}\right)^2+\eu{x}-6=0$. On pose $X=\eu{x}$. $(E_2)$ devient alors $X^2+X-6=0$. Ce trinôme possède deux solutions, qui sont $-3$ et $2$.\\ On revient à l'inconnue de départ : nous avons donc $\eu{x}=-3$, ce qui est impossible, ou $\eu{x}=2$ soit $x=\ln(2)$. Ainsi : \[\boxed{\mathcal{S}=\left \{ \ln(2) \right \}}\]
\end{align*}}
\question{$(E_3): \ln(x)^2+2\ln(x)-3=0$.}
\reponse{\begin{align*}
\item On pose $X=\ln(x)$. $(E_3)$ devient $X^2+2X-3=0$. Ce trinôme possède deux solutions : $-3$ et $1$. \\ On revient à l'inconnue de départ : on a donc $\ln(x)=-3$, soit $x=\eu{-3}$, ou $\ln(x)=1$, soit $x=\E$. Ainsi :
			\[\boxed{\mathcal{S}=\left \{ \eu{-3}; \E \right \}}\]
	\item On constate que $(E_4)$ s'écrit $(\sqrt{x})^2-\sqrt{x}-2=0$. On pose alors $X=\sqrt{x}$. $(E_4)$ s'écrit alors $X^2-X-2=0$. Ce trinôme possède deux solutions : $-1$ et $2$. \\On revient à l'inconnue de départ : $\sqrt{x}=-1$, ce qui est impossible, ou $\sqrt{x}=2$, c'est-à-dire $x=4$. Ainsi \[\boxed{\mathcal{S}=\left \{ 4 \right \}}\]
\end{align*}}
\question{$(E_4): x=\sqrt{x}+2$}
\reponse{\begin{align*}
\item $(E_5)$ s'écrit également $(\eu{x})^2+1-2\eu{x}=0$ en multipliant par $\eu{x}\neq 0$. On pose alors $X=\eu{x}$. $(E_5)$ s'écrit alors $X^2-2X+1=0$ qui possède une unique solution : $1$. \\On revient à l'inconnue de départ : $\eu{x}=1$, c'est-à-dire $x=0$. Ainsi \[\boxed{\mathcal{S}=\left \{ 0 \right \}}\]
\end{enumerate}
\end{align*}}
\question{$(E_5): \eu{x} +\eu{-x}=2$}
\reponse{}
}

%%% Fin exercice %%%

\uuid{aJ2m}
\niveau{PCSI}
\module{Analyse}
\chapitre{Sommes et produits de réels}
\sousChapitre{Manipulations et calculs}
%!TeX root=../../../encours.nouveau.tex
%%% Début exercice %%%

\duree{10}
\difficulte{1}
\auteur{Antoine Crouzet}
\datecreate{01/12/2024}
\titre{Premiers calculs}
\contenu{
\question{Soient $x$ un réel, et $0\leq p\leq n$ des entiers. Déterminer \[ \sum_{k=0}^p k,\quad \sum_{k=p}^n k,\quad \sum_{k=0}^n x^k \qeq \sum_{k=p}^n x^k.\]}
\reponse{On utilise les sommes usuelles. Si la somme ne commence pas à $0$, on peut utiliser la relation de Chasles pour s'y ramener :
\begin{align*}
 \sum_{k=0}^p k &= \frac{p(p+1)}{2}\\
 \sum_{k=p}^n k &= \sum_{k=0}^n k - \sum_{k=0}^{p-1} k \\
                &= \frac{n(n+1)}{2} - \frac{(p-1)p}{2} \\
 \sum_{k=0}^n x^k &= \left \{ \begin{array}{lll} \dfrac{1-x^{n+1}}{1-x} & \text{si} & x\neq 1 \\
                                                n+1 & \text{si} & x=1 \end{array}\right. \\
 \sum_{k=p}^n x^k &= \sum_{k=0}^n x^k - \sum_{k=0}^{p-1} x^k \\
                  &= \left \{ \begin{array}{lll} \dfrac{1-x^{n+1}}{1-x} - \dfrac{1-x^p}{1-x} & \text{si} & x\neq 1 \\
                                                (n+1) - p & \text{si} & x=1 \end{array}\right. \\
                  &= \left \{ \begin{array}{lll} \dfrac{x^p-x^{n+1}}{1-x} & \text{si} & x\neq  1 \\
                                                  n-p+1 & \text{si} & x=1 \end{array}\right.
\end{align*}}
}

%%% Fin exercice %%%

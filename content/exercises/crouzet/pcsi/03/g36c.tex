\uuid{g36c}
\niveau{PCSI}
\module{Analyse}
\chapitre{Sommes et produits de réels}
\sousChapitre{Manipulations et calculs}
%!TeX root=../../../encours.nouveau.tex
%%% Début exercice %%%

\duree{20}
\difficulte{1}
\auteur{Antoine Crouzet}
\datecreate{01/12/2024}
\titre{Des produits}
\contenu{
\question{Soit $n$ un entier naturel non nul. Calculer les produits suivants :
\[ \prod_{k=1}^n \frac{2k+3}{2k+1} \quad\quad \prod_{k=2}^n \left(1-\frac{1}{k^2}\right) \quad\quad \prod_{k=1}^n \left(1+\frac{1}{k}\right)^k.\]}
\reponse{Pour le premier, on remarque un produit télescopique :
\begin{align*}
  \prod_{k=1}^n \frac{2k+3}{2k+1} &= \prod_{k=1}^n \frac{2(k+1)+1}{2k+1}\\
  &= \frac{2(n+1)+1}{2\times 1+1} = \frac{2n+3}{3}
\end{align*}
Pour le deuxième, on ré-écrit pour faire apparaître deux produit télescopiques :
\begin{align*}
  \boxed{\prod_{k=2}^n \left(1-\frac{1}{k^2}\right)} &= \prod_{k=2}^n \frac{k^2-1}{k}\\
  &= \prod_{k=2}^n \frac{(k-1)(k+1)}{k\times k} \\
  &= \prod_{k=2}^n \frac{k-1}{k} \times \prod_{k=2}^n \frac{k+1}{k}\\
  &= \frac{1}{n}\times \frac{n+1}{2}=\boxed{\frac{n+1}{2n}}
\end{align*}
Pour le dernier produit, on fait apparaître un produit télescopique en ajoutant un terme :
\begin{align*}
  \boxed{\prod_{k=1}^n \left(1+\frac{1}{k}\right)^k} &= \prod_{k=1}^n \left(\frac{k+1}{k}\right)^k\\
  &=\prod_{k=1}^n \frac{(k+1)^k}{k^k}\\
  &= \prod_{k=1}^n \frac{(k+1)^{k+1}}{k^k} \times \frac{1}{k+1} \\
  &= \underbrace{\prod_{k=1}^n \frac{(k+1)^{k+1}}{k^k}}_{\text{telescopage}}\times \prod_{k=1}^n \frac{1}{k+1} \\
  &= \frac{(n+1)^{n+1}}{1^1} \times \frac{1}{(n+1)!} = \boxed{\frac{(n+1)^n}{n!}}
\end{align*}}
}

%%% Fin exercice %%%

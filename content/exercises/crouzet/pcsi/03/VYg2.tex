\uuid{VYg2}
\niveau{PCSI}
\module{Analyse}
\chapitre{Sommes et produits de réels}
\sousChapitre{Manipulations et calculs}
%!TeX root=../../../encours.nouveau.tex
%%% Début exercice %%%

\duree{5}
\difficulte{1}
\auteur{Antoine Crouzet}
\datecreate{01/12/2024}
\titre{Réécriture}
\contenu{
\texte{Réécrire les sommes suivantes à l'aide du symbole $\sum$ :}
\question{$1+3+5+\hdots+2021$.}
\reponse{Rapidement :
\begin{align*}
	1+3+5+\hdots+2021 &= \sum_{k=1}^{1010} 2k+1 \\
\end{align*}}
\question{$2+4+8+16+\hdots+1024$.}
\reponse{\begin{align*}
2+4+8+16+\hdots+1024 &= \sum_{k=1}^{10} 2^k \\
\end{align*}}
\question{$x^1-x^2+x^3-\hdots -x^{10}$	.}
\reponse{\begin{align*}
x^1-x^2+x^3-\hdots -x^{10} &= \sum_{k=1}^{10} (-1)^{k+1}x^k \\
\end{align*}}
\question{$\dfrac{1}{2}+\dfrac{2}{3}+\dfrac{3}{4}+\hdots + \dfrac{2021}{2022}$.}
\reponse{\begin{align*}
\dfrac{1}{2}+\dfrac{2}{3}+\dfrac{3}{4}+\hdots + \dfrac{2021}{2022} &= \sum_{k=1}^{2021} \frac{k}{k+1}
\end{align*}}
}

%%% Fin exercice %%%

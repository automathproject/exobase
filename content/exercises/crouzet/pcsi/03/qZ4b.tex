\uuid{qZ4b}
\niveau{PCSI}
\module{Analyse}
\chapitre{Sommes et produits de réels}
\sousChapitre{Manipulations et calculs}
%!TeX root=../../../encours.nouveau.tex
%%% Début exercice %%%

\duree{15}
\difficulte{1}
\auteur{Antoine Crouzet}
\datecreate{01/12/2024}
\titre{Des factorisations sympathiques}
\contenu{
\texte{Soient $x$ et $y$ deux réels.}
\question{Montrer que, pour tout entier naturel $n$, on a \[ x^n-y^n = (x-y)\sum_{k=0}^{n-1} x^{n-1-k}y^k \]}
\reponse{On part du membre de droite, et on va développer, puis faire un changement de variable :
	\begin{align*}
  (x-y) \sum_{k=0}^{n-1} x^{n-1-k}y^k &= x\sum_{k=0}^{n-1} x^{n-1-k}y^k - y\sum_{k=0}^{n-1} x^{n-1-k} y^k \\
																		  &= \sum_{k=0}^{n-1} x^{n-k}y^k - \underbrace{\sum_{k=0}^{n-1} x^{n-1-k}y^{k+1}}_{i=k+1} \\
																			&= \sum_{k=0}^{n-1} x^{n-k}y^k - \sum_{i=1}^n x^{n-i} y^i \\
																			&= x^n + \sum_{k=1}^{n-1} x^{n-k}y^k - \left(\sum_{i=1}^{n-1} x^{n-i} y^i + y^n\right)\\
																			&= x^n - y^n
	\end{align*}}
\question{En déduire que, pour tout entier naturel impair $n$, on a \[ x^n+y^n = (x+y)\sum_{k=0}^{n-1} (-1)^k x^{n-1-k}y^k \]}
\reponse{On utilise le cas précédent, en remplaçant $y$ par $-y$, et en utilisant le fait que $n$ est impair :
	\begin{align*}
		x^n - (-y)^n &= (x-(-y))\sum_{k=0}^{n-1} x^{n-1-k} (-y)^k \\
		&= (x+y)\sum_{k=0}^{n-1} (-1)^k x^{n-1-k} y^k
	\end{align*}
	Or, $n$ étant impair, $(-y)^n = (-1)^n y^n = -y^n$, d'où le résultat.}
}

%%% Fin exercice %%%

\uuid{kU9b}
\niveau{PCSI}
\module{Analyse}
\chapitre{Ensembles et applications}
%!TeX root=../../../encours.nouveau.tex
%%% Début exercice %%%

\duree{15}
\difficulte{2}
\auteur{Antoine Crouzet}
\datecreate{01/12/2024}
\titre{$\partie(\N)$ n'est pas dénombrable}
\contenu{
\question{Montrer qu'il n'existe pas de surjection de $\N$ dans $\partie(\N$).\\
En déduire que $\partie(\N)$ n'est pas dénombrable.

\textit{On pourra raisonner par l'absurde en introduisant une telle surjection $f$ et en posant \\$A=\left \{ x\in \N,\quad x\not \in f(x) \right \}$.}}
\reponse{\begin{align*}
Supposons qu'il existe une telle surjection, que l'on note $f$. Notons \[ A = \left \{ x\in \N,\quad x\notin f(x)\right\}. \]
Remarquons que $A$ est bien définie (puisque $f(x)\in \mathcal{P}(\N)$) et est une partie de $\N$.

Puisque $f$ est une surjection sur $\mathcal{P}(\N)$, il existe $x_0 \in \N$ tel que $f(x_0)=A$.  Montrons que ceci est une contradiction :
\begin{itemize}
  \item Si $x_0 \in A$, alors $x_0 \in f(x_0)=A$ et donc, par définition de $A$, $x_0\notin A$. C'est absurde.
  \item Si $x_0\notin A$, alors $x_0 \notin f(x_0)=A$ et donc, par définition de $A$, $x_0 \in A$. C'est absurde.
\end{itemize}
Dans tous les cas, l'existence de $x_0$ est impossible, d'où la contradiction : il n'existe pas de surjection de $\N$ dans $\partie(\N)$. Mais alors, il ne peut exister de bijection de $\N$ dans $\partie(\N)$ : ainsi, $\partie(\N)$ n'est pas dénombrable.

\begin{remarque}
  Cette démonstration, et cette manière de raisonner, est due à \textbf{Georg Cantor} (1845--1918).
\end{remarque}
\end{align*}}
}

%%% Fin exercice %%%

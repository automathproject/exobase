\uuid{hS6A}
\niveau{PCSI}
\module{Analyse}
\chapitre{Ensembles et applications}
%!TeX root=../../../encours.nouveau.tex
%%% Début exercice %%%

\duree{15}
\difficulte{3}
\auteur{Antoine Crouzet}
\datecreate{01/12/2024}
\titre{Des bijectivités compliquées}
\contenu{
\question{On note $E$ l'ensemble des suites $(u_n)$ géométriques, telles que $u_0\neq 0$.

Montrer que l'application $f:E\to \R*\times \R$ définie par 
\[ f( (u_n) ) = (u_0, u_1) \]
est bijective.}
\reponse{\begin{align*}
Soient deux suites $u$ et $v$ de $E$ tels que $f(u)=f(v)$. Par définition
	\[ (u_0, u_1) = (v_0,v_1) \iff u_0=v_0\qeq u_1=v_1. \]
	Remarquons donc que les suites $u$ et $v$ ont le même premier terme, mais également la même raison; en effet, puisque $u_0\in \R*$ et $v_0\in \R*$, on constate que 
	\[ \frac{v_1}{v_0} = \frac{u_1}{u_0}. \]
	Ce quotient donnant la raison des suites géométriques, on peut conclure que $u$ et $v$ ayant même premier terme et même raison, elles sont égales.	
	$f$ est donc injective.
	
	Elle est également surjective. Prenons deux réels $(a, b)\in \R*\times \R$. Posons la suite géométrique $u$ de premier terme $u_0=a\in \R*$ et de raison $q=\frac{b}{a}$ (qui a un sens car $a\neq 0$). Par construction, $u\in E$ et on a bien \[f(u) =(u_0, u_1) = (u_0, qu_0) = (a, b). \]
	Finalement $f$ est bien surjective, et donc bijective.
\end{align*}}
\question{Soit $E$ un ensemble, $A, B$ deux sous-ensembles de $E$. Soit $f:\partie(E)\to \partie(A)\times \partie(B)$ définie par 
\[ f(X) = (A\cap X, B\cap X).\]
\begin{enumerate}
\item Montrer que $f$ est injective si et seulement si $A\cup B = E$.
\item Montrer que $f$ est surjective si et seulement si $A\cap B = \vide$.
\item \`A quelle condition sur $A$ et $B$ $f$ est-elle bijective ? Si tel est le cas, déterminer $f^{-1}$.
\end{enumerate}}
\reponse{\begin{align*}
\begin{enumerate}
	\item Soient $X$ et $Y$ tels que $f(X)=f(Y)$. Alors
		\[ A\cap X = A\cap Y \qeq B\cap X=B\cap Y. \]
		Si $A\cup B = E$, alors les résultats précédents donnent 
		\[ (A\cap X) \cup (B\cap X) = (A\cap Y) \cup (B \cap Y) \]
		soit par distributivité :
		\[ (A\cup B)\cap X = (A\cup B)\cap Y \implies E\cap X = E\cap Y \implies X=Y. \]
	$f$ est injective.\\
	Supposons maintenant que $A\cup B \neq E$. Soit $C=E\backslash (A\cup B)\neq \vide$. Alors
	\[ f(C) = (\vide,\vide) \qeq f(\vide) = (\vide, \vide). \]
	Ainsi, $f(C)=f(\vide)$ et pourtant $C\neq \vide$ : $f$ n'est pas injective.
	\item Supposons $A\cap B = \vide$. Soit $Y \in \partie(A)$ et $Z\in \partie(B)$. Notons $X=Y\cup Z \in \partie(E)$. Par distributivité :
		\begin{itemize}
			\item $X\cap A = (Y\cap A)\cup (Z\cap A) = Y$ car $Y\subset A$ et $Z\subset B$ avec $A\cap B=\vide$.
			\item $X\cap B = (Y\cap B)\cup (Z\cap B) = Z$ car $Z\subset B$ et $Y\subset A$ avec $A\cap B=\vide$.
		\end{itemize}
		Ainsi, $f(X) = (Y,Z)$ : $f$ est donc surjective.
		
		Supposons $A\cap B \neq \vide$. Soit $x\in A\cap B$, et notons $Y=\{x\}$. $Y\in \partie(A)$. Prenons  $Z\in \partie(B)$ tel que $x\notin Z$. Alors le couple $(Y,Z)$ n'a pas d'antécédent par $f$. En effet, s'il existe $X\in \partie(E)$ tel que $f(X)=(Y,Z)$, cela implique que 
	\[ X \cap A = Y \qeq X\cap B = Z. \]
	Puisque $x \in Y$, nécessairement, puisque $X\cap A=Y$, $x\in X$. Mais alors $x\in X\cap B$ puisque $x\in B$ : c'est absurde puisque $x\notin Z$. $f$ n'est donc pas surjective.
	\item D'après les résultats précédents, $f$ est bijective si elle est à la fois injective et surjective. Donc $f$ est bijective si et seulement si $A\cap B =\vide$ et $A\cup B = E$, c'est-à-dire si $A$ et $B$ sont complémentaires : $B=\overline{A}$. 
	
	$f$ s'écrit alors $f:X\donne (A\cap X, \overline{A}\cap X)$.
Soit alors $(Y,Z)\in \partie{A}\times \partie{\overline{A}}$. Alors
\[ f(X) = (Y,Z) \iff A\cap X = Y \qeq \overline{A}\cap X = Z \iff X = Y \cup Z. \]
Ainsi, 
\[ \boxed{ f^{-1} : (Y,Z)\donne Y\cup Z.}\]
	\end{enumerate}
\end{align*}}
}

%%% Fin exercice %%%

\uuid{WaY8}
\niveau{PCSI}
\module{Analyse}
\chapitre{Ensembles et applications}
\sousChapitre{Fonctions}
%!TeX root=../../../encours.nouveau.tex
%%% Début exercice %%%

\duree{5}
\difficulte{1}
\auteur{Antoine Crouzet}
\datecreate{01/12/2024}
\titre{Image directe, image réciproque}
\contenu{
\question{Soit $f$ la fonction définie sur $\R$ par $f:x\mapsto \ln\left(x^2+1\right)$.
\\Déterminer l'image directe par $f$ de $\interff{-1 1}$. Déterminer l'image réciproque par $f$ de $\R^+$ et de $\interff{1 \ln(10)}$.}
\reponse{On raisonne par inégalité :
  \begin{align*}
    -1\leq x \leq 1 &\iff 0 \leq x^2 \leq 1 \\
    &\iff 1 \leq x^2+1 \leq 2 \\
    &\iff \ln(1) \leq \ln(x^2+1) \leq \ln(2) \text{ car $\ln$ est bijective.}
  \end{align*}
  Ainsi, \[ f\left(\interff{-1 1}\right)=\interff{0 \ln(2)}.\]
  Réciproquement :
  \begin{align*}
    1\leq f(x) \leq \ln(10) &\Leftrightarrow 1\leq \ln(x^2+1)\leq \ln(10) \\
    &\Leftrightarrow \E \leq x^2+1 \leq 10 \text{ car $\exp$ est bijective}\\
    &\Leftrightarrow \E - 1 \leq x^2 \leq 9 \\
    &\Leftrightarrow x \in \interff{-3 -\sqrt{\E-1}} \cup \interff{\sqrt{{\E-1}} 3}
  \end{align*}
  Ainsi, \[ f^{-1}\left(\interff{1 \ln(10)}\right) = \interff{-3 -\sqrt{\E-1}} \cup \interff{{\sqrt{\E-1}} 3} .\]
  
  Pour $\R+$ :
  \begin{align*}
  f(x) \geq 0 &\Leftrightarrow \ln(x^2+1)\geq 0\\
  &\Leftrightarrow x^2+1\geq 1 \\
  &\Leftrightarrow x^2 \geq 0 \text{ ce qui est vrai pour tout }x\in \R.	
  \end{align*}
  Ainsi, \[ f^{-1}\left(\R+\right) = \R.\]}
}

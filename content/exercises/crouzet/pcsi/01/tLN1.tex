\uuid{tLN1}
\niveau{PCSI}
\module{Analyse}
\chapitre{Éléments de logique et raisonnement}
\sousChapitre{Logiques}
%!TeX root=../../../encours.nouveau.tex

\duree{10}
\difficulte{1}
\auteur{Antoine Crouzet}
\datecreate{01/12/2024}
\titre{Négation et traducation}
\contenu{
\texte{Traduire en français, puis écrire la négation des propositions suivantes.}
\question{$\forall x\in \R+,\,\exists~y\in \R,\, y^2=x$.}
\reponse{\begin{align*}
Pour tout réel positif $x$, il existe un réel $y$ tel que $y^2=x$. C'est la définition d'une racine carrée. Sa négation : \[ \exists~x\in \R+,\, \forall y\in \R,\, y^2\neq x \]
\end{align*}}
\question{$\forall x>1,\,\exists~n\in \N,\,x^n\geq 2021$.}
\reponse{\begin{align*}
Pour tout $x>1$, il existe un entier $n$ tel que $x^n\geq 2021$. Sa négation : \[ \exists~x>1,\,\forall n\in \N,\, x^n<2021 \]
\end{align*}}
\question{$\forall x\in X,\,\forall y\in Y,\, xy=0 \implies (x=0 \ou y=0)$.}
\reponse{\begin{align*}
Pour tous éléments $x\in X$ et $y\in Y$, si $xy=0$ alors $x=0$ ou $y=0$. C'est le \og{} théorème du produit nul\fg{} (appelé \textbf{intégrité} du corps des réels). Sa négation : \[ \exists~x\in X,\, \exists~y\in Y,\, xy=0 \text{ et pourtant } x\neq 0 \text{ et } y\neq 0 \]
\end{align*}}
\question{$\exists!x \in E,\, f(x)=0$.}
\reponse{\begin{align*}
Il existe un unique élément $x\in E$ tel que $f(x)=0$. Cela traduit l'unicité de la solution de l'équation $f(x)=0$. Sa négation : \[ \forall x\in E,\, f(x)\neq 0 \text{ ou } \exists~x\in E,\exists~y\in E,\, x\neq y,\, f(x)=0,\, f(y)=0 \]
  \begin{attention}
La négation d'un \og{} il existe un unique \fg{} est \og{} il n'existe pas ou bien il en existe au moins 2\fg{}. 
  \end{attention}
\end{align*}}
}

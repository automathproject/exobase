\uuid{dPvd}
\niveau{PCSI}
\module{Analyse}
\chapitre{Éléments de logique et raisonnement}
\sousChapitre{Ensembles}
%!TeX root=../../../encours.nouveau.tex

\duree{5}
\difficulte{1}
\auteur{Antoine Crouzet}
\datecreate{01/12/2024}
\titre{Ensembles}
\contenu{
\question{\label{\formatexo{2}}
Soient les ensembles suivants : \[A = \anuplet{1 2 3 4 5 6 7} ,\quad  B = \anuplet{1 3 5 7 } , \quad C = \anuplet{2 4 6 } , \qeq D = \anuplet{3 6}\]

Déterminer $B\cap D, C\cap D, B\cup C, B \cup D$. Déterminer les complémentaires dans $A$ de $B, C$ et $D$.}
\reponse{\begin{align*}
Exercice rapide pour bien maitriser les différents concepts. On obtient :
\[B\cap D = \anuplet{3},\quad C\cap D = \anuplet{6},\quad B\cup C=\anuplet{1 2 3 4 5 6 7 }=A,\quad B\cup D = \anuplet{1 3 5 6 7 }\]
Enfin, dans $A$, on a :
\[\overline{B}=\anuplet{2 4 6}=C,\quad \overline{C}=\anuplet{1 3 5 7 }=B,\quad \overline{D}=\anuplet{1 2 4 5 7}\]
\end{align*}}
}

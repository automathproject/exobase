\uuid{0fy9}
\niveau{PCSI}
\module{Analyse}
\chapitre{Éléments de logique et raisonnement}
\sousChapitre{Logiques}
%!TeX root=../../../encours.nouveau.tex

\duree{10}
\difficulte{1}
\auteur{Antoine Crouzet}
\datecreate{01/12/2024}
\titre{Des équivalences}
\contenu{
\texte{Soient $\AA$, $\BB$ et $\CC$ trois propositions. Montrer les équivalences suivantes à l'aide de tables de vérité :}
\question{$((\AA \et \BB) \ou \CC) \Longleftrightarrow ((\AA \ou \CC) \et (\BB \ou \CC))$.}
\reponse{\begin{align*}
En écrivant les tables de vérités :
\[\begin{array}{|c|c|c|c|>{\columncolor{gray!20}}c|c|c|>{\columncolor{gray!20}}c|}\hline
 \AA & \BB & \CC & \AA \et \BB & (\AA \et \BB) \ou \CC & \AA \ou \CC & \BB \ou \CC & ((\AA \ou \CC) \et (\BB \ou \CC))   \\\hline
 V & V & V & V & V & V & V & V \\
 V & V & F & V & V & V & V & V \\
 V & F & V & F & V & V & V & V \\
 V & F & F & F & F & V & F & F \\
 F & V & V & F & V & V & V & V \\
 F & V & F & F & F & F & V & F \\
 F & F & V & F & V & V & V & V \\
 F & F & F & F & F & F & F & F \\\hline
 \end{array}
\]
\end{align*}}
\question{$((\AA \ou \BB) \et \CC) \Longleftrightarrow ((\AA \et \CC) \ou (\BB \et \CC))$.}
\reponse{\begin{align*}
\[\begin{array}{|c|c|c|c|>{\columncolor{gray!20}}c|c|c|>{\columncolor{gray!20}}c|}\hline
 \AA & \BB & \CC & \AA \ou \BB & (\AA \ou \BB) \et \CC & \AA \et \CC & \BB \et \CC & ((\AA \et \CC) \ou (\BB \et \CC))   \\\hline
 V & V & V & V & V & V & V & V \\
 V & V & F & V & F & F & F & F \\
 V & F & V & V & V & V & F & V \\
 V & F & F & V & F & F & F & F \\
 F & V & V & V & V & F & V & V \\
 F & V & F & V & F & F & F & F \\
 F & F & V & F & F & F & F & F \\
 F & F & F & F & F & F & F & F \\\hline
 \end{array}
\]
\end{align*}}
}

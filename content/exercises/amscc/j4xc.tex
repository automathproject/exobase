\uuid{j4xc}
\chapitre{Probabilité continue}
\niveau{L2}
\module{Probabilité et statistique}
\sousChapitre{Densité de probabilité}
\titre{Etude d'une loi de probabilité}
\theme{variables aléatoires à densité}
\auteur{Maxime NGUYEN}
\datecreate{2023-09-13}
\organisation{AMSCC}

\difficulte{1}
\contenu{

\texte{
On considère la fonction $f$ définie sur $\R$ par 
$$f(x)=\begin{cases} \lambda x & \text{ si } x\in[0,1] \\  
        0 &\text{ si } x\notin [0,1].
       \end{cases}
$$
}
\begin{enumerate}
 \item \question{ Calculer $\lambda$ pour que $f$ soit la densité d'une variable aléatoire absolument continue $X$. }
 \indication{Pour que $f$ soit une densité, il faut que $f$ soit positive sur $\R$ et que $\int_\R f(x) \dx =1$}
 \reponse{ Pour que $f$ soit une densité, il faut que $f$ soit positive sur $\R$ (ce qui est le cas ici) et que $\int_\R f(x) \dx =1$. Or
 \[ \int_\R f(x) \dx=\int_0^1 \lambda x \dx  =\left[ \frac{\lambda}{2}x^2\right]_0^1 =\frac{\lambda}{2}\]
 donc $\lambda=2$.
 }
 
 \item \question{ Déterminer $\prob\left(X\leq \frac{1}{3}\right)$ et $\prob\left(X\leq \frac{2}{3} \mid X>\frac{1}{3}\right)$. }
 \indication{Le deuxième calcul est une probabilité conditionnelle : appliquer la formule de Bayes. }
 \reponse{
 \begin{align*}
  &\prob\left(X\leq \frac{1}{3}\right)=\int_{-\infty}^\frac{1}{3} f(x)\dx =[x^2]_0^\frac{1}{3}=\frac{1}{9} \\
  &\prob\left(X\leq \frac{2}{3} \mid X>\frac{1}{3}\right)=\frac{\prob\left(\frac{1}{3}<X\leq \frac{2}{3}\right)}{\prob\left(X>\frac{1}{3}\right)}
  = \frac{\int_\frac{1}{3}^\frac{2}{3}2x \dx}{1-\prob\left(X\leq \frac{1}{3}\right)}
  =\frac{[x^2]_\frac{1}{3}^\frac{2}{3}}{1-\frac{1}{9}}
  = \frac{\frac{1}{3}}{\frac{8}{9}}
  =\frac{3}{8}
 \end{align*}
 }
 
 \item \question{ Calculer $\E(X)$ et $\var(X)$. }
 \indication{On applique la définition pour calculer $\E(X)$. Pour le calcul de $\var(X)$, on calcule dans un premier temps $\E(X^2)$ à l'aide du théorème de transfert. }
 \reponse{
 $$\E(X)=\int_\R xf(x)\dx =\int_0^1 2x^2 \dx =\left[ \frac{2}{3}x^3\right]_0^1=\frac{2}{3}$$
 \begin{align*}
  \var(X) &= \E(X^2)-\E(X)^2
  = \int_\R x^2f(x)\dx  -\left(\frac{2}{3}\right)^2 
  = \left[ \frac{1}{2}x^4\right]_0^1-\frac{4}{9}
  =\frac{1}{2}-\frac{4}{9}
  =\frac{1}{18}
 \end{align*}
 }
 
\end{enumerate}
}
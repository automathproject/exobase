\uuid{JPjn}
\chapitre{Probabilité continue}
\niveau{L2}
\module{Probabilité et statistique}
\sousChapitre{Densité de probabilité}
\titre{Densité, fonction de répartition}
\theme{variables aléatoires à densité}
\auteur{}
\datecreate{2022-09-21}
\organisation{AMSCC}
\difficulte{}
\contenu{

\texte{ On considère la fonction $f$ définie par  :
	\begin{align*}
		f(x)=
		\begin{cases}
			\frac{4}{3}(1-x)^{\frac{1}{3}} & \text{ si } 0\leq x \leq 1 \\
			0 & \text{ sinon.}
		\end{cases}
	\end{align*}
}
\begin{enumerate}
	\item \question{Montrer que $f$ est une densité d'une variable aléatoire $Y$.}
	\reponse{Il faut et il suffit de vérifier que $f$ est positive intégrable et $\int_\mathbb{R} f=1$. Le premier point est immédiat. La fonction $f$ est intégrable sur $\mathbb{R}$ car elle est continue par morceaux à support compact.
		Enfin, on a
		\begin{align*}
			\int_\mathbb{R} f(x)dx 
			= \int_0^1 \frac{4}{3} (1-x)^{1/3} dx
			= \left[ -(1-x)^{4/3}\right]_0^1
			=1.
	\end{align*}}
	\item \question{Déterminer la fonction de répartition $F$ de la variable $Y$.}
	\reponse{Par définition, pour tout $y\in\mathbb{R}$, on a $F_Y(y)=\int_{-\infty}^y f(t)dt$. Alors
		\begin{itemize}
			\item si $y<0$, $F_Y(y)=0$,
			\item si $0\leq y \leq 1$, 
			$F_Y(y)=\int_0^y \frac{4}{3} (1-x)^{1/3} dx
			= \left[ -(1-x)^{4/3}\right]_0^y
			=1-(1-y)^{4/3}$
			\item si $y>1$, $F_Y(y)=1$.
	\end{itemize}}
	\item \question{Calculer l'espérance de la variable $Y$.}
	\reponse{L'espérance de $Y$ se calcule de la manière suivante:
		\[ \mathbb{E}(Y)=
		\int_\mathbb{R} yf(y)dy
		= \int_0^1 \frac{4}{3}y (1-y)^{1/3} dy,\]
		% On effectue ensuite le changement de variable $t=1-y$:
		% \begin{align*}
		% \mathbb{E}(Y)&= \int_0^1 \frac{4}{3} (1-t) t^{\frac{1}{3}} dy
		% = \int_0^1 \frac{4}{3}t^{\frac{1}{3}}-\frac{4}{3}t^{\frac{4}{3}} dy
		% = \left[t^{\frac{4}{3}}-\frac{4}{7}t^{\frac{7}{3}} \right]_0^1
		% =1-\frac{4}{7}=\frac{3}{7}.
		% \end{align*}
		et par intégration par parties,
		\[ \E(Y)=\left[-y(1-y)^{4/3}\right]_0^1 + \int_0^1 (1-y)^{4/3}dy
		=\left[\frac{-3}{7}(1-y)^{7/3}\right]_0^1=\frac{3}{7}.
		\]}
	\item \question{Calculer la probabilité de l'événement $[0.488< Y \leq 1.2]$.}
	\reponse{Méthode 1:
		\begin{align*}
			\p(0.488< Y < 1.2)&= \int_{0.488}^{1.2} f(y) dy
			= \int_{0.488}^{1.2} \frac{4}{3} (1-y)^{1/3} dy
			=\left[ -(1-y)^{4/3}\right]_{0.488}^1 \\
			&=(1-0.488)^{1/3}
			=0.8
		\end{align*}
		Méthode 2:
		\begin{align*}
			\p(0.488< Y < 1.2)&= F_Y(1.2)-F_Y(0.488) \\
			&=1-[1-(1-0.488)^{1/3}] \\
			&=(1-0.488)^{1/3} \\
			&=0.8
	\end{align*}}
\end{enumerate}}

\uuid{TpFC}
\chapitre{Probabilité discrète}
\niveau{L2}
\module{Probabilité et statistique}
\sousChapitre{Lois de distributions}
\titre{Loi géométrique}
\theme{probabilités}
\auteur{}
\datecreate{2023-02-07}
\organisation{AMSCC}
\difficulte{}
\contenu{

\texte{ 	On note $p \in ]0;1[$ et $X$ une variable aléatoire suivant une loi géométrique $\mathcal{G}(p)$.  }
	\begin{enumerate}
		\item \question{ Déterminer la fonction de répartition de $X$. }
		\reponse{ $\PP(X \leq n) = 1-q^n$, $\PP(X > n)= q^{n}$ }
		\item \question{ Vérifier que la loi géométrique est une loi sans mémoire. }
	\end{enumerate}
}

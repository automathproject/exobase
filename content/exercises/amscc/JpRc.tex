\uuid{JpRc}
\chapitre{Probabilité discrète}
\niveau{L2}
\module{Probabilité et statistique}
\sousChapitre{Variable aléatoire discrète}
\titre{Loi d'une variable aléatoire discrète}
\theme{variables aléatoires discrètes}
\auteur{}
\datecreate{2023-07-05}
\organisation{AMSCC}



\difficulte{1}
\contenu{
    \question{ Soit $X$ une variable aléatoire discrète telle que $X(\Omega)=\{3,4,5,6\}$ telle que : 
$$\prob(X<5)=\frac{1}{6}, \quad \prob(X>5)=\frac{1}{2}, \quad \prob(X\leq 3)=\prob(X=4).$$
Déterminer la loi de $X$ et calculer son espérance.
    }
    
    \reponse{
        \begin{itemize}
    \item $\prob(X>5)=\prob(X=6)=\frac{1}{2}$ 
 \item $1=\prob(X<5)+\prob(X=5)+\prob(X>5)$ donc $\prob(X=5)=1-\frac{1}{6}-\frac{1}{2}=\frac{1}{3}$
 \item $\prob(X\leq 3)=\prob(X=3)=\prob(X=4)$ et $\frac{1}{6}=\prob(X<5)=\prob(X=3)+\prob(X=4)$ donc $\prob(X=3)=\prob(X=4)=\frac{1}{12}$
\end{itemize}
On a ainsi déterminé la loi de $X$ : \quad 
\begin{tabular}{|c|c|c|c|c|}
\hline
 $\omega$ & 3 & 4 & 5 & 6 \\
\hline
 $\prob(\omega)$ & $\frac{1}{12}$ & $\frac{1}{12}$ & $\frac{1}{3}$ & $\frac{1}{2}$ \\
\hline
\end{tabular}
L'espérance de $X$ est:
\[ \E(X)=3\times \prob(X=3)+4\times\prob(X=4)+5\times\prob(X=5)+6\times\prob(X=6) = \frac{21}{4}=5.25\]
}
}
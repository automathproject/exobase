\uuid{zmLP}
\chapitre{Continuité, limite et étude de fonctions réelles}
\niveau{L1}
\module{Analyse}
\sousChapitre{Fonctions équivalentes, fonctions négligeables}
\titre{Equivalents simples}
\theme{limite}
\auteur{}
\datecreate{2023-05-11}
\organisation{AMSCC}
\difficulte{}
\contenu{


\texte{Trouver un équivalent simple pour chaque fonction suivante :}
\colonnes{\solution}{3}{1}
%\begin{multicols}{3}
\begin{enumerate}
	\item \question{$\frac{x^2+2x+1}{x^3+x+2} \ \text{ en } 0$}
	\reponse{Réponse : $\frac{1}{2}$ \\ 
		Lorsque $x \to 0$, le numérateur tend vers $(0)^2+2(0)+1 = 1$ et le dénominateur vers $(0)^3+(0)+2 = 2$. Donc la fonction est équivalente à $\frac{1}{2}$ en $0$.}
	
	\item \question{$\frac{(x+1)^3}{(2x^4-3x^2+5)^2} \ \text{ en } +\infty$}
	\reponse{Réponse : $\frac{1}{4x^5}$ \\
		Quand $x \to +\infty$, on a $(x+1)^3 \sim x^3$ et $(2x^4-3x^2+5)^2 \sim (2x^4)^2 = 4x^8$. Donc la fonction est équivalente à $\frac{x^3}{4x^8} = \frac{1}{4x^5}$.}
	
	\item \question{$\sin x + \cos x \ \text{ en } 0$}
	\reponse{Réponse : $1$ \\
		En $0$, $\sin(0) = 0$ et $\cos(0) = 1$. Plus précisément, $\sin x \sim x$ et $\cos x \sim 1-\frac{x^2}{2}$ en $0$, donc $\sin x + \cos x \sim 1$ car le terme constant est dominant.}
	
	\item \question{$ \sqrt{\frac{1}{x^2}+1} \ \text{ en } +\infty$}
	\reponse{Réponse : $1$ \\
		Quand $x \to +\infty$, $\frac{1}{x^2} \to 0$, donc $\frac{1}{x^2}+1 \sim 1$. Par continuité de la racine carrée, $\sqrt{\frac{1}{x^2}+1} \sim \sqrt{1} = 1$.}
	
	\item \question{$1-e^{{x}} \ \text{ en } 0$}
	\reponse{Réponse : $-x$ \\
		En développant $e^x$ en $0$ : $e^x = 1 + x + o(x)$. Donc $1-e^x = 1-(1+x+o(x)) = -x + o(x)$.}
	
	\item \question{$1-e^{\frac{1}{x^2}} \ \text{ en } +\infty$}
	\reponse{Réponse : $-\frac{1}{x^2}$ \\
		Quand $x \to +\infty$, $\frac{1}{x^2} \to 0$. En posant $t = \frac{1}{x^2}$, on a $e^t = 1 + t + o(t)$. Donc $1-e^{\frac{1}{x^2}} = -\frac{1}{x^2} + o(\frac{1}{x^2})$.}
	
	\item \question{$ x\sin\left(\frac{1}{x}\right) \ \text{ en } +\infty$}
	\reponse{Réponse : $1$ \\
		Quand $x \to +\infty$, $\frac{1}{x} \to 0$. En posant $t = \frac{1}{x}$, on a $\sin(t) \sim t$. Donc $x\sin\left(\frac{1}{x}\right) \sim x \cdot \frac{1}{x} = 1$.}
	
	\item \question{$ \frac{\sqrt{x}+1}{2x^2+x} \ \text{ en } +\infty$}
	\reponse{Réponse : $\frac{1}{2x^{3/2}}$ \\
		Pour $x \to +\infty$, les termes dominants sont $\sqrt{x}$ au numérateur et $2x^2$ au dénominateur. Donc $\frac{\sqrt{x}+1}{2x^2+x} \sim \frac{\sqrt{x}}{2x^2} = \frac{1}{2x^{3/2}}$.}
	
	\item \question{$ \sqrt{x} \ln\left(1+\frac{1}{\sqrt{x}}\right) \ \text{ en } +\infty$}
	\reponse{Réponse : $1$ \\
		Pour $x \to +\infty$, $\frac{1}{\sqrt{x}} \to 0$. Avec $t = \frac{1}{\sqrt{x}}$, on a $\ln(1+t) \sim t$. Donc $\sqrt{x} \ln\left(1+\frac{1}{\sqrt{x}}\right) \sim \sqrt{x} \cdot \frac{1}{\sqrt{x}} = 1$.}
	
	\item \question{$ e^x+\sin(x^3) \ \text{ en } 0$}
	\reponse{Réponse : $1$ \\
		En $0$, $e^0 = 1$ et $\sin(0) = 0$. Plus précisément, $e^x \sim 1 + x$ et $\sin(x^3) \sim x^3$ au voisinage de $0$. Donc $e^x+\sin(x^3) \sim 1$.}
	
	\item \question{$ \ln(\cos(x)) \ \text{ en } 0 $}
	\reponse{Réponse : $-\frac{x^2}{2}$ \\
		Développons $\cos(x) \sim 1 - \frac{x^2}{2}$ en $0$. En utilisant $\ln(1+t) \sim t$ pour $t$ proche de $0$ et en posant $t = \cos(x) - 1 \sim -\frac{x^2}{2}$, on obtient $\ln(\cos(x)) \sim -\frac{x^2}{2}$.}
\end{enumerate}
\fincolonnes{\solution}{3}{1}
%\endcolumn
}

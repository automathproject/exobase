\uuid{ZMBI}
\chapitre{Probabilité discrète}
\niveau{L2}
\module{Probabilité et statistique}
\sousChapitre{Approximation de loi}
\titre{Estimation par inégalité}
\theme{variables aléatoires discrètes, approximation de loi}
\auteur{Maxime Nguyen}
\datecreate{2023-09-18}
\organisation{AMSCC}

\difficulte{}
\contenu{

\texte{ On sait qu'il y a à peu près une chance sur deux d'avoir une fille à chaque naissance, mais on veut avoir une estimation plus précise de cette probabilité $p$. \\
On regarde ce qui se passe sur $n$ naissances, et on appelle $S_n$ le nombre de filles parmi les $n$ naissances. }
\begin{enumerate}
	\item \question{ Que vaut la variable $Y_n$ égale à la proportion de filles dans les naissances ? Donner son espérance et sa variance. }
	\reponse{ $Y_n=\frac{S_n}{n}$, où $S_n$ est une variable aléatoire de loi $\mathcal{B}(n,p)$. On a donc 
		\[\E(Y_n)=\frac{1}{n}\E(S_n)=\frac{1}{n}\times np=p \text{ et } \sigma^2(X)=\frac{1}{n^2}\sigma^2(S_n)=\frac{1}{n^2}\times np(1-p)=\frac{p(1-p)}{n}.\]
	}
	
	\item \question{ Soit $a>0$. Montrer que $\prob(|Y_n-p|\geq a)\leq \frac{1}{4na^2}$. }
	\reponse{ Soit $a>0$. On applique l'inégalité de Bienaymé-Tchebychev à $Y_n$:
		\[ \prob(|Y_n-p|\geq a)\leq \frac{\sigma^2(Y_n)}{a^2}=\frac{p(1-p)}{a^2n} \leq \frac{1}{4na^2},\]
		en utilisant l'indication donnée dans l'énoncé.}
	
	\item \question{ Déterminer un nombre $n$ d'observations à faire pour que l'on puisse déduire de $Y_n$, avec moins de $1$\% de chances de se tromper, que $Y_n-0.01\leq p\leq Y_n+0.01$. }
	\reponse{On souhaite avoir $\prob(Y_n-0.01\leq p \leq Y_n+0.01)\geq 0.99$. Or
		\begin{align*}
		\prob(Y_n-0.01\leq p \leq Y_n+0.01)
		&=\prob(-0.01\leq p-Y_n \leq 0.01)
		=\prob(|Y_n-p|\leq 0.01)\\
		&=1-\prob(|Y_n-p|\geq 0.01)
		\end{align*}
		On souhaite donc avoir $\prob(|Y_n-p|\geq 0.01)\leq 0.01$.
		Or par la question 2., avec $a=0.01$, on a l'inégalité:
		\[\prob(|Y_n-p|\geq a)\leq \frac{\sigma^2(Y_n)}{a^2}\leq \frac{1}{4n\times(0.01)^2}. \]
		On prend donc $n$ tel que $\frac{1}{4n\times (0.01)^2}\leq 0.01$, c'est-à-dire $n\geq 250\ 000$.}
	
	\item \question{ Parmi $\nombre{250000}$ naissances, combien faut-il, au minimum, observer de naissances de filles pour conclure avec $99$\% de chances de ne pas se tromper, qu'il naît plus de filles que de garçons ? }
	\reponse{ Par la question 3., on sait que $\prob(p\in[Y_n-0.01;Y_n+0.01])\geq 0.99$. \\
		Soit $X_0$ le nombre de filles observées.
		Avec moins de $1$\% de chances de se tromper, on veut que $p>\frac{1}{2}$, c'est-à-dire $\frac{X_0}{n}-0.01>\frac{1}{2}$, donc $X_0>125\ 250$. \\
		Il faut donc observer au minimum $500$ filles de plus que de garçons pour être sûr à $99$\% qu'il y a plus de filles que de garçons.}
	
\end{enumerate}
}
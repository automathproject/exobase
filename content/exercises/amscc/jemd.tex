\uuid{jemd}
\chapitre{Probabilité discrète}
\niveau{L2}
\module{Probabilité et statistique}
\sousChapitre{Estimateur}
\titre{Maximum de vraisemblance pour une loi géométrique}
\theme{estimateurs, loi géométrique, maximum de vraisemblance}
\auteur{Maxime NGUYEN}
\datecreate{2022-10-04}
\organisation{AMSCC}
\difficulte{}
\contenu{

\texte{ 	On rappelle qu'une variable $X$ suit une loi géométrique de paramètre $p \in ]0;1[$ si $X$ est à valeurs dans $\N^*=\{1;2;3;...\}$ et si pour tout $k \in \N^*$, 
$$\PP(X=k)=p(1-p)^{k-1}$$

On cherche à estimer ce paramètre $p$ à partir d'un échantillon. }

\begin{enumerate}
	\item \texte{ On considère un échantillon $(X_1,X_2,X_3,X_4,X_5)$  ayant pour loi mère une loi géométrique de paramètre $p$ et on suppose que la suite  $(3;4;4;2;3)$ constitue une réalisation de cet échantillon. }
	\begin{enumerate}
		\item \question{ Exprimer la fonction de vraisemblance associée à cet échantillon. }
		\reponse{ D'après le cours, la fonction de vraisemblance associée à cet échantillon est donnée par :
		\begin{align*}
			L(p) &= \PP(X_1=3) \times \PP(X_2=4) \times \PP(X_3=4) \times \PP(X_4=2) \times \PP(X_5=3) \\
			&= p(1-p)^2 \times p(1-p)^3 \times p(1-p)^3 \times p(1-p) \times p(1-p)^2 \\
			&= p^5(1-p)^{11}
		\end{align*}
		}
		\item \question{ En déduire une estimation de $p$ par la méthode du maximum de vraisemblance.  }
		\reponse{ On cherche à maximiser la fonction $L$ sur $]0;1[$. Pour cela, on calcule la dérivée de $L$ :
		\begin{align*}
			L'(p) &= 5p^4(1-p)^{11} - 11p^5(1-p)^{10} \\
			&= p^4(1-p)^{10}(5-11p)
		\end{align*}
		La fonction $L$ est dérivable sur $]0;1[$ et $L'(p)=0$ si et seulement si $p=0$, $p=1$ ou $p=\frac{5}{11}$. \\
		Or, $L(0)=0$, $L(1)=0$ et $L(\frac{5}{11})>0$. \\
		Donc, $L$ admet un maximum en $p=\frac{5}{11}$.  

		En conclusion, la valeur la plus vraisemblable pour $p$ est $\frac{5}{11}$. Il s'agit donc d'une estimation du paramètre $p$ par la méthode du maximum de vraisemblance. 
		}
	\end{enumerate}
	
	\item \texte{ Afin de déterminer un estimateur de $p$, on considère maintenant un $n$-échantillon $(X_1,...,X_n)$ ayant pour loi mère une loi géométrique de paramètre $p$ et $n$ entiers non nuls $(x_1,...,x_n)$ constituant une réalisation de cet échantillon.  }	 
	\begin{enumerate}
		\item \question{ Exprimer la fonction de vraisemblance associée à cet échantillon. }
		\reponse{
		\begin{align*}
			L(x_1,...,x_n,p) &= \PP(X_1=x_1) \times ... \times \PP(X_n=x_n) \\
			&= p(1-p)^{x_1-1} \times ... \times p(1-p)^{x_n-1} \\
			&= p^n(1-p)^{x_1+...+x_n-n}
		\end{align*}
		}
		\item \question{ En utilisant la méthode du maximum de vraisemblance, déterminer un estimateur du paramètre $p$. }
		\reponse{
		On cherche à maximiser la fonction $L$ sur $]0;1[$. Pour cela, on calcule la dérivée partielle de $L$ par rapport à $p$ : 
		\begin{align*}
			\frac{\partial L}{\partial p}(x_1,...,x_n,p) &= np^{n-1}(1-p)^{x_1+...+x_n-n} - p^n(x_1+...+x_n-n)(1-p)^{x_1+...+x_n-n-1} \\
			&= p^{n-1}(1-p)^{x_1+...+x_n-n-1}(n-(x_1+...+x_n-n)p)
		\end{align*}
		La fonction $L$ est dérivable sur $]0;1[$ et $\frac{\partial L}{\partial p}(p)=0$ si et seulement si $p=0$, $p=1$ ou $p=\frac{n}{\sum\limits_{i=1}^n x_i}$. \\
		Or, $L(0)=0$, $L(1)=0$ et $L\left(\frac{n}{\sum\limits_{i=1}^n x_i}\right)>0$. \\
		Donc, $L$ admet un maximum en $p=\frac{n}{\sum\limits_{i=1}^n x_i}$. 

		Ceci étant vrai pour toute réalisation $(x_1,...,x_n)$ de l'échantillon, on en déduit que $\widehat{p}=\frac{n}{\sum\limits_{i=1}^n X_i}$ est un estimateur du paramètre $p$ par la méthode du maximum de vraisemblance. 
		}
	\end{enumerate}
\end{enumerate}
}

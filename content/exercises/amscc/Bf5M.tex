\uuid{Bf5M}
\chapitre{Probabilité discrète}
\niveau{L2}
\module{Probabilité et statistique}
\sousChapitre{Variable aléatoire discrète}
\titre{Lois usuelles}
\theme{variables aléatoires discrètes}
\auteur{}
\datecreate{2023-02-07}
\organisation{AMSCC}
\difficulte{}
\contenu{

\texte{ 	Un examen comporte un écrit et un oral. }

 \texte{ A l'écrit, il y a dix questions et chaque candidat a le choix entre 4 réponses (a, b, c ou d), à chaque question. Il y a une seule bonne réponse par question.
 	
		Un candidat décide d'une stratégie discutable : il va répondre au hasard à chacune des questions par une seule et unique réponse.
		Soit $X$ la variable aléatoire qui donne le nombre de réponses exactes à l'ensemble des 10 questions. }
		\begin{enumerate}
			\item \question{ Quelle est la loi de $X$ ? Calculer $\EX$, $V(X)$ et $\PP(X < 8)$. }
			\item \question{ Quelle est la probabilité que ce candidat soit reçu à l'examen, sachant que l'examen est réussi si on a répondu correctement à au moins six questions ? }
		\end{enumerate}
%		\item L'oral de l'examen comporte 25 sujets possibles. Le candidat interrogé tire 3 sujets au hasard ; parmi ces trois
%		sujets, il choisit le sujet qu'il souhaite traiter. Un candidat a révisé seulement 15 sujets.
%		On appelle $Y$ la variable aléatoire donnant le nombre de sujets révisés par ce candidat parmi les 3 sujets tirés. 
%		\begin{enumerate}
%			\item Quelle est la loi de probabilité de Y ?
%			\item Quelle est la probabilité que le candidat tire au moins un sujet révisé ?
%		\end{enumerate}
}

\uuid{Hhtu}
\chapitre{Probabilité discrète}
\niveau{L2}
\module{Probabilité et statistique}
\sousChapitre{Probabilité et dénombrement}
\titre{Probabilité d'événements}
\theme{probabilités}
\auteur{ }
\datecreate{2023-08-30}
\organisation{AMSCC}
%Proba360

\difficulte{2}
\contenu{
	\texte{ Soit $(\Omega, \mathcal{T}, \prob)$ un espace probabilisé et trois événements $A$, $B$ et $C$ tels que : 
		
	$$\prob(A) = 0.3 \quad , \quad \prob(B) = 0.4 \quad , \quad \prob(A \cap B) = 0.2 \quad , \quad \prob(C \mid A \cap B) = 0.5$$
			
		Calculer, si possible, les probabilités suivantes : }

\colonnes{\solution}{2}{1}
\begin{enumerate}
	\item \question{ $\prob(A \cup B)$ ; }
	\reponse{
		Par la formule des probabilités pour l'union :
		$$ \prob(A \cup B) = \prob(A) + \prob(B) - \prob(A \cap B) $$
		$$ \prob(A \cup B) = 0.3 + 0.4 - 0.2 = 0.5 $$
	}
	\item \question{ $\prob(A \mid B)$ ; }
	\reponse{
		Par la définition d'une probabilité conditionnelle :
		$$ \prob(A \mid B) = \frac{\prob(A \cap B)}{\prob(B)} $$
		$$ \prob(A \mid B) = \frac{0.2}{0.4} = 0.5 $$
	}
	\item \question{ $\prob(\overline{A} \mid B)$ ; }
	\reponse{
		La probabilité de l'événement complémentaire sous une condition :
		$$ \prob(\overline{A} \mid B) = 1 - \prob(A \mid B) $$
		$$ \prob(\overline{A} \mid B) = 1 - 0.5 = 0.5 $$
	}
	\item \question{ $\prob({A} \mid \overline{B})$ ; }
	\reponse{
		On calcule d'abord $\prob(\overline{B}) = 1 - \prob(B) = 1 - 0.4 = 0.6$. \\
		Puis, $\prob(A \cap \overline{B}) = \prob(A) - \prob(A \cap B) = 0.3 - 0.2 = 0.1$.
		$$ \prob(A \mid \overline{B}) = \frac{\prob(A \cap \overline{B})}{\prob(\overline{B})} $$
		$$ \prob(A \mid \overline{B}) = \frac{0.1}{0.6} = \frac{1}{6} $$
	}
	\item \question{ $\prob(B \mid A)$ ; }
	\reponse{
		Par la définition d'une probabilité conditionnelle :
		$$ \prob(B \mid A) = \frac{\prob(B \cap A)}{\prob(A)} $$
		$$ \prob(B \mid A) = \frac{0.2}{0.3} = \frac{2}{3} $$
	}
	\item \question{ $\prob(A \cap B \cap C)$ ; }
	\reponse{
		On utilise la formule des probabilités composées :
		$$ \prob(A \cap B \cap C) = \prob(C \mid A \cap B) \times \prob(A \cap B) $$
		$$ \prob(A \cap B \cap C) = 0.5 \times 0.2 = 0.1 $$
	}
	\item \question{ $\prob(\overline{A} \cap \overline{B})$ ;}
	\reponse{
		D'après les lois de De Morgan :
		$$ \prob(\overline{A} \cap \overline{B}) = \prob(\overline{A \cup B}) = 1 - \prob(A \cup B) $$
		$$ \prob(\overline{A} \cap \overline{B}) = 1 - 0.5 = 0.5 $$
	}
	\item \question{ $\prob(\overline{A} \cup \overline{B})$.}
	\reponse{
		D'après les lois de De Morgan :
		$$ \prob(\overline{A} \cup \overline{B}) = \prob(\overline{A \cap B}) = 1 - \prob(A \cap B) $$
		$$ \prob(\overline{A} \cup \overline{B}) = 1 - 0.2 = 0.8 $$
	}
\end{enumerate}
\fincolonnes{\solution}{2}{1}
}

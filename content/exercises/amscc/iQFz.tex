\uuid{iQFz}
\chapitre{Série entière}
\niveau{L2}
\module{Analyse}
\sousChapitre{Autre}
\titre{Etude de séries entières}
\theme{séries entières}
\auteur{}
\datecreate{2023-06-14}
\organisation{AMSCC}
\variante{MWOk}

\difficulte{}
\contenu{
	\begin{enumerate}
	\item \texte{ Déterminer le rayon de convergence des séries entières réelles suivantes : }
	\begin{enumerate}
		\item \question{ $\displaystyle \sum_{n \geq 0} \frac{3n^2-1}{2^{2n+1}\sqrt{n+1}} x^n$ }
		\reponse{On pose $u_n(x) = \frac{3n^2-1}{2^{2n+1}\sqrt{n+1}} x^n$ et on utilise le théorème de d'Alembert : 
			\begin{align*}
			\frac{|u_{n+1}(x)|}{|u_n(x)|} &= \frac{(3(n+1)^2-1) \times 2^{2n+1}\sqrt{n+1}}{\left(2^{2(n+1)+1}\sqrt{n+1+1}\right)(3n^2-1)}\frac{|x^{n+1}|}{|x^n|} \\
			&= \frac{1}{2^2}\sqrt{\frac{n+1}{n+2}} \frac{3n^2+6n+3-1}{3n^2-1} |x| \\
			&\xrightarrow[n\to+\infty]{} \frac{1}{4}|x|
			\end{align*}	
			Or $ \frac{1}{4}|x| <1 \iff |x| < 4$ donc le rayon de convergence est $R=4$.
		}
		\item  \question{ $\displaystyle \sum_{n \geq 1} 9^n \left(\frac{n-1}{n}\right)^nx^{n}$ }
		\reponse{On pose $u_n(x) = 9^n \left(\frac{n-1}{n}\right)^nx^{2n}$ et on utilise le théorème de Cauchy : 
			\begin{align*}
			\left(|u_n(x)|\right)^{\frac{1}{n}} &= \left(9^n \left(\frac{n-1}{n}\right)^n|x|^{2n}\right)^{\frac{1}{n}} \\
			&= 9 \left(\frac{n-1}{n}\right) |x| \\
			&\xrightarrow[n\to+\infty]{} 9|x|
			\end{align*}	
			Or $ 9|x| <1 \iff |x| < \frac{1}{9}$ donc le rayon de convergence est $R=\frac{1}{9}$.
		}
	\end{enumerate}
	\item \question{ Que peut-on dire de la série $\displaystyle \sum_{n \geq 0} \frac{3n^2-1}{2^{2n+1}\sqrt{n+1}} \left(-\frac{9}{8}\right)^n$ ? Justifier. }
	\reponse{On reconnaît la série entière de la question 1.(a) évaluée en $x = -\frac{9}{8}$. Or $-\frac{9}{8} \in ]-4;4[$ donc la série est absolument convergente.}
	\item \question{ Que peut-on dire de la série $\displaystyle \sum_{n \geq 1} 9^n \left(\frac{2n-2}{3n}\right)^n$ ? Justifier. }
	\reponse{On reconnaît la série entière de la question 1.(b) évaluée en $x = \frac{2}{3}$. Or $\frac{2}{3} \notin ]-\frac{1}{9};\frac{1}{9}[$ donc la série est grossièrement divergente.}
\end{enumerate}
}
\uuid{kbR8}
\chapitre{Série entière}
\niveau{L2}
\module{Analyse}
\sousChapitre{Calcul de la somme d'une série entière}
\titre{ Calcul d'une somme de série entière}
\theme {séries entières}
\auteur{ }
\datecreate{2023-06-01}
\organisation{ AMSCC }


\difficulte{}
\contenu{ 
\texte{ On considère la série entière $\displaystyle S(x)= \sum \frac{(3x)^n}{n^2+3n+2}$. }
\begin{enumerate}
	\item \question{ Calculer son rayon de convergence $R$. Si $|x|=R$, cette série converge-t-elle ? }
	\reponse{ 
		$R=\frac{1}{3}$ \\
		Pour $|x|=\frac{1}{3}$, la série converge absolument car $\frac{1}{n^2+3n+2} \underset{n\to +\infty}\sim \frac{1}{n^2}$.
	}
	
	\item \question{ Décomposer la fraction rationnelle $\displaystyle \frac{1}{n^2+3n+2}$ en éléments simples, c'est-à-dire trouver les réels $\alpha$ et $\beta$ tels que $\frac{1}{n^2+3n+2}=\frac{\alpha}{n+1}+\frac{\beta}{n+2}$. }
	\reponse{ 
		$\alpha=1$ et $\beta=-1$
	}
	
	\item \question{ Calculer la somme de la série pour $|x|<R$. }
	\reponse{ 
		Soit $|x|<R$. On a
		\[S(x)=\frac{1}{3x}\ln(1-3x)\left(\frac{1}{3x}-1\right) +\frac{1}{3x}. \]
	}
	
\end{enumerate} 
}
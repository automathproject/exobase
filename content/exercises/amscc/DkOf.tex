\uuid{DkOf}
\chapitre{Statistique}
\niveau{L2}
\module{Probabilité et statistique}
\sousChapitre{Tests d'hypothèses, intervalle de confiance}
\titre{Test d'adéquation à une loi}
\theme{statistiques}
\auteur{}
\datecreate{2022-09-28}
\organisation{AMSCC}
\difficulte{}
\contenu{

\texte{ Une enquête a été réalisée auprès de 100 demandeurs d'emploi sur le nombre d'entretiens d'embauche obtenus ces deux derniers mois. Elle donne les résultats suivants :
	
\begin{center}
		\begin{tabular}{|c|c|c|c|c|c|c|c|}
		\hline \textbf{Nombre d'entretiens :} & 0 & 1 & 2 & 3 & 4 & 5 & 6 et plus \\ 
		\hline \textbf{Effectifs :} & 15 & 30 & 27 & 13 & 9 & 5 & 1 \\ 
		\hline 
	\end{tabular} 
\end{center} }
	
\question{ Peut-on dire que le nombre d'entretien d'embauche est distribué selon une loi de Poisson ? }}

\uuid{wlig}
\chapitre{Fonction convexe}
\niveau{L3}
\module{Optimisation}
\sousChapitre{Multiplicateurs de Lagrange}
\titre{Optimisation sous contrainte (3)}
\theme{optimisation}
\auteur{Jean-François Culus}
\datecreate{2024-10-18}
\organisation{AMSCC}
\difficulte{}
\contenu{

\texte{
A présent, nous allons introduire un outil permettant d’aller plus vite dans la résolution de ces problèmes: le Lagrangien. 
Si l’on souhaite étudier les extremums de la fonction $f$ soumise à la contrainte $g(x,y)=0$, on introduit le Lagrangien $L(x,y,\lambda)$ défini par :
$$ L(x,y,\lambda)= f(x,y)-\lambda g(x,y)$$ 

Donnons un exemple d’étude pour étudier les extremas de la fonction $f(x,y)=x^2+y^2-4xy$ sous la contrainte $x^2+y^2=8$. 
}

\begin{enumerate}

\item 
\question{ Ecrire le Lagrangien associé à ce problème d’optimisation sous contrainte. }

\reponse{ Le Lagrangien est alors: 
$$L(x,y,\lambda)= f(x,y)-\lambda g(x,y) = x^2+y^2 - 4 xy - \lambda (x^2+y^2-8)$$ 
}

\item
\question{ Déterminer ses points stationnaires (ie. vérifiant 
$\frac{\partial L}{\partial x}=
 \frac{\partial L}{\partial y}=
\frac{\partial L}{\partial \lambda}=0$. 
}

\reponse{
$$\left\lbrace \begin{array}{ll}
\frac{\partial L}{\partial x}& =0 \\
\frac{\partial L}{\partial y}& =0 \\
\frac{\partial L}{\partial \lambda}& =0 \end{array}\right. 
~~ \Rightarrow 
\left\lbrace 
\begin{array}{ll}
2x-4y-2\lambda x &= 0 \\
2y -4x -2 \lambda y &= 0 \\
x^2+y^2 -8 &=0 \end{array}\right. 
~~\Rightarrow 
\left\lbrace 
\begin{array}{ll}
x &= \pm 2 \\
y &= \pm 2 \\
\lambda &= \frac{x-2y}{x}  \end{array}\right. $$


Ainsi, nous avons quatre points stationnaires: 
\\ $(2,2)$ correspondant  à $\lambda=-1$, 
\\ $(-2;2)$ correspondant à $\lambda =3$, 
\\ $(2;-2)$ correspondant à $\lambda = 3$ et 
\\ $(-2;-2)$ correspondant à $\lambda =-1$. 
}

\item
\question{
Donner l’équation de l’espace vectoriel tangent $T$ à la courbe de contrainte en chacun des points critiques.} 


\reponse{
Puisque le gradient $\nabla g(x_0;y_0)$ est orthogonal au plan tangent à la contrainte $\Gamma$ en $(x_0;y_0)$, nous en déduisons que:
$$(u,v)\in T \Leftrightarrow 
\nabla g(x,y) \cdot \begin{pmatrix} u \\ v \end{pmatrix} = 0 ~~\Leftrightarrow ~~
\frac{\partial g}{\partial x} (x_0;y_0) u + \frac{\partial g}{\partial y}(x_0;y_0) v=0 ~~\leftrightarrow 2x_0u+2y_0 v=0$$ 
}

\item 
\question{ On désigne par $H_L(x,y)$ la hessienne de la fonction $L$, vue comme fonction de deux variables. 
On définit alors la forme quadratique $Q(u,v)$ associé au Lagrangien par 
$$ Q(u,v)= \frac{1}{2} \begin{pmatrix} u & v\end{pmatrix} \cdot H_L \cdot \begin{pmatrix} u \\ v\end{pmatrix} 
~~~~\text{ avec }~~~~
 H_L(x,y) = \begin{pmatrix} 
\frac{\partial^2 L}{\partial x^2} & \frac{\partial^2 L}{\partial x \partial y} \\
\frac{\partial^2 L}{\partial y \partial x} & \frac{\partial^2 L}{\partial y^2}
\end{pmatrix}$$
Donner une expression de $Q(u,v)$. 
}

\reponse{
Déjà, exprimons la Hessienne du Lagrangien $L(x,y,\lambda)$:
Nous avons déjà 

$\frac{\partial L}{\partial x}(x,y)= 2x-4y-2\lambda x$ et 
$\frac{\partial L}{\partial y}(x,y)= 2y-4x-2\lambda y$.
Nous en déduisons alors que: 

$$\frac{\partial^2 L}{\partial x^2}(x,y)= 2-2\lambda;~~
\frac{\partial^2 L}{\partial x \partial y }= \frac{\partial^2 L}{\partial y \partial x}(x,y)=-4  ~~ \text{ et }~\frac{\partial^2 L}{\partial y^2}(x,y)=2-2\lambda$$ 

Aussi, nous avons :
$$ H_L(x,y) = \begin{pmatrix} 
\frac{\partial^2 L}{\partial x^2} & \frac{\partial^2 L}{\partial x \partial y} \\
\frac{\partial^2 L}{\partial y \partial x} & \frac{\partial^2 L}{\partial y^2}
\end{pmatrix}
= \begin{pmatrix} 2-2\lambda  & -4 \\ -4 & 2-2\lambda \end{pmatrix} 
$$

Ainsi, la forme quadratique associée est :

$$ Q(u,v)= \frac{1}{2} \begin{pmatrix} u&v \end{pmatrix} \cdot 
\begin{pmatrix} 2-2\lambda  & -4 \\ -4 & 2-2\lambda \end{pmatrix} \cdot 
\begin{pmatrix} u \\ v\end{pmatrix} =(1-\lambda) u^2-4uv+(1-2\lambda) v^2$$
}

\item
\question{ 
Pour chacun des points critiques obtenus précédemment, déterminer sa nature en étudiant la forme quadratique $Q(u,v)$, pour $(u,v)\in T$ un vecteur tangent à la contrainte en ce point.
On utilisera le fait que si $Q(u,v)>0$, ce point est un minimum alors que si $Q(u,v)<0$, ce point est un maximum. 
}

\reponse{
$\bullet$ Pour le point $(2;2)$ pour lequel $\lambda=-1$, nous avons 
$(u,v)\in T ~\Leftrightarrow ~ Q(u,v)=4u^2-8uv+4v^2$. 
De plus, nous devons avoir $\frac{\partial g}{\partial x}(x,y) u + \frac{\partial g}{\partial y} (x,y)=0$, soit $2xu +2yu=0$ d’où $u+v=0$, i.e. $u=-v$. 
\\ Il s’ensuit alors que $Q(u,v)=Q(u,-u)=16u^2>0$ d’où le point $(2,2)$ correspond à un minimum. 


$\bullet$ Pour le point $(-2;-2)$, les calculs sont les mêmes qu’au cas précédent et nous obtenons alors un minimum. 


$\bullet$ Pour le point $(-2;2)$, nous avions $\lambda =3$. Cela conduit à 
$Q(u,v)= -2u^2-4uv-2v^2$ et puisque $(u,v)\in T$, nous devons avoir
$\frac{\partial g}{\partial x}(x,y) u + \frac{\partial g}{\partial y} (x,y)=0$, soit $-4u +4v=0$, i.e. $u=v$. 
\\ Il s’ensuit alors que $Q(u,v)=Q(u,u)=-16u^2<0$ et donc ce point est un maximum. 

$\bullet$ Le point $(2;-2)$ revient aux mêmes calculs que précédemment et nous obtenons que c’est (aussi) un minimum. }


\end{enumerate}
}
\uuid{i64t}
\titre{Densité de probabilité et variables aléatoires}
\chapitre{Probabilité continue}
\niveau{L2}
\module{Probabilité et statistique}
\sousChapitre{Loi conjointe}
\theme{probabilités, densité, variables aléatoires}
\auteur{Erwan Hillion}
\datecreate{2025-06-11}
\organisation{AMSCC}
\difficulte{}
\contenu{
	\texte{Soit $\mathcal{T} \subset \mathbb{R}^2$ le triangle délimité par les sommets de coordonnées $(0,0)$, $(0,1)$ et $(2,0)$. On considère la densité $f: \R^2 \to \R$ définie par $f(x,y) = 1$ si $(x,y) \in \mathcal{T}$ et $f(x,y) = 0$ sinon. Soit $(X,Y)$ un vecteur aléatoire suivant la loi de densité $f$.}
	
	\begin{enumerate}
		\item \question{Vérifier que $f$ est une densité de probabilité sur $\R^2$.}
		\item \question{Montrer que la variable aléatoire marginale $X$ suit la loi de densité $f_X : x \mapsto \left(1-\frac{x}{2}\right) \mathbf{1}_{x \in [0,2]}$.}
		\item \question{Calculer de même la densité de la loi de la marginale $Y$.}
		\item \question{Calculer la covariance $\mathrm{Cov}(X,Y)$.}
		\item \question{Les variables aléatoires $X$ et $Y$ sont-elles indépendantes ?}
	\end{enumerate}
}
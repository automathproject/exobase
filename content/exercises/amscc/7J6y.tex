\uuid{7J6y}
\chapitre{Série numérique}
\niveau{L1}
\module{Analyse}
\sousChapitre{Autre}
\titre{Etude de séries numériques}
\theme{séries}
\auteur{}
\datecreate{2023-05-17}
\organisation{AMSCC}
\difficulte{}
\contenu{

\texte{ Dans chacun des cas, dire si la série de terme général $u_n$ est absolument convergente, semi-convergente, divergente, grossièrement divergente. }

\indication{
Attention, pour pouvoir répondre, certaines séries demandent deux démonstrations: par exemple, pour montrer que $\sum u_n$ est semi-convergente, il faut démontrer que $\sum u_n$ est convergente et que $\sum |u_n|$ est divergente. }

\colonnes{\solution}{3}{1}
\begin{enumerate}
	\item \question{$u_n=\frac{\sin(n^2)}{n^2}$}
	\reponse{La série converge car elle est absolument convergente (le terme général est dominé par $1/n^2$).}
	
	\item \question{$u_n=\frac{(-1)^n\ln(n)}{n}$}
	\reponse{On a $|u_n|\geq \frac{1}{n}$, qui est le terme général d'une série de Riemann divergente. Par comparaison, la série $\sum u_n$ ne converge pas absolument. \\
		Par le critère des séries alternées, comme la suite $\left(\frac{\ln(n)}{n}\right)_n$ est décroissante et tend vers $0$, la série $\sum u_n$ converge. \\
		Donc la série $\sum u_n$ est \textbf{semi-convergente}.}
	
	\item \question{$u_n=\frac{\cos(n^2\pi)}{n\ln(n)}$}
	\reponse{La série converge absolument, par comparaison à une série de Bertrand. }
	
	\item \question{$u_n=\frac{(2n+1)^4}{(7n^2+1)^3}$}
	\reponse{C'est une série à termes positifs, elle converge car le terme général est équivalent à $\frac{16}{7^3n^2}$ qui est une série de Riemann convergente de paramètre 2.}
	
	\item \question{$u_n=\left(1-\frac{1}{n}\right)^n$}
	\reponse{La série diverge grossièrement car le terme général tend vers $1/e$ et ne tend donc pas vers 0.}
	
	\item \question{$u_n=\ln(1+e^{-n})$}
	\reponse{C'est une série à termes positifs, elle converge  car le terme général est équivalent à $1/e^n$, c'est le terme général d'une série géométrique convergente.}
	
	\item \question{$u_n=\frac{1}{n\cos^2(n)}$}
	\reponse{Pour tout $n$, on a $u_n \geq \frac{1}{n} \geq 0$ donc par comparaison de séries à termes positifs, la série diverge. }
	
	\item \question{$u_n=\sin\left(\frac{n^2+1}{n}\pi \right)$}
	\reponse{On a: 
		$u_n=\sin((n+\frac{1}{n})\pi)=\sin(n\pi+\frac{\pi}{n})=
		=(-1)^n\sin(\frac{\pi}{n})$.
		$u_n$ n'est pas de signe constant: on commence par étudier la convergence absolue. $|u_n|\sim \frac{\pi}{n}$, qui est le terme d'une série de Riemann divergente donc la série $\sum u_n$ ne converge pas absolument. \\
		Par le critère des séries alternées, comme la suite $(\sin(\frac{\pi}{n}))_n$ est décroissante et tend vers $\sin(0)=0$, la série $\sum u_n$ converge. \\
		Finalement, la série $\sum u_n$ est \textbf{semi-convergente}.}
	
	\item \question{$u_n=\frac{n}{2^n}$}
	\reponse{On utilise le critère de d'Alembert: $\frac{u_{n+1}}{u_n}=\frac{n+1}{2n}$, qui tend vers $\frac{1}{2}<1$. On en conclut que la série $\sum u_n$ converge. Comme il s'agit d'une série à termes positifs, cette série \textbf{converge absolument}.}
	
	\item \question{$u_n=\left(1+\frac{1}{n}\right)^{n^2}$}
	\reponse{$u_n=(1+\frac{1}{n})^{n^2}>1$ donc la suite $(u_n)$ ne peut pas tendre vers $0$: la série $\sum u_n$ \textbf{diverge grossièrement}}
	
	\item \question{$u_n=\frac{n^{10000}}{n!}$}
	\reponse{ $\frac{u_{n+1}}{u_n}=(\frac{n+1}{n})^{10000}\frac{1}{n+1}$ tend vers $0$ en l'infini. Par le critère de D'Alembert, la série $\sum u_n$ converge. Comme elle est à termes positifs, cette série \textbf{converge absolument}.}
	
	\item \question{$u_n= \frac{(-1)^n}{\sqrt{n+1}}$}
	\reponse{ $u_n$ est de signe non constant; on commence par étudier la convergence absolue. $|u_n|\sim \frac{1}{n^{\frac{1}{2}}}$ donc la série $\sum u_n$ ne converge pas absolument. \\
	Par le critère des séries alternées, comme $(\frac{1}{\sqrt{n+1}})_n$ est une suite décroissante et de limite $0$, la série $\sum u_n$ converge. \\
	Conclusion: la série $\sum u_n$ est \textbf{semi-convergente}.}
	
	\item \question{$u_n=\frac{1}{n(n+1)(n+2)}$}
	\reponse{On a $u_n \sim \frac{1}{n^3}$ qui est le terme général d'une série de Riemann convergente. La série $\sum u_n$ est absolument convergente. Comme il s'agit d'une série télescopique, on peut calculer sa somme: on décompose la fraction en éléments simples:
		\[ \frac{1}{n(n+1)(n+2)} =\frac{\frac{1}{2}}{n}-\frac{1}{n+1}+\frac{\frac{1}{2}}{n+2}\]
		On calcule les sommes partielles:
		\[ \sum_{k=1}^{n} u_k=\frac{1}{4}+\frac{1}{2(n+1)} +\frac{1}{2(n+2)} \  \underset{n\rightarrow+\infty}\longrightarrow \frac{1}{4}.\]
		Donc la somme de la série $\sum u_n$ vaut $\frac{1}{4}$.}
	
	\item \question{$u_n=\frac{1}{(\ln(n))^n}$}
	\reponse{$(|u_n|)^{\frac{1}{n}}=\frac{1}{\ln(n)}$ qui tend vers $0$ en l'infini. Par le critère de Cauchy, la série est convergente. Comme elle est à termes positifs, elle est également \textbf{absolument convergente}.
		\item $u_n\sim \frac{1}{2n^2}$, qui est une série de Riemann convergente donc la série $\sum u_n$ converge. Étant à termes positifs, elle \textbf{converge absolument}.}
	
	\item \question{$u_n=1-\cos\left(\frac{1}{n}\right)$}
	\reponse{Par développement limité du cosinus, on obtient l'équivalent : $u_n\sim \frac{1}{2n^2}$, qui est le terme général d'une série de Riemann convergente. Donc la série $\sum u_n$ converge. Étant à termes positifs, elle \textbf{converge absolument}.}
	
	\item \question{$u_n=\Big(\frac{4n+1}{3n+2}\Big)^n$}
	\reponse{La série diverge grossièrement car le terme général ne tend pas vers $0$. }
	
	\item \question{$u_n=\frac{\ln(n)}{n}$}
	\reponse{C'est une série à termes positifs et $u_n \geq \frac{1}{n}$ donc par comparaison à une série de Riemann divergente, la série diverge. }
\end{enumerate}
\fincolonnes{\solution}{3}{1}}

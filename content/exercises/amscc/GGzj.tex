\uuid{GGzj}
\chapitre{Fonction de plusieurs variables}
\niveau{L2}
\module{Analyse}
\sousChapitre{Autre}
\titre{\'Etude d'une fonction de deux variables}
\theme{fonctions de plusieurs variables}
\auteur{}
\datecreate{2023-03-31}
\organisation{AMSCC}
\difficulte{}
\contenu{

\texte{ Soit la fonction $f \colon \R^2 \to \R$ définie par : $$f(x,y) = \frac{x+y}{x^2+y^2}$$ }

\begin{enumerate}
	\item \question{ Donner l'ensemble de définition de $f$. }
	\reponse{ La fonction $f$ est définie pour tout $(x,y) \in \R^2$ tels que $x^2+y^2 \neq 0$, c'est-à-dire $(x,y) \neq (0,0)$. L'ensemble de définition est donc $\mathcal{D}_f = \{(x,y) \in \R^2 \, , \, (x,y) \neq (0,0) \}$. }
	\item \question{ Montrer que la courbe de niveau $k=0$ est une droite dont on précisera l'équation. }
		\reponse{ $f(x,y) = 0 \iff y = -x$, la courbe de niveau $0$ est donc la droite d'équation $y=-x$. }
	\item \question{ Quelle est la forme des courbes de niveau $k \neq 0$ ? }
		\reponse{ Soit $k \neq 0$ : $f(x,y) = k \iff x+y = k(x^2+y^2) \iff x^2 + y^2 - \frac{1}{k}x - \frac{1}{k}y = 0$, la courbe de niveau $k$ est donc un cercle. }	
	\item \question{ Calculer $\lim\limits_{x \to 0} f(x,0)$, $\lim\limits_{x \to 0} f(x,x)$ et $\lim\limits_{x \to 0} f(x,-x)$.  }
	\reponse{ $\lim\limits_{x \to 0} f(x,0) = \lim\limits_{x \to 0}\frac{1}{x} = \begin{cases}
			+\infty & \text{ si } x > 0 \\
			-\infty & \text{ si } x < 0
			\end{cases}$. \\
		De même, $\lim\limits_{x \to 0} f(x,x) = \lim\limits_{x \to 0} \frac{2x}{2x^2} =  \begin{cases}
			+\infty & \text{ si } x > 0 \\
			-\infty & \text{ si } x < 0
		\end{cases}$. \\
	Enfin, $\lim\limits_{x \to 0} f(x,-x) = \lim\limits_{x \to 0} 0 = 0$.
		}
	\item \question{ Peut-on prolonger la fonction $f$ en $(0,0)$ afin qu'elle soit continue en $(0,0)$ ? Justifier. }
	\reponse{ Pour pouvoir prolonger la fonction $f$ en $(0,0)$ afin qu'elle soit continue en $(0,0)$, il faudrait que $f$ admette une limite quand $(x,y) \to (0,0)$. D'après la question précédente, ce n'est pas le cas. }
\end{enumerate}}

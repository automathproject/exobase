\uuid{2pit}
\titre{DSE et Equations différentielles}
\chapitre{Série entière}
\niveau{L2}
\module{Analyse}
\sousChapitre{Equations différentielles}
\theme{Analyse}
\auteur{Q. Liard}
\organisation{AMSCC}
\difficulte{}
\contenu{


\texte{Le but de cet exercice est de résoudre l'équation différentielle $$x(x-1)y''+3xy'+y=0$$ avec les conditions initiales $y(0)=0$ et $y'(0)=1$ par la méthode des séries entières.\\
On supposera qu'une solution $y$ peut s'exprimer sous la forme: $$y(x)=\displaystyle\sum_{n \in \mathbb{N}}a_n\,x^{n},$$ pour tout $|x|<R$ où $R$ est un réel fixé strictement positif et $(a_n)_{n\in \mathbb{N}}$ une suite de nombres réels.}
\begin{enumerate}
\item \question{Donner l'expression des dérivées successives de $y$: $y'$ et $y''$ sous la forme de série entière.}
\item \question{Déterminer $a_0$ et $a_1$ en utilisant les conditions initiales.}
\item \question{Montrer que $a_2=2$ et que pour tout $n\geq 2,$
$$a_{n+1}=\frac{n+1}{n}a_n.$$
}
\item En déduire une expression explicite de $(a_n)_{n\geq 2}$ puis l'expression générale de la solution $y$. On précisera l'intervalle de convergence de la série obtenue.
\item Déterminer la fonction somme de cette série entière.
\end{enumerate}
\reponse{
Nous cherchons une solution de la forme d'une série entière:
$$y(x) = \sum_{n=0}^{\infty} a_n x^n$$
\textbf{Conditions Initiales}
Les conditions initiales sont $y(0)=0$ et $y'(0)=1$.
D'après la forme de $y(x)$, $y(0) = a_0$. Donc, $a_0 = 0$.
D'après la forme de $y'(x)$, $y'(0) = 1 \cdot a_1 \cdot x^{1-1}|_{x=0} = a_1$. Donc, $a_1 = 1$.

\textbf{Substitution dans l'Équation Différentielle}
Substituons les séries dans l'équation $x(x-1)y''+3xy'+y=0$:
$x(x-1)\sum_{n=2}^{\infty} n(n-1) a_n x^{n-2} + 3x\sum_{n=1}^{\infty} n a_n x^{n-1} + \sum_{n=0}^{\infty} a_n x^n = 0$
$(x^2-x)\sum_{n=2}^{\infty} n(n-1) a_n x^{n-2} + \sum_{n=1}^{\infty} 3n a_n x^n + \sum_{n=0}^{\infty} a_n x^n = 0$
$\sum_{n=2}^{\infty} n(n-1) a_n x^n - \sum_{n=2}^{\infty} n(n-1) a_n x^{n-1} + \sum_{n=1}^{\infty} 3n a_n x^n + \sum_{n=0}^{\infty} a_n x^n = 0$

\textbf{Ajustement des Indices des Sommes}
Pour combiner ces sommes, nous les réécrivons pour avoir la même puissance générique $x^k$:
\begin{enumerate}
    \item $\sum_{n=2}^{\infty} n(n-1)a_n x^n = \sum_{k=2}^{\infty} k(k-1)a_k x^k$
    \item $-\sum_{n=2}^{\infty} n(n-1)a_n x^{n-1}$: Posons $k=n-1$, alors $n=k+1$. Quand $n=2, k=1$.
    $$-\sum_{k=1}^{\infty} (k+1)k a_{k+1} x^k$$
    \item $\sum_{n=1}^{\infty} 3n a_n x^n = \sum_{k=1}^{\infty} 3k a_k x^k$
    \item $\sum_{n=0}^{\infty} a_n x^n = \sum_{k=0}^{\infty} a_k x^k$
\end{enumerate}
L'équation différentielle devient:
$$\sum_{k=2}^{\infty} k(k-1)a_k x^k - \sum_{k=1}^{\infty} (k+1)k a_{k+1} x^k + \sum_{k=1}^{\infty} 3k a_k x^k + \sum_{k=0}^{\infty} a_k x^k = 0$$

\textbf{Relation de Récurrence}
Nous égalons la somme des coefficients de chaque puissance de $x$ à zéro.

Coefficient de $x^0$ (pour $k=0$):
Seul le terme $a_k x^k$ de la quatrième somme (où $k=0$) contribue.
$$a_0 = 0$$
Ceci est cohérent avec la condition initiale $y(0)=0$.

Coefficient de $x^1$ (pour $k=1$):
Les termes des sommes 2, 3 et 4 contribuent:
$$-(1+1)(1)a_{1+1} + 3(1)a_1 + a_1 = 0$$
$$-2a_2 + 4a_1 = 0 \implies a_2 = 2a_1$$
Comme $a_1=1$ (d'après $y'(0)=1$), nous avons $a_2=2$.

Coefficient de $x^k$ (pour $k \ge 2$):
$$k(k-1)a_k - (k+1)k a_{k+1} + 3k a_k + a_k = 0$$
Regroupons les termes en $a_k$ et $a_{k+1}$:
$$a_k [k(k-1) + 3k + 1] - k(k+1)a_{k+1} = 0$$
$$a_k [k^2-k+3k+1] - k(k+1)a_{k+1} = 0$$
$$a_k [k^2+2k+1] - k(k+1)a_{k+1} = 0$$
$$a_k (k+1)^2 = k(k+1)a_{k+1}$$
Pour $k \ge 2$, $k+1 \neq 0$, donc nous pouvons diviser par $k+1$:
$$a_k (k+1) = k a_{k+1}$$
La relation de récurrence est donc:
$$a_{k+1} = \frac{k+1}{k} a_k \quad \text{pour } k \ge 2$$
Vérifions si cette relation est valable pour $k=1$. Pour $k=1$, elle donne $a_2 = \frac{1+1}{1} a_1 = 2a_1$. Ceci est la même relation que nous avons trouvée pour le coefficient de $x^1$.
Donc, la relation de récurrence $a_{k+1} = \frac{k+1}{k} a_k$ est valable pour $k \ge 1$.

\textbf{Calcul des Coefficients}
Nous avons $a_0=0$ et $a_1=1$.
\begin{itemize}
    \item Pour $k=1$: $a_2 = \frac{1+1}{1} a_1 = 2a_1 = 2(1) = 2$.
    \item Pour $k=2$: $a_3 = \frac{2+1}{2} a_2 = \frac{3}{2}(2) = 3$.
    \item Pour $k=3$: $a_4 = \frac{3+1}{3} a_3 = \frac{4}{3}(3) = 4$.
    \item Pour $k=4$: $a_5 = \frac{4+1}{4} a_4 = \frac{5}{4}(4) = 5$.
\end{itemize}
Il apparaît que $a_k = k$ pour $k \ge 1$. Nous pouvons le prouver par récurrence:
\textit{Base}: Pour $k=1$, $a_1=1$.
\textit{Hypothèse de récurrence}: Supposons $a_k=k$ pour un certain entier $k \ge 1$.
\textit{Étape d'induction}: $a_{k+1} = \frac{k+1}{k} a_k = \frac{k+1}{k} (k) = k+1$.
La formule $a_k=k$ est donc vraie pour tout $k \ge 1$.
\newpage
\textbf{Solution de l'Équation}
La solution en série entière est $y(x) = \sum_{k=0}^{\infty} a_k x^k$.
Puisque $a_0=0$ et $a_k=k$ pour $k \ge 1$:
$$y(x) = \sum_{k=1}^{\infty} k x^k = x + 2x^2 + 3x^3 + 4x^4 + \dots$$
Cette série est une série géométrique dérivée. Rappelons que pour $|x|<1$:
$$\sum_{k=0}^{\infty} x^k = \frac{1}{1-x}$$
En dérivant par rapport à $x$:
$$\sum_{k=1}^{\infty} k x^{k-1} = \frac{1}{(1-x)^2}$$
En multipliant par $x$:
$$\sum_{k=1}^{\infty} k x^k = \frac{x}{(1-x)^2}$$
Ainsi, la solution de l'équation différentielle avec les conditions initiales données est:
$$y(x) = \frac{x}{(1-x)^2}$$
Cette solution est valable pour $|x|<1$.
}
}















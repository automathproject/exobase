\uuid{DnxY}
\chapitre{Statistique}
\niveau{L2}
\module{Probabilité et statistique}
\sousChapitre{Estimation}
\titre{ Estimation sur un jeu de données }
\theme{statistiques, tableur}
\auteur{Maxime NGUYEN}
\datecreate{2022-11-15}
\organisation{AMSCC}
\difficulte{}
\contenu{

Voici un extrait des prix de l'essence relevés le 7 novembre 2022 sur cette base de données ouvertes : 

https://data.economie.gouv.fr/explore/embed/dataset/prix-carburants-fichier-instantane-test-ods-copie/table/

L'échantillon est disponible en suivant \href{https://stcyrterrenetdefensegouvf-my.sharepoint.com/:x:/g/personal/maxime_nguyen_st-cyr_terre-net_defense_gouv_fr/EVXZBgQqo29LtxbAZJS0Y-IBz6WNroovMLS-hZVcisCSQQ?e=02c19c}{ce lien}. 

\begin{enumerate}
	\item \question{ A partir des données de l'échantillon, donner une estimation ponctuelle sans biais du prix moyen et de la variance du prix du carburant E10 en France. On précisera les estimateurs utilisés. }
	\reponse{ Avec l'estimateur de moyenne empirique (sans biais et convergent), on trouve une estimation de la moyenne de prix de $1{,}661759$ et une variance de $0{,}0139$. }
	\item \question{ Proposer une estimation par intervalle de confiance du prix moyen et de la variance du carburant E10 en France, respectivement pour une confiance de $95\%$ puis $99\%$.  }
	\reponse{ On travaille sur un échantillon de taille $108$ donc les calculs se font avec la loi normale.
	
	Pour $\alpha = 5\%$, on trouve $[1{,}6395 ; 1{,6840}]$. 
	
	Pour $\alpha = 5\%$, on trouve $[1{,}6325 ; 1{,6910}]$. 
 }
\end{enumerate}

}

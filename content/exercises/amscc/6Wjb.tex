\uuid{6Wjb}
\chapitre{Probabilité discrète}
\niveau{L2}
\module{Probabilité et statistique}
\sousChapitre{Variable aléatoire discrète}
\titre{Couple de variables}
\theme{variables aléatoires discrètes, loi conjointe}
\auteur{  }
\datecreate{2023-08-30}
\organisation{AMSCC}
%Proba192

\difficulte{2}
\contenu{
	\texte{ Deux urnes $U_1$ et $U_2$ contiennent des boules permettant des gains ou des pertes décrits comme suit : 
		\begin{itemize}
			\item dans l'urne $U_1$, il y a 1 boule << perdant 1\euro{} >>, 2 boules << sans gain >>, 3 boules << gagnant 2\euro{} >> ;
			\item dans l'urne $U_2$, il y a 3 boules << perdant 1\euro{} >>, 2 boules << sans gain >>, 1 boules << gagnant 2\euro{} >> ;
		\end{itemize}
	Un joueur lance un dé équilibré : s'il obtient 6, il pioche une boule dans l'urne $U_1$, sinon il pioche une boule dans l'urne $U_2$. 
	
	On note $X$ la variable aléatoire égale à $1$ si le dé sort $6$, $0$ sinon. On note $Y$ la variable aléatoire égale au gain (en euros.)
	}
	
	\begin{enumerate}
		\item \question{ Donner la loi du couple $(X,Y)$. }
		\indication{Utiliser la formule de Bayes pour calculer toutes les valeurs $\prob(X=k,Y=k')$. }
		\reponse{ Pour calculer par exemple $\prob(X=1,Y=-1)$, on utilise la formule de Bayes : 
	$$\prob(X=1,Y=-1) = \prob(\{X=1\} \cap \{Y=-1\}) = \prob(X=1)\prob(Y=-1 \mid X=1)$$
	Or d'après l'énoncé, $\prob(X=1) = \frac{1}{6}$ et $\prob(Y=-1 \mid X=1) = \frac{1}{6}$ d'où $\prob(X=1,Y=-1) = \frac{1}{36}$. 
	En procédant de même pour tous les couples de valeurs possibles, on obtient : 
\begin{center}
		\begin{tabular}{|c|c|c|c|}
		\hline
		& $Y=-1$ & $Y=0$ & $Y=2$ \\
		\hline
	$X=1$	& $\frac{1}{36}$ & $\frac{2}{36}$ & $\frac{3}{36}$  \\
		\hline
	$X=0$	& $\frac{15}{36}$ & $\frac{10}{36}$ & $\frac{5}{36}$ \\
		\hline
	\end{tabular}
\end{center}	
	 }
		\item \question{ Déterminer les lois marginales du couple $(X,Y)$.}
		\indication{On reconnaît une loi usuelle pour $X$.}
		\reponse{ On obtient directement que $X$ suit une loi de Bernoulli de paramètre $\frac{1}{6}$. De plus, par somme, on obtient la loi suivante pour $Y$ :
	\begin{center}
		\begin{tabular}{|c|c|c|c|}
			\hline
		$k$	& $-1$ & $0$ & $2$ \\
			\hline
			$\prob(Y=k$	& $\frac{4}{9}$ & $\frac{3}{9}$ & $\frac{2}{9}$  \\
			\hline
		\end{tabular}
	\end{center}			
 }
		\question{Les variables $X$ et $Y$ sont-elles indépendantes ?}
		\indication{
	Pour déterminer si les variables $X$ et $Y$ sont indépendantes, il faut vérifier si l'égalité 
	$$P(X=k, Y=k') = P(X=k) \times P(Y=k')$$
	est vraie pour tous les couples de valeurs possibles $(k,k')$.
	
	Testons cette égalité avec le couple $(1, -1)$. 	
	}
		\reponse{
	D'après la loi conjointe, nous avons :
	$$P(X=1, Y=-1) = \frac{1}{36}$$
	
	Calculons maintenant le produit des probabilités marginales :
	$$P(X=1) \times P(Y=-1) = \frac{1}{6} \times \frac{4}{9} = \frac{4}{54} = \frac{2}{27}$$
	
	On constate que $\frac{1}{36} \neq \frac{2}{27}$.
	
	Puisque l'égalité $P(X=k, Y=k') = P(X=k)P(Y=k')$ n'est pas vérifiée pour au moins un couple de valeurs, on peut conclure que les variables aléatoires $X$ et $Y$ ne sont pas indépendantes.	
	}
	\end{enumerate}
	
}

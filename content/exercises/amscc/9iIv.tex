\uuid{9iIv}
\chapitre{Statistique}
\niveau{L2}
\module{Probabilité et statistique}
\sousChapitre{Tests d'hypothèses, intervalle de confiance}
\titre{ Lois pour les statistiques }
\theme{statistiques, estimateurs, loi du chi2}
\auteur{Maxime NGUYEN}
\datecreate{2022-10-12}
\organisation{AMSCC}
\difficulte{}
\contenu{

\texte{ 	On considère un échantillon $(X_i)$ de taille $n=4$ dans une population suivant une loi normale de paramètres $\mu$ et $\sigma^2$. 
	
	
	
	On pose 
	$$T_1 = \frac{1}{4}\sum_{i=1}^{4} X_i  \qquad T_2 = \frac{1}{5}(2X_1+X_2)+\frac{1}{10}(3X_3+X_4)$$
	$$U = \frac{1}{\sigma^2}\sum_{i=1}^{4}  {(X_i-\mu)^2} \qquad V = \frac{1}{\sigma^2}\sum_{i=1}^{4}  {(X_i-T_1)^2}$$ }

\begin{enumerate}
	\item \question{ On cherche à estimer $\mu$ à l'aide des estimateurs $T_1$ et $T_2$. \'Etudier leur biais et comparer leurs efficacités.  }
	\reponse{ Par linéarité de l'espérance, on calcule $\mathbb{E}(T_1) = \frac{4\mu}{4} = \mu$, $\mathbb{E}(T_2) = \frac{3\mu}{5}+\frac{4\mu}{10} = \mu$. Par conséquent, $B(T_1)=B(T_2)=0$.
		
		Pour comparer l'efficacité des deux estimateurs sais biais, on calcule leur EQM (ce qui revient à calculer leur variance.) Par indépendance des variables, on a :
		
		$V(T_1) = \frac{\sigma^2}{4} < V(T_3) = \frac{(4^2+2^2+3^2+1^2)\sigma^2}{100}$. Le plus efficace est donc l'estimateur $T_1$ qui est la moyenne empirique.
		
	}
	\item \question{ Quelle est la loi suivie par la variable $T_1$ ? la variable $T_2$ ? la variable $U$ ? la variable $V$ ? justifier. }
	\reponse{$U = \sum_{i=1}^{4}  \left(\frac{X_i-\mu}{\sigma}\right)^2$ ; or les $X_i$ sont des variables aléatoires indépendantes et $\frac{X_i-\mu}{\sigma}$ suit une loi $\mathcal{N}(0,1)$ donc par définition, $U$ suit une loi de $\chi^2(4)$. 
		
		De plus, $T_1 = \overline{X}$ est l'estimateur de moyenne empirique donc d'après le théorème de Fisher, $V$ suit une loi de $\chi^2(4-1) = \chi^2(3)$. }
	\item \question{ Déterminer $x \in \R$ tel que $\PP(U>x) = 0{,}05$. }
	\reponse{On lit dans la table de loi  $\PP(U<x) = 0{,}95$ pour $x = 11{,}07$. }
	\item \question{ A l'aide du tableur, calculer $\PP(V > 5)$ avec une précision de $10^{-8}$. }
	\reponse{ On a constaté que $V$ suit une loi de $\chi^2(3)$ d'après le théorème de Fisher.

Ensuite, 	en tapant la formule \texttt{1-LOI.KHIDEUX.N(5;3;1)} dans le tableur, on trouve que $\PP(V > 5) \approx 0{,}17179714$. 
}
	\item \question{ Quelle est la loi de la variable aléatoire $\frac{4(T_1-\mu)}{{\sigma \sqrt{U}}}$ ? }
	\reponse{On pose $Z = \frac{T_1-\mu}{\frac{\sigma}{\sqrt{4} }} = \frac{2(T_1-\mu)}{\sigma}$ variable distribuée selon une loi $\mathcal{N}(0,1)$. Soit alors $Y = \frac{Z}{\sqrt{\frac{U}{4} }}$ : par définition, $Y$ suit une loi $St(4)$. Après simplifications, on peut réécrire $Y = \frac{4(T_1-\mu)}{{\sigma \sqrt{U}}}$.}
\end{enumerate}}

\uuid{td22}
\titre{Stabilisation de la variance versus “plug-in”}
\chapitre{Statistique}
\niveau{L2}
\module{Probabilité et statistique}
\sousChapitre{Tests d'hypothèses, intervalle de confiance}
\theme{}
\auteur{}
\datecreate{2025-03-20}
\organisation{}

\difficulte{}
\contenu{

\texte{
Soient $X_1, \ldots, X_n$ des variables aléatoires indépendantes de même loi de Poisson de paramètre $\theta > 0$. On note $\theta_n = \frac{1}{n} \sum_{i=1}^n X_i$.
}

\begin{enumerate}
    \item \question{Rappeler le Théorème Central Limite vérifié par $\theta_n$. En déduire un intervalle de confiance asymptotique à 95 \% pour $\theta$.}
    \indication{}
    \reponse{
D'après le théorème de la limite centrale appliqué aux variables $X_i$ i.i.d. :
$$
    \sqrt{n} (\hat{\theta}_n - \theta) \xrightarrow{\text{en loi}} \mathcal{N}(0, \theta),
$$
ce qui implique que :
$$
    \sqrt{n} \frac{\hat{\theta}_n - \theta}{\sqrt{\hat{\theta}_n}} \xrightarrow{\text{en loi}} \mathcal{N}(0,1).
$$
Un intervalle de confiance asymptotique à 95\% pour $\theta$ est donc donné par :
$$
    \left[ \hat{\theta}_n - q \frac{\sqrt{\hat{\theta}_n}}{\sqrt{n}}, \hat{\theta}_n + q \frac{\sqrt{\hat{\theta}_n}}{\sqrt{n}} \right]
$$
avec $q = \Phi^{-1}(0.975) \approx 1.96$.
}


    \item \question{Trouver une fonction $g : \mathbb{R}^+ \rightarrow \mathbb{R}$ telle que $\sqrt{n}(g(\theta_n) - g(\theta))$ tende en loi vers une gaussienne centrée réduite. En déduire un intervalle de confiance asymptotique à 95\% pour $\theta$.}
    \indication{}
    \reponse{En utilisant la méthode Delta, nous avons :
\begin{equation}
    \sqrt{n} (g(\hat{\lambda}_n) - g(\lambda)) \xrightarrow{d} \mathcal{N}(0, \lambda (g'(\lambda))^2).
\end{equation}
Si nous choisissons $g(\lambda) = 2\sqrt{\lambda}$, nous obtenons :
\begin{equation}
    \sqrt{n} \left( 2\sqrt{\hat{\lambda}_n} - 2\sqrt{\lambda} \right) \xrightarrow{d} \mathcal{N}(0,1).
\end{equation}
D'où un nouvel intervalle de confiance asymptotique à 95\% :
\begin{equation}
    \left[ 2\sqrt{\hat{\lambda}_n} - \frac{q}{\sqrt{n}}, 2\sqrt{\hat{\lambda}_n} + \frac{q}{\sqrt{n}} \right]^2.
\end{equation}
}
    \item \question{Vérifier que les deux résultats sont équivalents.}
    \indication{}
    \reponse{}
\end{enumerate}

}

\uuid{vnVD}
\exo7id{6698}
\titre{exo7 6698}
\auteur{queffelec}
\organisation{exo7}
\datecreate{2011-10-16}
\isIndication{false}
\isCorrection{false}
\chapitre{Formule de Cauchy}
\sousChapitre{Formule de Cauchy}
\module{Analyse complexe}
\niveau{L3}
\difficulte{}

\contenu{
\texte{
Soit $f(z)=\sum_{n\ge 0}a_nz^n$ une série entière de rayon de
convergence $R>0$. Pour $r<R$, soit $\gamma_r\colon t\mapsto re^{it}$,
$t\in [0,2\pi]$ et
$$I(r)={1\over 2i\pi}\int_{\gamma_r}\vert f(z)\vert ^2{dz\over z}.$$
}
\begin{enumerate}
    \item \question{Montrer que $I(r)=\sum_{n=0}^{+\infty}\vert a_n\vert ^2 r^{2n}$.}
    \item \question{En déduire une nouvelle démonstration des inégalités de Cauchy et
montrer que si $f$ n'est pas un monôme, ces inégalités sont strictes.}
    \item \question{En considérant $f(z)=1/(1-z)^2$, montrer que pour $r<1$,
$${1\over 2\pi}\int_0^{2\pi}{dt\over (1-2r\cos
t+r^2)^2}=\sum_{n=1}^{+\infty}n^2r^{2n-2}.$$}
\end{enumerate}
}

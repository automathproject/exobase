\uuid{QDHm}
\exo7id{6646}
\titre{exo7 6646}
\auteur{queffelec}
\organisation{exo7}
\datecreate{2011-10-16}
\isIndication{false}
\isCorrection{false}
\chapitre{Fonction holomorphe}
\sousChapitre{Fonction holomorphe}
\module{Analyse complexe}
\niveau{L3}
\difficulte{}

\contenu{
\texte{
Soit $f$ une fonction holomorphe sur un ouvert $\Omega $, $u$
sa partie réelle et $v$ sa partie imaginaire. On suppose que les dérivées
partielles secondes de $u$ et $v$ existent et sont continues sur $\Omega $.
Montrer que $u$ (resp. $v$) est  harmonique (c'est-à-dire $\displaystyle
{\partial ^2u\over \partial x^2}+ {\partial ^2u\over \partial y^2}=0$).
}
}

\uuid{EfLK}
\exo7id{7240}
\titre{exo7 7240}
\auteur{megy}
\organisation{exo7}
\datecreate{2021-03-06}
\isIndication{true}
\isCorrection{false}
\chapitre{Formule de Cauchy}
\sousChapitre{Formule de Cauchy}
\module{Analyse complexe}
\niveau{L3}
\difficulte{}

\contenu{
\texte{
(Généralisation de Liouville.) 
Soit \(f:\C\to \C\) une fonction entière non-constante. Montrer que l'image  \(f(\C)\) de \(\C\) par \(f\) est dense dans \(\C\).
}
\indication{Par l'absurde, supposer qu'il existe \(a\in \C\setminus \overline{f(\C)}\) et étudier la fonction \(z\mapsto \frac{1}{f(z)-a}\).)}
}

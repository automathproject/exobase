\uuid{MmAD}
\exo7id{6665}
\titre{exo7 6665}
\auteur{queffelec}
\organisation{exo7}
\datecreate{2011-10-16}
\isIndication{false}
\isCorrection{false}
\chapitre{Fonction logarithme et fonction puissance}
\sousChapitre{Fonction logarithme et fonction puissance}
\module{Analyse complexe}
\niveau{L3}
\difficulte{}

\contenu{
\texte{

}
\begin{enumerate}
    \item \question{En utilisant la détermination principale du
logarithme, on définit les fonctions $z\mapsto
z^{1/2}$, $z\mapsto (1-z)^{1/3}$, $z\mapsto \left(
(1-2i)z\right)^{2i/5}$. Donner leur domaine de définition.}
    \item \question{Soit $n\in{\Nn}$. Soit $f$ une détermination continue de $z^{1/n}$.
Montrer que $f(z)^n=z$ sur son domaine de définition.}
\end{enumerate}
}

\uuid{29UR}
\exo7id{7608}
\titre{exo7 7608}
\auteur{mourougane}
\organisation{exo7}
\datecreate{2021-08-10}
\isIndication{false}
\isCorrection{true}
\chapitre{Autre}
\sousChapitre{Autre}
\module{Analyse complexe}
\niveau{L3}
\difficulte{}

\contenu{
\texte{
On rappelle qu'à toute matrice $A=\begin{pmatrix} a&b\\c&d\end{pmatrix}$ de $SL(2,\Rr)$,
on associe l'application linéaire fractionnaire
$$\begin{array}{cccc}
 h_A :& \Cc-\{-\frac{d}{c}\}&\longrightarrow&\Cc-\{\frac{a}{c}\}\\&z&\longmapsto& \frac{az+b}{cz+d}
\end{array}$$
}
\begin{enumerate}
    \item \question{Montrer que $h_A$ envoie $\mathbb{H}$ sur $\mathbb{H}$.}
\reponse{Soit $z\in\mathbb{H}$.
\begin{eqnarray*}Im(h_A(z)&=&\frac{1}{2i}\left(\frac{az+b}{cz+d}-\overline{\frac{az+b}{cz+d}}\right)=\frac{1}{2i}\left(\frac{az+b}{cz+d}-\frac{a\overline{z}+b}{c\overline{z}+d}\right)\\
&=&\frac{1}{2i}\frac{(ad-bc)(z-\overline{z})}{|cz+d|^2}
=\frac{\det A}{|cz+d|^2}Im(z)>0.
\end{eqnarray*}
Donc, $h_A(z)\in\mathbb{H}$ et $h_A$ envoie le demi-plan $\mathbb{H}$ sur lui-même.}
    \item \question{Montrer que pour tout élément $z$ de $\mathbb{H}$, il existe $A\in SL(2,\Rr)$ 
tel que $h_A(i)=z$.}
\reponse{Soit $z=x+iy\in\mathbb{H}$. Soit $A\in SL(2,\Rr)$.
$$h_A(i)=z\iff ai+b=z(ci+d)\iff ai+b=(dx-cy)+i(cx+dy)$$
$$ \iff \left\{\begin{array}{ccc} b&=&dx-cy\\ a&=&cx+dy\end{array}\right..$$
    On peut par exemple choisir, puisque $y$ est strictement positif, $A=\frac{1}{\sqrt{y}}\begin{pmatrix}
    y&x\\
                    0&1
                   \end{pmatrix}$}
\end{enumerate}
}

\uuid{TiqK}
\exo7id{7572}
\titre{exo7 7572}
\auteur{mourougane}
\organisation{exo7}
\datecreate{2021-08-10}
\isIndication{false}
\isCorrection{false}
\chapitre{Fonction holomorphe}
\sousChapitre{Fonction holomorphe}
\module{Analyse complexe}
\niveau{L3}
\difficulte{}

\contenu{
\texte{

}
\begin{enumerate}
    \item \question{Soit une fonction holomorphe $f$ entière définie sur le plan complexe tout entier, on suppose que $ Re(f) \leq 0 $. 
 Montrer que $f$ est une fonction constante. On pourrait considérer la fonction $e^{f(z)}$.}
    \item \question{Soit une fonction holomorphe $f$ au voisinage de $0$.
 Montrer que si $ f \left( \frac{1}{n} \right) = f \left( \frac{1}{2n} \right)$ pour tout $n$ assez grand alors $f$ est une constante.}
    \item \question{Soit $D$ un ouvert de $\Cc$ et $f=u+iv : D\to\Cc$ une application holomorphe. On suppose que sur $D$,
$v=u^2$. Montrer que $f$ est une fonction constante.}
    \item \question{Soit une fonction entière $f$ telle que $\arrowvert f \arrowvert $ tend vers l'infini si $\arrowvert z \arrowvert $ tend vers l'infini. Montrer que :
\begin{enumerate}}
    \item \question{$f$ n'admet qu'un nombre fini de zéros.}
    \item \question{$f$ est un polynôme.}
\end{enumerate}
}

\uuid{LTGu}
\exo7id{2813}
\titre{exo7 2813}
\auteur{burnol}
\organisation{exo7}
\datecreate{2009-12-15}
\isIndication{false}
\isCorrection{true}
\chapitre{Formule de Cauchy}
\sousChapitre{Formule de Cauchy}
\module{Analyse complexe}
\niveau{L3}
\difficulte{}

\contenu{
\texte{
Soit $C$ un cercle de centre quelconque, parcouru dans le
sens direct, et ne passant pas par l'origine. Calculer
$\int_C z^n \,dz$ pour tout $n\in\Zz$ dans le cas où $C$
encercle l'origine, et dans le cas où $C$ n'encercle pas
l'origine. 
\emph{Indication} pour $n=-1$: soit $w$ l'affixe du
 centre du cercle, et $R$ son rayon. Paramétrer le cercle
 par $z = w(1 + \frac{R}{|w|}e^{i\theta})$,
 $-\pi<\theta\leq+\pi$, puis utiliser un développement en
 série en distinguant les cas $R>|w|$ et $R<|w|$. Ou encore
 invoquer la fonction $\mathrm{Log}(z/w)$.
}
\reponse{
Dans le cas o\`u $C$ n'encercle pas l'origine la fonction $z\mapsto z^n$ est holomorphe au voisinage du disque
bord\'e par $C$ et ceci pour tout $n\in \Z$. Par cons\'equent, $\int _C z^n dz=0$. Sinon on retrouve les valeurs
obtenues pr\'ec\'edemment. On rappelle \'egalement que :
$$\frac{1}{2i\pi}\int _C \frac{1}{z} dz$$
est l'indice $\mathrm{Ind}(C,0)$ de la courbe $C$ par rapport \`a l'origine. Cet indice est $1$ lorsque $C$ encercle
l'origine et $0$ sinon.
}
}

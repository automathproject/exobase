\uuid{ak5Q}
\exo7id{6649}
\titre{exo7 6649}
\auteur{queffelec}
\organisation{exo7}
\datecreate{2011-10-16}
\isIndication{false}
\isCorrection{false}
\chapitre{Fonction holomorphe}
\sousChapitre{Fonction holomorphe}
\module{Analyse complexe}
\niveau{L3}
\difficulte{}

\contenu{
\texte{
Soit $f(z)=u+iv$ une fonction holomorphe dans un ouvert connexe
$\Omega $. Montrer que les familles de courbes $u(x,y)=c_1$ et
$v(x,y)=c_2$ sont or\-tho\-go\-na\-les ; plus précisément, montrer qu'en
tout point d'intersection $z_0=x_0+iy_0$ de deux de ces courbes tel que
$f'(z_0)\ne 0$, leurs tangentes respectives sont perpendiculaires.
}
}

\uuid{lZFM}
\exo7id{6605}
\titre{exo7 6605}
\auteur{hueb}
\organisation{exo7}
\datecreate{2011-10-16}
\isIndication{false}
\isCorrection{false}
\chapitre{Tranformée de Laplace et de Fourier}
\sousChapitre{Tranformée de Laplace et de Fourier}
\module{Analyse complexe}
\niveau{L3}
\difficulte{}

\contenu{
\texte{
Vérifier les
propriétés suivantes de la transformation de Laplace $\mathcal{L}$
:
}
\begin{enumerate}
    \item \question{$\mathcal{L}(c_1 F_1 + c_2 F_2) =c_1 \mathcal{L}(F_1) + c_2
  \mathcal{L}(F_2)$.}
    \item \question{$\mathcal{L}(e^{at}F(t))(s) =(\mathcal{L} F)(s-a)$.}
    \item \question{Pour $G(t) = \begin{cases} F(t-a),\quad &t \geq a,\\
    0, \quad &t \leq a,
                \end{cases}$
                on a $(\mathcal{L}G)(s) =e^{-as}(\mathcal{L}F)(s)\ (a
                \geq 0)$.}
    \item \question{Pour $F_a(t) = F(at)$ on a $(\mathcal{L}F_a)(s) =
                \frac 1a (\mathcal{L}F)(\frac sa)$.}
    \item \question{Pour $G(t) = \int_0^t F(u) du$ on a
                $(\mathcal{L}G) (s) = \frac {(\mathcal{L} F)(s)} s$.}
    \item \question{$\mathcal{L}(t^n F(t)) = (-1)^n (\mathcal{L}
                F)^{(n)}$}
    \item \question{Si $\lim_{t\to 0} \frac {F(t)}t$ existe,
                $\mathcal{L}(\frac {F(t)}t)(s) = \int_s^{\infty}
                (\mathcal{L} F)(\zeta) d \zeta$.\}
    \item \question{Si $F$ est périodique, $F(t+T) = F(t)$, alors
                $$(\mathcal{L}F)(s) = \frac 1{1-e^{-sT}}\int_0^T F(t)
                e^{-st} dt$$}
    \item \question{$\lim_{s \to +\infty} (\mathcal{L} F)(s) = 0$.}
    \item \question{$\lim_{s \to +\infty} (\mathcal{L}F)(s) = 0$.}
    \item \question{$\lim_{t \to 0, t > 0} F(t) = \lim_{s \to \infty}
                s (\mathcal{L}F)(s)$ si les limites indiquées
                existent (Théorème de la valeur initiale).}
    \item \question{$\lim_{t \to \infty} F(t) = \lim_{s \to 0} s
                (\mathcal{L} F)(s)$ si les limites indiquées
                existent (Théorème de la valeur finale).}
\end{enumerate}
}

\uuid{7Rpf}
\exo7id{6712}
\titre{exo7 6712}
\auteur{queffelec}
\organisation{exo7}
\datecreate{2011-10-16}
\isIndication{false}
\isCorrection{false}
\chapitre{Formule de Cauchy}
\sousChapitre{Formule de Cauchy}
\module{Analyse complexe}
\niveau{L3}
\difficulte{}

\contenu{
\texte{
Soit $f$ une fonction holomorphe sur $\{z\in\C\vert\
\vert z\vert <R\}$. Si $f(z)=\sum_{n=0}^\infty a_nz^n$ pour $\vert
z\vert<R$, on pose pour $0\le r< R$ :
\begin{eqnarray*}\lefteqn{M(r,f)=\max_{\vert z\vert =r}\vert f(z)\vert
,\  M_1(r,f)=\sum_{n=0}^\infty \vert a_n\vert r^n,}\\ 
M_2(r,f)&=&\left({1\over 2\pi}\int_0^{2\pi}\vert f(re^{i\theta })\vert
^2d\theta \right)^{1/2}.\end{eqnarray*}
}
\begin{enumerate}
    \item \question{\begin{enumerate}}
    \item \question{Montrer que pour $0\le r<R$,
$$M_2(r,f)=\left( \sum_{n=0}^\infty\vert
a_n\vert^2r^{2n}\right)^{1/2}.$$}
    \item \question{Déduire de (a) que $r\mapsto M_2(r,f)$ est une fonction continue
croissante.}
    \item \question{Déduire de (a) une autre démonstration des inégalités de Cauchy.}
\end{enumerate}
}

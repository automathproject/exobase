\uuid{lxXI}
\exo7id{6714}
\titre{exo7 6714}
\auteur{queffelec}
\organisation{exo7}
\datecreate{2011-10-16}
\isIndication{false}
\isCorrection{false}
\chapitre{Formule de Cauchy}
\sousChapitre{Formule de Cauchy}
\module{Analyse complexe}
\niveau{L3}
\difficulte{}

\contenu{
\texte{
\label{gijsexophi}
Soit $D$ le disque unité ouvert : 
$$D=\{ z\in \C\vert\ \vert z\vert
<1\}$$
 et $C$ le cercle unité : 
$$C=\{ z\in \C\vert\ \vert
z\vert =1\} .$$
Pour $a\in D$, on considère l'application
homographique
$\Phi _a$ :
$$\Phi _a(z)={z-a\over 1-\overline{a}z}.$$
}
\begin{enumerate}
    \item \question{Montrer que $\Phi _a(C)\subset C$, puis que $\Phi_a$ est une bijection
de $D$ sur lui-même, de réciproque $\Phi _{-a}$.}
    \item \question{Soit $f$ un biholomorphisme de $D$, c'est-à-dire une fonction
holomorphe de
$D$ sur lui-même, bijective, telle que $f^{-1}$ soit aussi holomorphe.
Soit $a=f^{-1}(0)$. En considérant $f\circ (\Phi _a)^{-1}$ et sa
réciproque, montrer qu'il existe $\varphi\in {\Rr}$ tel que $$\forall
z\in D,\ f(z)=e^{i\varphi}\Phi _a(z),$$ c'est-à-dire qu'à rotation près, les
seuls biholomorphismes du disque sont les $\Phi _a$.}
\end{enumerate}
}

\uuid{hw1Z}
\exo7id{2829}
\titre{exo7 2829}
\auteur{burnol}
\organisation{exo7}
\datecreate{2009-12-15}
\isIndication{false}
\isCorrection{true}
\chapitre{Théorème des résidus}
\sousChapitre{Théorème des résidus}
\module{Analyse complexe}
\niveau{L3}
\difficulte{}

\contenu{
\texte{
Montrer que si une fonction entière $f$ a sa partie réelle
bornée supérieurement  alors elle est constante (considérer
$\exp(f)$).
}
\reponse{
Soit $g(z) = e^{f(z)}$. On a $|g(z)| =e^{\Re (f(z))}$. Par cons\'equent $g$ est une fonction enti\`ere born\'ee. Elle est donc constante par d'Alembert-Liouville.
}
}

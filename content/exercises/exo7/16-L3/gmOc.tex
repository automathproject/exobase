\uuid{gmOc}
\exo7id{2812}
\titre{exo7 2812}
\auteur{burnol}
\organisation{exo7}
\datecreate{2009-12-15}
\isIndication{false}
\isCorrection{true}
\chapitre{Formule de Cauchy}
\sousChapitre{Formule de Cauchy}
\module{Analyse complexe}
\niveau{L3}
\difficulte{}

\contenu{
\texte{
\label{ex:burnol2.2.3}
On note $C$ le cercle de rayon $1$ parcouru dans le sens
direct. Calculer $ \int_C z^n \,dz$ et  $\int_\gamma z^n
\,dz$ pour tout $n\in\Zz$, et vérifier qu'il y a toujours
égalité (ici $\gamma = \partial\mathcal{R}$ est à nouveau le bord du
carré qui a été 
 utilisé dans les exercices précédents). Calculer $\int_C
 \overline{z}^n \,dz$ et
$\int_\gamma \overline{z}^n \,dz$ et trouver les cas
 d'égalités et
d'inégalités.
}
\reponse{
Voir la correction de l'exercice \ref{ex:burnol2.2.1}.
}
}

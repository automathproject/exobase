\uuid{a2Ej}
\exo7id{6692}
\titre{exo7 6692}
\auteur{queffelec}
\organisation{exo7}
\datecreate{2011-10-16}
\isIndication{false}
\isCorrection{false}
\chapitre{Formule de Cauchy}
\sousChapitre{Formule de Cauchy}
\module{Analyse complexe}
\niveau{L3}
\difficulte{}

\contenu{
\texte{
Soit $f$ une fonction entière telle que $|f(z)|\to+\infty$ quand
$|z|\to+\infty$.
}
\begin{enumerate}
    \item \question{Montrer que $f$ n'a qu'un nombre fini de zéros dans $\Cc$ notés $z_1,\ldots
z_k$.}
    \item \question{En déduire que $f$ est un polyn\^ome.
(Pour cela considérer  $f(z)/P(z)=g(z)$ où $P(z)=(z-z_1)\cdots (z-z_k)$ et
montrer que
$1/g$ est entière) .}
\end{enumerate}
}

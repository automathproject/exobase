\uuid{cmHi}
\exo7id{6628}
\titre{exo7 6628}
\auteur{queffelec}
\organisation{exo7}
\datecreate{2011-10-16}
\isIndication{false}
\isCorrection{false}
\chapitre{Fonction holomorphe}
\sousChapitre{Fonction holomorphe}
\module{Analyse complexe}
\niveau{L3}
\difficulte{}

\contenu{
\texte{
Soit $(u_n)_{n\ge 0}$ une suite de réels. 
On dit que $a\in \overline{\Rr}$ est une valeur d'adhérence de la suite
$(u_n)_{n\ge 0}$ s'il existe une sous-suite de la suite $(u_n)_{n\ge 0}$
qui tend vers $a$. Montrer que $L=\limsup_{n\to\infty}u_n=\inf_n\sup_{k\ge n}u_k$ 
et $l=\liminf_{n\to\infty}u_n=\sup_n\inf_{k\ge n}u_k$ sont des valeurs 
d'adhérence de la suite $(u_n)_{n\ge 0}$. Vérifier que ce sont 
respectivement la plus grande et la plus petite des valeurs d'adhérence.
}
}

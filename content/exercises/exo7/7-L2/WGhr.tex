\uuid{WGhr}
\exo7id{6928}
\titre{exo7 6928}
\auteur{ruette}
\organisation{exo7}
\datecreate{2013-01-24}
\isIndication{false}
\isCorrection{true}
\chapitre{Probabilité continue}
\sousChapitre{Loi faible des grands nombres}
\module{Probabilité et statistique}
\niveau{L2}
\difficulte{}

\contenu{
\texte{
Pour étudier les particules émises par une substance radioactive, 
on dispose d'un détecteur. On note $X$ la variable aléatoire 
représentant le nombre de particules qui atteignent le détecteur pendant 
un intervalle de temps $\Delta t$. Le nombre maximal de particules que le 
détecteur peut compter pendant un intervalle de temps $\Delta t$ est de 
$10^3$. 
On suppose que  $X$ suit une loi de Poisson de paramètre 
$\lambda=10^2$. Donner  une majoration de la probabilité que $X$ 
dépasse~$10^3$.\\ \textit{(On rappelle que l'espérance et la variance d'une loi
de Poisson $\mathcal{P}(\lambda)$ sont égales à $\lambda$.)}
}
\reponse{
Inégalité de Bienaymé-Tchebychev : 
$P(|X-10^2|\ge 10^3-10^2)\le \frac{\text{Var}(X)}{(10^3-10^2)^2}\le 
\frac{\text{Var}(X)}{(10^3)^2}=10^{-4}$. Donc $P(X\ge 10^3)\le 10^{-4}$.
}
}

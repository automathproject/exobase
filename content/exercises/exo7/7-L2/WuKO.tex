\uuid{WuKO}
\exo7id{6007}
\titre{exo7 6007}
\auteur{quinio}
\organisation{exo7}
\datecreate{2011-05-20}
\isIndication{false}
\isCorrection{true}
\chapitre{Probabilité discrète}
\sousChapitre{Variable aléatoire discrète}
\module{Probabilité et statistique}
\niveau{L2}
\difficulte{}

\contenu{
\texte{
Un avion peut accueillir 20 personnes; des statistiques montrent 
que 25\% clients ayant réservé ne viennent pas.
Soit $X$ la variable aléatoire: <<nombre de clients qui viennent après
réservation parmi 20>>.
Quelle est la loi de $X$ ? (on ne donnera que la forme générale)
quelle est son espérance, son écart-type ?
Quelle est la probabilité pour que $X$ soit égal à 15 ?
}
\reponse{
Soit $X$ la variable aléatoire nombre de clients
qui viennent après réservation parmi 20.
La loi de $X$ est une loi binomiale de paramètres $n=20$, $p=0.75$.
Son espérance est $np=15$, son écart-type est $\sqrt{np(1-p)}=\sqrt{15\cdot 0.25}$.
La probabilité pour que $X$ soit égal à 15 est $\binom{20}{15}0.75^{15}0.25^{5}=0.202\,33$.
}
}

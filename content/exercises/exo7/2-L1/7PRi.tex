\uuid{7PRi}
\exo7id{641}
\titre{exo7 641}
\auteur{bodin}
\organisation{exo7}
\datecreate{1998-09-01}
\isIndication{false}
\isCorrection{false}
\chapitre{Continuité, limite et étude de fonctions réelles}
\sousChapitre{Continuité : théorie}
\module{Analyse}
\niveau{L1}
\difficulte{}

\contenu{
\texte{

}
\begin{enumerate}
    \item \question{Soit $f$ une fonction continue sur $]a,b[$ telle que
$f(]a,b[)\subset [a,b]$. Montrer, par consid\'eration de $\phi(x)=f(x)-x$, qu'il
existe $c$ dans $[a,b]$ tel que $f(c)=c$.}
    \item \question{Soit $f$ une fonction continue sur $[0,1]$ telle que $f(0)=f(1)$.
Montrer qu'il existe $c$ dans $[0,\frac{1}{2}]$ tel que $f(c) = f(c+\frac{1}{2})$.}
    \item \question{Un mobile parcours, \`a vitesse continue, une distance $d$ en
une unit\'e de temps. Montrer qu'il existe un intervalle d'une demi-unit\'e de temps
 pendant lequel il parcourt une distance $\frac{d}{2}$.}
\end{enumerate}
}

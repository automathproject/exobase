\uuid{hCOd}
\exo7id{3910}
\titre{exo7 3910}
\auteur{quercia}
\organisation{exo7}
\datecreate{2010-03-11}
\isIndication{false}
\isCorrection{true}
\chapitre{Continuité, limite et étude de fonctions réelles}
\sousChapitre{Etude de fonctions}
\module{Analyse}
\niveau{L1}
\difficulte{}

\contenu{
\texte{
Calculer $\tan p - \tan q$. En déduire la valeur de
$S_n = \sum_{k=1}^n \frac 1{\cos(k\theta)\cos((k+1)\theta)}$, $\theta \in \R$.
}
\reponse{
$\tan p - \tan q = \frac {\sin(p-q)}{\cos p\cos q}$.

$S_n = \begin{cases}\frac {\tan((n+1)\theta) - \tan\theta}{\sin\theta}
                     &\text{ si } \sin\theta \ne 0 \cr
              n      &\text{ si } \theta \equiv 0 (\mathrm{mod}\, {2\pi}) \cr
              -n     &\text{ si } \theta \equiv \pi (\mathrm{mod}\, {2\pi}).\end{cases}$
}
}

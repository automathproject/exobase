\uuid{I0Vq}
\exo7id{733}
\titre{exo7 733}
\auteur{ridde}
\organisation{exo7}
\datecreate{1999-11-01}
\video{_YeLn2_0Tso}
\isIndication{false}
\isCorrection{true}
\chapitre{Dérivabilité des fonctions réelles}
\sousChapitre{Autre}
\module{Analyse}
\niveau{L1}
\difficulte{}

\contenu{
\texte{
D\'eterminer les extremums de $f (x) = x^4-x^3 + 1$ sur $\Rr$.
}
\reponse{
$f'(x) = 4x^3-3x^2 = x^2(4x-3)$ donc les extremums
appartiennent à  $\{0,\frac 34\}$. Comme $f''(x) = 12x^2-6x
= 6x(2x-1)$. Alors $f''$ ne s'annule pas en $\frac 34$, donc
$\frac 34$ donne un extremum local (qui est même un minimum global).
Par contre $f''(0) = 0$ et $f'''(0)\not=0$ donc $0$ est
un point d'inflexion qui n'est pas un extremum (m\^eme pas local,
pensez \`a un fonction du type  $x \mapsto x^3$).
}
}

\uuid{5alP}
\exo7id{5227}
\titre{exo7 5227}
\auteur{rouget}
\organisation{exo7}
\datecreate{2010-06-30}
\isIndication{false}
\isCorrection{true}
\chapitre{Suite}
\sousChapitre{Convergence}
\module{Analyse}
\niveau{L1}
\difficulte{}

\contenu{
\texte{
Etudier la suite $(u_n)$ définie par $\sqrt{n+1}-\sqrt{n}=\frac{1}{2\sqrt{n+u_n}}$.
}
\reponse{
Soit $n\in\Nn$.

\begin{align*}
\frac{1}{2\sqrt{n+u_n}}=\sqrt{n+1}-\sqrt{n}&\Leftrightarrow 2\sqrt{n+u_n}=\frac{1}{\sqrt{n+1}-\sqrt{n}}\Leftrightarrow2\sqrt{n+u_n}=\sqrt{n+1}+\sqrt{n}\\
 &4(n+u_n)=(\sqrt{n+1}+\sqrt{n})^2\Leftrightarrow u_n=-n+\frac{1}{4}(2n+1+2\sqrt{n(n+1)})\\
 &\Leftrightarrow u_n=\frac{1}{4}(-2n+1+2\sqrt{n(n+1)})
\end{align*}
Par suite, quand $n$ tend vers $+\infty$,

\begin{align*}
u_n&=-\frac{n}{2}+\frac{1}{4}+\frac{1}{2}\sqrt{n^2+n}=
\frac{1}{4}+\frac{n}{2}\left(\sqrt{1+\frac{1}{n}}-1\right)=\frac{1}{4}+\frac{n}{2}\frac{1/n}{\sqrt{1+\frac{1}{n}}+1}\\
 &=
\frac{1}{4}+\frac{1}{2}\frac{1}{\sqrt{1+\frac{1}{n}}+1}=\frac{1}{4}+\frac{1}{4}+o(1)=\frac{1}{2}+o(1).
\end{align*}
La suite $(u_n)$ converge et a pour limite $\frac{1}{2}$.
}
}

\uuid{PDwZ}
\exo7id{5229}
\titre{exo7 5229}
\auteur{rouget}
\organisation{exo7}
\datecreate{2010-06-30}
\isIndication{false}
\isCorrection{true}
\chapitre{Suite}
\sousChapitre{Suite définie par une relation de récurrence}
\module{Analyse}
\niveau{L1}
\difficulte{}

\contenu{
\texte{
Soient $(u_n)$ et $(v_n)$ les suites définies par la donnée de $u_0$ et $v_0$ et les relations de récurrence 

$$u_{n+1}=\frac{2u_n+v_n}{3}\;\mbox{et}\;v_{n+1}=\frac{u_n+2v_n}{3}.$$
Etudier les suites $u$ et $v$ puis déterminer $u_n$ et $v_n$ en fonction de $n$ en recherchant des combinaisons linéaires intéressantes de $u$ et $v$. En déduire $\lim_{n\rightarrow +\infty}u_n$ et $\lim_{n\rightarrow +\infty}v_n$.
}
\reponse{
Pour tout entier naturel $n$, on a $\left\{
\begin{array}{l}
u_{n+1}-u_n=\frac{1}{3}(v_n-u_n)\\
v_{n+1}-v_n=-\frac{1}{3}(v_n-u_n)\\
v_{n+1}-u_{n+1}=\frac{1}{3}(v_n-u_n)
\end{array}
\right.$.
La dernière relation montre que la suite $v-u$ garde un signe constant puis les deux premières relations montrent que pour tout entier naturel $n$, $\mbox{sgn}(u_{n+1}-u_n)=\mbox{sgn}(v_n-u_n)$ et 
$\mbox{sgn}(v_{n+1}-v_n)=-\mbox{sgn}(v_n-u_n)$. Les suites $u$ et $v$ sont donc monotones de sens de variation opposés.
Si par exemple $u_0\leq v_0$, alors, pour tout naturel $n$, on a~:

$$u_0\leq u_n\leq u_{n+1}\leq v_{n+1}\leq v_n\leq v_0.$$
Dans ce cas, la suite $u$ est croissante et majorée par $v_0$ et donc converge vers un certain réel $\ell$. De même, la suite $v$ est décroissante et minorée par $u_0$ et donc converge vers un certain réel $\ell'$. Enfin, puisque pour tout entier naturel $n$, on a $u_{n+1}=\frac{2u_n+v_n}{3}$, on obtient par passage à la limite quand $n$ tend vers l'infini,  $\ell=\frac{2\ell+\ell'}{3}$ et donc $\ell=\ell'$. Les suites $u$ et $v$ sont donc adjacentes. Si $u_0>v_0$, il suffit d'échanger les rôles de $u$ et $v$.
\textbf{Calcul des suites $u$ et $v$.}
Pour $n$ entier naturel donné, on a $v_{n+1}-u_{n+1}=\frac{1}{3}(v_n-u_n)$. La suite $v-u$ est géométrique de raison $\frac{1}{3}$. Pour tout naturel $n$, on a donc $v_n-u_n=\frac{1}{3^n}(v_0-u_0)$.
D'autre part, pour $n$ entier naturel donné, $v_{n+1}+u_{n+1}=v_n+u_n$. La suite $v+u$ est constante et donc, pour tout entier naturel $n$, on a $v_n+u_n=v_0+u_0$.
En additionnant et en retranchant les deux égalités précédentes, on obtient pour tout entier naturel $n$~:

$$u_n=\frac{1}{2}\left(v_0+u_0+\frac{1}{3^n}(v_0-u_0)\right)\;\mbox{et}\;v_n=\frac{1}{2}\left(v_0+u_0-\frac{1}{3^n}(v_0-u_0)\right).$$
En particulier, $\ell=\ell'=\frac{u_0+v_0}{2}$.
}
}

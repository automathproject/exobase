\uuid{bxm2}
\exo7id{5084}
\titre{exo7 5084}
\auteur{rouget}
\organisation{exo7}
\datecreate{2010-06-30}
\isIndication{false}
\isCorrection{true}
\chapitre{Fonctions circulaires et hyperboliques inverses}
\sousChapitre{Fonctions circulaires inverses}
\module{Analyse}
\niveau{L1}
\difficulte{}

\contenu{
\texte{
Domaine de définition et calcul des fonctions suivantes :
}
\begin{enumerate}
    \item \question{$x\mapsto\sin(\Arcsin x)$,}
    \item \question{$x\mapsto\Arcsin(\sin x)$,}
    \item \question{$x\mapsto\cos(\Arccos x)$,}
    \item \question{$x\mapsto\Arccos(\cos x)$,}
    \item \question{$x\mapsto\tan(\Arctan x)$,}
    \item \question{$x\mapsto\Arctan(\tan
x)$.}
\reponse{
$\Arcsin x$ existe si et seulement si $x$ est dans $[-1,1]$. Donc, $\sin(\Arcsin x)$ existe si et seulement si
$x$ est dans $[-1,1]$ et pour $x$ dans $[-1,1]$, $\sin(\Arcsin x)=x$.
 \item  $\Arcsin(\sin x)$ existe pour tout réel $x$ mais ne vaut $x$ que si $x$ est 
dans $\left[-\frac{\pi}{2},\frac{\pi}{2}\right]$.
\textbullet~S'il existe un entier relatif $k$ tel que
$-\frac{\pi}{2}+2k\pi\leq x<\frac{\pi}{2}+2k\pi$, alors $-\frac{\pi}{2}\leq x-2k\pi<\frac{\pi}{2}$ et
donc

$$\Arcsin(\sin x)=\Arcsin(\sin(x-2k\pi))=x-2k\pi.$$
De plus, on a $k\leq\frac{x}{2\pi}+\frac{1}{4}<k+\frac{1}{2}$
et donc $k=E\left(\frac{x}{2\pi}+\frac{1}{4}\right)$.
\textbullet~S'il existe un entier relatif $k$ tel que $\frac{\pi}{2}+2k\pi\leq
x<\frac{3\pi}{2}+2k\pi$, alors $-\frac{\pi}{2}<\pi-x+2k\pi\leq\frac{\pi}{2}$ 
et donc

$$\Arcsin(\sin x)=\Arcsin(\sin(\pi-x+2k\pi))=\pi-x+2k\pi.$$
De plus, $k\leq\frac{x}{2\pi}-\frac{1}{4}<k+\frac{1}{2}$ et donc $k=E\left(\frac{x}{2\pi}-\frac{1}{4}\right)$.
 \item  $\Arccos x$ existe si et seulement si $x$ est dans $[-1,1]$. Donc, $\cos(\Arccos x)$ existe si et seulement si
$x$ est dans $[-1,1]$ et pour $x$ dans $[-1,1]$, $\cos(\Arccos x)=x$.
 \item  $\Arccos(\cos x)$ existe pour tout réel $x$ mais ne vaut $x$ que si $x$ est dans $[0,\pi]$.
\textbullet~S'il existe un
entier relatif $k$ tel que $2k\pi\leq x<\pi+2k\pi$, alors $\Arccos(\cos x)=x-2k\pi$ avec $k=E\left(\frac{x}{2\pi}\right)$.
\textbullet~S'il existe un entier relatif $k$ tel que $-\pi+2k\pi\leq x<2k\pi$ alors 
$\Arccos(\cos x)=\Arccos(\cos(2k\pi-x))=2k\pi-x$ avec $k=E\left(\frac{x+\pi}{2\pi}\right)$.
 \item  Pour tout réel $x$, $\tan(\Arctan x)=x$.
 \item  $\Arctan(\tan x)$ existe si et seulement si $x$ n'est pas
dans $\frac{\pi}{2}+\pi\Zz$ et pour ces $x$, il existe un entier relatif $k$ 
tel que $-\frac{\pi}{2}+k\pi<x<\frac{\pi}{2}+k\pi$. Dans ce cas, $\Arctan(\tan x)=\Arctan(\tan(x-k\pi))=x-k\pi$ avec
$k=E\left(\frac{x}{\pi}+\frac{1}{2}\right)$.
}
\end{enumerate}
}

\uuid{IfsT}
\exo7id{5445}
\titre{exo7 5445}
\auteur{rouget}
\organisation{exo7}
\datecreate{2010-07-10}
\isIndication{false}
\isCorrection{true}
\chapitre{Calcul d'intégrales}
\sousChapitre{Autre}
\module{Analyse}
\niveau{L1}
\difficulte{}

\contenu{
\texte{

}
\begin{enumerate}
    \item \question{Soit $f$ une application de classe $C^1$ sur $[0,1]$ telle que $f(1)\neq0$.

Pour $n\in\Nn$, on pose $u_n=\int_{0}^{1}t^nf(t)\;dt$. Montrer que $\lim_{n\rightarrow +\infty}u_n=0$ puis déterminer un équivalent simple de $u_n$ quand $n$ tend vers $+\infty$ (étudier $\lim_{n\rightarrow +\infty}nu_n$).}
\reponse{$f$ est continue sur le segment $[0,1]$ et est donc bornée sur ce segment. Soit $M$ un majorant de $|f|$ sur $[0,1]$. Pour $n\in\Nn$,

$$|u_n|\leq\int_{0}^{1}t^n|f(t)|\;dt\leq M\int_{0}^{1}t^n\;dt=\frac{M}{n+1},$$

et comme $\lim_{n\rightarrow +\infty}\frac{M}{n+1}=0$, on a montré que $\lim_{n\rightarrow +\infty}u_n=0$.

Soit $n\in\Nn$. Puisque $f$ est de classe $C^1$ sur $[0,1]$, on peut effectuer une intégration par parties qui fournit

$$u_n=\left[\frac{t^{n+1}}{n+1}f(t)\right]_{0}^{1}-\frac{1}{n+1}\int_{0}^{1}t^{n+1}f'(t)\;dt=\frac{f(1)}{n+1}-\frac{1}{n+1}\int_{0}^{1}t^{n+1}f'(t)\;dt.$$

Puisque $f'$ est continue sur $[0,1]$, $\lim_{n\rightarrow +\infty}\int_{0}^{1}t^{n+1}f'(t)\;dt=0$ ou encore $-\frac{1}{n+1}\int_{0}^{1}t^{n+1}f'(t)\;dt=o(\frac{1}{n})$. D'autre part, puisque $f(1)\neq0$, $\frac{f(1)}{n+1}\sim\frac{f(1)}{n}$ ou encore $\frac{f(1)}{n+1}=\frac{f(1)}{n}+o(\frac{1}{n})$. Finalement, $u_n=\frac{f(1)}{n}+o(\frac{1}{n})$, ou encore

$$u_n\sim\frac{f(1)}{n}.$$}
    \item \question{Mêmes questions en supposant que $f$ est de classe $C^2$ sur $[0,1]$ et que $f(1)=0$ et $f'(1)\neq0$.}
\reponse{Puisque $f$ est de classe $C^1$ sur $[0,1]$ et que $f(1)=0$, une intégration par parties fournit

$u_n=-\frac{1}{n+1}\int_{0}^{1}t^{n+1}f'(t)\;dt$. Puisque $f'$ est de classe $C^1$ sur $[0,1]$ et que $f'(1)\neq0$, le 1) appliqué à $f'$ fournit

$$u_n=-\frac{1}{n+1}\int_{0}^{1}t^{n+1}f'(t)\;dt\sim-\frac{1}{n}\frac{f'(1)}{n}=-\frac{f'(1)}{n^2}.$$

Par exemple, $\int_{0}^{1}t^n\sin\frac{\pi t}{2}\;dt\sim\frac{1}{n}$ et $\int_{0}^{1}t^n\cos\frac{\pi t}{2}\;dt\sim\frac{\pi}{2n^2}$}
\end{enumerate}
}

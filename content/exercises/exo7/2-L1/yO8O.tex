\uuid{yO8O}
\exo7id{1229}
\titre{exo7 1229}
\auteur{legall}
\organisation{exo7}
\datecreate{1998-09-01}
\isIndication{false}
\isCorrection{true}
\chapitre{Dérivabilité des fonctions réelles}
\sousChapitre{Applications}
\module{Analyse}
\niveau{L1}
\difficulte{}

\contenu{
\texte{
Pour tout $  n   $ entier sup\'erieur o\`u \'egal \`a $  2  $, on consid\`ere le polyn\^ome de
degr\' e $  n   $ \`a coefficients r\' eels~:
$$P_n(X)=X^n+X^{n-1}+X^2+X-1$$
}
\begin{enumerate}
    \item \question{Soit $  n \geq 2  .$ Montrer que $  P_n  $ a une unique racine r\' eelle positive que l'on nommera $  \lambda _n  .$ (On pourra \'etudier l'application $  X\mapsto P_n(X)  .$)}
\reponse{Pour tout $  n \geq 2  $ on a : $  P_n(0)=-1  $ et $  P_n(1)=3  .$ Comme l'application
$  X\mapsto P_n(X)  $ est continue, elle s'annulle en (au moins)
un point de l'intervalle $ \rbrack 0, 1 \lbrack .$  Comme par
ailleurs, pour tout $  X $ positif, $
P'_n(X)=nX^{n-1}+(n-1)X^{n-2}+2X+1    $ est strictement positif,
l'application $  X\mapsto P_n(X)  $ est strictement croissante sur
$  {\R}_+  $ et s'annule en au plus un point de $  {\R}_+  .$ En
cons\' equence $  P_n   $ a une unique racine positive $  \lambda
_n  $ qui de plus satisfait \`a l'in\' egalit\' e $  0<\lambda
_n<1   .$}
    \item \question{Montrer que la suite $  (\lambda _n)_{n\geq 2}  $ est croissante puis qu'elle
converge vers une limite que l'on notera $  \ell   .$}
\reponse{Pour tout $  X\in   \rbrack 0, 1
\lbrack   ,   P_n(X)-P_{n-1}(X)=X^n-X^{n-2}<0  .$ En particulier $
P_n(\lambda _{n-1})<0 $ donc $ \lambda _n>\lambda _{n-1}  .$ La
suite $  (\lambda _n)_{n\geq 2}  $ est donc croissante et major\'
ee (cf 1.) : elle est convergente.}
    \item \question{Montrer que $  \ell  $ est racine du polyn\^ome $  X^2+X-1  .$ En d\' eduire sa valeur.}
\reponse{Pour tout $  n \geq 2  $ on a  :
$  \lambda _n^n+\lambda _n^{n-1}=-\lambda _n^2-\lambda _n+1  .$ Or
$ \displaystyle{ P_n\Bigl( \frac{3}{ 4}\Bigr) >\Bigl( \frac{3}{
4}\Bigr) ^2+\frac{3}{ 4}-1>0} $ donc la suite $  (\lambda
_n^n+\lambda _n^{n-1})_{n\in {\N}}  $ satisfait aux in\' egalit\'
es $  0<\lambda _n^n+\lambda _n^{n-1}<\displaystyle{ \Bigl(
\frac{3}{ 4}\Bigr) ^n+\Bigl(\frac {3}{ 4} \Bigr) ^{n+1}}  $ et
converge vers $  0  .$ Il en va de m\^eme de la suite $  (-\lambda
_n^2-\lambda _n+1)_{n\geq 2}  .$ En passant \`a la limite, on
obtient l'\' egalit\' e~: $  \ell ^2+\ell -1=0  .$ La seule
solution positive de cette \' equation \' etant $
\displaystyle{\frac { -1+\sqrt{5} }{ 2}}  ,$ on a l'\' egalit\'
e~: $  \ell =\displaystyle{ \frac{ -1+\sqrt{5}}{ 2}}  .$}
\end{enumerate}
}

\uuid{hpIe}
\exo7id{5394}
\titre{exo7 5394}
\auteur{rouget}
\organisation{exo7}
\datecreate{2010-07-06}
\isIndication{false}
\isCorrection{true}
\chapitre{Continuité, limite et étude de fonctions réelles}
\sousChapitre{Autre}
\module{Analyse}
\niveau{L1}
\difficulte{}

\contenu{
\texte{
Soit $f$ définie sur $[0,+\infty[$ à valeurs dans $[0,+\infty[$, continue sur $[0,+\infty[$ telle que $\frac{f(x)}{x}$ a une limite réelle $\ell\in[0,1[$ quand $x$ tend vers $+\infty$. Montrer que $f$ a un point fixe.
}
\reponse{
Puisque $\frac{f(x)}{x}$ tend vers $\ell\in[0,1[$, il existe $A>0$ tel que pour $x\geq A$, $\frac{f(x)}{x}\leq\frac{\ell+1}{2}<1$. Mais alors, $f(A)<A$ (et $f(0)\geq0$) ce qui ramène à la situation de l'exercice \ref{exo:rouconti}~:~pour $x\in[0,A]$, soit $g(x)=f(x)-x$...
}
}

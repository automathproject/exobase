\uuid{emUv}
\exo7id{1219}
\titre{exo7 1219}
\auteur{ridde}
\organisation{exo7}
\datecreate{1999-11-01}
\isIndication{false}
\isCorrection{true}
\chapitre{Continuité, limite et étude de fonctions réelles}
\sousChapitre{Fonctions équivalentes, fonctions négligeables}
\module{Analyse}
\niveau{L1}
\difficulte{}

\contenu{
\begin{enumerate}
	\item \question{Calculer $ \lim_{x \to +\infty} \sqrt[3]{x^3 + x^2} - \sqrt[3]{x^3 - x^2} $}
	\reponse{$\frac{2}{3}$}
	
	\item \question{Trouver un équivalent en $ +\infty $ de $ \sqrt{x^2 + \sqrt{x^4 + 1}} - x\sqrt{2} $}
	\reponse{$\frac{\sqrt{2}}{8x^3}$}
	
	\item \question{Calculer $ \lim_{x \to 0} \frac{\tan(ax) - \sin(ax)}{\tan(bx) - \sin(bx)} $}
	\reponse{$\frac{a^3}{b^3}$}
	
	\item \question{Calculer $ \lim_{x \to \frac{\pi}{4}} \left(x - \frac{\pi}{4}\right) \tan\left(x + \frac{\pi}{4}\right) $}
	\reponse{$-1$}
	
	\item \question{Calculer $ \lim_{x \to \frac{\pi}{4}} \frac{\cos(x) - \sin(x)}{(4x - \pi)\tan(x)} $}
	\reponse{$-\frac{\sqrt{2}}{4}$}
	
	\item \question{Trouver un équivalent en 0 de $ \frac{\tan(x - x\cos(x))}{\sin(x) + \cos(x) - 1} $}
	\reponse{$\frac{1}{2}x^2$}
	
	\item \question{Trouver un équivalent en $ \frac{\pi}{4} $ de $ \left(\tan(2x) + \tan\left(x + \frac{\pi}{4}\right)\right) \left(\cos\left(x + \frac{\pi}{4}\right)\right)^2 $}
	\reponse{$-\frac{3}{2}\left(x - \frac{\pi}{4}\right)$}
	
	\item \question{Calculer $ \lim_{x \to 0} x^{\frac{1}{1 + 2\ln(x)}} $}
	\reponse{$\sqrt{e}$}
	
	\item \question{Calculer $ \lim_{x \to \frac{1}{2}} (2x^2 - 3x + 1)\tan(\pi x) $}
	\reponse{$\frac{1}{\pi}$}
	
	\item \question{Calculer $ \lim_{x \to 0} \frac{(\sin(x))^{\sin(x)} - 1}{(\tan(x))^{\tan(x)} - 1} $}
	\reponse{$1$}
	
	\item \question{Trouver un équivalent en $ +\infty $ de $ \frac{\sqrt{1 + x^2}}{\sin\left(\frac{1}{x}\right)} \ln\left(\frac{x}{x + 1}\right) $}
	\reponse{$x$}
\end{enumerate}

}

\uuid{jLkp}
\exo7id{4474}
\titre{exo7 4474}
\auteur{quercia}
\organisation{exo7}
\datecreate{2010-03-14}
\isIndication{false}
\isCorrection{true}
\chapitre{Série numérique}
\sousChapitre{Autre}
\module{Analyse}
\niveau{L1}
\difficulte{}

\contenu{
\texte{
Soit $(u_n)$ une série convergente à termes positifs décroissants.
}
\begin{enumerate}
    \item \question{Montrer que $nu_n\to 0$ lorsque $n\to\infty$.}
\reponse{$nu_{2n} \le \sum_{k=n+1}^{2n} u_k$,
             $nu_{2n+1} \le \sum_{k=n+2}^{2n+1} u_k$.}
    \item \question{Montrer que $\sum_{u_k \ge 1/n} \frac 1{u_k} = \text{o}(n^2)$.}
\reponse{$\varepsilon > 0$ : Pour $k$ suffisament grand,
             $u_k \le \frac \varepsilon k$,
             donc $u_k \ge \frac 1n  \Rightarrow  k\le n\varepsilon$.
             Alors $\sum_{u_k \ge 1/n} \frac 1{u_k} \le n^2\varepsilon + Kn$.}
\end{enumerate}
}

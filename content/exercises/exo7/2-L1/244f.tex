\uuid{244f}
\exo7id{5405}
\titre{exo7 5405}
\auteur{rouget}
\organisation{exo7}
\datecreate{2010-07-06}
\isIndication{false}
\isCorrection{true}
\chapitre{Continuité, limite et étude de fonctions réelles}
\sousChapitre{Autre}
\module{Analyse}
\niveau{L1}
\difficulte{}

\contenu{
\texte{
Trouver les fonctions bijectives de $[0,1]$ sur lui-même vérifiant $\forall x\in[0,1],\;f(2x-f(x))=x$.
}
\reponse{
$Id$ est solution.

Réciproquement, soit $f$ une bijection de $[0,1]$ sur lui-même vérifiant $\forall x\in[0,1],\;f(2x-f(x))=x$. Nécessairement, $\forall x\in[0,1],\;0\leq 2x-f(x)\leq 1$ et donc $\forall x\in[0,1],\;2x-1\leq f(x)\leq 2x$.

Soit $f^{-1}$ la réciproque de $f$.

\begin{align*}\ensuremath
\forall x\in[0,1],\;f(2x-f(x))=x&\Leftrightarrow\forall x\in[0,1],\;2x-f(x)=f^{-1}(x)\\
 &\Leftrightarrow\forall y\in[0,1],\;f(f(y))-2f(y)+y=0\;(\mbox{car}\;\forall x\in[0,1],\;\exists!y\;[0,1]/\;x=f(y))
\end{align*}

Soit $y\in[0,1]$ et $u_0=y$. En posant $\forall n\in\Nn,\;u_{n+1}=f(u_n)$, on définit une suite de réels de $[0,1]$ (car $[0,1]$ est stable par $f$). La condition $\forall y\in[0,1],\;f(f(y))-2f(y)+ y=0$ fournit $\forall n\in\Nn, \;u_{n+2}-2u_{n+1}+u_n=0$, ou encore $\forall n\in\Nn,\;u_{n+2}-u_{n+1}=u_{n+1}-u_n$. La suite $(u_{n+1}-u_n)_{n\in\Nn}$ est constante ou encore $u$ est arithmétique. Mais, $u$ est également bornée et donc $u$ est constante.

En particulier, $u_1=u_0$ ce qui fournit $f(y)=y$. On a montré que $\forall y\in[0,1],\;f(y)=y$ et donc $f=Id$.
}
}

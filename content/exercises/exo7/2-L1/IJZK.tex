\uuid{IJZK}
\exo7id{5415}
\titre{exo7 5415}
\auteur{rouget}
\organisation{exo7}
\datecreate{2010-07-06}
\isIndication{false}
\isCorrection{true}
\chapitre{Dérivabilité des fonctions réelles}
\sousChapitre{Théorème de Rolle et accroissements finis}
\module{Analyse}
\niveau{L1}
\difficulte{}

\contenu{
\texte{
Montrer que pour tout réel strictement positif $x$, on a~:$\left(1+\frac{1}{x}\right)^x<e<\left(1+\frac{1}{x}\right)^{x+1}$.
}
\reponse{
Montrons que ($\forall x>0,\;\left(1+\frac{1}{x}\right)^x<e<\left(1+\frac{1}{x}\right)^{x+1}$. Soit $x>0$.

\begin{align*}\ensuremath
\left(1+\frac{1}{x}\right)^x<e<\left(1+\frac{1}{x}\right)^{x+1}
&\Leftrightarrow x\ln(1+\frac{1}{x})<1<(x+1)\ln(1+\frac{1}{x})\\
 &\Leftrightarrow x(\ln(x+1)-\ln x)<1<(x+1)(\ln(x+1)-\ln x)\Leftrightarrow\frac{1}{x+1}<\ln(x+1)-\ln x<\frac{1}{x}.
\end{align*} 

Soit $x$ un réel strictement positif fixé. Pour $t\in[x,x+1]$, posons $f(t)=\ln t$. $f$ est continue sur $[x,x+1]$ et dérivable sur $]x,x+1[$. Donc, d'après le théorème des accroissements finis, il existe un réel $c$ dans $]x,x+1[$ tel que $f(x+1)-f(x)=(x+1-x)f'(c)$ ou encore

$$\exists c\in]x,x+1[/\;\ln(x+1)-\ln x=\frac{1}{c},$$

ce qui montre que $\forall x>0,\;\frac{1}{x+1}<\ln(x+1)-\ln x<\frac{1}{x}$, et donc que 

$$\forall x>0,\;\left(1+\frac{1}{x}\right)^x<e<\left(1+\frac{1}{x}\right)^{x+1}.$$
}
}

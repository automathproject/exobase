\uuid{IikC}
\exo7id{4255}
\titre{exo7 4255}
\auteur{quercia}
\organisation{exo7}
\datecreate{2010-03-12}
\isIndication{false}
\isCorrection{false}
\chapitre{Calcul d'intégrales}
\sousChapitre{Autre}
\module{Analyse}
\niveau{L1}
\difficulte{}

\contenu{
\texte{
On note $I_n =  \int_{t=0}^{\pi/2} \cos^nt\,d t$.
}
\begin{enumerate}
    \item \question{Comparer $I_n$ et $ \int_{t=0}^{\pi/2} \sin^nt\,d t$.}
    \item \question{En coupant $\left[0,\frac\pi2\right]$ en $[0,\alpha]$ et
      $\left[\alpha,\frac\pi2\right]$, démontrer que $I_n \to 0$ pour $n\to\infty$.}
    \item \question{Chercher une relation de récurrence entre $I_n$ et $I_{n+2}$.
      En déduire $I_{2k}$ et $I_{2k+1}$ en fonction de $k$.}
    \item \question{Démontrer que $nI_nI_{n-1} = \frac \pi2$.}
    \item \question{Démontrer que $I_n \sim I_{n-1}$ et en déduire un équivalent simple de $I_n$
      puis de $C_{2n}^n$ pour $n \to \infty$.}
\end{enumerate}
}

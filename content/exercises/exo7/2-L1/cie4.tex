\uuid{cie4}
\exo7id{553}
\titre{exo7 553}
\auteur{cousquer}
\organisation{exo7}
\datecreate{2003-10-01}
\isIndication{false}
\isCorrection{false}
\chapitre{Suite}
\sousChapitre{Suite définie par une relation de récurrence}
\module{Analyse}
\niveau{L1}
\difficulte{}

\contenu{
\texte{
Soit $f\colon [0,1]\rightarrow [0,1]$. On considère $a\in [0,1]$ 
et la suite $(u_n)_{n\in \mathbb{N}}$ vérifiant $u_0=a$ et $\forall n\in 
\mathbb{N},\; u_{n+1}=f(u_n)$. Les propriétés suivantes sont-elles 
vraies ou fausses~:
}
\begin{enumerate}
    \item \question{Si $f$ est croissante, alors $(u_n)$ est croissante.}
    \item \question{Si $(u_n)$ est croissante, alors $f$ est 
croissante.}
    \item \question{Si $(u_n)$ est croissante et $f$ monotone, alors $f$ est 
croissante.}
    \item \question{Si $(u_n)$ converge vers une limite~$l$, alors $l$ est point fixe 
de~$f$.}
    \item \question{Si $f$ est dérivable, alors $(u_n)$ est bornée.}
    \item \question{Si le graphe de~$f$ est au dessus de la droite d'équation $y=x$, 
alors $(u_n)$ est croissante.}
    \item \question{Si $(u_n)$ converge vers un point fixe~$l$ de~$f$, alors $f$ est 
continue en~$l$.}
\end{enumerate}
}

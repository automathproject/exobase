\uuid{b7QN}
\exo7id{4424}
\titre{exo7 4424}
\auteur{quercia}
\organisation{exo7}
\datecreate{2010-03-14}
\isIndication{false}
\isCorrection{true}
\chapitre{Série numérique}
\sousChapitre{Autre}
\module{Analyse}
\niveau{L1}
\difficulte{}

\contenu{
\texte{
Soit $(u_n)$ une suite réelle positive et $v_n = \frac1{1+n^2u_n}$.
Montrer que $\sum u_n$ converge $ \Rightarrow  \sum v_n$ diverge.
\'Etudier le cas où $\sum u_n$ diverge.
}
\reponse{
Si $\sum u_n$ et $\sum v_n$ convergent alors $n^2u_n\to \infty$ (lorsque $n\to\infty$)
donc $u_nv_n \sim 1/n^2$. Alors les suites $(\sqrt{u_n})$ et $(\sqrt{v_n})$
sont de carrés sommables tandis que la suite $(\sqrt{u_nv_n})$ n'est pas sommable,
c'est absurde.

Si $\sum u_n$ diverge on ne peut rien dire~: avec $u_n=1$ on a $\sum v_n$
convergente tandis qu'avec $u_n=\frac1{n}$ on a $\sum v_n$ divergente.
}
}

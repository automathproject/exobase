\uuid{sI9A}
\exo7id{688}
\titre{exo7 688}
\auteur{bodin}
\organisation{exo7}
\datecreate{1998-09-01}
\isIndication{false}
\isCorrection{false}
\chapitre{Continuité, limite et étude de fonctions réelles}
\sousChapitre{Etude de fonctions}
\module{Analyse}
\niveau{L1}
\difficulte{}

\contenu{
\texte{
Soient $n \in \N^*$ et $d \in \R^+$. D\'emontrer en utilisant
le th\'eor\`eme des valeurs interm\'ediaires que le polyn\^ome
$P(X)=X^n-d$ a au moins une racine dans $\R$.
}
}

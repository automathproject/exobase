\uuid{fE1J}
\exo7id{507}
\titre{exo7 507}
\auteur{bodin}
\organisation{exo7}
\datecreate{1998-09-01}
\video{HI2i2rdz3_A}
\isIndication{true}
\isCorrection{true}
\chapitre{Propriétés de R}
\sousChapitre{Les rationnels}
\module{Analyse}
\niveau{L1}
\difficulte{}

\contenu{
\texte{
Montrer que la suite $(u_n)_{n\in\Nn}$ d\'efinie par
$$u_n = (-1)^n+\frac{1}{n}$$
n'est pas convergente.
}
\indication{On prendra garde \`a ne pas parler de limite d'une suite sans savoir au pr\'ealable qu'elle converge !

Vous pouvez utiliser le r\'esultat du cours suivant :
Soit $(u_n)$ une suite convergeant  vers la limite $\ell$ alors toute sous-suite $(v_n)$ de $(u_n)$ a pour limite $\ell$.}
\reponse{
Il est facile de se convaincre que $(u_n)$ n'a pas de
 limite, mais plus délicat d'en donner une d\'emonstration
 formelle. En effet, d\`es lors qu'on ne sait pas qu'une suite $(u_n)$
converge, on ne peut pas \'ecrire $\lim u_n$, c'est un nombre qui
n'est pas d\'efini. Par exemple l'\'egalit\'e $$\lim_{n \rightarrow
\infty}\: (-1)^n+1/n=\lim_{n\rightarrow \infty} (-1)^n$$ n'a pas de
sens. Par contre voil\`a ce qu'on peut dire : \emph{Comme la
suite $1/n$ tend vers $0$ quand $n \rightarrow \infty$, la suite $u_n$ est
convergente si et seulement si la suite $(-1)^n$ l'est. De plus,
dans le cas o\`u elles sont toutes les deux convergentes, elles
ont m\^eme limite.} Cette affirmation provient tout simplement du
th\'eor\`eme suivant

\textbf{Th\'eor\`eme} : Soient $(u_n)$ et $(v_n)$ deux suites
convergeant vers deux limites $\ell$ et $\ell'$. Alors la suite $(w_n)$ définie par
$w_n=u_n+v_n$ est convergente (on peut donc parler de sa limite)
et $\lim w_n=\ell+\ell'$.

De plus, il n'est pas vrai que toute suite convergente
doit forc\'ement \^etre croissante et major\'ee ou d\'ecroissante
et minor\'ee. Par exemple, $(-1)^n/n$ est une suite qui converge
vers $0$ mais qui n'est ni croissante, ni d\'ecroissante. 

\bigskip

Voici maintenant un exemple de r\'edaction de l'exercice.  On veut
montrer que la suite $(u_n)$ n'est pas convergente. Supposons donc
par l'absurde qu'elle soit convergente et notons
$\ell=\lim_{n\rightarrow\infty} u_n$. (Cette expression a un sens puisqu'on
suppose que $u_n$ converge).

{\bf  Rappel.} Une {\em sous-suite} de $(u_n)$ (on dit aussi  {\em
suite extraite} de $(u_n)$) est une suite $(v_n)$ de la forme
$v_n=u_{\phi(n)}$ o\`u $\phi$ est une application strictement
croissante de $\N$ dans $\N$. Cette fonction $\phi$ correspond
``au choix des indices qu'on veut garder'' dans notre sous-suite.
Par exemple, si on ne veut garder dans la suite $(u_n)$ que les
termes pour lesquels $n$ est un multiple de trois, on pourra poser
$\phi(n)=3n$, c'est \`a dire $v_n=u_{3n}$.

\vspace{0.3cm}

Consid\'erons maintenant les sous-suites $v_n=u_{2n}$ et
$w_n=u_{2n+1}$ de $(u_n)$. On a que $v_n=1+1/2n\rightarrow1$ et que
$w_n=-1+1/(2n+1)\rightarrow -1$. Or on a le th\'eor\`eme suivant sur les
sous-suites d'une suite convergente:

\textbf{Th\'eor\`eme} : Soit $(u_n)$ une suite convergeant  vers la
limite $\ell$ (le th\'eor\`eme est encore vrai si $\ell=+\infty$ ou
$\ell=-\infty$). Alors, toute sous-suite $(v_n)$ de $(u_n)$ a pour limite
$\ell$.


Par cons\'equent, ici, on a que $\lim v_n=\ell$ et $\lim w_n=\ell$  donc
$\ell=1$ et $\ell=-1$ ce qui est une contradiction. L'hypoth\`ese disant
que $(u_n)$ \'etait convergente est donc fausse. Donc $(u_n)$ ne
converge pas.
}
}

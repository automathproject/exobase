\uuid{UFz1}
\exo7id{6325}
\titre{exo7 6325}
\auteur{queffelec}
\organisation{exo7}
\datecreate{2011-10-16}
\isIndication{false}
\isCorrection{false}
\chapitre{Solution maximale}
\sousChapitre{Solution maximale}
\module{Equation différentielle}
\niveau{L3}
\difficulte{}

\contenu{
\texte{
Soit l'équation différentielle
\begin{equation} \label{eq 10}
x''' - x x'' =0 \; \; .
\end{equation}
où $x$ est une application trois fois dérivable, définie sur
un intervalle ouvert de $\Rr$ et à valeurs dans $\Rr$.
}
\begin{enumerate}
    \item \question{Mettre cette équation différentielle sous la forme
  canonique
  $$ y'(t) = f(t,y(t)) \; , $$
  où $f$ est une application que l'on déterminera.}
    \item \question{Soient $t_0, a,b,c\in \Rr$. Montrer qu'il existe une
  unique solution maximale $\varphi$ de l'équation (\ref{eq 10}) qui
  satisfasse aux conditions initiales
  $$ \varphi (t_0) =a \; , \; \varphi ' (t_0)= b \;\; et \;\; \varphi '' (t_0) =c\;
  . $$}
    \item \question{Soit $\varphi $ une telle solution maximale. Calculer la
  dérivée de la fonction
  $$t\mapsto \varphi ''(t) \exp\left( -\int _a ^t \varphi (u) \, du
  \right)\; .$$
  En déduire que la fonction $\varphi $ est soit convexe, soit
  concave sur son intervalle de définition. Déterminer $\varphi $
  dans le cas où $\varphi ''(a)=0$.}
\end{enumerate}
}

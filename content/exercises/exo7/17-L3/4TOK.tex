\uuid{4TOK}
\exo7id{2229}
\titre{exo7 2229}
\auteur{matos}
\organisation{exo7}
\datecreate{2008-04-23}
\isIndication{false}
\isCorrection{true}
\chapitre{Autre}
\sousChapitre{Autre}
\module{Analyse numérique}
\niveau{L3}
\difficulte{}

\contenu{
\texte{

}
\begin{enumerate}
    \item \question{Soit $v$ un vecteur r\'eel v\'erifiant $v^Tv=1$. Montrer que la matrice de Householder
$$H(v)=I-2vv^T$$
repr\'esente une sym\'etrie par rapport au sous--espace vectoriel form\'e par les vecteurs orthogonaux aux vecgteurs $v$. En d\'eduire que $\det (H(v))=-1$.}
\reponse{Soit $P$ l'op\'erateur de projection dans le sous-espace $U$ de dimension 1 g\'en\'er\'e par $v$. Alors $Q=I-P$ est l'op\'erateur de projection sur l'hyperplan $U^{\bot}$ orthogonal \`a $U$. On a d\'ej\`a vu que $Pw=vv^Tw\quad \forall w$, et donc $Qw=w-vv^Tw$. On obtient

$P(H(v)w) =P(w_(2v^Tw)v)=(v^Tw)v-2v^Twvv^Tv=-(v^Tw)v=-Pw$


$Q(H(v)w)= H(v)w -P(H(v)w)=w-2vv^Tw+v^Twv=w-v^Twv=Qw$.

La matrice $H(v)$ repr\'esente donc une sym\'etrie par rapport \`a l'hyperplan $U^{\bot}$. On conclut que les vecteurs de $U^{\bot}$ sont invariants par $H(v)$.

$V(v)w=w\quad \forall w\in U^{\bot}, \quad \mbox{dim} U^{\bot}=n-1\Rightarrow \lambda =1$ est valeur propre de $H(v)$ avec multiplicit\'e $n-1$.

$H(v)v=-v\mp \lambda =-1$ est valeur propre de multiplicit\'e 1. Donc
 $$\det H(v)=\prod_{i=1}^n \lambda_i(H(v))=-1$$}
    \item \question{D\'emontrer que toute matrice orthogonale est le produit de au plus $n$ matrices de Householder. En d\'eduire une interpr\'etation g\'eom\'etrique des matrices orthogonales.}
\reponse{On sait qu'il exite des matrices de Householder $H_1, H_1, \ldots ,H_{n-1}$ telles que $H_{n-1}\cdots H_1A=A_{n}$ matrice triangulaire sup\'erieure. Comme $A$ est orthogonale on conclut que $A_n$ est orthogonale. Mais une matrice triangulaire sup\'erieure orthogonale est forc\'ement diagonale $\Rightarrow A_n=$diag$(\pm 1)$. On peut s'arranger pour que  $(A_n)_{ii} >0 \quad i=1,\ldots, n-1.$ Donc soit $A_n=I$ soit $A_n=$diag$(1,1,\ldots , 1,-1)=H(e_n)$ et finalement la matrice orthogonale $A$ s'\'ecrit
$$A=H_1 \cdots H_{n-1}H(e_n)$$}
\end{enumerate}
}

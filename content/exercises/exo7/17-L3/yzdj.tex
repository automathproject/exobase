\uuid{yzdj}
\exo7id{2213}
\titre{exo7 2213}
\auteur{matos}
\organisation{exo7}
\datecreate{2008-04-23}
\isIndication{false}
\isCorrection{true}
\chapitre{Autre}
\sousChapitre{Autre}
\module{Analyse numérique}
\niveau{L3}
\difficulte{}

\contenu{
\texte{
Montrer que si $\rho (A) <1$ alors
\begin{itemize}
\item $I-A$ est r\'eguli\`ere;
\item $(I-A)^{-1}=\lim_{k \rightarrow \infty} C_k$ avec $C_k=I+A+\cdots +A^k$.
\end{itemize}
}
\reponse{
$\rho (A)<1 \Rightarrow 1$ n'est pas valeur propre de $A \Rightarrow 0$ n'est pas valeur propre de $I-A\Rightarrow I-A$ inversible
$$(I-A)C_k= (I-A)(I+A+\cdots +A^k)=I-A^{k+1}$$
$C_k=(I-A)^{-1} (I-A^{k+1}) \Rightarrow (I-A)^{-1} -C_k =(I-A)^{-1} A^{k+1}$ et conc
$$\|(I-A)^{-1} -C_k\|\leq \|(I-A)^{-1}\|\|A^{k+1}\|\leq \|(I-A)^{-1}\|\|A\|^{k+1}$$
Comme $\|A\|<1$ pour au moins une norme subordonn\'ee on obtient finalement
$$\lim_{k\rightarrow \infty}\|(I-A)^{-1}-C_k\|=0$$
}
}

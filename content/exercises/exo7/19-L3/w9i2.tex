\uuid{w9i2}
\exo7id{6941}
\titre{exo7 6941}
\auteur{ruette}
\organisation{exo7}
\datecreate{2013-01-24}
\isIndication{false}
\isCorrection{true}
\chapitre{Loi, indépendance, loi conditionnelle}
\sousChapitre{Loi, indépendance, loi conditionnelle}
\module{Probabilité et statistique}
\niveau{L3}
\difficulte{}

\contenu{
\texte{
Soit $X$ une variable aléatoire réelle positive intégrable. Montrer que
$\displaystyle E(X)=\int_0^{+\infty}(1-F(t))\,dt$.
}
\reponse{
$\displaystyle 1-F(t)=P(X>t)=P_X(]t,+\infty[)=\int_{]t,+\infty[}1 \,dP_X(x)$.
Donc
$$ \int_0^{+\infty}(1-F(t))\,dt=\int_0^{+\infty}\left(\int_{]t,+\infty[}1 \,dP_X(x)\right)dt.$$  
On applique Fubini-Tonelli, le domaine d'intégration
étant $\{(t,x)\in\Rr^2 \mid 0\leq t<x\}$ :
$$ \int_0^{+\infty}(1-F(t))\,dt=\int_{]0,+\infty[}\left(\int_{[0,x[}
1\,dt\right)dP_X(x)=\int_{]0,+\infty[} xdP_X(x).$$
Par ailleurs, comme $X$ est positive, 
$$E(X)=\displaystyle\int_0^{+\infty}x\,dP_X(x),
\quad\mbox{donc}\quad E(X)=0 \times P_X(\{0\})+\int_{]0,+\infty[}x\,dP_X(x).$$ 
On en déduit
que $\displaystyle E(X)=\int_0^{+\infty}(1-F(t))\,dt$.

\medskip
\textit{Remarque : si la loi de $X$ est continue, on n'a pas à se préoccuper
si les intervalles sont ouverts ou fermés puisque $P(X>a)=P(X\geq a)$.}
}
}

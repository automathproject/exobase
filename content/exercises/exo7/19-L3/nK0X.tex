\uuid{nK0X}
\exo7id{6950}
\titre{exo7 6950}
\auteur{ruette}
\organisation{exo7}
\datecreate{2013-01-24}
\isIndication{false}
\isCorrection{true}
\chapitre{Convergence de variables aléatoires}
\sousChapitre{Convergence de variables aléatoires}
\module{Probabilité et statistique}
\niveau{L3}
\difficulte{}

\contenu{
\texte{
Soit $(X_n)_{n\geq 1}$ une suite de variables aléatoires indépendantes de même
loi exponentielle $\mathcal{E}(1)$.
%de densité $\I1_{\Rr_+}(x)e^{-x}$.
}
\begin{enumerate}
    \item \question{En utilisant le lemme de Borel-Cantelli, montrer que
$$
P(X_n>\alpha \ln n\mbox{ pour une infinité de }n)=\left\{
\begin{array}[c]{ll}0 \mbox{ si } \alpha>1\\ 1\mbox{ si }\alpha\leq 1
\end{array}\right.
$$}
\reponse{On pose $A_n(\alpha)=\{X_n>\alpha \ln n\}$
et $A(\alpha)=\limsup A_n(\alpha)$.
Les $(A_n(\alpha))_{n\geq 1}$ sont des événements indépendants. 
Si $\alpha\geq 0$ alors $\alpha\ln n\geq 0$ et 
$P(A_n(\alpha))=\displaystyle\int_{\alpha\ln n}^{+\infty}e^{-x}\,dx=e^{-\alpha\ln n}
=\frac{1}{n^{\alpha}}$, donc $\displaystyle\sum P(A_n(\alpha))=
\sum\frac{1}{n^{\alpha}}$. C'est une série de Riemann. 
Si $\alpha>1$ la série converge et le lemme de Borel-Cantelli implique que
$P(A(\alpha))=0$.
Si $0\leq \alpha\leq 1$ la série diverge et le lemme de Borel-Cantelli
implique que $P(A(\alpha))=1$. Si $\alpha<0$ alors $P(A_n(\alpha))=1$ pour tout $n$
parce que $X_n$ est positive presque sûrement, donc on a $P(A(\alpha))=1$. Ce qui
répond à la question 1.}
    \item \question{En déduire que $\displaystyle\limsup_{n\to+\infty} \frac{X_n}{\ln n}=1$ 
presque sûrement.}
\reponse{Pour tout $\epsilon>0$, on
a $P(A(1-\epsilon))=1$. Or
$A(1-\epsilon)=\left\{\frac{X_n}{\ln n}>1-\epsilon \mbox{ pour une 
infinité de }n\right\}$, donc si $\omega$ appartient à $A(1-\epsilon)$ alors
$\displaystyle\limsup_{n\to+\infty}\frac{X_n(\omega)}{n}\geq 1-\epsilon$. Soit $N_k$
le complémentaire de
$A(1-1/k)$ et $N=\bigcup_{k\geq 1} N_k$.
On a $P(N_k)=0$ donc $P(N)\leq \sum_{k\geq 1}P(N_k)=0$
(somme dénombrable d'ensembles de mesures nulles). Si
$\omega\not \in N$ alors $\forall k\geq 1$, $\omega\not \in N_k$
donc $\displaystyle\limsup_{n\to+\infty}\frac{X_n(\omega)}{n}\geq 1-1/k$, et
comme $1/k$ tend vers $0$ alors $\displaystyle\limsup_{n\to+\infty}\frac{X_n(\omega)}{n}\geq 1$ $\forall \omega\not\in N$.

\medskip
$\forall\epsilon>0$, on a $P(A(1+\epsilon))=0$ donc $P((A(1+\epsilon))^c)=1$.
Or 
$$
(A(1+\epsilon))^c=\liminf (A_n(1+\epsilon))^c=\left\{\frac{X_n}{n}\leq \alpha 
\mbox{ à partir d'un certain rang}\right\},$$ 
donc si $\omega\in (A(1+\epsilon))^c$ 
alors $\displaystyle \limsup_{n\to +\infty}\frac{X_n}{\ln n}\leq 1+\epsilon$.
Soit $N'=\displaystyle\bigcup_{k\geq 1} A(1+1/k)$.
On a $P(A(1+1/k))=0$ pour tout $k\geq 1$ donc $P(N')=0$. Si
$\omega\not \in N'$ alors pour tout $k\geq 1$, $\omega\in (A(1+1/k))^c$
donc 
$$\limsup_{n\to+\infty}\frac{X_n(\omega)}{n}\leq 1+1/k,$$ 
et comme $1/k$ tend vers $0$ alors $\displaystyle\limsup_{n\to+\infty}\frac{X_n(\omega)}{n}\leq 1$.


Conclusion : $\forall \omega\not\in N\cup N'$ alors $\displaystyle \limsup_{n\to+\infty}
\frac{X_n(\omega)}{\ln n}=1$. Autrement dit
$\displaystyle\limsup_{n\to+\infty}\frac{X_n}{\ln n}=1$ presque sûrement, puisque
$P(N\cup N')\leq P(N)+P(N')=0$.}
\end{enumerate}
}

\uuid{iY9a}
\exo7id{5943}
\titre{exo7 5943}
\auteur{tumpach}
\organisation{exo7}
\datecreate{2010-11-11}
\isIndication{false}
\isCorrection{true}
\chapitre{Lemme de Fatou, convergence monotone}
\sousChapitre{Lemme de Fatou, convergence monotone}
\module{Théorie de la mesure, intégrale de Lebesgue}
\niveau{L3}
\difficulte{}

\contenu{
\texte{
Soit $\Omega = \mathbb{R}$, $\Sigma = \mathcal{B}(\mathbb{R})$ et
$\mu$ la mesure de Lebesgue sur $\mathbb{R}$. Soit $f_{n} =
-\frac{1}{n} \mathbf{1}_{[0, n]}$, $n\in\mathbb{N}$, et $f = 0$. Montrer
que $f_{n}$ converge uniform\'ement vers $f$ sur $\mathbb{R}$ mais
que
$$\lim \inf_{n\rightarrow+\infty} \int_{\Omega} f_{n}\,d\mu  ~<~
\int_{\Omega} f\,d\mu.
$$
Est-ce que cela contredit le lemme de Fatou ?
}
\reponse{
En effet, pour tout $\varepsilon>0$, il existe $N_{\varepsilon} =
\left[\frac{1}{\varepsilon} \right] + 1$ tel que $\forall n \geq
N_{\varepsilon}$, $$ \sup_{x\in\mathbb{R}} |f_{n}(x) - f(x)| <
\varepsilon,
$$
i.e. $f_{n}$ converge uniform\'ement vers $f$ sur $\mathbb{R}$. On
a~:
$$
\lim \inf_{n\rightarrow +\infty} \int_{\Omega} f_{n}\,d\mu = \lim
\inf_{n\rightarrow +\infty} - \int_{0}^{n} \frac{1}{n}\,d\mu = -1.
$$
D'autre part $\int_{\Omega} f\,d\mu = 0.$ Le lemme de Fatou ne
s'applique pas car les fonctions $f_{n}$ ne sont pas \`a valeurs
dans $[0, +\infty]$.
}
}

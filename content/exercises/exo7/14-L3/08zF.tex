\uuid{08zF}
\exo7id{5935}
\titre{exo7 5935}
\auteur{tumpach}
\organisation{exo7}
\datecreate{2010-11-11}
\isIndication{false}
\isCorrection{true}
\chapitre{Tribu, fonction mesurable}
\sousChapitre{Tribu, fonction mesurable}
\module{Théorie de la mesure, intégrale de Lebesgue}
\niveau{L3}
\difficulte{}

\contenu{
\texte{
Soit $\Omega = \mathbb{N}$, $\Sigma = \mathcal{P}(\mathbb{N})$ et
$\mu$ la mesure de comptage sur $\mathbb{N}$ d\'efinie par~:
$$
\mu(E) = \sharp E = \sum_{k \in E} 1,
$$
o\`u $E\in \Sigma$. Soit $f~:\mathbb{N} \rightarrow \mathbb{R}$
une fonction positive ou nulle. Montrer que $f$ est
($\Sigma$-$\mathcal{B}(\mathbb{R})$)-mesurable et que~:
$$
\int_{\Omega} f d\mu = \sum_{n=1}^{\infty} f(n).
$$
}
\reponse{
Soit $\Omega = \mathbb{N}$, $\Sigma = \mathcal{P}(\mathbb{N})$ et
$\mu$ la mesure de comptage sur $\mathbb{N}$ d\'efinie par~:
$$
\mu(E) = \sharp E = \sum_{k \in E} 1,
$$
o\`u $E\in \Sigma$. Soit $f~:\mathbb{N} \rightarrow \mathbb{R}$
une fonction positive ou nulle. Pour tout bor\'elien $E$,
$f^{-1}(E)$ appartient \`a $\mathcal{P}(\mathbb{N})$, donc $f$ est
($\Sigma$-$\mathcal{B}(\mathbb{R})$)-mesurable. Par d\'efinition
de l'int\'egrale,
$$
\int_{\Omega} f d\mu = \int_{0}^{\infty}
\mu\left(S_{f}(t)\right)\,dt,
$$
o\`u $S_{f}(t) = \{n\in\Sigma, \,f(n)>t\}$. Pour tout
$y\in[0,+\infty[$, posons $A_{y} := \{n\in\mathbb{N},\, f(n) =
y\}$. Alors $$S_{f}(t) = \cup_{y
> t}A_{y}$$ o\`u l'union est disjointe et o\`u $A_y$ est vide sauf
pour un ensemble d\'enombrable $\{y_{i}\}_{i\in\mathbb{N}}$ de
valeurs de $y$. Par $\sigma$-additivit\'e de la mesure $\mu$,
$$
\mu\left(S_{f}(t)\right) = \mu\left(\cup_{y_i > t} A_{y_{i}}
\right) = \sum_{y_i
> t} \mu\left(A_{y_{i}}\right) = \sum_{y_i > t}\mu\left(\{f = y_i
\}\right).
$$
Ainsi~: 
\begin{eqnarray*} \int_{\Omega} f d\mu & =&
\int_{0}^{\infty}\sum_{y_i
> t} \mu\left(\{f = y_i \}\right)\,dt = \sum_{i=0}^{\infty}
\int_{0\leq t<y_{i}}\mu\left(\{f = y_i \}\right)\,dt \\&=&
\sum_{i=0}^{\infty} y_i \cdot \mu\left(\{f = y_i \}\right) =
\sum_{i=0}^{\infty} y_{i}\cdot\sharp\{n\in\mathbb{N}, f(n) =
y_{i}\} = \sum_{n=0}^{\infty} f(n).
\end{eqnarray*}
}
}

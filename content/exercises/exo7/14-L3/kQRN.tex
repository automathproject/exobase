\uuid{kQRN}
\exo7id{5973}
\titre{exo7 5973}
\auteur{tumpach}
\organisation{exo7}
\datecreate{2010-11-11}
\isIndication{false}
\isCorrection{true}
\chapitre{Autre}
\sousChapitre{Autre}
\module{Théorie de la mesure, intégrale de Lebesgue}
\niveau{L3}
\difficulte{}

\contenu{
\texte{
Soit $\Omega'$ l'ouvert de $\mathbb{R}^{n}$ d\'efini par
$$
\Omega' = \{(r, \theta_1, \dots, \theta_{n-1})\in\mathbb{R}^{n}~|~
0< r, 0 <\theta_1, \dots, \theta_{n-2} < \pi, 0 < \theta_{n-1} <
2\pi\}.
$$
Soit l'application $S$ de $\Omega'$ dans $\mathbb{R}^{n}$
d\'efinie par
$$
\begin{array}{lcl}
x_1 &= &r \cos \theta_1,\\
x_2 &=& r \sin \theta_1 \cos \theta_2,\\
\vdots & & \\
x_{n-2} &=&  r \sin \theta_1 \sin \theta_2 \dots \cos \theta_{n-2}\\
x_{n-1} &=& r \sin \theta_1 \sin\theta_2 \dots \sin \theta_{n-2}
\cos \theta_{n-1}\\
x_{n} &=& r \sin \theta_1 \sin \theta_2 \dots \sin\theta_{n-2}
\sin \theta_{n-1},
\end{array}
$$
o\`u $(x_1, \dots, x_n)$ sont les coordonn\'ees cart\'esiennes de
$x\in\mathbb{R}^n$.
}
\begin{enumerate}
    \item \question{Soit $\Omega = \mathbb{R}^n\setminus\{
x\in\mathbb{R}^n~|~x_{n} = 0~\text{et}~x_{n-1} \geq 0~\}.$ Montrer
que $\Omega$ est une partie ouverte de $\mathbb{R}^n$ dont le
compl\'ementaire est de mesure nulle, et que $S$ est un
diff\'eomorphisme de $\Omega'$ sur $\Omega$.}
\reponse{Posons $\Omega = \mathbb{R}^n\setminus\{
x\in\mathbb{R}^n~|~x_{n} = 0~\text{et}~x_{n-1} \geq 0~\}.$ Comme
$0 < \theta_{n-1} < 2\pi$, l'image de $\Omega'$ par $S$ est
incluse dans $\Omega$. R\'eciproquement, soit $x$ un \'el\'ement
de $\Omega$. Posons $r = |x|$, alors pour tout $i \in\{1, \dots,
n-2\}$, on peut d\'efinir par r\'ecurrence $\theta_i\in(0, \pi)$
gr\^ace \`a son cosinus~:
$$
\cos\theta_{i} = \frac{x_{i}}{r
\sin\theta_1\dots\sin\theta_{i-1}}.
$$
Quant \`a $\theta_{n-1}$, il est d\'etermin\'e par son sinus et
son cosinus. Comme $x_{n} \neq 0$ ou $x_{n-1} < 0$,
n\'ecessairement $\theta_{n-1} \neq 0(\text{modulo}~2\pi)$.

L'application $S$ est contin\^ument diff\'erentiable, car chacune
de ses composantes l'est. La matrice jacobienne a ses vecteurs
colonnes orthogonaux, et de norme respectivement $1$, $r$,
$r\sin\theta_{1}$, $\dots$,
$r\sin\theta_{1}\dots\sin\theta_{n-2}$. Son d\'eterminant vaut
alors $r^{n-1}\left(\sin \theta_{1}\right)^{n-2}\dots
\sin\theta_{n-2}.$ Comme ce d\'eterminant ne s'annule jamais, $S$
est un diff\'eomorphisme de $\Omega'$ sur $\Omega$.}
    \item \question{Soit $f$
une fonction bor\'elienne positive sur $\mathbb{R}^{n}$. Montrer
que
$$
\begin{array}{lcl}
\int_{\mathbb{R}^n} f(x)\,dx &=& \int_{\Omega'} (f\circ S)(r,
\theta_1, \dots, \theta_{n-1})\,r^{n-1}
\sin^{n-2}\theta_{1}\,\sin^{n-3}\theta_2 \dots \sin
\theta_{n-2}\,dr\,d\theta_1 \dots d\theta_{n-1}\\ & & \\ & =&
\int_{\Omega'} (f\circ S)(r, \theta_1, \dots,
\theta_{n-1})\,r^{n-1}\,dr\,d\sigma,
\end{array}
$$
o\`u $d\sigma$ est la mesure uniforme sur la sph\`ere unit\'e de
$\mathbb{R}^{n}$.}
\reponse{C'est la formule du changement de variable.}
    \item \question{En utilisant les coordonn\'ees
sph\'eriques, calculer le volume $\mathcal{V}_4$ de la boule
unit\'e de $\mathbb{R}^4$ et l'aire $\mathcal{A}_{3}$ de la
sph\`ere unit\'e $\mathcal{S}^3$ de $\mathbb{R}^4$.}
\reponse{On a~:
$$
\begin{array}{lcl} \mathcal{V}_{4} &=& \int_{r = 0}^{1}\int_{\theta_1 =
0}^{\pi}\int_{\theta_{2} = 0}^{\pi} \int_{\theta_3 = 0}^{2\pi}
r^3\sin^{2}\theta_{1}\sin\theta_{2}\,dr
d\theta_1 d\theta_2 d\theta_3\\
& = &
2\pi\left[\frac{r^4}{4}\right]_{0}^{1}\left(\int_{0}^{\pi}\sin^{2}\theta_1
d\theta_1\right)\left(\int_{0}^{\pi}\sin\theta_2
d\theta_2\right)\\
& = & \frac{\pi}{2}\left(\int_{0}^{\pi}\frac{1 -\cos 2\theta_1}{2}
d\theta_1\right)\left[-\cos\theta_2\right]_{0}^{\pi}\\
& = & \frac{\pi^2}{2}.\\
 \mathcal{A}_{3} &=& \int_{\theta_1 =0}^{\pi}\int_{\theta_{2} = 0}^{\pi} \int_{\theta_3 = 0}^{2\pi}
\sin^{2}\theta_{1}\sin\theta_{2}\,
d\theta_1 d\theta_2 d\theta_3\\
& = & 2\pi^2.
\end{array}
$$}
\end{enumerate}
}

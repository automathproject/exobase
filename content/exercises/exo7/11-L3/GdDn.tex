\uuid{GdDn}
\exo7id{7710}
\titre{exo7 7710}
\auteur{mourougane}
\organisation{exo7}
\datecreate{2021-08-11}
\isIndication{false}
\isCorrection{true}
\chapitre{Sous-variété}
\sousChapitre{Sous-variété}
\module{Géométrie différentielle}
\niveau{L3}
\difficulte{}

\contenu{
\texte{
Soit $I$ un intervalle de $\Rr$ et $c~:I\to \Rr^2$ une courbe plane paramétrée par la longueur d'arc. 
On suppose que $c$ reste dans le disque de rayon $r>0$ et qu'au point de paramètre $\tau$,
$\| c(\tau)\| = r$.
}
\begin{enumerate}
    \item \question{Rappeler la valeur absolue de la courbure d'un cercle de rayon $r$.}
\reponse{La courbure d'un cercle de rayon $r$ est en valeur absolue $1/r$.}
    \item \question{Montrer en dérivant une fois la fonction $\phi : t\mapsto \| c(t)\|^2$ 
que $\ddot{c}(\tau)$ est colinéaire à $c(\tau)$.}
\reponse{Puisque la fonction norme (au carré) $\phi$ est différentiablement continue et maximale au point de paramètre $\tau$,
 sa dérivée en $\tau$ s'annule. On trouve $\phi'(\tau)=2<c(\tau),\dot{c}(\tau)>=0$.
 Par ailleurs, comme la fonction $c$ donne un paramétrage par la longueur d'arc,
 la fonction $t\mapsto \| \dot{c}(t)\|^2$ est constante égale à $1$,
 donc aussi de dérivée nulle en $\tau$. On trouve $2<\ddot{c}(\tau),\dot{c}(\tau)>=0$.
 Par conséquent, en notant $\vec{n}$ un vecteur normal unitaire à $\dot{c}(\tau)$,
 on trouve $c(\tau)=+/-\vec{n}$ et $\ddot{c}(\tau)=\kappa(\tau)\vec{n}$.}
    \item \question{Montrer en dérivant à nouveau la fonction $\phi$ 
que la courbure en $\tau$ vérifie $|\kappa(\tau)|\geq 1/r$.}
\reponse{Puisque la fonction norme $\phi$ est deux fois différentiablement continue et maximale au point de paramètre $\tau$,
 sa dérivée seconde en $\tau$ est négative. On trouve $\phi''(\tau)=2<c(\tau),\ddot{c}(\tau)>+2<\dot{c}(\tau),\dot{c}(\tau)>=+/-2r\kappa(\tau)+2\leq 0$.
 Donc, $-/+2r\kappa(\tau)\geq 2$ et par suite $-/+r\kappa(\tau)$ est positif de valeur absolue au moins $1$.
 En divisant par $r>0$, on trouve le résultat.}
    \item \question{Interpréter graphiquement ce résultat.}
\reponse{La courbe est en ses points extrémaux plus courbée que le cercle de rayon $r$ qui la borde.}
\end{enumerate}
}

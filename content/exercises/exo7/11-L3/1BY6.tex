\uuid{1BY6}
\exo7id{7711}
\titre{exo7 7711}
\auteur{mourougane}
\organisation{exo7}
\datecreate{2021-08-11}
\isIndication{false}
\isCorrection{true}
\chapitre{Sous-variété}
\sousChapitre{Sous-variété}
\module{Géométrie différentielle}
\niveau{L3}
\difficulte{}

\contenu{
\texte{

}
\begin{enumerate}
    \item \question{Déterminer une équation cartésienne du plan tangent 
à la surface $\mathcal{S}$ de $\Rr^3$ paramétrée par $F : (u,v)\mapsto (u+v^2,u^2-v^2,v)$
au point $M(u_0,v_0)$ de paramètres $(u_0,v_0)$.}
\reponse{Le plan $T_M\mathcal{S}$ tangent à $\mathcal{S}$ au point $M$ est engendré par les deux vecteurs
 $$\frac{\partial F}{\partial u}=\begin{pmatrix}1\\2u_0\\0\end{pmatrix} \quad \text{ et } \quad \frac{\partial F}{\partial v}=\begin{pmatrix}2v_0\\-2v_0\\1\end{pmatrix}.$$
 Une équation cartésienne du plan $T_M\mathcal{S}$ est obtenue par
 $$\begin{pmatrix}X\\Y\\Z\end{pmatrix}\in T_M\mathcal{S}\iff \begin{vmatrix}
                                1&2v_0&X\\2u_0&-2v_0&Y\\0&1&Z
                               \end{vmatrix}=0
 \iff -2u_0X+Y+2(1+2u_0)v_0Z=0.$$}
    \item \question{Déterminer une équation cartésienne du plan tangent 
à la surface $\Sigma$ de $\Rr^3$ d'équation $x^5+y^5+z^5=1$
au point $M(x_0,y_0,z_0)$ de coordonnées $(x_0,y_0,z_0)$.}
\reponse{Il suffit de dire que le plan tangent $T_M\Sigma$ est le noyau de la différentielle de la fonction 
 $\psi : (x,y,z)\mapsto x^5+y^5+z^5-1$ en $(x_0,y_0,z_0)$.
 On trouve $$\begin{pmatrix}X\\Y\\Z\end{pmatrix}\in T_M\Sigma\iff
 5(x_0^4X+y_0^4Y+z_0^4Z)=0\iff x_0^4X+y_0^4Y+z_0^4Z=0.$$}
    \item \question{Soit $U$ un ouvert de $\Rr^2$ et $f~:U\to \Rr$ une application de classe $\mathcal{C}^\infty$.
Déterminer une équation cartésienne du plan tangent au graphe de $f$ en chacun de ses points.}
\reponse{Une équation du graphe $\mathcal{G}$ de $f$ est $z=f(x,y)$.
 Une équation du plan tangent en $M$ de coordonnées $(x,y,f(x,y))$ est donc
 $$\begin{pmatrix}X\\Y\\Z\end{pmatrix}\in T_M\Sigma\iff
 Z=\frac{\partial f}{\partial x}(x,y)X+\frac{\partial f}{\partial y}(x,y)Y.$$}
\end{enumerate}
}

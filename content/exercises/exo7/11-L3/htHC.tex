\uuid{htHC}
\exo7id{6286}
\titre{exo7 6286}
\auteur{mayer}
\organisation{exo7}
\datecreate{2011-10-16}
\isIndication{false}
\isCorrection{false}
\chapitre{Sous-variété}
\sousChapitre{Sous-variété}
\module{Géométrie différentielle}
\niveau{L3}
\difficulte{}

\contenu{
\texte{
Soit $E$ un espace vectoriel de dimension finie, $a\in E$ et
$f:E\to E$ un difféomorphisme de classe $C^1$. On suppose que
$f^n =id$ et $f(a)=a$. On pose $A=Df(a)$ et $u(x) = \sum_{p=1}^n
A^{-p} f^p(x)$ pour $x\in E$.
}
\begin{enumerate}
    \item \question{Montrer que $u$ est un difféomorphisme local en $a$ tel
que $u\circ f = A\circ u$.}
    \item \question{Soit $F$ l'ensemble des points fixes de $f$. Montrer que $F$
est une sous-variété de $E$.}
    \item \question{Soit $g:\Rr^2\to \Rr^2$, $g(x,y) = (x,y+y^3-x^2)$.
Montrer que $g$ est un difféomorphisme de $\Rr^2$. En
déduire que $2)$ n'est plus nécessairement vrai si on supprime
l'hypothèse $f^n =id$.}
\end{enumerate}
}

\uuid{uYTw}
\exo7id{7704}
\titre{exo7 7704}
\auteur{mourougane}
\organisation{exo7}
\datecreate{2021-08-11}
\isIndication{false}
\isCorrection{false}
\chapitre{Sous-variété}
\sousChapitre{Sous-variété}
\module{Géométrie différentielle}
\niveau{L3}
\difficulte{}

\contenu{
\texte{
Soit $S$ une surface différentiable munie d'une métrique riemannienne $g$.
Soit $p$ et $q$ deux points fixés sur $S$.
Soit $\epsilon$ un nombre réel strictement positif. Soit $a\leq b$ deux nombres réels.
Soit $C :]-\epsilon,\epsilon [\times [a,b]\to S$ une application différentiable vérifiant 
pour tout $s\in ]-\epsilon,\epsilon [$,
$C(s,a)=p$ et $C(s,b)=q$.
On notera $c_s(t):=C(s,t)$, $V(t):=\frac{\partial C}{\partial s}_{(0, t)}$.
On admettra l'identité des dérivées covariantes
$$\nabla_{V(t)}\dot{c_0}(t)=\nabla_{\dot{c_0}(t)}V(t).$$
}
\begin{enumerate}
    \item \question{Représenter sur un dessin $S$, l'application $C$ et le champs de vecteurs $V$,
en particulier $V(a)$ et $V(b)$.}
    \item \question{Rappeler la définition de l'énergie $E[c_s]$ de la courbe $c_s : [a,b]\to S$.}
    \item \question{Exprimer à l'aide de la dérivée covariante $\nabla_{\dot{c_0}(t)}\dot{c_0}(t)$, 
la dérivée $\frac{dE[c_s]}{ds}_{s=0}$.}
\end{enumerate}
}

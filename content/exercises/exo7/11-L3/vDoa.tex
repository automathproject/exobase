\uuid{vDoa}
\exo7id{2552}
\titre{exo7 2552}
\auteur{tahani}
\organisation{exo7}
\datecreate{2009-04-01}
\isIndication{false}
\isCorrection{false}
\chapitre{Sous-variété}
\sousChapitre{Sous-variété}
\module{Géométrie différentielle}
\niveau{L3}
\difficulte{}

\contenu{
\texte{
Soit $E$ un espace vectoriel de dimension finie, $a \in E$ et $f:
E \rightarrow E$ un diff\'eomorphisme de classe $C^1$. On suppose
que $f^n=Id$ et $f(a)=a$. On pose $A=D_af$ et $u(x)=\sum_{p=1}^n
A^{-p}f^p(x)$ pour $x \in E$.
}
\begin{enumerate}
    \item \question{Montrer que $u$ est un diff\'eomorphisme local en $a$ tel
que $u \circ f=A \circ u$.}
    \item \question{Soit $F$ l'ensemble des points
fixes de $f$. Montrer que $F$ est une sous-vari\'et\'e de $E$.}
    \item \question{Soit $g: \mathbb{R}^2 \rightarrow \mathbb{R}^2$,
$g(x,y)=(x,y+y^3-x^2)$. Montrer que $g$ est un diff\'eomorphisme
de $\mathbb{R}^2$. En d\'eduire que $2/$ n'est plus
n\'ecessairement vrai si on supprime l'hypoth\`ese $f^n=Id$.}
\end{enumerate}
}

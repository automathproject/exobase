\uuid{uMYH}
\exo7id{6785}
\titre{exo7 6785}
\auteur{gijs}
\organisation{exo7}
\datecreate{2011-10-16}
\isIndication{false}
\isCorrection{false}
\chapitre{Forme différentielle}
\sousChapitre{Forme différentielle}
\module{Géométrie différentielle}
\niveau{L3}
\difficulte{}

\contenu{
\texte{
Dans tout ce qui suit, on note
$(x,y)$ les coordonnées sur $\Rr^2 = \Cc$, $z=x+
iy$, et on sépare systématiquement les parties
réelle et imaginaire de tous les objets. 
Soit  $z_o \in \Cc$, et soit  $C_r$, $r\in \Rr^+$  la
courbe  donnée par l'équation $|z-z_o|=r$.
}
\begin{enumerate}
    \item \question{Soit $f:\Cc\setminus\{z_o\} \to \Cc$ l'application
$f(z) = (z-z_o)^n$. Calculer pour tout $n\in \Zz$
l'intégrale $\int_{C_r} f(z)\,dz$ (indication~:
utiliser une variante des coordonnées polaires).


\medskip
Soit $g,h : \Cc\setminus\{z_o\} \to \Rr$
deux applications de classe $C^1$ verifiant les
équations~:
$$
\frac{\partial g}{\partial x} = 
\frac{\partial h}{\partial y}\qquad\&\qquad
\frac{\partial g}{\partial y} = 
-\frac{\partial h}{\partial x}\ ,
$$
et soit $f:\Cc\setminus\{z_o\} \to \Cc$
l'application $f(x,y) = g(x,y) + i h(x,y)$.}
    \item \question{Montrer que la 1-forme
$ \frac{f(z)}{z-z_o}\,dz$ sur $\Rr^2\setminus\{z_o\}$ est fermée (ne pas oublier de
séparer la partie réelle et imaginaire). En déduire
(Stokes!) que $ \int_{C_r}  \frac{f(z)}{z-z_o}\,dz$
est indépendant de $r\in \Rr^+$.}
    \item \question{En faisant un développement limité de $g$
et de $h$ d'ordre 1 autour $z_o = (x_o,y_o)$, montrer que 
$$
\int_{C_r} \frac{f(z)}{z-z_o}\,dz = 2\pi i\,\, f(z_o)\ .
$$
Indication : utiliser le résultat de 1.; vous avez le
droit d'être un petit  peu vague en ce qui concerne les
$\varepsilon$ dans le développement limité.}
\end{enumerate}
}

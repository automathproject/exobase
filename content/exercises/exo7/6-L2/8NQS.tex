\uuid{8NQS}
\exo7id{4980}
\titre{exo7 4980}
\auteur{quercia}
\organisation{exo7}
\datecreate{2010-03-17}
\isIndication{false}
\isCorrection{true}
\chapitre{Géométrie affine euclidienne}
\sousChapitre{Géométrie affine euclidienne de l'espace}
\module{Géométrie}
\niveau{L2}
\difficulte{}

\contenu{
\texte{
Déterminer {\it toutes\/} les isométries
}
\begin{enumerate}
    \item \question{d'un tétraèdre régulier.}
\reponse{$24$ éléments :
      \halign{ \hskip 5cm #\hfill&\quad(#)\cr
      identité &1\cr
      rotations autour de l'axe d'une face &8\cr
      symétries \% plan médiateur d'une arête &6\cr
      $\frac12$-tour autour de la perpendiculaire commune à deux arêtes opposées &3\cr
      $\frac14$-tour autour de la perp. \dots + symétrie \% plan médian &6\cr }}
    \item \question{d'un cube.}
\reponse{$48$ éléments :
      \halign{ \hskip 5cm #&\hfill#\hfill&\quad(#)\cr
      identité &&1\cr
      symétries &\% plan médian d'une face &3\cr
      &\% plan diagonal d'une face &6\cr
      &\% axe d'une face &3\cr
      &\% axe d'une arête &6\cr
      &\% centre du cube &1\cr
      rotation &$\pm 2\pi/3$ autour d'une diagonale &8\cr
      &$\pm \pi/2$ autour de l'axe d'une face &6\cr
      symétries-rotations &$\pm \pi/2$ \% axe face &6\cr
      &$\pm \pi/3$ \% diagonale &8\cr }}
    \item \question{de deux droites non coplanaires.}
\reponse{si les droites ne sont pas perpendiculaires :
      \halign{ \hskip 5cm #\hfill&\quad(#)\cr
      identité &1\cr
      $\frac12$-tour autour de la perpendiculaire commune &1\cr
      $\frac12$-tour autour d'une bissectrice &2\cr }
      si elles sont perpendiculaires, il y a aussi :
      \halign{ \hskip 5cm #\hfill&\quad(#)\cr
      symétrie \% plan contenant une droite et la perp. commune &2\cr
      symétrie-$\frac14$-tour autour de la perp. commune &2\cr }}
\end{enumerate}
}

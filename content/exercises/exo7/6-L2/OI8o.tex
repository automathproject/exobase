\uuid{OI8o}
\exo7id{7084}
\titre{exo7 7084}
\auteur{megy}
\organisation{exo7}
\datecreate{2017-01-21}
\isIndication{true}
\isCorrection{false}
\chapitre{Géométrie affine euclidienne}
\sousChapitre{Géométrie affine euclidienne du plan}
\module{Géométrie}
\niveau{L2}
\difficulte{}

\contenu{
\texte{
% tags: homothéties et translations
On donne deux cercles. Déterminer le nombre de tangentes communes aux deux cercles et tracer ces tangentes.
}
\indication{Si les cercles ne sont pas confondus, il peut y avoir entre $0$ et $4$ tangentes. Faire des figures pour tous les cas possibles. Puis, déterminer les homothéties (ou les translations) envoyant un cercle sur l'autre et tracer leur centre.}
}

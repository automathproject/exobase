\uuid{F3Mf}
\exo7id{7472}
\titre{exo7 7472}
\auteur{mourougane}
\organisation{exo7}
\datecreate{2021-08-10}
\isIndication{false}
\isCorrection{false}
\chapitre{Géométrie affine dans le plan et dans l'espace}
\sousChapitre{Géométrie affine dans le plan et dans l'espace}
\module{Géométrie}
\niveau{L2}
\difficulte{}

\contenu{
\texte{
On considére le plan euclidien muni d'un un repère orthonormé ($O, \overrightarrow {\imath},\overrightarrow{\jmath}$) et la courbe $(C)$ d'équation 

\begin{center}$sqrt{3}x^{2} +6xy + \sqrt{3}y^{2} +2(\sqrt{3} - 6)x -2(3+\sqrt{3})y + 1 =0 $ \end{center}
}
\begin{enumerate}
    \item \question{Montrer que cette courbe posséde un centre de symétrie $\Omega$ et donner son équation dans le repère ($\Omega, \overrightarrow {\imath},\overrightarrow{\jmath}$)}
    \item \question{Montrer que dans le repère ($\Omega, \overrightarrow {\imath},\overrightarrow{\jmath}$) la droite $\Delta$ d'équation $y = \sqrt{3}x$ est axe de symétrie.}
    \item \question{On considère le repère orthonormé direct ($\Omega, \overrightarrow {I},\overrightarrow{J}$) où $\overrightarrow {I}$ est un vecteur unitaire de $\Delta$. Donner l'équation de $(C)$ dans ce repère.}
    \item \question{Montrer que dans le repère ($\Omega, \overrightarrow {\imath},\overrightarrow{\jmath}$) la perpendiculaire $\Delta^{'}$ à $\Delta$ est axe de symétrie.}
    \item \question{Sachant que $(C)$ est une hyperbole dessiner graphiquement $(C)$ dans le repère ($O, \overrightarrow {\imath},\overrightarrow{\jmath})$.}
\end{enumerate}
}

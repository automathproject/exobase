\uuid{qzvB}
\exo7id{2036}
\titre{exo7 2036}
\auteur{liousse}
\organisation{exo7}
\datecreate{2003-10-01}
\isIndication{false}
\isCorrection{false}
\chapitre{Géométrie affine euclidienne}
\sousChapitre{Géométrie affine euclidienne du plan}
\module{Géométrie}
\niveau{L2}
\difficulte{}

\contenu{
\texte{
Soit  $P$ un plan muni d'un rep\`ere $(O,{\buildrel\rightarrow \over i},
{\buildrel\rightarrow \over j})$ quelconque.
}
\begin{enumerate}
    \item \question{Donner l'expression 
analytique de la translation $t_1$ de vecteur $(1,2).$\\}
    \item \question{Donner l'expression analytique de la translation $t_2$ de vecteur $(-1,2).$\\}
    \item \question{Donner l'expression analytique de l'homoth\'etie $h_1$ de centre l'origine 
du rep\`ere et de rapport 2 et de l'homoth\'etie $h_2$ de centre $A(2,-1)$ de 
rapport 3.\\}
    \item \question{Donner l'expression analytique de $t_1\circ h_1,$ $t_2\circ h_2,$ 
$h_1\circ t_1,$ $h_2\circ t_2.$}
    \item \question{Soit  $M(x,y)$ un point de $P$. Donner les coordonn\'ees du sym\'etrique de 
$M$ par rapport \`a la droite d'\'equation $y=ax+b.$}
\end{enumerate}
}

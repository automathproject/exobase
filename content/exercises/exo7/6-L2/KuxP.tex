\uuid{KuxP}
\exo7id{2487}
\titre{exo7 2487}
\auteur{matexo1}
\organisation{exo7}
\datecreate{2002-02-01}
\isIndication{false}
\isCorrection{false}
\chapitre{Analyse vectorielle}
\sousChapitre{Forme différentielle, champ de vecteurs, circulation}
\module{Géométrie}
\niveau{L2}
\difficulte{}

\contenu{
\texte{
Soit $C$ une courbe ferm\'ee du plan, et $P$ et $Q$  deux polyn\^omes de
degr\'e 1 en $x$, $y$\,; montrer que la valeur de $\oint_C 
P(x,y)\,dx+Q(x,y)\,dy$ ne change pas si l'on effectue une
translation sur $C$. En d\'eduire la valeur de $\oint_C
(3x+4y)\,dx+(x-3y)\,dy$, o\`u $C$ est un cercle quelconque de rayon
$a>0$.
}
}

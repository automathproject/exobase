\uuid{MBON}
\exo7id{5537}
\titre{exo7 5537}
\auteur{rouget}
\organisation{exo7}
\datecreate{2010-07-15}
\isIndication{false}
\isCorrection{true}
\chapitre{Courbes planes}
\sousChapitre{Propriétés métriques : longueur, courbure,...}
\module{Géométrie}
\niveau{L2}
\difficulte{}

\contenu{
\texte{
Trouver le point de la courbe d'équation $y=\ln x$ en lequel la valeur absolue du rayon de courbure est minimum.
}
\reponse{
$\mathcal{C}$ est le support de l'arc paramétré $t\mapsto\left(
\begin{array}{c}
t\\
\ln t
\end{array}
\right)$, $t>0$. 

\begin{center}
$\overrightarrow{\frac{dM}{dt}}=\left(\begin{array}{c}
1\\
1/t
\end{array}
\right)=\sqrt{1+\frac{1}{t^2}}\left(\begin{array}{c}
1/\sqrt{1+\frac{1}{t^2}}\\
1/\left(t\sqrt{1+\frac{1}{t^2}}\right)
\end{array}
\right)=\frac{\sqrt{1+t^2}}{t}\left(
\begin{array}{c}
\frac{t}{\sqrt{1+t^2}}\\
\frac{1}{\sqrt{1+t^2}}
\end{array}
\right)$.
\end{center}
Donc, $\frac{ds}{dt}=\frac{\sqrt{1+t^2}}{t}$ et on peut prendre $\alpha(t)=\Arcsin\left(\frac{1}{\sqrt{1+t^2}}\right)$ puis 

\begin{center}
$\frac{d\alpha}{dt}=-\frac{t}{(t^2+1)^{3/2}}\frac{1}{\sqrt{1-\frac{1}{1+t^2}}}=-\frac{1}{t^2+1}$,
\end{center}
et finalement

\begin{center}
$R(t)=\frac{ds/dt}{d\alpha/dt}=-\frac{1}{t}(t^2+1)^{3/2}$.
\end{center}
Pour $t>0$, posons $f(t)=|R(t)|=\frac{1}{t}(t^2+1)^{3/2}$. $f$ est dérivable sur $]0,+\infty[$ et pour $t>0$, 

\begin{center}
$f'(t)=-\frac{1}{t^2}(t^2+1)^{3/2}+3(t^2+1)^{1/2}=\frac{\sqrt{t^2+1}}{t^2}(-(t^2+1)+3t^2)=\frac{\sqrt{t^2+1}}{t^2}(2t^2-1)$.
\end{center}
$f$ admet un minimum en $t=\frac{1}{\sqrt{2}}$ égal à $\sqrt{2}\left(\frac{1}{2}+1\right)^{3/2}=\frac{3\sqrt{3}}{2}$.

\begin{center}
\shadowbox{
Le rayon de courbure minimum est $\frac{3\sqrt{3}}{2}$ et est le rayon de courbure en $M\left(\frac{1}{\sqrt{2}}\right)$, $-\frac{\pi}{2}<t<\frac{\pi}{2}$.
}
\end{center}
}
}

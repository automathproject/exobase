\uuid{pYTz}
\exo7id{7049}
\titre{exo7 7049}
\auteur{megy}
\organisation{exo7}
\datecreate{2017-01-08}
\isIndication{true}
\isCorrection{true}
\chapitre{Géométrie affine euclidienne}
\sousChapitre{Géométrie affine euclidienne du plan}
\module{Géométrie}
\niveau{L2}
\difficulte{}

\contenu{
\texte{
%tags : triangle rectangle, cercle circonscrit
On donne deux points $A$ et $B$. Construire un triangle $ABC$ rectangle en $C$ tel que $AB = 2AC$.
}
\indication{Tracer le milieu $I$ de $AB$ puis procéder par analyse-synthèse.}
\reponse{
On relie les points $A$ et $B$. On construit le milieu $I$ de $[AB]$. On a donc $AI = \frac{1}{2}AB$. On trace le cercle $\mathcal{C}$ de centre $A$ et de rayon $AI$. Ceci fait, On trace le cercle $\mathcal{C}'$ centr\'e en $I$ de diam\`etre $AB$. Nommons $C$ l'un des deux points d'intersection de $\mathcal{C}$ et $\mathcal{C}'$. Il s'agit de l'une des deux solutions possibles.

En effet, par construction $AC = \frac{1}{2} AB$, puisque $C$ appartient \`a $\mathcal{C}$. En outre $ACB$ est rectangle en $C$. En effet, $ACB$ est inscrit dans le cercle $\mathcal{C}'$ et $AB$ est un diam\`etre de $\mathcal{C}'$.
}
}

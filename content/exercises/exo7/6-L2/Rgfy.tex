\uuid{Rgfy}
\exo7id{7254}
\titre{exo7 7254}
\auteur{mourougane}
\organisation{exo7}
\datecreate{2021-08-10}
\isIndication{false}
\isCorrection{false}
\chapitre{Géométrie affine euclidienne}
\sousChapitre{Géométrie affine euclidienne du plan}
\module{Géométrie}
\niveau{L2}
\difficulte{}

\contenu{
\texte{
On rappelle qu'un quadrilatère d'un espace euclidien \(E\) est un 
\emph{parallélogramme} si ses diagonales se coupent en leur milieu, 
appelé centre du parallélogramme.

Cette définition est aussi valable en dimension 1 et pour les cas où 
deux sommets coïncident. (Dans ces cas, le parallélogramme est plat).

On dit que deux bipoints \((A,B)\) et \((C,D)\) sont \emph{équipollents} 
si le quadrilatère \((ABDC)\) est un parallélogramme.
}
\begin{enumerate}
    \item \question{Vérifier que pour tout couple de points \((A,B)\), les bipoints 
\((A,B)\) et \((A,B)\) sont équipollents. On dit alors que la relation 
d'équipollence est \emph{réflexive}.}
    \item \question{Montrer que pour tous bipoints \((A,B)\) et \((C,D)\), si les 
bipoints \((A,B)\) et \((C,D)\) sont équipollents alors les bipoints 
\((C,D)\) et \((A,B)\) le sont aussi. On dit alors que la relation 
d'équipollence est \emph{symétrique}.}
    \item \question{Démontrer que la relation d'équipollence est \emph{transitive}, 
c'est à dire que pour tous triplets \((A,B)\), \((C,D)\) et \((F,G)\) 
de bipoints, si les bipoints \((A,B)\) et \((C,D)\) sont équipollents 
et si les bipoints \((C,D)\) et \((F,G)\) sont équipollents alors les 
bipoints \((A,B)\) et \((F,G)\) le sont aussi. (Indication : dans le 
cas où le quadrilatère \((ABGF)\) n'est pas plat, on pourra considérer 
la droite joignant les centres des parallélogrammes \((ABDC)\) et 
\((CDGF)\); dans le cas où le quadrilatère \((ABGF)\) est plat, on 
pourra utiliser le théorème de Thalès.)}
    \item \question{On résume les trois propriétés précédentes en disant que la 
relation d'équipollence est \emph{une relation d'équivalence}.
La \emph{classe d'équipollence} du bipoint \((A,B)\) est par 
définition l'ensemble des bipoints équipollents à \((A,B)\).
Elle est appelée \emph{vecteur} et notée \(\vec{AB}\).
Si \((C,D)\) est équipollent à \((A,B)\), on dit que \((C,D)\) est 
un \emph{représentant} de \(\vec{AB}\). Montrer qu'étant donné un 
point \(A\) et un vecteur \(\vec{u}\), il existe un unique point \(B\) 
tel que \(\vec{AB} = \vec{u}\). On notera \(B = t_{\vec{u}}(A)\).}
    \item \question{Étant donnés un point \(A\) et un représentant \((F, G)\) 
du vecteur \(\vec{u}\), construire à la règle et au compas le 
point \(t_{\vec{u}}(A)\).}
    \item \question{Montrer que si deux bipoints \((A,B)\) et \((C,D)\) sont 
équipollents, alors les bipoints \((A,C)\) et \((B,D)\) le sont aussi.}
    \item \question{On définit la somme de deux vecteurs \(\vec{u}\) et \(\vec{v}\) 
par le procédé suivant:
\begin{itemize}}
    \item \question{On choisit un point \(A\).}
    \item \question{On détermine le point \(B\) tel que \(\vec{AB} = \vec{u}\).}
    \item \question{On détermine le point \(C\) tel que \(\vec{BC} = \vec{v}\).}
    \item \question{On définit \(\vec{u} + \vec{v} := \vec{AC}\).
\end{itemize}
Montrer que la \emph{somme} ainsi définie est indépendante du choix du 
point de base \(A\), c'est à dire, montrer que si on choisit un autre 
point \(A'\) comme point de base, le bipoint \((A',C')\) construit 
alors est équipollent au bipoint \((A,C)\) construit en partant du 
point \(A\). (Indication : On pourra montrer que \((A,A')\) est 
équipollent à \((C,C')\).)}
\end{enumerate}
}

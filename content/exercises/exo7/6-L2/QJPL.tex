\uuid{QJPL}
\exo7id{7452}
\titre{exo7 7452}
\auteur{mourougane}
\organisation{exo7}
\datecreate{2021-08-10}
\isIndication{false}
\isCorrection{false}
\chapitre{Géométrie affine dans le plan et dans l'espace}
\sousChapitre{Géométrie affine dans le plan et dans l'espace}
\module{Géométrie}
\niveau{L2}
\difficulte{}

\contenu{
\texte{
On considère dans le plan euclidien muni d'un repère orthonormé la
courbe $C$ d'équation polaire $r=f(\theta )$ où $f$ est une fonction
continue positive sur $[-\frac{\pi}{2},\frac{\pi}{2}]$ avec
$f(-\frac{\pi}{2})=f(\frac{\pi}{2})=0$.
}
\begin{enumerate}
    \item \question{Représenter $C$ dans le cas où $f=cos$.}
    \item \question{On rappelle que l'aire de la surface $S$ délimitée par la courbe $C$
est donnée par $$a(S)=\int_{-\frac{\pi}{2}}^{\frac{\pi}{2}}
\frac{1}{2}f^2(\theta )d\theta. $$
Montrer que $$a(S)\leq \frac{\pi}{4}\left(diam (S)\right)^2.$$}
\end{enumerate}
}

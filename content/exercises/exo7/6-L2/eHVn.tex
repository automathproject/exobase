\uuid{eHVn}
\exo7id{5833}
\titre{exo7 5833}
\auteur{rouget}
\organisation{exo7}
\datecreate{2010-10-16}
\isIndication{false}
\isCorrection{true}
\chapitre{Conique}
\sousChapitre{Quadrique}
\module{Géométrie}
\niveau{L2}
\difficulte{}

\contenu{
\texte{
Equation cartésienne du cylindre $(\mathcal{C})$ de direction $\overrightarrow{u}$ et de directrice $(C)$ dans les cas suivants :
}
\begin{enumerate}
    \item \question{$\overrightarrow{u}(1,0,1)$ et $(C)$ : $x = a\cos t$,  $y = b\sin t$,  $z =a\sin t\cos t$ ($a$ et $b$ tous deux non nuls).}
\reponse{\begin{align*}\ensuremath
M(x,y,z)\in(\mathcal{C})&\Leftrightarrow\exists\lambda\in\Rr,\;\exists m\in(C)/\;M = m +\lambda\overrightarrow{u}\Leftrightarrow\exists\lambda\in\Rr,\;\exists t\in\Rr/\;\left\{
\begin{array}{l}
x=a\cos t+\lambda\\
y=b\sin t\\
z=a\cos t\sin t+\lambda
\end{array}
\right.\\
 &\Leftrightarrow\exists\lambda\in\Rr,\;\exists t\in\Rr/\;\left\{
\begin{array}{l}
\lambda=x-a\cos t\\
y=b\sin t\\
z=a\cos t\sin t+x-a\cos t
\end{array}
\right.\Leftrightarrow\exists t\in\Rr/\;\left\{
\begin{array}{l}
y=b\sin t\\
z=a\cos t\sin t+x-a\cos t
\end{array}
\right.\\
 &\Leftrightarrow\exists t\in\Rr/\;\left\{
\begin{array}{l}
y=b\sin t\\
z-x=a\cos t(\sin t-1)
\end{array}
\right.\Leftrightarrow\exists t\in\Rr/\;\left\{
\begin{array}{l}
y=b\sin t\\
b(z-x)=a\cos t(y-b)
\end{array}
\right.\\
 &\Leftrightarrow b^4(z-x)^2+y^2a^2(y-b)^2=a^2b^2(y-b)^2.
\end{align*}

En effet,

\textbullet~$\Rightarrow/$ s'il existe $t\in\Rr$ tel que $y=b\sin t$ et $b(z-x)=a\cos t(y-b)$ alors

\begin{align*}\ensuremath
b^4(z-x)^2+y^2a^2(y-b)^2&=b^2a^2\cos^2t(y-b)^2+b^2\sin^2ta^2(y-b)^2=a^2b^2(y-b)^2(\cos^2t+\sin^2t)\\
 &=a^2b^2(y-b)^2.
\end{align*}

\textbullet~$\Leftarrow/$ Réciproquement, si $b^4(z-x)^2+y^2a^2(y-b)^2=a^2b^2(y-b)^2$ alors $
b^4(z-x)^2=a^2(y-b)^2 (b^2-y^2)$ et donc

ou bien $y=b$, ou bien $b^2-y^2\geqslant 0$. Par suite, il existe un réel $t$ tel que $y = b\sin t=b\sin(\pi-t)$ puis

\begin{align*}\ensuremath
b^4(z-x)^2=a^2(y-b)^2 (b^2-y^2)&\Rightarrow b^4(z-x)^2=a^2(b\sin t-b)^2b^2\cos^2t\Rightarrow b(z-x) =\pm a\cos t(b\sin t-b)\\
 &\Rightarrow b(z-x) = a\cos t(y-b)\;\text{ou}\;b(z-x) = a\cos(\pi-t)(y-b)
\end{align*}

et il existe un réel $t'$ tel que $y=b\sin t'$ et $b(z-x) = a\cos t'(y-b)$.}
    \item \question{$\overrightarrow{u}(0,1,1)$ et $(C)$ :  $\left\{
\begin{array}{l}
y+z=1\\
x^2+y^2=1
\end{array}
\right.$.}
\reponse{\begin{align*}\ensuremath
M(x,y,z)\in(\mathcal{C})&\Leftrightarrow\exists\lambda\in\Rr,\;\exists m\in(C)/\;M = m +\lambda\overrightarrow{u}\Leftrightarrow\exists\lambda\in\Rr,\;\exists (X,Y,Z)\in\Rr^3/\;\left\{
\begin{array}{l}
x=X\\
y=Y+\lambda\\
z=Z+\lambda\\
Y+Z=1\\
X^2+Y^2=1
\end{array}
\right.\\
 &\Leftrightarrow\exists\lambda\in\Rr,\;\left\{
\begin{array}{l}
(y-\lambda)+(z-\lambda)=1\\
x^2+(y-\lambda)^2=1
\end{array}
\right.\Leftrightarrow x^2 +\left(y- \frac{1}{2}(y+z-1)\right)^2 = 1\\
 &\Leftrightarrow4x^2+(y-z+1)^2 = 4.
\end{align*}}
\end{enumerate}
}

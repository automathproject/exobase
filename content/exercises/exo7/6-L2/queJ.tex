\uuid{queJ}
\exo7id{7513}
\titre{exo7 7513}
\auteur{mourougane}
\organisation{exo7}
\datecreate{2021-08-10}
\isIndication{false}
\isCorrection{true}
\chapitre{Géométrie affine euclidienne}
\sousChapitre{Géométrie affine euclidienne du plan}
\module{Géométrie}
\niveau{L2}
\difficulte{}

\contenu{
\texte{
Soit $P$ un plan euclidien orienté.
}
\begin{enumerate}
    \item \question{Donner la liste des éléments du groupe des isométries du plan qui conservent un carré.}
\reponse{Les éléments du groupe des isométries du plan qui conservent un carré ont tous le centre $O$ du carré comme points fixes. Ce sont donc outre l'identité, des réflexions d'axe passant par $O$ et des rotations de centre $O$.
    Il y a quatre rotations $Id,r,r^2, r^3$ où $r$ est la rotation de centre $O$ et d'angle $+\pi/2$. Il y a aussi quatre réflexions $s,rs,r^2s,r^3s$.}
    \item \question{Ce groupe est-il commutatif ?}
\reponse{Comme la composée des réflexions dépend de l'ordre, $(rs)s=r$ et $s(rs)=r^{-1}$, ce groupe n'est pas commutatif.}
    \item \question{Le groupe des déplacements du plan qui conservent un carré est-il commutatif ?}
\reponse{Le groupe des déplacements du plan qui conservent un carré est un sous-groupe du groupe commutatif des rotation de centre $O$. Il est donc commutatif.}
\end{enumerate}
}

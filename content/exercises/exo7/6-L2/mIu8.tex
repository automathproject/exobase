\uuid{mIu8}
\exo7id{2697}
\titre{exo7 2697}
\auteur{matexo1}
\organisation{exo7}
\datecreate{2002-02-01}
\isIndication{false}
\isCorrection{false}
\chapitre{Courbes planes}
\sousChapitre{Autre}
\module{Géométrie}
\niveau{L2}
\difficulte{}

\contenu{
\texte{
Sur l'{\'e}cran d'un oscilloscope on observe la courbe dont les {\'e}quations
param{\'e}triques sont les suivantes: \[\hspace*{-3cm}\left\{\begin{array}{l}
x(t)=a\sin \omega t \\ y(t)=a\sin(\omega t-\varphi) \end{array} \right.\]
\begin{itemize}
\item Exprimer puis factoriser la somme et la diff{\'e}rence $x+y$ et $x-y$. 
\item
Soient $X$ et $Y$ les coordonn{\'e}es par rapport aux axes d{\'e}duits des axes ${\rm
O}x$ et ${\rm O}y$ par une rotation de $\pi/4$. Donner les {\'e}quations
param{\'e}triques de la courbe dans ce syst{\`e}me de coordonn{\'e}es. 
\item Tracer la
courbe et discuter de sa forme et du sens de parcours sur celle-ci en fonction du
param{\`e}tre $\varphi \in [0,2\pi]\, $ (consid{\'e}rer les valeurs multiples de
$\pi/2$ et les r{\'e}gions qu'elles d{\'e}limitent). 
\item La courbe {\'e}tant
suppos{\'e}e donn{\'e}e, en d{\'e}duire g{\'e}om{\'e}triquement la valeur de $\varphi$.
\end{itemize}
}
}

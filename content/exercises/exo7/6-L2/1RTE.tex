\uuid{1RTE}
\exo7id{7087}
\titre{exo7 7087}
\auteur{megy}
\organisation{exo7}
\datecreate{2017-01-21}
\isIndication{false}
\isCorrection{false}
\chapitre{Géométrie affine euclidienne}
\sousChapitre{Géométrie affine euclidienne du plan}
\module{Géométrie}
\niveau{L2}
\difficulte{}

\contenu{
\texte{
% homothéties
Soient $[AB]$ et $[CD]$ deux segments parallèles de longueurs différentes. Montrer qu'il existe des homothéties transformant l'un en l'autre. Combien y a-t-il de telles homothéties ? Tracer leurs centres. Montrer que la droite reliant ces centres coupe les segments en leur moitié.

Application : construire à la règle seule le symétrique de $A$ par rapport à $B$. Expliquer comment construire à la règle seule n'importe quel barycentre à coefficients rationnels de $A$ et de $B$.
}
}

\uuid{dRpp}
\exo7id{7449}
\titre{exo7 7449}
\auteur{mourougane}
\organisation{exo7}
\datecreate{2021-08-10}
\isIndication{false}
\isCorrection{false}
\chapitre{Géométrie affine dans le plan et dans l'espace}
\sousChapitre{Géométrie affine dans le plan et dans l'espace}
\module{Géométrie}
\niveau{L2}
\difficulte{}

\contenu{
\texte{
Dans le plan euclidien orienté $\mathcal{P}$,
 on considère deux demi-droites $d_1$ et $d_2$ de même origine $\Omega$.
}
\begin{enumerate}
    \item \question{Rappeler la forme générale d'une matrice de rotation vectorielle dans une
 base orthonormée directe de $\vec{\mathcal{P}}$. Comment change cette
 matrice quand on change de base orthonormée ?}
    \item \question{Montrer qu'il existe une unique rotation de centre $\Omega$ qui
 envoie $d_1$ sur $d_2$. On pourra donc définir l'angle orienté de
 deux demi-droites comme l'angle de la rotation qui envoie la
 première sur la seconde.}
\end{enumerate}
}

\uuid{8IGN}
\exo7id{7073}
\titre{exo7 7073}
\auteur{megy}
\organisation{exo7}
\datecreate{2017-01-11}
\isIndication{true}
\isCorrection{false}
\chapitre{Géométrie affine dans le plan et dans l'espace}
\sousChapitre{Propriétés des triangles}
\module{Géométrie}
\niveau{L2}
\difficulte{}

\contenu{
\texte{
%  cercle inscrit, bissectrice, triangles isocèles
Un quadrilatère convexe est dit \emph{tangentiel} ou \emph{circonscriptible} s'il possède un cercle inscrit, c'est-à-dire si ses quatre côtés sont tangents à un même cercle.
}
\begin{enumerate}
    \item \question{Montrer qu'un quadrilatère est tangentiel ssi ses bissectrices intérieures sont concourantes.}
    \item \question{Montrer le théorème de Pitot (1725) : dans un quadrilatère tangentiel, la somme des longueurs de deux côtés opposés est égale à la somme des deux autres. Réciproque ?}
    \item \question{Montrer qu'un cerf-volant isocèle (ou rhomboïde) est tangentiel.}
\indication{Décomposer les longueurs suivant les points de tangence du cercle inscrit.}
\end{enumerate}
}

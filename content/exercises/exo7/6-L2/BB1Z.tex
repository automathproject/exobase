\uuid{BB1Z}
\exo7id{7266}
\titre{exo7 7266}
\auteur{mourougane}
\organisation{exo7}
\datecreate{2021-08-10}
\isIndication{false}
\isCorrection{false}
\chapitre{Géométrie affine euclidienne}
\sousChapitre{Géométrie affine euclidienne du plan}
\module{Géométrie}
\niveau{L2}
\difficulte{}

\contenu{
\texte{
Soit $ABC$ un triangle dans un plan euclidien orienté $E$.
On notera $a=BC, b=CA$ et $c=AB$ et $\hat{A}=\widehat{(BAC)}$.
Notons $p$ la moitié de son périmètre.
}
\begin{enumerate}
    \item \question{Montrer que \begin{eqnarray*}
2p(p-a)=bc+\vec{AB}\cdot\vec{AC}\\
          2(p-b)(p-c)=bc-\vec{AB}\cdot\vec{AC}
         \end{eqnarray*}}
    \item \question{En déduire que l'aire du triangle $ABC$ est $\sqrt{p(p-a)(p-b)(p-c)}$.}
\end{enumerate}
}

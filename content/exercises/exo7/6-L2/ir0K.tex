\uuid{ir0K}
\exo7id{4899}
\titre{exo7 4899}
\auteur{quercia}
\organisation{exo7}
\datecreate{2010-03-17}
\isIndication{false}
\isCorrection{true}
\chapitre{Géométrie affine dans le plan et dans l'espace}
\sousChapitre{Propriétés des triangles}
\module{Géométrie}
\niveau{L2}
\difficulte{}

\contenu{
\texte{
Soit $ABC$ un triangle.
On note :
$\alpha \equiv (\overline{ \vec{AB}, \vec{AC}})$,
$\beta  \equiv (\overline{ \vec{BC}, \vec{BA}})$,
$\gamma \equiv (\overline{ \vec{CA}, \vec{CB}})$.
}
\begin{enumerate}
    \item \question{Soit $A'$ le pied de la hauteur issue de $A$.
    Calculer $\frac{\overline{A'B}}{\overline{A'C}}$.}
\reponse{$-\frac{\tan \gamma}{\tan \beta}$.}
    \item \question{En déduire les coordonnées barycentriques de l'orthocentre $H$.}
\reponse{$H = \text{Bar}(A:\tan\alpha, B:\tan\beta, C:\tan\gamma)
                = \text{Bar}\left(A:\frac a{\cos\alpha},
                                  B:\frac b{\cos\beta},
                                 C:\frac c{\cos\gamma}\right)$.}
\end{enumerate}
}

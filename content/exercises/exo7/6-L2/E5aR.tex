\uuid{E5aR}
\exo7id{7500}
\titre{exo7 7500}
\auteur{mourougane}
\organisation{exo7}
\datecreate{2021-08-10}
\isIndication{false}
\isCorrection{false}
\chapitre{Géométrie affine euclidienne}
\sousChapitre{Géométrie affine euclidienne de l'espace}
\module{Géométrie}
\niveau{L2}
\difficulte{}

\contenu{
\texte{
Dans l'espace vectoriel euclidien $\Rr^3$ muni du produit scalaire
standard  et  de  la  base  canonique $\mathcal{C}$,  appliquer  le  procédé
d'orthonormalisation de Gram-Schmidt à la base $\mathcal{B}$
$$v_1=\left( \begin{array}{c}  3\\1\\1\end{array}\right)\ ; \ 
v_2=\left(
\begin{array}{c}   2\\ 1\\ 0\end{array}\right)\   ;   \  
 v_3=\left(
\begin{array}{c} -1\\ -1\\ -1 \end{array}\right)
$$ pour obtenir une base orthonormée $\mathcal{B'}$.
}
}

\uuid{RDfg}
\exo7id{7111}
\titre{exo7 7111}
\auteur{megy}
\organisation{exo7}
\datecreate{2017-01-21}
\isIndication{true}
\isCorrection{false}
\chapitre{Géométrie affine euclidienne}
\sousChapitre{Géométrie affine euclidienne du plan}
\module{Géométrie}
\niveau{L2}
\difficulte{}

\contenu{
\texte{
% tags : rotation, dur, utile
On considère trois droites parallèles $D_1$, $D_2$ et $D_3$. Construire un triangle équilatéral dont les sommets appartiennent respectivement à $D_1$, $D_2$ et $D_3$.
}
\indication{Si $ABC$ est un tel triangle, considérer les rotations d'angles $\pm \pi/3$ et centrées sur les sommets. Déterminer les images des différents points et droites par ces rotations.}
}

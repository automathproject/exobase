\uuid{ngT0}
\exo7id{7754}
\titre{exo7 7754}
\auteur{mourougane}
\organisation{exo7}
\datecreate{2021-08-11}
\isIndication{false}
\isCorrection{false}
\chapitre{Géométrie projective}
\sousChapitre{Géométrie projective}
\module{Algèbre et géométrie}
\niveau{L3}
\difficulte{}

\contenu{
\texte{

}
\begin{enumerate}
    \item \question{Soit $\vec{V}$ un espace vectoriel de dimension $n+1$ muni d'une
 base $\mathcal{B}:=(\vec{v_i})$ et $(x_i)$ les coordonnées cartésiennes
 associées. Soit $\mathcal{R}$ le repère projectif associé de
 $P(\vec{V})$. Soit $\vec{v}$ un vecteur de $\vec{V}$.
 Déterminer un système de coordonnées homogènes dans $\mathcal{R}$ pour
 $vect(\vec{v})$ en fonction des coordonnées cartésiennes de
 $\vec{v}$ dans $\mathcal{B}$.}
    \item \question{Soit $E$ un espace affine de dimension $n$ et $\widehat{E}$ son
 complété vectoriel. On identifie $E$ à un ouvert affine de
 $P(\widehat{E})$ par l'application naturelle $M\mapsto vect\left(
 (\!( 1,M )\!)\right)$. Soit 
$\mathcal{A}:=(A_i)_{0\leq i\leq n}$ un repère affine de
 $E$.
 \begin{enumerate}}
    \item \question{Considérons d'abord la base 
$\mathcal{B}_1:=\left( (\!( 1,A_i)\!)\right) $ de
$\widehat{E}$ et $\mathcal{R}_1$ le repère projectif associé. 
Déterminer un sytème de coordonnées homogènes dans
$\mathcal{R}_1$ de $M\in
E$ considéré dans $P(\widehat{E})$ en fonction de ses coordonnées
barycentriques dans $\mathcal{A}$. Donner une équation de l'hyperplan
à l'infini.}
    \item \question{Considérons maintenant la base 
$\mathcal{B}_2:=\left( (\!( 1,A_0)\!) , (\!( 0,\vec{A_0A_i})\!)\right) $ de
$\widehat{E}$ et $\mathcal{R}_2$ le repère projectif associé. 
Déterminer un sytème de coordonnées homogènes dans
$\mathcal{R}_2$ de $M\in
E$ considéré dans $P(\widehat{E})$ en fonction de ses coordonnées
cartésiennes dans $\mathcal{A}$. Donner une équation de l'hyperplan
à l'infini.}
\end{enumerate}
}

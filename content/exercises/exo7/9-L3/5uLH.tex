\uuid{5uLH}
\exo7id{6404}
\titre{exo7 6404}
\auteur{potyag}
\organisation{exo7}
\datecreate{2011-10-16}
\isIndication{false}
\isCorrection{false}
\chapitre{Géométrie et trigonométrie sphérique}
\sousChapitre{Géométrie et trigonométrie sphérique}
\module{Algèbre et géométrie}
\niveau{L3}
\difficulte{}

\contenu{
\texte{
Soit $ S^n$ est la sphère unité dans l'espace linéaire $E$
de dimension $n+1.$
}
\begin{enumerate}
    \item \question{Montrer que la distance sphérique induit sur $ S^n$ une topologie équivalente à
celle induite de l'espace ambiant $E.$}
    \item \question{Montrer que l'intersection d'un sous-espace  linéaire $L$ de $E$ de dimension $k$
  avec $S^n$ est une sphère de dimension $k-1$ (si $k=2$ cette intersection est
  un cercle appelée grand cercle de $ S^n$).}
\end{enumerate}
}

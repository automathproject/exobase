\uuid{L7r1}
\exo7id{2204}
\titre{exo7 2204}
\auteur{debes}
\organisation{exo7}
\datecreate{2008-02-12}
\isIndication{true}
\isCorrection{false}
\chapitre{Théorème de Sylow}
\sousChapitre{Théorème de Sylow}
\module{Théorie des groupes}
\niveau{L3}
\difficulte{}

\contenu{
\texte{
\label{ex:deb104}
Soient $p<q$ deux nombres premiers distincts et $G$ un
groupe d'ordre $pq$. Montrer que $G$ admet un unique $q$-Sylow $Q$
qui est distingu\'e et que $G=QP$, o\`u $P$ est un $p$-Sylow de $G$.
Montrer que $G$ est isomorphe au produit semi-direct d'un groupe
cyclique d'ordre $q$ par un groupe cyclique d'ordre $p$. Montrer que
si $q-1$ n'est pas divisible par $p$, ce produit semi-direct est en
fait un produit direct.
}
\indication{Les th\'eor\`emes de Sylow montrent qu'il n'y a qu'un seul
$q$-Sylow, n\'ecessairement distingu\'e. La suite est standard. Pour le dernier
point, utiliser que $\hbox{\rm Aut}(\Z/q\Z) \simeq (\Z/q\Z)^\times$ (exercice \ref{ex:le17}) et donc que $\Z/p\Z$ ne peut agir non trivialement sur
$\Z/q\Z$ que si
$p$ divise $q-1$.}
}

\uuid{DdYz}
\exo7id{7882}
\titre{exo7 7882}
\auteur{mourougane}
\organisation{exo7}
\datecreate{2021-08-11}
\isIndication{false}
\isCorrection{false}
\chapitre{Sous-groupe distingué}
\sousChapitre{Sous-groupe distingué}
\module{Théorie des groupes}
\niveau{L3}
\difficulte{}

\contenu{
\texte{
Soit $E$ un $k$-espace vectoriel. Soit $f$ une forme linéaire sur $E$ et $a$ un élément non nul de $H=ker(f)$.
On appelle transvection associée à $f$ et $a$ l'application $u : E\to E$, $x\mapsto x+f(x)a$.
On rappelle que les transvections de $E$ engendrent $SL(E)$.
}
\begin{enumerate}
    \item \question{Soit $u$ une transvection. En considérant une base $(e_i)$ de $E$ avec 
$e_{n-1}=a$, $(e_j)_{1\leq j\leq n-1}$ base de $H$, et $e_n$ tel que $f(e_n)=1$, écrire la matrice de $u$.}
    \item \question{Montrer que si $u$ est une transvection, $Ker(u-Id)=H$, $\det u=1$, $u$ n'est pas diagonalisable.}
    \item \question{Montrer que si $\dim E\geq 3$, les transvections de $E$ sont conjuguées dans $SL(E)$.}
    \item \question{On suppose $k$ de caractéristique différente de $2$ et $\dim E\geq 3$.
 Montrer que $$D(GL(E)=D(SL(E))=SL(E)$$ et donc que ni $GL(E)$, ni $SL(E)$ ne sont résolubles.}
\end{enumerate}
}

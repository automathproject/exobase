\uuid{V4oE}
\exo7id{7881}
\titre{exo7 7881}
\auteur{mourougane}
\organisation{exo7}
\datecreate{2021-08-11}
\isIndication{false}
\isCorrection{false}
\chapitre{Sous-groupe distingué}
\sousChapitre{Sous-groupe distingué}
\module{Théorie des groupes}
\niveau{L3}
\difficulte{}

\contenu{
\texte{
Soit $p>q>r$ trois nombres premiers.
Le but de l'exercice est de montrer qu'un groupe d'ordre $pqr$ est résoluble.
}
\begin{enumerate}
    \item \question{Montrer qu'un groupe d'ordre $pq$ est résoluble.}
    \item \question{Soit $G$ un groupe d'ordre $pqr$. Supposons qu'il n'admette pas de sous-groupe distingué.
On note $N_p$ (resp. $N_q$, $N_r$ le nombre de sous-groupes de Sylow d'ordre $p$ (resp. $q$, $r$).
Montrer que $m_p=qr$, $m_q\geq p$ et $m_r\geq q$.}
    \item \question{Conclure.}
\end{enumerate}
}

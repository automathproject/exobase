\uuid{6yTT}
\exo7id{4569}
\titre{exo7 4569}
\auteur{quercia}
\organisation{exo7}
\datecreate{2010-03-14}
\isIndication{false}
\isCorrection{true}
\chapitre{Série entière}
\sousChapitre{Rayon de convergence}
\module{Analyse}
\niveau{L2}
\difficulte{}

\contenu{
\texte{
On considère les suites $(a_n)$ et $(b_n)$ définies par :
$a_n = \frac{\cos(n\pi/3)}{n^{1/3}}$, $b_n = \sin(a_n)$.
\smallskip
}
\begin{enumerate}
    \item \question{Déterminer les rayons de convergence des séries $\sum a_nx^n$ et $\sum b_nx^n$.}
\reponse{$(a_n)$ est bornée et $(na_n)$ ne l'est pas, donc $R_a = 1$. $|b_n|\sim|a_n|$ donc $R_b=1$.}
    \item \question{Déterminer la nature de $\sum a_nx^n$ et $\sum b_nx^n$ en fonction de $x$.}
\reponse{Il y a doute seulement pour $x=\pm 1$. Le critère de convergence d'Abel (hors programme)
    s'applique, $\sum a_nx^n$ converge si $x=\pm1$.
    $b_n = a_n -\frac16a_n^3 +  O(n^{-5/3})$ et le critère d'Abel s'applique aussi
    à $\sum a_n^3x^n$ (linéariser le $\cos^3$), il y a aussi convergence pour $x=\pm 1$.
    
    Résolution conforme au programme : regrouper par paquets de six termes.}
\end{enumerate}
}

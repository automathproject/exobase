\uuid{e1XU}
\exo7id{6994}
\titre{exo7 6994}
\auteur{blanc-centi}
\organisation{exo7}
\datecreate{2015-07-04}
\video{4M0txOmYb1I}
\isIndication{true}
\isCorrection{true}
\chapitre{Equation différentielle}
\sousChapitre{Résolution d'équation différentielle du premier ordre}
\module{Analyse}
\niveau{L2}
\difficulte{}

\contenu{
\texte{
Résoudre les équations différentielles suivantes en trouvant 
une solution particulière par la méthode de variation de la constante :
}
\begin{enumerate}
    \item \question{$y' - (2x - \frac{1}{x})y = 1$ sur $]0;+\infty[$}
    \item \question{$y'-y = x^k \exp(x)$ sur $\R$, avec $k \in \Nn$}
    \item \question{$x(1+\ln^2(x))y'+2\ln(x)y=1$ sur $]0;+\infty[$}
\reponse{
$y' - (2x - \frac{1}{x})y = 1$ sur $]0;+\infty[$
  \begin{enumerate}
\textbf{Résolution de l'équation homogène $y' - (2x - \frac{1}{x})y = 0$.}
    
    Une primitive de $a(x) = 2x - \frac1x$ est $A(x) = x^2 - \ln x$,
    donc les solutions de l'équation homogène sont les $y(x) = \lambda \exp(x^2 - \ln x) = \lambda \frac1x\exp(x^2)$,
    pour $\lambda$ une constante réelle quelconque.
\textbf{Recherche d'une solution particulière.}
    
    Nous allons utiliser la méthode de variation de la constante pour trouver une solution particulière 
    à l'équation $y' - (2x - \frac{1}{x})y = 1$.
    On cherche une telle solution sous la forme $y_0(x) = \lambda(x) \frac1x\exp(x^2)$ 
    où $x \mapsto \lambda(x)$ est maintenant une fonction.
    
    On calcule d'abord 
    $$y_0'(x) = \lambda'(x) \frac1x\exp(x^2) + \lambda(x) \left(-\frac{1}{x^2}+2\right) \exp(x^2)$$
    
    Maintenant :
    \begin{align*}
          & y_0 \quad \text{ est  solution de }  y' - (2x + \frac{1}{x})y = 1 \\
    \iff & y_0' - (2x - \frac{1}{x})y_0 = 1 \\
    \iff& \lambda'(x) x\exp(x^2) + \lambda(x) \left(-\frac{1}{x^2}+2\right) \exp(x^2)
        - (2x - \frac{1}{x})\lambda(x) \frac1x\exp(x^2) = 1 \\
    \iff& \lambda'(x) \frac1x\exp(x^2) = 1 \qquad \text{cela doit se simplifier !}\\
    \iff& \lambda'(x) = x\exp(-x^2)
    \end{align*}
    
    Ainsi on peut prendre $\lambda(x) = -\frac12\exp(x^2)$, ce qui fournit la solution particulière :
    $$y_0(x) = \lambda(x) \frac1x\exp(x^2) = -\frac12\exp(-x^2)\frac1x\exp(x^2) = -\frac1{2x}$$
    
    Pour se rassurer, on n'oublie pas de vérifier que c'est bien une solution !
\textbf{Solution générale.}    
    
    L'ensemble des solutions s'obtient par la somme de la solution particulière avec les solutions de l'équation
    homogène. Autrement dit, les solutions sont les :
    $$y(x) = -\frac1{2x} + \lambda \frac1x\exp(x^2)\qquad (\lambda\in\Rr).$$
}
\indication{Solution particulière :
\begin{enumerate}
  \item $-\frac1{2x}$
  
  \item $\frac{x^{k+1}}{k+1} \exp(x)$
  
  \item $\frac{\ln x}{1+\ln^2(x)}$
\end{enumerate}}
\end{enumerate}
}

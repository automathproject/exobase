\uuid{ESW3}
\exo7id{4821}
\titre{exo7 4821}
\auteur{quercia}
\organisation{exo7}
\datecreate{2010-03-16}
\isIndication{false}
\isCorrection{true}
\chapitre{Topologie}
\sousChapitre{Topologie des espaces vectoriels normés}
\module{Analyse}
\niveau{L2}
\difficulte{}

\contenu{
\texte{
Soit $E$ un espace vectoriel complexe de dimension finie.
On consid{\`e}re un endomorphisme $f$ de~$E$ et on note
$\rho(f) = \sup\{|\lambda|\text{ tq }\lambda\in\mathrm{Sp}(f)\}$
(rayon spectral de~$f$).
Soit $\nu$ une norme sur $\mathcal{L}(E)$.
}
\begin{enumerate}
    \item \question{Montrer que $\rho(f)\le \lim_{p\to\infty}(\nu(f^p)^{1/p})$.
    On pourra pour commencer supposer que $\nu$ est la norme subordonn{\'e}e
    {\`a} une norme sur~$E$.}
\reponse{Si $\nu$ est subordonn{\'e}e {\`a} $\|\ \|$~:
    on a $|\lambda| \le \nu(f^p)^{1/p}$ pour toute valeur propre $\lambda$
    et tout $p\ge 1$, donc il suffit de prouver que la suite $(x_p = \nu(f^p)^{1/p})$
    est convergente. Soit $\ell = \inf\{x_p,\ p\ge 1\}$, $\varepsilon > 0$
    et $p\ge 1$ tel que $x_p \le \ell+\varepsilon$. Pour $n> p$ on note
    $n-1=pq+r$ la division euclidienne de $n-1$ par $p$ et l'on a~:
    $$\nu(f^n) = \nu((f^p)^q\circ f^{r+1})
               \le \nu(f^p)^q\nu(f^{r+1})$$
    d'o{\`u}~:
    $$\ell \le x_n \le x_p^{pq/n}x_{r+1}^{(r+1)/n}\le (\ell+\varepsilon)^{pq/n}\max(x_1,\dots,x_p)^{(r+1)/n}.$$
    Le majorant tend vers $\ell+\varepsilon$ quand $n$ tend vers l'infini
    donc pour $n$ assez grand on a $\ell \le x_n \le \ell+2\varepsilon$
    ce qui prouve la convergence demand{\'e}e.
    
    Dans le cas o{\`u} $\nu$ est une norme quelconque sur $\mathcal{L}(E)$, il existe
    une norme subordonn{\'e}e $\mu$ et deux r{\'e}els $a,b>0$ tels que $a\mu \le \nu\le b\mu$
    et donc les suites $(\nu(f^p)^{1/p})$ et  $(\mu(f^p)^{1/p})$ ont m{\^e}me
    limite par le th{\'e}or{\`e}me des gens d'armes.
    Remarque~: il r{\'e}sulte de ceci que $\lim_{p\to\infty}(\nu(f^p)^{1/p})$
    est ind{\'e}pendant de $\nu$.}
    \item \question{Montrer que si $f$ est diagonalisable l'in{\'e}galit{\'e} pr{\'e}c{\'e}dente est une {\'e}galit{\'e}.}
\reponse{Consid{\'e}rer la matrice de $f^p$ dans une base propre pour~$f$.}
    \item \question{{\'E}tudier le cas g{\'e}n{\'e}ral.}
\reponse{On sait que $f^p = \sum_{\lambda\in\mathrm{spec}(f)} \lambda^pP_\lambda(p)$
    o{\`u} $P_\lambda$ est un polyn{\^o}me. D'o{\`u} $\rho(f) \le \nu(f^p)^{1/p}\le \rho(f)+ o(1)$
    et donc $\nu(f^p)^{1/p}\xrightarrow[p\to\infty]{} \rho(f)$ (thm du rayon spectral).}
\end{enumerate}
}

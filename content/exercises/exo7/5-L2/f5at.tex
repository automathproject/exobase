\uuid{f5at}
\exo7id{4833}
\titre{exo7 4833}
\auteur{quercia}
\organisation{exo7}
\datecreate{2010-03-16}
\isIndication{false}
\isCorrection{true}
\chapitre{Topologie}
\sousChapitre{Compacité}
\module{Analyse}
\niveau{L2}
\difficulte{}

\contenu{
\texte{
Soit $E$ un espace vectoriel norm{\'e},
$(K_n)$ une suite d{\'e}croissante de compacts non vides de $E$ et $K = \bigcap_n K_n$.
}
\begin{enumerate}
    \item \question{Montrer que $K \ne \varnothing$.}
    \item \question{Soit $U$ un ouvert contenant $K$. Montrer qu'il existe $n$ tel que $K_n \subset U$.}
    \item \question{Montrer que $\delta(K) = \lim_{n\to\infty} \delta(K_n)$ ($\delta$ est le diam{\`e}tre).}
\reponse{
Soit $\ell = \lim_{n\to\infty} \delta(K_n)$.
Il existe $x_n,y_n \in K_n$ tels que $d(x_n,y_n) = \delta(K_n)$.
Apr{\`e}s extraction de sous-suites, on peut supposer que $x_n \to x$ et
$y_n \to y$. Pour $\varepsilon > 0$, $(\overline B(x,\varepsilon) \cap K_n)$
et $(\overline B(y,\varepsilon) \cap K_n)$ forment des suites d{\'e}croissantes de
compacts non vides, donc $\overline B(x,\varepsilon) \cap K$ et $\overline B(y,\varepsilon) \cap K$
sont non vides. Par cons{\'e}quent, $\delta(K) \ge \ell - 2\varepsilon$.
}
\end{enumerate}
}

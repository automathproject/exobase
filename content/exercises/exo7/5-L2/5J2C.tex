\uuid{5J2C}
\exo7id{4515}
\titre{exo7 4515}
\auteur{quercia}
\organisation{exo7}
\datecreate{2010-03-14}
\isIndication{false}
\isCorrection{true}
\chapitre{Suite et série de fonctions}
\sousChapitre{Convergence simple, uniforme, normale}
\module{Analyse}
\niveau{L2}
\difficulte{}

\contenu{
\texte{
On considère la suite $(f_n)$ de fonctions sur $[0,1]$ définie par les relations :
$f_0 = 0$, $f_{n+1}(t) = f_n(t) + \frac{t-f_n^2(t)}2$.

\'Etudier la convergence simple, uniforme, des fonctions $f_n$.
}
\reponse{
$f_n(t) \to \sqrt t$ par valeurs croisantes, il y a convergence uniforme.
}
}

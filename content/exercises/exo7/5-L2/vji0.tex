\uuid{vji0}
\exo7id{5755}
\titre{exo7 5755}
\auteur{rouget}
\organisation{exo7}
\datecreate{2010-10-16}
\isIndication{false}
\isCorrection{true}
\chapitre{Série entière}
\sousChapitre{Autre}
\module{Analyse}
\niveau{L2}
\difficulte{}

\contenu{
\texte{
Soit $A$ une matrice carrée complexe de format $p\in\Nn^*$.
Rayon de convergence et somme en fonction de $\chi_A$ de la série entière $\sum_{n=0}^{+\infty}\text{Tr}(A^n)z^n$.
}
\reponse{
Posons $\text{Sp}_\Cc A=\left(\lambda_1,\ldots,\lambda_p\right)$. On sait que pour tout entier naturel $n$, $\text{Tr}(A^n) =\lambda_1^n+...+\lambda_p^n$.

Soit $\lambda$ un nombre complexe. 

\textbullet~Si $\lambda= 0$, la série entière associée à la suite $(\lambda^n)_{n\in\Nn}$ est de rayon infini et pour tout nombre complexe $z$, $\sum_{n=0}^{+\infty}\lambda^nz^n=1=\frac{1}{1-\lambda z}$.

\textbullet~Si $\lambda\neq0$, la série entière associée à la suite $(\lambda^n)$ est de rayon $\frac{1}{|\lambda|}$ et pour $|z| <\frac{1}{|\lambda|}$,  $\sum_{n=0}^{+\infty}\lambda^nz^n=\frac{1}{1-\lambda z}$.

Soit $\rho=\text{Max}\left(|\lambda_1|,\ldots,|\lambda_p|\right)$ ($\rho$ est le rayon spectral de la matrice $A$) et $R=\frac{1}{\rho}$  si $\rho\neq0$ et $R = +\infty$ si $\rho= 0$.

Pour $|z| < R$,  

\begin{align*}\ensuremath
\sum_{n=0}^{+\infty}\text{Tr}(A^n)z^n&=\sum_{n=0}^{+\infty}\left(\sum_{k=1}^{p}(\lambda_kz)^n\right)
=\sum_{k=1}^{p}\left(\sum_{n=0}^{+\infty}(\lambda_kz)^n\right)\;(\text{somme de}\;p\;\text{séries convergentes})\\
 &=\sum_{k=1}^{p}\frac{1}{1-\lambda_kz}.
\end{align*}

Il est alors clair que $R$ est le rayon de convergence de la série entière proposée (développement en série entière d'une fraction rationnelle).

Si de plus, $0< |z| < R$,  $\sum_{n=0}^{+\infty}\text{Tr}(A^n)z^n=\frac{1}{z}\left(\sum_{k=1}^{p}\frac{1}{\frac{1}{z}-\lambda_k}\right)=\frac{\chi_A'\left(\frac{1}{z}\right)}{\frac{1}{z}\chi_A\left(\frac{1}{z}\right)}$ (décomposition usuelle de $\frac{P'}{P}$).
}
}

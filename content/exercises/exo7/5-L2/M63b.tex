\uuid{M63b}
\exo7id{4853}
\titre{exo7 4853}
\auteur{quercia}
\organisation{exo7}
\datecreate{2010-03-16}
\isIndication{false}
\isCorrection{true}
\chapitre{Topologie}
\sousChapitre{Fonctions vectorielles}
\module{Analyse}
\niveau{L2}
\difficulte{}

\contenu{
\texte{
Soit $\varphi : {[a,b]} \to {\R^2}, t \mapsto {M_t}$ une courbe param{\'e}tr{\'e}e de classe
$\mathcal{C}^1$ de longueur non nulle.
Le centre de gravit{\'e} de la courbe est le point $G$ tel que
$ \int_{t=a}^b \vec{GM_t}\|\vec M\,'(t)\|\,dt = \vec 0$.
}
\begin{enumerate}
    \item \question{Montrer l'existence et l'unicit{\'e} de $G$.}
    \item \question{D{\'e}terminer le centre de gravit{\'e} d'un demi-cercle.
    (On admet que $G$ est ind{\'e}pendant du param{\'e}trage)}
    \item \question{Montrer que $G$ appartient {\`a} l'enveloppe convexe de la courbe.}
    \item \question{Montrer que si la courbe admet un axe de sym{\'e}trie, $\Delta$, alors $G \in \Delta$.
    (Si $\sigma$ est la sym{\'e}trie associ{\'e}e, consid{\'e}rer la courbe d{\'e}crite par
    $N_t = \sigma(M_t)$)}
    \item \question{Soit $\Phi : {\R^2} \to {\R^2}$ une isom{\'e}trie affine.
    Montrer que si $G$ est le centre de gravit{\'e} de $\mathcal{C}$, alors $\Phi(G)$ est le centre
    de gravit{\'e} de $\Phi(\mathcal{C})$.}
\reponse{
demi-cercle unit{\'e} $ \Rightarrow  x = 0$, $y = \frac 2\pi$.
Sommes de Riemann + l'enveloppe convexe d'un compact est compacte.
$\vec N\,'(t) = \vec\sigma(\vec M\,'(t))  \Rightarrow 
             \|\vec N\,'(t)\| = \|\vec M\,'(t))\|$.\par
             $ \int_{t=a}^b \vec{GN_t}\|\vec N\,'(t)\|\,dt = \vec 0
             =  \int_{t=a}^b \vec{\sigma(G)N_t}\|\vec M\,'(t)\|\,dt$,
             donc $\sigma(G) = G$.
}
\end{enumerate}
}

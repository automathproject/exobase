\uuid{TiBd}
\exo7id{5855}
\titre{exo7 5855}
\auteur{rouget}
\organisation{exo7}
\datecreate{2010-10-16}
\isIndication{false}
\isCorrection{true}
\chapitre{Topologie}
\sousChapitre{Application linéaire continue, norme matricielle}
\module{Analyse}
\niveau{L2}
\difficulte{}

\contenu{
\texte{
On munit $E =\ell^\infty(\Cc)$ le $\Cc$-espace vectoriel des suites bornées de la norme $\|u\|_\infty=\underset{n\in\Nn}{\text{sup}}\;|u_n|$.

On considère les endomorphismes $\Delta$ et $C$ de $\ell^\infty(\Cc)$ définis par :

\begin{center}
$\forall u\in E$, $\Delta(u) = v$ où $\forall n\in\Nn$, $v_n = u_{n+1}-u_n$ et $\forall u\in E$, $C(u)=w$ où $\forall n\in\Nn$, $w_n= \frac{1}{n+1}\sum_{k=0}^{n}u_k$.
\end{center}

Montrer que $\Delta$ et $C$ sont continus sur $(E,\|\;\|_\infty)$ et calculer leur norme.
}
\reponse{
(La linéarité de $\Delta$ est claire et de plus $\Delta$ est un endomorphisme de $E$ car si $u$ est une suite bornée, $\Delta(u)$ l'est encore. Plus précisément,)

\begin{center}
$\forall u\in E$, $\forall n\in\Nn$, $\left|\Delta(u)_n\right|\leqslant|u_n|+|u_{n+1}|\leqslant 2\|u\|_\infty$ et donc $\forall u\in E$, $\|\Delta(u)\|_\infty\leqslant 2\|u\|_\infty$.
\end{center}

Ceci montre que $\Delta$ est continu sur $E$ et $|||\Delta|||\leqslant2$. Ensuite, si $u$ est la suite définie par $\forall n\in\Nn$, $u_n = (-1)^n$ alors $u$ est un élément non nul de $E$ tel que $\|u\|_\infty=1$ et $\|\Delta(u)\|_\infty= 2$. En résumé,

\textbullet~$\forall u\in E\setminus\{0\}$, $ \frac{\|\Delta(u)\|_\infty}{\|u\|_\infty}\leqslant2$,

\textbullet~$\exists u\in E\setminus\{0\}$, $ \frac{\|\Delta(u)\|_\infty}{\|u\|_\infty}=2$.

On en déduit que

\begin{center}
\shadowbox{
$\Delta$ est continu sur $(E,\|\;\|_\infty)$ et $|||\Delta|||=2$.
}
\end{center}

(La linéarité de $C$ est claire et $C$ est un endomorphisme de $E$ car si $u$ est bornée, $C(u)$ l'est encore. Plus précisément,)

\begin{center}
$\forall u\in E$, $\forall n\in\Nn$, $\left|(C(u))_n\right|\leqslant \frac{1}{n+1}\sum_{k=0}^{n}\|u\|_\infty =\|u\|_\infty$ et donc $\forall u\in E$, $\|C(u)\|_\infty\leqslant\|u\|_\infty$.
\end{center}

Par suite $T$ est continue sur $E$ et $|||T|||\leqslant1$. Ensuite, si $u$ est la suite définie par $\forall n\in\Nn$, $u_n = 1$ alors $u$ est un élément non nul de $E$ tel que $\|u\|_\infty=1$ et $\|C(u)\|_\infty=1$. En résumé,

\textbullet~$\forall u\in E\setminus\{0\}$, $ \frac{\|C(u)\|_\infty}{\|u\|_\infty}\leqslant1$,

\textbullet~$\exists u\in E\setminus\{0\}$, $ \frac{\|C(u)\|_\infty}{\|u\|_\infty}=1$.

On en déduit que

\begin{center}
\shadowbox{
$C$ est continu sur $(E,\|\;\|_\infty)$ et $|||C|||=1$.
}
\end{center}
}
}

\uuid{7LuQ}
\exo7id{5900}
\titre{exo7 5900}
\auteur{rouget}
\organisation{exo7}
\datecreate{2010-10-16}
\isIndication{false}
\isCorrection{true}
\chapitre{Fonction de plusieurs variables}
\sousChapitre{Continuité}
\module{Analyse}
\niveau{L2}
\difficulte{}

\contenu{
\texte{
Soit $(E,\|\;\|)$ un espace vectoriel normé et $B=\{x\in E/\;\|x\| < 1\}$. Montrer que $\begin{array}[t]{cccc}
f~:&E&\rightarrow&B\\
 &x&\mapsto& \frac{x}{1+\|x\|}
\end{array}$ est un homéomorphisme.
}
\reponse{
\textbullet~Pour tout $x\in E$, $\|f(x)\|= \frac{\|x\|}{1+\|x\|}< \frac{\|x\|+1}{\|x\|+1}=1$. Donc $f$ est bien une application de $E$ dans $B$.

\textbullet~Si $y=0$, pour $x\in E$, $f(x)=y\Leftrightarrow \frac{1}{1+\|x\|}x=0\Leftrightarrow x=0$.

Soit alors $y\in B\setminus\{0\}$. Pour $x\in E$,

\begin{center}
$f(x)=y\Rightarrow x=(1+\|x\|)y\Rightarrow\exists\lambda\in\Kk/\;x=\lambda y$.
\end{center}

Donc un éventuel antécédent de $y$ est nécessairement de la forme $\lambda y$, $\lambda\in\Rr$. Réciproquement, pour $\lambda\in\Rr$, $f(\lambda y)= \frac{\lambda}{1+|\lambda|\|y\|}y$ et donc

\begin{align*}\ensuremath
f(\lambda y)=y&\Leftrightarrow \frac{\lambda}{1+|\lambda|\|y\|}=1\Leftrightarrow\lambda=1+|\lambda|\|y\|\\
 &\Leftrightarrow(\lambda\geqslant0\;\text{et}\;(1-\|y\|)\lambda=1)\;\text{ou}\;(\lambda<0\;\text{et}\;(1+\|y\|)\lambda=1)\\
 &\Leftrightarrow\lambda= \frac{1}{1-\|y\|}\;(\text{car}\;\|y\|<1).
\end{align*}

Dans tous les cas, $y$ admet un antécédent par $f$ et un seul à savoir $x= \frac{1}{1-\|y\|}y$. Ainsi,

\begin{center}
$f$ est bijective et $\forall x\in B$, $f^{-1}(x)= \frac{1}{1-\|x\|}x$.
\end{center}

\textbullet~On sait que l'application $x\mapsto\|x\|$ est continue sur $\Rr^2$. Donc l'application $x\mapsto \frac{1}{1+\|x\|}$ est continue sur $\Rr^2$ en tant qu'inverse d'une fonction continue sur $\Rr^2$ à valeurs dans $\Rr$, ne s'annulant pas sur $\Rr^2$. L'application $x\mapsto \frac{1}{1-\|x\|}$ est continue sur $B$ pour les mêmes raisons. Donc les applications $f$ et $f^{-1}$ sont continues sur $\Rr^2$ et $B$ respectivement et on a montré que

\begin{center}
\shadowbox{
l'application $\begin{array}[t]{cccc}
f~:&E&\rightarrow&B\\
 &x&\mapsto& \frac{x}{1+\|x\|}
\end{array}$ est un homéomorphisme.
}
\end{center}
}
}

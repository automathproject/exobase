\uuid{T7hi}
\exo7id{6992}
\titre{exo7 6992}
\auteur{blanc-centi}
\organisation{exo7}
\datecreate{2015-07-04}
\video{hVq0LPY0kV0}
\isIndication{true}
\isCorrection{true}
\chapitre{Equation différentielle}
\sousChapitre{Résolution d'équation différentielle du premier ordre}
\module{Analyse}
\niveau{L2}
\difficulte{}

\contenu{
\texte{
Déterminer toutes les fonctions $f:[0;1]\to\R$, dérivables, telles que 
$$\forall x\in[0;1],\ f'(x)+f(x)=f(0)+f(1)$$
}
\indication{Une telle fonction $f$ est solution d'une équation différentielle $y'+y=c$.}
\reponse{
Une fonction $f:[0;1]\to\R$ convient si et seulement si 
\begin{itemize}
\item $f$ est dérivable
\item $f$ est solution de $y'+y=c$ 
\item $f$ vérifie $f(0)+f(1)=c$ (où $c$ est un réel quelconque)
\end{itemize}
Or les solutions de l'équation différentielle $y'+y=c$ sont exactement les 
$f:x\mapsto \lambda e^{-x}+c$, où $\lambda\in\R$ (en effet, on voit facilement que 
la fonction constante égale à $c$ est une solution particulière de $y'+y=c$). \'Evidemment ces fonctions sont dérivables, et $f(0)+f(1)=\lambda(1+e^{-1})+2c$, donc la troisième condition est satisfaite si et seulement si $-\lambda(1+e^{-1})=c$.

Ainsi les solutions du problème sont exactement les 
$$f(x)=\lambda(e^{-x}-1-e^{-1})$$
pour $\lambda\in\R$.
}
}

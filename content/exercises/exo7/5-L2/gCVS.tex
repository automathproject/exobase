\uuid{gCVS}
\exo7id{5761}
\titre{exo7 5761}
\auteur{rouget}
\organisation{exo7}
\datecreate{2010-10-16}
\isIndication{false}
\isCorrection{true}
\chapitre{Série entière}
\sousChapitre{Développement en série entière}
\module{Analyse}
\niveau{L2}
\difficulte{}

\contenu{
\texte{
Pour $x\in\left]-\frac{\pi}{2},\frac{\pi}{2}\right[$, on pose $f(x) =\tan x$.
}
\begin{enumerate}
    \item \question{Montrer qu'il existe une suite de polynômes $(P_n)_{n\in\Nn}$ telle que pour tout entier naturel $n$, $f^{(n)}= P_n\circ f$ et que les $P_n$ sont à coefficients entiers naturels. (Utiliser $\tan' = 1+\tan^2$).}
\reponse{La fonction $f$ est de classe $C^\infty$ sur $\left]-\frac{\pi}{2},\frac{\pi}{2}\right[$ en tant que quotient de fonctions de classe $C^\infty$ sur $\left]-\frac{\pi}{2},\frac{\pi}{2}\right[$ dont le dénominateur ne s'annule pas sur $\left]-\frac{\pi}{2},\frac{\pi}{2}\right[$ et de plus $f'= 1 + f^2$.

Montrons par récurrence que pour tout naturel $n$, il existe un polynôme $P_n$ à coefficients entiers naturels tel que $f^{(n)}= P_n\circ f$ (ou encore $\forall x\in\left]-\frac{\pi}{2},\frac{\pi}{2}\right[$, $\tan^{(n)}(x) = P_n(\tan x)$).

\textbullet~C'est vrai pour $n=0$ avec $P_0=X$ et pour $n =1$ avec $P_1= 1+X^2$.

\textbullet~Soit $n\geqslant1$. Supposons que pour tout $k\in\llbracket0,n\rrbracket$, il existe un polynôme $P_k$ à coefficients entiers naturels tel que $f^{(k)}= P_k\circ f$. D'après la formule de \textsc{Leibniz},

\begin{center}
$f^{(n+1)}=(1+f^2)^{(n)}= (f^2)^{(n)}=\sum_{k=0}^{n}\dbinom{n}{k}f^{(k)}f^{(n-k)}=\left(\sum_{k=0}^{n}\dbinom{n}{k}P_kP_{n-k}\right)\circ f$
\end{center}

et le polynôme $P_{n+1}=\sum_{k=0}^{n}\dbinom{n}{k}P_kP_{n-k}$ est un polynôme à coefficients entiers naturels tel que $\tan^{(n+1)}=P_{n+1}\circ f$.

\textbf{Remarque.} On aurait pu aussi dériver l'égalité $f^{(n)}=P_n\circ f$ pour obtenir $f^{(n+1)}= f'\times P_n'\circ f = (P_1\times P_n')\circ f$ mais on a déjà dans l'idée une relation de récurrence sur les coefficients du développement de $\tan$ qui n'est pas fournie par cette dernière égalité.}
    \item \question{En utilisant la formule de \textsc{Taylor}-\textsc{Laplace}, montrer que la série de \textsc{Taylor} à l'origine de $f$ a un rayon de convergence $R$ supérieur ou égal à $\frac{\pi}{2}$.}
\reponse{Soient $x\in\left[0,\frac{\pi}{2}\right[$ et $n\in\Nn$. La formule de \textsc{Taylor}-\textsc{Laplace} à l'ordre $n$ en $0$ fournit

\begin{center}
$f(x)=\sum_{k=0}^{n}\frac{f^{(k)}(0)}{k!}x^k+\int_{0}^{x}\frac{(x-t)^n}{n!}f^{(n+1)}(t)\;dt $.
\end{center}

Le 1) montre que pour tout réel $t$ de $\left[0,\frac{\pi}{2}\right[$ et tout entier naturel $k$, $f^{(k)}(t)=P_{k}(\tan t)\geqslant 0$.

Donc, d'une part $\frac{f^{(k)}(0)}{k!}x^k\geqslant0$ et d'autre part,

\begin{center}
$\sum_{k=0}^{n}\frac{f^{(k)}(0)}{k!}x^k=f(x)-\int_{0}^{x}\frac{(x-t)^n}{n!}f^{(n+1)}(t)\;dt$.
\end{center}

La suite des sommes partielles de la série de terme général $\frac{f^{(k)}(0)}{k!}x^k\geqslant0$ est majorée et donc la série de terme général $\frac{f^{(k)}(0)}{k!}x^k$ converge.

Ainsi, la série de \textsc{Taylor} de $f$ à l'origine converge pour tout réel $x$ de $\left[0,\frac{\pi}{2}\right[$. Son rayon de convergence $R$ est donc supérieur ou égal à $\frac{\pi}{2}$ (et donc la série de terme général $\frac{f^{(k)}(0)}{k!}x^k$ converge aussi pour $x\in\left]-\frac{\pi}{2},0\right]$). Il n'y a par contre aucune raison pour le moment pour que sa somme soit $f$.}
    \item \question{On note $a_n$ les coefficients du développement précédent et $g$ la somme de la série entière associée à la suite $(a_n)_{n\in\Nn}$. Montrer que pour tout entier naturel non nul $n$, $(n+1)a_{n+1}= \sum_{k=0}^{n}a_ka_{n-k}$. En déduire que pour tout $x$ de $\left]-\frac{\pi}{2},\frac{\pi}{2}\right[$, $f(x) = g(x)$ et que $R=\frac{\pi}{2}$.}
\reponse{Pour $n$ entier naturel donné, posons $a_n =\frac{f^{(n)}(0)}{n!}$ puis pour $x$ dans $\left]-\frac{\pi}{2},\frac{\pi}{2}\right[$, posons $g(x) =\sum_{n=0}^{+\infty}a_nx_n$.

On a vu que $\forall n\in\Nn^*$, $P_{n+1}=\sum_{k=0}^{n}P_kP_{n-k}$. On divise les deux membres de ces égalités par $n!$ et on prend la valeur en $0$ ($=\tan0$). On obtient

\begin{center}
$\forall n\in\Nn^*$, $(n+1)a_{n+1}=a_ka_{n-k}$ et aussi $a_0 = 0$ et $a_1 = 1$.
\end{center}

Donc, pour $x\in\left]-\frac{\pi}{2},\frac{\pi}{2}\right[$,

\begin{align*}\ensuremath
g'(x)&=\sum_{n=0}^{+\infty}(n+1)a_{n+1}x^n=1+\sum_{n=1}^{+\infty}\left(\sum_{k=0}^{n}a_ka_{n-k}\right)x^n =1+\sum_{n=0}^{+\infty}\left(\sum_{k=0}^{n}a_ka_{n-k}\right)x^n=1 +\left(\sum_{n=0}^{+\infty}a_nx^n\right)^2\\
 &= 1 + g^2(x).
\end{align*}

De plus, $g(0) = a_0 = 0$.

Pour $x\in\left]-\frac{\pi}{2},\frac{\pi}{2}\right[$, posons alors $h(x) =\Arctan(g(x))$. La fonction $h$ est dérivable sur $\left]-\frac{\pi}{2},\frac{\pi}{2}\right[$ et pour $x\in\left]-\frac{\pi}{2},\frac{\pi}{2}\right[$

\begin{center}
$h'(x)=\frac{g'(x)}{1+g^2(x)}= 1$ puis $h(x) =h(0)+(x-0) = x$.
\end{center}

Ainsi, pour tout $x\in\left]-\frac{\pi}{2},\frac{\pi}{2}\right[$, $g(x) =\tan x = f(x)$. Ceci montre déjà que $f$ est développable en série entière sur $\left]-\frac{\pi}{2},\frac{\pi}{2}\right[$. Mais quand $x$ tend vers $\frac{\pi}{2}$  par valeurs inférieures, $g(x) = f(x)$ tend vers $+\infty$ et donc $R\leqslant\frac{\pi}{2}$ puis $R=\frac{\pi}{2}$.

En résumé, la fonction tangente est développable en série entière sur $\left]-\frac{\pi}{2},\frac{\pi}{2}\right[$ et pour $x\in\left]-\frac{\pi}{2},\frac{\pi}{2}\right[$, $\tan x =\sum_{n=0}^{+\infty}a_nx^n$ où
$a_0 = 0$, $a_1 = 1$ et $\forall n\in\Nn^*$, $(n+1)a_{n+1}=\sum_{k=0}^{n}a_ka_{n-k}$. De plus, $\forall n\in\Nn$, $a_{2n}= 0$ puisque la fonction tangente est impaire.}
    \item \question{Calculer $a_0$, $a_1$, $a_2$,..., $a_7$.}
\reponse{$a_0 = a_2 = a_4 = a_6 = 0$ puis $a_1 = 1$.

$3a_3 = a_0a_2 + a_1^2 + a_2a_0 = 1$ et  donc $a_3 =\frac{1}{3}$.

$5a_5 =2a_1a_3=\frac{2}{3}$ et donc $a_5 =\frac{2}{15}$.

$7a_7 =2a_1a_5 + a_3^2=\frac{4}{15}+\frac{1}{9}=\frac{51}{135}=\frac{17}{45}$ et $a_7=\frac{17}{315}$.

\begin{center}
\shadowbox{
$\forall x\in\left]-\frac{\pi}{2},\frac{\pi}{2}\right[$, $\tan x = x +\frac{x^3}{2}+\frac{2x^5}{15}+\frac{17x^7}{315}+ ...$
}
\end{center}}
    \item \question{Vérifier que la fonction $x\mapsto\tanh x$ est développable en série entière. Préciser le rayon et la valeur des coefficients en fonction des $a_n$.}
\reponse{Pour tout  réel $x$, $\tanh(x) =\frac{1}{x}\tan(ix)$ et donc pour $x\in\left]-\frac{\pi}{2},\frac{\pi}{2}\right[$,

\begin{center}
$\tanh(x) =\frac{1}{i}\sum_{n=0}^{+\infty}a_{2n+1}(ix)^{2n+1}=\sum_{n=0}^{+\infty}(-1)^na_{2n+1}x^{2n+1}$.
\end{center}

Cette série entière a aussi pour rayon de convergence $\frac{\pi}{2}$.}
\end{enumerate}
}

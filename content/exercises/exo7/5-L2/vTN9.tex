\uuid{vTN9}
\exo7id{5850}
\titre{exo7 5850}
\auteur{rouget}
\organisation{exo7}
\datecreate{2010-10-16}
\isIndication{false}
\isCorrection{true}
\chapitre{Topologie}
\sousChapitre{Borne supérieure}
\module{Analyse}
\niveau{L2}
\difficulte{}

\contenu{
\texte{
Calculer $\underset{\alpha\in]0,\pi[}{\text{Inf}}\left\{\underset{n\in\Zz}{\text{Sup}}|\sin(n\alpha)|\right\}$.
}
\reponse{
Pour $\alpha\in]0,\pi[$, posons $f(\alpha) =\underset{n\in\Zz}{\text{Sup}}|\sin(n\alpha)|=\underset{n\in\Nn}{\text{Sup}}|\sin(n\alpha)|$.

\textbullet~Tout d'abord $\forall\alpha\in]0,\pi[$, $\forall n\in\Nn$, $|\sin(n(\pi-\alpha))| = |\sin(n\alpha)|$ et donc $\forall\alpha\in]0,\pi[$, $f(\pi-\alpha) = f(\alpha)$.

On en déduit que $\underset{\alpha\in]0,\pi[}{\text{Inf}}f(\alpha) =\underset{\alpha\in\left]0, \frac{\pi}{2}\right]}{\text{Inf}}f(\alpha)$.

\textbullet~$f\left( \frac{\pi}{3}\right)=\text{sup}\left\{0, \frac{\sqrt{3}}{2}\right\}= \frac{\sqrt{3}}{2}$.

\textbullet~Ensuite, si $\alpha\in\left[ \frac{\pi}{3}, \frac{\pi}{2}\right]$, $f(\alpha)\geqslant\sin(\alpha)\geqslant\sin\left( \frac{\pi}{3}\right)= \frac{\sqrt{3}}{2}=f\left( \frac{\pi}{3}\right)$.
Par suite $\underset{\alpha\in\left]0, \frac{\pi}{2}\right]}{\text{Inf}}f(\alpha)=\underset{\alpha\in\left]0, \frac{\pi}{3}\right]}{\text{Inf}}f(\alpha)$.

\textbullet~Soit alors $\alpha\in\left]0, \frac{\pi}{3}\right]$. Montrons qu'il existe un entier naturel non nul $n_0$ tel que $n_0\alpha\in\left[ \frac{\pi}{3}, \frac{2\pi}{3}\right]$.

Il existe un unique entier naturel $n_1$ tel que $n_1\alpha\leqslant \frac{\pi}{3}<(n_1+1)\alpha$ à savoir $n_1= E\left( \frac{\pi}{3\alpha}\right)$.

Mais alors, $ \frac{\pi}{3}<(n_1+1)\alpha= n\alpha+\alpha\leqslant \frac{\pi}{3}+ \frac{\pi}{3}= \frac{2\pi}{3}$ et l'entier $n_0=n_1+1$ convient.

Ceci montre que $f(\alpha)\geqslant\sin \frac{\pi}{3}= \frac{\sqrt{3}}{2}=f\left( \frac{\pi}{3}\right)$.

Finalement $\forall\alpha\in]0,\pi[$, $f(\alpha)\geqslant f\left( \frac{\pi}{3}\right)$ et donc 
$\underset{\alpha\in]0,\pi[}{\text{Inf}}\left\{\underset{n\in\Zz}{\text{Sup}}|\sin(n\alpha)|\right\}=\underset{\alpha\in]0,\pi[}{\text{Min}}\left\{\underset{n\in\Zz}{\text{Sup}}|\sin(n\alpha)|\right\}=f\left( \frac{\pi}{3}\right)= \frac{\sqrt{3}}{2}$.

\begin{center}
\shadowbox{
$\underset{\alpha\in]0,\pi[}{\text{Inf}}\left\{\underset{n\in\Zz}{\text{Sup}}|\sin(n\alpha)|\right\}= \frac{\sqrt{3}}{2}$.
}
\end{center}
}
}

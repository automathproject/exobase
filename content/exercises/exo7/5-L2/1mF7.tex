\uuid{1mF7}
\exo7id{4083}
\titre{exo7 4083}
\auteur{quercia}
\organisation{exo7}
\datecreate{2010-03-11}
\isIndication{false}
\isCorrection{true}
\chapitre{Equation différentielle}
\sousChapitre{Equations différentielles linéaires}
\module{Analyse}
\niveau{L2}
\difficulte{}

\contenu{
\texte{
On considère l'équation différentielle $y'=\sin(x+y)$. Montrer que
pour toute condition initiale l'intervalle maximal est
$\R$. Ensemble des points d'inflexion des courbes solutions ?
}
\reponse{
$y'$ étant bornée, $y$ admet une limite finie en tout point fini
donc la solution non prolongeable est définie sur~$\R$.

Une solution $y$ est de classe $\mathcal{C}^\infty$ vu l'équation et
$y'' = (1+\sin(x+y))\cos(x+y)$ est du signe de $\cos(x+y)$.
En un point $(x_0,y_0)$ tel que $x_0+y_0\equiv \frac\pi2(\mathrm{mod}\,{\pi})$
on a $y'' = 0$ et $\frac{d}{d x}(x+y)\ne 0$ donc $y''$ change
de signe et il y a inflexion.
En un point $(x_0,y_0)$ tel que $x_0+y_0\equiv \frac{3\pi}2(\mathrm{mod}\,{\pi})$
on a $x+y=$cste (car $y=\text{cste}-x$ est solution et il y a unicité)
donc il n'y a pas inflexion.
}
}

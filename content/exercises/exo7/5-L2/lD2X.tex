\uuid{lD2X}
\exo7id{4504}
\titre{exo7 4504}
\auteur{quercia}
\organisation{exo7}
\datecreate{2010-03-14}
\isIndication{false}
\isCorrection{false}
\chapitre{Suite et série de fonctions}
\sousChapitre{Convergence simple, uniforme, normale}
\module{Analyse}
\niveau{L2}
\difficulte{}

\contenu{
\texte{
On pose $f_n(x) = x^n(1-x)$ et $g_n(x) = x^n\sin(\pi x)$.
}
\begin{enumerate}
    \item \question{Montrer que la suite $(f_n)$ converge uniformément vers la fonction nulle sur
    $[0,1]$.}
    \item \question{En déduire qu'il en est de même pour la suite $(g_n)$.
    (On utilisera la concavité de sin sur $[0,\pi]$)}
\end{enumerate}
}

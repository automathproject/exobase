\uuid{HD4D}
\exo7id{1405}
\titre{exo7 1405}
\auteur{barraud}
\organisation{exo7}
\datecreate{2003-09-01}
\isIndication{false}
\isCorrection{false}
\chapitre{Groupe, anneau, corps}
\sousChapitre{Groupe de permutation}
\module{Algèbre}
\niveau{L2}
\difficulte{}

\contenu{
\texte{
Représenter graphiquement les permutations suivantes. Les décomposer
en produit de cycles à supports disjoints, puis en produits de
transpositions.
$$
\sigma_{1}=
 \left(
\begin{array}{@{}c@{}} 1234567\\1425376 \end{array}
 \right)
\qquad
\sigma_{2}=
 \left(
\begin{array}{@{}c@{}} 1234567\\2471635 \end{array}
 \right)
\qquad
\sigma_{3}=
 \left(
\begin{array}{@{}c@{}} 1234567\\3261547 \end{array}
 \right)
\qquad
\sigma_{4}=
 \left(
\begin{array}{@{}c@{}} 1234567\\7146253 \end{array}
 \right)
$$

  Calculer la signature des permutations ci-dessus. Calculer le produit
  $\sigma_{1}\sigma_{2}\sigma_{3}$ et sa signature. Comparer ce 
  résultat aux précédents.
}
}

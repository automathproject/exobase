\uuid{60kT}
\exo7id{3627}
\titre{exo7 3627}
\auteur{quercia}
\organisation{exo7}
\datecreate{2010-03-10}
\isIndication{false}
\isCorrection{true}
\chapitre{Endomorphisme particulier}
\sousChapitre{Autre}
\module{Algèbre}
\niveau{L2}
\difficulte{}

\contenu{
\texte{
Pour $\vec x = (x_1,\dots,x_n) \in  K^n$ on pose $f_i(\vec x) = x_i+x_{i+1}$
et $f_n(\vec x) = x_n+x_1$.
Déterminer si ${\cal F} = (f_1,\dots,f_n)$ est une base de $( K^n)^*$ et,
le cas échéant, déterminer la base duale.
}
\reponse{
Base ssi $n$ est impair, $2\vec e_1 = (1,1,-1,1,-1,\dots,1,-1)$ et
         les autres vecteurs s'obtiennent par rotation :\par
         $2\vec e_2 = (-1,1,1,-1,1,-1,\dots,1)$.
}
}

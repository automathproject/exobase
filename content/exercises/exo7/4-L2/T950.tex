\uuid{T950}
\exo7id{5300}
\titre{exo7 5300}
\auteur{rouget}
\organisation{exo7}
\datecreate{2010-07-04}
\isIndication{false}
\isCorrection{true}
\chapitre{Arithmétique}
\sousChapitre{Arithmétique de Z}
\module{Algèbre}
\niveau{L2}
\difficulte{}

\contenu{
\texte{
Montrer que la somme de cinq carrés parfaits d'entiers consécutifs n'est jamais un carré parfait.
}
\reponse{
Soit $n$ un entier supérieur ou égal à $2$.

$$(n-2)^2+(n-1)^2+n^2+(n+1)^2+(n+2)^2=5n^2+10=5(n^2+2).$$

$5(n^2+2)$ devant être un carré parfait, $n^2+2$ doit encore être divisible par $5$ mais si $n$ est dans $5\Zz$, $n^2+2$ est dans $2+5\Zz$, si $n$ est dans $\pm1+5\Zz$, $n^2+2$ est dans $3+5\Zz$ et si $n$ est dans $\pm2+5\Zz$, $n^2+2$ est dans $1+5\Zz$ et $n^2+2$ n'est jamais divisible par $5$. Une somme de cinq carrés d'entiers consécutifs n'est donc pas un carré parfait.
}
}

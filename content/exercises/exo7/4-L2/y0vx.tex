\uuid{y0vx}
\exo7id{2588}
\titre{exo7 2588}
\auteur{delaunay}
\organisation{exo7}
\datecreate{2009-05-19}
\isIndication{false}
\isCorrection{true}
\chapitre{Réduction d'endomorphisme, polynôme annulateur}
\sousChapitre{Polynôme annulateur}
\module{Algèbre}
\niveau{L2}
\difficulte{}

\contenu{
\texte{
Soit $N$ une matrice nilpotente, il existe $q\in\N$ tel que $N^q=0$. Montrer que la matrice $I-N$ est inversible et exprimer son inverse en fonction de $N$.
}
\reponse{
{\it Soit $N$ une matrice nilpotente, il existe $q\in\N$ tel que $N^q=0$. Montrons que la matrice $I-N$ est inversible et exprimons son inverse en fonction de $N$.}

On remarque que 
$(I-N)(I+N+N^2+\dots+N^{q-1})=I-N^q=I.$
Ainsi, la matrice $I-N$ est inversible, et son inverse est 
$(I-N)^{-1}=I+N+N^2+\dots+N^{q-1}.$
}
}

\uuid{U9KV}
\exo7id{5295}
\titre{exo7 5295}
\auteur{rouget}
\organisation{exo7}
\datecreate{2010-07-04}
\isIndication{false}
\isCorrection{true}
\chapitre{Arithmétique}
\sousChapitre{Arithmétique de Z}
\module{Algèbre}
\niveau{L2}
\difficulte{}

\contenu{
\texte{
Montrer que, pour tout entier naturel $n$, $2^{n+1}$ divise $E((1+\sqrt{3})^{2n+1})$.
}
\reponse{
Posons $(1+\sqrt{3})^n=a_n+b_n\sqrt{3}$ où $a_n$ et $b_n$ sont des entiers naturels. On a alors $(1-\sqrt{3})^n=a_n-b_n\sqrt{3}$ et donc 

$$(1+\sqrt{3})^{2n+1}+(1-\sqrt{3})^{2n+1}=2a_{2n+1}\in\Nn.$$

Mais de plus, $-1<1-\sqrt{3}< 0$ et donc, puisque $2n+1$ est impair, $-1<(1-\sqrt{3})^{2n+1}<0$. Par suite, 

$$2a_{2n+1}<(1+\sqrt{3})^{2n+1}<2a_{2n+1}+1,$$

ce qui montre que $E((1+\sqrt{3})^{2n+1})=2a_{2n+1}=(1+\sqrt{3})^{2n+1}+(1-\sqrt{3})^{2n+1}$ et montre déjà que $E((1+\sqrt{3})^{2n+1})$ est un entier pair. Mais on en veut plus~:

\begin{align*}\ensuremath
(1+\sqrt{3})^{2n+1}+(1-\sqrt{3})^{2n+1}&=(1+\sqrt{3})((1+\sqrt{3})^2)^n+(1-\sqrt{3})((1-\sqrt{3})^2)^n\\
 &=(1+\sqrt{3})(4+2\sqrt{3})^n+(1-\sqrt{3})(4-2\sqrt{3})^n\\
 &=2^n((1+\sqrt{3})(2+\sqrt{3})^n+(1-\sqrt{3})(2-\sqrt{3})^n)
\end{align*}

Montrons enfin que $(1+\sqrt{3})(2+\sqrt{3})^n+(1-\sqrt{3})(2-\sqrt{3})^n$ est un entier, pair. Mais, $(1+\sqrt{3})(2+\sqrt{3})^n$ est de la forme $A+B\sqrt{3}$ où $A$ et $B$ sont des entiers naturels et donc, puisque $(1-\sqrt{3})(2-\sqrt{3})^n=A-B\sqrt{3}$, on a finalement $(1+\sqrt{3})(2+\sqrt{3})^n+(1-\sqrt{3})(2-\sqrt{3})^n=2A$ où $A$ est un entier.

Donc, $(1+\sqrt{3})(2+\sqrt{3})^n+(1-\sqrt{3})(2-\sqrt{3})^n$ est un entier pair, ou encore $(1+\sqrt{3})^{2n+1}+(1-\sqrt{3})^{2n+1}=E((1+\sqrt{3})^{2n+1})$ est un entier divisible par $2^{n+1}$.
}
}

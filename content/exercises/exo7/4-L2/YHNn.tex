\uuid{YHNn}
\exo7id{3724}
\titre{exo7 3724}
\auteur{quercia}
\organisation{exo7}
\datecreate{2010-03-11}
\isIndication{false}
\isCorrection{true}
\chapitre{Espace euclidien, espace normé}
\sousChapitre{Forme quadratique}
\module{Algèbre}
\niveau{L2}
\difficulte{}

\contenu{
\texte{
Soit $n\ge 2$ et $A$ une matrice réelle symétrique $n\times n$
représentant une forme quadratique~$q$.
On appelle mineurs principaux de~$A$ les déterminants~:
$$\Delta_k(A) = \det\bigl((a_{i,j})_{i,j\le k}\bigr).$$
On suppose que tous les mineurs principaux de~$A$ sont non nuls,
montrer que la signature de~$q$ est $(r,s)$ où $s$ est le nombre
de changements de signe dans la suite $(1,\Delta_1,\dots,\Delta_n)$
et $r=n-s$.
}
\reponse{
Récurrence sur~$n$.

Soit $(e_1,\dots,e_n)$ la base dans laquelle $A$ est la matrice de~$q$.
$\Delta_{n-1}(A) \ne 0$ donc il existe des coefficients
$\alpha_1,\dots,\alpha_{n-1}$ tels que $u_n = e_n-\sum_{i<n}\alpha_i e_i$
soit $q$-orthogonal à~$e_1,\dots,e_{n-1}$. Alors $A$ a mêmes
mineurs principaux que la matrice de $q$ dans la base
$(e_1,\dots,e_{n-1},u_n)$.
}
}

\uuid{ljBb}
\exo7id{1530}
\titre{exo7 1530}
\auteur{legall}
\organisation{exo7}
\datecreate{1998-09-01}
\isIndication{false}
\isCorrection{false}
\chapitre{Endomorphisme particulier}
\sousChapitre{Endomorphisme auto-adjoint}
\module{Algèbre}
\niveau{L2}
\difficulte{}

\contenu{
\texte{
Soit $  E  $ un espace euclidien de dimension $  3  .$
}
\begin{enumerate}
    \item \question{Soit $  \{ e_1, e_2 ,e_3 \}   $ une base orthonorm\'ee de $  E  .$ Soient $ 
x=x_1e_1+x_2e_2+x_3e_3  $ et $ 
y=y_1e_1+y_2e_2+y_3e_3  $ deux vecteurs de $  E  .$ Calculer $  \langle x,y\rangle   $ en
fonction des coefficients $  x_i  $ et $  y_i  $ (pour $  i\in \{ 1,2,3\}  $).}
    \item \question{On consid\`ere $  u \in \mathcal{L} (E)  $ un endomorphisme auto-adjoint. On note $  \lambda   $ sa plus petite valeur propre et $  \lambda '  $ sa
plus grande valeur propre. Montrer que l'on a, pour tout $  x   $ appartenant \`a $  E  ,$ les
in\'egalit\'es~:
$$ \lambda  \Vert x\Vert ^2 \leq \langle u(x),x\rangle \leq \lambda ' \Vert x\Vert ^2.$$
(On utilisera une base orthonorm\'ee convenable.)}
    \item \question{Soit $  v \in \mathcal{L} (E)  $ un endomorphisme quelconque. Montrer que $  \displaystyle {u=
\frac{1}{ 2} (v+v^*)}  $ est auto-adjoint. Soient $  \mu   $ une valeur propre de $  v  ,$
$  \lambda   $ la
plus petite valeur propre de $  u  $ et $  \lambda '  $ la
plus grande valeur propre de $  u  .$ Montrer que $  \lambda  \leq \mu \leq \lambda '  .$}
\end{enumerate}
}

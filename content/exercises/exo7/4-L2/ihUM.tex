\uuid{ihUM}
\exo7id{5659}
\titre{exo7 5659}
\auteur{rouget}
\organisation{exo7}
\datecreate{2010-10-16}
\isIndication{false}
\isCorrection{true}
\chapitre{Réduction d'endomorphisme, polynôme annulateur}
\sousChapitre{Polynôme annulateur}
\module{Algèbre}
\niveau{L2}
\difficulte{}

\contenu{
\texte{
\label{ex:rou9}
Soient $f$ et $g$ deux endomorphismes d'un espace vectoriel de dimension finie vérifiant $fg - gf = f$. Montrer que $f$ est nilpotent.
}
\reponse{
Soit $k\in\Nn^*$.

\begin{align*}\ensuremath
f^kg-fg^k&=f^kg-f^{k-1}gf+f^{k-1}gf-f^{k-2}gf^2+f^{k-2}gf^2-\ldots-fgf^{k-1}+fgf^{k-1}-gf^k\\
 &=\sum_{i=0}^{k-1}(f^{k-i}gf^i-f^{k-i-1}gf^{i+1})=\sum_{i=0}^{k-1}f^{k-i-1}(fg-gf)f^i=\sum_{i=0}^{k-1}f^{k-i-1}ff^i\\
 &=kf^k.
\end{align*}

Ainsi,

\begin{center}
\shadowbox{
si $fg-gf=f$, alors $\forall k\in\Nn$, $f^kg-gf^k=kf^k$\quad$(*)$.
}
\end{center}

\textbf{1ère solution.} Soit $\begin{array}[t]{cccc}
\varphi~:&\mathcal{L}(E)&\rightarrow&\mathcal{L}(E)\\
 &h&\mapsto&hg-gh
\end{array}$. $\varphi$ est un endomorphisme de $\mathcal{L}(E)$ et $\forall k\in\Nn^*$, $\varphi(f^k)=kf^k$. Si, pour $k\in\Nn^*$ donné, $f^k$ n'est pas nul, $f^k$ est valeur propre de $\varphi$ associé à la valeur propre $k$. Par suite, si aucun des $f^k$ n'est nul, $\varphi$ admet une infinité de valeurs propres deux à deux distinctes. Ceci est impossible car $\text{dim}(\mathcal{L}(E))<+\infty$. Donc, $f$ est nilpotent.

\textbf{2ème solution.} Les égalités $(*)$ peuvent s'écrire $P(f)g-gP(f)=fP'(f)$, $(**)$, quand $P$ est un polynôme de la forme $X^k$, $k\in\Nn$. Par linéarité, les égalités $(**)$ sont vraies pour tout polynôme $P$.

En particulier, l'égalité $(**)$ est vraie quand $P$ est $Q_f$ le polynôme minimal de $f$ et donc

\begin{center}
$fQ_f'(f)=Q_f(f)g-gQ_f(f)=0$.
\end{center}

Le polynôme $XQ_f'$ est donc un polynôme annulateur de $f$ et on en déduit que le polynôme $Q_f$ divise le polynôme $XQ_f'$. Plus précisément, si $p\in\Nn^*$ est le degré de $Q_f$, les polynômes $pQ_f$ ayant mêmes degrés et mêmes coefficients dominants, on en déduit que $pQ_f=XQ_f'$ ou encore que

\begin{center}
$\frac{Q_f'}{Q_f}=\frac{p}{X}$.
\end{center}

Par identification à la décomposition en éléments simples usuelles de $\frac{Q_f'}{Q_f}$, on en déduit que $Q_f=X^p$. En particulier, $f^p=0$ et encore une fois $f$ est nilpotent.
}
}

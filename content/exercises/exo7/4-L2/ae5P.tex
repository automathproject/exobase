\uuid{ae5P}
\exo7id{1387}
\titre{exo7 1387}
\auteur{legall}
\organisation{exo7}
\datecreate{1998-09-01}
\isIndication{false}
\isCorrection{true}
\chapitre{Groupe, anneau, corps}
\sousChapitre{Groupe, sous-groupe}
\module{Algèbre}
\niveau{L2}
\difficulte{}

\contenu{
\texte{
Soit $  G  $ un groupe d'ordre $  pn  $ avec $  p
$ premier.
}
\begin{enumerate}
    \item \question{On consid\`ere deux sous-groupes $  H  $ et $  H'  $ de $  G  $ d'ordre $  p  $ avec $  H\not = H'  .$ Montrer que $  H\cap H'=\{ e\}  .$}
\reponse{$H\cap H'$ est un sous-groupe  de $H$ donc
$\mathrm{Card} H\cap H'$  divise $\mathrm{Card} H = p$. Or $p$ est premier donc
$\mathrm{Card} H\cap H' = 1$ ou $p$. Mais $H\cap H'\not= H$ donc $\mathrm{Card}
H\cap H' \not= p$ et donc $ H\cap H' = \{ e \}$.}
    \item \question{En d\'eduire que le nombre d'\'el\'ements d'ordre $  p   $ dans $  G  $ est un multiple de $  p-1  .$}
\reponse{Soit $E$ l'ensemble des \'el\'ements
d'ordre $p$ que l'on suppose non vide. Notons que pour $x\in E$ le
sous-groupe $H_x$ engendr\'e par $x$ est d'ordre $p$ et de plus
tout $z\in H_x\setminus \{ e \}$ est d'ordre $p$ car $H_x$ est
cyclique et $p$ est premier. Donc $H_x$ contient $p-1$ \'el\'ement
d'ordre $p$.

Si $E$ ne contient qu'un seule \'el\'ement $x$ alors $E = H_x
\setminus\{ e \}$ et donc $E$ contienet $p-1$ \'el\'ements.

Sinon, soit $x,y \in  E$ avec $x\not=y$. Alors d'apr\`es la
premi\`ere question  $H_x \cap H_y = \{ e \}$. Donc $E$ se
d\'ecompose en une union disjointe de $H_x \setminus \{ e \}$.
Donc $\mathrm{Card} E$ est multiple de $p-1$.}
\end{enumerate}
}

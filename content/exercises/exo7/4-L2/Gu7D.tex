\uuid{Gu7D}
\exo7id{5666}
\titre{exo7 5666}
\auteur{rouget}
\organisation{exo7}
\datecreate{2010-10-16}
\isIndication{false}
\isCorrection{true}
\chapitre{Réduction d'endomorphisme, polynôme annulateur}
\sousChapitre{Polynôme caractéristique, théorème de Cayley-Hamilton}
\module{Algèbre}
\niveau{L2}
\difficulte{}

\contenu{
\texte{
\label{ex:rou16}
Soit $A$ une matrice antisymétrique réelle. Etudier la parité de son polynôme caractéristique.
}
\reponse{
Soit $A\in\mathcal{M}_n(\Rr)$. $A$ est antisymétrique si et seulement si ${^t}A =-A$. Dans ce cas

\begin{center}
$\chi_A =\text{det}(A-XI) =\text{det}({^t}(A-XI))=\text{det}(-A-XI)=(-1)^n\text{det}(A+XI) =(-1)^n\chi_A(-X)$
\end{center}

Ainsi, $\chi_A$ a la parité de $n$.
}
}

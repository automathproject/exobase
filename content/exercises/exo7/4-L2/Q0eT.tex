\uuid{Q0eT}
\exo7id{5371}
\titre{exo7 5371}
\auteur{rouget}
\organisation{exo7}
\datecreate{2010-07-06}
\isIndication{false}
\isCorrection{true}
\chapitre{Déterminant, système linéaire}
\sousChapitre{Calcul de déterminants}
\module{Algèbre}
\niveau{L2}
\difficulte{}

\contenu{
\texte{
Soit $A=
\left(
\begin{array}{cccccc}
a_1&a_2&\ldots& &\ldots&a_n\\
a_n&a_1&a_2& & &a_{n-1}\\
a_{n-1}&a_n&a_1& & &a_{n-2}\\
\vdots& &\ddots&\ddots& &\vdots\\
 & & &\ddots&\ddots&\vdots\\
a_2&a_3&\ldots& &a_n&a_1
\end{array}
\right)
$ et $P=(\omega^{(k-1)(l-1)})_{1\leq k,l\leq n}$ où $\omega=e^{2i\pi/n}$. Calculer $P^2$ et $PA$. En déduire $\mbox{det}A$.
}
\reponse{
\textbullet~Soit $(k,l)\in\llbracket1,n\rrbracket^2$. Le coefficient ligne $k$, colonne $l$ de $P^2$ est

$$\alpha_{k,l}=\sum_{u=1}^{n}\omega^{(k-1)(u-1)}\omega^{(u-1)(l-1)}=\sum_{u=1}^{n}\omega^{(k+l-2)(u-1)}=\sum_{u=0}^{n-1}(\omega^{k+l-2})^u.$$
 

Or, $\omega^{k+l-2}=1\Leftrightarrow k+l-2\in n\Zz$. Mais, $0\leq k+l-2\leq 2n-2<2n$ et donc, $k+l-2\in n\Zz\Leftrightarrow k+l-2\in\{0,n\}\Leftrightarrow k+l=2\;\mbox{ou}\;k+l=n+2$. Dans ce cas, $\alpha_{k,l}=n$. Sinon, 

$$\alpha_{k,l}=\frac{1-(\omega^{k+l-2})^n}{1-\omega^{k+l-2}}=\frac{1-1}{1-\omega^{k+l-2}}=0.$$
Ainsi, $P^2=n\left(
\begin{array}{ccccc}
1&0&\ldots&\ldots&0\\
0& & &0&1\\
\vdots& &0&1&0\\
 & & & & \\
0&1&0&\ldots&0
\end{array}
\right)$.
\textbullet~Soit $(k,l)\in\llbracket1,n\rrbracket^2$. Le coefficient ligne $k$, colonne $l$ de $P\overline{P}$ est

$$\beta_{k,l}=\sum_{u=1}^{n}\omega^{(k-1)(u-1)}\omega^{-(u-1)(l-1)}=\sum_{u=1}^{n}(\omega^{k-l})^{u-1}.$$
Or, $\omega^{k-l}=1\Leftrightarrow k-l\in n\Zz$. Mais, $-n<-(n-1)\leq k-l\leq n-1<n$ et donc $k-l\in n\Zz\Leftrightarrow k=l$. Dans ce cas, $\beta_{k,l}=n$. Sinon, $\beta_{k,l}=0$. Ainsi, $P\overline{P}=nI_n$ (ce qui montre que $P\in GL_n(\Cc)$ et $P^{-1}=\frac{1}{n}\overline{P}$).
Calculons enfin $PA$. Il faut d'abord écrire proprement les coefficients de $A$. Le coefficient ligne $k$, colonne $l$ de $A$ peut s'écrire $a_{l-k+1}$ si l'on adopte la convention commode $a_{n+1}=a_1$, $a_{n+2}=a_2$ et plus généralement pour tout entier relatif $k$, $a_{n+k}=a_k$.
Avec cette convention d'écriture, le coefficient ligne $k$, colonne $l$ de $PA$ vaut

$$\sum_{u=1}^{n}\omega^{(k-1)(u-1)}a_{l-u+1}=\sum_{v=l-n+1}^{l}\omega^{(k-1)(l-v)}a_v.$$
Puis on réordonne cette somme pour qu'elle commence par $a_1$.

\begin{align*}\ensuremath
\sum_{v=l-n+1}^{l}\omega^{(k-1)(l-v)}a_v&=\sum_{v=1}^{l}\omega^{(k-1)(l-v)}a_v
+\sum_{v=l-n+1}^{0}\omega^{(k-1)(l-v)}a_v\\
 &=\sum_{v=1}^{l}\omega^{(k-1)(l-v)}a_v+\sum_{w=l+1}^{n}\omega^{(k-1)(l-w+n)}a_{w+n}\;(\mbox{en posant}\;w=v+n)\\
 &=\sum_{v=1}^{l}\omega^{(k-1)(l-v)}a_v+\sum_{w=l+1}^{n}\omega^{(k-1)(l-w)}a_{w}\\
 &=\sum_{v=1}^{n}\omega^{(k-1)(l-v)}a_v=\omega^{(k-1)(l-1)}\sum_{v=1}^{n}\omega^{(k-1)(1-v)}a_v
\end{align*}
(le point clé du calcul précédent est que les suites $(a_k)$ et $(\omega^k)$ ont la même période $n$ ce qui s'est traduit par
$\omega^{(k-1)(l-w+n)}a_{w+n}=\omega^{(k-1)(l-v)}a_v$).
Pour $k$ élément de $\llbracket1,n\rrbracket$, posons alors $S_k=\sum_{v=1}^{n}\omega^{(k-1)(1-v)}a_v$. On a montré que $PA=(\omega^{(k-1)(l-1)}S_k)_{1\leq k,l\leq n}$.

Par linéarité par rapport à chaque colonne, on a alors

\begin{center}
$\mbox{det}(PA)=\mbox{det}(\omega^{(k-1)(l-1)}S_k)_{1\leq k,l\leq n}=\left(\prod_{k=1}^{n}S_k\right)\times\mbox{det}(\omega^{(k-1)(l-1)})_{1\leq k,l\leq n}=\left(\prod_{k=1}^{n}S_k\right)\times\mbox{det}P$.
\end{center}
Donc $(\mbox{det}P)(\mbox{det}A)=\left(\prod_{k=1}^{n}S_k\right)\mbox{det}P$ et finalement, puisque $\mbox{det}P\neq0$,

\begin{center}
\shadowbox{
$\mbox{det}A=\prod_{k=1}^{n}\left(\sum_{v=1}^{n}\omega^{(k-1)(1-v)}a_v\right).$
}
\end{center}
Par exemple, pour $n=3$, $\mbox{det}A=(a_1+a_2+a_3)(a_1+ja_2+j^2a_3)(a_1+j^2a_2+ja_3)$.
}
}

\uuid{x0Ts}
\exo7id{3010}
\titre{exo7 3010}
\auteur{quercia}
\organisation{exo7}
\datecreate{2010-03-08}
\isIndication{false}
\isCorrection{true}
\chapitre{Groupe, anneau, corps}
\sousChapitre{Anneau}
\module{Algèbre}
\niveau{L2}
\difficulte{}

\contenu{
\texte{
Soit $A$ un anneau non nul tel que : $\forall\ x \in A,\ x^2 = x$.
}
\begin{enumerate}
    \item \question{Exemple d'un tel anneau ?}
    \item \question{Quels sont les {\'e}l{\'e}ments inversibles de $A$ ?}
    \item \question{Montrer que : $\forall\ x \in A,\ x+x = 0$.
    En d{\'e}duire que $A$ est commutatif.}
    \item \question{Pour $x,y \in A$ on pose :
    $x \le y \iff \exists\ a \in A \text{ tel que } x=ay$.
    Montrer que c'est une relation d'ordre.}
\reponse{
$1$.
$x+y = (x+y)^2 = x^2+y^2+xy+yx = x+y+xy+yx  \Rightarrow  xy+yx = 0$.\par
             Pour $y = 1$ : $x+x = 0  \Rightarrow  1 = -1$.\par
             Pour $y$ quelconque : $xy = -yx = yx$.
Antisym{\'e}trie : si $x = ay$, alors $xy = ay^2 = ay = x$.\par
             Donc $(x\le y)$ et $(y\le x)  \Rightarrow  xy = x = y$.
}
\end{enumerate}
}

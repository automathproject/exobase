\uuid{D0OG}
\exo7id{1621}
\titre{exo7 1621}
\auteur{liousse}
\organisation{exo7}
\datecreate{2003-10-01}
\isIndication{false}
\isCorrection{false}
\chapitre{Réduction d'endomorphisme, polynôme annulateur}
\sousChapitre{Diagonalisation}
\module{Algèbre}
\niveau{L2}
\difficulte{}

\contenu{
\texte{
Soit $T$ l'application lin\'eaire de $\mathbb R^3$ dans  $\mathbb R^3$ d\'efinie par sa matrice $A$
dans la base canonique $(e_1,e_2,e_3)$ de $\mathbb R^3$ :
 $$ A= {\begin{pmatrix} 1 & 2 & 0 \\ 1 & 3& -1
\\ 1 & -1& 3\end{pmatrix}}.$$
}
\begin{enumerate}
    \item \question{Donner un base de Ker $T$ et Im$T$.}
    \item \question{\begin{enumerate}}
    \item \question{Calculer le polyn{\^o}me caract\'eristique de $T$, puis ses valeurs propres.}
    \item \question{Justifier, sans calcul, que $T$ soit diagonalisable et
 \'ecrire une matrice diagonale semblable \`a $A$ .}
    \item \question{Calculer une base de $\mathbb R^3$ form\'ee de vecteurs propres de $T$.}
\end{enumerate}
}

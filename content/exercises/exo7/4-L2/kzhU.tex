\uuid{kzhU}
\exo7id{2571}
\titre{exo7 2571}
\auteur{delaunay}
\organisation{exo7}
\datecreate{2009-05-19}
\isIndication{false}
\isCorrection{true}
\chapitre{Réduction d'endomorphisme, polynôme annulateur}
\sousChapitre{Valeur propre, vecteur propre}
\module{Algèbre}
\niveau{L2}
\difficulte{}

\contenu{
\texte{
Soit $f$ un endomorphisme de $E$ v\'erifiant $f^2={\rm mathrm{Id}}_E$.
\begin {enumerate}
  \item  D\'emontrer que les seules valeurs propres possibles de $f$ sont $1$ et $-1$.
  \item V\'erifier que pour tout $\vec x\in E$, on a
$$f(\vec x-f(\vec x))=-(\vec x-f(\vec x))\ \ {\hbox{et}}\ \ f(\vec x+f(\vec x))=(\vec x+f(\vec x))$$
et en d\'eduire que $f$ admet toujours une valeur propre.
  \item  D\'emontrer que si $1$ et $-1$ sont valeurs propres, alors $E$ est somme directe des sous-espaces propres correspondants.

  \item Traduire g\'eom\'etriquement sur un dessin dans le cas $n=2$.
\end {enumerate}
}
\reponse{
Soit $f$ un endomorphisme de $E$ v\'erifiant $f^2={\rm mathrm{Id}}_E$.
\begin {enumerate}
  \item {\it D\'emontrons que les seules valeurs propres possibles de $f$ sont $1$ et $-1$}.

Si $\lambda$ est une valeur propre de $f$, il existe un vecteur non nul $\vec x\in E$ tel que $f(\vec x)=\lambda\vec x$. On a donc
$$f^2(\vec x)=f(\lambda\vec x)=\lambda f(\vec x)=\lambda^2\vec x.$$
Mais, $f^2={\rm mathrm{Id}}_E$ donc si $\vec x$ est un vecteur propre associ\'e \`a la valeur propre $\lambda$ on a
$$\vec x=f^2(\vec x)=\lambda^2\vec x,$$
d'o\`u $\lambda^2=1$, c'est-\`a-dire (dans $\R$ ou $\C$), $\lambda=1$ ou $\lambda=-1$. ce qui prouve que les seules valeurs propres possibles de $f$
 sont $1$ et $-1$.
  \item {\it  V\'erifions que pour tout $\vec x\in E$, on a}
$$f(\vec x-f(\vec x))=-(\vec x-f(\vec x))\ \ {\hbox{et}}\ \ f(\vec x+f(\vec x))=(\vec x+f(\vec x))$$
Soit $\vec x\in E$, on a
$$f(\vec x-f(\vec x))=f(\vec x)-f^2(\vec x)=f(\vec x)-\vec x=-(\vec x-f(\vec x))$$ et
$$f(\vec x+f(\vec x))=f(\vec x)+f^2(\vec x)=f(\vec x)+\vec x$$

{\it Nous allons en d\'eduire que $f$ admet toujours une valeur propre.}

Supposons que $1$ ne soit pas valeur propre de $f$, alors, $\vec x=f(\vec x)\Rightarrow \vec x=\vec 0$. Or, pour tout $\vec x\in E$, on a
$f(\vec x+f(\vec x))=f(\vec x)+\vec x$, donc pour tout $\vec x\in E$, on a $f(\vec x)+\vec x=\vec 0$, c'est-\`a-dire, $f(\vec x)=-\vec x$.
Ce qui prouve que $-1$ est valeur propre de $f$. On a m\^eme dans ce cas $f=-{\rm mathrm{Id}_E}$. 

Si $-1$ n'est pas valeur propre de $f$, on montre par un raisonnement
analogue que pour tout $\vec x\in E$ on a $f(\vec x)-\vec x=\vec 0$. Ce qui prouve que $1$ est valeur propre de $f$, et dans ce cas $f={\rm mathrm{Id}_E}$.

  \item  {\it D\'emontrons que si $1$ et $-1$ sont valeurs propres, alors $E$ est somme directe des sous-espaces propres correspondants.}

Supposons maintenant que $1$ et $-1$ sont valeurs propres de $f$. Ce sont alors les seules et on a, pour tout $\vec x\in E$,
$$\vec x={\frac{1}{2}}(\vec x+f(\vec x))+{\frac{1}{2}}(\vec x-f(\vec x))$$
Et, quelque soit $\vec x\in E$, $f(\vec x-f(\vec x))=-(\vec x-f(\vec x))$ et $f(\vec x+f(\vec x))=(\vec x+f(\vec x))$, c'est-\`a-dire 
$\vec x+f(\vec x)$ est dans le sous-espace propre associ\'e \`a la valeur propre $1$ et $\vec x-f(\vec x)$ est dans le sous-espace propre associ\'e \`a
la valeur propre $-1$. Par ailleurs on sait que les sous-espaces propres sont en somme directe ( on peut le v\'erifier \'egalement puisque leur intersection
est l'ensemble des vecteurs $\vec x$ tels que $\vec x=-\vec x$, donc r\'eduite au vecteur nul). par cons\'equent $E$ est bien somme directe des sous-espaces
propres correspondants aux valeurs propres $1$ et $-1$.

  \item {\it Traduisons g\'eom\'etriquement le cas $n=2$}.

Rappelons que si il n'y a qu'une valeur propre, $f$ est l'identit\'e ou son oppos\'ee. Dans le cas o\`u $1$ et $-1$ sont valeur propres, leurs sous-espaces
propres sont des droites vectorielles. Soit $u$ un vecteur
propre tel que $f(u)=u$ et $v$ un vecteur propre tel que $f(v)=-v$, alors si $w=au+bv$, $f(w)=au-bv$.
\end {enumerate}
}
}

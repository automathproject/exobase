\uuid{eu7r}
\exo7id{1511}
\titre{exo7 1511}
\auteur{ortiz}
\organisation{exo7}
\datecreate{1999-04-01}
\isIndication{false}
\isCorrection{false}
\chapitre{Espace euclidien, espace normé}
\sousChapitre{Forme quadratique}
\module{Algèbre}
\niveau{L2}
\difficulte{}

\contenu{
\texte{
Soit $q$ la forme quadratique de $\Rr^3$ de
matrice $A=
\left(\begin{smallmatrix}
2 & 1 & 1 \\
1 & 1 & 1 \\
1 & 1 & 2
\end{smallmatrix}\right)
$ dans la base canonique
$\mathcal{B}=(e_1,e_2,e_3)$ de $\Rr^3.$
}
\begin{enumerate}
    \item \question{Donner l'expression analytique de $q$ dans $\mathcal{B}$ et
expliciter sa forme polaire $f$.}
    \item \question{V\'erifier que $\mathcal{B}^{^{\prime }}=(e_1,-\frac
12e_1+e_2,-e_2+e_3)$ est une base $\Rr^3$ et
donner la matrice de $q$ dans cette base.
Expliciter $q$ dans cette base.}
    \item \question{Trouver le rang et la signature de $q$.}
\end{enumerate}
}

\uuid{MiMx}
\exo7id{3669}
\titre{exo7 3669}
\auteur{quercia}
\organisation{exo7}
\datecreate{2010-03-11}
\isIndication{false}
\isCorrection{false}
\chapitre{Espace euclidien, espace normé}
\sousChapitre{Produit scalaire, norme}
\module{Algèbre}
\niveau{L2}
\difficulte{}

\contenu{
\texte{
Soit $E$ un espace vectoriel muni d'un produit scalaire (de dimension éventuellement infinie)
et $(\vec u_1,\dots,\vec u_n)$ une famille orthonormée de $E$.
On note $F = \text{vect}(\vec u_1,\dots,\vec u_n)$.
}
\begin{enumerate}
    \item \question{Démontrer que $F \oplus F^\perp = E$ et $F^{\perp\perp} = F$
    (on utilisera la projection associée aux $\vec u_i$).}
    \item \question{Soit $\vec x \in E$. Démontrer que $\sum_{i=1}^n (\vec x\mid \vec u_i)^2 \le \|\vec x\,\|^2$.
    Quand a-t-on égalité ?}
    \item \question{Application : Soit $f : {[0,2\pi]} \to {\R}$ continue. On appelle
{\it coefficients de Fourier de $f$\/} les réels :
$$ c_k(f) =  \int_{t=0}^{2\pi} f(t)\cos(kt)\,d t   \qquad \text{et}  \qquad
   s_k(f) =  \int_{t=0}^{2\pi} f(t)\sin(kt)\,d t.$$


Démontrer l'inégalité de Bessel :
$ \int_{t=0}^{2\pi} f^2(t)\,d t \ge \frac {c_0(f)^2}{2\pi}
 + \sum_{k=1}^\infty \frac {c_k(f)^2 + s_k(f)^2}\pi$.}
\end{enumerate}
}

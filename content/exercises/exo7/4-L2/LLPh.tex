\uuid{LLPh}
\exo7id{3596}
\titre{exo7 3596}
\auteur{quercia}
\organisation{exo7}
\datecreate{2010-03-10}
\isIndication{false}
\isCorrection{true}
\chapitre{Réduction d'endomorphisme, polynôme annulateur}
\sousChapitre{Applications}
\module{Algèbre}
\niveau{L2}
\difficulte{}

\contenu{
\texte{
En se déplaçant uniquement sur les arêtes d'un cube de côté 1, combien y a-t-il
de chemins de longueur~$n$ pour aller d'un point à un autre~?
}
\reponse{
Soit $d_n(i,j)$ le nombre de chemins de longueur~$n$ allant du sommet
$i$ au sommet $j$. $j$ admet trois voisins $k_1,k_2,k_3$ et l'on a~:
$d_n(i,j) = d_{n-1}(i,k_1) + d_{n-1}(i,k_2) + d_{n-1}(i,k_3)$.
On numérote les sommets de $0$ à $7$ de sorte que les voisins du sommet~$i$
sont les sommets $i+1 \bmod 8$, $i+2 \bmod 8$ et $i+4 \bmod 8$.
Le vecteur $d_n = (d_n(0,0),\dots,d_n(0,7))$ vérifie la relation de récurrence
$d_n = A d_{n-1}$ où $A$ est la matrice suivante ($.$ désigne $0$)~:
$$A = \begin{pmatrix}. &1 &1 &. &1 &. &. &.\cr
               1 &. &. &1 &. &1 &. &.\cr
               1 &. &. &1 &. &. &1 &.\cr
               . &1 &1 &. &. &. &. &1\cr
               1 &. &. &. &. &1 &1 &.\cr
               . &1 &. &. &1 &. &. &1\cr
               . &. &1 &. &1 &. &. &1\cr
               . &. &. &1 &. &1 &1 &.\cr\end{pmatrix}
 = \begin{pmatrix}B &I_4\\ I_4&B\\\end{pmatrix}
 =  P\begin{pmatrix}B+I_4 &0\\ 0&B-I_4\end{pmatrix}P^{-1}$$
avec $$B = \begin{pmatrix}. &1 &1 &.\cr
                   1 &. &. &1\cr
                   1 &. &. &1\cr
                   . &1 &1 &.\cr\end{pmatrix}
\text{ et } P = \begin{pmatrix}I_4 &I_4\\ I_4&-I_4\\ \end{pmatrix}.$$
De même,
$$B\pm I_4 = \begin{pmatrix}C\pm I_2&I_2\\ I_2&C\pm I_2\\\end{pmatrix}
= Q\begin{pmatrix}C\pm I_2+I_2 &0\cr 0&C\pm I_2 - I_2\end{pmatrix}Q^{-1}$$
et enfin,
$$C\pm I_2\pm I_2 = \begin{pmatrix}\pm I_1\pm I_1&I_1\cr I_1&\pm I_1\pm I_1\cr\end{pmatrix}
= R\begin{pmatrix}\pm I_1\pm I_1+I_1 &0\cr 0&\pm I_1\pm I_1 - I_1\end{pmatrix}R^{-1}.$$
Donc $A$ est diagonalisable de valeurs propres $-3,-1,1,3$ et on
peut certainement terminer les calculs pour obtenir $d_n = A^nd_0$.
}
}

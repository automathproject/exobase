\uuid{wWVI}
\exo7id{2599}
\titre{exo7 2599}
\auteur{delaunay}
\organisation{exo7}
\datecreate{2009-05-19}
\isIndication{false}
\isCorrection{true}
\chapitre{Réduction d'endomorphisme, polynôme annulateur}
\sousChapitre{Diagonalisation}
\module{Algèbre}
\niveau{L2}
\difficulte{}

\contenu{
\texte{
Soit $m\in\R$, et $A$ la matrice
$$A=\begin{pmatrix}1+m&1+m&1 \\  -m&-m&-1 \\  m&m-1&0\end{pmatrix}$$
}
\begin{enumerate}
    \item \question{Factoriser le polyn\^ome caract\'eristique de $A$ et montrer que les valeurs propres de $A$ sont $-1$ et $1$.}
\reponse{{\it Factorisons le polyn\^ome caract\'eristique de $A$ et montrons que les valeurs propres de $A$ sont $-1$ et $1$.} (1,5 points)
\begin{align*}
P_A(X)&=\begin{vmatrix}1+m-X&1+m&1 \\  -m&-m-X&-1 \\  m&m-1&-X\end{vmatrix} \\ 
&=\begin{vmatrix}1+m-X&1+m&1 \\  -m&-m-X&-1 \\  0&-X-1&-X-1\end{vmatrix} \\ 
&=\begin{vmatrix}1+m-X&m&1 \\  -m&1-m-X&-1 \\  0&0&-X-1\end{vmatrix} \\ 
&=(-X-1)\begin{vmatrix}1+m-X&m \\  -m&1-m-X\end{vmatrix} \\ 
&=-(1+X)\left((1-X)^2-m^2+m^2\right) \\ 
&=-(1+X)(1-X)^2
\end{align*}
Ainsi, les valeurs propres de la matrice $A$ sont $-1$, valeur propre simple, et $1$, valeur propre double.}
    \item \question{Pour quelles valeurs de $m$ la matrice est-elle diagonalisable ? (justifier). D\'eterminer suivant les valeurs de $m$ le polyn\^ome minimal de $A$ (justifier).}
\reponse{{\it Pour quelles valeurs de $m$ la matrice est-elle diagonalisable ? } (1,5 points)

La matrice $A$ est diagonalisable si et seulement si le sous-espace propre associ\'e \`a la valeur propre $1$ est de dimension $2$. D\'eterminons donc ce sous-espace propre $E_1=\{\vec u=(x,y,z)\in\R^3,\ A\vec u=\vec u\}$.
$$\begin{align*}
A\vec u=\vec u
&\iff\left\{{\begin{align*}mx+(1+m)y+z&=0 \\  -mx-(1+m)y-z&=0 \\  mx+(m-1)y-z&=0\end{align*}} \right. \\ 
&\iff\left\{{\begin{align*}mx+(1+m)y+z&=0 \\  mx+(m-1)y-z&=0\end{align*}}\right. \\ 
&\iff\left\{{\begin{align*}2y+2z&=0 \\  2mx+2my&=0\end{align*}}\right. \\ 
&\iff\left\{{\begin{align*}y+z&=0 \\  m(x+y)&=0\end{align*}}\right. \\ 
\end{align*}$$
Ainsi, l'espace $E_1$ est de dimension $2$ si et seulement si $m=0$, c'est alors le plan d'\'equation $y+z=0$, sinon c'est une droite, intersection des deux plans $y+z=0$ et $x+y=0$.

{\it D\'eterminons suivant les valeurs de $m$ le polyn\^ome minimal de $A$.} (1 point)

Si $m=0$, la matrice $A$ est diagonalisable, son polyn\^ome minimal n'a que des racines simples, il est \'egal \`a 
$$Q(X)=(X-1)(X+1).$$
Si $m\neq 0$, la matrice $A$ n'est pas diagonalisable, son polyn\^ome minimal ne peut pas avoir uniquement des racines simples, il est donc \'egal \`a
$$Q(X)=(X-1)^2(X+1).$$}
\end{enumerate}
}

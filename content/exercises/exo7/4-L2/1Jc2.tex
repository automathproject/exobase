\uuid{1Jc2}
\exo7id{5509}
\titre{exo7 5509}
\auteur{rouget}
\organisation{exo7}
\datecreate{2010-07-15}
\isIndication{false}
\isCorrection{true}
\chapitre{Espace euclidien, espace normé}
\sousChapitre{Projection, symétrie}
\module{Algèbre}
\niveau{L2}
\difficulte{}

\contenu{
\texte{
Dans $\Rr^3$, soient $(D)$ $\left\{
\begin{array}{l}
x+y+z=1\\
x-2y-z=0
\end{array}
\right.$ et $(\Delta)~:~6x=2y=3z$ puis $(P)~:~x+3y+2z=6$. Déterminer la projection de $(D)$ sur $(P)$ parallèlement à $(\Delta)$.
}
\reponse{
Notons $p$ la projection sur $(P)$ parallèlement à $(\Delta)$.
\textbullet~Déterminons un repère de $(D)$. 
\begin{center}$\left\{
\begin{array}{l}
x+y+z=1\\
x-2y-z=0
\end{array}
\right.\Leftrightarrow
\left\{
\begin{array}{l}
y+z=-x+1\\
2y+z=x
\end{array}
\right.
\Leftrightarrow
\left\{
\begin{array}{l}
y=2x-1\\
z=-3x+2
\end{array}
\right.
$
\end{center}
$(D)$ est la droite de repère $(A,\overrightarrow{u})$ où $A(0,-1,2)$ et $\overrightarrow{u}(1,2,-3)$.
\textbullet~$(\Delta)$ est dirigée par le vecteur $\overrightarrow{u'}(1,3,2)$. $\overrightarrow{u}$ n'est pas colinéaire à $\overrightarrow{u'}$ et donc $(D)$ n'est pas parallèle à $(\Delta)$. On en déduit que $p(D)$ est une droite. 
Plus précisément, $p(D)$ est la droite intersection du plan $(P)$ et du plan $(P')$ contenant $(D)$ et parallèle à $(\Delta)$. Déterminons une équation de $(P')$. Un repère de $(P')$ est $(A,\overrightarrow{u},\overrightarrow{u'})$. Donc

\begin{center}
$M(x,y,z)\in(P')\Leftrightarrow\left|
\begin{array}{ccc}
x&1&1\\
y+1&2&3\\
z-2&-3&2
\end{array}
\right|=0\Leftrightarrow 13x-5(y+1)+(z-2)=0\Leftrightarrow13x-5y+z=7$.
\end{center}

Finalement

\begin{center}
\shadowbox{
$p(D)$ est la droite dont un système d'équations cartésiennes est $\left\{
\begin{array}{l}
13x-5y+z=7\\
x+3y+2z=6
\end{array}
\right.$
}
\end{center}
}
}

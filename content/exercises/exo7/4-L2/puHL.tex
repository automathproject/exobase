\uuid{puHL}
\exo7id{7357}
\titre{exo7 7357}
\auteur{mourougane}
\organisation{exo7}
\datecreate{2021-08-10}
\isIndication{false}
\isCorrection{false}
\chapitre{Groupe, anneau, corps}
\sousChapitre{Groupe, sous-groupe}
\module{Algèbre}
\niveau{L2}
\difficulte{}

\contenu{
\texte{

}
\begin{enumerate}
    \item \question{Construire à la règle et au compas un hexagone régulier $E$.}
    \item \question{Une translation de vecteur non nul peut-elle conserver globalement l'hexagone $E$ ?}
    \item \question{Quels sont les centres et les angles des rotations du plan euclidien orienté qui conservent globalement l'hexagone $E$ ?}
    \item \question{Quelles sont les symétries orthogonales du plan euclidien orienté qui conservent globalement l'hexagone $E$ ?}
    \item \question{Quelles sont les symétries orthogonales glissées du plan euclidien orienté qui conservent globalement l'hexagone $E$ ?}
    \item \question{Soit $r$ une rotation d'angle minimal non nul qui conserve globalement l'hexagone $E$ et $s$ une symétrie orthogonale qui conserve globalement $E$. Quel est l'ordre de $r$ ? et celui de $s$ ?}
    \item \question{\'Ecrire $r$ comme composée de $s$ suivie d'une autre symétrie orthogonale.
Déterminer la nature et les éléments caractéristiques (centre, axe, angle...) de l'isométrie $rs$.}
    \item \question{Donner la liste des éléments du groupe $D_{12}$ des isométries du plan euclidien orienté qui conservent globalement l'hexagone $E$. On n'utilisera dans les notations seulement une rotation $r$ et une symétrie $s$.}
    \item \question{\`A l'aide de l'écriture de $r$ comme composée de deux symétries, calculer $sr$.}
    \item \question{En déduire les produits $rsrs$, $rsr^2s$, $r^3sr^4s$.}
\end{enumerate}
}

\uuid{EP63}
\exo7id{2753}
\titre{exo7 2753}
\auteur{tumpach}
\organisation{exo7}
\datecreate{2009-10-25}
\video{d4ZZDotSUcw}
\isIndication{false}
\isCorrection{true}
\chapitre{Déterminant, système linéaire}
\sousChapitre{Calcul de déterminants}
\module{Algèbre}
\niveau{L2}
\difficulte{}

\contenu{
\texte{

}
\begin{enumerate}
    \item \question{Calculer l'aire du parall\'elogramme construit sur les vecteurs 
$\vec{u} = \left(\begin{array}{c}2\\3\end{array}\right)$ et 
$\vec{v} = \left(\begin{array}{c}1\\4\end{array}\right)$.}
\reponse{L'aire $\mathcal{A}$ du parallélogramme construit sur les vecteurs $\vec{u} = \left(\begin{array}{c}a\\c\end{array}\right)$ et 
$\vec{v} = \left(\begin{array}{c}b\\d\end{array}\right)$ est la valeur absolue du déterminant $\begin{vmatrix} a & b \\ c & d \end{vmatrix}$
donc $\mathcal{A} =  |ad-bc|$.
Ici on trouve 
$\mathcal{A} = \text{abs} \begin{vmatrix} 2 & 1 \\ 3 & 4 \end{vmatrix} = + 5 $
où $\text{abs}$ désigne la fonction valeur absolue.}
    \item \question{Calculer le volume du  parall\'el\'epip\`ede construit sur les vecteurs \\
$\vec{u} = \left(\begin{array}{c}1\\2\\0\end{array}\right)$, 
$\vec{v} = \left(\begin{array}{c}0\\1\\3\end{array}\right)$ et 
$\vec{w} = \left(\begin{array}{c}1\\1\\1\end{array}\right)$.}
\reponse{Le volume  du  parall\'el\'epip\`ede construit sur trois vecteurs de $\Rr^3$ est la valeur absolue du déterminant
de la matrice formée des trois vecteurs.
Ici $$\mathcal{V} = \text{abs} \begin{vmatrix} 1 & 0 & 1 \\ 2 & 1 & 1 \\ 0 & 3 & 1\end{vmatrix} 
= \text{abs} \Big(+1 \begin{vmatrix} 1 & 1 \\ 3 & 1\end{vmatrix}  + 1  \begin{vmatrix} 2 & 1 \\ 0 & 3 \end{vmatrix}
\Big)= 4$$
où l'on a développé par rapport à la première ligne.}
    \item \question{Montrer que le volume d'un parall\'el\'epip\`ede dont les sommets sont des points de 
$\Rr^3$ \`a coefficients entiers est un nombre entier.}
\reponse{Si un parall\'el\'epip\`ede est construit sur trois vecteurs de $\Rr^3$
dont les coefficients sont des entiers alors le volume correspond au déterminant d'une matrice à coefficients
entiers. C'est donc un entier.}
\end{enumerate}
}

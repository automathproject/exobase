\uuid{z9kP}
\exo7id{5631}
\titre{exo7 5631}
\auteur{rouget}
\organisation{exo7}
\datecreate{2010-10-16}
\isIndication{false}
\isCorrection{true}
\chapitre{Espace euclidien, espace normé}
\sousChapitre{Produit scalaire, norme}
\module{Algèbre}
\niveau{L2}
\difficulte{}

\contenu{
\texte{
Soit $E$ un $\Kk$-espace vectoriel et $\varphi$ et $\psi$ deux formes linéaires sur $E$. On suppose que pour tout $x$ de $E$, on a  $\varphi(x)\psi(x) = 0$. Montrer que $\varphi= 0$ ou $\psi= 0$.
}
\reponse{
\textbf{1 ère solution.} On utilise le fait qu'une réunion de deux sous-espaces vectoriels est un sous-espace vectoriel si et seulement si l'un des deux contient l'autre. Donc

\begin{center}
$\varphi\psi= 0\Rightarrow\text{Ker}\varphi\cup\text{Ker}\psi = E\Rightarrow\text{Ker}\psi \subset\text{Ker}\varphi=\text{Ker}\varphi\cup\text{Ker}\psi= E\;\text{ou}\;\text{Ker}\varphi\subset\text{Ker}\psi=\text{Ker}\varphi\cup\text{Ker}\psi = E\Rightarrow \varphi= 0\;\text{ou}\;\psi = 0$.
\end{center}

\textbf{2ème solution.} Supposons que $\varphi\psi= 0$ et qu'il existe $x$ et $y$ tels que $\varphi(x)\neq 0$ (et donc $\psi(x) = 0$) et $\psi(y)\neq 0$ (et donc $\varphi(y) = 0$). Alors $0 = \varphi(x+y)\psi(x+y)=(\varphi(x)+\varphi(y))(\psi(x)+\psi(y))=\varphi(x)\psi(y)$ ce qui est une contradiction.

\begin{center}
\shadowbox{
$\forall(\varphi,\psi)\in(E^*)^2,\;(\forall x\in E,\;\varphi(x)\psi(x)=0)\Rightarrow\varphi=0\;\text{ou}\;\psi=0$.
}
\end{center}
}
}

\uuid{ZQ6q}
\exo7id{5311}
\titre{exo7 5311}
\auteur{rouget}
\organisation{exo7}
\datecreate{2010-07-04}
\isIndication{false}
\isCorrection{true}
\chapitre{Arithmétique}
\sousChapitre{Arithmétique de Z}
\module{Algèbre}
\niveau{L2}
\difficulte{}

\contenu{
\texte{
Soit $p$ un nombre premier.
}
\begin{enumerate}
    \item \question{Montrer que, pour tout entier $k$ tel que $1\leq k\leq p-1$, $p$ divise $C_p^k$.}
\reponse{Soit $p$ un nombre premier et $k$ un entier tel que $1\leq k\leq p-1$. On a $kC_p^k=pC_{p-1}^{k-1}$. Donc, $p$ divise $kC_p^k$. Mais, $p$ est premier et donc $p$ est premier à tous les entiers compris entre $1$ et $p-1$ au sens large. D'après le théorème de \textsc{Gauss}, $p$ divise $C_p^k$.}
    \item \question{Montrer que $\forall a\in\Nn^*$, $a^p\equiv a$ $(p)$ (par récurrence sur $a$).}
\reponse{Soit $p$ un nombre premier. Montrons par récurrence que $\forall a\in\Nn^*$, $a^p\equiv a\;(p)$.

C'est clair pour $a=1$.

Soit $a\geq1$. Supposons que $a^p\equiv a\;(p)$. On a alors 

\begin{align*}\ensuremath
(a+1)^p&=\sum_{k=0}^{p}C_p^ka^k=a^p+1+\sum_{k=1}^{p-1}C_p^ka^k\\
 &\equiv a^p+1\;(p)\quad(\mbox{d'après 1)})\\
 &\equiv a+1\;(p)\quad(\mbox{par hypothèse de récurrence})
\end{align*}

On a montré par récurrence que $\forall a\in\Nn^*,\;a^p\equiv a\;(p)$.}
\end{enumerate}
}

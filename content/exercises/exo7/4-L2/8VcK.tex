\uuid{8VcK}
\exo7id{1691}
\titre{exo7 1691}
\auteur{ortiz}
\organisation{exo7}
\datecreate{1999-04-01}
\isIndication{false}
\isCorrection{false}
\chapitre{Réduction d'endomorphisme, polynôme annulateur}
\sousChapitre{Trigonalisation}
\module{Algèbre}
\niveau{L2}
\difficulte{}

\contenu{
\texte{
Les questions sont ind\'ependantes. $K$ d\'esigne
$\Rr$ ou $\Cc$, $E$ est un $K$-espace vectoriel de
dimension finie $n$, $\mathcal{B}=(e_1,...,e_n)$
est une base fix\'ee de $E$ et $f$. un
endomorphisme de $E$.
}
\begin{enumerate}
    \item \question{Donner un exemple de matrice de $M_2(K)$ non trigonalisable.}
    \item \question{Donner un exemple de matrice de $M_n(K)$ \`a la fois non
diagonalisable et trigonalisable.}
    \item \question{D\'eterminer sans calculs les valeurs propres complexes de $f$ s
i sa matrice dans $\mathcal{B}$ est
$M=\left(\begin{smallmatrix}
1&0&1\\0&1&0\\1&0&1\end{smallmatrix}\right)$.}
    \item \question{On suppose que $n=3$ et que la matrice de $f$ dans la base $\mathcal{B}$
est $M=\left(\begin{smallmatrix}
3&2&4\\-1&3&-1\\-2&-1&-3\end{smallmatrix}\right)$.
Montrer que le plan d'\'equation $x+2z=0$ est
stable par $f.$}
    \item \question{Que peut-on dire d'un vecteur g\'en\'erateur d'une droite stable par $f$ ?}
    \item \question{Montrer que si l'endomorphisme $f$ est trigonalisable alors il admet
au moins un sous-espace vectoriel stable par $f$
et de dimension $k\in[ 0,n]$ fix\'ee.}
\end{enumerate}
}

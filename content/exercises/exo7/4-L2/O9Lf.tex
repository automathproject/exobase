\uuid{O9Lf}
\exo7id{2575}
\titre{exo7 2575}
\auteur{delaunay}
\organisation{exo7}
\datecreate{2009-05-19}
\isIndication{false}
\isCorrection{true}
\chapitre{Réduction d'endomorphisme, polynôme annulateur}
\sousChapitre{Diagonalisation}
\module{Algèbre}
\niveau{L2}
\difficulte{}

\contenu{
\texte{
({\it 4 points}) On suppose qu'une population $x$ de lapins et une population $y$ de loups sont gouvern\'ees par le syst\`eme suivant d'\'equations diff\'erentielles :
$$(S)\ \ \left\{ \begin{align*}x'&= 4x-2y \\  y'&= x+y\end{align*}\right.$$
\begin {enumerate}
  \item  Diagonaliser la matrice 
$$A=\begin{pmatrix}4&-2 \\ 1&1 \\ \end{pmatrix}.$$
  \item Exprimer le syst\`eme $(S)$ et ses solutions dans une base de vecteurs propres de $A$.
  \item  Repr\'esenter graphiquement les trajectoires de $(S)$ dans le rep\`ere $(Oxy)$.
  \item Discuter graphiquement l'\'evolution de la population des lapins en fonction des conditions initiales.
\end {enumerate}
}
\reponse{
{\it On suppose qu'une population $x$ de lapins et une population $y$ de loups sont gouvern\'ees par le syst\`eme suivant d'\'equations diff\'erentielles} :
$$(S)\ \ \left\{\begin{align*}x'&= 4x-2y \\  y'&= x+y\end{align*}\right.$$
\begin {enumerate}
  \item On diagonalise la matrice 
$$A=\begin{pmatrix}4&-2 \\ 1&1 \\ \end{pmatrix}.$$
Pour cela on d\'etermine ses valeurs propres :
$$\det(A-\lambda I)=\begin{vmatrix}4-\lambda &-2 \\ 1&1-\lambda \\ \end{vmatrix}=(4-\lambda)(1-\lambda)+2=\lambda^2-5\lambda+6=(\lambda-2)(\lambda-3).$$
Ainsi, la matrice $A$ admet deux valeurs propres distinctes, qui sont $\lambda_1=2$ et $\lambda_2=3$. Elle est diagonalisable. D\'eterminons une base de vecteurs propres :
$$\begin{pmatrix}4&-2 \\ 1&1 \\ \end{pmatrix}\begin{pmatrix}x \\  y\end{pmatrix}=
\begin{pmatrix}2x \\  2y\end{pmatrix}\iff x=y,$$
d'o\`u le vecteur propre $u_1=(1,1)$ associ\'e \`a la valeur propre $\lambda_1=2$ . 
$$\begin{pmatrix}4&-2 \\ 1&1 \\ \end{pmatrix}\begin{pmatrix}x \\  y\end{pmatrix}=
\begin{pmatrix}3x \\  3y\end{pmatrix}\iff x=2y,$$
d'o\`u le vecteur propre $u_2=(2,1)$ associ\'e \`a la valeur propre $\lambda_2=3$ .
Dans la base $(u_1, u_2)$, la matrice s'\'ecrit
$$A'=\begin{pmatrix}2&0 \\ 0&3 \\ \end{pmatrix}.$$
On a $A=PA'P^{-1}$ o\`u
$$P=\begin{pmatrix}1&2 \\ 1&1 \\  \end{pmatrix}\ {\hbox{et}}\ P^{-1}=\begin{pmatrix}-1&2 \\ 1&-1 \\ \end{pmatrix}.$$
  \item Exprimons le syst\`eme $(S)$ et ses solutions dans une base de vecteurs propres de $A$.

Dans la base $(u_1, u_2)$, le syst\`eme $(S)$ devient
$$(S')\ \ \left\{\begin{align*}X'&= 2X \\  Y'&= 3Y\end{align*}\right.$$
Ses solutions sont les fonctions
$$\left\{\begin{align*}X(t)&= X(0)e^{2t} \\  Y(t)&=Y(0)e^{3t}\end{align*}\right.$$

  \item  Pour repr\'esenter graphiquement les trajectoires de $(S)$ dans le rep\`ere $(Oxy)$, on trace d'abord le rep\`ere $(O, u_1,u_2)$ dans le rep\`ere $(0xy)$, puis, on trace les courbes $$Y={\frac{Y(0)}{X(0)}}X^{3/2}$$
dans le rep\`ere $(O, u_1,u_2)$ (ou $OXY$).

  \item On voit sur le dessin que si $Y(0)$ est strictement positif, alors la population des lapins, $x(t)$ tend vers $+\infty$ quand $t$ tend vers $+\infty$. Si $Y(0)$ est strictement n\'egatif alors la populations des lapins s'\'eteint dans la mesure ou $x(t)$ dans ce cas tendrait vers $-\infty$.
\end {enumerate}
}
}

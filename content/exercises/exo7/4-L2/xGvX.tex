\uuid{xGvX}
\exo7id{1626}
\titre{exo7 1626}
\auteur{barraud}
\organisation{exo7}
\datecreate{2003-09-01}
\isIndication{false}
\isCorrection{false}
\chapitre{Réduction d'endomorphisme, polynôme annulateur}
\sousChapitre{Diagonalisation}
\module{Algèbre}
\niveau{L2}
\difficulte{}

\contenu{
\texte{
Soit $A_{t}$ la matrice
 $
 A_{t}=
 \begin{pmatrix}
     t   &    1   & \cdots &    1   \\
     1   &    t   & \ddots & \vdots \\
  \vdots & \ddots & \ddots &    1   \\
     1   & \cdots &    1   &    t
 \end{pmatrix}
 $.
Sans calculer le polyn\^{o}me caract\'{e}ristique de $A_{t}$, montrer que $(t-1)$ est valeur
propre. D\'{e}terminer l'espace propre associ\'{e}. Que dire de la multiplicit\'{e} de la valeur
propre $(t-1)$ ? En d\'{e}duire le spectre de $A_{t}$. $A_{t}$ est-elle diagonalisable ?
}
}

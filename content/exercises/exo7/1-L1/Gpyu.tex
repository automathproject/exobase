\uuid{Gpyu}
\exo7id{934}
\titre{exo7 934}
\auteur{cousquer}
\organisation{exo7}
\datecreate{2003-10-01}
\video{pNH1v_tXHvg}
\isIndication{true}
\isCorrection{true}
\chapitre{Application linéaire}
\sousChapitre{Image et noyau, théorème du rang}
\module{Algèbre}
\niveau{L1}
\difficulte{}

\contenu{
\texte{
Soit $E$ un espace vectoriel et soient $E_1$ et $E_2$ deux sous-espaces vectoriels de dimension finie
de $E$, on d\'efinit l'application $f\colon E_1\times E_2 \to E$ par $f(x_1,x_2)=x_1+x_2$.
}
\begin{enumerate}
    \item \question{Montrer que $f$ est lin\'eaire.}
\reponse{Aucun problème...}
    \item \question{D\'eterminer le noyau et l'image de $f$.}
\reponse{Par d\'efinition de $f$ et de ce qu'est la somme de deux sous-espaces vectoriels, l'image est
$$\Im f =  \{ f(x_1,x_2) \mid x_1 \in E_1, x_2\in E_2 \} = \{ x_1+x_2 \mid x_1 \in E_1, x_2\in E_2 \} = E_1 + E_2.$$

Pour le noyau :
$$\ker f = \{ (x_1,x_2) \mid f(x_1,x_2)=0 \} = \{ (x_1,x_2) \mid x_1+x_2=0 \}$$

Mais on peut aller un peu plus loin. En effet un \'el\'ement $(x_1,x_2) \in \ker f$,
v\'erifie $x_1\in E_1$, $x_2\in E_2$ et $x_1=-x_2$. Donc $x_1 \in E_2$. Donc $x_1\in E_1 \cap E_2$.
R\'eciproquement si $x\in E_1 \cap E_2$, alors $(x,-x)\in \ker f$.
Donc 
$$\ker f = \{ (x,-x) \mid x \in  E_1 \cap E_2 \}. $$
De plus l'application $x \mapsto (x,-x)$ montre que $\ker f$ est isomorphe \`a $E_1 \cap E_2$.}
    \item \question{Que donne le th\'eor\`eme du rang ?}
\reponse{Le th\'eor\`eme du rang s'\'ecrit :
$$\dim \ker f+ \dim \Im f = \dim (E_1\times E_2).$$
Compte tenu de  l'isomorphisme entre $\ker f$ et $E_1 \cap E_2$ on obtient :
$$\dim (E_1 \cap E_2) + \dim (E_1+E_2) = \dim (E_1\times E_2).$$
Mais $\dim (E_1\times E_2) = \dim E_1 + \dim E_2$, donc on retrouve ce que l'on appelle le th\'eor\`eme des quatre dimensions :
$$\dim (E_1+E_2) = \dim E_1+\dim E_2-\dim (E_1 \cap E_2).$$}
\indication{Faire un dessin de l'image et du noyau pour $f: \Rr\times \Rr \longrightarrow \Rr$.
Montrer que le noyau est isomorphe à $E_1 \cap E_2$.}
\end{enumerate}
}

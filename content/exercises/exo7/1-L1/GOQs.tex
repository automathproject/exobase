\uuid{GOQs}
\exo7id{5336}
\titre{exo7 5336}
\auteur{rouget}
\organisation{exo7}
\datecreate{2010-07-04}
\isIndication{false}
\isCorrection{true}
\chapitre{Polynôme, fraction rationnelle}
\sousChapitre{Fraction rationnelle}
\module{Algèbre}
\niveau{L1}
\difficulte{}

\contenu{
\texte{
Décomposer en éléments simples dans $C(X)$ les fractions rationnelles suivantes

$$\begin{array}{lll}
1)\;\frac{1}{X^n-1}&2)\;\frac{1}{(X-1)(X^n-1)}&3)\;\frac{n!}{(X-1)(X-2)...(X-n)}\\
4)\;\frac{X^2}{X^4-2X^2\cos(2a)+1}&5)\;\frac{1}{X^{2n}+1}.
\end{array}$$
}
\reponse{
Soit $P=X^n-1$ et $F=\frac{1}{P}$. La partie entière de $F$ est nulle et les pôles de F sont simples (car $P=X^n-1$ et $P'=nX^{n-1}$ n'ont pas de racines communes dans $\Cc$). De plus, $P=\prod_{k=0}^{n-1}(X-\omega_k)$ où $\omega_k=e^{2ik\pi/n}$. Donc, $F=\sum_{k=0}^{n-1}\frac{\lambda_k}{X-\omega_k}$ où 

$$\lambda_k=\frac{1}{P'(\omega_k)}=\frac{1}{n\omega_k^{n-1}}=\frac{\omega_k}{n\omega_k^n}=\frac{\omega_k}{n}.$$

Ainsi,

$$\frac{1}{X^n-1}=\frac{1}{n}\sum_{k=0}^{n-1}\frac{e^{2ik\pi/n}}{X-e^{2ik\pi/n}}.$$
Soit $P=(X-1)(X^n-1)=(X-1)^2\prod_{k=1}^{n-1}\omega_k$ où $\omega_k=e^{2ik\pi/n}$. Soit $F=\frac{1}{P}$. La partie entière de $F$ est nulle. D'autre part, F admet un pôle double, à savoir $1$ et $n-1$ pôles simples à savoir les $\omega_k=e^{2ik\pi/n}$, $\leq k\leq n-1$. Donc, 

$$F=\frac{a}{X-1}+\frac{b}{(X-1)^2}+\sum_{k=1}^{n-1}\frac{\lambda_k}{X-\omega_k}.$$ 

$\lambda_k=\frac{1}{(n+1)\omega_k^n-n\omega_k^{n-1}-1}=\frac{1}{n(1-\omega_k^{n-1})}=\frac{\omega_k}{n(\omega_k-1)}$. Ensuite, 

$$b=\lim_{x\rightarrow 1}(x-1)^2F(x)=\frac{1}{1^{n-1}+...+1^1+1}=\frac{1}{n}.$$

Il reste à calculer $a$.

\begin{align*}\ensuremath
F-\frac{1}{n(X-1)^2}&=\frac{n-(X^{n-1}+X^{n-2}+...+X+1)}{n(X-1)^2(X^{n-1}+...+X+1)}
=\frac{-X^{n-2}-2X^{n-3}-...-(n-2)X-(n-1))}{n(X-1)(X^{n-1}+...+X+1)}.
\end{align*}

Donc, $a=\lim_{x\rightarrow 1}(x-1)(F(x)-\frac{1}{n(x-1)^2})=\frac{-[(n-1)+(n-2)+...+2+1]}{n(1+1...+1)}=-\frac{n-1}{2n}$.

Finalement,

$$F=\frac{1}{n}(-\frac{n-1}{2n(X-1)}+\frac{1}{(X-1)^2}+\sum_{k=1}^{n-1}\frac{\omega_k}{\omega_k-1}\frac{1}{X-\omega_k}).$$
$\frac{n!}{(X-1)...(X-n)}=\sum_{k=1}^{n}\frac{\lambda_k}{X-k}$ avec

$$\lambda_k=\lim_{x\rightarrow k}(x-k)F(x)=\frac{n!}{\prod_{j\neq k}^{}(j-k)}=\frac{n!}{(-1)^{n-k}(k-1)!(n-k)!}=n(-1)^{n-k}C_{n-1}^{k-1}.$$

Donc,

$$\frac{n!}{(X-1)...(X-n)}=\sum_{k=1}^{n}\frac{(-1)^{n-k}nC_{n-1}^{k-1}}{X-k}.$$
Posons $P=X^4-2X^2\cos(2a)+1$.

\begin{align*}\ensuremath
X^4-2X^2\cos(2a)+1&=(X^2-e^{2ia})(X^2-e^{-2ia})=(X-e^{ia})(X-e^{-ia})(X+e^{ia})(X+e^{-ia})\\
 &(=(X^2-2X\cos a+1)(X^2+2X\cos a+1)).
\end{align*}

$P$ est à racines simples si et seulement si $e^{ia}\neq\pm e^{-ia}$ ce qui équivaut à $a\notin\frac{\pi}{2}\Zz$.

\begin{itemize}
[1er cas.] Si $a\in\pi\Zz$,

$$F=\frac{X^2}{(X^2-1)^2}=\frac{a}{X-1}+\frac{b}{(X-1)^2}-\frac{a}{X+1}+\frac{b}{(X+1)^2}.$$

$b=\lim_{x\rightarrow 1}(x-1)^2F(x)=\frac{1}{4}$ puis $x=0$ fournit $0=-2a+2b$ et donc $a=b=\frac{1}{4}$.

$$F=\frac{X^2}{(X^2-1)^2}=\frac{1}{4}(\frac{1}{X-1}+\frac{1}{(X-1)^2}-\frac{1}{X+1}+\frac{1}{(X+1)^2}).$$
[2ème cas.] Si $a\in\frac{\pi}{2}+\pi\Zz$,

$$F=\frac{X^2}{(X^2+1)^2}=\frac{a}{X-i}+\frac{b}{(X-i)^2}-\frac{a}{X+i}+\frac{b}{(X+i)^2}.$$

$b=\lim_{x\rightarrow i}(x-i)^2F(x)=\frac{i^2}{(i+i)^2}=\frac{1}{4}$ puis $x=0$ fournit $0=2ia-2b$ et donc $a=-ib=-\frac{i}{4}$.

$$F=\frac{X^2}{(X^2+1)^2}=\frac{1}{4}(-\frac{i}{X-i}+\frac{1}{(X-i)^2}+\frac{i}{X+i}+\frac{1}{(X+i)^2}).$$
[3ème cas.] Si $a\notin\frac{\pi}{2}\Zz$, puisque $F$ est réelle et paire,

$$F=\frac{A}{X-e^{ia}}+\frac{\overline{A}}{X-e^{-ia}}-\frac{A}{X+e^{ia}}-\frac{\overline{A}}{X+e^{-ia}},$$

avec 

$$A=\frac{e^{2ia}}{(e^{ia}-e^{-ia})(e^{ia}+e^{ia})(e^{ia}+e^{-ia})}=\frac{e^{2ia}}{8i\sin a\cos ae^{ia}}=\frac{-ie^{ia}}{4\sin(2a)}.$$

Donc,

$$F=\frac{1}{4\sin(2a)}(-\frac{ie^{ia}}{X-e^{ia}}+\frac{ie^{-ia}}{X-e^{-ia}}+\frac{ie^{ia}}{X+e^{ia}}+\frac{ie^{-ia}}{X+e^{-ia}}).$$

\end{itemize}
Le polynôme $X^{2n}+1=\prod_{k=0}^{2n-1}(X-e^{i(\frac{\pi}{2n}+\frac{2k\pi}{2n})})$ est à racines simples car n'a pas de racine commune avec sa dérivée. En posant $\omega_k=e^{i(\frac{\pi}{2n}+\frac{2k\pi}{2n})}$, on a

$$\frac{1}{X^{2n}+1}=\sum_{k=0}^{2n-1}\frac{\lambda_k}{X-\omega_k},$$

où 

$$\lambda_k=\frac{1}{2n\omega_k^{2n-1}}=\frac{\omega_k}{2n\omega_k^{2n}}=-\frac{\omega_k}{2n}.$$

Finalement,

$$\frac{1}{X^{2n}+1}=-\frac{1}{2n}\sum_{k=0}^{2n-1}\frac{\omega_k}{X-\omega_k}.$$
}
}

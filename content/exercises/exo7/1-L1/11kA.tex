\uuid{11kA}
\exo7id{5626}
\titre{exo7 5626}
\auteur{rouget}
\organisation{exo7}
\datecreate{2010-10-16}
\isIndication{false}
\isCorrection{true}
\chapitre{Matrice}
\sousChapitre{Changement de base, matrice de passage}
\module{Algèbre}
\niveau{L1}
\difficulte{}

\contenu{
\texte{
Soient $M(a)=\left(
\begin{array}{ccc}
4-a&1&-1\\
-6&-1-a&2\\
2&1&1-a
\end{array}
\right)$  et $N(a)=\left(
\begin{array}{ccc}
1-a&1&0\\
0&1-a&0\\
0&0&2-a
\end{array}
\right)$.
$M(a)$ et $N(a)$ sont-elles semblables ?
}
\reponse{
Si $M(a)$ et $N(a)$ sont semblables alors nécessairement $\text{Tr}(M(a))=\text{Tr}(N(a))$. Or, pour tout scalaire $a$, $\text{Tr}(M(a))=4-3a =\text{Tr}(N(a))$. La trace ne fournit aucun renseignement.

On doit aussi avoir $\text{det}(M(a))=\text{det}(N(a))$. Or, $\text{det}(N(a))=(1-a)^2(2-a)$ et 

\begin{align*}\ensuremath
\text{det}(M(a))&= (4-a)(a^2-1-2)+6(1-a+1)+2(2-1-a) = (4-a)(a^2-3)+14-8a=-a^3+4a^2-5a+2\\
 &= (a-1)^2(2-a)=\text{det}(N(a)).
\end{align*}

Le déterminant ne fournit aucun renseignement.

Soit $f$ l'endomorphisme de $\Kk^3$ de matrice $M(a)$ dans la base canonique $\mathcal{B}_0=(i,j,k)$ de $\Kk^3$.

Le problème posé équivaut à l'existence d'une base $\mathcal{B}=(e_1,e_2,e_3)$ de $\Kk^3$ telle que 
$f(e_1)=(1-a)e_1$, $f(e_2) =(1-a)e_2 +e_1$ et $f(e_3)= (2-a)e_3$. Soit $(x,y,z)$ un élément de $\Kk^3$.

\textbullet~$f((x,y,z)) =(1-a)(x,y,z)\Leftrightarrow\left\{
\begin{array}{l}
3x+y-z=0\\
-6x-2y+2z=0\\
2x+y=0
\end{array}
\right.\Leftrightarrow\left\{
\begin{array}{l}
y=-2x\\
z=x
\end{array}
\right.$. On peut prendre $e_1=(1,-2,1)$.

\textbullet~$f((x,y,z)) =(1-a)(x,y,z)+(1,-2,1)\Leftrightarrow\left\{
\begin{array}{l}
3x+y-z=1\\
-6x-2y+2z=-2\\
2x+y=1
\end{array}
\right.\Leftrightarrow\left\{
\begin{array}{l}
y=-2x-1\\
z=x-2
\end{array}
\right.$. On peut prendre $e_2=(0,-1,-2)$.

\textbullet~$f((x,y,z)) =(2-a)(x,y,z)\Leftrightarrow\left\{
\begin{array}{l}
2x+y-z=0\\
-6x-3y+2z=0\\
2x+y-z=0
\end{array}
\right.\Leftrightarrow\left\{
\begin{array}{l}
y=-2x\\
z=0
\end{array}
\right.$. On peut prendre $e_3=(1,-2,0)$.

La matrice de la famille $\mathcal{B}=(e_1,e_2,e_3)$ dans la base $\mathcal{B}_0$ est $P=\left(
\begin{array}{ccc}
1&0&1\\
-2&-1&-2\\
1&-2&0
\end{array}
\right)$. $\text{det}P=-4+4+1 =1\neq 0$ et donc la famille $\mathcal{B}=(e_1,e_2,e_3)$ est une base de $\Kk^3$. Puisque $\text{Mat}_{\mathcal{B}_0}f=
M(a)$ et $\text{Mat}_{\mathcal{B}}f=N(a)$, les matrices $M(a)$ et $N(a)$ sont semblables.
}
}

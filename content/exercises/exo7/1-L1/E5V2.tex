\uuid{E5V2}
\exo7id{1080}
\titre{exo7 1080}
\auteur{monthub}
\organisation{exo7}
\datecreate{2001-11-01}
\isIndication{false}
\isCorrection{true}
\chapitre{Matrice}
\sousChapitre{Propriétés élémentaires, généralités}
\module{Algèbre}
\niveau{L1}
\difficulte{}

\contenu{
\texte{
Soient $(x_n)_{n \in \N}$ et  $(y_n)_{n \in \N}$
deux suites réelles, vérifiant la relation de récurrence linéaire
suivante :
$$\Big\{
  \begin{array}{rccc}
x_{n+1}&=&-9 x_n &-18 y_n \\
y_{n+1}&=&6 x_n &+ 12y_n\\
  \end{array}$$
avec $x_0=-137$ et $y_0=18$. On se propose dans ce problème de
trouver les termes généraux de ces deux suites.
}
\begin{enumerate}
    \item \question{Montrer qu'il existe une matrice $A \in M_2(\R)$ telle que la relation
  de récurrence linéaire ci-dessus soit équivalente à la relation
  $U_{n+1}=AU_n$, où $U_n=  \begin{pmatrix}    x_n\\y_n  \end{pmatrix}$.}
\reponse{$A=
  \begin{pmatrix}
    -9&-18\\6&12
  \end{pmatrix}$}
    \item \question{Trouver une expression de $U_n$ en fonction de $A$ et de $U_0$.}
\reponse{$U_n=A^nU_0$}
    \item \question{Trouver le noyau de $A$, et en donner une base $B_1$. Calculer le rang de $A$.}
\reponse{C'est la droite engendrée par $
  \begin{pmatrix}
    -2\\1
  \end{pmatrix}$. Le rang est 1.}
    \item \question{Montrer que l'ensemble des vecteurs $X \in \R^2$ tels que $AX=3X$ est un
  sous-espace vectoriel de $\R^2$. Quelle
  est sa dimension ? En donner une base, qu'on notera $B_2$.}
\reponse{C'est la droite engendrée par $
  \begin{pmatrix}
    -3\\2
  \end{pmatrix}$.}
    \item \question{Montrer que la réunion $B_1 \cup B_2$ forme une base $B$ de $\R^2$. Soit
  $P$ la matrice formée des composantes des vecteurs de $B$ relativement à la
  base canonique de $\R^2$. Montrer que $P$ est inversible, et que  le produit
  $P^{-1}AP$ est une matrice diagonale $D$ qu'on calculera.}
\reponse{Ce sont deux vecteurs non colinéaires. On a
$$P^{-1}AP=D=
\begin{pmatrix}
  3&0\\0&0
\end{pmatrix}$$}
    \item \question{Montrer que $A^n=PD^nP^{-1}$. Calculer $D^n$, et en déduire $A^n$, pour tout $n \in \N$.}
\reponse{On a $A=PDP^{-1}$ donc $A^n=PD^nP^{-1}=
  \begin{pmatrix}
   -3^{n+1} & -2\cdot 3^{n+1} \\
2\cdot 3^n & 4\cdot 3^n \\
  \end{pmatrix}
$}
    \item \question{Donner les termes généraux  $x_n$ et $y_n$.}
\reponse{Donc
$$\Big\{
\begin{array}{rcc}
x_n &=& -137\cdot 3^{n+1}-36\cdot 3^{n+1}\\
y_n &=& 274(3^n)+72\cdot 3^n
\end{array}$$}
\end{enumerate}
}

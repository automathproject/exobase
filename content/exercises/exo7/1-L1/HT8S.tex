\uuid{HT8S}
\exo7id{1063}
\titre{exo7 1063}
\auteur{ridde}
\organisation{exo7}
\datecreate{1999-11-01}
\video{-ew4tp-RHJY}
\isIndication{true}
\isCorrection{true}
\chapitre{Matrice}
\sousChapitre{Propriétés élémentaires, généralités}
\module{Algèbre}
\niveau{L1}
\difficulte{}

\contenu{
\texte{
Soient $A$ et $B \in \mathcal{M}_n(\Rr)$ telles que $\forall X \in \mathcal{M}_n (\Rr)$,
$\text{tr} (AX) = \text{tr} (BX)$. Montrer que $A = B$.
}
\indication{Essayer avec $X$ la matrice élémentaire $E_{ij}$ (des zéros partout sauf le coefficient $1$ à la
 $i$-ème ligne et la $j$-ème colonne).}
\reponse{
Notons $E_{ij}$ la matrice élémentaire (des zéros partout sauf le coefficient $1$ à la
 $i$-ème ligne et la $j$-ème colonne).

Soit $A = (a_{ij})  \in \mathcal{M}_n(\Rr)$.
Alors 
$$A \times E_{ij} = 
\begin{pmatrix}  
0 & 0 & \cdots & 0 & a_{1i} & 0 & \cdots \\
0 & 0 & \cdots & 0 & a_{2i} & 0 & \cdots \\
\vdots&  & \cdots & & \vdots &  & \cdots \\
0 & 0 & \cdots & 0 & a_{ji} & 0 & \cdots \\
\vdots&  & \cdots & & \vdots &  & \cdots \\
0 & 0 & \cdots & 0 & a_{ni} & 0 & \cdots \\
\end{pmatrix}$$
La seule colonne non nulle est la $j$-ème colonne.

La trace est la somme des éléments sur la diagonale. Ici le seul élément non nul de la diagonale est
$a_{ji}$, on en déduit donc 
$$\text{tr} (A \times E_{ij})=a_{ji}$$
(attention à l'inversion des indices).

Maintenant prenons deux matrices $A, B$ telles que $\text{tr} (AX) = \text{tr} (BX)$
pour toute matrice $X$. Alors pour $X=E_{ij}$ on en déduit $a_{ji}=b_{ji}$.
On fait ceci pour toutes les matrices élémentaires $E_{ij}$ avec $1 \le i,j \le n$
ce qui implique $A=B$.
}
}

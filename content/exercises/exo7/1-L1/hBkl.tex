\uuid{hBkl}
\exo7id{5575}
\titre{exo7 5575}
\auteur{rouget}
\organisation{exo7}
\datecreate{2010-10-16}
\isIndication{false}
\isCorrection{true}
\chapitre{Espace vectoriel}
\sousChapitre{Dimension}
\module{Algèbre}
\niveau{L1}
\difficulte{}

\contenu{
\texte{
Soient $F$ et $G$ deux sous-espaces vectoriels d'un espace vectoriel de dimension finie sur $\Kk$.

Démontrer que $\text{dim}(F+G)=\text{dim}F+\text{dim}G-\text{dim}(F\cap G)$.
}
\reponse{
Soit $f$ l'application de $F\times G$ dans $E$ qui à un élément $(x,y)$ de $F\times G$ associe $x+y$.

$f$ est clairement linéaire et d'après le thèorème du rang 

\begin{center}
$\text{dim}(F\times G)=\text{dim}(\text{Ker}f)+\text{dim}(\text{Im}f)$ avec $\text{dim}(FxG)=\text{dim}F +\text{dim}G$ et $\text{dim}(\text{Im}f)=\text{dim}(F+G)$.
\end{center}

Il reste à analyser $\text{Ker}f$.

Soit $(x,y)\in E^2$. $(x,y)$ est élément de $\text{Ker}f$ si et seulement si $x$ est dans $F$, $y$ est dans $G$ et $x+y=0$ ou encore si et seulement si $x$ et $y$ sont dans $F\cap G$ et $y=-x$. Donc $\text{Ker}f=\{(x,-x),\;x\in F\cap G\}$.

Montrons enfin que $\text{Ker}f$ est isomorphe à $F\cap G$. Soit $\varphi$ l'application de $F\cap G$ dans $\text{Ker}f$ qui à l'élément $x$ de $F\cap G$ associe $(x,-x)$ dans $\text{Ker}f$. $\varphi$ est clairement une application linéaire, clairement injective et clairement surjective. Donc $\varphi$ est un isomorphisme de $F\cap G$ sur $\text{Ker}f$ et en particulier $\text{dim}(\text{Ker}f)=\text{dim}(F\cap G)$. Finalement

\begin{center}
\shadowbox{
$\text{dim}(F+G)=\text{dim}F+\text{dim}G-\text{dim}(F\cap G)$.
}
\end{center}
}
}

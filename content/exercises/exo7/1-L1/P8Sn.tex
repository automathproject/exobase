\uuid{P8Sn}
\exo7id{1084}
\titre{exo7 1084}
\auteur{legall}
\organisation{exo7}
\datecreate{1998-09-01}
\isIndication{false}
\isCorrection{false}
\chapitre{Matrice}
\sousChapitre{Matrice et application linéaire}
\module{Algèbre}
\niveau{L1}
\difficulte{}

\contenu{
\texte{
Soit $ f : \C \rightarrow \C  $
l'application $ z \mapsto e^{i\theta }\bar{z} .$ On consid\`ere $ \C $ comme
un $\R $-espace vectoriel et on fixe la base $ \epsilon =\{ 1,i\} .$
}
\begin{enumerate}
    \item \question{Montrer que $ f $ est $\R $-lin\'eaire.}
    \item \question{Calculer $ A=\hbox{Mat}(f, \epsilon , \epsilon ) .$}
    \item \question{Existent-ils $ x $ et $ y \in \C -\{ 0\}  $ tels que $
f(x)=x $ et $ f(y) =-y ?$ Si c'est le cas d\'eterminer un tel $ x $ et un tel $ y  .$}
    \item \question{D\'ecrire g\'eom\'etriquement $ f .$}
    \item \question{Soit $ g : \C \rightarrow \C  $
l'application $ z \mapsto e^{i\rho }\bar{z} .$ Calculer
$ A=\hbox{Mat}(g\circ f, \epsilon , \epsilon ) $ et d\'ecrire
 g\'eom\'etriquement $ g\circ f .$}
\end{enumerate}
}

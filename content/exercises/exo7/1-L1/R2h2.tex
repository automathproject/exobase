\uuid{R2h2}
\exo7id{5344}
\titre{exo7 5344}
\auteur{rouget}
\organisation{exo7}
\datecreate{2010-07-04}
\isIndication{false}
\isCorrection{true}
\chapitre{Polynôme, fraction rationnelle}
\sousChapitre{Autre}
\module{Algèbre}
\niveau{L1}
\difficulte{}

\contenu{
\texte{
Résoudre dans $\Cc$ les équations suivantes~:
}
\begin{enumerate}
    \item \question{$z^4+2z^3+3z^2+2z+1=0$ en posant $Z=z+\frac{1}{z}$ (ou autrement).}
\reponse{\begin{align*}\ensuremath
P&=X^4+2X^3+3X^2+2X+1=X^2(X^2+\frac{1}{X^2}+2(X+\frac{1}{X})+3)=X^2((X+\frac{1}{X})^2+2(X+\frac{1}{X})+1)\\
 &=X^2(X+\frac{1}{X}+1)^2=(X^2+X+1)^2=(X-j)^2(X-j^2)^2.
\end{align*}}
    \item \question{$z^6-5z^5+5z^4-5z^2+5z-1=0$.}
\reponse{$1$ et $-1$ sont racines de $P$. On écrit donc $P=(X^2-1)(X^4-5X^3+6X^2-5X+1)$ puis

\begin{align*}\ensuremath
X^4-5X^3+6X^2-5X+1&=X^2((X^2+\frac{1}{X^2})-5(X+\frac{1}{X})+6)=X^2((X+\frac{1}{X})^2-5(X+\frac{1}{X})+4)\\
 &=X^2(X+\frac{1}{X}-1)(X+\frac{1}{X}-4)=(X^2-X+1)(X^2-4X+1)
\end{align*}

et donc, $P=(X-1)(X+1)(X+j)(X+j^2)(X-2+\sqrt{3})(X-2-\sqrt{3})$.}
    \item \question{$z^7-z^6-7z^5+7z^4+7z^3-7z^2-z+1=0$.}
\reponse{\begin{align*}\ensuremath
P&=X^7-X^6-7X^5+7X^4+7X^3-7X^2-X+1=(X^2-1)(X^5-X^4-6X^3+6X^2+X-1)\\
 &=(X-1)^2(X+1)(X^4-6X^2+1)\\
 &=(X-1)^2(X+1)(X^2(3+2\sqrt{2}))(X^2-(3-2\sqrt{2}))
\end{align*}

Les racines de $P$ dans $\Cc$ sont $1$, $-1$, $\sqrt{3+2\sqrt{2}}$,$-\sqrt{3+2\sqrt{2}}$, $\sqrt{3-2\sqrt{2}}$ et $-\sqrt{3-2\sqrt{2}}$.}
\end{enumerate}
}

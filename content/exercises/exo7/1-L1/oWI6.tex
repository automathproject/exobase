\uuid{oWI6}
\exo7id{5172}
\titre{exo7 5172}
\auteur{rouget}
\organisation{exo7}
\datecreate{2010-06-30}
\isIndication{false}
\isCorrection{true}
\chapitre{Espace vectoriel}
\sousChapitre{Définition, sous-espace}
\module{Algèbre}
\niveau{L1}
\difficulte{}

\contenu{
\texte{
Soient $E$ un $\Kk$-espace vectoriel et $A$, $B$ et $C$ trois sous-espaces de $E$.
}
\begin{enumerate}
    \item \question{Montrer que~:~$(A\cap B)+(A\cap C)\subset A\cap(B+C)$.}
\reponse{Soit $x\in E$.

$x\in(A\cap B)+(A\cap C)\Rightarrow\exists y\in A\cap B,\;\exists z\in A\cap C/\;x=y+z$.

$y$ et $z$ sont dans $A$ et donc $x=y+z$ est dans $A$ car $A$ est un sous-espace vectoriel de $E$.

Puis $y$ est dans $B$ et $z$ est dans $C$ et donc $x=y+z$ est dans $B+C$.
Finalement,

$$\forall x\in E,\;[x\in(A\cap B)+(A\cap C)\Rightarrow x\in A\cap(B+C)].$$

Autre démarche.

$(A\cap B\subset B$ et $A\cap C\subset C)\Rightarrow(A\cap B)+(A\cap C)\subset B+C$ puis
$(A\cap B\subset A$ et $A\cap C\subset A\Rightarrow (A\cap B)+(A\cap C)\subset A+A=A$, et finalement
$(A\cap B)+(A\cap C)\subset A\cap(B+C)$.}
    \item \question{A-t-on toujours l'égalité~?}
\reponse{Si on essaie de démontrer l'inclusion contraire, le raisonnement coince car la somme $y+z$ peut être dans
$A$ sans que ni $y$, ni $z$ ne soient dans $A$.

Contre-exemple. Dans $\Rr^2$, on considère $A=\Rr.(1,0)=\{(x,0),\;x\in\Rr\}$, $B=\Rr.(0,1)$ et $C=\Rr.(1,1)$.

$B+C=\Rr^2$ et $A\cap(B+C)=A$ mais $A\cap B=\{0\}$ et $A\cap C=\{0\}$ et donc $(A\cap B)+(A\cap C)=\{0\}\neq
A\cap(B+C)$.}
    \item \question{Montrer que~:~$(A\cap B)+(A\cap C)=A\cap(B+(A\cap C))$.}
\reponse{$A\cap B\subset B\Rightarrow (A\cap B)+(A\cap C)\subset B+(A\cap C)$ mais aussi $(A\cap B)+(A\cap C)\subset
A+A=A$. Donc, $(A\cap B)+(A\cap C)\subset A\cap(B+(A\cap C))$.

Inversement, soit $x\in A\cap(B+(A\cap C))$ alors $x=y+z$ où $y$ est dans $B$ et $z$ est dans $A\cap C$. Mais alors,
$x$ et $z$ sont dans $A$ et donc $y=x-z$ est dans $A$ et même plus précisément dans $A\cap B$. Donc,
$x\in(A\cap B)+(A\cap C)$. Donc,  $A\cap(B+(A\cap C))\subset(A\cap B)+(A\cap C)$ et finalement, $A\cap(B+(A\cap
C))=(A\cap B)+(A\cap C)$.}
\end{enumerate}
}

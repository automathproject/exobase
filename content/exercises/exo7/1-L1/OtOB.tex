\uuid{OtOB}
\exo7id{949}
\titre{exo7 949}
\auteur{ridde}
\organisation{exo7}
\datecreate{1999-11-01}
\isIndication{true}
\isCorrection{true}
\chapitre{Application linéaire}
\sousChapitre{Image et noyau, théorème du rang}
\module{Algèbre}
\niveau{L1}
\difficulte{}

\contenu{
\texte{
Soit $f \in \mathcal{L} (E)$. Montrer que $\ker (f)
\cap \text{Im} (f) = f (\ker (f\circ f))$.
}
\indication{Montrer la double inclusion.}
\reponse{
Pour montrer l'\'egalit\'e $\ker f \cap \mathop{\mathrm{Im}}\nolimits f = f(\ker f^2)$, nous
montrons la double inclusion.

Soit $y\in \ker f \cap \mathop{\mathrm{Im}}\nolimits f$, alors $f(y) = 0$ et il existe $x$
tel que $y=f(x)$. De plus $f^2(x) = f(f(x))=f(y) =0$ donc $x\in
\ker f^2$. Comme $y = f(x)$ alors $y \in f(\ker f^2)$. Donc $\ker
f \cap \mathop{\mathrm{Im}}\nolimits f \subset f(\ker f^2)$.


Pour l'autre inclusion, nous avons d\'ej\`a que $ f(\ker f^2) \subset
f(E) = \mathop{\mathrm{Im}}\nolimits f$. De plus $ f(\ker f^2) \subset \ker f$, car si $y\in
f(\ker f^2)$ il existe $x\in \ker f^2$ tel que $y=f(x)$, et
$f^2(x) = 0$ implique $f(y)=0$ donc $y\in \ker f$. Par cons\'equent
$ f(\ker f^2) \subset \ker f \cap \mathop{\mathrm{Im}}\nolimits f $.
}
}

\uuid{Sxal}
\exo7id{168}
\titre{exo7 168}
\auteur{cousquer}
\organisation{exo7}
\datecreate{2003-10-01}
\isIndication{false}
\isCorrection{false}
\chapitre{Logique, ensemble, raisonnement}
\sousChapitre{Récurrence}
\module{Algèbre}
\niveau{L1}
\difficulte{}

\contenu{
\texte{

}
\begin{enumerate}
    \item \question{Calculer les restes de la division euclidienne de $1,4,4^2,4^3$ par
  $3$.}
    \item \question{Formuler, pour tout $n\in\mathbb{N}$, une hypothèse $\mathcal{P}(n)$ concernant le
  reste de la division euclidienne de $4^n$ par $3$. Démontrer que $\mathcal{P}(n)$
  est vérifiée pour tout $n\in\mathbb{N}$.}
    \item \question{Pour tout $n\in\mathbb{N}$, le nombre $16^n+4^n +3$ est-il divisible par $3$.}
\end{enumerate}
}

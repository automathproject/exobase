\uuid{73LF}
\exo7id{5608}
\titre{exo7 5608}
\auteur{rouget}
\organisation{exo7}
\datecreate{2010-10-16}
\isIndication{false}
\isCorrection{true}
\chapitre{Matrice}
\sousChapitre{Autre}
\module{Algèbre}
\niveau{L1}
\difficulte{}

\contenu{
\texte{
Soient $I=\left(
\begin{array}{cc}
1&0\\
0&1
\end{array}
\right)$ et $J=\left(
\begin{array}{cc}
1&1\\
0&1
\end{array}
\right)$. Soit $E=\{M(x,y)=xI+yJ,\;(x,y)\in\Rr^2\}$.
}
\begin{enumerate}
    \item \question{Montrer que $(E,+,.)$ est un $\Rr$-espace vectoriel et préciser sa dimension.}
\reponse{$E=\text{Vect}(I,J)$ est un sous-espace vectoriel de $\mathcal{M}_2(\Rr)$ de dimension inférieure ou égale à 2. De plus, la famille $(I,J)$ est libre car la matrice $J$ n'est pas une matrice scalaire et donc $\text{dim}E= 2$.}
    \item \question{Montrer que $(E,+,\times)$ est un anneau commutatif.}
\reponse{Puisque $(E,+,.)$ est un espace vectoriel, $(E,+)$ est un groupe commutatif.

Ensuite, $I^2=I\in E$, $IJ=JI=J\in E$ et $J^2=(I + E_{1,2})^2=I +2E_{1,2}= I+2(J-I) - I =2J -I\in E$. Par bilinéarité du produit matriciel, la multiplication est interne dans $E$ et commutative. De plus, $I\in E$ et finalement $E$ est un sous-anneau commutatif de $\mathcal{M}_2(\Rr)$.

 
\textbf{Remarque.} $M(x,y)M(x',y')= xx'I +(xy'+yx')J +yy'(2J-I)= (xx'-yy')I + (xy'+yx'+2yy')J$.}
    \item \question{Quels sont les éléments inversibles de l'anneau $(E,+,\times)$ ?}
\reponse{Soit $(x,y)\in\Rr^2$. 

\begin{align*}\ensuremath
M(x,y)\;\text{est inversible dans}\;E&\Leftrightarrow\exists(x',y')\in\Rr^2/;(xx'-yy')I + (xy'+yx'+2yy')J=I\\
 &\Leftrightarrow\exists(x',y')\in\Rr^2/;\left\{
 \begin{array}{l}
 xx'-yy'=1\\
 yx'+(x+2y)y'=0
 \end{array}
 \right.\;(\text{car la famille}\;(I,J)\;\text{est libre})\quad(*).
\end{align*}

Le déterminant de ce système d'inconnue $(x',y')$ est $x(x+2y)+y^2=(x+y)^2$.

 

\textbullet~Si $x+y\neq0$, le système $(*)$ admet une et une seule solution. Dans ce cas, $M(x,y)$ est inversible dans $E$.

\textbullet~Si $x+y=0$, le système $(*)$ s'écrit $\left\{
 \begin{array}{l}
 x(x'+y')=1\\
 -x(x'+y')=0
 \end{array}
 \right.$ et n'a pas de solution. Dans ce cas, $M(x,y)$ n'est pas inversible dans $E$.
 
 \begin{center}
 \shadowbox{
 $M(x,y)$ est inversible dans $E\Leftrightarrow x+y\neq0$.
 }
 \end{center}
 

\textbf{Remarque.} Puisque $I\in E$, $M(x,y)$ est inversible dans $E$ si et seulement si $M(x,y)$ est inversible dans $\mathcal{M}_2(\Rr)$.}
    \item \question{Résoudre dans $E$ les équations : 
  \begin{enumerate}}
\reponse{Posons $X= xI+yJ$, $(x,y)\in\Rr^2$. 

  \begin{enumerate}}
    \item \question{$X^2=I$}
\reponse{D'après 1), $X^2=(x^2-y^2)I+(2xy+2y^2)J$. Donc

\begin{align*}\ensuremath
X^2= I&\Leftrightarrow x^2-y^2 = 1\;\text{et}\;2xy+2y^2= 0\;(\text{car la famille}\;(I,J)\;\text{est libre})\\
 &\Leftrightarrow (y = 0\;\text{et}\;x^2 = 1)\;\text{ou}\;(y = -x et 0 = 1)\Leftrightarrow(y = 0\;\text{et}\;x = 1)\;\text{ou}\;(y = 0\;\text{et}\;x = -1)\\
  &\Leftrightarrow X = I\;\text{ou}\;X = -I.
\end{align*}
 
\begin{center}
\shadowbox{
$\mathcal{S}=\{I,-I\}$.
}
\end{center}}
    \item \question{$X^2=0$}
\reponse{\begin{align*}\ensuremath
X^2=0&\Leftrightarrow x^2-y^2 =0\;\text{et}\;2xy+2y^2= 0\Leftrightarrow (y = 0\;\text{et}\;x^2 = 0)\;\text{ou}\;(y = -x et 0 = 0)\Leftrightarrow y=-x.
\end{align*}

\begin{center}
\shadowbox{
$\mathcal{S}=\{x(I-J),\;x\in\Rr\}$.
}
\end{center}

\textbf{Remarque.} L'équation $X^2=0$, de degré 2, admet une infinité de solutions dans $E$ ce qui montre une nouvelle fois que $(E,+,\times)$ n'est pas un corps.}
    \item \question{$X^2=X$.}
\reponse{\begin{align*}\ensuremath
X^2=X&\Leftrightarrow x^2-y^2 =x\;\text{et}\;2xy+2y^2= y\Leftrightarrow y(2x+2y-1)= 0\;\text{et}\;x^2-y^2=x\\
 &\Leftrightarrow(y = 0\;\text{et}\;x^2= x)\;\text{ou}\;(2(x+y)=1\;\text{et}\;(x+y)(x-y)=x)\Leftrightarrow(X = 0\;\text{ou}\;X = I)\;\text{ou}\;(2(x+y)=1\;\text{et}\;x-y=2x)\\
  &\Leftrightarrow X = 0\;\text{ou}\;X =I.
\end{align*}

\begin{center}
\shadowbox{
$\mathcal{S}=\{0,I\}$.
}
\end{center}}
\end{enumerate}
}

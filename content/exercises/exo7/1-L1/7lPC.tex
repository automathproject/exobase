\uuid{7lPC}
\exo7id{5579}
\titre{exo7 5579}
\auteur{rouget}
\organisation{exo7}
\datecreate{2010-10-16}
\isIndication{false}
\isCorrection{true}
\chapitre{Espace vectoriel}
\sousChapitre{Dimension}
\module{Algèbre}
\niveau{L1}
\difficulte{}

\contenu{
\texte{
Soient $(x_1,..,x_n)$ une famille de $n$ vecteurs de rang $r$ et $(x_1,...,x_m)$ une sous famille de rang $s$ ($m\leqslant n$ et $s\leqslant r$). Montrer que $s\geqslant r+m-n$. Cas d'égalité ?
}
\reponse{
Si $m=n$, c'est immédiat.

Supposons $m<n$.

\begin{align*}\ensuremath
r&=\text{dim}(\text{Vect}(x_1,...,x_n))=\text{dim}\left(\text{Vect}(x_1,...,x_m)+\text{Vect}(x_{m+1},...x_n)\right)\\
 &\leqslant\text{dim}(\text{Vect}(x_1,...,x_m))+\text{dim}(\text{Vect}(x_{m+1},...,x_n))\\
&\leqslant s + (n-m)
\end{align*}

et donc $s\geqslant r+m-n$. On a l'égalité si et seulement si chaque inégalité est une égalité, c'est à dire si et seulement si
$\text{Vect}(x_1,...,x_m)\cap\text{Vect}(x_{m+1},...x_n)=\{0\}$ (pour la première) et la famille $(x_{m+1},...,x_n)$ est libre (pour la deuxième).
}
}

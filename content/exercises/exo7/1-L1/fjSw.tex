\uuid{fjSw}
\exo7id{161}
\titre{exo7 161}
\auteur{cousquer}
\organisation{exo7}
\datecreate{2003-10-01}
\isIndication{false}
\isCorrection{false}
\chapitre{Logique, ensemble, raisonnement}
\sousChapitre{Récurrence}
\module{Algèbre}
\niveau{L1}
\difficulte{}

\contenu{
\texte{
Que pensez-vous de la démonstration suivante~?
}
\begin{enumerate}
    \item \question{Pour tout $n\geq2$, on considère la propriété~:
$$P(n) :\quad \mbox{$n$ points distincts du plan sont toujours alignés}$$}
    \item \question{Initialisation~: $P(2)$ est vraie car deux points distincts 
sont toujours alignés.}
    \item \question{Hérédité~: On suppose que $P(n)$ est vraie et on 
va démontrer $P(n+1)$.

 \noindent Soit donc $A_{1}, A_{2},\ldots,A_{n}, A_{n+1}$ des points distincts. 
D'après l'hypothèse de récurrence, $A_{1}, A_{2},\ldots, A_{n}$ sont 
alignés sur une droite~$d$, et $A_{2},\ldots, A_{n}, A_{n+1}$ sont 
alignés sur une droite~$d'$. Les deux droites $d$ et $d'$ ayant $n-1$
points communs $ A_{2},\ldots, A_{n}$ sont confondues. Donc
$A_{1}, A_{2},\ldots, A_{n}, A_{n+1}$ sont alignés,
ce qui montre l'hérédité de la propriété.}
    \item \question{Conclusion~: la propriété $P(n)$ est vraie pour tout $n \geq 2$.}
\end{enumerate}
}

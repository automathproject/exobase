\uuid{yJiu}
\exo7id{951}
\titre{exo7 951}
\auteur{gourio}
\organisation{exo7}
\datecreate{2001-09-01}
\isIndication{false}
\isCorrection{true}
\chapitre{Application linéaire}
\sousChapitre{Image et noyau, théorème du rang}
\module{Algèbre}
\niveau{L1}
\difficulte{}

\contenu{
\texte{
Donner des exemples d'applications lin\'{e}aires de $\Rr^{2}$ dans $\Rr^{2}$
v\'{e}rifiant :
}
\begin{enumerate}
    \item \question{$\mathop{\mathrm{Ker}}\nolimits(f)=\mathop{\mathrm{Im}}\nolimits (f).$}
\reponse{Par exemple $f(x,y)=(0,x)$ alors $\mathop{\mathrm{Ker}}\nolimits f=\mathop{\mathrm{Im}}\nolimits f = \{0\}\times \Rr = \{ (0,y) \mid y\in \Rr\}$.}
    \item \question{$\mathop{\mathrm{Ker}}\nolimits(f)$ inclus strictement dans $\mathop{\mathrm{Im}}\nolimits (f).$}
\reponse{Par exemple l'identit\'e : $f(x,y)=(x,y)$. En fait un petit exercice est de montrer que les seules applications possibles sont les applications bijectives (c'est tr\`es particulier aux applications de $\Rr^2$ dans $\Rr^2$).}
    \item \question{$\mathop{\mathrm{Im}}\nolimits (f)$ inclus strictement dans $\mathop{\mathrm{Ker}}\nolimits(f).$}
\reponse{L'application nulle : $f(x,y)=(0,0)$. Exercice : c'est la seule possible !}
\end{enumerate}
}

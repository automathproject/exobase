\uuid{mTQq}
\exo7id{6968}
\titre{exo7 6968}
\auteur{exo7}
\organisation{exo7}
\datecreate{2014-04-08}
\video{cMc2VRPKaKI}
\isIndication{true}
\isCorrection{true}
\chapitre{Polynôme, fraction rationnelle}
\sousChapitre{Fraction rationnelle}
\module{Algèbre}
\niveau{L1}
\difficulte{}

\contenu{
\texte{
Décomposer les fractions suivantes en éléments simples sur $\Rr$, 
en raisonnant par substitution pour obtenir les coefficients.
}
\begin{enumerate}
    \item \question{% De 444, cousquer
$F=\frac{X^5+X^4+1}{X^3-X}$}
    \item \question{% De 444, cousquer
$G=\frac{X^3+X+1}{(X-1)^3(X+1)}$}
    \item \question{% De 444, cousquer
$H=\frac{X}{(X^2+1)(X^2+4)}$}
    \item \question{% De 445, cousquer
$K=\frac{2X^4+X^3+3X^2-6X+1}{2X^3-X^2}$}
\reponse{
$F=\frac{X^5+X^4+1}{X^3-X}$.

Pour obtenir la partie polynomiale, on fait une division euclidienne : 
$X^5+X^4+1=(X^3-X)(X^2+X+1)+ X^2+X+1$. Ce qui donne $F=X^2+X+1+F_1$, 
où $F_1=\frac{X^2+X+1}{X^3-X}$. Puisque $X^3-X=X(X-1)(X+1)$, 
la décomposition en éléments simples est de la forme 
$$F_1 = \frac{X^2+X+1}{X(X-1)(X+1)}=\frac{a}{X}+\frac{b}{X-1}+\frac{c}{X+1}$$

Pour obtenir $a$ :
\begin{itemize}
on multiplie l'égalité par $X$ : $\frac{X(X^2+X+1)}{X(X-1)(X+1)}=X \left(\frac{a}{X}+\frac{b}{X-1}+\frac{c}{X+1}\right)$,
on simplifie $\frac{X^2+X+1}{(X-1)(X+1)}= a+\frac{bX}{X-1}+\frac{cX}{X+1}$,
on remplace $X$ par $0$ et on obtient $-1= a+0+0$, donc $a=-1$.
\end{itemize}

 De même, en multipliant par $X-1$ et en remplaçant $X$ par $1$, il vient $b=\frac{3}{2}$.
 Puis en multipliant par $X+1$ et en remplaçant $X$ par $-1$, on trouve $c=\frac{1}{2}$.

D'où
$$\frac{X^5+X^4+1}{X^3-X} = X^2+X+1-\frac{1}{X}+\frac{\tfrac12}{X+1}+\frac{\tfrac32}{X-1}$$
$G=\frac{X^3+X+1}{(X-1)^3(X+1)}$. 

La partie polynomiale est nulle. La décomposition en éléments simples est de la forme
$G=\frac{a}{(X-1)^3}+\frac{b}{(X-1)^2}+\frac{c}{X-1}+\frac{d}{X+1}$.

\begin{itemize}
En multipliant les deux membres de l'égalité par $(X-1)^3$, 
en simplifiant puis en remplaçant $X$ par $1$, 
on obtient $a=\frac32$.
De même, en multipliant par $X+1$, 
en simplifiant puis en remplaçant $X$ par $-1$, 
on obtient $d=\frac18$.
En multipliant par $X$ et en regardant la limite 
  quand $X\to +\infty$, on obtient $1=c+d$. Donc $c=\frac78$.
En remplaçant $X$ par $0$, il vient $-1=-a+b-c+d$.
  Donc $b = \frac54$.
\end{itemize}

Ainsi :
$$G = \frac{X^3+X+1}{(X-1)^3(X+1)} = \frac{\tfrac32}{(X-1)^3} + \frac{\tfrac54}{(X-1)^2} 
+ \frac{\tfrac78}{X-1}  + \frac{\tfrac18}{X+1}$$
$H=\frac{X}{(X^2+1)(X^2+4)}$.

Puisque $X^2+1$ et $X^2+4$ sont irréductibles sur $\Rr$, la décomposition en éléments simples sera de la forme 
$$\frac{X}{(X^2+1)(X^2+4)} = \frac{aX+b}{X^2+1} + \frac{cX+d}{X^2+4}$$

  \begin{itemize}
En remplaçant $X$ par $0$, on obtient $0=b+\frac{1}{4}d$.
En multipliant les deux membres par $X$, on obtient 
    $\frac{X^2}{(X^2+1)(X^2+4)} = \frac{aX^2+bX}{X^2+1} + \frac{cX^2+dX}{X^2+4}$.
    En calculant la limite quand $X\to +\infty$, on a $0=a+c$.
Enfin, en évaluant les fractions en $X=1$ et $X=-1$, 
    on obtient $\frac{1}{10}=\frac{a+b}{2}+\frac{c+d}{5}$ et 
    $\frac{-1}{10}=\frac{-a+b}{2}+\frac{-c+d}{5}$.
  \end{itemize}

La résolution du système donne $b=d=0$, $a=\frac{1}{3}$, $c=-\frac{1}{3}$ et donc 
$$\frac{X}{(X^2+1)(X^2+4)} = \frac{\frac{1}{3}X}{X^2+1} - \frac{\frac{1}{3}X}{X^2+4}$$
$K=\frac{2X^4+X^3+3X^2-6X+1}{2X^3-X^2}$.

Pour la partie polynomiale, on fait la division euclidienne:
$$2X^4+X^3+3X^2-6X+1=(2X^3-X^2)(X+1)+(4X^2-6X+1)$$ 
ce qui donne 
$K=X+1+K_1$ où $K_1=\frac{4X^2-6X+1}{2X^3-X^2}$.
Pour trouver la décomposition en éléments simples de $K_1$, on factorise son numérateur: 
$2X^3-X^2=2X^2(X-\frac{1}{2})$, ce qui donne une décomposition de la forme
$K_1=\frac{a}{X^2}+\frac{b}{X}+\frac{c}{X-\frac{1}{2}}$.

On obtient alors $a$ en multipliant les deux membres de l'égalité par
$X^2$ puis en remplaçant $X$ par 0: $a=-1$. On obtient de
m\^eme $c$ en multipliant par $X-\frac{1}{2}$ et en remplaçant $X$
par $\frac{1}{2}$: $c=-2$. Enfin on trouve $b$ en identifiant pour une
valeur particuli\`ere non encore utilisée, par exemple $X=1$, ou mieux en
multipliant les deux membres par $X$ et en passant \`a la limite pour
$X\to+\infty$: $b=4$. Finalement:
$$\frac{2X^4+X^3+3X^2-6X+1}{2X^3-X^2}=X+1-\frac{1}{X^2}+\frac{4}{X}-\frac{2}{X-\frac{1}{2}}$$
}
\indication{Les fractions $F, K$ ont une partie polynomiale, elles s'écrivent

$F=X^2+X+1+\frac{X^2+X+1}{X^3-X}$

$K=X+1+\frac{4X^2-6X+1}{2X^3-X^2}$}
\end{enumerate}
}

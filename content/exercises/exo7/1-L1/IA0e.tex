\uuid{IA0e}
\exo7id{1054}
\titre{exo7 1054}
\auteur{legall}
\organisation{exo7}
\datecreate{1998-09-01}
\isIndication{false}
\isCorrection{false}
\chapitre{Matrice}
\sousChapitre{Propriétés élémentaires, généralités}
\module{Algèbre}
\niveau{L1}
\difficulte{}

\contenu{
\texte{
Soit $ E  $ le sous ensemble de $ M_3({\Rr}) $
d\' efini par
$ E = \Bigl \{  M(a,b,c)=\begin{pmatrix} a & 0 & c \cr
                    0 & b & 0 \cr
                    c & 0 & a \cr \end{pmatrix}  a , b  , c \in {\Rr} \Bigr \} .$
}
\begin{enumerate}
    \item \question{Montrer que $ E  $ est un sous-espace vectoriel de $ M_3(\Rr) $ stable
pour la multiplication des matrices. Calculer $ \hbox{dim } (E) .$}
    \item \question{Soit $ M(a,b,c)  $ un \' el\' ement de $ E  .$
D\' eterminer, suivant les valeurs des param\`etres $  a , b \hbox{ et }
 c \in {\Rr} $ son rang. Calculer (lorsque cela est possible) l'inverse $ M(a,b,c)
^{-1} $ de $ M(a,b,c)  .$}
    \item \question{Donner une base de $ E  $ form\' ee de matrices inversibles et une
autre form\' ee de matrices de rang $ 1 .$}
\end{enumerate}
}

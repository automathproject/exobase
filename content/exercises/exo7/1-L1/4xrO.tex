\uuid{4xrO}
\exo7id{7414}
\titre{exo7 7414}
\auteur{mourougane}
\organisation{exo7}
\datecreate{2021-08-10}
\isIndication{false}
\isCorrection{true}
\chapitre{Espace vectoriel}
\sousChapitre{Somme directe}
\module{Algèbre}
\niveau{L1}
\difficulte{}

\contenu{
\texte{
Soit $E$ l'espace vectoriel des fonctions de classe $\mathcal C^1$ de $\Rr$ dans $\Rr$. Soit $F$ l'ensemble de toutes les fonctions de $\Rr$ dans $\Rr$ qui s'\'{e}crivent $x\mapsto ax+b$, pour certains r\'{e}els $a$ et $b$.
Soit enfin $G=\{f\in E | f(0)=0, f'(0)=0\}$.
}
\begin{enumerate}
    \item \question{Montrer que $F$ et $G$ sont des sous-espaces vectoriels de $E$.}
    \item \question{Montrer que $$E=F\oplus G.$$}
\reponse{
On v\'{e}rifie ais\'{e}ment que $F$ et $G$ sont des sous-ensembles non vides de $E$ stables par combinaison lin\'{e}aire.
\begin{itemize}
L'intersection des sous-espaces $F$ et $G$ est triviale:
si $f\in F\cap G$, $f\in F$ implique que $f$ est une fonction polynomiale de degr\'{e} 1, soit $x\mapsto ax+b$, pour certains r\'{e}els $a$ et $b$,
mais alors on a $f(0)=b$ et $f'(0)=a$ donc $f\in G$ implique $a=b=0$ et finalement $f=0$.\\
Il reste \`{a} v\'{e}rifier que $E$ est la somme de $F$ et $G$. Comme $F$ et $G$ sont des sous-espaces vectoriels de $E$ on a \'{e}videmment $F+G\subset E$.
Pour l'inclusion inverse, soit $f\in E$, soit $g$ la fonction d\'{e}finie sur $\Rr$ par $x\mapsto f'(0)x+f(0)$.
Soit $h=f-g$. On a $g\in F$, et $h\in G$ puisque la d\'{e}rivation est lin\'{e}aire; de plus, $f=g+h$.
On a donc montr\'{e} $E=F+G$. 
\end{itemize}
Des deux points pr\'{e}c\'{e}dents, il r\'{e}sulte que $E=F\oplus G$.
}
\end{enumerate}
}

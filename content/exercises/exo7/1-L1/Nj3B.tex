\uuid{Nj3B}
\exo7id{2475}
\titre{exo7 2475}
\auteur{matexo1}
\organisation{exo7}
\datecreate{2002-02-01}
\video{zf0ETNslBnc}
\isIndication{true}
\isCorrection{true}
\chapitre{Matrice}
\sousChapitre{Autre}
\module{Algèbre}
\niveau{L1}
\difficulte{}

\contenu{
\texte{
\label{exo2475}
Trouver toutes les matrices de $\mathcal{M}_3(\Rr)$ qui vérifient
}
\begin{enumerate}
    \item \question{$M^2 = 0$ ;}
    \item \question{$M^2 = M$ ;}
    \item \question{$M^2 = I$.}
\reponse{
Soit $M$ une matrice telle que $M^2=0$ et soit $f$ l'application linéaire associée à $M$.
Comme $M^2=0$ alors $f\circ f = 0$. Cela entraîne $\Im f \subset \Ker f$. Discutons suivant la dimension
du noyau :
   \begin{enumerate}
Si $\dim \Ker f=3$ alors $f=0$ donc $M=0$ (la matrice nulle).
Si $\dim \Ker f=2$ alors prenons une base de $\Rr^3$ formée de deux vecteurs du noyau et d'un troisième vecteur.
Dans cette base la matrice de $f$ est
$M'=\begin{pmatrix}0&0&a\\0& 0 & b\\0&0&c\end{pmatrix}$ mais comme $f\circ f=0$ alors $M'^2=0$ ;
un petit calcul implique $c=0$. Donc $M$ et $M'$ sont les matrices de la même application linéaire $f$ mais exprimées dans des bases différentes,
donc $M$ et $M'$ sont semblables.
Si $\dim \Ker f=1$ alors comme $\Im f \subset \Ker f$ on a $\dim \Im f \le 1$ mais alors cela contredit le théorème
du rang : $\dim \Ker f + \dim \Im f = \dim \Rr^3$. Ce cas n'est pas possible.
Conclusion : $M$ est une matrice qui vérifie $M^2=0$ si et seulement si 
il existe une matrice inversible $P$ et des réels $a,b$ tels que 
$$M=P^{-1} \begin{pmatrix}0&0&a\\0& 0 & b\\0&0&0\end{pmatrix}P$$
}
\indication{Il faut trouver les propriétés de l'application linéaire $f$ associée à chacune de ces matrices.
Les résultats s'expriment en explicitant une (ou plusieurs) matrice $M'$ qui est la matrice de $f$ dans une base bien choisie
et ensuite en montrant que toutes les autres matrices sont de la forme
$M=P^{-1}M'P$.

Plus en détails pour chacun des cas :
\begin{enumerate}
  \item $\Im f \subset \Ker f$ et discuter suivant la dimension du noyau.
  \item Utiliser l'exercice \ref{exo1093} : $\Ker f \oplus \Im f$ et il existe une base telle que 
$f(e_i)=0$ ou $f(e_i)=e_i$.
  \item Poser $N= \frac{I+M}{2}$ (et donc $M=\cdots$) chercher à quelle condition $M^2=I$.
\end{enumerate}}
\end{enumerate}
}

\uuid{78p9}
\exo7id{105}
\titre{exo7 105}
\auteur{bodin}
\organisation{exo7}
\datecreate{1998-09-01}
\isIndication{false}
\isCorrection{true}
\chapitre{Logique, ensemble, raisonnement}
\sousChapitre{Logique}
\module{Algèbre}
\niveau{L1}
\difficulte{}

\contenu{
\texte{
D\'emontrer que $(1=2) \Rightarrow (2=3)$.
}
\reponse{
Il ne faut pas se laisser impressionner par l'allure de cette
assertion. En effet $A \Rightarrow B$ est une \'ecriture pour $B
\text{ ou } (\text{non} A)$ ;
 ici $A$ (la proposition $(1=2)$) est fausse, donc $(\text{non} A)$ est vraie
et $B \text{ ou } (\text{non} A)$ l'est \'egalement. Donc
l'assertion $A \Rightarrow B$ est vraie, quand $A$ est fausse et
quelque soit la proposition $B$.
}
}

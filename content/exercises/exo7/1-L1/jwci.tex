\uuid{jwci}
\exo7id{243}
\titre{exo7 243}
\auteur{bodin}
\organisation{exo7}
\datecreate{1998-09-01}
\isIndication{false}
\isCorrection{false}
\chapitre{Dénombrement}
\sousChapitre{Autre}
\module{Algèbre}
\niveau{L1}
\difficulte{}

\contenu{
\texte{

}
\begin{enumerate}
    \item \question{({\it principe des bergers}) Soient $E ,F$ deux ensembles avec $F$  ensemble fini,
    et $f$ une surjection de $E$ sur $F$ v\'erifiant :
    $$\forall y\in F ,\ \mathrm{Card}(f^{-1}({y}))=p$$
    Montrer que E est alors un ensemble fini et $\mathrm{Card}(E) =p\mathrm{Card}(F)$.}
    \item \question{({\it principe des tiroirs}) Soient
     $\alpha _1,\alpha _2,\ldots,\alpha _p ,$ $p$ \'elements distincts d'un ensemble
    $E$, r\'epartis entre une famille de $n$ sous-ensembles de $E$.
    Si $n<p$ montrer qu'il existe au moins un ensemble de la famille contenant
     au moins deux \'el\'ements parmi les $\alpha _i$.(on pourra raisonner par l'absurde)}
\end{enumerate}
}

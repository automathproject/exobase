\uuid{r8mL}
\exo7id{123}
\titre{exo7 123}
\auteur{bodin}
\organisation{exo7}
\datecreate{1998-09-01}
\video{I6E-O_PNk1Y}
\isIndication{true}
\isCorrection{true}
\chapitre{Logique, ensemble, raisonnement}
\sousChapitre{Ensemble}
\module{Algèbre}
\niveau{L1}
\difficulte{}

\contenu{
\texte{
Soit $A,B$ deux ensembles, montrer
$\complement(A\cup B) =\complement A\cap \complement B$ et
$\complement (A\cap B) =\complement A\cup \complement B$.
}
\indication{Il est plus facile de raisonner en prenant un \'el\'ement $x\in E$.
Par exemple, soit $F, G$ des sous-ensembles de $E$. Montrer que $F\subset G$
revient à montrer que pour tout $x\in F$ alors $x\in G$.
Et montrer $F=G$ est \'equivalent \`a $x\in F$ si et seulement si $x\in G$,
et ce pour tout $x$ de $E$.
Remarque : pour montrer $F=G$ on peut aussi montrer $F\subset G$ puis $G\subset F$.

Enfin, se rappeler que $x\in \complement F$ si et seulement si $x\notin F$.}
\reponse{
\begin{align*}
x\in\complement (A\cup B) & \Leftrightarrow x\notin A\cup B \\
&\Leftrightarrow x\notin A \text{ et } x\notin B\\
&\Leftrightarrow x\in\complement A \text{ et } x\in\complement B\\
&\Leftrightarrow x\in \complement A \cap \complement B.\\
\end{align*}
\begin{align*}
x\in\complement (A\cap B) & \Leftrightarrow x\notin A\cap B\\
&\Leftrightarrow x\notin A \text{ ou } x\notin B\\
&\Leftrightarrow x\in\complement A \text{ ou } x\in\complement\\
&\Leftrightarrow x\in \complement A \cup \complement B.\\
\end{align*}
}
}

\uuid{qiEV}
\exo7id{378}
\titre{exo7 378}
\auteur{gourio}
\organisation{exo7}
\datecreate{2001-09-01}
\video{ypZfDoacMUs}
\isIndication{false}
\isCorrection{true}
\chapitre{Polynôme, fraction rationnelle}
\sousChapitre{Division euclidienne}
\module{Algèbre}
\niveau{L1}
\difficulte{}

\contenu{
\texte{
Quels sont les polynômes $P\in\Cc[X]$ tels que $P'$ divise $P$?
}
\indication{Si $P=P'Q$ avec $P\not=0$, regarder le degré de $Q$.}
\reponse{
Le polynôme nul convient. Dans la suite on suppose que $P$ n'est pas le polynôme nul.

Notons $n = \deg P$ son degré. Comme $P'$ divise $P$, alors $P$ est non constant, 
donc $n\ge 1$. Soit $Q\in\Cc[X]$ tel que $P=P'Q$.
Puisque $\deg(P')=\deg(P)-1\ge 0$, alors $Q$ est de degré $1$.
Ainsi $Q(X)=aX+b$ avec $a\neq0$, et donc  
$$P(X)=P'(X)(aX+b)=aP'(X)(X+\frac{b}{a})$$
Donc si $z\not=\frac{-b}{a}$ et si $z$ est racine de $P$ de multiplicité $k \ge 1$, 
alors $z$ est aussi racine de $P'$ avec la même multiplicité, ce qui est impossible. 
Ainsi la seule racine possible de $P$ est $\frac{-b}{a}$.

Réciproquement, soit $P$ un polynôme avec une seule racine $z_0 \in \Cc$ : 
il existe $\lambda\not=0$, $n\ge 1$ tels que $P=\lambda(X-z_0)^n$, 
qui est bien divisible par son polynôme dérivé.
}
}

\uuid{1A0w}
\exo7id{2512}
\titre{exo7 2512}
\auteur{mayer}
\organisation{exo7}
\datecreate{2009-04-01}
\isIndication{false}
\isCorrection{false}
\chapitre{Différentiabilité, calcul de différentielles}
\sousChapitre{Différentiabilité, calcul de différentielles}
\module{Calcul différentiel}
\niveau{L3}
\difficulte{}

\contenu{
\texte{
Soit $X= {\cal C} ([0,1])$
muni de la norme uniforme et soit $f$ une application de ${\cal
C}^1 (\Rr , \Rr )$. On note $F$ l'application $\varphi \mapsto
f\circ \varphi $ de $X$ dans $X$. Montrer que pour chaque $\varphi
\in X$, $DF(\varphi ) $ est l'op\'erateur lin\'eaire de
multiplication par $f'\circ \varphi $ dans $X$:
$$ DF(\varphi )\cdot ( h ) = h \, f'\circ \varphi \; ,$$
et que $DF$ est continue.
}
}

\uuid{OVsz}
\exo7id{2502}
\titre{exo7 2502}
\auteur{sarkis}
\organisation{exo7}
\datecreate{2009-04-01}
\isIndication{false}
\isCorrection{false}
\chapitre{Différentiabilité, calcul de différentielles}
\sousChapitre{Différentiabilité, calcul de différentielles}
\module{Calcul différentiel}
\niveau{L3}
\difficulte{}

\contenu{
\texte{
Soit $X={\mathcal C}([0,1])$
avec la norme $||f||=\int_0^1 |f(t)|dt$. Montrez que la forme
lin\'eaire $T: X \rightarrow \mathbb{R}$ d\'efinie par $T(f)=f(0)$
n'est pas continue en $0$. Que peut-on en d\'eduire pour le
sous-espace des fonctions de $X$ nulles en $0$ ?
}
}

\uuid{otPp}
\exo7id{2535}
\titre{exo7 2535}
\auteur{queffelec}
\organisation{exo7}
\datecreate{2009-04-01}
\isIndication{false}
\isCorrection{false}
\chapitre{Difféomorphisme, théorème d'inversion locale et des fonctions implicites}
\sousChapitre{Difféomorphisme, théorème d'inversion locale et des fonctions implicites}
\module{Calcul différentiel}
\niveau{L3}
\difficulte{}

\contenu{
\texte{
Soit $U$ un ouvert de
$\Rr^2$ et $\varphi : U\to \Rr^2$ une application de classe $C^1$
$\varphi=(f, g)$. On consid\`ere $u,v$ r\'eels et on cherche $x,y$
tels que
$$(*)\quad f(x,y)=u,\ g(x,y)=v.$$
}
\begin{enumerate}
    \item \question{On suppose que la diff\'erentielle de $\varphi$ est de rang
$2$ en tout point de $U$. Montrer que pour tout $(u,v)$ le
syst\`eme $(*)$ admet une solution, unique localement. Que peut-on
dire si la diff\'erentielle est de rang $2$ en un point de $U$
seulement ?}
    \item \question{A-t-on des solutions si la diff\'erentielle est de rang $0$
?}
    \item \question{On suppose maintenant que la diff\'erentielle de $\varphi$
est de rang $1$ en tout point de $U$. Si $f'_x$ ne s'annule pas
sur $U$, montrer que $\psi:(x,y)\to (f(x,y),y)$ d\'efinit un
diff\'eomorphisme d'un ouvert $V\subset U$ sur $\psi(V)$. En
d\'eduire $G$ telle que $g(x,y)=G(f(x,y))$ sur $V$. Que peut-on
dire des solutions du syst\`eme $(*)$ ?}
\end{enumerate}
}

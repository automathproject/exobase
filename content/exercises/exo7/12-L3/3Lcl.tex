\uuid{3Lcl}
\exo7id{2525}
\titre{exo7 2525}
\auteur{mayer}
\organisation{exo7}
\datecreate{2009-04-01}
\isIndication{false}
\isCorrection{false}
\chapitre{Différentiabilité, calcul de différentielles}
\sousChapitre{Différentiabilité, calcul de différentielles}
\module{Calcul différentiel}
\niveau{L3}
\difficulte{}

\contenu{
\texte{
On consid\`ere l'application
$F:\Rr ^2 \to \Rr ^2$ d\'efinie par $$F(x,y) = (x^2+y^2, y^2)\;
.$$ Soit $\Omega = \{p\in \Rr^2 \, ; \; \lim _{k\rightarrow \infty
} F^k (p) = (0,0)\}$.
}
\begin{enumerate}
    \item \question{V\'erifier que $p\in \Omega$ si et seulement si $F(p)\in
\Omega$.}
    \item \question{Montrer qu'il existe $\delta >0$ tel que $\| | D
F(p) \| | <\frac{1}{2}$ si $\|p\|<\delta$. En d\'eduire que $(0,0)
$ est dans l'int\'erieur de $\Omega$ puis que $\Omega$ est un
ouvert.}
    \item \question{Utiliser l'homog\'en\'eit\'e de $F$ pour montrer que
$\Omega $ est connexe.}
\end{enumerate}
}

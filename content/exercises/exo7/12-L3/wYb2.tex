\uuid{wYb2}
\exo7id{2531}
\titre{exo7 2531}
\auteur{queffelec}
\organisation{exo7}
\datecreate{2009-04-01}
\isIndication{false}
\isCorrection{false}
\chapitre{Difféomorphisme, théorème d'inversion locale et des fonctions implicites}
\sousChapitre{Difféomorphisme, théorème d'inversion locale et des fonctions implicites}
\module{Calcul différentiel}
\niveau{L3}
\difficulte{}

\contenu{
\texte{
Soit $U$ l'ouvert $\Rr^3 \setminus \{0\}$.
Soit $(x,y,z) \to (X,Y,Z)$ l'application inversion de p\^ole $0$, de
puisssance 1, d\'efinie dans $U$, \`a valeurs dans $\Rr^3$, par
les formules
$$X = {\frac x{x^2 + y^2 +
z^2}}\quad ; \quad Y = {\frac y {x^2 + y^2 + z^2}} \quad ; \quad Z =
{\frac z {x^2 + y^2 + z^2}}
$$
Calculer la matrice jacobienne de cette transformation (on posera
$\rho = \sqrt{x^2 + y^2 + z^2}$) et v\'erifier que cette matrice
est \'egale \`a son inverse.
}
}

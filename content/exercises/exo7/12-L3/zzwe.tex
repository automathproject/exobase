\uuid{zzwe}
\exo7id{6273}
\titre{exo7 6273}
\auteur{queffelec}
\organisation{exo7}
\datecreate{2011-10-16}
\isIndication{false}
\isCorrection{false}
\chapitre{Difféomorphisme, théorème d'inversion locale et des fonctions implicites}
\sousChapitre{Difféomorphisme, théorème d'inversion locale et des fonctions implicites}
\module{Calcul différentiel}
\niveau{L3}
\difficulte{}

\contenu{
\texte{
On considère le système d'équations d'inconnues $x$ et $y$ :
$$x={1\over2}\sin(x+y)+t-1,\quad y={1\over2}\cos(x-y)-t+{1\over2}.$$
}
\begin{enumerate}
    \item \question{Montrer que pour chaque $t_0\in\Rr$, il existe une unique solution
$(x_0,y_0)$, et que la fonction ainsi définie est continue..}
    \item \question{Montrer en considérant la fonction
$F(x,y,t)=(x-{1\over2}\sin(x+y)+t-1,\quad y-{1\over2}\cos(x-y)-t+{1\over2})$,
 que le système admet une unique solution $x=x(t),\ y=y(t)$ constituée
de fonctions $C^\infty$.}
    \item \question{Donner un développement limité à l'ordre $2$ de $x(t),y(t)$ au point
$(0,0)$.}
\end{enumerate}
}

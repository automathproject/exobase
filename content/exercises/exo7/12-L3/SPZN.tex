\uuid{SPZN}
\exo7id{2534}
\titre{exo7 2534}
\auteur{queffelec}
\organisation{exo7}
\datecreate{2009-04-01}
\isIndication{false}
\isCorrection{false}
\chapitre{Difféomorphisme, théorème d'inversion locale et des fonctions implicites}
\sousChapitre{Difféomorphisme, théorème d'inversion locale et des fonctions implicites}
\module{Calcul différentiel}
\niveau{L3}
\difficulte{}

\contenu{
\texte{

}
\begin{enumerate}
    \item \question{Montrer que si $a,b$ sont voisins de $1$, on peut trouver
$x,y\in{\Rr}$ tels que $y+e^{xy}=a,\ x+e^{-xy}=b$.}
    \item \question{Soit $f$ l'application de ${\Rr}^2$ dans lui-m\^eme
d\'efinie par $f(x,y)=(x\sin(xy)+y, y\cos(xy)+x)$, et soit
$(a_n,b_n)$ une suite tendant vers $(0,0)$. Montrer que si
$f(a_n,b_n)=0$ pour tout $n$, la suite $(a_n,b_n)$ stationne.}
\end{enumerate}
}

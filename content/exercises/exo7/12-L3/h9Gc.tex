\uuid{h9Gc}
\exo7id{2539}
\titre{exo7 2539}
\auteur{mayer}
\organisation{exo7}
\datecreate{2009-04-01}
\isIndication{false}
\isCorrection{false}
\chapitre{Difféomorphisme, théorème d'inversion locale et des fonctions implicites}
\sousChapitre{Difféomorphisme, théorème d'inversion locale et des fonctions implicites}
\module{Calcul différentiel}
\niveau{L3}
\difficulte{}

\contenu{
\texte{
Soit $G$ un ouvert born\'e de
$\Rr^n$ et soit $f:\overline{G} \to \Rr ^n$ une application
continue dans $\overline{G}$ et $C^1$ dans $G$. Pour tout $x\in
G$,  on suppose $Df(x)$ inversible. D\'emontrer que, sous ces
conditions, l'application $x\mapsto \|f(x)\|$ atteint son maximum
en un point du bord $\partial G = \overline{G}\setminus G$.
}
}

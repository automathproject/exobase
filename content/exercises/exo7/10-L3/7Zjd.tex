\uuid{7Zjd}
\exo7id{7727}
\titre{exo7 7727}
\auteur{mourougane}
\organisation{exo7}
\datecreate{2021-08-11}
\isIndication{false}
\isCorrection{true}
\chapitre{Forme bilinéaire}
\sousChapitre{Forme bilinéaire}
\module{Algèbre et théorie des nombres}
\niveau{L3}
\difficulte{}

\contenu{
\texte{

}
\begin{enumerate}
    \item \question{On considère l'application $\sigma~:~\mathbb{F}_{25}\to \mathbb{F}_{25}, \lambda\mapsto\lambda^5$. Montrer que c'est un automorphisme involutif du corps $\mathbb{F}_{25}$.}
\reponse{Il faut montrer que $\sigma$ est un morphisme de corps,
 puis que c'est une bijection. La bijection inverse est alors automatiquement un morphisme de corps.
 Pour le premier point, $\sigma$ est facilement multiplicative et $\sigma (1)=1$. Pour montrer que $\sigma (a+b)=\sigma (a)+\sigma(b)$, on utilise la formule du binôme puis le fait que le corps $\mathbb{F}_5$ est de caractéristique $5$.
 Pour le second point, il suffit de montrer que $\sigma$ est injective puisque son ensemble de départ a le même cardinal fini que son ensemble d'arrivée. Pour cela, il suffit de dire que $\sigma$ est un morphisme de corps.}
    \item \question{Les formes suivantes sur $E:=\mathbb{F}_{25}^3$ sont-elles $\sigma$-sesquilinéaires ?
\begin{eqnarray*}
 f_1( 
\left(\begin{array}{c}x\\ y\\ z\end{array}\right),
\left(\begin{array}{c}x'\\ y'\\ z'\end{array}\right)
)
&=&x(x')^5+3z(y')^5+3y(z')^5.\\
f_2( 
\left(\begin{array}{c}x\\ y\\ z\end{array}\right),
\left(\begin{array}{c}x'\\ y'\\ z'\end{array}\right)
)
&=&x^5(x')^5+x^5(y')^5+y^5(x')^5.\\
f_3(
\left(\begin{array}{c}x\\ y\\ z\end{array}\right),
\left(\begin{array}{c}x'\\ y'\\ z'\end{array}\right)
)
&=&3x(x')^5+z(y')^5+y(z')^5.
\end{eqnarray*}}
\reponse{Les formes $f_1$ et $f_3$ sont linéaires par rapport à la première variable, et en utilisant l'expression 
 $$f_1 \left(\begin{array}{c}x\\ y\\ z\end{array}\right),
 \left(\begin{array}{c}x'\\ y'\\ z'\end{array}\right)
 )
 =x\sigma(x')+3z\sigma(y')+3y\sigma(z')$$
 on montre aussi qu'elles sont $\sigma$-semi linéaire par rapport à la seconde variable. Elles sont donc sesquilinéaires.
 Par contre, comme il n'y a au plus que cinq élément de $\mathbb{F}_{25}$ qui vérifient $\lambda^5=\lambda$, il existe un $\lambda\in\mathbb{F}_{25}$ tel que $f_2(\lambda\cdot(1,1,1),(1,1,1))=3\lambda^5
 \not=\lambda\cdot f_2((1,1,1),(1,1,1))$.
 La forme $f_2$ n'est donc pas sesquilinéaires.}
    \item \question{Parmi les formes $\sigma$-sesquilinéaires précédentes, lesquelles sont équivalentes ?}
\reponse{Par un théorème du cours, c'est le rang qui classifie les formes sesquilinéaires sur les corps finis. Comme $f_1$ et $f_3$ ont même rang $3$, elles sont équivalentes.}
\end{enumerate}
}

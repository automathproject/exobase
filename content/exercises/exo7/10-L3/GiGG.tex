\uuid{GiGG}
\exo7id{2263}
\titre{exo7 2263}
\auteur{barraud}
\organisation{exo7}
\datecreate{2008-04-24}
\isIndication{false}
\isCorrection{true}
\chapitre{Polynôme}
\sousChapitre{Polynôme}
\module{Algèbre et théorie des nombres}
\niveau{L3}
\difficulte{}

\contenu{
\texte{
Soit $f(x)\in A[x]$ un polyn\^ome
sur un anneau $A$. Supposons que $(x-1)\,|\,f(x^n)$. Montrer que
$(x^n-1)\,|\,f(x^n)$.
}
\reponse{
Notons $f(x^{n})=P(x-1)$. Alors $f(1)=0\cdot P(1)=0$ et donc $(x-1)|f$.
  Notons $f=Q(x-1)$. On a alors $f(x^{n})=Q(x^{n})(x^{n}-1)$. $(x^{n}-1)$
  divise bien $f$.
}
}

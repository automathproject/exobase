\uuid{4XuP}
\exo7id{6549}
\titre{exo7 6549}
\auteur{drutu}
\organisation{exo7}
\datecreate{2011-10-16}
\isIndication{false}
\isCorrection{false}
\chapitre{Polynôme}
\sousChapitre{Polynôme}
\module{Algèbre et théorie des nombres}
\niveau{L3}
\difficulte{}

\contenu{
\texte{
Soient $x_1,x_2,\dots ,x_n$ les zéros du polyn\^ome $X^n+a_1x_{n-1}+\dots +a_n$. Démontrer que tout polyn\^ome symétrique en $x_2,x_3,\dots ,x_n$ peut être représenté sous forme de polyn\^ome en $x_1$.
}
}

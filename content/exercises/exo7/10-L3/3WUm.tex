\uuid{3WUm}
\exo7id{7892}
\titre{exo7 7892}
\auteur{mourougane}
\organisation{exo7}
\datecreate{2021-08-11}
\isIndication{false}
\isCorrection{false}
\chapitre{Forme bilinéaire}
\sousChapitre{Forme bilinéaire}
\module{Algèbre et théorie des nombres}
\niveau{L3}
\difficulte{}

\contenu{
\texte{

}
\begin{enumerate}
    \item \question{Montrer que deux formes quadratiques équivalentes sur un espace $E$ prennent les mêmes valeurs dans $k$.}
    \item \question{Soit dans tout l'exercice $(E,f)$ un espace muni d'une forme bilinéaire symétrique non dégénérée de forme quadratique associée $q$. Montrer que si $f$ admet un vecteur non nul isotrope, la forme $q$ prend toutes les valeurs de $k$.}
    \item \question{Montrer que $E$ se décompose comme somme directe orthogonale de plans hyperboliques et d'un sous-espace sur lequel la forme quadratique n'a pas de vecteur isotrope non nul.}
    \item \question{Montrer que le nombre de plans hyperboliques dans une telle décomposition est indépendant de la décomposition.}
\end{enumerate}
}

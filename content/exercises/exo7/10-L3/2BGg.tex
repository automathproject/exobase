\uuid{2BGg}
\exo7id{7840}
\titre{exo7 7840}
\auteur{mourougane}
\organisation{exo7}
\datecreate{2021-08-11}
\isIndication{false}
\isCorrection{true}
\chapitre{Forme bilinéaire}
\sousChapitre{Forme bilinéaire}
\module{Algèbre et théorie des nombres}
\niveau{L3}
\difficulte{}

\contenu{
\texte{
Démontrer le théorème de Witt dans le cas particulier suivant :

Soit $E$ et $E'$ deux espaces symplectiques non-singuliers de dimension $4$.
Soit $d\subset E$ et $d'\subset E'$ deux droites 
et $f$ une application linéaire bijective de $d$ sur $d'$.
Montrer qu'il existe une isométrie de $E$ sur $E'$ qui prolonge $f$.
}
\reponse{
Soit $(E,f)$ et $(E,f')'$ deux espaces symplectiques non-singuliers de dimension $4$.
Soit $d\subset E$ et $d'=vect(x')\subset E'$ deux droites 
et $f$ une application linéaire bijective de $d$ sur $d'$.
On écrit $d=vect(x)$ et $d'=vect(x')$ où $x'=f(x)$.

Notons que comme $d$ et $d'$ sont de dimension $1$ donc isotropes, $f$ est une isométrie (pour les formes bilinéaires induites). Puisque $f$ est non dégénérée, on choisit un vecteur $z$ (non colinéaire à $x$) tel que $f(x,z)$ est non nul donc inversible dans $k$. 
On définit $y$ sous la forme $y=\alpha x+ f(x,z)^{-1}z$. Alors, $(x,y)$ est une paire symplectique.
L'orthogonal de $vect(x,y)$ (muni de la forme induite) est un plan non-singulier puisque $vect(x,y)$ est non-singulier.
Par la construction précédente, on y trouve une paire symplectique $(X,Y)$.
Le même raisonnement permet de trouver une base $(x',y',X',Y')$ de $E'$ telle que $(x',y')$ et $(X',Y')$ soient deux paires symplectiques orthogonales de $E'$.
L'application $f$ définie par $x\mapsto x'$, $y\mapsto y'$, $X\mapsto X'$ et $Y\mapsto Y'$ est une isométrie de $E$ sur $E'$ qui prolonge $f$.
}
}

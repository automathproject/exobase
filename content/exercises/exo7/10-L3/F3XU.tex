\uuid{F3XU}
\exo7id{7798}
\titre{exo7 7798}
\auteur{mourougane}
\organisation{exo7}
\datecreate{2021-08-11}
\isIndication{false}
\isCorrection{false}
\chapitre{Forme bilinéaire}
\sousChapitre{Forme bilinéaire}
\module{Algèbre et théorie des nombres}
\niveau{L3}
\difficulte{}

\contenu{
\texte{
Soit $E=\Rr^2$ muni de la forme quadratique $q(x_1,x_2)=x_1^2-x_2^2$ non dégénérée de signature $(1,1)$. Le groupe $O(1,1)$ est par définition le groupe des isométries de $(E,q)$.
}
\begin{enumerate}
    \item \question{Le groupe $O(1,1)$ agit-il transitivement sur les droites de
 $\Rr^2$ ?}
    \item \question{Soit $q$ une forme quadratique réelle non dégénérée.
 Montrer que si $q$ a la
 même signature en restriction à $F$ et à $F'$ alors $F$ et $F'$ sont
 dans la même orbite sous l'action de $O(q)$.}
    \item \question{Décrire les orbites de l'action de $O(2,1)$ sur les droites de
 $\Rr^3$.}
\end{enumerate}
}

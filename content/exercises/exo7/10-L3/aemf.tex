\uuid{aemf}
\exo7id{2286}
\titre{exo7 2286}
\auteur{barraud}
\organisation{exo7}
\datecreate{2008-04-24}
\isIndication{false}
\isCorrection{true}
\chapitre{Anneau, corps}
\sousChapitre{Anneau, corps}
\module{Algèbre et théorie des nombres}
\niveau{L3}
\difficulte{}

\contenu{
\texte{

}
\begin{enumerate}
    \item \question{Trouver le nombre d'\'el\'ements de l'anneau
quotient $\Zz[\sqrt{d}]/(m)$ o\`u $m\in \Zz$ et $m\ne 0$.}
    \item \question{L'id\'eal principal endendr\'e par $2$ est-il premier
dans l'anneau $\Zz[\sqrt{d}]$ ?}
\reponse{
Soit $\alpha=a+b\sqrt{d}\in\Zz[\sqrt{d}]$. Soit $a=mp+a'$ la division
  euclidienne de $a$ par $m$, et $b=mq+b'$ celle de $b$ par $m$. Alors
  $\alpha=m(p+q\sqrt{d})+a'+b'\sqrt{d}$. On en déduit que chaque classe
  du quotient $\Zz[\sqrt{d}]/(m)$ a un représentant dans
  $$
    \mathcal{C}=\Big\{a+b\sqrt{d},\  (a,b)\in\{0,\dots,m-1\}^{2}\Big\}
  $$
  Par ailleurs si deux éléments $a+b\sqrt{d}$ et $a'+b'\sqrt{d}$ de cet
  ensemble sont dans la même classe, alors $\exists c,d\in\Zz, \
  a+b\sqrt{d}=(a'+b'\sqrt{d})+m(c+d\sqrt{d})$. On en déduit que $a=a'+mc$
  et $b=b'+md$, et donc $a=a'$, $b=b'$.
 
  Ainsi chaque classe de $\Zz[\sqrt{d}]/(m)$ a un représentant unique
  dans $\mathcal{C}$. $\Zz[\sqrt{d}]/(m)$ et $\mathcal{C}$ sont donc en
  bijection~: en particulier, $\Zz[\sqrt{d}]/(m)$ a $m^{2}$ éléments.
  

  \emph{Remarque}~: on a
  $$
    \Zz[\sqrt{d}]\sim \Zz[X]/(X^{2}-d).
  $$
  En effet l'application $\phi:\Zz[X]/(X^{2}-d)\to\Zz[\sqrt{d}]$,
  $\bar{P}\mapsto P(\sqrt{d})$ est bien définie (si $\bar(P)=\bar{Q}$,
  alors $P(\sqrt{d})=Q(\sqrt{d})$), et c'est un morphisme d'anneaux. De
  plus, si $\phi(P)=0$, notons $P=Q(X^{2}-d)+(aX+b)$ la division
  euclidienne de $P$ par $X^{2}-d$. En évaluant en $\sqrt{d}$, on a
  $a\sqrt{d}+b=0$ donc $R=0$. On en déduit que $(X^{2}-d)|P$, i.e.
  $\bar{P}=0$. On en déduit que $\ker\phi=\{0\}$, donc $\phi$ est
  injective. Par ailleurs $\forall(a,b)\in\Zz^{2},
  \phi(a+bX)=a+b\sqrt{d}$ donc $\phi$ est surjective.

  \medskip

  Si $d$ est pair, comme $\sqrt{d}\cdot\sqrt{d}=|d|\in(2)$ alors que
  $\sqrt{d}\notin(2)$, $(2)$ n'est pas premier.

  Si $d$ est impair~: $(1+\sqrt{d})(1+\sqrt{d})=(1+d)+2\sqrt{d}\in(2)$,
  mais $(1+\sqrt{d})\notin(2)$ donc $(2)$ n'est pas premier.

  \emph{Remarque}~: $\Zz[\sqrt{d}]/(2)\sim\Zz_{2}[X]/(X^{2}+\bar{d})$.
  $(X^{2}+\bar{d})$ est $X^{2}$ ou $X^{2}+1$. Aucun de ces deux polynômes
  n'est irréductible. Donc le quotient ne saurait être intègre.
}
\end{enumerate}
}

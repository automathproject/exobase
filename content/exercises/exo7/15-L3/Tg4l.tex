\uuid{Tg4l}
\exo7id{2383}
\titre{exo7 2383}
\auteur{mayer}
\organisation{exo7}
\datecreate{2003-10-01}
\isIndication{true}
\isCorrection{true}
\chapitre{Connexité}
\sousChapitre{Connexité}
\module{Topologie}
\niveau{L3}
\difficulte{}

\contenu{
\texte{
\label{exocon}
Soit $X$ un espace m\'etrique.
}
\begin{enumerate}
    \item \question{Montrer que $X$ est connexe si et seulement si toute application
continue $f:X\to \{0,1\}$ est constante.}
\reponse{C'est du cours.}
    \item \question{Soit $A$ une partie de $X$ connexe. Montrer que toute partie
$B \subset E$ v\'erifiant $A \subset B\subset \overline{A}$ est
connexe.}
\reponse{Si $f : B \longrightarrow \{ 0,1 \}$ est continue alors elle induit une application restreinte $f_{|A} : A \longrightarrow \{ 0,1 \}$ continue. Donc $f$ est constante sur $A$. Soit $b \in B$ et soit $(a_n)$ une suite d'éléments de $A$ qui tendent vers $b$ (c'est possible car $B \subset \bar A$), alors $f(a_n)$ est constante, par exemple égal à $1$, car $A$ est connexe. Mais $f$ est continue sur $B$, donc $f(b) = \lim f(a_n) = 1$. On montre ainsi que $f$ est constante sur $B$. Donc $B$ est connexe. (Au passage on a montrer que $\bar A$ était connexe.)}
    \item \question{Si $(A_n)_{n\geq 0}$ est une suite de parties connexes de
$X$ telle que $A_n \cap A_{n+1}\neq \emptyset$ pour tout $n\geq
0$. Prouver que $\bigcup _{n\geq 0} A_n $ est connexe.}
\reponse{Soit $f : A \longrightarrow \{ 0,1 \}$ une fonction continue, o\`u $A = \bigcup A_n$. $A_0$ est connexe donc $f$ est constante sur $A_0$ et vaut $v_0$,
de même $A_1$ est connexe donc $f$ est constante sur $A_1$ et vaut $v_1$. Mais 
pour $a \in A_0\cap A_1 \neq \varnothing$, on a $f(a)=v_0$ car $a\in A_0$ et
$f(a)=v_1$ car $a\in A_1$. Donc $v_0=v_1$. Donc $f$ est constante sur $A_0 \cup A_1$. Par récurrence $f$ est constante sur $A$.}
\indication{Utiliser la première question pour les deux suivantes.}
\end{enumerate}
}

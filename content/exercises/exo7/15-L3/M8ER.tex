\uuid{M8ER}
\exo7id{6177}
\titre{exo7 6177}
\auteur{queffelec}
\organisation{exo7}
\datecreate{2011-10-16}
\isIndication{false}
\isCorrection{false}
\chapitre{Compacité}
\sousChapitre{Compacité}
\module{Topologie}
\niveau{L3}
\difficulte{}

\contenu{
\texte{
Soit $X$ un espace topologique compact et $C(X)$ l'espace des fonctions
conti\-nues sur $X$ avec la norme uniforme.
}
\begin{enumerate}
    \item \question{Soit $J$ un idéal propre de $C(X)$ ; montrer que toutes les fonctions de $J$
s'annulent en un même point de $X$.
\emph{Indication :} raisonner par l'absurde, utiliser le fait qu'une fonction continue
$\neq0$ en $x$, est $\neq0$ sur un voisinage de $x$ et recouvrir $X$ avec de
tels voisinages. 

\medskip

Pour $f\in J$, on note $Z_f=f^{-1}(\{0\})$, l'ensemble des zéros de $f$.}
    \item \question{Soit $J$ un idéal de $C(X)$ et $Z=\cap_{f\in J}Z_f$; $Z$ est fermé.
  \begin{enumerate}}
    \item \question{Soit $K$ un fermé de $X$ disjoint de $Z$. Par un raisonnement
analogue à celui du 1., construire $f\in J$, $f\geq 0$ et ne s'annulant pas sur
$K$.

Etudier la limite $F$ de ${nf\over{1+nf}}$ dans $C(X)$.}
    \item \question{Montrer que si $g\in C(X)$ s'annule sur un ouvert contenant $Z$,
alors $g\in J$ et $Z\neq\emptyset$.}
    \item \question{Soit $g\in C(X)$ nulle sur $Z$; par un bon choix de $K$, montrer
que $g\in \overline J$.}
\end{enumerate}
}

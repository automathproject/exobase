\uuid{FnEz}
\exo7id{6040}
\titre{exo7 6040}
\auteur{queffelec}
\organisation{exo7}
\datecreate{2011-10-16}
\isIndication{false}
\isCorrection{false}
\chapitre{Espace topologique, espace métrique}
\sousChapitre{Espace topologique, espace métrique}
\module{Topologie}
\niveau{L3}
\difficulte{}

\contenu{
\texte{
On considère dans ${\Nn}^*$, la famille de progressions arithmétiques
$$P_{a,b}=\{a+bn/n\in {\Nn}^*\},$$ 
où $a$ et $b$ sont deux entiers premiers entre eux.
}
\begin{enumerate}
    \item \question{Montrer que l'intersection de deux telles progressions est soit vide,
soit une progression arithmétique de même nature, plus
précisément,
$$P_{a,b}\cap P_{a',b'}=P_{\alpha,\beta}$$
où $\alpha$ est le minimum de l'ensemble $P_{a,b}\cap P_{a',b'}$, et
$\beta=\mathrm{ppcm}(b,b')$.}
    \item \question{En déduire que cette famille d'ensembles (en y adjoignant $\emptyset$) 
forme une base de topologie sur ${\Nn}^*$ dont on décrira les ouverts.}
    \item \question{Montrer que cette topologie est séparée.}
\end{enumerate}
}

\uuid{U01x}
\exo7id{6169}
\titre{exo7 6169}
\auteur{queffelec}
\organisation{exo7}
\datecreate{2011-10-16}
\isIndication{false}
\isCorrection{false}
\chapitre{Compacité}
\sousChapitre{Compacité}
\module{Topologie}
\niveau{L3}
\difficulte{}

\contenu{
\texte{
Soit $X$ un espace topologique compact et $C(X)$ l'espace des fonctions réelles
conti\-nues sur $X$ avec la norme uniforme.

Soit $J$ un idéal propre de $C(X)$; on va montrer par l'absurde que toutes les
fonctions de
$J$ s'annulent en un même point de $X$.
}
\begin{enumerate}
    \item \question{Sinon, montrer qu'on peut trouver $n$ points de $X$, $x_1,\cdots,x_n$,
$V_1,\cdots,V_n$ où $V_i$ voisinage de $x_i$ et $n$ fonctions de $J$,
$f_1,\cdots,f_n$ tels que
 $$X=\cup_i\ V_i, \  \  f_i{}_{|V_i}\not=0.$$}
    \item \question{Construire alors une fonction $g$ dans $J$ ne s'annulant jamais et en déduire
que ${\bf 1}\in J$, d'où la contradiction.}
\end{enumerate}
}

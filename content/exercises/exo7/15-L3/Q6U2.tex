\uuid{Q6U2}
\exo7id{6151}
\titre{exo7 6151}
\auteur{queffelec}
\organisation{exo7}
\datecreate{2011-10-16}
\isIndication{false}
\isCorrection{false}
\chapitre{Connexité}
\sousChapitre{Connexité}
\module{Topologie}
\niveau{L3}
\difficulte{}

\contenu{
\texte{
Soit $P\in{\Cc}[X]$ un polyn\^ome de racines $z_1,\ldots ,z_n$  distinctes ou
non, situées dans un convexe $K$ de ${\Cc}$.
}
\begin{enumerate}
    \item \question{On suppose que $P'(z)=0$ et $z\notin \{z_1,\ldots ,z_n\}$~; montrer qu'il
existe des réels
$\lambda _1(z),\ldots ,\lambda _n(z)$, inconnus mais $>0$, tels que l'on ait :
$\sum^n_{k=1} \lambda _k(z)(z-z_k)=0$.  (Indication~: considérer
${P'(z)\over P(z)}$ et son conjugué).}
    \item \question{Montrer que $P'$ a aussi toutes ses racines dans $K$ (théorème de
Gauss-Lucas).}
\end{enumerate}
}

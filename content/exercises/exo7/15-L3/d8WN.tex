\uuid{d8WN}
\exo7id{6065}
\titre{exo7 6065}
\auteur{queffelec}
\organisation{exo7}
\datecreate{2011-10-16}
\isIndication{false}
\isCorrection{false}
\chapitre{Espace topologique, espace métrique}
\sousChapitre{Espace topologique, espace métrique}
\module{Topologie}
\niveau{L3}
\difficulte{}

\contenu{
\texte{
On dit qu'un espace topologique $X$ a la propriété (P) si la famille de
parties de $X$ qui sont à la fois ouvertes et fermées est une base pour les
ouverts de $X$.
}
\begin{enumerate}
    \item \question{Montrer qu'un espace topologique discret a cette propriété.}
    \item \question{Montrer que la topologie induite sur
${\Qq}$ par la topologie usuelle de
${\Rr}$ n'est pas la topologie discrète, mais qu'elle possède aussi la
propriété (P).}
    \item \question{Autre exemple ?}
\end{enumerate}
}

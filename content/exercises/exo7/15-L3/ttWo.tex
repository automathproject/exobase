\uuid{ttWo}
\exo7id{6819}
\titre{exo7 6819}
\auteur{gijs}
\organisation{exo7}
\datecreate{2011-10-16}
\isIndication{false}
\isCorrection{false}
\chapitre{Espace métrique complet, espace de Banach}
\sousChapitre{Espace métrique complet, espace de Banach}
\module{Topologie}
\niveau{L3}
\difficulte{}

\contenu{
\texte{
Soit $(E, \|\cdot\|)$ un espace vectoriel normé. Le
but de cet exercice est de montrer que $E$ est complet si
et seulement si toute série absolument convergente
converge.
}
\begin{enumerate}
    \item \question{Soit $E$ complet et $(a_n)_{n\in \Nn}$ une suite dans $E$
telle que $\sum_{n=0}^\infty \|a_n\|$ converge.
Démon\-trer que la suite $s_N = \sum_{n=0}^N a_n$ est
une suite de Cauchy~; en déduire que $
\sum_{n=0}^\infty a_n$ converge.}
    \item \question{On suppose que toute série absolument convergente
converge, c'est-à-dire si $\sum_{n=0}^\infty
\|b_n\|$ converge, alors $\sum_{n=0}^\infty b_n$
converge. Soit $(a_n)_{n\in \Nn}$ une suite de Cauchy dans
$E$. Trouver une suite strictement croissante $i\mapsto n_i
\in \Nn$ telle que $\forall i : \|a_{n_{i+1}} -
a_{n_i}\| < 2^{-i}$. En déduire que
$\sum_{i=0}^\infty (a_{n_{i+1}} - a_{n_i})$
converge. Déduire de ce résultat que $E$ est complet.}
\end{enumerate}
}

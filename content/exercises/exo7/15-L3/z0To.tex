\uuid{z0To}
\exo7id{6088}
\titre{exo7 6088}
\auteur{queffelec}
\organisation{exo7}
\datecreate{2011-10-16}
\isIndication{false}
\isCorrection{false}
\chapitre{Espace vectoriel normé}
\sousChapitre{Espace vectoriel normé}
\module{Topologie}
\niveau{L3}
\difficulte{}

\contenu{
\texte{
Soit $E$ un espace vectoriel normé sur $\R$ ou $\C$.
}
\begin{enumerate}
    \item \question{Vérifier que l'application $(\lambda,x)\to \lambda x$ est continue; que
$(x,y)\to x+y$ est lipschitzienne ainsi que l'applica\-tion $x\to\Vert
x\Vert$; et que les translations et les homothéties sont des
homéomor\-phismes de
$E$.}
    \item \question{Montrer que la boule unité ouverte est
homéomorphe à
$E$ tout entier (considérer l'application $x\to {x\over 1-||x||}$).}
    \item \question{Montrer que deux
boules ouvertes de $(E,||.||)$ sont homéomorphes entre elles.}
    \item \question{Montrer que le seul sous-espace ouvert de $E$ est $E$ lui-même, et que tout
sous-espace propre est d'intérieur vide dans $E$.}
    \item \question{Montrer que l'adhérence d'un sous-espace vectoriel est encore un sous-espace
vectoriel; en déduire qu'un hyperplan de $E$ est fermé ou partout dense dans $E$.}
\end{enumerate}
}

\uuid{hZhE}
\exo7id{2404}
\titre{exo7 2404}
\auteur{mayer}
\organisation{exo7}
\datecreate{2003-10-01}
\isIndication{true}
\isCorrection{true}
\chapitre{Théorème du point fixe}
\sousChapitre{Théorème du point fixe}
\module{Topologie}
\niveau{L3}
\difficulte{}

\contenu{
\texte{
\label{exn13}
Soit $\alpha _n >0$ tel que la s\'erie $\sum_{n=1}^\infty \alpha _n $
converge. Soit $(X,d)$ un espace m\'etrique complet et $f:X\to X$
une application pour laquelle
$$ d(f^n(x),f^n(y))\leq \alpha_n d(x,y) \quad \text{pour tout } x,y\in X \;
\text {et }\; n\in \Nn\; .$$ Montrer que, sous ces conditions, $f$
poss\`ede un unique point fixe $p\in X$, que pour tout point
initial $x_0 \in X$, la suite  des it\'er\'ees $(x_n=f^n
(x_0))_{n\geq 0}$ converge vers $p$ et que la vitesse de
convergence d'une telle suite est contr\^ol\'ee par
$$ d(p,x_n) \leq \left ( \sum_{\nu =n}^\infty \alpha _\nu \right ) d(x_1,x_0) \; .$$
}
\indication{C'est à peu prés la même démonstration que pour le théorème du point fixe d'une fonction contractante.}
\reponse{
Commen\c{c}ons par l'unicité, si $x,y$ sont deux points fixes alors $f(x)=x$ et $f(y)=y$ donc la relation pour $f$ s'écrit
$$d(x,y) \le \alpha_n d(x,y) \quad \forall n \in \Nn.$$
Comme $\sum_{n\ge1}\alpha_n$ converge alors $(\alpha_n)$ tend vers $0$, donc il existe $n_0$ assez grand avec $\alpha_{n_0} < 1$, la relation devient 
$$ d(x,y) \le \alpha_{n_0} d(x,y) < d(x,y),$$
ce qui est contradictoire.
Soit $x_0 \in X$, notons $x_n = f^n(x_0)$. Alors
$$d(x_{n+1},x_n) \le \alpha_n d(x_1,x_0) \quad \forall n \in \Nn.$$
On va montrer que $(x_n)$ est une suite de Cauchy, c'est-à-dire
$$\forall \epsilon >0 \quad \exists N\in\Nn \quad
\forall n \ge N \quad \forall p \ge0\qquad d(x_{n+p},x_n) \le \epsilon.$$
Pour $n,p$ fixés, évaluons $d(x_{n+p},x_n)$.
\begin{align*}
d(x_{n+p},x_n) 
   &\le \sum_{k=n}^{n+p-1} d(x_{k+1},x_k) \\
   &\le \sum_{k=n}^{n+p-1} \alpha_k d(x_{1},x_{0}) \\
   &= d(x_{1},x_{0})\sum_{k=n}^{n+p-1} \alpha_k \\ 
\end{align*}

De plus la série $\sum_{n\ge1}\alpha_n$ converge donc la suite $(S_n)$ définie par
$S_n = \sum_{k=1}^n \alpha_k$ est de Cauchy et donc il existe $ N$ tel que
pour tout $n\ge N$ et tout $p\ge 0$ on a
$$\sum_{k=n}^{n+p-1} \alpha_k = S_{n+p-1}-S_{n-1} \le \epsilon.$$
Donc pour tout $n\ge N$ et tout $p\ge 0$ on $ d(x_{n+p},x_n) \le d(x_1,x_0)\epsilon$.
Quitte à poser $\epsilon'= d(x_1,x_0)\epsilon$, ceci prouve que $(x_n)$ est une suite de Cauchy.  
 Comme l'espace est complet alors cette suite converge, notons $x$ sa limite.

Pour tout $n\in \Nn$ nous avons
$$x_{n+1}=f(x_n).$$
\`A la limite, la suite $(x_{n+1})$ tend vers $x$, et comme $f$ est continue
(elle est $\alpha_1$-lipschitziènne : $d(f(x),f(y)) \le \alpha_1 d(x,y)$) alors
$(f(x_n))$  converge vers $f(x)$. Par unicité de la limite nous obtenons
$$x=f(x).$$
Donc $f$ possède un point fixe, qui est unique et est obtenu en partant d'un point quelconque $x_0\in X$ comme limite de $(f^n(x_0))_n$.
Il reste à estimer la vitesse de convergence, nous avons vu
$$ d(x_{n+p},x_n) \le d(x_{1},x_{0})\sum_{k=n}^{n+p-1} \alpha_k,$$
On fait tendre $p$ vers $+\infty$ dans cette inégalité alors
$$ d(x,x_n) \le d(x_{1},x_{0})\sum_{k=n}^{+\infty} \alpha_k.$$
Ce qui était l'estimation recherchée.
}
}

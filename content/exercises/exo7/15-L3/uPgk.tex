\uuid{uPgk}
\exo7id{2350}
\titre{exo7 2350}
\auteur{queffelec}
\organisation{exo7}
\datecreate{2003-10-01}
\isIndication{true}
\isCorrection{true}
\chapitre{Espace topologique, espace métrique}
\sousChapitre{Espace topologique, espace métrique}
\module{Topologie}
\niveau{L3}
\difficulte{}

\contenu{
\texte{
Soit $E=C^1([0,1],{\Rr})$. Comparer les normes 
$N_1(f)=||f||_\infty,\  N_2(f)=||f||_\infty+||f||_1,\
N_3(f)=||f'||_\infty+||f||_\infty,\  N_4(f)=||f'||_1+||f||_\infty.$
}
\indication{Les seules relations sont :
$$N_1 \le N_2 \le 2N_1 \le 2N_4 \le 2N_3.$$}
\reponse{
On montre facilement 
$$N_1 \le N_2 \le 2N_1 \le 2N_4 \le 2N_3.$$
Par contre il n'existe pas de constante $C>0$ telle que $N_3 \le CN_4$ ou $N_2 \le CN_4$.
On suppose qu'il existe $C>0$ telle que $N_3 \le CN_4$ on
regarde $f_k$ définie par $f_k(x) = x^k$, après calcul on obtient $N_3(f_k)=k+1$ et $N_4(f_k)=2$, pour $k$ suffisament grand on obtient une contradiction.
Comme $N_1$ et $N_2$ sont équivalentes on va prouver qu'il n'existe pas de constante $C>0$ telle que $N_3 \le CN_1$. On prend $g_k$, définie par
$g_k(x) = 1+\sin(2\pi k x)$. Alors $N_1(g_k) = 2$ et $N_3(g_k) = 4k$, ce qui prouve le résultat souhaité.
}
}

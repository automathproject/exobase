\uuid{k2gq}
\exo7id{6121}
\titre{exo7 6121}
\auteur{queffelec}
\organisation{exo7}
\datecreate{2011-10-16}
\isIndication{false}
\isCorrection{false}
\chapitre{Continuité, uniforme continuité}
\sousChapitre{Continuité, uniforme continuité}
\module{Topologie}
\niveau{L3}
\difficulte{}

\contenu{
\texte{
On note $\mathbb{S}^1$ le cercle unité dans ${\Rr}^2$, et $h$ l'application de
${\Rr}$ dans $\mathbb{S}^1$: 
$t\to(\cos 2\pi t, \sin 2\pi t)$.
}
\begin{enumerate}
    \item \question{Montrer que le cercle privé d'un point, $\mathbb{S}^1\backslash
\{a\}$, est homéomorphe à l'inter\-valle $]0,1[$.}
    \item \question{Montrer que $h$ est une bijection continue de $[0,1[$ sur $\mathbb{S}^1$, mais
n'est pas un homéomorphisme.}
    \item \question{Soit $f$ une application continue de $\Rr$ dans $\mathbb{S}^1\backslash
\{a\}$, cette fois plongé dans $\Cc$. Montrer que $f$ admet un ``logarithme
continu", c'est-à-dire qu'il existe $g$ continue de $\Rr$ dans $\Rr$ telle
que
$f=e^{ig}$.}
\end{enumerate}
}

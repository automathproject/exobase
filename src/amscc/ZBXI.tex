\uuid{ZBXI}
\chapitre{Fonction de plusieurs variables}
\niveau{L2}
\module{Analyse}
\sousChapitre{Extremums locaux}
\titre{Recherche d'extremum d'une fonction de deux variables}
\theme{optimisation}
\auteur{legall}
\datecreate{2024-12-02}
\organisation{exo7}

\difficulte{}
\contenu{
\texte{ 
	Soit $f : \Rr 
	^2\rightarrow \Rr $ l'application $(x,y) \mapsto 6xy+(y-x)^3.$ On note
	$\Delta =\{ (x,y)\in \Rr^2 , -1\leq x\leq y \leq 1\} .$
 }
\begin{enumerate}
	\item \question{ Dessiner $\Delta $. Montrer que $f$ est bornée et atteint ses 	bornes sur $\Delta .$}
	\indication{}
    \reponse{\textbf{Description de $\Delta$ :}
    	
    	La région $\Delta$ est définie par les inégalités :
    	\[
    	-1 \leq x \leq y \leq 1.
    	\]
    	Dans le plan $(x, y)$, $\Delta$ est la région délimitée par :
    	\begin{itemize}
    		\item Les droites verticales $x = -1$ et $x = 1$.
    		\item La droite $y = x$.
    		\item La droite horizontale $y = 1$.
    	\end{itemize}
    	Graphiquement, $\Delta$ est un triangle situé délimité par les points $(-1, -1)$, $(1, 1)$ et $(-1, 1)$.
    	
    	\medskip
    	
    	\textbf{Continuité et compacité :}
    	
    	La fonction $f$ est une fonction polynomiale en $x$ et $y$, donc continue sur $\mathbb{R}^2$. L'ensemble $\Delta$ est fermé et borné dans $\mathbb{R}^2$, donc compact.
    	
    	\medskip
    	
    	\textbf{Conclusion :}
    	
    	D'après le théorème des valeurs extrêmes, toute fonction continue sur un compact atteint ses bornes sur cet ensemble. Ainsi, $f$ est bornée sur $\Delta$ et atteint ses bornes sur $\Delta$.
    }
	\item \question{ La fonction $f$ admet-elle un minimum local ou maximum local dans l'intérieur de $\Delta$ ? Si oui, les déterminer.}
	\reponse{ Calculons les points stationnaires en résolvant $\nabla f = 0$ :
		\[
		\begin{cases}
			\frac{\partial f}{\partial x}(x,y) = 6y - 3(y - x)^2 = 0, \\
			\frac{\partial f}{\partial y}(x,y) = 6x + 3(y - x)^2 = 0.
		\end{cases}
		\]
		$\iff$
		\[
		\begin{cases}
			6y + 6x = 0, \\
			6x + 3(2x)^2 = 0.
		\end{cases}
		\]
		$\iff$
		\[
		\begin{cases}
			x = -y, \\
			6x(1 + 2x) = 0.
		\end{cases}
		\]
		$\iff$
		$(x,y) = (0,0)$ ou $(x,y) = \left(-\frac{1}{2}, \frac{1}{2}\right)$. Le point $(0,0)$ n'est pas dans l'intérieur de $\Delta$. 

		On détermine la nature en calculant la matrice hessienne de $f$ : 
		\[
		H_f(x,y) = \begin{pmatrix}
			\frac{\partial^2 f}{\partial x^2}(x,y) & \frac{\partial^2 f}{\partial x \partial y}(x,y) \\
			\frac{\partial^2 f}{\partial y \partial x}(x,y) & \frac{\partial^2 f}{\partial y^2}(x,y)
		\end{pmatrix} = \begin{pmatrix}
			6(y-x) & 6-6(y-x) \\
			6-6(y-x) & 6(y-x)
		\end{pmatrix}.
		\]
		En $\left(-\frac{1}{2}, \frac{1}{2}\right)$, $H_f\left(-\frac{1}{2}, \frac{1}{2}\right) = \begin{pmatrix} 6 & 0 \\ 0 & 6 \end{pmatrix}$, 
		de déterminant $36 > 0$  et définie positive, donc $\left(-\frac{1}{2}, \frac{1}{2}\right)$ est un minimum local de $f$ et ce minimum vaut $f\left(-\frac{1}{2}, \frac{1}{2}\right) = -\frac{1}{2}$.
		 }
    \item \question{Calculer les extrema de $f$ sur le bord de $\Delta $.}
    \indication{}
    \reponse{
    \textbf{Sur le bord $\mathbf{y = x}$, avec $\mathbf{x \in [-1, 1]}$ :}
    
    On a $f(x, x) = 6x^2 + (x - x)^3 = 6x^2$.
    
    \begin{itemize}
    	\item Le minimum est atteint en $x = 0$ : $f(0, 0) = 0$.
    	\item Le maximum est atteint en $x = \pm 1$ : $f(\pm 1, \pm 1) = 6$.
    \end{itemize}
    
    \medskip
    
    \textbf{Sur le bord $\mathbf{y = 1}$, avec $\mathbf{x \in [-1, 1]}$ :}
    
    On a $f(x, 1) = 6x \cdot 1 + (1 - x)^3 = 6x + (1 - x)^3$.
    
    Calcul de la dérivée :
    \[
    f'(x) = 6 - 3(1 - x)^2.
    \]
    En résolvant $f'(x) = 0$ :
    \[
    6 - 3(1 - x)^2 = 0 \implies (1 - x)^2 = 2 \implies x = 1 \pm \sqrt{2}.
    \]
    Seule la solution $x = 1 - \sqrt{2} \approx -0{,}4142$ appartient à $[-1, 1]$.
    
    \begin{itemize}
    	\item $f(1, 1) = 6 + 0 = 6$ (maximum).
    	\item $f(-1, 1) = -6 + 8 = 2$.
    	\item $f(1 - \sqrt{2}, 1) = 6(1 - \sqrt{2}) + (1 - (1 - \sqrt{2}))^3 = 6 - 6\sqrt{2} + (\sqrt{2})^3 = 6 - 6\sqrt{2} + 2\sqrt{2} = 6 - 4\sqrt{2} \approx 0{,}3432$ (minimum).
    \end{itemize}
    
    \medskip
    
    \textbf{Sur le bord $\mathbf{x = -1}$, avec $\mathbf{y \in [-1, 1]}$ :}
    
    On a $f(-1, y) = -6y + (y + 1)^3$.
    
    Calcul de la dérivée :
    \[
    f'(y) = -6 + 3(y + 1)^2.
    \]
    En résolvant $f'(y) = 0$ :
    \[
    -6 + 3(y + 1)^2 = 0 \implies (y + 1)^2 = 2 \implies y = -1 \pm \sqrt{2}.
    \]
    Seule la solution $y = -1 + \sqrt{2} \approx 0{,}4142$ appartient à $[-1, 1]$.
    
    \begin{itemize}
    	\item $f(-1, -1) = 6 + 0 = 6$ (maximum).
    	\item $f(-1, 1) = -6 + 8 = 2$.
    	\item $f(-1, -1 + \sqrt{2}) = 6 - 4\sqrt{2} \approx 0{,}3432$ (minimum).
    \end{itemize}
}
    \item \question{ En déduire les bornes de $f$ sur $\Delta $. }
\reponse{ D'après les calculs précédents :
	
	\begin{itemize}
		\item Le maximum de $f$ sur $\Delta$ est $f_{\text{max}} = 6$, atteint en $(1, 1)$ et $(-1, -1)$.
		\item Le minimum de $f$ sur $\Delta$ est $f_{\text{min}} = 0{,}3432$, atteint sur le bord en $(1 - \sqrt{2}, 1)$ et $(-1, -1 + \sqrt{2})$, et $f(0, 0) = 0$ à l'intérieur.
	\end{itemize}
	
	Cependant, en comparant $f\left(-\frac{1}{2}, \frac{1}{2}\right) = -\frac{1}{2}$ avec $f_{\text{min}}$, on conclut que le minimum global de $f$ sur $\Delta$ est $-\frac{1}{2}$, atteint en $\left(-\frac{1}{2}, \frac{1}{2}\right)$.
 }
\end{enumerate}
}
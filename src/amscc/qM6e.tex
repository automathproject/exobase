\uuid{qM6e} 
\titre{Calcul de la somme d'une série entière}
\chapitre{Série entière}
\niveau{L2}
\module{Analyse}
\sousChapitre{Calcul de la somme d'une série entière}
\theme{Séries entières, Sommation}
\auteur{}
\datecreate{2023-10-27} 
\organisation{}

\difficulte{}
\contenu{
	
	\texte{
		On considère la série entière : 
		$$S(x) = \sum_{n=0}^{+\infty} \frac{n}{n+2}x^n$$
	}
	
	\begin{enumerate}
		\item \question{Vérifier que : $\forall n \in \N$, $ \frac{n}{n+2} = 1 - \frac{2}{n+2}$.}
		\indication{}
		\reponse{Il suffit de voir que $\frac{n}{n+2} = \frac{n+2-2}{n+2} = \frac{n+2}{n+2} - \frac{2}{n+2} = 1 - \frac{2}{n+2} $.}
		\item \question{Déterminer le domaine de convergence $I$ de cette série entière.}
		\indication{}
		\reponse{On pose $a_n = \frac{n}{n+2}$. On utilise le critère de d'Alembert pour la série entière $\sum a_n x^n$. Soit $u_n(x) =  \frac{n}{n+2}x^n$.
			\begin{align*}
				\frac{|u_{n+1}(x)|}{|u_n(x)|} &= \frac{ (n+1) }{ (n+1)+2 } \frac{ n+2 }{n} \frac{|x^{n+1}|}{|x^{n}|} \\
				&= \frac{ (n+1)(n+2) }{n(n+3)}|x| \\
				& \sim  \frac{n^2}{n^2} |x| \quad \text{quand } n \to +\infty \\
				&\xrightarrow[n\to+\infty]{}  |x|
			\end{align*}	
			Donc le rayon de convergence est $R=1$.
			
			Étude aux bornes :
			Pour $x=1$, le terme général est $u_n(1) = \frac{n}{n+2}$. On a $\lim\limits_{n\to+\infty} u_n(1) = \lim\limits_{n\to+\infty} \frac{n}{n+2} = 1 \neq 0$. Donc la série $\sum u_n(1)$ diverge grossièrement.
			Pour $x=-1$, le terme général est $u_n(-1) = \frac{n}{n+2}(-1)^n$. On a $\lim\limits_{n\to+\infty} |u_n(-1)| = \lim\limits_{n\to+\infty} \frac{n}{n+2} = 1 \neq 0$. Donc le terme général $u_n(-1)$ ne tend pas vers $0$, et la série $\sum u_n(-1)$ diverge grossièrement.
			
			Par conséquent, le domaine de convergence est $I = ]-1;1[$. 
		}
		\item \question{Calculer la valeur de $\displaystyle x^2 \sum_{n=0}^{+\infty} \frac{x^n}{n+2}$ pour tout $x \in I$.}
		\indication{Pensez à un changement d'indice et à la série du logarithme.}
		\reponse{Pour tout $x \in I = ]-1;1[$:
			\begin{align*}
				x^2 \sum_{n=0}^{+\infty} \frac{x^n}{n+2} &= \sum_{n=0}^{+\infty}\frac{x^{n+2}}{n+2} \\
				&= \sum_{k=2}^{+\infty}\frac{x^{k}}{k} \quad (\text{en posant } k=n+2) \\
				&= \left( \sum_{k=1}^{+\infty}\frac{x^{k}}{k} \right) - x \\
				&= -\ln(1-x) -x
			\end{align*}	
		}
		\item \question{En déduire le calcul de la somme $S(x)$ pour tout $x \in I$.}
		\indication{Utiliser les questions précédentes.}
		\reponse{D'après la question 1, $\frac{n}{n+2} = 1 - \frac{2}{n+2}$. Donc pour $x \in I = ]-1;1[$ :
			\begin{align*}
				S(x) &= \sum_{n=0}^{+\infty} \left(1 - \frac{2}{n+2}\right)x^n \\
				&=  \sum_{n=0}^{+\infty} x^n - 2\sum_{n=0}^{+\infty}\frac{x^{n}}{n+2}
			\end{align*}
			On sait que $\sum_{n=0}^{+\infty} x^n = \frac{1}{1-x}$ pour $x \in ]-1;1[$.
			
			Pour $x \neq 0$, d'après la question précédente, $\sum_{n=0}^{+\infty}\frac{x^{n}}{n+2} = \frac{-\ln(1-x)-x}{x^2}$.
			Donc, pour $x \in ]-1;1[$ et $x \neq 0$ :
			\begin{align*}
				S(x) &= \frac{1}{1-x} - 2 \left( \frac{-\ln(1-x)-x}{x^2} \right) \\
				&= \frac{1}{1-x} + \frac{2\ln(1-x)}{x^2} + \frac{2x}{x^2} \\
				&= \frac{1}{1-x} + \frac{2\ln(1-x)}{x^2} + \frac{2}{x} 
			\end{align*}	
			Pour $x=0$:
			$S(0) = \frac{0}{0+2}x^0 + \sum_{n=1}^{+\infty} \frac{n}{n+2}0^n = 0 \cdot 1 + 0 = 0$.
			
			Vérifions si la formule est cohérente en $x=0$ via un développement limité.
			Pour $x \to 0$:
			$\frac{1}{1-x} = 1+x+x^2+o(x^2)$
			$\ln(1-x) = -x - \frac{x^2}{2} - \frac{x^3}{3} + o(x^3)$
			$\frac{2\ln(1-x)}{x^2} = \frac{2(-x - x^2/2 - x^3/3 + o(x^3))}{x^2} = -\frac{2}{x} - 1 - \frac{2x}{3} + o(x)$
			$\frac{2}{x}$
			$S(x) = (1+x+x^2) + (-\frac{2}{x} - 1 - \frac{2x}{3}) + \frac{2}{x} + o(x)$
			$S(x) = 1+x+x^2 - \frac{2}{x} - 1 - \frac{2x}{3} + \frac{2}{x} + o(x)$
			$S(x) = x - \frac{2x}{3} + x^2 + o(x) = \frac{x}{3} + x^2 + o(x)$.
			Le terme constant $\frac{0}{0+2} = 0$. Le terme en $x$ est $\frac{1}{1+2}x = \frac{x}{3}$. Le terme en $x^2$ est $\frac{2}{2+2}x^2 = \frac{x^2}{2}$.
			Il y a une cohérence.
			$S(x) = \sum_{n=0}^{\infty} \frac{n}{n+2}x^n = 0 \cdot x^0 + \frac{1}{3}x^1 + \frac{2}{4}x^2 + \frac{3}{5}x^3 + \dots = \frac{1}{3}x + \frac{1}{2}x^2 + \frac{3}{5}x^3 + \dots$
			La formule donne $S(x) \xrightarrow{x \to 0} \frac{x}{3}$.
			Effectivement, $S(0) = 0$. La formule trouvée pour $x \neq 0$ n'est pas définie en $x=0$.
			Donc, on écrit :
			$S(x) = \begin{cases} \frac{1}{1-x} + \frac{2\ln(1-x)}{x^2} + \frac{2}{x} & \text{si } x \in ]-1;1[ \setminus \{0\} \\ 0 & \text{si } x=0 \end{cases}$
		}
	\end{enumerate}
}
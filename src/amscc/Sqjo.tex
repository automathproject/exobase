\uuid{Sqjo}
\chapitre{Statistique}
\niveau{L2}
\module{Probabilité et statistique}
\sousChapitre{Tests d'hypothèses, intervalle de confiance}
\titre{Test de consommation }
\theme{tests d'hypothèses}
\auteur{}
\datecreate{2023-12-06}
\organisation{AMSCC}

\difficulte{}
\contenu{
En France, la moyenne de consommation électrique annuelle individuelle est de 1400 kWh. On s'intéresse à la consommation de certains quartiers de la ville de Nouille-Orque.

\paragraph{Partie 1 :}
Des relevés ont été effectués dans le quartier Manatane  (échantillon 1 du fichier de données) et on se demande si la consommation de ce quartier peut être considérée comme conforme à la moyenne nationale ou si elle est significativement supérieure, auquel cas la politique du quartier devra être modifiée. 

\begin{enumerate}
	\item \`A l'aide d'un test d'hypothèse, répondre à cette question avec un risque de première espèce $\alpha = 5\%$.
	\item \`A partir de quelle moyenne observée doit-on rejeter l'hypothèse que la consommation de ce quartier est conforme à la moyenne nationale ? On note cette valeur $\mu_{critique}$.
	\item Calculer la puissance de ce test pour une hypothèse alternative $H_1 \colon \mu = 1500$.
\end{enumerate}

\paragraph{Partie 2 :}
D'autres relevés ont été effectués dans le quartier Brouqueline (échantillon 2 du fichier de données).

\begin{enumerate}
	\item La consommation de Brouqueline est-elle significativement différente de la moyenne nationale ? On répondra avec un risque de première espèce de 5\%.
	\item La différence de consommation entre ces deux quartiers est-elle significative ?   On répondra avec un risque de première espèce de 5\%.
\end{enumerate}
}
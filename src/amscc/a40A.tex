\uuid{a40A}
\chapitre{Polynôme, fraction rationnelle}
\niveau{L1}
\module{Algèbre}
\sousChapitre{Fraction rationnelle}
\titre{ Décomposition d'une fraction rationnelle }
\theme{polynômes, fractions rationnelles}
\auteur{}
\datecreate{2024-01-31}
\organisation{AMSCC}

\difficulte{}
\contenu{
    Soit le polynôme $P \in \R[X]$ défini par : $$P(X) = (X^{2}+1)^2-X^2.$$

    \begin{enumerate}
    \item Démontrer l'égalité $P(X)=(X^2+X+1)(X^2-X+1)$.
    \item Déterminer les racines réelles ou complexes des polynômes $X^2+X+1$ et $X^2-X+1$.
    \item Déterminer des constantes $a,b,c,d \in \R$ telles que : 
    $$\frac{1}{P(X)}=\frac{aX+b}{X^2+X+1}+\frac{cX+d}{X^2+X+1}.$$
    \end{enumerate}
}
\uuid{CLjr}
\titre{Multiplicité d'une racine et factorisation d'un polynôme}
\theme{polynômes}
\auteur{}
\datecreate{2023-01-23}
\organisation{AMSCC}
\contenu{


\texte{ Déterminer l'ordre de multiplicité de la racine $x_0$ du polynôme $P$, et en déduire la factorisation du polynôme, dans les cas suivants : }

\begin{enumerate}
	\item  \question{ $P(X)=X^4-X^3-3 X^2+5 X-2 \quad x_0=1$.}
	\indication{Calculer $P(1)$, puis $P'(1)$...}
\reponse{
$$
P(X)=X^4-X^3-3 X^2+5 X-2 .
$$ 

 On a :

\begin{align*}
P(1) & =1-1-3+5-2=0 \\
P^{\prime}(X) & =4 X^3-3 X^2-6 X+5 \\
P^{\prime}(1) & =4-3-6+5=0 \\
P^{\prime \prime}(X) & =12 X^2-6 X-6 \\
P^{\prime \prime}(1) & =12-6-6=0 \\
P^{(3)}(X) & =24 X-6 \\
P^{(3)}(1) & \neq 0
\end{align*}

Ainsi $x_0=1$ est racine d'ordre 3 de $P$, donc par définition il existe un polynôme $Q$ tel que : 
$$P(X) = (X-1)^3 Q(X)$$

Or $P$ est d'ordre $4$ donc $Q$ est de degré $1$ De plus, le coefficient de plus haut degré de $P$ est $1$, donc celui de $Q$ aussi et il est existe $a \in \mathbb{C}$ tel que $Q(X) = X-a$. Ainsi :
$$P(X)=(X-1)^3 (X-a)$$
Or $P(0) = -1 \times (-a) = a$ d'une part et $P(0) = -2$ d'après la définition de $P$ donc $a=-2$ et :
$$P(X)=(X-1)^3 (X+2)$$
}

\item \question{ $P(X)=X^3-iX^2+X-i \quad x_0=i$. }
	\indication{Calculer $P(i)$, puis $P'(i)$...}
\reponse{ $$
P(X)=X^3-iX^2+X-i .
$$

On a :

\begin{align*}
P(i) & =i^3-i . i^2+i-i=0 \\
P^{\prime}(X) & =3 X^2-2 i X+1 \\
P^{\prime}(i) & =3 i^2-2 . i . i+1=0 \\
P^{\prime \prime}(X) & =6 X-2 i \\
P^{\prime \prime}(i) & =6 i-2 i \neq 0
\end{align*}

Ainsi $x_0=1$ est racine double de $P$, donc par définition il existe $a \in \mathbb{C}$ tel que :


$$P(X)=  (X-i)^2 (X-a)$$

Or $P(0) = a = -i$ 


d'où 
$$P(X) = (X-i)^2(X+i)$$ }

\end{enumerate}}

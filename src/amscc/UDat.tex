\uuid{UDat}
\chapitre{Probabilité discrète}
\niveau{L2}
\module{Probabilité et statistique}
\sousChapitre{Loi, indépendance, loi conditionnelle}
\titre{Indépendance}
\theme{probabilités}
\auteur{}
\datecreate{2023-01-24}
\organisation{AMSCC}
\difficulte{}
\contenu{

\texte{ Soient $A_1, \ldots, A_n$ une suite de $n$ événements d'un espace probabilisé $(\Omega, P)$. On les suppose mutuellement indépendants et de probabilités respectives $p_i=P\left(A_i\right)$. } 

\question{Donner une expression simple de $P\left(A_1 \cup \cdots \cup A_n\right)$ en fonction de $p_1, \ldots, p_n$. }

\reponse{ Si les événements $A_1, \ldots, A_n$ sont mutuellement indépendants, les événements complémentaires $\overline{A_1}, \ldots, \overline{A_n}$ le sont aussi. On en déduit que
$$
\begin{aligned}
	P\left(A_1 \cup \cdots \cup A_n\right) & =1-P\left(\overline{A_1} \cap \cdots \cap \overline{A_n}\right) \\
	& =1-\prod_{i=1}^n P\left(\overline{A_i}\right) \\
	& =1-\prod_{i=1}^n\left(1-p_i\right) .
\end{aligned}
$$
 }}

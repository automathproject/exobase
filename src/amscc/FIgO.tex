\uuid{FIgO}
\chapitre{Probabilité discrète}
\niveau{L2}
\module{Probabilité et statistique}
\sousChapitre{Loi, indépendance, loi conditionnelle}
\titre{Loi d'un couple}
\theme{loi conjointe}
\auteur{}
\datecreate{2023-02-07}
\organisation{AMSCC}
\difficulte{}
\contenu{

\texte{ 	On considère un fast food muni de $N$ comptoirs. Le nombre de personnes se présentant en une heure au fast food est $X$ suivant une loi de poisson $\mathscr{P}(\lambda)$ avec $\lambda >0$. Pour commander son sandwich, chaque personne choisit un comptoir au hasard, de manière équiprobable. On note $Y$ le nombre de personnes se présentant au comptoir numéro 1. On note $Y_i$ le nombre de personnes se présentant au comptoir numéro $i$.  }
\begin{enumerate}
	\item \question{ Quelle est  la loi de $Y_1$ ? }
	\item \question{ Calculer le coefficient de corrélation linéaire de $(Y_i,Y_j)$. }
	\indication{On pourra tout d'abord calculer la loi $Y$ conditionnée par $X$, puis la loi mutuelle $(X,Y)$. }
\end{enumerate}
}

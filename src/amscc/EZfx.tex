\uuid{EZfx}
\titre{Lois pour les statistiques}
\theme{loi normale, loi du chi2, loi de Student}

\auteur{}
\datecreate{2022-08-25}
\organisation{AMSCC}
\contenu{

\texte{La bestiole est un animal dont le poids est distribué selon une loi normale de moyenne 100 g et d'écart-type 5 g.

 On prélève un échantillon aléatoire de 16 bestioles. On note $X_i$ le poids de la bestiole numéro $i$ ($1 \leq i \leq 16$).}

 \begin{enumerate}
  \item \question{Déterminer la loi suivie par  $$\overline{X}=\frac{\sum\limits_{i=1}^{16} X_i}{16}$$}
 \reponse{Par propriétés de somme de lois normales, on obtient que $\overline{X}$ suit une loi normale. Il reste à calculer $\mathbb{E}(\overline{X}) = \frac{1}{16} \times 16 \times 100 = 100$ et $V(\overline{X}) = \frac{1}{16^2} \times 16 \times 5^2 = \frac{5^2}{16}$. On en déduit que $\overline{X}$ suit une loi normale $\mathcal{N}(100, \sigma = \frac{5}{4})$. }
  \item \question{Déterminer la loi suivie par $$Q=\frac{\sum\limits_{i=1} ^{16} (X_i-100)^2}{25}$$
Déterminer le réel $q$ tel que $\PP(Q > q) = 0.05$.}
\reponse{On réécrit $Q= \sum\limits_{i=1} ^{16} \left(\frac{X_i-100}{5}\right)^2$ or $\frac{X_i-100}{5}$ suit une loi $\mathcal{N}(0,1)$ et les variables $X_i$ sont indépendantes donc par définition, $Q$ suit une loi $\chi^2(16)$. 
On cherche maintenant $q$ tel que $\PP(Q \leq q) = 0.95$ dans la table de valeurs soit $q = 26.296$.
 }
  \item \question{Déterminer la loi suivie par $$V=\frac{\sum\limits_{i=1} ^{16} (X_i-\overline{X})^2}{25}$$
  puis déterminer le réel $v$ tel que $\PP(V > v) = 0.05$.}
\reponse{D'après le théorème de Fisher, $V$ suit une loi  $\chi^2(15)$. Par lecture de table, on trouve $\PP(V \leq v) = 0.95$ pour $v=24.996$.}
  \item \question{Déterminer la loi suivie par $$W=\frac{(\overline{X}-100)4\sqrt{15}}{\sqrt{\sum\limits_{i=1} ^{16} (X_i-\overline{X})^2}}$$
    Déterminer le réel $w$ tel que $\PP(W > w) = 0.05$.}
\reponse{D'après la question 1, la variable $\frac{\overline{X}-100}{\frac{5}{4}}$ suit une loi $\mathcal{N}(0,1)$.  On réécrit maintenant :
$$W = \frac{\frac{\overline{X}-100}{\frac{5}{4}} \times \frac{5}{4} \times 4\sqrt{15}}{\sqrt{\sum\limits_{i=1} ^{16} (X_i-\overline{X})^2}} = \frac{\frac{\overline{X}-100}{\frac{5}{4}}}{\frac{\sqrt{\sum\limits_{i=1} ^{16} (X_i-\overline{X})^2}}{5\sqrt{15}}} = \frac{\frac{\overline{X}-100}{\frac{5}{4}}}{\frac{\sqrt{V}}{\sqrt{15}}} $$
	Or $V$ suit une  $\chi^2(15)$ d'après la question précédente. Donc par définition, $W$ suit une loi de Student $St(15)$. 
}
 \end{enumerate}
}

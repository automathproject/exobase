\uuid{DkWo}
\chapitre{Statistique}
\niveau{L2}
\module{Probabilité et statistique}
\sousChapitre{Tests d'hypothèses, intervalle de confiance}
\titre{ Effet d'un traitement }
\theme{statistiques, tests d'hypothèses}
\auteur{}
\datecreate{2023-11-27}
\organisation{AMSCC}
\difficulte{}
\contenu{

\texte{ 	On a mesuré les dimensions (en cm$^2$) d'une tumeur chez des souris. Certaines sont traitées avec une substance, d'autres non. On a obtenu les résultats suivants :

\begin{center}
	
	\begin{tabular}{|c|c|c|c|}
		\hline  & taille échantillon & moyenne & écart type \\ 
		\hline souris témoins & 20 & 7.075 cm$^2$ & 0.576 cm$^2$ \\ 
		\hline souris traitées & 18 & 5.850  cm$^2$ & 0.614 cm$^2$ \\ 
		\hline 
	\end{tabular} 
	
\end{center} }


\question{ A partir de ces résultats, peut-on affirmer que le traitement a permis de réduire significativement la taille des tumeurs ? }

\reponse{ \href{https://stcyrterrenetdefensegouvf-my.sharepoint.com/:x:/g/personal/maxime_nguyen_st-cyr_terre-net_defense_gouv_fr/EYlCPJY62BdMpTYscg1T6zABc9_WCIR_pzt6XSOajALYhw?e=u1SXuH&nav=MTVfezAwMDAwMDAwLTAwMDEtMDAwMC0wNjAwLTAwMDAwMDAwMDAwMH0}{lien vers tableur} }
}
\uuid{6YTv}
\titre{ Calcul d'une somme de série entière}
\theme {séries entières}
\auteur{ }
\datecreate{2023-06-01}
\organisation{ AMSCC }

\contenu{
\question{  À l'aide des développements en séries entières des fonctions usuelles, calculer les développements en séries entières en $0$ des fonctions réelles suivantes:
\[ f(x)=\frac{1}{ 2+3x}, \qquad g(x)=\ln \Big(\frac{1+x}{1-x}\Big), \qquad h(x)=\frac{1}{\sqrt{1-x^2}},\]
et déterminer le rayon de convergence de ces développements. }
\reponse{ 
	On écrit les développements sur l'intervalle $]-R;R[$ où $R$ est le rayon de convergence :
	\begin{itemize}
		\item $\displaystyle\forall x \in \left]\frac{-2}{3};\frac{2}{3}\right[, \qquad f(x)=\frac{1}{2}\frac{1}{1-\left(-\frac{3}{2}x\right)} = \frac{1}{2}\sum_{n=0}^{+\infty} (-1)^n\left( \frac{3}{2}x\right)^n$
		\item $\displaystyle\forall x \in ]-1;1[, \qquad g(x)=\ln(1+x) - \ln(1-x) = \sum_{n=1}^{+\infty} \frac{(-1)^{n+1}}{n}x^n + \sum_{n=1}^{+\infty} \frac{x^n}{n} = \sum_{n=1}^{+\infty} \frac{(-1)^{n+1}+1}{n}x^n = \sum_{n=1}^{+\infty} \frac{2}{2n-1}x^{2n-1}$ 
		\item $\displaystyle\forall x \in ]-1;1[, \qquad h(x)=\sum_{n=0}^{+\infty} \frac{1}{n!}\left(\frac{1}{2}\times \frac{3}{2}\times \cdots \times \left(\frac{1}{2}+n-1\right) \right) x^{2n}$
	\end{itemize}
}
}
\uuid{2sPY}
\titre{Résolution de système linéaire à paramètre, homogène et non homogène}
\theme{systèmes linéaires}
\auteur{}
\datecreate{2023-01-17}
\organisation{AMSCC}
\contenu{

\texte{ On considère les systèmes suivants :
$$
\left(\mathcal{S}_m\right)\left\{\begin{aligned}
(m-1) x&+(1-m) y & +\left(m^2-1\right) z & =0 \\
&m y & +z & =0 \\
(1-m) x&-y & -z & =0
\end{aligned}\right.
$$
et :
$$
\left(\mathcal{S}_m^{\prime}\right)\left\{\begin{aligned}
(m-1) x&+(1-m) y+&\left(m^2-1\right) z & =1 \\
&my & +z & =1 \\
(1-m)x &-my & -z & =-2
\end{aligned}\right.
$$
où $m \in \mathbb{R}$ est un paramètre réel.  }

\begin{enumerate}
	\item \question{ Déterminer, suivant la valeur de $m$, le déterminant de la matrice :
$$
A_m=\left(\begin{array}{ccc}
m-1 & 1-m & m^2-1 \\
0 & m & 1 \\
1-m & -1 & -1
\end{array}\right) .
$$ }
\reponse{ 
$$
\begin{aligned}
\left|\begin{array}{ccc}
m-1 & 1-m & m^2-1 \\
0 & m & 1 \\
1-m & -1 & -1
\end{array}\right| & =(m-1) \cdot\left|\begin{array}{ccc}
1 & -1 & m+1 \\
0 & m & 1 \\
1-m & -1 & -1
\end{array}\right| \\
& =(m-1) \cdot[(m-1) \cdot m \cdot(m+1)+1-m+(m-1)] \\
& =m \cdot(m-1)^2 \cdot(m+1)
\end{aligned}
$$
Donc $\mathrm{det} (A_m ) \neq 0$ si et seulement si $m \notin\{-1,0,1\}$.   }

\item \question{ En déduire les valeurs du paramètre $m$ pour lesquelles le système $\left(\mathcal{S}_m\right)$ admet des solutions non nulles et, dans ces cas, résoudre le système. }

\reponse{ Le système $\left(\mathcal{S}_m\right)$ est homogène. Il admet des solutions non nulles si et seulement si $\mathrm{det} (A_m )= 0 \iff m \in\{-1,0,1\}$.

Pour $m=0$ :
$$
\begin{aligned}
& \left(\mathcal{S}_0\right) \Leftrightarrow\left\{\begin{array} { r l } 
{ - x + y - z } & { = 0 } \\
{ z } & { = 0 } \\
{ x - y - z } & { = 0 }
\end{array} \Leftrightarrow \left\{\begin{array}{l}
x=y \\
z=0
\end{array}\right.\right. \\
& \text { Sol }=\{(x, x, 0) \in \mathbb{R}^3 \, \mid \, x \in \mathbb{R}\} \\
&
\end{aligned}
$$
Pour $m=1$ :
$$
\begin{aligned}
& \left(\mathcal{S}_1\right) \Leftrightarrow\left\{\begin{array} { r l } 
{ 0 } & { = 0 } \\
{ y + z } & { = 0 } \\
{ - y - z } & { = 0 }
\end{array} \Leftrightarrow \left\{\begin{array}{l}
x \text { quelconque } \\
y=-z
\end{array}\right.\right. \\
& \text { Sol }=\{(x, y,-y) \in \mathbb{R}^3 \, \mid \, x, y \in \mathbb{R}\} \\
&
\end{aligned}
$$
Pour $m=-1$ :
$$
\begin{aligned}
&\left(\mathcal{S}_{-1}\right) \Leftrightarrow\left\{\begin{array}{r}
-2 x+2 y=0 \\
-y+z=0 \\
2 x-y-z=0
\end{array}\right. \\
& \Leftrightarrow\left\{\begin{array}{l}
x=y \\
y=z
\end{array}\right. \\
& \text { Sol }=\{(x, x, x) \in \mathbb{R}^3 \, \mid \, x \in \mathbb{R}\}
\end{aligned}
$$ }
\item \question{ Trouver les solutions du système $\left(\mathcal{S}_m^{\prime}\right)$ pour $m=2$ et $m=-1$. }
\reponse{ $$
\left(\mathcal{S}_2^{\prime}\right) \Leftrightarrow\left\{\begin{aligned}
x-y+3 z & =1 \\
2 y+z & =1 \\
-x-y-z & =-2
\end{aligned}\right.
$$
On a $\operatorname{det} A_2=2 \cdot(2-1)^2 \cdot(2+1)=6 \neq 0$ donc le système est de CRAMER, il admet une unique solution que l'on détermine par les formules de CRAMER :
$$
\begin{aligned}
x=\frac{\left|\begin{array}{ccc}
1 & -1 & 3 \\
1 & 2 & 1 \\
-2 & -1 & -1
\end{array}\right|}{6}=\frac{9}{6}=\frac{3}{2} \quad y & =\frac{\left|\begin{array}{ccc}
1 & 1 & 3 \\
0 & 1 & 1 \\
-1 & -2 & -1
\end{array}\right|}{6}=\frac{3}{6}=\frac{1}{2} \quad z=\frac{\left|\begin{array}{ccc}
1 & -1 & 1 \\
0 & 2 & 1 \\
-1 & -1 & -2
\end{array}\right|}{6}=\frac{0}{2}=0 \\
& \Rightarrow \mathrm{Sol}=\left\{\left(\frac{3}{2}, \frac{1}{2}, 0\right)\right\}
\end{aligned}
$$
$$
\left(\mathcal{S}_1^{\prime}\right) \Leftrightarrow\left\{\begin{array} { r l } 
{ - 2 x + 2 y } & { = 1 } \\
{ - y + z } & { = 1 } \\
{ 2 x - y - z } & { = - 2 }
\end{array} \Leftrightarrow \left\{\begin{array}{rl}
2 x-y-z & =-2 \\
-y+z & =1 \\
y-z & =-1
\end{array}\right.\right.
$$

Donc $$S=\left\{\left(z-\frac{3}{2}, z-1, z\right) \in \mathbb{R}^3 \, \mid \, z \in \mathbb{R}\right\}$$ }
\end{enumerate}}

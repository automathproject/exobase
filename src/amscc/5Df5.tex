\uuid{5Df5}
\chapitre{Probabilité continue}
\niveau{L2}
\module{Probabilité et statistique}
\sousChapitre{Loi normale}
\titre{ Distribution normale }
\theme{loi normale}
\auteur{}
\datecreate{2022-09-22}
\organisation{AMSCC}
\difficulte{}
\contenu{

\texte{ Le nombre de déjeuners servis chaque jour ouvrable par un restaurant d'entreprise est une variable aléatoire $X$ suivant une loi normale de moyenne $\mu$ et d'écart-type $\sigma$ telle que :
	$$\PP(X \geq 1522) = 0{,}33 \quad \text{ et } \quad \PP(X \leq 1598) = 0{,}975$$ }
\question{ 	Quel est le nombre minimal de repas qui doivent être préparés chaque midi si l'on veut avoir 99 chances sur 100 de satisfaire la demande ? }
\reponse{Commençons par déterminer $m$ et $\sigma$ grâce aux données de l'énoncé:
	\begin{align*}
		&\p(X\geq 1522)=0.33 \ \Leftrightarrow \ \p\left(\frac{X-m}{\sigma}\leq \frac{1522-m}{\sigma}\right)=0.67
		\ \Leftrightarrow \ \frac{1522-m}{\sigma}=0.44 \\
		&\p(X\leq 1598)=0.975 \ \Leftrightarrow \ \p\left(\frac{X-m}{\sigma}\leq \frac{1598-m}{\sigma}\right)=0.975
		\ \Leftrightarrow \ \frac{1598-m}{\sigma}=1.96
	\end{align*}
	Il vient ainsi
	\begin{align*}
		\begin{cases}
			1522-m=0.44\sigma \\
			1598-m=1.96\sigma
		\end{cases}
	\end{align*}
	soit $\sigma=50$ et $m=1500$. Donc $X\sim \mathcal{N}(1500,\sigma=50)$.
	\vspace{1em}
	
	On note $n$ le nombre de repas à prévoir pour satisfaire $99$\% des demandes. Par définition: $\p(X\leq n)=0.99$, ce qui donne
	\[ \p\left(\frac{X-1500}{50}\leq \frac{n-1500}{50}\right)= 0.99\]
	soit $\frac{n-1500}{50}=2.33$ et finalement $n=1617$.
	Il faut donc prévoir $1617$ repas au minimum pour satisfaire la demande dans $99$\% des cas.
	
}}

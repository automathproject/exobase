\uuid{g19x}
\chapitre{Matrice}
\niveau{L1}
\module{Algèbre}
\sousChapitre{Inverse, méthode de Gauss}
\titre{Méthode itérative d'inversion d'une matrice}
\theme{}
\auteur{}
\datecreate{2024-04-29}
\organisation{AMSCC}
\difficulte{}
\contenu{

\texte{ Soient $n \in \mathbb{N}$ tel que $n \geq 3$ et $b \in \mathbb{R}^n$, de composantes $(b_1, \ldots, b_n)$. On cherche $x \in \mathbb{R}^n$, de composantes $(x_1, \ldots, x_n)$, solution de :

$$
\begin{cases}	
	4x_1 + x_2 &= b_1, \\
	x_{i-1} + 4x_i + x_{i+1} &= b_i, \quad \forall i \in [\![ 2, n - 1 ]\!] \\	
	x_{n-1} + 4x_n &= b_n.	
\end{cases}
$$ }

\begin{enumerate}
	\item \question{ Montrer que le syst\`eme ci-dessus peut s'\'ecrire sous la forme $Ax=b$ avec une matrice $A$ que l'on donnera pour $n=4$. %Donner un code Python qui permet de générer la matrice $A$ pour $n$ quelconque. 
	}
	\reponse{Soit $x$ le vecteur $\begin{pmatrix}
x_1 \\ x_2 \\ \vdots{} \\ x_n \end{pmatrix},$ b le vecteur  $\begin{pmatrix}
b_1 \\ b_2 \\ \vdots{} \\ b_n \end{pmatrix},$ le syst\`eme proposé est donc équivalent \`a l'égalité matricielle $Ax=b$ avec 
$A=$ $\begin{pmatrix} 4 & 1 & 0 & 0 \\
1 & 4 & 1 & 0 \\
0 & 1 & 4 & 1 \\
0 & 0 & 1 & 4 
\end{pmatrix}.
$
}
	\item On suppose, dans cette question uniquement, que $b = 0$.
	
	\begin{enumerate}
		\item \question{  Montrer que : $\forall i \in [\![1, n]\!]\,, \quad 4 |x_i| \leq 2 \|x\|_\infty$. }
  \reponse{Pour tout $i \in [\![1, n]\!]\,,$ on a:\\
$4 |x_i|\leq |x_{i-1}|+|x_{i+1}|\leq 2 \|x\|_\infty,$
avec $4 |x_1|\leq |x_2|\leq \|x\|_\infty$ et $4 |x_{n}|\leq |x_2|\leq \|x\|_\infty,$
d'o\`u le r\'esultat. }
		\item \question{ En déduire que $x = 0$. }
  \reponse{Ainsi on a pour tout $i \in [\![1, n]\!]\,,$ $|x_i|\leq \frac{1}{2}\|x\|_{\infty}$. Or si $\|x\|_{\infty}\neq 0,$ il existe $i_0  \in \{1, \ldots, n\}$ tel que $\|x\|_{\infty}=|x_{i_{0}}|\neq 0$ ce qui contredit l'in\'egalit\'e pr\'ec\'edente.}
	\end{enumerate}
	
	
	\item \question{ Montrer que dans le cas d’un second membre quelconque $b$, il existe une unique $x \in \mathbb{R}^n$ solution du système linéaire. }
	\reponse{On a montr\'e \`a la question précédente que $Ker(A)=0$ avec $A$ la matrice carr\'ee d\'efinie \`a la question 2). Ainsi la matrice $A$ est inversible et la solution du syst\`eme est $x=A^{-1}b$.}
	\item \texte{ Afin de résoudre le système, on considère la méthode itérative suivante : $x^{(0)} = 0 \in \mathbb{R}^n$ et
	
	$$	
	\begin{cases}	
		x^{(k+1)}_1 = \alpha x^{(k)}_1 + \frac{\alpha - 1}{4} (x^{(k)}_2 - b_1), \\	
		x^{(k+1)}_i = \alpha x^{(k)}_i + \frac{\alpha - 1}{4} (x^{(k)}_{i-1} + x^{(k)}_{i+1} - b_i), & \forall i \in [\![ 2, n - 1 ]\!] \\	
		x^{(k+1)}_n = \alpha x^{(k)}_n + \frac{\alpha - 1}{4} (x^{(k)}_{n-1} - b_n).	
	\end{cases}
	$$
	
	avec pour paramètre $\alpha \in \mathbb{R}$.  }
	
	\begin{enumerate}
		
		\item \question{ Montrer que pour tout $\alpha \in \mathbb{R}$, on a
		
		$$\|x^{(k+1)} - x\|_\infty \leq \left(|\alpha| + \frac{|\alpha - 1|}{2}\right) \|x^{(k)} - x\|_\infty.$$ }
		\reponse{Pour tout $i \in \{2, \ldots, n-1\},$ on a:
    $$|x^{(k+1)}_{i}-x_i|=|\alpha x_i^{(k)}+\frac{(\alpha-1)}{4}(x_2^{(k)}-b_1)-x_1|\leq |\alpha(x_1^{(k)}-x_1)+\alpha x_1-x_1+(\frac{\alpha-1}{4})(b_i-x_{i+1}-x_{i-1})|$$
$$\leq |\alpha (x_1^{(k)}-x_1)+(\frac{\alpha-1}{4})(b_i-x_{i+1}-x_{i-1})+(\frac{\alpha-1}{4})(x_{i+1}^{(k)}-x_{i+1}+x_{i+1}^{(k)}-x_{i-1})| $$ 
$$\leq |\alpha|\times\|x^{(k)}-x\|_{\infty}+|\frac{\alpha-1}{4}|\times 2 \times \|x^{(k)}-x\|_{\infty}\leq \big(\,|\alpha|+|\frac{\alpha-1}{2}|\,\big)\|x^{(k)}-x\|_{\infty}.$$
Pour $j=2$ et $j=n$ on a la majoration suivante:
  $$|x^{(k+1)}_{j}-x_j|\leq (|\alpha|+|\frac{\alpha-1}{4}|)\,\|x^{(k)}-x\|_{\infty}.$$
Le passage \`a la borne sup\'erieure \`a gauche de l'in\'egalit\'e ach\`eve la preuve.}
		\item \question{ Trouver $\alpha_{\text{min}}, \alpha_{\text{max}} \in \mathbb{R}$, tels que $\alpha \in ]\alpha_{\text{min}}, \alpha_{\text{max}}[$ si et seulement si $|\alpha| + \frac{|\alpha - 1|}{2} < 1$. }
		\reponse{La condition $|\alpha|>1$ n'est pas compatible avec l'in\'egalit\'e $|\alpha| + \frac{|\alpha - 1|}{2} < 1$ on cherche l'intervalle $]\alpha_{min};\alpha_{max}[$ dans l'intervalle $]-1;1[.$
Or pour $\alpha \in [0;1]$ on a: $|\alpha| + \frac{|\alpha - 1|}{2} < 1 \Longleftrightarrow \alpha<1$ et pour $\alpha\in ]-1;0[$ on a: $-\alpha+\frac{1-\alpha}{2}<1 \Longleftrightarrow -3\alpha<1 \Longleftrightarrow \alpha>-\frac{1}{3}$.\\
Ainsi l'intervalle cherch\'e est $]-\frac{1}{3};1[$.       }
		\item \question{ Montrer que la méthode itérative converge vers $x$ pour $\alpha \in ]\alpha_{\text{min}}, \alpha_{\text{max}}[$. }
		\reponse{Pour $\alpha \in ]\frac{1}{3},1[,$ on a par r\'ecurrence pour tout $k\geq 1$:
        $$\|x^{(k)}-x\|_{\infty}\leq \big(|\alpha|+|\frac{\alpha-1}{4}|\big)^{k}\,\|x^{(0)}-x\|_{\infty},$$
et la convergence vers $0$ est assur\'ee par la condition sur $\alpha$. }
		\item\question{  Trouver $\alpha_0 \in \mathbb{R}$ qui minimise la quantité $|\alpha| + \frac{|\alpha - 1|}{2}.$ Que peut-on d\'eduire de la convergence du syst\`eme pour ce $\alpha_0$ ? Quelle méthode du cours reconnaissez-vous ? }
		\reponse{Le minimum de la quantité $|\alpha| + \frac{|\alpha - 1|}{2}$ est le minimum sous la contrainte $|\alpha| + \frac{|\alpha - 1|}{2} < 1$. Ce minimum vaut $\frac{1}{2}$ et est atteint en $\alpha_{0}=0$. En effet il suffit de minimiser $\frac{\alpha+1}{2}$ sur $]0;1[$ et $\frac{-3\alpha+1}{2}$ sur $]-\frac{1}{3};1[$. On reconnait pour $\alpha_0=0$ la m\'ethode de Jacobi. }
		
		
	\end{enumerate}
	
\end{enumerate}
}

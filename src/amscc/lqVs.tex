\uuid{lqVs}
\chapitre{Résolution de systèmes linéaires : méthode directe}
\niveau{L3}
\module{Analyse numérique}
\sousChapitre{Résolution de systèmes linéaires : méthode directe}
\titre{Résolution de système linéaire}
\theme{analyse numérique}
\auteur{}
\datecreate{2023-03-02}
\organisation{AMSCC}
\difficulte{}
\contenu{


\texte{ On consid\`ere le syst\`eme lin\'eaire :
\begin{equation}
	\left\{\begin{array}{rcl}
		5 x +  y + z & = & 7\\
		x + 5 y -z & = & 5 \\
		x -y + 4z & = & 4
	\end{array}\right.
	\tag{$S$}
	\label{eq:syslin1}
\end{equation}

dont la solution est $(1,1,1)$. }

\begin{enumerate}
	\item \question{ Montrer que l'on peut utiliser la m\'ethode de Jacobi pour r\'esoudre ce syst\`eme et justifier que dans ce cas, la m\'ethode converge. }
	\reponse{Résoudre ce système revient à résoudre l'équation $Ax=b$ où $A=\left(\begin{matrix} 5 & 1 & 1\\1 & 5 & -1\\1 & -1 & 4 \end{matrix}\right)$ et $b=\left(\begin{matrix} 7\\5\\4 \end{matrix}\right)$. Cette matrice $A$ est à diagonale strictement dominante car $5>1+1$ et $4>1+1$. Par théorème, on en déduit que la méthode de Jacobi converge vers la solution. }
	
	\item \question{ Calculer la premi\`ere it\'eration de la m\'ethode de Jacobi en partant de $X_0 = (0,0,0)$ puis la 50ème itération à l'aide d'un programme Python. }
	\reponse{Pour appliquer la méthode de Jacobi, on décompose $A$ sous la forme $A=M-N$ où 
		$M=\left(\begin{matrix} 5 & 0 & 0\\0 & 5 & 0\\0 & 0 & 4 \end{matrix}\right)$ et $N=\left(\begin{matrix} 0 & -1 & -1\\-1 & 0 & 1\\-1 & 1 & 0 \end{matrix}\right)$. Pour tout entier $n$, on définit $X_{n+1}=F(X_n)$ où $F(X)=M^{-1}NX+M^{-1}b$ et $M^{-1}=\left(\begin{matrix} \frac15 & 0 & 0\\0 & \frac15 & 0\\0 & 0 & \frac14 \end{matrix}\right)$. On décide d'initialiser l'itération avec $X_0=\left(\begin{matrix} 0\\0\\0 \end{matrix}\right)$.
		
		On calcule :
		$$X_1 = M^{-1}NX_0+M^{-1}b= M^{-1}b = \left(\begin{matrix} 1.4\\1\\1 \end{matrix}\right) $$
		$$X_2 = M^{-1}NX_1+M^{-1}b = \left(\begin{matrix} 1\\0.92\\0.9 \end{matrix}\right) $$
		$$X_3 = M^{-1}NX_2+M^{-1}b = \left(\begin{matrix} 1.036\\0.98\\0.98 \end{matrix}\right) $$}
	
	\item \question{ Montrer que la matrice $A$ est symétrique définie positive et en déduire la convergence de la m\'ethode de Gauss-Seidel pour ce probl\`eme. }
	\reponse{Pour utiliser la méthode de Gauss-Seidel, on peut s'assurer que la matrice $A$ est symétrique définie positive. Elle est visiblement symétrique réelle donc diagonalisable. Il reste donc à vérifier que ses valeurs propres sont toutes strictement positives. 
		
		On se lance dans le calcul du polynôme caractéristique : 
		$$P_A(X) = \begin{vmatrix}
			5-X & 1 & 1 \\
			1 & 5-X & -1 \\
			1 & -1 & 4-X 
		\end{vmatrix} = \begin{vmatrix}
			6-X & 1 & 1 \\
			6-X & 5-X & -1 \\
			0 & -1 & 4-X 
		\end{vmatrix} 		=(6-X)(X^2-8X+14)$$
		Une valeur propre évidente est donc $\lambda_1=6$. En analysant le polynôme du second degré $(X^2-8X+14)$, on déduit que $\lambda_2\lambda_3=14$ donc $\lambda_2$ et $\lambda_3$ sont de même signe. De plus, $\lambda_2+\lambda_3=8$ donc on est assuré que $\lambda_2>0$ et $\lambda_3>0$. }
	
	\item \question{ Calculer les cinquante premi\`eres it\'erations de la m\'ethode de Gauss-Seidel en partant de $X_0 = (0,0,0)$. }
	\reponse{On calcule :
		$$X_1 = M^{-1}NX_0+M^{-1}b= M^{-1}b = \left(\begin{matrix} 1.4\\0.72\\0.83 \end{matrix}\right) $$
		$$X_2 = M^{-1}NX_1+M^{-1}b = \left(\begin{matrix} 1.09\\0.948\\0.9645 \end{matrix}\right) $$
		$$X_3 = M^{-1}NX_2+M^{-1}b = \left(\begin{matrix} 1.0175\\0.9894\\0.992975 \end{matrix}\right) $$
	}
	
\end{enumerate}}

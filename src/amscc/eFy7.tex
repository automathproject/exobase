\uuid{eFy7}
\titre{ Analyse de covariance et moments d'une loi normale }
\theme{loi normale}
\auteur{}
\datecreate{2023-11-15}
\organisation{AMSCC}
\contenu{


\texte{ On considère deux variables aléatoires réelles \( X \) et \( U \)  indépendantes, \( X \) suivant la loi normale \( \mathcal{N}(0, 1) \) et \( U \) suivant la loi discrète uniforme sur \( \{-1,1\} \).

On pose \( Y = UX \) et on admet que \( Y \) est une variable aléatoire absolument continue.  }

\begin{enumerate}
	\item  \question{ En utilisant la formule des probabilités totales, montrer que :
$$ \forall x \in \mathbb{R}, \quad \mathbb{P}(Y \leq x) = \mathbb{P}(U = 1) \mathbb{P}(X \leq x) + \mathbb{P}(U = -1) \mathbb{P}(X > -x) $$

et en déduire que \( Y \) suit la même loi que \( X \). }
\reponse{
Soient \( X \sim \mathcal{N}(0,1) \) et \( U \) une variable discrète uniforme sur \(\{-1,1\}\), indépendantes. Pour tout \( x \in \mathbb{R} \), on a
\[
\mathbb{P}(Y\le x) = \mathbb{P}(UX\le x)
= \mathbb{P}(\{U=1\}\cap\{X\le x\}) + \mathbb{P}(\{U=-1\}\cap\{-X\le x\}).
\]
Lorsque \( U = -1 \), l'inégalité \(-X\le x\) s'écrit \( X\ge -x \). Par indépendance de \( U \) et \( X \) et en utilisant la formule des probabilités totales, on obtient
\[
\mathbb{P}(Y\le x)
=\mathbb{P}(U=1)\,\mathbb{P}(X\le x) + \mathbb{P}(U=-1)\,\mathbb{P}(X\ge -x).
\]
Étant donné que \( \mathbb{P}(U=1)=\mathbb{P}(U=-1)=\frac{1}{2} \) et que la symétrie de la loi normale implique que
\[
\mathbb{P}(X\ge -x)=\mathbb{P}(X\le x),
\]
il vient
\[
\mathbb{P}(Y\le x)
=\frac{1}{2}\,\mathbb{P}(X\le x) + \frac{1}{2}\,\mathbb{P}(X\le x)
=\mathbb{P}(X\le x).
\]
Ainsi, la fonction de répartition de \( Y \) est identique à celle de \( X \) et donc \( Y \sim \mathcal{N}(0,1) \).
}


\item \question{  Calculer l'espérance de \( U \), puis montrer que \( \mathbb{E}(XY) = 0 \). En déduire que \( \mathrm{Cov}(X, Y) = 0 \). }
\reponse{
Tout d'abord, comme \( U \) prend les valeurs \( -1 \) et \( 1 \) avec même probabilité,
\[
\mathbb{E}(U)=\frac{1}{2}(-1)+\frac{1}{2}(1)=0.
\]
En écrivant \( Y = UX \), on a
\[
\mathbb{E}(XY)=\mathbb{E}(X\,(UX))=\mathbb{E}(U\,X^2).
\]
Comme \( U \) et \( X \) sont indépendantes, on peut séparer l'espérance :
\[
\mathbb{E}(U\,X^2)=\mathbb{E}(U)\,\mathbb{E}(X^2)=0\times\mathbb{E}(X^2)=0.
\]
La covariance est donnée par
\[
\mathrm{Cov}(X,Y)=\mathbb{E}(XY)-\mathbb{E}(X)\,\mathbb{E}(Y).
\]
Or, \( \mathbb{E}(X)=0 \) (pour une loi normale centrée) et, comme \( Y \sim \mathcal{N}(0,1) \), \( \mathbb{E}(Y)=0 \). Ainsi,
\[
\mathrm{Cov}(X,Y)=0-0=0.
\]
}

\item \question{ Rappeler la valeur de \( \mathbb{E}(X^2) \) et en déduire que :
$$ \int_{0}^{+\infty} x^2 e^{-\frac{x^2}{2}} \, dx = \frac{\sqrt{\pi}}{2} $$ }
\reponse{
Pour \( X \sim \mathcal{N}(0,1) \), on connaît que
\[
\mathbb{E}(X^2)=1.
\]
D'autre part, en utilisant la densité de \( X \),
\[
f_X(x)=\frac{1}{\sqrt{2\pi}}\,e^{-x^2/2},
\]
on a
\[
\mathbb{E}(X^2)=\int_{-\infty}^{+\infty} x^2\,f_X(x)\,dx
=\frac{1}{\sqrt{2\pi}} \int_{-\infty}^{+\infty} x^2 e^{-x^2/2}\,dx.
\]
La fonction \( x\mapsto x^2e^{-x^2/2} \) est paire, donc
\[
\int_{-\infty}^{+\infty} x^2 e^{-x^2/2}\,dx = 2\int_{0}^{+\infty} x^2 e^{-x^2/2}\,dx.
\]
Ainsi,
\[
1=\frac{1}{\sqrt{2\pi}} \cdot 2\int_{0}^{+\infty} x^2 e^{-x^2/2}\,dx,
\]
ce qui donne
\[
\int_{0}^{+\infty} x^2 e^{-x^2/2}\,dx
=\frac{\sqrt{2\pi}}{2}.
\]
}

\item \question{ En déduire, s'il existe, le moment d'ordre $4$ de $X$. }

\reponse{
Nous souhaitons calculer
\[
\mathbb{E}(X^4)=\frac{1}{\sqrt{2\pi}}\int_{-\infty}^{+\infty} x^4 e^{-x^2/2}\,dx.
\]
Comme la fonction \( x\mapsto x^4e^{-x^2/2} \) est paire, on peut écrire
\[
\mathbb{E}(X^4)=\frac{2}{\sqrt{2\pi}}\int_{0}^{+\infty} x^4 e^{-x^2/2}\,dx.
\]
Pour évaluer l'intégrale
\[
I = \int_{0}^{+\infty} x^4 e^{-x^2/2}\,dx,
\]
nous effectuons une intégration par parties en posant
\[
\begin{cases}
u = x^3, \quad & du = 3x^2\,dx,\\[1mm]
dv = x\,e^{-x^2/2}\,dx, \quad & v = -e^{-x^2/2}.
\end{cases}
\]
Ainsi, par intégration par parties,
\[
I = \Bigl[-x^3e^{-x^2/2}\Bigr]_0^{+\infty} + \int_0^{+\infty} 3x^2e^{-x^2/2}\,dx.
\]
Le terme d'évaluation aux bornes s'annule (en effet, \( x^3 e^{-x^2/2}\to 0 \) quand \( x\to+\infty \) et vaut 0 en 0), ce qui donne
\[
I = 3\int_{0}^{+\infty} x^2 e^{-x^2/2}\,dx.
\]
D'après la question précédente, on a établi que
\[
\int_{0}^{+\infty} x^2 e^{-x^2/2}\,dx = \frac{\sqrt{2\pi}}{2}.
\]
Il s'ensuit que
\[
I = 3\,\frac{\sqrt{2\pi}}{2}.
\]
Finalement,
\[
\mathbb{E}(X^4)=\frac{2}{\sqrt{2\pi}}\;I=\frac{2}{\sqrt{2\pi}}\cdot\frac{3\sqrt{2\pi}}{2}=3.
\]
Ainsi, le moment d'ordre 4 de \( X \) existe et vaut \( 3 \).
}
\end{enumerate}
}
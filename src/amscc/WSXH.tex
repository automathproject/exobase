\uuid{WSXH}
\chapitre{Statistique}
\niveau{L2}
\module{Probabilité et statistique}
\sousChapitre{Tests d'hypothèses, intervalle de confiance}
\titre{ Comparaison de moyennes }
\theme{tests d'hypothèses}
\auteur{}
\datecreate{2023-11-27}
\organisation{AMSCC}
\difficulte{}
\contenu{

%labrousse 27 p 92
\texte{ Une usine fabrique des rouleaux de pellicules photographiques. On mesure l'épaisseur des couches sensibles de ces pellicules. pour cela, on utilise un échantillon de $N=616$ pellicules . On répartit les épaisseurs par tranche de un micron et on désigne par $x_i$ l'épaisseur moyenne de la classe $i$. On regroupe les épaisseurs par classe dans le fichier \texttt{TP\_photos.xls}.  }

\question{ Peut-on admettre que l'épaisseur des couches sensibles suit une loi normale ? }

\reponse{ \href{https://stcyrterrenetdefensegouvf-my.sharepoint.com/:x:/g/personal/maxime_nguyen_st-cyr_terre-net_defense_gouv_fr/EYlCPJY62BdMpTYscg1T6zABc9_WCIR_pzt6XSOajALYhw?e=MnGFyL&nav=MTVfezAwMDAwMDAwLTAwMDEtMDAwMC0wNTAwLTAwMDAwMDAwMDAwMH0}{lien vers tableur} }
}
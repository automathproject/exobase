\uuid{CuTJ}
\titre{Calcul de reste de division euclidienne}
\theme{polynômes}
\auteur{}
\datecreate{2023-01-23}
\organisation{AMSCC}
\contenu{


\texte{ Soit $P \in \mathbb{R}[X]$ un polynôme dont le reste de la division euclidienne par $\left(X^2-1\right)$ est $(X+1)$. } 

\question{ Quels sont les restes de la division de $P$ par $(X+1)$ ? par $(X-1)$ ? }

\indication{Revenir à la définition en écrivant qu'il existe un polynôme $Q$ tel que $P(X)=Q(X) \left(X^2-1\right)+(X+1)$. }

\reponse{ 
On a :
$$
P(X)=Q(X) \left(X^2-1\right)+(X+1)
$$
Que l'on peut encore écrire :
$$
P(X)=Q(X) \cdot(X-1)(X+1)+(X+1)=(X+1)[Q(X) \cdot(X-1)+1]+0
$$
Le reste de la division par $(X+1)$ est 0.

On peut aussi écrire :
$$
P(X)=Q(X) \cdot(X-1)(X+1)+\underbrace{(X-1)+2}_{X+1}=(X-1)[Q(X) \cdot(X+1)+1]+2
$$
Le reste de la division par $(X-1)$ est 2. }}

\uuid{uR6d}
\chapitre{Série numérique}
\niveau{L1}
\module{Analyse}
\sousChapitre{Série à termes positifs}
\titre{\'Etude de la nature d'une série}
\theme{séries}
\auteur{ }
\datecreate{2023-05-30}
\organisation{AMSCC}
\difficulte{}
\contenu{

\begin{enumerate}

\item \question{ Prouver qu'un équivalent de $e^{x^2}-\cos(x)$ au voisinage de $0$ est $\frac{3x^2}{2}$. On pourra utiliser un développement limité.  }
\reponse{ Pour démontrer ceci, nous devons d'abord obtenir les développements limités de $e^{x^2}$ et $\cos(x)$ autour de $0$ jusqu'à l'ordre $2$ :
	
	\begin{align*}
		e^{x^2} & = 1 + x^2 + o(x^2)\\
		\cos(x) & = 1 - \frac{x^2}{2} + o(x^2)
	\end{align*}

	
	Ensuite, nous soustrayons ces deux séries terme à terme pour obtenir le développement limité de $e^{x^2}-\cos(x)$ :
	
	\begin{align*}
		e^{x^2} - \cos(x) & = (1 + x^2 + o(x^2)) - (1 - \frac{x^2}{2}  + o(x^2))\\
		& =  \frac{3x^2}{2}  + o(x^2)
	\end{align*}
	
	On en déduit que l'équivalent de $e^{x^2}-\cos(x)$ au voisinage de $0$ est bien $\frac{3x^2}{2}$.
 }
\item \question{ Étudier la convergence de la série $\displaystyle   \sum_{n \geq 1} u_n$ où pour tout $n \geq 1$, $$u_n = e^{\frac{1}{n^2}}-\cos\left(\frac1n\right).$$ }
\reponse{ On déduit de la question précédente que $$u_n \underset{+\infty}\sim \frac{3}{2n^2}$$
Cela prouve que $u_n \geq 0$ à partir d'un certain rang et que la série $\displaystyle   \sum_{n \geq 1} u_n$ converge par comparaison à une série de Riemann convergente.
 }
\end{enumerate}}

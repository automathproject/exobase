\uuid{tPpR}
\chapitre{Série numérique}
\niveau{L1}
\module{Analyse}
\sousChapitre{Séries divergentes}
\titre{Séries divergentes}
\theme{séries}
\auteur{}
\datecreate{2023-05-17}
\organisation{AMSCC}
\difficulte{}
\contenu{



\texte{ 	En examinant la limite du terme général, montrer que les séries suivantes divergent : }
\indication{Si le terme général $u_n$ ne converge pas vers $0$ alors par théorème du cours, $\sum u_n$ est une série divergente. }
	\begin{enumerate}
		\item \question{ $\displaystyle \sum_{n\geq 1} \left(1+(-1)^n\cos\left(\frac{1}{n}\right)\right)$ ; }
		\reponse{On pose $u_n = \left(1+(-1)^n\cos\left(\frac{1}{n}\right)\right)$ : alors $u_{2n} = 1+\cos\left(\frac{1}{n^2}\right) \xrightarrow[n \to +\infty]{} 1$ donc $\lim\limits u_n \neq 0$ donc la série est divergente.}
		\item \question{ $\displaystyle \sum_{n\geq 1} \frac{(-1)^n}{1+n^{-1}}$ ; }
		\reponse{On pose $u_n = \frac{(-1)^n}{1+n^{-1}}$ : alors $u_{2n} = \frac{1}{1+\frac1n} \xrightarrow[n \to +\infty]{} 1$ donc $\lim\limits u_n \neq 0$ donc la série est divergente.}
		\item \question{ $\displaystyle \sum_{n\geq 1} \left(\left(\frac{n}{n+1}\right)^n-1\right)$. }
		\reponse{Contrairement aux apparences, le terme général ne tend pas vers vers $0$. En effet : \\ $\frac{n}{n+1} = \frac{n+1-1}{n+1} = 1-\frac{1}{n+1}$ donc $\left(\frac{n}{n+1}\right)^n = e^{n\ln\left(1-\frac{1}{n+1}\right)}$. \\
			Or $\ln\left(1-\frac{1}{n+1}\right) \underset{n\to +\infty}{\sim} -\frac{1}{n+1}$ donc $n\ln\left(1-\frac{1}{n+1}\right) \underset{n\to +\infty}{\sim} -1$. \\
			Par composition de limites, on en déduit que $\left(\frac{n}{n+1}\right)^n \xrightarrow[n \to +\infty]{} e^{-1} = \frac{1}{e} \neq 0$. Donc la série est divergente. 
		}
	\end{enumerate}
}

\uuid{M7Z5}
\chapitre{Statistique}
\niveau{L2}
\module{Probabilité et statistique}
\sousChapitre{Tests d'hypothèses, intervalle de confiance}
\titre{ Comparaison de moyennes }
\theme{tests d'hypothèses}
\auteur{Maxime Nguyen}
\datecreate{2023-11-27}
\organisation{AMSCC}
\difficulte{}
\contenu{
\texte{ 		Un professeur affirme qu'il existe une différence entre les
hommes et les femmes sur un test d'habileté spatio-cognitivo-émotive.
Voici les données sur lesquelles repose son
affirmation: \\ Hommes: 82, 80, 81, 84, 75 \\ Femmes: 74, 79, 78, 71 }

\question{ La différence observée est-elle significative au seuil de 5\% (on suppose que $\sigma_H = \sigma_F$)?  }

\reponse{ \href{https://stcyrterrenetdefensegouvf-my.sharepoint.com/:x:/g/personal/maxime_nguyen_st-cyr_terre-net_defense_gouv_fr/EYlCPJY62BdMpTYscg1T6zABc9_WCIR_pzt6XSOajALYhw?e=WsbIMH&nav=MTVfezAwMDAwMDAwLTAwMDEtMDAwMC0wNDAwLTAwMDAwMDAwMDAwMH0}{lien vers tableur} }
}

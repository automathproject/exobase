\uuid{fU9E}
\chapitre{Probabilité continue}
\niveau{L2}
\module{Probabilité et statistique}
\sousChapitre{Densité de probabilité}
\titre{Couple de variables aléatoires}
\theme{variables aléatoires à densité}
\auteur{}
\datecreate{2022-11-15}
\organisation{AMSCC}
\difficulte{}
\contenu{



\texte{ Soit $f(x,y)=2e^{-x}e^{-2y}\textbf{1}_{(\R^+)^2}(x,y)$.  }
\begin{enumerate}
	\item \question{  Vérifier que $f$ définit une densité de probabilité. }
	\reponse{ On vérifie que $f$ est positive sur $\R^2$ et par application du théorème de Fubini que $\int_{-\infty}^{+\infty}\int_{-\infty}^{+\infty} f(x,y) dxdy=1$ }
	\item \question{ Calculer $\PP(X>1,Y<1)$, $\PP(X<Y)$ et $\PP(X<a)$. }
	\reponse{ On applique la définition et on trouve $\PP(X>1,Y<1) = e^{-1}(1-e^{-2})$, $\PP(X<Y)=\frac{1}{3}$ et $\PP(X<a)=1-e^{-a}$. }
\end{enumerate}}

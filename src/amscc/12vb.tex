\uuid{12vb}
\chapitre{Statistique}
\niveau{L2}
\module{Probabilité et statistique}
\sousChapitre{Probabilité et Statistique}
\titre{Vraisemblance et Méthode des moments}
\theme{Statistique}
\auteur{}
\organisation{AMSCC}
\difficulte{}
\contenu{






\texte{
Soit \( \theta \in ]0, 1[ \) un paramètre inconnu, on note \( X \) une variable aléatoire de loi définie par
$$
P_\theta(X = k) = (k + 1)(1 - \theta)^2 \theta^k, \quad \text{pour tout } k \in \mathbb{N}.
$$
On donne
$$
E_\theta[X] = \frac{2\theta}{1 - \theta}
\quad \text{et} \quad
\text{Var}_\theta[X] = \frac{2\theta}{(1 - \theta)^2}.
$$
On souhaite estimer \( \theta \) à partir d’un échantillon \( (X_1, \dots, X_n) \) de même loi que \( X \).
\begin{enumerate}
  \item \question{Donner un estimateur \( \hat{\theta}_n \) de \( \theta \) par la méthode des moments.}
  \item \question{L’estimateur du maximum de vraisemblance \( \hat{\theta}_n \) de \( \theta \) est-il bien défini ?}
  \item \question{Étudier la consistance de \( \hat{\theta}_n \) et déterminer sa loi limite.}
\end{enumerate}


}
}
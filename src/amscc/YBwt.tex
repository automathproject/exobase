\uuid{YBwt}
\chapitre{Probabilité discrète}
\niveau{L2}
\module{Probabilité et statistique}
\sousChapitre{Lois de distributions}
\titre{Loi d'un couple}
\theme{variables aléatoires discrètes, loi conjointe}
\auteur{}
\datecreate{2023-02-07}
\organisation{AMSCC}
\difficulte{}
\contenu{

\texte{ 	Soient $X$ et $Y$ deux variables aléatoires réelles discrètes dont la loi de probabilité conjointe est
	donnée par le tableau ci-contre:
	
	\begin{center}
		\begin{tabular}{|c|c|c|c|}
			\hline
			$X \backslash  Y$ & 0 & 1 & 2 \\ 
			\hline
			0 & 1/12 &  & 5/24 \\
			\hline 
			1 & 1/24 & 1/48 & 5/48 \\ 
			\hline
			2 & 1/8 & 1/16 & 5/16\\
			\hline
		\end{tabular} 
	\end{center} }
	\begin{enumerate}
		\item \question{ Calculer la probabilité de l'événement $(X, Y) = (0, 1)$. }
		\item \question{ Calculer les lois marginales de $X$ et de $Y$. }
		\item \question{ Les variables aléatoires $X$ et $Y$ sont-elles indépendantes ? }
		\item \question{ Calculer la loi de probabilité, l'espérance et la variance de $X + Y$. }
	\end{enumerate}}

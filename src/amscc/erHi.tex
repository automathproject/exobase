\uuid{erHi}
\titre{Racine multiple}
\theme{polynômes}
\auteur{}
\datecreate{2023-01-23}
\organisation{AMSCC}
\contenu{

\question{ Montrer que le polynôme $P(X)=1+X+\frac{X^2}{2 !}+\frac{X^3}{3 !}+\ldots+\frac{X^n}{n !}$ n'a pas de racine multiple. }

\indication{Supposer qu'il existe une racine multiple, c'est-à-dire qu'il existe $\alpha$ tel que $P(\alpha) = P'(\alpha) = 0$ et raisonner par l'absurde. }

\reponse{ Raisonnement par l'absurde :
$a$ est racine multiple ssi $P(a)=P^{\prime}(a)=0$.
On a :
$$
\begin{aligned}
& P(X)=1+X+\frac{X^2}{2 !}+\ldots+\frac{X^{n-1}}{(n-1) !}+\frac{X^n}{n !} \\
& P^{\prime}(X)=1+X+\frac{X^2}{2 !}+\ldots+\frac{X^{n-1}}{(n-1) !} \\
& \left.\begin{array}{l}
P(a)=0 \quad \Leftrightarrow 1+a+\frac{a^2}{2 !}+\ldots+\frac{a^{n-1}}{(n-1) !}+\frac{a^n}{n !}=0 \\
P^{\prime}(a)=0 \Leftrightarrow 1+a+\frac{a^2}{2 !}+\ldots+\frac{a^{n-1}}{(n-1) !}=0
\end{array}\right\} \Rightarrow \frac{a^n}{n !}=0 \Leftrightarrow a=0 \\
&
\end{aligned}
$$
or 0 n'est pas racine car $P(0)=1$, donc $a$ ne peut pas être racine simultanément de $P$ et $P^{\prime}$. $a$ ne peut pas être racine multiple. }}

\uuid{3zDK}
\chapitre{Statistique}
\niveau{L2}
\module{Probabilité et statistique}
\sousChapitre{Tests d'hypothèses, intervalle de confiance}
\titre{Application de la loi de Benford}
\theme{statistiques, tests d'hypothèses}
\auteur{Maxime NGUYEN}
\datecreate{2024-12-10}
\organisation{AMSCC}

\difficulte{}
\contenu{

\texte{
La loi de Benford, initialement appelée loi des nombres anormaux par Benford, fait référence à une fréquence de distribution statistique observée empiriquement sur de nombreuses sources de données dans la vraie vie, ainsi qu’en mathématiques. Dans une série de données numériques, on pourrait s’attendre à voir les chiffres de 1 à 9 apparaître à peu près aussi fréquemment comme premier chiffre significatif, soit avec une fréquence de $1/9 = 11,1\%$ pour chacun. Or, contrairement à cette intuition (biais d’équiprobabilité), la série suit très souvent approximativement la loi de Benford :

% Échelle verticale : 1% = 0.2 (ensuite on convertit en cm via [y=1cm])
% Donc 35% = 7 (c'est juste un facteur numérique, on multiplie par 1cm grâce au y=1cm)
\def\yscale{0.2}

% Largeur des barres en cm (1 unité = 1cm)
\def\barwidth{0.5}

% Données (X,Y) sans unités, Y étant un nombre
\def\data{{1/30.10},{2/17.61},{3/12.49},{4/9.69},{5/7.92},{6/6.69},{7/5.80},{8/5.12},{9/4.58}}

\begin{center}
\begin{tikzpicture}[x=1cm,y=1cm,font=\small]

% Axe Y de 0 à 35% -> 35 * 0.2 = 7 cm
\draw[->,thick] (0.5,0) -- (0.5,7.5) node[above]{Fréquence (\%)};

% Graduations Y : 0 à 35% par pas de 5%
\foreach \v in {0,5,10,15,20,25,30,35} {
   \draw (0.5,\v*\yscale) -- (0.45,\v*\yscale) node[left]{\v};
}

% Axe X de 0.5 à 9.5
\draw[->,thick] (0.5,0) -- (9.5,0) node[right]{1ère décimale};

% Titre et sous-titre
\node[font=\bfseries,anchor=south] at (5,7.7) {Loi de BENFORD};
\node[font=\small,anchor=south] at (5,6.4) {Fréquences relatives d'apparition de la 1ère décimale};

% Graduation X (1 à 9)
\foreach \x in {1,2,3,4,5,6,7,8,9} {
  \draw (\x,0) -- (\x,-0.1) node[below]{\x};
}

% Dessin des barres
\foreach \X/\Y in \data {
  % Calcul de la hauteur : h = Y * yscale
  \pgfmathsetmacro{\h}{\Y*\yscale}
  % Dessin de la barre :
  % La barre est centrée sur X, avec une largeur de barwidth cm 
  % et une hauteur de h (qu'on interprète en cm grâce au [y=1cm])
  \filldraw[fill=cyan!70!black,draw=black] (\X-\barwidth/2,0) rectangle (\X+\barwidth/2,\h);
  % Étiquette du pourcentage au-dessus de la barre
  \node[font=\footnotesize,anchor=south] at (\X,\h) {\Y\%};
}

\end{tikzpicture}
\end{center}

Cette loi est utilisée notamment dans la détection des fraudes. 

En 2019, le mathématicien Mickäel Launay a relevé 1226 prix dans un supermarché, et a obtenu comme fréquences successives pour les premiers chiffres de 1 à 9 : 32\%, 26\%, 15\%, 9\%, 5\%, 4\%, 3\%, 2\%, 4\%.
}


\begin{enumerate}
    \item \question{Avec un risque de première espèce de 5\%, peut-on affirmer que l’observation du mathématicien est incompatible avec la loi de Benford ?}
    \indication{Effectuer un test du $\chi^2$ pour comparer les fréquences observées avec celles attendues selon la loi de Benford.}
    \reponse{À compléter.}
\end{enumerate}

}

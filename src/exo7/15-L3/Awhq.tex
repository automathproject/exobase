\uuid{Awhq}
\exo7id{2351}
\auteur{queffelec}
\organisation{exo7}
\datecreate{2003-10-01}
\isIndication{false}
\isCorrection{true}
\chapitre{Espace topologique, espace métrique}
\sousChapitre{Espace topologique, espace métrique}

\contenu{
\texte{
Soit $(x_n)$ une suite d'un espace topologique $X$ séparé; on note $A$
l'ensemble $\{x_1,x_2,\ldots\}$.
}
\begin{enumerate}
    \item \question{Toute valeur d'adhérence $a$ de la suite est un point de $\overline
A$ : donner un exemple où $a$ est un point isolé de $A$; un exemple où $a$ est
un point d'accumulation dans $A$; un exemple où $a$ est
un point d'accumulation dans $\overline A\backslash A$.}
    \item \question{Montrer que tout point d'accumulation de $A$ est valeur d'adhérence de la
suite.}
\reponse{
\begin{enumerate}
Par exemple une suite constante $x_n=a$ pour tout $n$.
Par exemple $x_n = \frac1n$ et $a=0$.
Comme $\Qq$ est dénombrable on peut trouver une suite $x_n$ telle
que $A = \{x_1,x_2,\ldots\}= \Qq$. On prend $a = \sqrt2$ alors $a\in \bar A \setminus A = \Rr \setminus \Qq$.
}
\end{enumerate}
}

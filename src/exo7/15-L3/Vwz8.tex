\uuid{Vwz8}
\exo7id{6101}
\auteur{queffelec}
\organisation{exo7}
\datecreate{2011-10-16}
\isIndication{false}
\isCorrection{false}
\chapitre{Espace topologique, espace métrique}
\sousChapitre{Espace topologique, espace métrique}

\contenu{
\texte{
Soit $E$ un ensemble non vide, et $X=E^{\Nn}$ l'ensemble des suites $x=(x_n)$
d'éléments de $E$. Pour $x,y\in X$, on pose $p(x,y)=\min\{n/x_n\neq y_n\}$ si
$x\neq y$, et $\infty$ si $x=y$.

Montrer que $d(x,y)={1\over p(x,y)}$ (avec ${1\over\infty}=0$) est une distance
sur $X$ qui vérifie l'inégalité ultramétrique
$$d(x,z)\leq\max(d(x,y),d(y,z)).$$
}
}

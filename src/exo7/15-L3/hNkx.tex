\uuid{hNkx}
\exo7id{6138}
\auteur{queffelec}
\organisation{exo7}
\datecreate{2011-10-16}
\isIndication{false}
\isCorrection{false}
\chapitre{Espace topologique, espace métrique}
\sousChapitre{Espace topologique, espace métrique}

\contenu{
\texte{
On considère la suite de polyn\^omes sur $[-1,1]$ 
$$f_n(x)={{\int_0^x(1-t^2)^n\ dt}\over {\int_0^1(1-t^2)^n\ dt}}.$$
}
\begin{enumerate}
    \item \question{Montrer que
pour tout $\varepsilon$, cette suite converge uniformément vers
$1$ sur l'intervalle $[\varepsilon,1]$, et  vers
$-1$ sur l'intervalle $[-1,-\varepsilon]$.

\emph{Indication :} Comparer $\int_0^1(1-t^2)^n\ dt$ à $\int_0^1(1-t)^n\ dt$.}
    \item \question{En déduire que la suite $g_n(x)=\int_0^x f_n(t)\ dt$ converge uniformément
vers $\vert x\vert$ sur $[-1,1]$.}
    \item \question{Montrer que dans l'exercice \ref{gijsexopol} la convergence est aussi
uniforme sur $[-1,1]$, en établissant une relation de récurrence satisfaite par
l'erreur  $\epsilon_n(t)=\vert t\vert-p_n(t).$}
\end{enumerate}
}

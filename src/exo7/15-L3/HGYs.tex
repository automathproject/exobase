\uuid{HGYs}
\exo7id{2406}
\titre{exo7 2406}
\auteur{mayer}
\organisation{exo7}
\datecreate{2003-10-01}
\isIndication{true}
\isCorrection{true}
\chapitre{Théorème du point fixe}
\sousChapitre{Théorème du point fixe}
\module{Topologie}
\niveau{L3}
\difficulte{}

\contenu{
\texte{
Soit $X = ({\cal C}^1([0,1]),N)$ avec $N(f) = \|f\|_\infty+\|f'\|_\infty$.
Montrer qu'il existe une fonction $f\in X$ qui est
point fixe de l'op\'erateur $T$ donn\'e par
$$ Tf(x)= 1 + \int _0^x f(t-t^2)\, dt \; .$$
On pourra commencer par \'etablir que $T\circ T$ est une
contraction. Utiliser ceci pour \'etablir l'existence d'une
fonction unique $f\in X$ qui v\'erifie $f(0)=1 $
et $f'(x) = f(x-x^2) $.
}
\indication{Faire soigneusement le calcul : $(T\circ T f)(x)=1+x+\int_0^x\int_0^{t-t^2}f(u-u^2)dudt$.
Se souvenir que $X$ est complet et utiliser l'exercice \ref{exoiter}.}
\reponse{
$(T\circ T f)(x)= 1+\int _0^x Tf(t-t^2) dt = 1+\int _0^x(1+\int_0^{t-t^2}f(u-u^2)du)dt
=1+x+\int_0^x\int_0^{t-t^2}f(u-u^2)dudt$.
De plus $(T\circ T f)'(x) = 1+\int_0^{x-x^2}f(u-u^2)du$.
En remarquant que pour $t\in[0,1]$, $t-t^2\le \frac 14$, on montre que
$|T\circ Tf(x)-T\circ Tg(x)| \le \frac 14 \|f-g\|_\infty$
et que $|(T\circ Tf)'(x)-(T\circ Tg)'(x)| \le \frac 14 \|f-g\|_\infty$
Donc $N(T\circ Tf-T\circ T g)\le \frac 12 \|f-g\|_\infty \le \frac 12 N(f-g)$.
Donc $T\circ T$ est une contraction et $X$ est complet donc $T\circ T$ admet 
un unique point fixe, par l'exercice \ref{exoiter}, $T$ admet un unique point fixe.
Remarquons que $Tf=f$ est équivalent à $f(0)=1$ et $f'(x)=f(x-x^2)$.
Donc l'existence et l'unicité du point fixe pour $T$ donne l'existence et l'unicité de la solution
au problème posé.
}
}

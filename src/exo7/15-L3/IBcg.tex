\uuid{IBcg}
\exo7id{6168}
\titre{exo7 6168}
\auteur{queffelec}
\organisation{exo7}
\datecreate{2011-10-16}
\isIndication{false}
\isCorrection{false}
\chapitre{Compacité}
\sousChapitre{Compacité}
\module{Topologie}
\niveau{L3}
\difficulte{}

\contenu{
\texte{
Soit $(f_n)$ une suite croissante de fonctions réelles définies sur un espace
topolo\-gique compact $X$, convergeant simplement vers une fonction $f$; on
suppose que les fonctions $f_n$ et $f$ sont continues.
Montrer que la convergence est uniforme sur $X$.

Application : montrer que la suite de fonctions $f_n$ définies sur $[0,1]$
par $f_n(x)=\sum_1^{n-1} x^k(1-x)^{n-k}$ converge vers $0$ uniformément sur
$[0,1]$.
}
}

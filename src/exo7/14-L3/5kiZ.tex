\uuid{5kiZ}
\exo7id{5945}
\titre{exo7 5945}
\auteur{tumpach}
\organisation{exo7}
\datecreate{2010-11-11}
\isIndication{false}
\isCorrection{true}
\chapitre{Théorème de convergence dominée}
\sousChapitre{Théorème de convergence dominée}
\module{Théorie de la mesure, intégrale de Lebesgue}
\niveau{L3}
\difficulte{}

\contenu{
\texte{
Soit $(\Omega, \Sigma, \mu)$ un espace mesur\'e avec
$\mu(\Omega)~<~+\infty$. Soit $\{ f_{n} \}_{n\in\mathbb{N}}$ une
suite de fonctions mesurables convergeant presque partout vers une
fonction mesurable $f$. On suppose qu'il existe une constante
$C>0$ telle que $|f_{n}| \leq C$ pour tout $n\geq 1$. Montrer que
$$
\lim_{n\rightarrow+\infty}\int_{\Omega} f_{n}\,d\mu =
\int_{\Omega} f\,d\mu.
$$
}
\reponse{
Puisque $\mu(\Omega) < +\infty$, la fonction constante \'egale \`a
$C$ est int\'egrable, d'int\'egrale $C \, \mu(\Omega)$. Une
application directe du th\'eor\`eme de convergence domin\'ee donne
$$
\lim_{n\rightarrow+\infty}\int_{\Omega} f_{n}\,d\mu =
\int_{\Omega} f\,d\mu.
$$
}
}

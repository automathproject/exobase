\uuid{xNxI}
\exo7id{5957}
\auteur{tumpach}
\organisation{exo7}
\datecreate{2010-11-11}
\isIndication{false}
\isCorrection{true}
\chapitre{Intégrales multiples, théorème de Fubini}
\sousChapitre{Intégrales multiples, théorème de Fubini}

\contenu{
\texte{
Soit $f(x,y)=
\frac{x^2-y^2}{(x^2+y^2)^2}$. Montrer que
$$
\int_{-1}^{1} \left(\int_{-1}^{1}f(x,y)dx\right) dy \neq
\int_{-1}^{1} \left(\int_{-1}^{1} f(x,y)dy\right) dx.
$$
Y a-t-il contradiction avec le th\'{e}or\`{e}me de Fubini ? (on
pourra calculer l'int\'egrale de $|f|$ sur l'anneau
$S_\varepsilon=\{(x,y)\in\mathbb{R}^2| \varepsilon\leq x^2+y^2\leq
1\}$.)
}
\reponse{
On a
\begin{align*}\int_{-1}^1\left(\int_{-1}^1
\frac{x^2-y^2}{(x^2+y^2)^2}dx\right) dy=&\int_{-1}^1\left(\left.-
\frac{x}{(x^2+y^2)}\right|_{-1}^1\right) dy \\=&\left.-\int_{-1}^1
\frac{2}{(1+y^2)} dy =-2 \arctan y \right|_{-1}^1=-\pi.
\end{align*}
\begin{align*}\int_{-1}^1\left(\int_{-1}^1 \frac{x^2-y^2}{(x^2+y^2)^2}dy\right) dx=&\int_{-1}^1\left(\left.
\frac{y}{(x^2+y^2)}\right|_{-1}^1\right) dx \\=&\left.\int_{-1}^1
\frac{2}{(x^2+1)} dx =2 \arctan x \right|_{-1}^1=\pi.
\end{align*}
Il n'y a pas de contradiction avec le th\'{e}or\`{e}me de Fubini
car la fonction $f$ n'appartient pas \`a
$\mathcal{L}^1([-1,1]\times[-1,1])$. En effet, soit
$S_\varepsilon=\{(x,y)\in\mathbb{R}^2| \varepsilon\leq x^2+y^2\leq
1\}$. On a
$$\int_{[-1,1]\times[-1,1]}|f| d\mu 
\geq \int_{S_\varepsilon} |f| d\mu =\int_{\theta=0}^{2\pi}\int_{r=\varepsilon}^1\frac{|\cos 2\theta|}{r}dr
d\theta
= 4\int_{\theta=0}^{\frac{\pi}{2}}\int_{r=\varepsilon}^1 \frac{|\cos 2\theta|}{r} dr
d\theta = -4 \log \varepsilon\rightarrow \infty$$ lorsque
$\varepsilon\rightarrow 0,$ et donc $f\notin
\mathcal{L}^1([-1,1]\times[-1,1]).$
}
}

\uuid{KEvK}
\exo7id{5933}
\titre{exo7 5933}
\auteur{tumpach}
\organisation{exo7}
\datecreate{2010-11-11}
\isIndication{false}
\isCorrection{true}
\chapitre{Tribu, fonction mesurable}
\sousChapitre{Tribu, fonction mesurable}
\module{Théorie de la mesure, intégrale de Lebesgue}
\niveau{L3}
\difficulte{}

\contenu{
\texte{
Montrer les \'egalit\'es ensemblistes suivantes~:
$$
[a, b] =\bigcap_{n=1}^{\infty}\, ]a-\frac{1}{n}, b+\frac{1}{n}[
\quad\quad\text{et}\quad\quad]a, b[ = \bigcup_{n=1}^{\infty}\, [a+
\frac{1}{n}, b-\frac{1}{n}]
$$
}
\reponse{
Montrons que $ [a, b] =\bigcap_{n=1}^{\infty}
]a-\frac{1}{n}, b+\frac{1}{n}[$.
\begin{itemize}
Pour tout $n\in\mathbb{N}$, on a $[a, b] \subset\,
]a-\frac{1}{n}, b+\frac{1}{n}[$. Donc $[a,
b]\subset\bigcap_{n=1}^{\infty} ]a-\frac{1}{n}, b+\frac{1}{n}[$.
Soit $x\in \bigcap_{n=1}^{\infty} ]a-\frac{1}{n},
b+\frac{1}{n}[$. Alors pour tout $n\in\mathbb{N}$, on a~:
$$a-\frac{1}{n} < x < b+\frac{1}{n}.$$ Ainsi
$$\lim_{n\rightarrow+\infty}(a-\frac{1}{n})\leq x \leq
\lim_{n\rightarrow+\infty}(b+\frac{1}{n}),
$$
c'est-\`a-dire $x\in[a, b]$. Donc $\bigcap_{n=1}^{\infty}
]a-\frac{1}{n}, b+\frac{1}{n}[ \subset [a, b]$ et on a
d\'emontr\'e l'\'egalit\'e entre ces deux ensembles.
\end{itemize}
Montrons que $]a, b[ = \bigcup_{n=1}^{\infty} [a+
\frac{1}{n}, b-\frac{1}{n}]$.
\begin{itemize}
Pour tout $n\in\mathbb{N}$, on a $[a+ \frac{1}{n},
b-\frac{1}{n}]\subset ]a,b[$, donc $\bigcup_{n=1}^{\infty} [a+
\frac{1}{n}, b-\frac{1}{n}]\subset]a,b[$.
Soit
$x\in\bigcup_{n=1}^{\infty} [a+ \frac{1}{n}, b-\frac{1}{n}]$.
Alors il existe  $n\in\mathbb{N}$ tel que $x\in[a+ \frac{1}{n},
b-\frac{1}{n}]$. Ainsi $x\in]a, b[$ et $\bigcup_{n=1}^{\infty} [a+
\frac{1}{n}, b-\frac{1}{n}] \subset ]a, b[$, d'o\`u l'\'egalit\'e
de ces deux ensembles.
\end{itemize}
}
}

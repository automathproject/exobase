\uuid{hp54}
\exo7id{5936}
\auteur{tumpach}
\organisation{exo7}
\datecreate{2010-11-11}
\isIndication{false}
\isCorrection{true}
\chapitre{Tribu, fonction mesurable}
\sousChapitre{Tribu, fonction mesurable}

\contenu{
\texte{
\label{ex:barb20}
Soit $(\Omega, \Sigma)$ un espace mesurable. On dit que
$\varphi~:\Omega \rightarrow \mathbb{R}$ est une \emph{fonction
simple} ou \emph{\'etag\'ee} si $\varphi$ est mesurable et ne
prend qu'un nombre fini de valeurs, i.e. si $\varphi$ s'\'ecrit~:
$$
\varphi = \sum_{j \in J} c_{j} \mathbf{1}_{E_{j}},
$$
o\`u $J$ est un ensemble fini, les ensembles $E_{j}$ sont
mesurables et o\`u, pour $i\neq j$, $c_{i} \neq c_{j}$ et
$E_{i}\cap E_{j} = \emptyset$. Soit $\varphi$ une fonction simple
positive. On rappelle que l'int\'egrale de $\varphi$ par rapport
\`a une mesure $\mu$ est d\'efinie par~:
$$
\int_{\Omega} \varphi \,d\mu = \int_{0}^{\infty}
\mu\left(S_{\varphi}(t) \right)\,dt,
$$
o\`u $S_{\varphi}(t) = \{x\in \Omega, \varphi(x)>t\}$.
}
\begin{enumerate}
    \item \question{Montrer que $$\int_{\Omega} \varphi \,d\mu = \sum_{j \in J}
c_{j} \mu(E_{j}).$$}
    \item \question{Montrer que pour toute fonction r\'eelle
mesurable positive, $f \in \mathcal{M}^{+}(\Omega, \Sigma)$, il
existe une suite $\{\varphi_{n}\}_{n\in\mathbb{N}}$ de fonctions
simples positives telle que~:
\begin{enumerate}}
    \item \question{[(a)] $0 \leq \varphi_{n}(x) \leq \varphi_{n+1}(x)$ pour tout
$x\in\Omega$ et pour tout $n\in\mathbb{N}$~;}
    \item \question{[(b)]
$\lim_{n\rightarrow +\infty} \varphi_{n}(x) = f(x)$ pour tout
$x\in\Omega$.}
\reponse{
On a
$$
\int_{\Omega} \varphi \,d\mu = \int_{0}^{\infty}
\mu\left(S_{\varphi}(t) \right)\,dt,
$$
o\`u $S_{\varphi}(t) = \{x\in \Omega, \varphi(x)>t\} =
\cup_{c_{j}>t} E_{j}$ et o\`u $\mu\left(S_{\varphi}(t) \right) =
\sum_{c_{j}>t} \mu\left(E_{j}\right)$. Ainsi
$$
\int_{\Omega} \varphi \,d\mu = \int_{0}^{\infty}\sum_{c_{j}>t}
\mu\left(E_{j}\right)\,dt = \sum_{j\in
J}\int_0^{c_{j}}\mu\left(E_{j}\right)\,dt =  \sum_{j \in J}\,
c_{j} \mu(E_{j}).
$$
Pour tout $n\in\mathbb{N}$, posons
\begin{eqnarray*}
E_{k,n} &:=& \{x\in\Omega,~ k\, 2^{-n} \leq f(x) <(k+1) 2^{-n}\}
 \quad \text{pour}\quad k = 0, \dots, n \,2^{n} - 1,\\
E_{n, n} &:=& \{x\in\Omega,~ f(x)\geq n\} \quad \text{pour}\quad k
= n\, 2^{n}.
\end{eqnarray*}
Puisque $f$ est mesurable, les ensembles $E_{k, n}$ appartiennent
\`a $\Sigma$. Pour tout $n\in\mathbb{N}$ fix\'e, les ensembles
$E_{k,n}$, $0\leq k\leq n 2^n - 1$ sont deux \`a deux disjoints et
$\cup_{k} E_{k,n} = \Omega$. Posons
$$
\varphi_{n} = \sum_{k=0}^{n 2^{-n}-1} k\, 2^{-n} \mathbf{1}_{E_{k,n}}.
$$
Alors $\varphi_{n}$ est une fonction simple positive v\'erifiant
$\varphi_{n} \leq f$. En outre $0 \leq \varphi_{n}(x) \leq
\varphi_{n+1}(x)$ pour tout $x\in\Omega$ et pour tout
$n\in\mathbb{N}$. De plus,  $\lim_{n\rightarrow +\infty}
\varphi_{n}(x) = f(x)$ pour tout $x\in\Omega$.
}
\end{enumerate}
}

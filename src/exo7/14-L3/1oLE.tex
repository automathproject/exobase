\uuid{1oLE}
\exo7id{5980}
\auteur{tumpach}
\organisation{exo7}
\datecreate{2010-11-11}
\isIndication{false}
\isCorrection{true}
\chapitre{Autre}
\sousChapitre{Autre}

\contenu{
\texte{
\textbf{Th\'eor\`eme}
Soit $\{\varphi_{n}\}_{n\in\mathbb{N}}$ une suite de fonctions de
$\mathbb{R}^n$ dans $\mathbb{R}$  telles que~:
\begin{itemize}
\item[(i)] $\int_{\mathbb{R}^n} \varphi_{n} = 1$ \item[(ii)] il
existe une constante $K
> 0$ telle que $~\sup_{n\in\mathbb{N}} \int_{\mathbb{R}^n} |\varphi_{n}|(x) \,dx \leq
K$ \item[(iii)] Pour tout $\varepsilon>0$, on a
$\lim_{n\rightarrow +\infty}\int_{\|x\|>\varepsilon}
|\varphi_{n}(x)|\,dx = 0$.
\end{itemize}
Alors pour tout $f\in L^p(\mathbb{R}^n)$, $1\leq p <+\infty$,
$\lim_{n\rightarrow+\infty}\|\varphi_{n}*f - f\|_p = 0$.


\bigskip

Le but de cet exercice est de d\'emontrer ce th\'eor\`eme.


Soit $\{\varphi_{n}\}_{n\in\mathbb{N}}$ une suite de fonctions
v\'erifiant les hypoth\`eses (i), (ii) et (iii) du
th\'eor\`eme, et soit $1\leq p <+\infty$.
}
\begin{enumerate}
    \item \question{En notant $q$ l'exposant conjugu\'e de $p$ ($\frac{1}{p} +
\frac{1}{q} = 1$), et en utilisant l'in\'egalit\'e de H\"older
pour la mesure $d\nu(x) = |\varphi_n|(x)\,dx$, montrer que
$$
|\varphi_n * f - f|^p(x) \leq \left(\int_{\mathbb{R}^n}
|\varphi_n|(x)\,dx \right)^{\frac{p}{q}} \left(\int_{\mathbb{R}^n}
|f(x - y) - f(x)|^p |\varphi_n|(y)\,dy\right).
$$}
\reponse{En notant $q$ l'exposant conjugu\'e de $p$ ($\frac{1}{p} +
\frac{1}{q} = 1$), on a
$$
\begin{array}{lcl}
|\varphi_n * f - f|^p(x)& = &|\int_{\mathbb{R}^n} f(x - y)
\varphi_n(y)\,dy - f(x) \int_{\mathbb{R}^n}\varphi_n(y)\,dy|^{p}\\
& & \\ & \leq & \left(\int_{\mathbb{R}^n} |f(x - y) - f(x)|
|\varphi_n(y)|\,dy \right)^p.
\end{array}
$$
En utilisant l'in\'egalit\'e de H\"older pour la mesure $d\nu(x) =
|\varphi_n|(x)\,dx$, on a
$$
\begin{array}{lcl}
|\varphi_n * f - f|^p(x) & \leq & \left(\int_{\mathbb{R}^{n}}
1^{q}\,d\nu(y) \right)^{\frac{p}{q}}\left(
\int_{\mathbb{R}^{n}}|f(x - y) -
f(x)|^p\,d\nu(y)\right)^{\frac{p}{p}}\\
& \leq & \left(\int_{\mathbb{R}^n} |\varphi_n|(y)\,dy
\right)^{\frac{p}{q}} \left(\int_{\mathbb{R}^n} |f(x - y) -
f(x)|^p |\varphi_n|(y)\,dy\right)\\ & \leq &
K^{\frac{p}{q}}\left(\int_{\mathbb{R}^n} |f(x - y) - f(x)|^p
|\varphi_n|(y)\,dy\right).
\end{array}
$$}
    \item \question{En d\'eduire que
$$
\|\varphi_n * f - f\|_{p}^{p} \leq K^{\frac{p}{q}}
\int_{\mathbb{R}^{n}} \|\tau_{y}f - f\|_{p}^p |\varphi_n(y)|\,dy.
$$}
\reponse{On en d\'eduit que
$$
\begin{array}{lcl}
\|\varphi_n * f - f\|_{p}^{p} & \leq &
K^{\frac{p}{q}}\int_{x\in\mathbb{R}^n}\left(\int_{y\in\mathbb{R}^n}|f(x
- y) - f(x)|^p |\varphi_n|(y)\,dy\right)\,dx
\end{array}
$$
D'apr\`es le th\'eor\`eme de Tonelli~:
$$
\begin{array}{lcl}
\|\varphi_n * f - f\|_{p}^{p} & \leq &
K^{\frac{p}{q}}\int_{y\in\mathbb{R}^n}\left(\int_{x\in\mathbb{R}^n}|f(x
- y) - f(x)|^p \,dx\right)|\varphi_n|(y)\,dy\\ & \leq &
 K^{\frac{p}{q}}
\int_{\mathbb{R}^{n}} \|\tau_{y}f - f\|_{p}^p |\varphi_n(y)|\,dy.
\end{array}
$$}
    \item \question{Soit $\delta>0$, montrer que
$$
\|\varphi_n * f - f\|_{p}^{p} \leq
K^{\frac{p}{q}}\left(\sup_{|y|\leq\delta}\|\tau_{y}f - f\|_{p}^p +
2^p\|f\|_{p}^{p}\int_{|y|>\delta} |\varphi_{n}(y)|\,dy \right).
$$}
\reponse{Soit $\delta>0$, on a
$$
\begin{array}{lcl}
 && \|\varphi_n * f - f\|_{p}^{p}  \\ & \leq &  K^{\frac{p}{q}}\left(
\int_{|y|\leq \delta}\|\tau_{y}f - f\|_{p}^p |\varphi_n(y)|\,dy +
\int_{|y|>\delta} \|\tau_{y}f - f\|_{p}^p |\varphi_n(y)|\,dy
\right)\\ & \leq &
K^{\frac{p}{q}}\left(\sup_{|y|\leq\delta}\|\tau_{y}f - f\|_{p}^p
\int_{|y|\leq \delta} |\varphi_n(y)|\,dy + \int_{|y|>\delta}
\left(\|\tau_{y}f\|_p + \|f\|_{p}\right)^p |\varphi_n(y)|\,dy
\right)\\& \leq &
K^{\frac{p}{q}}\left(K\,\sup_{|y|\leq\delta}\|\tau_{y}f -
f\|_{p}^p
 + \left(2\|f\|_{p}\right)^p \int_{|y|>\delta}
|\varphi_n(y)|\,dy \right)\\
& \leq & K^{\frac{p}{q}}\left(K\,\sup_{|y|\leq\delta}\|\tau_{y}f -
f\|_{p}^p + 2^p\|f\|_{p}^{p}\int_{|y|>\delta} |\varphi_{n}(y)|\,dy
\right).
\end{array}
$$}
    \item \question{En d\'eduire le th\'eor\`eme cherch\'e.}
\reponse{Soit $\varepsilon>0$. Par continuit\'e des translations dans
$L^p(\mathbb{R}^n)$ (cf l'exercice pr\'ec\'edent), il existe un
$\delta>0$ tel que
$$
|y|\leq\delta\quad \Rightarrow \|\tau_{y}f - f\|_{p}^p <
\frac{K^{-\left(\frac{p}{q}+1\right)}}{2}~ \varepsilon.
$$
D'apr\`es l'hypoth\`ese (iii), il existe un $N\in\mathbb{N}$ tel
que pour $n>N$, on a
$$
\int_{|y|>\delta} |\varphi_{n}(y)|\,dy <
\frac{K^{-\frac{p}{q}}}{2^{p+1}\|f\|_{p}^{p}}~ \varepsilon.
$$
Ainsi pour tout $n>N$,
$$
\|\varphi_n * f - f\|_{p}^{p} < \varepsilon,
$$
i.e. $\lim_{n\rightarrow+\infty}\|\varphi_n * f - f\|_{p} = 0$.}
\end{enumerate}
}

\uuid{aju5}
\exo7id{5977}
\titre{exo7 5977}
\auteur{tumpach}
\organisation{exo7}
\datecreate{2010-11-11}
\isIndication{false}
\isCorrection{true}
\chapitre{Transformée de Fourier}
\sousChapitre{Transformée de Fourier}
\module{Théorie de la mesure, intégrale de Lebesgue}
\niveau{L3}
\difficulte{}

\contenu{
\texte{
\emph{Le but de cet exercice est de d\'emontrer le th\'eor\`eme
de Plancherel.} 

\textbf{D\'efinition.}
Soient $f, g\in L^1(\mathbb{R}^n)$. On note $\hat{f}$ la
transform\'ee de Fourier d\'efinie par
$$
\hat{f}(y) = \int_{\mathbb{R}^n} f(x)\,e^{-2\pi i (y, x)} \,dx,
$$
o\`u $(\cdot, \cdot)$ d\'esigne le produit scalaire de
$\mathbb{R}^n.$ 

\bigskip

\textbf{Th\'eor\`eme de Plancherel.}
Si $f \in L^{1}(\mathbb{R}^n)\cap L^2(\mathbb{R}^{n})$, alors
$\|\hat{f}\|_{2} = \| f\|_{2}$.

\bigskip


Soit $f \in L^{1}(\mathbb{R}^n)\cap
L^2(\mathbb{R}^{n})$.
}
\begin{enumerate}
    \item \question{Montrer que $\|\hat{f}\|_{\infty}\leq \|f\|_{1}$.}
    \item \question{Montrer que la fonction $g_{\varepsilon}(k) = |\hat{f}(k)|^2
\,e^{-\varepsilon\pi|k|^2}$ appartient \`a
$L^{1}(\mathbb{R}^{n})$.}
    \item \question{Montrer que $$\int_{\mathbb{R}^{n}} g_{\varepsilon}(k)\,dk =
\int_{\mathbb{R}^{3n}} \bar{f}(x) f(y) e^{2\pi i (k, x-y)}
e^{-\varepsilon \pi |k|^2}\,dx dy dk.$$}
    \item \question{Sachant que la transform\'ee de Fourier de la
gaussienne $h_{\varepsilon}(x) = e^{-\pi \varepsilon |x|^2}$
($\varepsilon>0$, $x\in\mathbb{R}^n$) est donn\'ee par
$\hat{h}_{\varepsilon}(k) = {\varepsilon}^{-\frac{n}{2}}
e^{-\frac{\pi|k|^2}{\varepsilon}}$, montrer que
$$\int_{\mathbb{R}^{n}} g_{\varepsilon}(k)\,dk =
\int_{\mathbb{R}^{2n}}{\varepsilon}^{-\frac{n}{2}}
e^{-\frac{\pi|x-y|^2}{\varepsilon}} \bar{f}(x) f(y) \,dx dy.$$}
    \item \question{Soit $\{s_{\varepsilon}\}_{\varepsilon>0}$ la famille de
fonctions d\'efinies par~:
$$
s_{\varepsilon} = \int_{\mathbb{R}^n} {\varepsilon}^{-\frac{n}{2}}
e^{-\frac{\pi|x-y|^2}{\varepsilon}} \bar{f}(x)\,dx.
$$
Quelle est la limite dans $L^{2}(\mathbb{R}^{n})$ de
$s_{\varepsilon}$ lorsque $\varepsilon$ tend vers $0$?}
    \item \question{Montrer que $$\lim_{\varepsilon\rightarrow 0}
\int_{\mathbb{R}^{n}} g_{\varepsilon}(k)\,dk =
\int_{\mathbb{R}^{n}} (\lim_{\varepsilon\rightarrow
0}s_{\varepsilon}) f(y) \,dy.$$}
    \item \question{Montrer que $\lim_{\varepsilon\rightarrow 0}
\int_{\mathbb{R}^{n}} g_{\varepsilon}(k)\,dk =
\|\hat{f}\|_{2}^2$.}
    \item \question{En d\'eduire que $\|\hat{f}\|_{2} = \| f\|_{2}$.}
\reponse{
cf E.~Lieb et M.~Loss, \emph{Analysis}, p.118, American Mathematical Society (2001).
(Pour la question~6, on peut utiliser la continuit\'e du produit
scalaire dans $L^{2}(\mathbb{R}^{n})$.)
}
\end{enumerate}
}

\uuid{4rXe}
\exo7id{5947}
\auteur{tumpach}
\organisation{exo7}
\datecreate{2010-11-11}
\isIndication{false}
\isCorrection{true}
\chapitre{Théorème de convergence dominée}
\sousChapitre{Théorème de convergence dominée}

\contenu{
\texte{
On rappelle qu'une fonction $f~:\Omega \rightarrow \mathbb{R}$ est
dite int\'egrable si $f_{+} := \max\{f, 0\}$ et $f_{-} = \max\{-f,
0\}$ v\'erifient $\int_{\Omega} f_{+} \,d\mu ~<~+\infty$ et
$\int_{\Omega} f_{-} \,d\mu ~<~+\infty$. On note
$\mathcal{L}^{1}(\Omega, \Sigma, \mu)$ l'ensemble des fonctions
r\'eelles int\'egrables. Pour $f\in \mathcal{L}^{1}(\Omega,
\Sigma, \mu)$, on pose
$$
\int_{\Omega} f\,d\mu = \int_{\Omega} f_{+}\,d\mu - \int_{\Omega}
f_{-}\,d\mu.
$$
}
\begin{enumerate}
    \item \question{Montrer l'\'equivalence
$$
f\in \mathcal{L}^{1}(\Omega, \Sigma, \mu) \Leftrightarrow |f| \in
\mathcal{L}^{1}(\Omega, \Sigma, \mu)
$$
et \begin{equation}\label{leq} \left|\int_{\Omega} f\,d\mu
\right|~\leq~ \int_{\Omega} |f|\,d\mu.
\end{equation}}
\reponse{Par d\'efinition, $ f\in \mathcal{L}^{1}(\Omega,
\Sigma, \mu)$ si et seulement si $f_{+}$ et $f_{-}$ sont
int\'egrables. On note que $|f| = f_{+} + f_{-}$. Donc $f\in
\mathcal{L}^{1}(\Omega, \Sigma, \mu)\Rightarrow |f| \in
\mathcal{L}^{1}(\Omega, \Sigma, \mu)$. R\'eciproquement, on a $0
\leq f_{\pm} \leq |f|$, donc $ |f| \in \mathcal{L}^{1}(\Omega,
\Sigma, \mu)\Rightarrow f\in \mathcal{L}^{1}(\Omega, \Sigma,
\mu)$. D'autre part~:
\begin{equation*} \left|\int_{\Omega} f\,d\mu
\right| ~=~ \left|\int_{\Omega} f_{+}\,d\mu - \int_{\Omega}
f_{-}\,d\mu\right|~\leq~ \int_{\Omega} f_{+}\,d\mu + \int_{\Omega}
f_{-}\,d\mu ~=~\int_{\Omega} |f|\,d\mu.
\end{equation*}}
    \item \question{Montrer que si $f$ est mesurable, $g$ int\'egrable et
$|f|~\leq |g|$, alors $f$ est int\'egrable et
$$
\int_{\Omega} |f|\,d\mu ~\leq~\int_{\Omega} |g|\,d\mu.
$$}
\reponse{Par monotonie de l'int\'egrale, on a
$$
\int_{\Omega} |f|\,d\mu ~\leq~\int_{\Omega} |g|\,d\mu < +\infty.
$$
D'apr\`es la question $(a)$, il en d\'ecoule que $f$ est
int\'egrable.}
    \item \question{On rappelle qu'une fonction $f~:\Omega \rightarrow
\mathbb{C}$ est dite int\'egrable si la partie r\'eelle $\text{Re}
f$ et la partie imaginaire $\text{Im} f$ de $f$ sont
int\'egrables. On pose alors
$$
\int_{\Omega} f\,d\mu =  \int_{\Omega}\text{Re} f\,d\mu + i
\int_{\Omega}\text{Im} f\,d\mu.
$$
Montrer que l'in\'egalit\'e  \eqref{leq} est v\'erifi\'ee.}
\reponse{D\'efinissons $z = \int_{\Omega} f\,d\mu$. Comme $z$
est un nombre complexe, il s'\'ecrit $z = |z|e^{i\theta}$. Soit
$u$ la partie r\'eelle de $e^{-i\theta} f$. On a $u \leq
|e^{-i\theta}f| = |f|.$ Donc
$$
\left|\int_{\Omega} f\,d\mu \right|~=~ e^{-i\theta}\int_{\Omega}
f\,d\mu ~=~\int_{\Omega}e^{-i\theta} f\,d\mu ~=~ \int_{\Omega}
u\,d\mu ~\leq~ \int_{\Omega} |f|\,d\mu,
$$
o\`u la troisi\`eme \'egalit\'e d\'ecoule du fait que le nombre
$\int_{\Omega}e^{-i\theta} f\,d\mu$ est r\'eel donc est
l'int\'egrale de la partie r\'eelle de $e^{-i\theta} f$
c'est-\`a-dire de $u$.}
\end{enumerate}
}

\uuid{BH33}
\exo7id{2518}
\titre{exo7 2518}
\auteur{queffelec}
\organisation{exo7}
\datecreate{2009-04-01}
\isIndication{false}
\isCorrection{true}
\chapitre{Différentiabilité, calcul de différentielles}
\sousChapitre{Différentiabilité, calcul de différentielles}
\module{Calcul différentiel}
\niveau{L3}
\difficulte{}

\contenu{
\texte{

}
\begin{enumerate}
    \item \question{Soit $f$ une application r\'eelle continue et d\'erivable
sur $]a,b[$ telle que $f'(x)$ ait une limite quand
$x\buildrel{<}\over{\to}b$; alors $f$ se prolonge en une fonction
continue et d\'erivable \`a gauche au point $b$.}
    \item \question{Soit $f$ une application continue et d\'erivable sur un
intervalle $I\subset\Rr$, et de d\'eriv\'ee croissante; montrer
que $f$ est convexe sur $I$ i.e. $f((1-t)x+ty)\leq
(1-t)f(x)+tf(y)$ pour tous $x<y$ de $I$ et $t\in[0,1]$. (Poser
$z=(1-t)x+ty$ et appliquer les AF \`a $[x,z]$ puis $[z,y]$.)}
\reponse{
Montrons que $f$ se prolonge par continuit\'e au point $b$, on
montrera alors que $f$ est d\'erivable \`a gauche au point $b$ est
que cette d\'eriv\'ee est $\lim_{x \rightarrow b^-}f'(x)$. Pour
cel\`a montrons qu'il existe un r\'eel $k$ tel que toute suite
$\{x_n\}$ tendant vers $b$ v\'erifie $\lim_{n\rightarrow \infty}
f(x_n)=k$. Remarquons que la d\'eriv\'ee $f'(x)$ admettant une
limite au point $b$, elle est born\'ee sur un petit voisinage (\`a
gauche) de $b$ (notons $M$ ce majorant).
 Soit $y_n$ une suite
convergent vers $b$. Alors la suite $f(y_n)$ est de Cauchy. En
effet, pour tout $\epsilon >0$, posons
$\epsilon'=\frac{\epsilon}{2M}$. La suite $\{y_n\}$ \'etant de
cauchy, $$\exists N \in \mathbb{N}, p,q \geq N \Rightarrow
|y_p-y_q| \leq  \epsilon' \leq \frac{\epsilon}{2M}.$$ Or d'apr\`es
les accroissements finis:
$$f(y_p)-f(y_q)=(y_p-y_q)f'(c_{p,q}) \mbox { o\`u } c_{p,q} \in
]y_p,y_q[.$$ Par cons\'equent, $$|f(y_p)-f(y_q)|\leq
|y_p-y_q|.|f'(c_{p,q})| \leq \frac{\epsilon}{2M}M \leq
\frac{epsilon}{2} < \epsilon$$ et donc la suite $\{f(y_n)\}$ est
de cauchy et converge vers un r\'eel que nous noterons $l$.
Montrons que c'est le cas pour toute autre suite $\{x_n\}$ qui
tend vers $b$. On a $$f(x_n)=f(x_n)-f(y_n)+f(y_n).$$ D'apr\`es les
accroissements finis,  $|f(x_n)-f(y_n)| \leq M.|x_n-y_n|$ et donc
tend vers zero car les suites $x_n$ et $y_n$ tendent vers $b$. De
plus, comme on l'a vu, $f(y_n)$ tend vers $k$ et donc $f(x_n)$
aussi. Prolongeons $f$ par continuit\'e au point $b$ en posant
$f(b)=k$. On a alors le taux d'accroissement
$$T_xf=\frac{f(b)-f(x)}{b-x}=\frac{(b-x)f'(c_x)}{b-x}=f'(c_x)
\mbox{ o\`u } c_x \in ]x,b[.$$ Quand $x$ tend vers $b$, $c_x$
aussi et donc $T_xf$ tend vers $l$.
}
\end{enumerate}
}

\uuid{ntOd}
\exo7id{2521}
\auteur{queffelec}
\organisation{exo7}
\datecreate{2009-04-01}
\isIndication{false}
\isCorrection{true}
\chapitre{Différentiabilité, calcul de différentielles}
\sousChapitre{Différentiabilité, calcul de différentielles}

\contenu{
\texte{
On consid\`ere
l'application $F:{\Rr^2}\to{\Rr^2}$ d\'efinie par $F(x,y)=(\cos
x-\sin y,\ \sin x-\cos y)$; on note $F^{(k)}$ l'application $F$
compos\'ee $k$-fois
}
\begin{enumerate}
    \item \question{Montrer que $||DF(x,y)||\leq \sqrt2$ pour tout $(x,y)$.}
\reponse{On a
$$Df(x,y)=\left(\begin{array}{cc}
-\sin x & - \cos y \\ \cos x & \sin y
\end{array}\right)$$

On a
$$|||Df(x,y)|||=\sup_{(a,b) \in \mathbb{R}^2\backslash
\{(0,0)\}}\frac{ ||Df(x,y).(a,b)||}{||(a,b)||}=$$
$$\frac{\sqrt{a^2 \sin^2 x + b^2 \cos^2 y +2ab \sin x \cos x + a^2 \cos^2 x + b^2 \sin^2 y + 2ab \cos x \sin y}}{\sqrt{a^2+b^2}}=$$
$$\frac{\sqrt{a^2+b^2+2ab \sin(x+y)}}{\sqrt{a^2+b^2}} \leq \sqrt{1+\frac{2|a||b|}{a^2+b^2}} \leq \sqrt 2$$
car $$(|a|-|b|)^2 \geq0\Rightarrow a^2+b^2 \geq 2 |a||b|.$$}
    \item \question{En d\'eduire que la suite r\'ecurrente d\'efinie par
$x_0,y_0$ et pour $n\geq 1$
$$x_{n+1}=
{\frac 1 2}(\cos x_n-\sin y_n),\quad y_{n+1}= {\frac 1 2}(\sin
x_n-\cos y_n)$$ converge pour tout $(x_0,y_0)$. Donnez
l'\'equation que v\'erifie sa limite ?}
\reponse{Soient $U_n=(x_n,y_n)$ et $G(x,y)=1/2F(x,y)$, alors $|||G|||
\leq \frac{\sqrt 2}{2}$ et $U_{n+1}=G(U_n)$. D'apr\`es les
accroissements finis, $G$ est contractante et donc le th\'eor\`eme
du point fixe donne le r\'esultat demand\'e.}
\end{enumerate}
}

\uuid{NSfg}
\exo7id{6832}
\titre{exo7 6832}
\auteur{gijs}
\organisation{exo7}
\datecreate{2011-10-16}
\isIndication{false}
\isCorrection{false}
\chapitre{Différentiabilité, calcul de différentielles}
\sousChapitre{Différentiabilité, calcul de différentielles}
\module{Calcul différentiel}
\niveau{L3}
\difficulte{}

\contenu{
\texte{
Soit $U$ un ouvert de $\Rr^n$ et $f: U \to \Rr^p$ une
application dérivable.
}
\begin{enumerate}
    \item \question{\'Enoncer l'inégalité des accroissements
finis. (Pour cette question on peut supposer que $U$ est
con{\bf v}exe.)}
    \item \question{Démontrer, à l'aide de 1., la proposition
suivante :

Si pour tout $x\in U$ la dérivée
de $f$ en $x$ est nulle~: $Df(x) = 0$, alors pour tout $x$
dans $U$ il existe un voisinage $V$ de $x$ dans $U$ (par
exemple une boule centrée en $x$) tel que $f$ est
constante sur $V$.}
    \item \question{\`A l'aide de 2., démontrer que si en plus
$U$ est co{\bf nn}exe, alors $f$ est constante sur $U$.}
\end{enumerate}
}

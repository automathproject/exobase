\uuid{jmqK}
\exo7id{6289}
\titre{exo7 6289}
\auteur{mayer}
\organisation{exo7}
\datecreate{2011-10-16}
\isIndication{false}
\isCorrection{false}
\chapitre{Différentielle d'ordre supérieur, formule de Taylor}
\sousChapitre{Différentielle d'ordre supérieur, formule de Taylor}
\module{Calcul différentiel}
\niveau{L3}
\difficulte{}

\contenu{
\texte{
Soit $f:\Rr^n\to\Rr^n$ de classe $C^2$ telle que, pour tout $x\in
\Rr^n$, l'application $Df(x)$ est un automorphisme orthogonal, i.e.
$Df(x)$ est linéaire bijective et conserve le produit scalaire:
$$\langle Df(x)(h), Df(x)(k)\rangle=\langle h,k\rangle \quad \text{pour tout} \;\; h,k\in
\Rr^n\; .$$ Montrer que l'application $f$ est elle même un
automorphisme orthogonal.

Indications:
}
\begin{enumerate}
    \item \question{Déterminer la différentielle de $x\mapsto \langle Df(x)(h),
Df(x)(k)\rangle$.}
    \item \question{Vérifier que $A(h,k,l) = \langle Df(x)(h), D^2f(x)(k,l)\rangle$ est
antisymétrique par rapport aux deux premières variables et
symétrique par rapport aux deux dernières variables.}
    \item \question{En déduire que $A(h,k,l)=0$ pour tous $h,k,l\in \Rr^n$ puis conclure.}
\end{enumerate}
}

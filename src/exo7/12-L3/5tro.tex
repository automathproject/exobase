\uuid{5tro}
\exo7id{1843}
\auteur{legall}
\organisation{exo7}
\datecreate{2003-10-01}
\isIndication{false}
\isCorrection{false}
\chapitre{Extremum, extremum lié}
\sousChapitre{Extremum, extremum lié}

\contenu{
\texte{
$S=\{ z\in \Cc ; \vert z\vert = 1\} .$ Soit $f $ l'application
de $D$ dans $\Rr$ d\'efinie par $f(z)=\vert \sin z\vert .$
}
\begin{enumerate}
    \item \question{Pour quelle raison $f$ est-elle born\'ee sur $D$~? On note 
$\displaystyle{ M=\sup_{z\in D}f(z)}$ et $\displaystyle{m=\inf_{z\in 
D}f(z)}.$
Est-ce que $M$ et $m$ sont atteints~? Donner la valeur de $m$.}
    \item \question{Soit $z= x+iy\in \Cc , x, y \in \Rr.$ Montrer que $\vert \sin z\vert ^2=
\frac{1}{2}(\ch 2y-\cos 2x).$ (On rappelle que $\sin z 
=\frac{e^{i(x+iy)}-e^{-i(x+iy)}}{2i}$ et $\ch 
y=\frac{e^{y}-e^{-y}}{2}.$)}
    \item \question{En d\'eduire que $M$ est atteint en un point de $S$.}
    \item \question{Montrer que $\displaystyle{M=\frac{e^2-1}{2e}}$.}
\end{enumerate}
}

\uuid{3qI3}
\exo7id{4199}
\titre{exo7 4199}
\auteur{quercia}
\organisation{exo7}
\datecreate{2010-03-11}
\isIndication{false}
\isCorrection{true}
\chapitre{Fonction de plusieurs variables}
\sousChapitre{Extremums locaux}
\module{Analyse}
\niveau{L2}
\difficulte{}

\contenu{
\texte{
D{\'e}terminer le plus court chemin entre les p{\^o}les nord et sud d'une sph{\`e}re en dimension $3$.
}
\reponse{
On param{\`e}tre le chemin en coordonn{\'e}ees sph{\'e}riques par $t\mapsto (\theta(t),\phi(t))$.

La longueur du chemin est ${ \int_{t=0}^{1}\sqrt{\phi'^{2}(t)+\sin^{2}(\phi(t))\theta'^{2}(t)}d t\ge \Bigl| \int_{t=0}^{1}\phi'(t)d t\Bigr|}$ avec {\'e}galit{\'e} si et 
seulement si $\theta'=0$ et $\phi'$ est de signe constant. On trouve donc les m{\'e}ridiens.
}
}

\uuid{Qmjj}
\exo7id{4184}
\auteur{quercia}
\organisation{exo7}
\datecreate{2010-03-11}
\isIndication{false}
\isCorrection{true}
\chapitre{Fonction de plusieurs variables}
\sousChapitre{Dérivée partielle}

\contenu{
\texte{
Soit ${f} : {\R^n} \to {\R^n}$ de classe $\mathcal{C}^1$ et $c>0$ tels que, pour tous $x,y$, 
$\| f(x)-f(y)\| \ge c \|x-y\|$.
}
\begin{enumerate}
    \item \question{Montrer que pour tous $x,h$, $\|d f_x(h)\|\ge c\|h\|$.}
    \item \question{Montrer que $f$ est un $\mathcal{C}^1$-difféomorphisme sur $\R^n$ (pour la surjectivité on 
    considèrera, si $a\in \R^n$, le minimum de $\|f(x)-a\|^2$).}
\reponse{
$\|f(x)\|\to+\infty$ lorsque $\|x\|\to\infty$ donc $x \mapsto\|f(x)-a\|^2$
    admet un minimum sur~$\R^n$. En ce point on a pour tout $h\in\R^n$~:
    $(f(x)-a\mid d f_x(h)) = 0$ et $d f_x$ est surjective
    (linéaire injective en dimension finie) donc $f(x)=a$.
}
\end{enumerate}
}

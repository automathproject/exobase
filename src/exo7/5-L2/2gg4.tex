\uuid{2gg4}
\exo7id{1787}
\auteur{drutu}
\organisation{exo7}
\datecreate{2003-10-01}
\isIndication{true}
\isCorrection{true}
\chapitre{Fonction de plusieurs variables}
\sousChapitre{Limite}

\contenu{
\texte{
D\'eterminer les limites
lorsqu'elles existent:
}
\begin{enumerate}
    \item \question{$\lim_{(x,y)\to (0,0)} \frac{x}{x^2+y^2}  $}
    \item \question{$ \lim_{(x,y)\to (0,0)} \frac{(x+2y)^3}{x^2+y^2} $}
    \item \question{$ \lim_{(x,y)\to (1,0)} \frac{\log (x+e^y)}{\sqrt{x^2+y^2}} $}
    \item \question{$\lim_{(x,y)\to (0,0)} \frac{x^4+y^3-xy}{x^4+y^2} $}
    \item \question{$ \lim_{(x,y)\to (0,0)} \frac{x^3y}{x^4+y^4}  $ ;}
    \item \question{$ \lim_{(x,y)\to (0,0)} \frac{(x^2+y^2)^2}{x^2-y^2} $ ;}
    \item \question{$ \lim_{(x,y)\to (0,0)} \frac{1-\cos xy}{y^2}  $ ;}
    \item \question{$ \lim_{(x,y)\to (0,0)} \frac{\sin x}{\cos y-\cosh x} $}
\reponse{
$\lim_{(x,y)\to (0,0),y=0} \frac{x}{x^2+y^2}$ n'existe pas
d'o\`u $\lim_{(x,y)\to (0,0)} \frac{x}{x^2+y^2}  $ n'existe pas.
$\frac{(x+2y)^3}{x^2+y^2} =r(\cos \varphi +2 \sin \varphi)^3$ d'o\`u
$\left|\frac{(x+2y)^3}{x^2+y^2}\right| \leq 27 r$ et
\[
\lim_{(x,y)\to (0,0)} \frac{(x+2y)^3}{x^2+y^2} =0
\]
car $\lim_{(x,y)\to (0,0)} r=0$.
$\lim_{(x,y)\to (1,0)} {\sqrt{x^2+y^2}} =1 \ne 0$ et
$\lim_{(x,y)\to (1,0)} {\log (x+e^y)}=\log 2$  d'o\`u
\[
\lim_{(x,y)\to (1,0)} \frac{\log (x+e^y)}{\sqrt{x^2+y^2}}=\log 2. 
\]
$\mathrm{lim}_{\begin{smallmatrix}(x,y)\to (0,0)\\y=2\end{smallmatrix}} \frac{x^4+y^3-xy}{x^4+y^2} = 1$ tandis que
$\mathrm{lim}_{\begin{smallmatrix}(x,y)\to (0,0)\\x=0\end{smallmatrix}} \frac{x^4+y^3-xy}{x^4+y^2} =0$ 
 d'o\`u
\[
\lim_{(x,y)\to (0,0)} \frac{x^4+y^3-xy}{x^4+y^2} 
\]
n'existe pas.
}
\indication{\begin{enumerate}
\item 
R\'efuter l'existence de la limite \`a l'aide
de l'\'etude des limites le long de deux courbes adapt\'ees.

\item Utiliser les coordonn\'ees polaires dans le plan.

\item Si $\lim_{(x,y) \to (x_0,y_0)}h(x,y)$ existe et est non nul
alors 
\[
\lim_{(x,y) \to (x_0,y_0)}\frac{f(x,y)}{h(x,y)}=
\frac{\lim_{(x,y) \to (x_0,y_0)}f(x,y)}{\lim_{(x,y) \to (x_0,y_0)}h(x,y)} .
\]
\item
Chercher deux courbes dans
le domaine de d\'efinition
qui tendent vers l'origine
telles que les limites, calcul\'ees le long de ces courbes,
existent mais ont des valeures distinctes.
\end{enumerate}}
\end{enumerate}
}

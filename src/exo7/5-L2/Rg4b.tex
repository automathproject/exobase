\uuid{Rg4b}
\exo7id{4144}
\titre{exo7 4144}
\auteur{quercia}
\organisation{exo7}
\datecreate{2010-03-11}
\isIndication{false}
\isCorrection{false}
\chapitre{Fonction de plusieurs variables}
\sousChapitre{Dérivée partielle}
\module{Analyse}
\niveau{L2}
\difficulte{}

\contenu{
\texte{
Soient $f : {\R^3} \mapsto \R$ de classe $\mathcal{C}^2$,
$\Phi : {\R^3} \to {\R^3}, {(r,\theta,\varphi)} \mapsto
{(x,y,z)}$ avec $\begin{cases} x = r\cos\theta\cos\varphi\cr
                         y = r\sin\theta\cos\varphi\cr
                         z = r\sin\varphi,\cr \end{cases}$
et $F = f\circ \Phi$.
Vérifier que :
$$(\Delta f)\circ \Phi = \frac{\partial^2 F}{\partial r^2} + \frac2r\frac{\partial F}{\partial r} + \frac1{r^2}\frac{\partial^2 F}{\partial \varphi^2}
  -\frac{\tan\varphi}{r^2}\frac{\partial F}{\partial \varphi} +\frac1{r^2\cos^2\varphi}\frac{\partial^2 F}{\partial \theta^2}.$$

{\it Pour cet exercice, il est conseillé de prendre la feuille dans le sens de
la longueur, et d'y aller calmement, en vérifiant ses calculs.}
}
}

\uuid{OCRX}
\exo7id{2625}
\auteur{debievre}
\organisation{exo7}
\datecreate{2009-05-19}
\isIndication{true}
\isCorrection{true}
\chapitre{Fonction de plusieurs variables}
\sousChapitre{Dérivée partielle}

\contenu{
\texte{
Soit $f\colon \R^2\rightarrow \R$ la fonction
d\'efinie par 
\begin{eqnarray*}
f(x,y)&=&\frac{x^2y +3y^3}{x^2+y^2} \ \mathrm{ pour }\ (x,y)\not=(0,0),\\
f(0,0)&=& 0.
\end{eqnarray*}
}
\begin{enumerate}
    \item \question{La fonction $f$ est-elle continue en $(0,0)$? Justifier la
r\'eponse.}
\reponse{Puisque $f(x,y)=\frac{x^2y +3y^3}{x^2+y^2}
=r(\cos^2 \varphi \sin \varphi + 3 \sin^3 \varphi)$, il s'ensuit que
\[
\mathrm{lim}_{\begin{smallmatrix} (x,y) \to (0,0)\\ (x,y) \ne (0,0)
\end{smallmatrix}} f(x,y) = 0
\]
car $\cos^2 \varphi \sin \varphi + 3 \sin^3 \varphi$ reste born\'e.
Par cons\'equent la fonction $f$ est continue en $(0,0)$.}
    \item \question{La fonction $f$ admet-elle des d\'eriv\'ees partielles par
rapport \`a $x$, \`a $y$ en $(0,0)$? Donner la ou les valeurs le cas
\'ech\'eant et justifier la r\'eponse.}
\reponse{Les d\'eriv\'ees partielles
\begin{align*}
\frac{\partial f}{\partial x}(0,0)&=
\mathrm{lim}_{\begin{smallmatrix} x \to 0\\ x \ne 0
\end{smallmatrix}}\frac {f(x,0)}x =
\mathrm{lim}_{\begin{smallmatrix} x \to 0\\ x \ne 0
\end{smallmatrix}}\frac{0}{x^2} = 0
\\
\frac{\partial f}{\partial y}(0,0)&=
\mathrm{lim}_{\begin{smallmatrix} y \to 0\\ y \ne 0
\end{smallmatrix}}\frac {f(0,y)}y =
\mathrm{lim}_{\begin{smallmatrix} y \to 0\\ y \ne 0
\end{smallmatrix}}\frac{3y^3}{y^2} = 3
\end{align*}
existent.}
    \item \question{La fonction $f$ est-elle diff\'erentiable en $(0,0)$? Justifier
la r\'eponse.}
\reponse{Puisque $f(x,x)=\frac{4x^3}{2x^2}= 2x$, la d\'eriv\'ee
directionnelle $D_vf(0,0)$ suivant le vecteur $v=(1,1)$ est non nulle.
Par cons\'equent, la fonction $f$ n'est pas diff\'erentiable en $(0,0)$.}
    \item \question{D\'eterminer les d\'eriv\'ees partielles de $f$ en un point
$(x_0,y_0)\not=(0,0)$.}
\reponse{\begin{align*}
\frac{\partial f}{\partial x}&=
\frac{\partial }{\partial x}\frac{x^2y +3y^3}{x^2+y^2}
=\frac{2xy(x^2+y^2)-2x(x^2y +3y^3)}{(x^2+y^2)^2}=
-4\frac{xy^3}{(x^2+y^2)^2}
\\
\frac{\partial f}{\partial y}&=
\frac{\partial }{\partial y}\frac{x^2y +3y^3}{x^2+y^2}
=\frac{(x^2+3y^2)(x^2+y^2)-2y(x^2y +3y^3)}{(x^2+y^2)^2}=
\frac{x^4 +8x^2y^2 +3y^4}{(x^2+y^2)^2}
\end{align*}}
    \item \question{D\'eterminer l'\'equation du plan tangent au graphe de $f$ au 
point $(1,1,2)$.}
\reponse{D'apr\`es \eqref{eqt}, cette \'equation s'\'ecrit
\[
z-2 =\frac{\partial f}{\partial x}(1,1)(x-1)+
\frac{\partial f}{\partial y}(1,1)(y-1)
=1-x+3(y-1)
\]
d'où $z=3y-x$.}
    \item \question{Soit $F:\R^2\rightarrow \R^2$ la fonction d\'efinie par
$F(x,y)=(f(x,y),f(y,x))$. D\'eterminer la matrice jacobienne de $F$ au
point $(1,1)$. La fonction $F$ admet-elle une r\'eciproque locale au
voisinage du point $(2,2)$?}
\reponse{La fonction $F\colon \R^2 \to \R^2$ s'\'ecrit 
$F(x,y)=\left(\frac{x^2y +3y^3}{x^2+y^2}, \frac{y^2x +3x^3}{x^2+y^2}\right)$
et sa matrice jacobienne
\[
\mathrm{J}_F(1,1)=
\left[
\begin{matrix} 
\frac{\partial f}{\partial x}(1,1)
&
\frac{\partial f}{\partial y}(1,1)
\\
\frac{\partial f}{\partial y}(1,1)
&
\frac{\partial f}{\partial x}(1,1)
\end{matrix}
\right]
= \left[
\begin{matrix} -1 & 3\\ 3 & -1 \end{matrix}
\right]
\]
au point $(1,1)$ est inversible.
Par cons\'equent, la fonction $F$ admet une r\'eciproque locale au
voisinage du point $(1,1)$.
Au point $(2,2)$,
\[
\mathrm{J}_F(2,2)=
\left[
\begin{matrix} 
\frac{\partial f}{\partial x}(2,2)
&
\frac{\partial f}{\partial y}(2,2)
\\
\frac{\partial f}{\partial y}(2,2)
&
\frac{\partial f}{\partial x}(2,2)
\end{matrix}
\right]
= \left[
\begin{matrix} -1 & 3\\ 3 & -1 \end{matrix}
\right]
\]
d'o\`u la fonction $F$ admet \'egalement une r\'eciproque locale au
voisinage du point $(2,2)$.}
\indication{\begin{enumerate}
\item Pour r\'efuter la diff\'erentiabilit\'e de $f$ en $(0,0)$, il suffit de 
trouver
une d\'eriv\'ee directionnelle qui n'est pas combinaison lin\'eaire
des d\'eriv\'ees partielles (par rapport aux deux variables).

\item Le plan tangent au point $(x_0,y_0,f(x_0,y_0))$
du graphe $z=f(x,y)$ de $F$ est donn\'ee par l'\'equation
\begin{equation}
z-f(x_0,y_0) =\frac{\partial f}{\partial x}(x_0,y_0)(x-x_0)+
\frac{\partial f}{\partial y}(x_0,y_0)(y-y_0) .
\label{eqt}
\end{equation}
\end{enumerate}}
\end{enumerate}
}

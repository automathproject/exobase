\uuid{6Z4K}
\exo7id{6997}
\auteur{blanc-centi}
\organisation{exo7}
\datecreate{2015-07-04}
\video{P_IrLQoPqYs}
\isIndication{false}
\isCorrection{true}
\chapitre{Equation différentielle}
\sousChapitre{Résolution d'équation différentielle du deuxième ordre}

\contenu{
\texte{
Résoudre
}
\begin{enumerate}
    \item \question{$y''-3y'+2y=0$}
\reponse{Il s'agit d'une équation homogène du second ordre. L'équation caractéristique associée est $r^2-3r+2=0$, qui admet deux solutions: $r=2$ et $r=1$. Les solutions sont donc les fonctions définies sur $\R$ par $y(x)=\lambda e^{2x}+\mu e^x$ ($\lambda,\mu\in\R$).}
    \item \question{$y''+2y'+2y=0$}
\reponse{L'équation caractéristique associée est $r^2+2r+2=0$, qui admet deux solutions: $r=-1+i$ et $r=-1-i$. On sait alors que les solutions sont donc les fonctions définies sur $\R$ par $y(x)=e^{-x}(A\cos x+ B\sin x)$ ($A, B\in\R$). Remarquons que, en utilisant l'expression des fonctions $\cos$ et $\sin$ à l'aide d'exponentielles, ces solutions peuvent aussi s'écrire sous la forme $\lambda e^{(-1+i)x}+\mu e^{(-1-i)x}$ $(\lambda,\mu\in\R)$.}
    \item \question{$y''-2y'+y=0$}
\reponse{L'équation caractéristique est $r^2-2r+1=0$, dont 1 est racine double. 
Les solutions de l'équation homogène sont donc de la forme $(\lambda x+\mu)e^x$.}
    \item \question{$y''+y=2\cos^2x$}
\reponse{Les solutions de l'équation homogène sont les $\lambda\cos x+\mu \sin x$. 
Le second membre peut en fait se réécrire $\cos^2 x=1+\cos(2x)$: 
d'après le principe de superposition, on cherche une solution particulière sous 
la forme $a+b\cos(2x)+c\sin(2x)$. En remplaçant, on trouve qu'une telle fonction est solution si $a=1$, $b=-\frac{1}{3}$, $c=0$. Les solutions générales sont donc les $\lambda\cos x+\mu \sin x-\frac{1}{3}\cos(2x)+1$.}
\end{enumerate}
}

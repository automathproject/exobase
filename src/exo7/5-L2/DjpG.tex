\uuid{DjpG}
\exo7id{1785}
\auteur{drutu}
\organisation{exo7}
\datecreate{2003-10-01}
\isIndication{true}
\isCorrection{true}
\chapitre{Fonction de plusieurs variables}
\sousChapitre{Limite}

\contenu{
\texte{
Soit $f \colon \R^2 \setminus \{(0,0)\} \to \R$ la fonction d\'efinie par
\[
f(x,y) = \frac{x^2y^2}{x^2y^2 + (x-y)^2} .
\]
Montrer que
\begin{equation}
\lim_{x\to 0}\lim_{y\to 0}f(x,y) =  \lim_{y\to 0}\lim_{x\to 0}f(x,y) = 0
\label{non}
\end{equation}
et que $\lim_{(x,y)\to (0,0)} f(x,y)$ n'existe pas.
}
\indication{Diviser le num\'erateur et le d\'enominateur
par $x^2$ resp. $y^2$ pour d\'eterminer $\lim_{y\to 0}f(x,y)$
resp. $\lim_{x\to 0}f(x,y)$. Montrer que, 
calcul\'ee le long d'une autre courbe
convenable, $\lim_{(x,y)\to (0,0)} f(x,y)$ existe et ne vaut pas z\'ero.}
\reponse{
\[ 
\lim_{y\to 0}f(x,y) =  \lim_{y\to 0}\frac{y^2}{y^2 + (1-y/x)^2} 
=\frac{\lim_{y\to 0} y^2}{\lim_{y\to 0} y^2 + (1-y/x)^2}
=\frac{0}{1} =0
\]
De m\^eme $\lim_{x\to 0}f(x,y) =0$ d'o\`u \eqref{non}.
D'autre part,
$f(x,x) =  \frac{x^4}{x^4} =1$ d'o\`u
$\lim_{x\to 0}f(x,x) = 1$ et
$\lim_{(x,y)\to (0,0)} f(x,y)$ ne peut pas exister.
}
}

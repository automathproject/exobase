\uuid{PIXt}
\exo7id{4372}
\auteur{quercia}
\organisation{exo7}
\datecreate{2010-03-12}
\isIndication{false}
\isCorrection{true}
\chapitre{Intégration}
\sousChapitre{Intégrale de Riemann dépendant d'un paramètre}

\contenu{
\texte{
Soit $\alpha\in{]0,\frac\pi2[}$ et $\lambda\in\R$.
Chercher un équivalent pour $n\to\infty$ de
$I_n =  \int_{x=0}^{\alpha}\sin(x)\exp(\lambda n\sin^2(x))\,d x$.
}
\reponse{
Pour $\lambda\ne 0$~:
$I_n = \Bigl[\frac{\exp(\lambda n\sin^2(x))}{2\lambda n\cos(x)}\Bigr]_{x=0}^\alpha
- \int_{x=0}^{\alpha}\frac{\sin(x)}{2\lambda n\cos^2(x)}\exp(\lambda n\sin^2(x))\,d x
= \frac{\exp(\lambda n\sin^2(\alpha))}{2\lambda n\cos(\alpha)} - \frac1{2\lambda n}
- \frac{J_n}{2\lambda n}$
avec $0\le J_n\le \frac{I_n}{\cos^2(\alpha)}$.
Donc $I_n\sim \frac{\exp(\lambda n\sin^2(\alpha))}{2\lambda n\cos(\alpha)}$ si $\lambda > 0$
et $I_n\sim - \frac1{2\lambda n}$ si $\lambda < 0$.
}
}

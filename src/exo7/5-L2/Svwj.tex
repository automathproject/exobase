\uuid{Svwj}
\exo7id{4586}
\titre{exo7 4586}
\auteur{quercia}
\organisation{exo7}
\datecreate{2010-03-14}
\isIndication{false}
\isCorrection{true}
\chapitre{Série entière}
\sousChapitre{Calcul de la somme d'une série entière}
\module{Analyse}
\niveau{L2}
\difficulte{}

\contenu{
\texte{
Rayon et somme de $\sum P(n)x^n$ où $P$ est un polynôme de degré $p$.
}
\reponse{
$R=1$. On décompose $P$ sous la forme :
	     $P = a_0 + a_1(X+1) + a_2(X+1)(X+2) + \dots + a_p(X+1)\dots(X+p)$.\par
	     Alors $\sum_{n=0}^\infty P(n)x^n = \frac{a_0}{1-x} +
	     \frac{a_1}{(1-x)^2} + \dots + \frac{p!\,a_p}{(1-x)^{p+1}}$.
}
}

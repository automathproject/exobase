\uuid{wvbu}
\exo7id{2630}
\titre{exo7 2630}
\auteur{debievre}
\organisation{exo7}
\datecreate{2009-05-19}
\isIndication{true}
\isCorrection{true}
\chapitre{Fonction de plusieurs variables}
\sousChapitre{Différentiabilité}
\module{Analyse}
\niveau{L2}
\difficulte{}

\contenu{
\texte{
Trouver les points sur 
le parabolo\"\i de $z=4x^2 +y^2$ o\`u le plan tangent est parall\`ele au plan
$x+2y+z=6$. M\^eme question avec le plan $3x+5y-2z=3$.
}
\indication{Le plan tangent \`a la surface d'\'equation 
$z=f(x,y)$ au point $(x_0,y_0,z_0)$ 
est donn\'e par l'\'equation
\begin{equation}
z-z_0=\frac{\partial f}{\partial x}(x_0,y_0) (x-x_0) 
+\frac{\partial f}{\partial y}(x_0,y_0)
(y-y_0) .
\label{tang3}
\end{equation}}
\reponse{
Suivant l'indication,
le plan tangent \`a la surface d'\'equation 
$z=4x^2 +y^2$ au point $(x_0,y_0,z_0)$ 
est donn\'e par l'\'equation
\begin{align*}
z&=z_0 +8x_0 (x-x_0) + 2y_0 (y-y_0) 
\\
&= 8x_0x +2y_0 y +z_0 - 8x^2_0 -  2y^2_0= 
8x_0x +2y_0 y  - z_0 
\end{align*}
d'o\`u par
\begin{equation}
z-8x_0x -2y_0 y  = z_0.
\label{tang2}
\end{equation}
Pour que
ce plan soit parall\`ele au plan d'\'equation $x+2y+z=6$ il faut et il 
suffit que
$(1,2)=(-8x_0, -2y_0)$ d'o\`u que $x_0=-1/8$ et $y_0=-1$.
Par cons\'equent, 
le point cherch\'e sur le parabolo\"\i de $z=4x^2 +y^2$
est le point
$(-1/8,-1,17/16)$.
De m\^eme, 
pour que
le plan \eqref{tang2} 
soit parall\`ele au plan d'\'equation $3x+5y-2z=3$ il faut et il 
suffit que
$(3/2,5/2)=(8x_0, 2y_0)$ d'o\`u que $x_0=3/16$ et $y_0=5/4$,
et le point cherch\'e sur le parabolo\"\i de $z=4x^2 +y^2$
est alors le point
$(3/16,5/4,9/64+25/16)$ =$(3/16,5/4,109/64)$.
}
}

\uuid{sNVd}
\exo7id{5879}
\auteur{rouget}
\organisation{exo7}
\datecreate{2010-10-16}
\isIndication{false}
\isCorrection{true}
\chapitre{Equation différentielle}
\sousChapitre{Equations différentielles linéaires}

\contenu{
\texte{
Résoudre les systèmes suivants :
}
\begin{enumerate}
    \item \question{$\left\{
\begin{array}{l}
x'=4x-2y\\
y'=x+y
\end{array}
\right.$\quad}
\reponse{Posons $A=\left(
\begin{array}{cc}
4&-2\\
1&1
\end{array}
\right)$.

$\chi_A=\lambda^2-5\lambda+6=(\lambda-2)(\lambda-3)$ puis $A=PDP^{-1}$ où $D=\text{diag}(2,3)$ et $P=\left(
\begin{array}{cc}
1&2\\
1&1
\end{array}
\right)$.

Posons $X=\left(
\begin{array}{c}
x\\
y
\end{array}
\right)$ puis $X_1=P^{-1}X=\left(
\begin{array}{c}
x_1\\
y_1
\end{array}
\right)$.

\begin{align*}\ensuremath
\left\{
\begin{array}{l}
x'=4x-2y\\
y'=x+y
\end{array}
\right.&\Leftrightarrow X'=AX\Leftrightarrow X'=PDP^{-1}X\Leftrightarrow P^{-1}X'=DP^{-1}X\Leftrightarrow(P^{-1}X)'=D(P^{-1}X)\Leftrightarrow X_1'=DX_1\\
 &\Leftrightarrow\left\{
\begin{array}{l}
x_1'=2x_1\\
y_1'=3y_1
\end{array}
\right.\Leftrightarrow\exists(a,b)\in\Rr/\;\forall t\in\Rr,\;\left\{
\begin{array}{l}
x_1(t)=ae^{2t}\\
y_1(t)=be^{3t}
\end{array}
\right.\\
&\Leftrightarrow\exists(a,b)\in\Rr/\;\forall t\in\Rr,\;\left(
\begin{array}{l}
x(t)\\
y(t)
\end{array}
\right)=\left(
\begin{array}{cc}
1&2\\
1&1
\end{array}
\right)\left(
\begin{array}{l}
ae^{2t}\\
be^{3t}
\end{array}
\right)\\
 &\Leftrightarrow\exists(a,b)\in\Rr/\;\forall t\in\Rr,\;\left(
\begin{array}{l}
x(t)\\
y(t)
\end{array}
\right)=\left(
\begin{array}{l}
ae^{2t}+2be^{3t}\\
ae^{2t}+be^{3t}
\end{array}
\right)\\
\end{align*}

\begin{center}
\shadowbox{
$\mathcal{S}=\left\{t\mapsto \left(
\begin{array}{l}
ae^{2t}+2be^{3t}\\
ae^{2t}+be^{3t}
\end{array}
\right),\;(a,b)\in\Rr^2\right\}$.
}
\end{center}}
    \item \question{$\left\{
\begin{array}{l}
x'=x-y+ \frac{1}{\cos t}\\
y'=2x-y
\end{array}
\right.$  sur $\left]- \frac{\pi}{2}, \frac{\pi}{2}\right[$\quad}
\reponse{Puisque la fonction $t\mapsto\left(
\begin{array}{c}
 \frac{1}{\cos t}\\
\rule{0mm}{5mm}0
\end{array}
\right)$ est continue sur $\left]- \frac{\pi}{2}, \frac{\pi}{2}\right[$, les solutions réelles sur $\left]- \frac{\pi}{2}, \frac{\pi}{2}\right[$ du système proposé constituent un $\Rr$-espace affine de dimension $2$.

\textbf{Résolution du système homogène associé.} Posons $A=\left(
\begin{array}{cc}
1&-1\\
2&-1
\end{array}
\right)$. $\chi_A=\lambda^2+1=(\lambda-i)(\lambda+i)$ et en particulier $A$ est diagonalisable dans $\Cc$. Un vecteur propre de $A$ associé à la valeur propre $i$ est $\left(
\begin{array}{c}
1\\
1-i
\end{array}
\right)$ et un vecteur propre de $A$ associé à la valeur propre $-i$ est $\left(
\begin{array}{c}
1\\
1+i
\end{array}
\right)$. On sait alors que les solutions complexes sur $\Rr$ du système homogène associé sont les fonctions de la forme $X~:~t\mapsto ae^{it}\left(
\begin{array}{c}
1\\
1-i
\end{array}
\right)+be^{-it}\left(
\begin{array}{c}
1\\
1+i
\end{array}
\right)$, $(a,b)\in\Cc^2$.

Déterminons alors les solutions réelles du système homogène.

\begin{align*}\ensuremath
X\;\text{réelle}&\Leftrightarrow\forall t\in\Rr,\;ae^{it}\left(
\begin{array}{c}
1\\
1-i
\end{array}
\right)+be^{-it}\left(
\begin{array}{c}
1\\
1+i
\end{array}
\right)=\overline{a}e^{-it}\left(
\begin{array}{c}
1\\
1+i
\end{array}
\right)+\overline{b}e^{it}\left(
\begin{array}{c}
1\\
1-i
\end{array}
\right)\\
 &\Leftrightarrow b=\overline{a}\;(\text{car la famille de fonctions}\;(e^{it},e^{-it})\;\text{est libre}.)
\end{align*}

Les solutions réelles sur $\Rr$ du système homogène sont les fonctions de la forme $X~:~t\mapsto ae^{it}\left(
\begin{array}{c}
1\\
1-i
\end{array}
\right)+\overline{a}e^{-it}\left(
\begin{array}{c}
1\\
1+i
\end{array}
\right)=2\text{Re}\left(ae^{it}\left(
\begin{array}{c}
1\\
1-i
\end{array}
\right)\right)$, $a\in\Cc$. En posant $a=\lambda+i\mu$, $(\lambda,\mu)\in\Rr^2$, 

\begin{center}
$2\text{Re}\left(ae^{it}\left(
\begin{array}{c}
1\\
1-i
\end{array}
\right)\right)=2\text{Re}\left(\left(
\begin{array}{c}
(\lambda+i\mu)(\cos t+i\sin t)\\
(\lambda+i\mu)(1-i)(\cos t+i\sin t)
\end{array}
\right)\right)=2\left(
\begin{array}{c}
\lambda\cos t-\mu\sin t\\
\lambda(\cos t+\sin t)+\mu(\cos t-\sin t)
\end{array}
\right)$.
\end{center}

Maintenant, le couple $(\lambda,\mu)$ décrit $\Rr^2$ si et seulement si le couple $(2\lambda,2\mu)$ décrit $\Rr^2$ et en renommant les constantes $\lambda$ et $\mu$, on obtient les solutions réelles du système homogène : $t\mapsto\lambda\left(\begin{array}{c}
\cos t\\
\cos t+\sin t
\end{array}
\right)+\mu\left(\begin{array}{c}
-\sin t\\
\cos t-\sin t
\end{array}
\right)$, $(\lambda,\mu)\in\Rr^2$.

\textbf{Résolution du système.} D'après la méthode de variation de la constante, il existe une solution particulière du système de la forme $t\mapsto\lambda(t)\left(
\begin{array}{c}
\cos t\\
\cos t+\sin t
\end{array}
\right)+\mu(t)
\left(
\begin{array}{c}
-\sin t\\
\cos t-\sin t
\end{array}
\right)$ où $\lambda$ et $\mu$ sont deux fonctions dérivables sur $\left]- \frac{\pi}{2}, \frac{\pi}{2}\right[$ telles que pour tout réel $t$ de $\left]- \frac{\pi}{2}, \frac{\pi}{2}\right[$, $\lambda'(t)\left(
\begin{array}{c}
\cos t\\
\cos t+\sin t
\end{array}
\right)+\mu'(t)
\left(
\begin{array}{c}
-\sin t\\
\cos t-\sin t
\end{array}
\right)=\left(
\begin{array}{c}
 \frac{1}{\cos t}\\
\rule{0mm}{5mm}0
\end{array}
\right)$. Les formules de \textsc{Cramer} fournissent $\lambda'(t)= \frac{1}{\cos t}(\cos t-\sin t)=1- \frac{\sin t}{\cos t}$ et $\mu'(t)=- \frac{1}{\cos t}(\cos t+\sin t)=-1- \frac{\sin t}{\cos t}$. On peut prendre $\lambda(t)=t+\ln(\cos t)$ et $\mu(t)=-t+\ln(\cos t)$ et on obtient la solution particulière 
\begin{center}
$X(t)=(t+\ln(\cos t))\left(
\begin{array}{c}
\cos t\\
\cos t+\sin t
\end{array}
\right)+(-t+\ln(\cos t))\left(
\begin{array}{c}
-\sin t\\
\cos t-\sin t
\end{array}
\right)=\left(
\begin{array}{c}
t(\cos t+\sin t)+\ln(\cos t)(\cos t-\sin t)\\
2t\sin t+2\cos t\ln(\cos t)
\end{array}.
\right)$.
\end{center}

\begin{center}
\shadowbox{
$\mathcal{S}_{\left]- \frac{\pi}{2}, \frac{\pi}{2}\right[}=\left\{t\mapsto \left(
\begin{array}{l}
t(\cos t+\sin t)+\ln(\cos t)(\cos t-\sin t)+\lambda\cos t-\mu\sin t\\
2t\sin t+2\cos t\ln(\cos t)+\lambda(\cos t+\sin t)+\mu(\cos t-\sin t)
\end{array}
\right),\;(\lambda,\mu)\in\Rr^2\right\}$.
}
\end{center}}
    \item \question{$\left\{
\begin{array}{l}
x'=5x-2y+e^t\\
y'=-x+6y+t
\end{array}
\right.$}
\reponse{Puisque la fonction $t\mapsto\left(
\begin{array}{c}
e^t\\
t
\end{array}
\right)$ est continue sur $\Rr$, les solutions sur $\Rr$ du système proposé constituent un $\Rr$-espace affine de dimension $2$.

\textbf{Résolution du système homogène associé.} Posons $A=\left(
\begin{array}{cc}
5&-2\\
-1&6
\end{array}
\right)$. $\chi_A=\lambda^2-11\lambda+28=(\lambda-4)(\lambda-7)$. Un vecteur propre de ${^t}A$ associé à la valeur propre $4$ est $\left(
\begin{array}{c}
1\\
1
\end{array}
\right)$ et un vecteur propre de ${^t}A$ associé à la valeur propre $7$ est $\left(
\begin{array}{c}
1\\
-2
\end{array}
\right)$. Ces vecteurs fournissent des combinaisons linéaires intéressantes des équations :

\begin{align*}\ensuremath
\left\{
\begin{array}{l}
x'=5x-2y+e^t\\
y'=-x+6y+t
\end{array}
\right.&\Leftrightarrow\left\{
\begin{array}{l}
(x+y)'=4(x+y)+e^t+t\\
(x-2y)'=7(x-2y)+e^t-2t
\end{array}
\right.\\
 &\Leftrightarrow\exists(\lambda,\mu)\in\Rr^2/\;\forall t\in\Rr,\;\left\{
\begin{array}{l}
x(t)+y(t)=- \frac{e^t}{3}- \frac{t}{4}- \frac{1}{16}+\lambda e^{4t}\\
x(t)-2y(t)=- \frac{e^t}{6}+ \frac{2t}{7}+ \frac{2}{49}+\mu e^{7t}
\end{array}
\right.\\
 &\Leftrightarrow\exists(\lambda,\mu)\in\Rr^2/\;\forall t\in\Rr,\;\left\{
\begin{array}{l}
x(t)=- \frac{5e^t}{6}- \frac{3t}{14}- \frac{33}{392}++2\lambda e^{4t}+\mu e^{4t}\\
y(t)=- \frac{e^t}{6} \frac{15t}{28}- \frac{81}{784}+\lambda e^{4t}-\mu e^{7t}
\end{array}
\right.
\end{align*}}
    \item \question{$\left\{
\begin{array}{l}
x'=5x+y-z\\
y'=2x+4y-2z\\
z'=x-y+z
\end{array}
\right.$\quad}
\reponse{Posons $A=\left(
\begin{array}{ccc}  
5&1&-1\\
2&4&-2\\
1&-1&1
\end{array}
\right)$.

\begin{align*}\ensuremath
\chi_A&=\left|
\begin{array}{ccc}  
5-\lambda&1&-1\\
2&4-\lambda&-2\\
1&-1&1-\lambda
\end{array}
\right|=(5-\lambda)(\lambda^2-5\lambda+2)-2(-\lambda)+(-\lambda+2)=-\lambda^3+10\lambda^2-26\lambda+12\\
 &=-(\lambda-6)(\lambda^2-4\lambda+2)=-(\lambda-6)(\lambda-2+\sqrt{2})(\lambda-2-\sqrt{2}),
\end{align*}

et en particulier $A$ est diagonalisable dans $\Rr$.

\begin{center}
$(x,y,z)\in\text{Ker}(A-6I)\Leftrightarrow\left\{
\begin{array}{l}
-x+y-z=0\\
2x-2y-2z=0\\
x-y-5z=0
\end{array}
\right.\Leftrightarrow z=0\;\text{et}\;x=y$.
\end{center}

$\text{Ker}(A-6I)$ est la droite vectorielle engendrée par le vecteur $(1,1,0)$.

\begin{align*}\ensuremath
(x,y,z)\in\text{Ker}(A-(2+\sqrt{2})I)&\Leftrightarrow\left\{
\begin{array}{l}
(3-\sqrt{2})x+y-z=0\\
2x+(2-\sqrt{2})y-2z=0\\
x-y-(1+\sqrt{2})z=0
\end{array}
\right.\Leftrightarrow\left\{
\begin{array}{l}
z=(3-\sqrt{2})x+y\\
2x+(2-\sqrt{2})y-2((3-\sqrt{2})x+y)=0\\
x-y-(1+\sqrt{2})((3-\sqrt{2})x+y)=0
\end{array}
\right.
\\
 &\Leftrightarrow\left\{
\begin{array}{l}
z=(3-\sqrt{2})x+y\\
(-4+2\sqrt{2})x-\sqrt{2}y=0\\
-2\sqrt{2}x-(2+\sqrt{2})y=0
\end{array}
\right.\Leftrightarrow\left\{
\begin{array}{l}
z=(3-\sqrt{2})x+y\\
y=(-2\sqrt{2}+2)x
\end{array}
\right.\\
 &\Leftrightarrow\left\{
\begin{array}{l}
y=(-2\sqrt{2}+2)x\\
z=(5-3\sqrt{2})x
\end{array}
\right..
\end{align*} $\text{Ker}(A-(2+\sqrt{2})I)$ est la droite vectorielle engendrée par le vecteur $(1,2-2\sqrt{2},5-3\sqrt{2})$. Un calcul conjugué montre alors que $\text{Ker}(A-(2-\sqrt{2})I)$ est la droite vectorielle engendrée par le vecteur $(1,2+2\sqrt{2},5+3\sqrt{2})$.

On sait alors que les solutions du système homogène $t\mapsto ae^{6t}\left(
\begin{array}{c}
1\\
1\\
0
\end{array}
\right)+be^{(2+\sqrt{2})t}\left(
\begin{array}{c}
1\\
2-2\sqrt{2}\\
5-3\sqrt{2})
\end{array}
\right)+ce^{(2-\sqrt{2})t}\left(
\begin{array}{c}
1\\
2+2\sqrt{2}\\
5+3\sqrt{2})
\end{array}
\right)$, $(a,b,c)\in\Rr^3$.}
    \item \question{$\left\{
\begin{array}{l}
x'=2x+y\\
y'=-x\\
z'=x+y+z
\end{array}
\right.$ (trouver la solution telle que $x(0)=0$, $y(0)=1$ et $z(0)=-1$).}
\reponse{Posons $A=\left(
\begin{array}{ccc}
2&1&0\\
-1&0&0\\
1&1&1
\end{array}
\right)$. $\chi_A=\left|
\begin{array}{ccc}
2-\lambda&1&0\\
-1&-\lambda&0\\
1&1&1-\lambda
\end{array}
\right|=(1-\lambda)(\lambda^2-2\lambda+1)=(1-\lambda)^3$. Le théorème de \textsc{Cayley}-\textsc{Hamilton} permet alors d'affirmer que $(A-I)^3=0$. 

On sait que les solutions du système $X'=AX$ sont les fonctions de la forme $t\mapsto e^{tA}X_0$ où $X_0\in\mathcal{M}_{3,1}(\Rr)$. Or, pour $t\in\Rr$,

\begin{align*}\ensuremath
e^{tA}&=e^{t(A-I)}\times e^{tI}\;(\text{car les matrices}\;t(A-I)\;\text{et}\;tI\;\text{commutent})\\
 &=\left(\sum_{n=0}^{+\infty} \frac{t^n}{n!}(A-I)^n\right)\times e^tI=e^t\left(\sum_{n=0}^{2} \frac{t^n}{n!}(A-I)^n\right)\\
 &=e^t\left(\left(
 \begin{array}{ccc}
 1&0&0\\
 0&1&0\\
 0&0&1
 \end{array}
 \right)+t\left(
 \begin{array}{ccc}
 1&1&0\\
 -1&-1&0\\
 1&1&0
 \end{array}
 \right)+ \frac{t^2}{2}\left(
 \begin{array}{ccc}
 1&1&0\\
 -1&-1&0\\
 1&1&0
 \end{array}
 \right)\left(
 \begin{array}{ccc}
 1&1&0\\
 -1&-1&0\\
 1&1&0
 \end{array}
 \right)\right)\\
 &=e^t\left(
 \begin{array}{ccc}
 1+t&t&0\\
 -t&1-t&0\\
 t&t&1
 \end{array}
 \right).
\end{align*}

Les solutions du système sont les fonctions de la forme $t\mapsto e^{tA}X_0=e^t\left(
 \begin{array}{ccc}
 1+t&t&0\\
 -t&1-t&0\\
 t&t&1
 \end{array}
 \right)\left(
 \begin{array}{c}
 a\\
 b\\
 c
 \end{array}
 \right)=\left(
 \begin{array}{c}
 (a+(a+b)t)e^t\\
 (b-(a+b)t)e^t\\
 ((a+b)t+c)e^t
 \end{array}
 \right)$, $(a,b,c)\in\Rr^3$. Maintenant,
 
\begin{center}
$\left(
\begin{array}{c}
x(0)\\
y(0)\\
z(0)
\end{array}
\right)=\left(
\begin{array}{c}
0\\
1\\
-1
\end{array}
\right)\Leftrightarrow\left\{
\begin{array}{l}
a=0\\
b=1\\
c=-1
\end{array}
\right.$.
\end{center}

La solution cherchée est $t\mapsto\left(
 \begin{array}{c}
 te^t\\
 (1-t)e^t\\
 (t-1)e^t
 \end{array}
 \right)$.}
\end{enumerate}
}

\uuid{2of8}
\exo7id{5555}
\auteur{rouget}
\organisation{exo7}
\datecreate{2010-07-15}
\isIndication{false}
\isCorrection{true}
\chapitre{Fonction de plusieurs variables}
\sousChapitre{Différentiabilité}

\contenu{
\texte{
Déterminer la classe de $f$ sur $\Rr^2$ où $f(x,y)=\left\{\begin{array}{l}
0\;\text{si}\;(x,y)=(0,0)\\
\frac{xy(x^2-y^2)}{x^2+y^2}\;\text{si}\;(x,y)\neq(0,0)
\end{array}
\right.$.
}
\reponse{
\textbullet~Pour $(x,y)\in\Rr^2$, $x^2+y^2=0\Leftrightarrow x=y=0$ et donc $f$ est définie sur $\Rr^2$.
\textbullet~$f$ est de classe $C^\infty$ sur $\Rr^2\setminus\{(0,0)\}$ en tant que quotient de fonctions de classe $C^\infty$ sur $\Rr^2\setminus\{(0,0)\}$ dont le dénominateur ne s'annule pas sur $\Rr^2\setminus\{(0,0)\}$.
 

\textbullet~Pour $(x,y)\neq(0,0)$, $|f(x,y)|\leqslant\frac{|xy|(x^2+y^2)}{x^2+y^2}=|xy|$. Comme $\lim_{(x,y)\rightarrow (0,0)}|xy|=0$, on en déduit que $\displaystyle\lim_{\substack{(x,y)\rightarrow(0,0)\\(x,y)\neq(0,0)}}f(x,y)=0=f(0,0)$. Ainsi, $f$ est continue en $(0,0)$ et donc sur $\Rr^2$.
\textbullet~\textbf{Existence de $\frac{\partial f}{\partial x}(0,0)$.} Pour $x\neq0$,

\begin{center}
$\frac{f(x,0)-f(0,0)}{x-0}=\frac{x\times0\times(x^2-0^2)}{x\times(x^2+0^2)}=0$,
\end{center}
et donc $\lim_{x\rightarrow 0}\frac{f(x,0)-f(0,0)}{x-0}=0$. Ainsi, $f$ admet une dérivée partielle par rapport à sa première variable en $(0,0)$ et $\frac{\partial f}{\partial x}(0,0)=0$.
\textbullet~Pour $(x,y)\neq(0,0)$, $\frac{\partial f}{\partial x}(x,y)=y\frac{(3x^2-y^2)(x^2+y^2)-(x^3-y^2x)(2x)}{(x^2+y^2)^2}=\frac{y(x^4+4x^2y^2-y^4)}{(x^2+y^2)^2}$.

Finalement, $f$ admet sur $\Rr^2$ une dérivée partielle par rapport à sa première variable définie par 

\begin{center}
$\forall(x,y)\in\Rr^2$, $\frac{\partial f}{\partial x}(x,y)=\left\{
\begin{array}{l}
\rule[-4mm]{0mm}{0mm}0\;\text{si}\;(x,y)=(0,0)\\
\frac{y(x^4+4x^2y^2-y^4)}{(x^2+y^2)^2}\;\text{si}\;(x,y)\neq(0,0)
\end{array}
\right.$.
\end{center}
\textbullet~Pour $(x,y)\in\Rr^2$, $f(y,x)=-f(x,y)$. Par suite, $\forall(x,y)\in\Rr^2$, $\frac{\partial f}{\partial y}(x,y)=-\frac{\partial f}{\partial x}(y,x)$.
En effet, pour $(x_0,y_0)$ donné dans $\Rr^2$

\begin{center}
$\frac{f(x_0,y)-f(x_0,y_0)}{y-y_0}=\frac{-f(y,x_0)+f(y_0,x_0)}{y-y_0}=-\frac{f(y,x_0)-f(y_0,x_0)}{y-y_0}\underset{y\rightarrow y_0}{\rightarrow}-\frac{\partial f}{\partial x}(y_0,x_0)$.
\end{center}
Donc, $f$ admet sur $\Rr^2$ une dérivée partielle par rapport à sa deuxième variable définie par 

\begin{center}
$\forall(x,y)\in\Rr^2$, $\frac{\partial f}{\partial y}(x,y)=-\frac{\partial f}{\partial x}(y,x)=\left\{
\begin{array}{l}
\rule[-4mm]{0mm}{0mm}0\;\text{si}\;(x,y)=(0,0)\\
\frac{x(x^4-4x^2y^2-y^4)}{(x^2+y^2)^2}\;\text{si}\;(x,y)\neq(0,0)
\end{array}
\right.$.
\end{center}
\textbullet~\textbf{Continuité de $\frac{\partial f}{\partial x}$ et $\frac{\partial f}{\partial y}$ en $(0,0)$.} Pour $(x,y)\neq(0,0)$,

\begin{center}
$\left|\frac{\partial f}{\partial x}(x,y)-\frac{\partial f}{\partial x}(0,0)\right|=\frac{|y(x^4+4x^2y^2-y^4)|}{(x^2+y^2)^2}\leqslant\frac{|y|(x^4+4x^2y^2+y^4)}{(x^2+y^2)^2}\leqslant\frac{|y|(2x^4+4x^2y^2+2y^4)}{(x^2+y^2)^2}=2|y|$.
\end{center}
Comme $2|y|$ tend vers $0$ quand $(x,y)$ tend vers $(0,0)$, $\left|\frac{\partial f}{\partial x}(x,y)-\frac{\partial f}{\partial x}(0,0)\right|$ tend vers $0$ quand $(x,y)$ tend vers $(0,0)$. On en déduit que l'application $\frac{\partial f}{\partial x}$ est continue en $(0,0)$ et donc sur $\Rr^2$.
Enfin, puisque $\forall(x,y)\in\Rr^2$, $\frac{\partial f}{\partial y}(x,y)=-\frac{\partial f}{\partial x}(y,x)$, $\frac{\partial f}{\partial y}$ est continue sur $\Rr^2$. $f$ est donc au moins de classe $C^1$ sur $\Rr^2$.
\textbullet~Pour $x\neq0$, $\frac{\frac{\partial f}{\partial y}(x,0)-\frac{\partial f}{\partial y}(0,0)}{x-0}=\frac{x^4}{x^4}=1$ et donc $\lim_{x\rightarrow 0}\frac{\frac{\partial f}{\partial y}(x,0)-\frac{\partial f}{\partial y}(0,0)}{x-0}=1$. Donc $\frac{\partial^2f}{\partial y\partial x}(0,0)$ existe et $\frac{\partial^2f}{\partial y\partial x}(0,0)=1$.
Pour $y\neq0$, $\frac{\frac{\partial f}{\partial x}(y,0)-\frac{\partial f}{\partial x}(0,0)}{y-0}=-\frac{y^4}{y^4}=-1$ et donc $\lim_{y\rightarrow 0}\frac{\frac{\partial f}{\partial x}(y,0)-\frac{\partial f}{\partial x}(0,0)}{y-0}=-1$. Donc $\frac{\partial^2f}{\partial x\partial y}(0,0)$ existe et $\frac{\partial^2f}{\partial x\partial y}(0,0)=-1$.
$\frac{\partial^2f}{\partial y\partial x}(0,0)\neq\frac{\partial^2f}{\partial x\partial y}(0,0)$ et donc $f$ n'est pas de classe $C^2$ sur $\Rr^2$ d'après le théorème de \textsc{Schwarz}.

\begin{center}
\shadowbox{
$f$ est de classe $C^1$ exactement sur $\Rr^2$.
}
\end{center}
}
}

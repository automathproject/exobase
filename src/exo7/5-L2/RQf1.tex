\uuid{RQf1}
\exo7id{4855}
\titre{exo7 4855}
\auteur{quercia}
\organisation{exo7}
\datecreate{2010-03-16}
\isIndication{false}
\isCorrection{false}
\chapitre{Topologie}
\sousChapitre{Fonctions vectorielles}
\module{Analyse}
\niveau{L2}
\difficulte{}

\contenu{
\texte{
Soit ${\vec f} : {I\subset \R} \to {\R^3}$ une fonction de classe $\mathcal{C}^1$
telle que :
$$\forall\ t\in I,\ \vec f(t) \ne \vec 0 \text{ et la famille }
  (\vec f(t),\vec f\,'(t)) \text{ est li{\'e}e}.$$
On pose $\vec g(t) = \frac{\vec f(t)}{\|\vec f(t)\|}$.
}
\begin{enumerate}
    \item \question{Montrer que $g$ est de classe $\mathcal{C}^1$ et que $\vec g\,'(t)$ est {\`a} la fois orthogonal
    et colin{\'e}aire {\`a} $\vec g(t)$.}
    \item \question{En d{\'e}duire que $\vec f(t)$ garde une direction constante.}
    \item \question{Chercher un contre-exemple lorsqu'on retire la propri{\'e}t{\'e} :
    $\forall\ t\in I,\ \vec f(t) \ne \vec 0$.}
\end{enumerate}
}

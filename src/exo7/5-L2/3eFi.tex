\uuid{3eFi}
\exo7id{4559}
\titre{exo7 4559}
\auteur{quercia}
\organisation{exo7}
\datecreate{2010-03-14}
\isIndication{false}
\isCorrection{true}
\chapitre{Suite et série de fonctions}
\sousChapitre{Autre}
\module{Analyse}
\niveau{L2}
\difficulte{}

\contenu{
\texte{
Déterminer un équivalent au voisinage de $0$ de
$S_{1}(x) = \sum_{n=1}^{\infty}{\frac{1}{\sh{(nx)}}}$ et
$S_{2}(x) = \sum_{n=1}^{\infty}{\frac{1}{\sh^{2}{(nx)}}}$.
}
\reponse{
On a
$ \int_{t=x}^{+\infty}\frac{d t}{\sh t} \le xS_1(x)
\le \frac x{\sh x} +  \int_{t=x}^{+\infty}\frac{d t}{\sh t}$
et $\frac1{\sh t}=\frac1t + O(t)$ donc
$ \int_{t=x}^{+\infty}\frac{d t}{\sh t}=-\ln(x)+ O(1)$.
On en déduit $S_1(x)\sim-\frac{\ln x}x$.

La même méthode ne marche pas pour $S_2$ car le terme résiduel,
$\frac x{\sh^2(x)}$ n'est pas négligeable devant
$ \int_{t=x}^{+\infty}\frac{d t}{\sh^2(t)}$.
Par contre, on peut remarquer que la série
$\sum_{n=1}^{\infty}{\frac{x^2}{\sh^{2}{(nx)}}}$ est normalement
convergente sur~$\R$, d'où
$S_2(x)\sim\frac{\zeta(2)}{x^2}$.
}
}

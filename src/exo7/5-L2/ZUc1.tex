\uuid{ZUc1}
\exo7id{2640}
\titre{exo7 2640}
\auteur{debievre}
\organisation{exo7}
\datecreate{2009-05-19}
\isIndication{true}
\isCorrection{true}
\chapitre{Fonction de plusieurs variables}
\sousChapitre{Différentielle seconde}
\module{Analyse}
\niveau{L2}
\difficulte{}

\contenu{
\texte{
Les variables \'etant not\'ees $x$ et $t$,
trouver la solution g\'en\'erale 
$f\colon\R^2\to\R$
de ``l'\'equation des ondes'', \`a savoir
\begin{equation}\label{eq:ondes}
\frac{\partial^2 f}{\partial x^2}-\frac{\partial^2 f}{\partial t^2}=0.
\end{equation}
Trouver ensuite la solution unique de l'\'equation des ondes qui
satisfait aux conditions initiales
\begin{equation}
f(x,0)=\sin x, \ \frac{\partial f}{\partial t}(x,0)=-\cos x.
\label{in}
\end{equation}
}
\indication{\begin{enumerate}
\item Grace au changement de variables
\[
\R^2 \longrightarrow \R^2,
\ (u,v) \longmapsto (x,y)=\left(\frac{u-v}{2}, \frac{u+v}{2}\right),
\]
la fonction $f$ s'\'ecrit 
$F(u,v)=f(\frac{u-v}{2}, \frac{u+v}{2})$.
Montrer que pour que $f$ soit solution de (\ref{eq:ondes}) 
il faut et il suffit que
\begin{equation}\label{eq:ondesbis}
\frac{\partial^2F}{\partial u\partial v}=0 .
\end{equation}
 \item  Montrer que, si $F$ satisfait \`a (\ref{eq:ondesbis}), il existe deux fonctions $g_1,g_2\colon\R\to\R$ telles que
\[
F(u,v)=g_1(u)+g_2(v).
\]
 \item  \'Ecrire la solution g\'en\'erale de (\ref{eq:ondes}) et expliquer la phrase: ``En une dimension d'espace, toute solution de l'\'equation des ondes s'\'ecrit comme somme d'une onde qui se d\'eplace vers la droite et une qui se d\'eplace vers la gauche.''
\end{enumerate}}
\reponse{
Avec $\frac{\partial }{\partial u}
=1/2(\frac{\partial}{\partial x}+\frac{\partial}{\partial t})$ et
$\frac{\partial }{\partial v}
=1/2(-\frac{\partial}{\partial x}+\frac{\partial}{\partial t})$
nous obtenons les identit\'es
\begin{align*}
\frac{\partial F}{\partial u}&= 
\frac 12\frac{\partial f}{\partial x}+\frac 12\frac{\partial f}{\partial t}
\\
\frac{\partial F}{\partial v}&= 
-\frac 12\frac{\partial f}{\partial x}+\frac 12\frac{\partial f}{\partial t}
\\
\frac{\partial^2 F}{\partial u \partial v}&= 
-\frac 14\frac{\partial^2 f}{\partial x^2}
+\frac 14\frac{\partial^2 f}{\partial t^2}
\end{align*}
d'o\`u pour que $f$ satisfasse \`a l'\'equation \eqref{eq:ondes} il faut et il suffit que $F$ satisfasse \`a l'\'equation
 \eqref{eq:ondesbis}.
Supposons que $F$ satisfasse \`a l'\'equation \eqref{eq:ondesbis}.
Alors la fonction $\frac{\partial F}{\partial u}$ est une fonction
disons $h_1$ seulement de la variable $u$
et la fonction $\frac{\partial F}{\partial v}$ est une fonction
disons $h_2$ seulement de la variable $v$.
Par cons\'equent,
$F(u,v)= g_1(u)+g_2(v)$ o\`u $g'_1=h_1$ et $g'_2=h_2$.
La solution g\'en\'erale de
\eqref{eq:ondes} s'\'ecrit alors
\[
f(x,t)=g_1(u)+g_2(v)=g_1(x+t)+g_2(t-x) .
\]
La fonction $g_1$ d\'ecrit
une onde qui se d\'eplace vers la droite et 
la fonction $g_1$ d\'ecrit
une onde qui se d\'eplace vers la gauche.

Enfin, pour trouver la solution unique
satisfaisant aux condition initiales \eqref{in} nous constatons que
les conditions initiales entra\^ \i nent les identit\'es
\begin{align*}
f(x,0)&=g_1(x)+g_2(-x)=\sin x
\\
\frac {\partial f}{\partial x}(x,0)&=g'_1(x)-g'_2(-x)=\cos x
\\
\frac {\partial f}{\partial t}(x,0)&=g'_1(x)+g'_2(-x)=-\cos x
\end{align*}
d'o\`u $g'_1=0$ et $g'_2(-x)=-\cos x$, c.a.d. 
$g_2(x)=\sin (-x)$. Par cons\'equent,
la solution unique cherch\'ee $f$ s'\'ecrit
\[
f(x,t)= \sin(x-t).
\]
}
}

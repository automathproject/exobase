\uuid{Fau2}
\exo7id{5841}
\titre{exo7 5841}
\auteur{rouget}
\organisation{exo7}
\datecreate{2010-10-16}
\isIndication{false}
\isCorrection{true}
\chapitre{Topologie}
\sousChapitre{Topologie des espaces vectoriels normés}
\module{Analyse}
\niveau{L2}
\difficulte{}

\contenu{
\texte{
Soit $E = C^2([0,1],\Rr)$. Pour $f$ élément de $E$, on pose $N(f) =\int_{0}^{1}|f(t)|\;dt$, $N'(f) = |f(0)| +\int_{0}^{1}|f'(t)|\;dt$ et

$N''(f) =|f(0)|+|f'(0)|+\int_{0}^{1}|f''(t)|\;dt$. Montrer que $N$, $N'$ et $N''$ sont des normes et les comparer.
}
\reponse{
\textbullet~Il est connu que $N$ est une norme sur $E$.

\textbullet~Montrons que $N'$ est une norme sur $E$.

(1) $N'$ est une application de $E$ dans $\Rr^+$ car pour $f$ dans $E$, $f'$ est continue sur le segment $[0,1]$ et donc $f'$ est intégrable

sur le segment $[0,1]$.

(2) Soit $f\in E$. Si $N'(f) = 0$ alors  $f(0)=0\;\text{et}\;f'= 0$ (fonction continue positive d'intégrale nulle). Par suite, $f$ est un

polynôme de degré inférieur ou égal à $0$ tel que $f(0)=0$ et on en déduit que $f = 0$.

(3) $\forall f\in E$, $\forall \lambda\in\Rr$, $N'(\lambda f) =|\lambda f(0)|+\int_{0}^{1}|\lambda f'(t)|\;dt=|\lambda|\left(|f(0)|+\int_{0}^{1}|f'(t)|\;dt\right)=|\lambda| N'(f)$.

(4) Soit $(f,g)\in E^2$.

\begin{center}
$N'(f+g)\leqslant|f(0)|+|g(0)|+\int_{0}^{1}|f'(t)|\;dt +\int_{0}^{1}|g'(t)|dt= N'(f)+N'(g)$.
\end{center}

Donc $N'$ est une norme sur $E$.

\textbullet~Montrons que $N''$ est une norme sur $E$. On note que $\forall f\in E$, $N''(f)=|f(0)|+N'(f')$ et tout est immédiat.

\begin{center}
\shadowbox{
$N$, $N'$ et $N''$ sont des normes sur $E$.
}
\end{center}

\textbullet~Soit $f\in E$ et $t\in[0,1]$. Puisque la fonction $f'$ est continue sur $[0,1]$

\begin{center}
$|f(t)| = |f(0) +\int_{0}^{t}f'(u)\;du|\leqslant|f(0)|+\int_{0}^{t}|f'(u)|du\leqslant |f(0)|+\int_{0}^{1}|f'(u)|\;du =N'(f)$,
\end{center}

et donc $N(f) =\int_{0}^{1}|f(t)|\;dt\leqslant\int_{0}^{1}N'(f)\;dt=N'(f)$.

Ensuite en appliquant le résultat précédent à $f'$,  on obtient 

\begin{center}
$N'(f)=|f(0)|+N(f')\leqslant|f(0)|+N'(f')=N''(f)$.
\end{center}

Finalement 

\begin{center}
\shadowbox{
$\forall f\in E$, $N(f)\leqslant N'(f)\leqslant N''(f)$.
}
\end{center}

Pour $n\in\Nn$ et $t\in[0,1]$, on pose $f_n(t) = t^n$. 

$N(f_n)=\int_{0}^{1}t^n\;dt= \frac{1}{n+1}$ et donc la suite $(f_n)_{n\in\Nn}$ tend vers $0$ dans l'espace vectoriel normé $(E,N)$.

Par contre, pour $n\geqslant1$, $N'(f_n)=n\int_{0}^{1}t^{n-1}\;dt=1$ et la suite $(f_n)_{n\in\Nn}$  ne tend pas vers $0$ dans l'espace vectoriel normé $(E,N')$. On en déduit que

\begin{center}
\shadowbox{
les normes $N$ et $N'$ ne sont pas des normes équivalentes.
}
\end{center}

De même en utilisant $f_n(t)= \frac{t^n}{n}$, on montre que les normes $N'$ et $N''$ ne sont pas équivalentes.
}
}

\uuid{Ph0R}
\exo7id{4541}
\titre{exo7 4541}
\auteur{quercia}
\organisation{exo7}
\datecreate{2010-03-14}
\isIndication{false}
\isCorrection{true}
\chapitre{Suite et série de fonctions}
\sousChapitre{Autre}
\module{Analyse}
\niveau{L2}
\difficulte{}

\contenu{
\texte{
\'Etudier la convergence de la suite de fonctions définies par :
$f_n(x) = \frac{n(n+1)}{x^{n+1}} \int_0^x (x-t)^{n-1}\sin t\,d t$.
}
\reponse{
Poser $t = xu$ puis intégrer deux fois par parties :
         $f_n(x) = 1 -  \int_{u=0}^1 (1-u)^{n+1}x\sin(xu)\,d u$
         donc $(f_n)$ converge simplement vers la fonction constante $1$, et
         la convergence est uniforme sur tout intervalle borné.
}
}

\uuid{Ab19}
\exo7id{4107}
\titre{exo7 4107}
\auteur{quercia}
\organisation{exo7}
\datecreate{2010-03-11}
\isIndication{false}
\isCorrection{true}
\chapitre{Equation différentielle}
\sousChapitre{Equations différentielles linéaires}
\module{Analyse}
\niveau{L2}
\difficulte{}

\contenu{
\texte{
Soit $A$ coninue de $\mathcal{M}_n(\R)$ dans lui-même. On suppose
que les $a_{ij}(t)$ restent positifs quand $t$ décrit $\R^+$, et l'on se donne un
vecteur $X_0$ dont toutes les composantes sont positives. Montrer
qu'en désignant par $X(t)$ la valeur en $t$ du système $Y'=AY$ valant
$X_0$ en $t=0$, on a pour tout $t\ge0$ et pour tout $i$ l'inégalité
$x_i(t)\ge 0$.
}
\reponse{
La suite $(X_k)$ de fonctions définie par $X_k(t) = X_0$,
$X_{k+1}(t) = X_0 +  \int_{u=0}^t A(u)X_k(u)\,d u$ converge localement
uniformément vers $X$ et $X_k(t)$ est clairement à composantes positives
pour $t\ge 0$.
}
}

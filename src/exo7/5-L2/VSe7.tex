\uuid{VSe7}
\exo7id{4188}
\auteur{quercia}
\organisation{exo7}
\datecreate{2010-03-11}
\isIndication{false}
\isCorrection{true}
\chapitre{Fonction de plusieurs variables}
\sousChapitre{Dérivée partielle}

\contenu{
\texte{
Soit $f:{\R^2}\to\R$ de classe $\mathcal{C}^2$ telle que 
$|f(x)|/\|x\| \to +\infty$ lorsque $\|x\|\to\infty$.
Prouver que $\nabla f$ est surjective sur $\R^2$.
}
\reponse{
Sinon il existe $a\in\R^2$ telle que $g$~: $x \mapsto f(x)-(a\mid x)$
n'a pas de point critique, donc pas de minimum ni de maximum.
On a aussi $|g(x)|/\|x\|\to +\infty$ lorsque $\|x\|\to\infty$ d'où $\sup(g) = +\infty$
et $\inf(g) = -\infty$.
Considérons pour $r> 0$ $E_r = \{x\in\R^2\text{ tq }\|x\|\ge r\}$~: $g(E_r)$ est une partie connexe
de~$\R$ donc un intervalle, et $\sup(g(E_r)) = \sup(g) = +\infty$,
$\inf(g(E_r)) = \inf(g) = -\infty$, d'où $g(E_r) = \R$. Ainsi il existe des $x$ de normes arbitrairement
grandes tels que $g(x) = 0$ en contradiction avec la
propriété $|g(x)|/\|x\| \to +\infty$ lorsque $\|x\|\to\infty$.

Remarque~: l'hypothèse $f$ de classe $\mathcal{C}^2$ est surabondante, la classe $\mathcal{C}^1$
suffit à conclure.
}
}

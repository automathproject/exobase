\uuid{92Xj}
\exo7id{1802}
\auteur{ridde}
\organisation{exo7}
\datecreate{1999-11-01}
\isIndication{false}
\isCorrection{false}
\chapitre{Fonction de plusieurs variables}
\sousChapitre{Différentiabilité}

\contenu{
\texte{
Soit $f : \begin{cases} \Rr^{2} \rightarrow \Rr \\ (x, y) \mapsto \frac{x^{5}}
{ (y-x^{2})^{2} + x^{6}} \text{ si } (x, y) \neq (0,0)\\
 (0, 0) \mapsto 0\\ \end{cases}$.\\
 Montrer que $f$ admet une d\'eriv\'ee en $ (0, 0)$ suivant tout vecteur mais
 n'admet pas de d\'eveloppement limit\'e \`a l'ordre 1 en $ (0, 0)$.
}
}

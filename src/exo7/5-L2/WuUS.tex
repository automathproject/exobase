\uuid{WuUS}
\exo7id{4546}
\titre{exo7 4546}
\auteur{quercia}
\organisation{exo7}
\datecreate{2010-03-14}
\isIndication{false}
\isCorrection{true}
\chapitre{Suite et série de fonctions}
\sousChapitre{Autre}
\module{Analyse}
\niveau{L2}
\difficulte{}

\contenu{
\texte{
Pour $x>1$ on pose $\zeta(x) = \sum_{n=1}^\infty \frac1{\strut n^x}$
et pour $x>0$~: $\eta(x) = \sum_{n=1}^\infty \frac{(-1)^{n-1}}{\strut n^x}$.
}
\begin{enumerate}
    \item \question{\'Etablir pour $x>1$~: $\eta(x) = (1-2^{1-x})\zeta(x)$.
    En déduire $\zeta(x) \sim \frac 1{x-1}$ pour $x\to1^+$.}
    \item \question{Montrer que $\zeta(x) = \frac 1{x-1} + \gamma +  o(1)$.
    On remarquera que $\frac 1{x-1} =  \int_{t=1}^{+\infty} \frac{d t}{\strut t^x}$.}
    \item \question{En déduire la valeur de $\sum_{n=1}^\infty \frac{(-1)^n\ln n}n$.}
\reponse{
$\zeta(x) - \frac 1{x-1} =
    \sum_{n=1}^\infty \left(\frac 1{\strut n^x} -  \int_{t=n}^{n+1} \frac{d t}{\strut t^x} \right)$.

    A $n$ fixé, $\frac 1{n^x} -  \int_{t=n}^{n+1} \frac{d t}{t^x}\to
                 \frac 1n -  \int_{t=n}^{n+1} \frac{d t}t$
   lorsque $x\to1^+$ et la convergence est monotone donc lorsque $x\to1^+$

    $\zeta(x) - \frac 1{x-1} \to
    \sum_{n=1}^\infty \left(\frac 1n -  \int_{t=n}^{n+1} \frac{d t}t\right) = \gamma$.
$\sum_{n=1}^\infty \frac{(-1)^n\ln n}n = \eta'(1) = \gamma\ln 2 - \frac12\ln(2)^2$.
}
\end{enumerate}
}

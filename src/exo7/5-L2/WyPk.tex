\uuid{WyPk}
\exo7id{4063}
\titre{exo7 4063}
\auteur{quercia}
\organisation{exo7}
\datecreate{2010-03-11}
\isIndication{false}
\isCorrection{true}
\chapitre{Equation différentielle}
\sousChapitre{Equations différentielles linéaires}
\module{Analyse}
\niveau{L2}
\difficulte{}

\contenu{
\texte{
Soit $E=\mathcal{C}(\R^+,\R)$, $b\in \R$ et $a>0$.
}
\begin{enumerate}
    \item \question{Montrer que, pour tout $f\in E$, il existe un unique $g$ de $\mathcal{C}^1(\R^+,\R)$ tel que 
    $\begin{cases}g'+ag=f\cr g(0)=b.\cr\end{cases}$}
\reponse{$g(x) = be^{-ax} +  \int_{t=0}^x e^{a(t-x)}f(t)\,d t$.}
    \item \question{Montrer que si $f$ est intégrable sur $\R^+$, $g$ l'est également. Relation entre
    $ \int_{t=0}^{+\infty}f(t)\,d t$ et $ \int_{t=0}^{+\infty}g(t)\,d t$.}
\reponse{$ \int_{x=0}^Xg(x)\,d x
    = \frac{b}{\strut a}(1-e^{-aX}) +  \int_{t=0}^X \int_{x=t}^Xe^{a(t-x)}f(t)\,d x\,d t
    = \frac{b}{\strut a}(1-e^{-aX}) +  \int_{t=0}^X\frac{1-e^{a(t-X)}}{\strut a}f(t)\,d t$

    $\phantom{ \int_{x=0}^Xg(x)\,d x}\to
     \frac{b}{\strut a}+ \frac{1}{\strut a} \int_{t=0}^{+\infty}f(t)\,d t$ lorsque $X\to+\infty$.

    Donc l'intégrale de~$g$ converge. On montre la convergence
    absolue par majoration élémentaire.}
\end{enumerate}
}

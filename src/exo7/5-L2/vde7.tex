\uuid{vde7}
\exo7id{4721}
\titre{exo7 4721}
\auteur{quercia}
\organisation{exo7}
\datecreate{2010-03-16}
\isIndication{false}
\isCorrection{true}
\chapitre{Topologie}
\sousChapitre{Topologie de la droite réelle}
\module{Analyse}
\niveau{L2}
\difficulte{}

\contenu{
\texte{
Soit $a > 1$. Montrer qu'il existe une suite r{\'e}elle born{\'e}e, $(x_n)$,
telle que : $\forall\ i\ne j,\ |x_i-x_j| \ge \frac 1{|i-j|^a}$.
}
\reponse{
On construit un ensemble de type Cantor dont les trous ont pour
         longueur $1$, $\frac 1{2^a}$, $\frac 1{4^a}$, \dots,
         et on r{\'e}partit les $x_k$ de part et d'autre des trous en fonction
         de l'{\'e}criture d{\'e}cimale de $k$ (0 $\rightarrow$ {\`a} gauche,
         1 $\rightarrow$ {\`a} droite).
}
}

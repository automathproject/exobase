\uuid{icAN}
\exo7id{4656}
\auteur{quercia}
\organisation{exo7}
\datecreate{2010-03-14}
\isIndication{false}
\isCorrection{true}
\chapitre{Série de Fourier}
\sousChapitre{Convergence, théorème de Dirichlet}

\contenu{
\texte{
Soit $f : \R \to \C$ $2\pi$-périodique continue, $f_n$ sa
$n$-ème somme de Fourier et $g_n = \frac{f_0 + \dots + f_n}{n+1}$.
}
\begin{enumerate}
    \item \question{Exprimer $g_n$ à l'aide d'un produit de convolution, $g_n = f*k_n$.}
    \item \question{Montrer que la suite $(k_n)$ constitue une suite d'approximations de la
    mesure de Dirac sur $]-\pi,\pi[$. Ceci montre que la moyenne des
    sommes partielles de la série de Fourier de~$f$ converge uniformément
    vers~$f$ {\it pour toute $f$ continue}.}
\reponse{
$k_n(x)= \frac{1-\cos((n+1)x)}{(n+1)(1-\cos x)}
                    = \frac{\sin^2((n+1)x/2)}{(n+1)\sin^2(x/2)}$.
}
\end{enumerate}
}

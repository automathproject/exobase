\uuid{ulOe}
\exo7id{4572}
\auteur{quercia}
\organisation{exo7}
\datecreate{2010-03-14}
\isIndication{false}
\isCorrection{true}
\chapitre{Série entière}
\sousChapitre{Rayon de convergence}

\contenu{
\texte{
Soit $a(z) = \sum_{n=0}^\infty a_nz^n$ une série entière de rayon de
convergence infini et $\rho > 0$.

On définit la série entière $b(z) = \sum_{n=0}^\infty b_nz^n$ de sorte que
$(z-\rho)b(z) = a(z)$ en cas de convergence de $b(z)$.
}
\begin{enumerate}
    \item \question{Prouver l'existence et l'unicité des coefficients $b_n$.}
\reponse{Série produit de $a(z)$ et $\frac1{z-\rho}  \Rightarrow 
             b_n = \sum_{k=0}^\infty a_{k+n+1}\rho^k$.}
    \item \question{Quel est le rayon de convergence de $b(z)$ ?}
\reponse{Si $a(\rho) \ne 0$ : $b(z)$ converge pour $|z|<\rho$ et tend vers
             l'infini pour $z\to\rho^-  \Rightarrow  R = \rho$.\par
             Si $a(\rho) = 0$ : $\forall\ r > \rho$, $|a_p| \le \frac M{r^p} \Rightarrow 
             |b_n| \le \frac M{r^n(r-\rho)}  \Rightarrow  R=\infty$.}
\end{enumerate}
}

\uuid{DL0B}
\exo7id{4174}
\auteur{quercia}
\organisation{exo7}
\datecreate{2010-03-11}
\isIndication{false}
\isCorrection{true}
\chapitre{Fonction de plusieurs variables}
\sousChapitre{Dérivée partielle}

\contenu{
\texte{
Soit $U$ un ouvert convexe de~$\R^n$ et $f : U \to \R$. On dit que
$f$ est convexe lorsque~:
$$\forall\ x,y\in U,\ \forall\ t\in{[0,1]},\ f(tx+(1-t)y) \le tf(x) + (1-t)f(y).$$
On dit que $f$ est strictement convexe si l'inégalité précédente est
stricte lorsque $x\ne y$ et $0<t<1$.
}
\begin{enumerate}
    \item \question{On suppose que $f$ est convexe.
  \begin{enumerate}}
\reponse{\begin{enumerate}}
    \item \question{Soient $x\in U$, $h\in \R^n$ et $t\in{[0,1]}$ tel que $x-h\in U$ et $x+h\in U$. Montrer~:
$$(1+t)f(x) - tf(x-h) \le f(x+th) \le (1-t)f(x) + tf(x+h).$$}
\reponse{thm des trois cordes pour $u \mapsto f(x+uh)$ sur $[-1,1]$.}
    \item \question{Montrer que $f$ est continue (raisonner sur le cas $n=2$ puis généraliser).}
\reponse{Soit $\overline{B}_\infty(x,r)\subset U$ et $y$ tel
que $0<\|x-y\|_\infty \le r$. On note $t = \|x-y\|_\infty/r$ et $h=(y-x)/t$.
Alors $y = x+th$ et $\|x\pm h\| = r$ donc~:
$$|f(x)-f(y)| \le t\max(|f(x+h)-f(x)|,|f(x-h)-f(x)|) \le Mt$$
car la restriction de~$f$ aux côtés de $\overline{B}_\infty(x,r)$ est
bornée (fonction convexe en dimension~1).}
\end{enumerate}
}

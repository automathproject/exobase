\uuid{Nge3}
\exo7id{4848}
\titre{exo7 4848}
\auteur{quercia}
\organisation{exo7}
\datecreate{2010-03-16}
\isIndication{false}
\isCorrection{true}
\chapitre{Topologie}
\sousChapitre{Espaces complets}
\module{Analyse}
\niveau{L2}
\difficulte{}

\contenu{
\texte{
Soit $E$ un evn complet et $(B_n(a_n,r_n))$ une suite d{\'e}croissante de boules
ferm{\'e}es dont le rayon ne tend pas vers 0.
Montrer que $\bigcap_n B_n$ est une boule ferm{\'e}e.
}
\reponse{
Soit $r = \lim r_n$ :
$\|a_n - a_{n+k}\| \le r_n - r_{n+k}$ donc la suite $(a_n)$ est de Cauchy,
et converge vers $a$.

On a $\|a_n - a\| \le r_n - r$ donc $B(a,r) \subset B_n$.

R{\'e}ciproquement, si $x \in \bigcap_n B_n$, alors $\|x-a_n\| \le r_n$ donc
$\|x-a\| \le r$.
}
}

\uuid{8cFR}
\exo7id{5869}
\titre{exo7 5869}
\auteur{rouget}
\organisation{exo7}
\datecreate{2010-10-16}
\isIndication{false}
\isCorrection{true}
\chapitre{Suite et série de fonctions}
\sousChapitre{Suite et série de matrices}
\module{Analyse}
\niveau{L2}
\difficulte{}

\contenu{
\texte{
Calculer $\text{exp}(tA)$, $t\in\Rr$, si
}
\begin{enumerate}
    \item \question{$A=\left(
\begin{array}{ccc}
3&2&2\\
1&0&1\\
-1&1&0
\end{array}
\right)$}
\reponse{$\chi_A=\left|
\begin{array}{ccc}
3-X&2&2\\
1&-X&1\\
-1&1&-X
\end{array}
\right|=(3-X)(X^2-1)-(-2X-2)-(2X+2)=-(X+1)(X-1)(X-3)$.

Soit $n\in\Nn$. La division euclidienne de $X^n$ par $\chi_A$ s'écrit $X^n=Q_n\times\chi_A+a_nX^2+b_nX+c_n$ où $Q_n\in\Rr[X]$ et $(a_n,b_n,c_n)\in\Rr^3$. En évaluant les deux membres de cette égalité en $-1$, $1$ et $3$, on obtient

\begin{center}
$\left\{
\begin{array}{l}
a_n-b_n+c_n=(-1)^n\\
a_n+b_n+c_n=1\\
9a_n+3b_n+c_n=3^n
\end{array}
\right.
\Rightarrow\left\{
\begin{array}{l}
b_n= \frac{1}{2}(1-(-1)^n)\\
a_n+c_n= \frac{1}{2}(1+(-1)^n)\\
8a_n+ \frac{3}{2}(1-(-1)^n)+ \frac{1}{2}(1+(-1)^n)=3^n
\end{array}
\right.\Rightarrow\left\{
\begin{array}{l}
a_n= \frac{1}{8}(3^n-2+(-1)^n)\\
\rule[-4mm]{0mm}{10mm}b_n= \frac{1}{2}(1-(-1)^n)\\
c_n= \frac{1}{8}(-3^n+6+3(-1)^n)
\end{array}
\right.$.
\end{center}

Le théorème de \textsc{Cayley}-\textsc{Hamilton} fournit alors 

\begin{center}
$\forall n\in\Nn$, $A^n= \frac{1}{8}((3^n-2+(-1)^n)A^2+4(1-(-1)^n)A+(-3^n+6+3(-1)^n)I_3)$.
\end{center}

Maintenant,

\begin{center}
$A^2=\left(
\begin{array}{ccc}
3&2&2\\
1&0&1\\
-1&1&0
\end{array}
\right)\left(
\begin{array}{ccc}
3&2&2\\
1&0&1\\
-1&1&0
\end{array}
\right)=\left(
\begin{array}{ccc}
9&8&8\\
2&3&2\\
-2&-2&-1
\end{array}
\right)$
\end{center}

et donc, pour tout réel $t$,

\begin{align*}\ensuremath
\text{exp}(tA)&=\sum_{n=0}^{+\infty} \frac{t^n}{n!}A^n=\sum_{n=0}^{+\infty} \frac{t^n}{n!}. \frac{1}{8}((3^n-2+(-1)^n)A^2+4(1-(-1)^n)A+(-3^n+6+3(-1)^n)I_3)\\
 &= \frac{e^{3t}-2e^t+e^{-t}}{8}\left(
\begin{array}{ccc}
9&8&8\\
2&3&2\\
-2&-2&-1
\end{array}
\right)+ \frac{4(e^t-e^{-t})}{8}\left(
\begin{array}{ccc}
3&2&2\\
1&0&1\\
-1&1&0
\end{array}
\right)+ \frac{-e^{3t}+6e^t+3e^{-t}}{8}\left(
\begin{array}{ccc}
1&0&0\\
0&1&0\\
0&0&1
\end{array}
\right)\\
&= \frac{1}{8}\left(
\begin{array}{ccc}
8e^{3t}&8e^{3t}-8e^t&8e^{3t}-8e^t\\
2e^{3t}-2e^{-t}&2e^{3t}+6e^{-t}&2e^{3t}-2e^{-t}\\
-2e^{3t}+2e^{-t}&-2e^{3t}+8e^t-6e^{-t}&2e^{3t}+8e^t+2e^{-t}
\end{array}
\right)\\
 &= \frac{1}{4}\left(
\begin{array}{ccc}
4e^{3t}&4e^{3t}-4e^t&4e^{3t}-4e^t\\
e^{3t}-e^{-t}&e^{3t}+3e^{-t}&e^{3t}-e^{-t}\\
-e^{3t}+e^{-t}&-e^{3t}+4e^t-3e^{-t}&e^{3t}+4e^t+e^{-t}
\end{array}
\right).
\end{align*}

\begin{center}
\shadowbox{
$\forall t\in\Rr$, $\text{exp}(tA)= \frac{1}{4}\left(
\begin{array}{ccc}
4e^{3t}&4e^{3t}-4e^t&4e^{3t}-4e^t\\
e^{3t}-e^{-t}&e^{3t}+3e^{-t}&e^{3t}-e^{-t}\\
-e^{3t}+e^{-t}&-e^{3t}+4e^t-3e^{-t}&e^{3t}+4e^t+e^{-t}
\end{array}
\right)$.
}
\end{center}}
    \item \question{$A=\left(
\begin{array}{ccc}
4&1&1\\
6&4&2\\
-10&-4&-2
\end{array}
\right)$}
\reponse{$\chi_A=\left|
\begin{array}{ccc}
4-X&1&1\\
6&4-X&2\\
-10&-4&-2-X
\end{array}
\right|=(4-X)(X^2-2X)-6(-X+2)-10(X-2)=(X-2)[-X(X-4)+6-10]=-(X-2)(X^2-4X+4)=-(X-2)^3$. On est dans la situation où $A$ a une unique valeur propre. D'après le théorème de \textsc{Cayley}-\textsc{Hamilton}, $(A-2I_3)^3=0$ et donc pour tout réel $t$,

\begin{align*}\ensuremath
\text{exp}(tA)&=\text{exp}(t(A-2I_3)+2tI_3)=\text{exp}(t(A-2I_3))\times\text{exp}(2tI_3)\;(\text{car les matrices}\;t(A-2I_3)\;\text{et}\;2tI_3\;\text{commutent})\\
 &=\left(I_3+t(A-2I_3)+ \frac{t^2}{2}(A-2I_3)^2\right)\times e^{2t}I_3\\
 &=e^{2t}\left(
\begin{array}{ccc}
1&0&0\\
0&1&0\\
0&0&1
\end{array}
\right)+te^{2t}\left(
\begin{array}{ccc}
2&1&1\\
6&2&2\\
-10&-4&-4
\end{array}
\right)+ \frac{t^2e^{2t}}{2}\left(
\begin{array}{ccc}
2&1&1\\
6&2&2\\
-10&-4&-4
\end{array}
\right)\left(
\begin{array}{ccc}
2&1&1\\
6&2&2\\
-10&-4&-4
\end{array}
\right)
\\
 &=e^{2t}\left(
\begin{array}{ccc}
1&0&0\\
0&1&0\\
0&0&1
\end{array}
\right)+te^{2t}\left(
\begin{array}{ccc}
2&1&1\\
6&2&2\\
-10&-4&-4
\end{array}
\right)+ \frac{t^2e^{2t}}{2}\left(
\begin{array}{ccc}
0&0&0\\
4&2&2\\
-4&-2&-2
\end{array}
\right)\\
 &=\left(
\begin{array}{ccc}
(2t+1)e^{2t}&te^{2t}&te^{2t}\\
(2t^2+6t)e^{2t}&(t^2+2t+1)e^{2t}&(t^2+2t)e^{2t}\\
(-2t^2-10t)e^{2t}&(-t^2-4t)e^{2t}&(-t^2-4t)e^{2t}
\end{array}
\right).
\end{align*}

\begin{center}
\shadowbox{
$\forall t\in\Rr$, $\text{exp}(tA)=\left(
\begin{array}{ccc}
(2t+1)e^{2t}&te^{2t}&te^{2t}\\
(2t^2+6t)e^{2t}&(t^2+2t+1)e^{2t}&(t^2+2t)e^{2t}\\
(-2t^2-10t)e^{2t}&(-t^2-4t)e^{2t}&(-t^2-4t)e^{2t}
\end{array}
\right)$.
}
\end{center}}
\end{enumerate}
}

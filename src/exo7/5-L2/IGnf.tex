\uuid{IGnf}
\exo7id{4376}
\titre{exo7 4376}
\auteur{quercia}
\organisation{exo7}
\datecreate{2010-03-12}
\isIndication{false}
\isCorrection{true}
\chapitre{Intégration}
\sousChapitre{Intégrale de Riemann dépendant d'un paramètre}
\module{Analyse}
\niveau{L2}
\difficulte{}

\contenu{
\texte{
Soit $(f_n)_{n\in\N^*}$ une suite de fonctions définie par~:
$\forall\ n\in\N^*,\ \forall\ x\in{[0,1]},\ f_n(x) = \Bigl(\frac{x+x^n}2\Bigr)^n$.
}
\begin{enumerate}
    \item \question{Montrer que~$(f_n)$ converge simplement vers une fonction~$\varphi$.}
\reponse{$0\le f_n(x)\le x^n$ et $f_n(1) = 1$ donc  lorsque $n\to\infty$ $f_n(x)\to \begin{cases}0 & \text{ si }  x<1 \cr 1 & \text{ si }x=1.\cr\end{cases}$}
    \item \question{\begin{enumerate}}
\reponse{\begin{enumerate}}
    \item \question{La convergence est-elle uniforme~?}
\reponse{Non, la continuité n'est pas conservée.}
    \item \question{La convergence est-elle monotone~?}
\reponse{Oui, il y a décroissance évidente.}
\end{enumerate}
}

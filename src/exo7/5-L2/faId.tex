\uuid{faId}
\exo7id{2629}
\titre{exo7 2629}
\auteur{debievre}
\organisation{exo7}
\datecreate{2009-05-19}
\isIndication{true}
\isCorrection{true}
\chapitre{Fonction de plusieurs variables}
\sousChapitre{Différentiabilité}
\module{Analyse}
\niveau{L2}
\difficulte{}

\contenu{
\texte{
On demande \`a un \'etudiant de trouver l'\'equation du plan tangent \`a
la surface d'\'equation
$z=x^4-y^2$ au point $(x_0,y_0,z_0)=(2,3,7)$. Sa r\'eponse est
\[
z=4x^3(x-2)-2y(y-3).
\]
}
\begin{enumerate}
    \item \question{Expliquer, sans calcul, pourquoi cela ne peut en aucun 
cas \^etre la
bonne r\'eponse.}
\reponse{L'\'equation d'un plan tangent doit \^etre une \'equation lin\'eaire !}
    \item \question{Quelle est l'erreur commise par l'\'etudiant?}
\reponse{La confusion est exactement celle \`a \'eviter suivant les indications
donn\'ees.}
    \item \question{Donner la r\'eponse correcte.}
\reponse{D'apr\`es \eqref{tang1}, le plan tangent \`a la surface d'\'equation
$z=f(x,y)=x^4-y^2$ au point $(x_0,y_0,z_0)=(2,3,7)$ est donn\'e 
par l'\'equation 
\[
z-7=\frac{\partial f}{\partial x}(2,3) (x-2) 
+\frac{\partial f}{\partial y}(2,3)
(y-3)
\]
c.a.d.
\[
z-7=32 (x-2) -6 (y-3) .
\]}
\indication{Ne pas confondre les variables pour l'\'equation de la surface,
les variables pour l'\'equation de la tangente en un point,
et les coordonn\'ees du point de contact.}
\end{enumerate}
}

\uuid{vyyp}
\exo7id{5894}
\auteur{rouget}
\organisation{exo7}
\datecreate{2010-10-16}
\isIndication{false}
\isCorrection{true}
\chapitre{Fonction de plusieurs variables}
\sousChapitre{Différentiabilité}

\contenu{
\texte{
Extremums des fonctions suivantes :
}
\begin{enumerate}
    \item \question{$f(x,y) = x^3+3x^2y -15x-12y$}
\reponse{$f$ est de classe $C^1$ sur $\Rr^2$ qui est un ouvert de $\Rr^2$. Donc si $f$ admet un extremum local en un point $(x_0,y_0)$ de $\Rr^2$, $(x_0,y_0)$ est un point critique de $f$.

\begin{center}
$df_{(x,y)}=0\Leftrightarrow\left\{
\begin{array}{l}
3x^2+6xy-15=0\\
3x^2-12=0
\end{array}
\right.\Leftrightarrow\left\{
\begin{array}{l}
x=2\\
y= \frac{1}{4}
\end{array}
\right.\;\text{ou}\;\left\{
\begin{array}{l}
x=-2\\
y=- \frac{1}{4}
\end{array}
\right.$.
\end{center}

Réciproquement, $r=6x+6y$, $t=0$ et $s=6x$ puis $rt-s^2=-36x^2$. Ainsi, $(rt-s^2)\left(2, \frac{1}{4}\right)=(rt-s^2)\left(-2,- \frac{1}{4}\right)=-144<0$ et $f$ n'admet pas d'extremum local en $\left(2, \frac{1}{4}\right)$ ou $\left(-2,- \frac{1}{4}\right)$.

\begin{center}
\shadowbox{
$f$ n'admet pas d'extremum local sur $\Rr^2$.
}
\end{center}}
    \item \question{$f(x,y) = -2(x-y)^2+x^4+y^4$.}
\reponse{La fonction $f$ est de classe $C^1$ sur $\Rr^2$ en tant que polynôme à plusieurs variables. Donc, si $f$ admet un extremum local en $(x_0,y_0)\in\Rr^2$, $(x_0,y_0)$ est un point critique de $f$. Soit $(x,y)\in\Rr^2$.

\begin{align*}\ensuremath
\left\{
\begin{array}{l}
 \frac{\partial f}{\partial x}(x,y)=0\\
\rule{0mm}{7mm} \frac{\partial f}{\partial y}(x,y)=0
\end{array}
\right.&\Leftrightarrow\left\{
\begin{array}{l}
-4(x-y)+4x^3=0\\
4(x-y)+4y^3=0
\end{array}
\right.\Leftrightarrow\left\{
\begin{array}{l}
x^3+y^3=0\\
-4(x-y)+4x^3=0
\end{array}
\right.\Leftrightarrow\left\{
\begin{array}{l}
y=-x\\
x^3-2x=0
\end{array}
\right.\\
 &\Leftrightarrow(x,y)\in\left\{(0,0),\left(\sqrt{2},\sqrt{2}\right),\left(-\sqrt{2},-\sqrt{2}\right)\right\}.
\end{align*}

Réciproquement, $f$ est plus précisément de classe $C^2$ sur $\Rr^2$ et

\begin{center}
$r(x,y)t(x,y)-s^2(x,y)=(-4+12x^2)(-4+12y^2)-(4)^2=-48x^2-48y^2+144x^2y^2=48(3x^2y^2-x^2-y^2)$
\end{center}

\textbullet~$(rt-s^2)\left(\sqrt{2},\sqrt{2}\right)=48(12-2-2)>0$. Donc $f$ admet un extremum local en $\left(\sqrt{2},\sqrt{2}\right)$. Plus précisément, puisque $r\left(\sqrt{2},\sqrt{2}\right)=2\times12-4=20>0$, $f$ admet un minimum local en $\left(\sqrt{2},\sqrt{2}\right)$. De plus, pour $(x,y)\in\Rr^2$,

\begin{align*}\ensuremath
f(x,y)-f\left(\sqrt{2},\sqrt{2}\right)&=-2(x-y)^2+x^4+y^4-8=x^4+y^4-2x^2-2y^2+4xy+8\\
 &\geqslant x^4+y^4-2x^2-2y^2-2(x^2+y^2)+8=(x^4-4x^2+4)+(y^4-4y^2+4)=(x^2-2)^2+(y^2-2)^2\\
 &\geqslant0.
\end{align*}

et $f\left(\sqrt{2},\sqrt{2}\right)$ est un minimum global.

\textbullet~Pour tout $(x,y)\in\Rr^2$, $f(-x,-y)=f(x,y)$ et donc $f$ admet aussi un minimum global en $\left(-\sqrt{2},-\sqrt{2}\right)$ égal à $8$.

\textbullet~$f(0,0)=0$. Pour $x\neq0$, $f(x,x)=2x^4>0$ et donc $f$ prend des valeurs strictement supérieures à $f(0,0)$ dans tout voisinage de $(0,0)$. Pour $x\in\left]-\sqrt{2},\sqrt{2}\right[\setminus\{0\}$, $f(x,0)=x^4-2x^2=x^2(x^2-2)<0$ et $f$ prend des valeurs strictement inférieures à $f(0,0)$ dans tout voisinage de $(0,0)$. Finalement, $f$ n'admet pas d'extremum local en $(0,0)$.

\begin{center}
\shadowbox{
$f$ admet un minimum global égal à $8$, atteint en $\left(\sqrt{2},\sqrt{2}\right)$ et $\left(-\sqrt{2},-\sqrt{2}\right)$.
}
\end{center}}
\end{enumerate}
}

\uuid{8CJn}
\exo7id{5889}
\auteur{rouget}
\organisation{exo7}
\datecreate{2010-10-16}
\isIndication{false}
\isCorrection{true}
\chapitre{Fonction de plusieurs variables}
\sousChapitre{Différentielle seconde}

\contenu{
\texte{
Soit $f(x,y) =\left\{
\begin{array}{l}
y^2\sin\left( \frac{x}{y}\right)\;\text{si}\;y\neq0\\
\rule{0mm}{5mm}0\;\text{si}\;y=0
\end{array}
\right.$.

Déterminer le plus grand sous-ensemble de $\Rr^2$ sur lequel $f$ est de classe $C^1$. Vérifier que $ \frac{\partial^2f}{\partial x\partial y}(0,0)$ et $ \frac{\partial^2f}{\partial y\partial x}(0,0)$ existent et sont différents.
}
\reponse{
On pose $D=\{(x,0),\;x\in\Rr\}$ puis $\Omega=\Rr^2\setminus D$.

\textbullet~$f$ est définie sur $\Rr^2$.

\textbullet~$f$ est de classe $C^1$ sur $\Omega$ en vertu de théorèmes généraux et pour $(x,y)\in\Omega$,

\begin{center}
$ \frac{\partial f}{\partial x}(x,y)=y\cos\left( \frac{x}{y}\right)$ et $ \frac{\partial f}{\partial y}(x,y)=2y\sin\left( \frac{x}{y}\right)-x\cos\left( \frac{x}{y}\right)$.
\end{center}

\textbullet~Etudions la continuité de $f$ en $(0,0)$. Pour $(x,y)\neq(0,0)$,

\begin{center}
$|f(x,y)-f(0,0)|=\left\{
\begin{array}{l}
y^2\left|\sin\left( \frac{x}{y}\right)\right|\;\text{si}\;y\neq0\\
\rule{0mm}{5mm}0\;\text{si}\;y=0
\end{array}
\right.\leqslant\left\{
\begin{array}{l}
y^2\;\text{si}\;y\neq0\\
\rule{0mm}{5mm}0\;\text{si}\;y=0
\end{array}
\right.\leqslant y^2$.
\end{center}

Comme $y^2$ tend vers $0$ quand $(x,y)$ tend vers $0$, $\displaystyle\lim_{\substack{(x,y)\rightarrow(0,0)\\(x,y)\neq(0,0)}}f(x,y)=f(0,0)$ et donc $f$ est continue en $(0,0)$ puis

\begin{center}
$f$ est continue sur $\Rr^2$.
\end{center}

\textbullet~Etudions l'existence et la valeur éventuelle de $ \frac{\partial f}{\partial x}(x_0,0)$, $x_0$ réel donné. Pour $x\neq x_0$,

\begin{center}
$ \frac{f(x,0)-f(x_0,0)}{x-x_0}= \frac{0-0}{x-x_0}=0$.
\end{center}

Donc $ \frac{f(x,x_0)-f(x_0,0)}{x-x_0}$ tend vers $0$ quand $x$ tend vers $x_0$. On en déduit que $ \frac{\partial f}{\partial x}(x_0,0)$ existe et $ \frac{\partial f}{\partial x}(x_0,0)=0$. Finalement, la fonction $ \frac{\partial f}{\partial x}$ est définie sur $\Rr^2$ par

\begin{center}
$\forall (x,y)\in\Rr^2$, $ \frac{\partial f}{\partial x}(x,y)=\left\{
\begin{array}{l}
y\cos\left( \frac{x}{y}\right)\;\text{si}\;y\neq0\\
\rule{0mm}{5mm}0\;\text{si}\;y=0
\end{array}
\right.$.
\end{center}

\textbullet~Etudions l'existence et la valeur éventuelle de $ \frac{\partial f}{\partial y}(x_0,0)$, $x_0$ réel donné. Pour $y\neq 0$,

\begin{center}
$ \frac{f(x_0,y)-f(x_0,0)}{y-0}= \frac{y^2\sin\left( \frac{x_0}{y}\right)}{y}=y\sin\left( \frac{x_0}{y}\right)$.
\end{center}

On en déduit que $\left| \frac{f(x_0,y)-f(x_0,0)}{y-0}\right|\leqslant|y|$ puis que $ \frac{f(x_0,y)-f(x_0,0)}{y-0}$ tend vers $0$ quand $y$ tend vers $0$. Par suite, $ \frac{\partial f}{\partial y}(x_0,0)$ existe et $ \frac{\partial f}{\partial y}(x_0,0)=0$. Finalement, la fonction $ \frac{\partial f}{\partial y}$ est définie sur $\Rr^2$ par

\begin{center}
$\forall (x,y)\in\Rr^2$, $ \frac{\partial f}{\partial y}(x,y)=\left\{
\begin{array}{l}
2y\sin\left( \frac{x}{y}\right)-x\cos\left( \frac{x}{y}\right)\;\text{si}\;y\neq0\\
\rule{0mm}{5mm}0\;\text{si}\;y=0
\end{array}
\right.$.
\end{center}

\textbullet~Etudions la continuité de $ \frac{\partial f}{\partial x}$ en $(x_0,0)$, $x_0$ réel donné. Pour $(x,y)\in\Rr^2$,

\begin{center}
$\left| \frac{\partial f}{\partial x}(x,y)- \frac{\partial f}{\partial x}(x_0,0)\right|=\left\{
\begin{array}{l}
|y|\left|\cos\left( \frac{x}{y}\right)\right|\;\text{si}\;y\neq0\\
\rule{0mm}{5mm}0\;\text{si}\;y=0
\end{array}
\right.\leqslant|y|$.
\end{center}

Quand $(x,y)$ tend vers $(0,0)$, $|y|$ tend vers $0$ et donc $ \frac{\partial f}{\partial x}(x,y)$ tend vers $ \frac{\partial f}{\partial x}(x_0,0)$ quand $(x,y)$ tend vers $(x_0,0)$. La fonction $ \frac{\partial f}{\partial x}$ est donc continue en $(x_0,0)$ et finalement

\begin{center}
la fonction $ \frac{\partial f}{\partial x}$ est continue sur $\Rr^2$.
\end{center}

\textbullet~Etudions la continuité de $ \frac{\partial f}{\partial y}$ en $(x_0,0)$, $x_0$ réel donné. Supposons tout d'abord $x_0=0$. Pour $(x,y)\in\Rr^2$,

\begin{center}
$\left| \frac{\partial f}{\partial y}(x,y)- \frac{\partial f}{\partial y}(0,0)\right|=\left\{
\begin{array}{l}
\left|2y\sin\left( \frac{x}{y}\right)-x\cos\left( \frac{x}{y}\right)\right|\;\text{si}\;y\neq0\\
\rule{0mm}{5mm}0\;\text{si}\;y=0
\end{array}
\right.\leqslant2|y|+|x|$.
\end{center}

Quand $(x,y)$ tend vers $(0,0)$, $|x|+2|y|$ tend vers $0$ et donc $ \frac{\partial f}{\partial y}(x,y)$ tend vers $ \frac{\partial f}{\partial y}(0,0)$ quand $(x,y)$ tend vers $(0,0)$. 

Supposons maintenant $x_0\neq0$. Pour $y\neq0$, $ \frac{\partial f}{\partial y}(x_0,y)=2y\sin\left( \frac{x_0}{y}\right)-x_0\cos\left( \frac{x_0}{y}\right)$. Quand $y$ tend vers $0$, $2y\sin\left( \frac{x_0}{y}\right)$ tend vers $0$ car $\left|2y\sin\left( \frac{x_0}{y}\right)\right|$ et $x_0\cos\left( \frac{x_0}{y}\right)$ n'a pas de limite réelle car $x_0\neq0$. Donc $ \frac{\partial f}{\partial y}(x_0,y)$ n'a pas de limite quand $y$ tend vers $0$ et la fonction $ \frac{\partial f}{\partial y}$ n'est pas continue en $(x_0,0)$ si $x_0\neq0$. On a montré que

\begin{center}
$f$ est de classe $C^1$ sur $\Omega\cup\{(0,0)\}$.
\end{center}

\textbullet~Etudions l'existence et la valeur éventuelle de $ \frac{\partial^2f}{\partial x\partial y}(0,0)$. Pour $x\neq0$,

\begin{center} 
$ \frac{ \frac{\partial f}{\partial y}(x,0)- \frac{\partial f}{\partial y}(0,0)}{x-0}= \frac{0-0}{x}=0$.
\end{center}

Donc $ \frac{ \frac{\partial f}{\partial y}(x,0)- \frac{\partial f}{\partial y}(0,0)}{x-0}$ tend vers $0$ quand $x$ tend vers $0$. On en déduit que $ \frac{\partial^2}{\partial x\partial y}(0,0)$ existe et $ \frac{\partial^2}{\partial x\partial y}(0,0)=0$. 

\textbullet~Etudions l'existence et la valeur éventuelle de $ \frac{\partial^2f}{\partial y\partial x}(0,0)$. Pour $y\neq0$,

\begin{center} 
$ \frac{ \frac{\partial f}{\partial x}(0,y)- \frac{\partial f}{\partial x}(0,0)}{y-0}= \frac{y\cos\left( \frac{0}{y}\right)}{y}=1$.
\end{center}

Donc $ \frac{ \frac{\partial f}{\partial x}(0,y)- \frac{\partial f}{\partial x}(0,0)}{y-0}$ tend vers $1$ quand $y$ tend vers $0$. On en déduit que $ \frac{\partial^2}{\partial y\partial x}(0,0)$ existe et $ \frac{\partial^2}{\partial y\partial x}(0,0)=1$. On a montré que $ \frac{\partial^2}{\partial x\partial y}(0,0)$ et $ \frac{\partial^2}{\partial y\partial x}(0,0)$ existent et sont différents. D'après le théorème de \textsc{Schwarz}, $f$ n'est pas de classe $C^2$ sur $\Omega\cup\{(0,0)\}$.
}
}

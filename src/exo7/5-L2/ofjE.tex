\uuid{ofjE}
\exo7id{2651}
\auteur{debievre}
\organisation{exo7}
\datecreate{2009-05-19}
\isIndication{false}
\isCorrection{false}
\chapitre{Fonction de plusieurs variables}
\sousChapitre{Différentiabilité}

\contenu{
\texte{

}
\begin{enumerate}
    \item \question{Soit $f:D\subset \R^m\to\R^n$ et $a\in D$. Donner la d\'efinition de 
``$f$ est diff\'erentiable en $a$''.}
    \item \question{Montrer que, si $f$ est diff\'erentiable en $a$, alors toutes ses d\'eriv\'ees partielles existent. Exprimer le lien entre la diff\'erentielle $d f_a$ de $f$ en $a$ et les d\'eriv\'ees partielles de $f$ en $a$.}
    \item \question{Les affirmations suivantes, sont-elles vraies ou fausses? On justifiera bri\`evement sa r\'eponse.

(A) Si $f$ est diff\'erentiable en $a$, alors elle y est continue.

(B) Si toutes les d\'eriv\'ees partielles de $f$ en $a$ existent, alors $f$ est diff\'erentiable en $a$.}
\end{enumerate}
}

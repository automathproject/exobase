\uuid{jHI7}
\exo7id{4828}
\titre{exo7 4828}
\auteur{quercia}
\organisation{exo7}
\datecreate{2010-03-16}
\isIndication{false}
\isCorrection{true}
\chapitre{Topologie}
\sousChapitre{Compacité}
\module{Analyse}
\niveau{L2}
\difficulte{}

\contenu{
\texte{
Soit $A$ une partie compacte d'un evn $E$ et $f : A \to  A$ telle que :
$\forall\ x,y \in A,\ x\ne y  \Rightarrow  d(f(x),f(y)) < d(x,y)$.
}
\begin{enumerate}
    \item \question{Montrer que $f$ admet un point fixe unique, $a$.}
    \item \question{Soit $(x_n)$ une suite d'{\'e}l{\'e}ments de $A$ telle que $x_{n+1} = f(x_n)$.
    Montrer qu'elle converge vers $a$.}
\reponse{
$d(x_n,a)$ d{\'e}croit, donc tend vers $d$.  Il existe une
    sous-suite $(x_{n_k})$ convergeant vers $\ell$ et $d(\ell,a) = d$.
    La suite $(f(x_{n_k}))$ converge vers $f(\ell)$ et on a
    $d(f(\ell),a) = d$, donc $\ell = a$. Il y a une seule valeur
    d'adh{\'e}rence, donc la suite converge.
}
\end{enumerate}
}

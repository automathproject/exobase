\uuid{2yD6}
\exo7id{4076}
\titre{exo7 4076}
\auteur{quercia}
\organisation{exo7}
\datecreate{2010-03-11}
\isIndication{false}
\isCorrection{false}
\chapitre{Equation différentielle}
\sousChapitre{Equations différentielles linéaires}
\module{Analyse}
\niveau{L2}
\difficulte{}

\contenu{
\texte{
Soit $\lambda \in \C$ et $\varphi : \R \to \C$ $T$-périodique.
On considère l'équation : $(*) \Leftrightarrow y' + \lambda y = \varphi(x)$.
}
\begin{enumerate}
    \item \question{Montrer que si $y$ est solution de $(*)$, alors $y(x+T)$ est aussi solution.}
    \item \question{En déduire que $y$, solution de $(*)$, est $T$-périodique si et seulement si $y(0) = y(T)$.}
    \item \question{Montrer que, sauf pour des valeurs exceptionnelles de $\lambda$, l'équation
    $(*)$ admet une et une seule solution $T$-périodique.}
\end{enumerate}
}

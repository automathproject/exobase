%%%%%%%%%%%%%%%%%% PREAMBULE %%%%%%%%%%%%%%%%%%

\documentclass[12pt,a4paper]{article}

\usepackage{amsfonts,amsmath,amssymb,amsthm}
\usepackage[francais]{babel}
\usepackage[utf8]{inputenc}
\usepackage[T1]{fontenc}

%----- Ensemles : entiers, reels, complexes -----
\newcommand{\Nn}{\mathbb{N}} \newcommand{\N}{\mathbb{N}}
\newcommand{\Zz}{\mathbb{Z}} \newcommand{\Z}{\mathbb{Z}}
\newcommand{\Qq}{\mathbb{Q}} \newcommand{\Q}{\mathbb{Q}}
\newcommand{\Rr}{\mathbb{R}} \newcommand{\R}{\mathbb{R}}
\newcommand{\Cc}{\mathbb{C}} \newcommand{\C}{\mathbb{C}}

%----- Modifications de symboles -----
\renewcommand {\epsilon}{\varepsilon}
\renewcommand {\Re}{\mathop{\mathrm{Re}}\nolimits}
\renewcommand {\Im}{\mathop{\mathrm{Im}}\nolimits}

%----- Fonctions usuelles -----
\newcommand{\ch}{\mathop{\mathrm{ch}}\nolimits}
\newcommand{\sh}{\mathop{\mathrm{sh}}\nolimits}
\renewcommand{\tanh}{\mathop{\mathrm{th}}\nolimits}
\newcommand{\Arcsin}{\mathop{\mathrm{Arcsin}}\nolimits}
\newcommand{\Arccos}{\mathop{\mathrm{Arccos}}\nolimits}
\newcommand{\Arctan}{\mathop{\mathrm{Arctan}}\nolimits}
\newcommand{\Argsh}{\mathop{\mathrm{Argsh}}\nolimits}
\newcommand{\Argch}{\mathop{\mathrm{Argch}}\nolimits}
\newcommand{\Argth}{\mathop{\mathrm{Argth}}\nolimits}
\newcommand{\pgcd}{\mathop{\mathrm{pgcd}}\nolimits} 

%----- Commandes special dessin a ajouter localement ------
\usepackage{geometry}
\usepackage{pstricks}
\usepackage{pst-plot}
\usepackage{pst-node}
\usepackage{graphics,epsfig}

\pagestyle{empty}

% Que faire avec ce fichier monimage.tex ?
%   1/ latex monimage.tex
%   2/ dvips monimage.dvi
%   3/ ps2eps monimage.ps
%   4/ ps2pdf -dEPSCrop monimage.eps
%   5/ Dans votre fichier d'exos \includegraphics{monimage}

\begin{document}

\begin{pspicture}(-6.1,-3.2)(0.1,3.2)
\psset{xunit=2cm,yunit=2cm}
\psaxes{->}(0,0)(-2.9,-1.2)(4.2,1.2)
\psplot[plotpoints=10000]{-2.9}{4}{x 3.1415 div 360 mul sin}
\psline(-1.1,-1.1)(1.1,1.1)
\psline[linestyle=dashed](2.9,0)(2.9,-0.464)
\psline[linestyle=dashed](2.9,-0.464)(-0.464,-0.464)
\psline[linestyle=dashed](-0.464,-0.464)(-0.464,-0.801)
\psline[linestyle=dashed](-0.464,-0.801)(-0.801,-0.801)
\psline[linestyle=dashed](-0.801,-0.801)(-0.801,-0.999)
\psline[linestyle=dashed](-0.801,-0.999)(-0.999,-0.999)
\psline[linestyle=dashed](-0.999,-0.999)(-0.999,-0.909)
\psline[linestyle=dashed](-0.999,-0.909)(-0.909,-0.909)

\psline[linestyle=dashed](3.6,0)(3.6,0.793)
\psline[linestyle=dashed](3.6,0.793)(0.793,0.793)
\psline[linestyle=dashed](0.793,0.793)(0.793,0.999)
\psline[linestyle=dashed](0.793,0.999)(0.999,0.999)
\psline[linestyle=dashed](0.999,0.999)(0.999,0.909)
\psline[linestyle=dashed](0.999,0.909)(0.909,0.909)
\psline[linestyle=dashed](0.909,0.909)(0.909,0.969)
\psline[linestyle=dashed](0.909,0.969)(0.969,0.969)
\uput[u](2.9,0){$u_0$}
\uput[d](3.6,0){$u_0'$}
\uput[ur](0.947,0){$\alpha$}
\uput[u](-0.947,0){$-\alpha$}
\psline[linestyle=dashed](0.947,0)(0.947,0.947)
\psline[linestyle=dashed](-0.947,0)(-0.947,-0.947)
\uput[u](1.1,1.1){$y=x$}
\uput[r](4,1){$y=\sin(2x)$}
\end{pspicture}

\end{document}

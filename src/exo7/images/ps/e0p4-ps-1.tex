%%%%%%%%%%%%%%%%%% PREAMBULE %%%%%%%%%%%%%%%%%%

\documentclass[12pt,a4paper]{article}

\usepackage{amsfonts,amsmath,amssymb,amsthm}
\usepackage[francais]{babel}
\usepackage[utf8]{inputenc}
\usepackage[T1]{fontenc}

%----- Ensemles : entiers, reels, complexes -----
\newcommand{\Nn}{\mathbb{N}} \newcommand{\N}{\mathbb{N}}
\newcommand{\Zz}{\mathbb{Z}} \newcommand{\Z}{\mathbb{Z}}
\newcommand{\Qq}{\mathbb{Q}} \newcommand{\Q}{\mathbb{Q}}
\newcommand{\Rr}{\mathbb{R}} \newcommand{\R}{\mathbb{R}}
\newcommand{\Cc}{\mathbb{C}} \newcommand{\C}{\mathbb{C}}

%----- Modifications de symboles -----
\renewcommand {\epsilon}{\varepsilon}
\renewcommand {\Re}{\mathop{\mathrm{Re}}\nolimits}
\renewcommand {\Im}{\mathop{\mathrm{Im}}\nolimits}

%----- Fonctions usuelles -----
\newcommand{\ch}{\mathop{\mathrm{ch}}\nolimits}
\newcommand{\sh}{\mathop{\mathrm{sh}}\nolimits}
\renewcommand{\tanh}{\mathop{\mathrm{th}}\nolimits}
\newcommand{\Arcsin}{\mathop{\mathrm{Arcsin}}\nolimits}
\newcommand{\Arccos}{\mathop{\mathrm{Arccos}}\nolimits}
\newcommand{\Arctan}{\mathop{\mathrm{Arctan}}\nolimits}
\newcommand{\Argsh}{\mathop{\mathrm{Argsh}}\nolimits}
\newcommand{\Argch}{\mathop{\mathrm{Argch}}\nolimits}
\newcommand{\Argth}{\mathop{\mathrm{Argth}}\nolimits}
\newcommand{\pgcd}{\mathop{\mathrm{pgcd}}\nolimits} 

%----- Commandes special dessin a ajouter localement ------
\usepackage{geometry}
\usepackage{pstricks}
\usepackage{pst-plot}
\usepackage{pst-node}
\usepackage{graphics,epsfig}

\newcmykcolor{DarkGreen}{0.4 0 0.76 0.3}
\newcmykcolor{Goldenrod}{0 0.1 0.84 0}
\newcmykcolor{Fuchsia}{0.47 0.91 0 0.08}

\pagestyle{empty}

% Que faire avec ce fichier monimage.tex ?
%   1/ latex monimage.tex
%   2/ dvips monimage.dvi
%   3/ ps2eps monimage.ps
%   4/ ps2pdf -dEPSCrop monimage.eps
%   5/ Dans votre fichier d'exos \includegraphics{monimage}

\begin{document}

\begin{center}
\begin{pspicture}(-1,-1)(5,5)
\pscustom[fillstyle=solid,fillcolor=Goldenrod,linestyle=none]
{
\psplot{1}{1.442}{x dup mul}
\psline(1.442,0)(1,0)
}
\pscustom[fillstyle=solid,fillcolor=white,linestyle=none]
{
\psplot{0.9}{1.442}{x sqrt}
\psline(1.442,0)(0.9,0)
}
\pscustom[fillstyle=solid,fillcolor=Goldenrod,linestyle=none]
{
\psplot{1.442}{3}{x 3 mul sqrt}
\psline(3,0)(1.442,0)
}
\pscustom[fillstyle=solid,fillcolor=white,linestyle=none]
{
\psplot{1.4}{2.1}{x sqrt}
\psline(2.1,0)(1.4,0)
}
\pscustom[fillstyle=solid,fillcolor=white,linestyle=none]
{
\psplot{2.08}{3.1}{x dup mul 3 div}
\psline(3.1,0)(2.08,0)
}
\psline{->}(-0.5,0)(5,0)
\psline{->}(0,-0.5)(0,5)
\psplot[linecolor=blue]{0}{5}{x sqrt}
\psplot[linecolor=blue]{0}{5}{x 3 mul sqrt}
\psplot[linecolor=blue]{0}{2.2}{x dup mul}
\psplot[linecolor=blue]{0}{3.9}{x dup mul 3 div}
\psplot[linestyle=dashed]{0}{3}{x dup mul 2 div}
\psplot[linestyle=dashed]{0}{5}{x 1.5 mul sqrt}
%\psdots(3,3)
%\psdots(1.442,2.08)
%\psdots(1,1)
%\psdots(2.08,1.442)
\uput[r](5,2.23){\textcolor{blue}{$y^2=2p_1x$}}
\uput[r](5,3.87){\textcolor{blue}{$y^2=2p_1x$}}
\uput[u](2.2,4.84){\textcolor{blue}{$x^2=2q_1y$}}
\uput[u](4.1,4.9){\textcolor{blue}{$x^2=2q_2y$}}
\uput[u](3,4.5){$x^2=2qy$}
\uput[r](5,2.73){$y^2=2px$}
\psdots(1.817,1.65)
\end{pspicture}
\end{center}

\end{document}

%%%%%%%%%%%%%%%%%% PREAMBULE %%%%%%%%%%%%%%%%%%

\documentclass[12pt,a4paper]{article}

\usepackage{amsfonts,amsmath,amssymb,amsthm}
\usepackage[francais]{babel}
\usepackage[utf8]{inputenc}
\usepackage[T1]{fontenc}

%----- Ensemles : entiers, reels, complexes -----
\newcommand{\Nn}{\mathbb{N}} \newcommand{\N}{\mathbb{N}}
\newcommand{\Zz}{\mathbb{Z}} \newcommand{\Z}{\mathbb{Z}}
\newcommand{\Qq}{\mathbb{Q}} \newcommand{\Q}{\mathbb{Q}}
\newcommand{\Rr}{\mathbb{R}} \newcommand{\R}{\mathbb{R}}
\newcommand{\Cc}{\mathbb{C}} \newcommand{\C}{\mathbb{C}}

%----- Modifications de symboles -----
\renewcommand {\epsilon}{\varepsilon}
\renewcommand {\Re}{\mathop{\mathrm{Re}}\nolimits}
\renewcommand {\Im}{\mathop{\mathrm{Im}}\nolimits}

%----- Fonctions usuelles -----
\newcommand{\ch}{\mathop{\mathrm{ch}}\nolimits}
\newcommand{\sh}{\mathop{\mathrm{sh}}\nolimits}
\renewcommand{\tanh}{\mathop{\mathrm{th}}\nolimits}
\newcommand{\Arcsin}{\mathop{\mathrm{Arcsin}}\nolimits}
\newcommand{\Arccos}{\mathop{\mathrm{Arccos}}\nolimits}
\newcommand{\Arctan}{\mathop{\mathrm{Arctan}}\nolimits}
\newcommand{\Argsh}{\mathop{\mathrm{Argsh}}\nolimits}
\newcommand{\Argch}{\mathop{\mathrm{Argch}}\nolimits}
\newcommand{\Argth}{\mathop{\mathrm{Argth}}\nolimits}
\newcommand{\pgcd}{\mathop{\mathrm{pgcd}}\nolimits} 

%----- Commandes special dessin a ajouter localement ------
\usepackage{geometry}
\usepackage{pstricks}
\usepackage{pst-plot}
\usepackage{pst-node}
\usepackage{graphics,epsfig}

\pagestyle{empty}

% Que faire avec ce fichier monimage.tex ?
%   1/ latex monimage.tex
%   2/ dvips monimage.dvi
%   3/ ps2eps monimage.ps
%   4/ ps2pdf -dEPSCrop monimage.eps
%   5/ Dans votre fichier d'exos \includegraphics{monimage}

\begin{document}

\begin{pspicture}(-6.1,-0.6)(0.1,8.1)
\psset{xunit=2cm,yunit=2cm}
\psaxes{->}(0,0)(-1.5,-0.5)(4.5,3.5)
\psplot[plotpoints=10000]{-0.5}{2.7}{x dup mul x 2 mul sub 2 add}
\psline(-1,-1)(3.5,3.5)

\psline[linestyle=dashed](2.1,0)(2.1,2.21)
\psline[linestyle=dashed](2.1,2.21)(2.21,2.21)
\psline[linestyle=dashed](2.21,2.21)(2.21,2.464)
\psline[linestyle=dashed](2.21,2.464)(2.464,2.464)
\psline[linestyle=dashed](2.464,2.464)(2.464,3.143)
\psline[linestyle=dashed](2.464,3.143)(3.143,3.143)
\psline[linestyle=dashed](3.143,3.143)(3.143,3.5)

\psline[linestyle=dashed](1.8,0)(1.8,1.64)
\psline[linestyle=dashed](1.8,1.64)(1.64,1.64)
\psline[linestyle=dashed](1.64,1.64)(1.64,1.409)
\psline[linestyle=dashed](1.64,1.409)(1.409,1.409)
\psline[linestyle=dashed](1.409,1.409)(1.409,1.167)
\psline[linestyle=dashed](1.409,1.167)(1.167,1.167)
\psline[linestyle=dashed](1.167,1.167)(1.167,1.028)
\psline[linestyle=dashed](1.167,1.028)(1.028,1.028)

\psline[linestyle=dashed](0.5,0)(0.5,1.25)
\psline[linestyle=dashed](0.5,1.25)(1.25,1.25)
\psline[linestyle=dashed](1.25,1.25)(1.25,1.062)

\psline[linestyle=dashed](-0.4,0)(-0.4,2.96)
\psline[linestyle=dashed](-0.4,2.96)(2.96,2.96)
\psline[linestyle=dashed](2.96,2.96)(2.96,3.4)

\psline[linestyle=dashed](1,0)(1,1)
\psline[linestyle=dashed](2,0)(2,2)
\uput[dr](2.1,0){$u_0$}
\uput[d](1.8,0){$u_0'$}
\uput[d](0.5,0){$u_0''$}
\uput[d](-0.4,0){$u_0'''$}
\uput[r](3.5,3.5){$y=x$}
\uput[u](2.7,3.8){$y=x^2-2x+2$}
\end{pspicture}


\end{document}

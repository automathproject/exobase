\uuid{BEor}
\exo7id{7685}
\titre{exo7 7685}
\auteur{mourougane}
\organisation{exo7}
\datecreate{2021-08-11}
\isIndication{false}
\isCorrection{false}
\chapitre{Sous-variété}
\sousChapitre{Sous-variété}
\module{Géométrie différentielle}
\niveau{L3}
\difficulte{}

\contenu{
\texte{
Soit l'ouvert $W =]0; 1[\times \Rr$ dans $\Rr^2$. On considère la surface $M$ de $\Rr^3$
paramétrée par 
$$\begin{array}{ccc}
 F: W&\to&\Rr ^3\\(u, v)&\mapsto& (u \cos v , u \sin v , v)
 \end{array}$$
}
\begin{enumerate}
    \item \question{Faire un dessin donnant l'allure de $M$.}
    \item \question{Calculer l'aire de la partie de $M$ comprise entre les plans d'équations $z = 0$ et $z = 2\pi$.}
    \item \question{Calculer, dans le paramétrage $F$, la courbure de Gauss $K$, la courbure
moyenne $H$, les courbures principales de la surface $M$.}
\end{enumerate}
}

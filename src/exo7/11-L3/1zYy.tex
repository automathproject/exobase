\uuid{1zYy}
\exo7id{6800}
\titre{exo7 6800}
\auteur{gijs}
\organisation{exo7}
\datecreate{2011-10-16}
\isIndication{false}
\isCorrection{false}
\chapitre{Champ de vecteurs}
\sousChapitre{Champ de vecteurs}
\module{Géométrie différentielle}
\niveau{L3}
\difficulte{}

\contenu{
\texte{
Soit $M \subset \Rr^3$ le cylindre défini par
l'équation $x^2 + y^2 = 1$. Dans la carte
$(\theta,z) \mapsto (\cos(\theta), \sin(\theta), z)$ de
$M$ on  donne le champ de vecteurs $X$ défini par~:
$$
X(\theta,z) =
\frac{\partial}{\partial\theta}{\Big\vert_{(\theta,z)}} +
\tfrac12 z(z^2-1) 
\frac{\partial}{\partial z}{\Big\vert_{(\theta,z)}}\ . $$
}
\begin{enumerate}
    \item \question{Dire pourquoi $X$ définit un champ de vecteurs sur
$M$.}
    \item \question{Esquisser le champ $X$ dans la carte et calculer son
flot $\phi_t$  sur $M$.}
    \item \question{Le champ $X$ est-il complet~?}
    \item \question{Calculer les points $(t,\theta,z)$ tel que
$\phi(t,\theta,z) = (\theta,z)$ {\bf sur le cylindre $M$}.}
\end{enumerate}
}

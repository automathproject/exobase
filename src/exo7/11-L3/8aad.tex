\uuid{8aad}
\exo7id{6795}
\titre{exo7 6795}
\auteur{gijs}
\organisation{exo7}
\datecreate{2011-10-16}
\isIndication{false}
\isCorrection{false}
\chapitre{Forme différentielle}
\sousChapitre{Forme différentielle}
\module{Géométrie différentielle}
\niveau{L3}
\difficulte{}

\contenu{
\texte{
Soit $M \subset \Rr^3$ le cylindre défini par
l'équation $x^2 + y^2 = 1$, soit $V\subset \Rr^3$
l'ensemble défini par $V = \{\,(x,y,z)\mid z\ge 0\ \&
\ (x/2)^2 + (2y)^2 + z^2 \le 1\,\}$, soit $K = M \cap V$,
et soit $\partial K$ le bord de $K$. Soit finalement
$\alpha$ la 1-forme sur $\Rr^3$ définie par $\alpha
= z^2x\,dy - z^2y\,dx + (xz-yz^3)\,dz$.
}
\begin{enumerate}
    \item \question{Calculer $d\alpha$.}
    \item \question{Exprimer $\alpha$ et $d\alpha$ dans la carte
$(\theta,z) \mapsto (\cos(\theta), \sin(\theta), z)$ de
$M$.}
    \item \question{Exprimer $K$ et $\partial K$ dans cette carte.}
    \item \question{Calculer séparément $\int_K d\alpha$ et
$\int_{\partial K} \alpha$ sans utiliser le théorème
de Stokes. Est ce que votre résultat confirme ce
théorème\,?}
\end{enumerate}
}

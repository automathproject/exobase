\uuid{C8KC}
\exo7id{7700}
\titre{exo7 7700}
\auteur{mourougane}
\organisation{exo7}
\datecreate{2021-08-11}
\isIndication{false}
\isCorrection{true}
\chapitre{Sous-variété}
\sousChapitre{Sous-variété}
\module{Géométrie différentielle}
\niveau{L3}
\difficulte{}

\contenu{
\texte{
Trouver les extrema de la fonction $f: (x,y,z)\mapsto xy$ sur la sphère unité.
}
\reponse{
On note $S$ la sphère unité de $\Rr ^3$.
La fonction $f$ est différentiable sur la sphère comme restriction d'une fonction polynomiale sur $\Rr ^3$.
La fonction $f$ continue sur le compact $S$ atteint ses extrema.
 En un point extremal, la différentielle de la fonction $f$ restreinte à l'espace tangent doit être nulle.
Le gradient de $f$ doit donc être normal à l'espace tangent.
On trouve que le vecteur 
$\left(\begin{array}{c}y\\x\\0\end{array}\right)$ doit être parallèle au vecteur $\left(\begin{array}{c}x\\y\\z\end{array}\right)$.
Comme $x$, $y$ et $z$ ne sont pas simultanément nuls sur $S$, $z=0$ et $x^2=y^2$.
Puisque $x^2+y^2+z^2=1$, $x=\pm 1/\sqrt{2}$ et $y=\pm 1/\sqrt{2}$.
En comparant les valeurs en ces quatre points, on trouve que
le minimum absolu de la fonction $f$ est donc $-1/2$ atteint au deux points $(1/\sqrt{2},-1/\sqrt{2},0)$ et
$(-1/\sqrt{2},1/\sqrt{2},0)$.
Le maximum absolu de la fonction $f$ est donc $1/2$ atteint au deux points $(1/\sqrt{2},1/\sqrt{2},0)$ et
$(-1/\sqrt{2},-1/\sqrt{2},0)$.
}
}

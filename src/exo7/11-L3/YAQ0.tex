\uuid{YAQ0}
\exo7id{2550}
\titre{exo7 2550}
\auteur{tahani}
\organisation{exo7}
\datecreate{2009-04-01}
\isIndication{false}
\isCorrection{false}
\chapitre{Sous-variété}
\sousChapitre{Sous-variété}
\module{Géométrie différentielle}
\niveau{L3}
\difficulte{}

\contenu{
\texte{
Soit $f:{\cal M}_n(\mathbb{R}) \rightarrow \mathbb{R}$ de classe
$C^\infty$ d\'efinie par $f(A)=det(A)$.
}
\begin{enumerate}
    \item \question{Montrer que $\lim_{\lambda \rightarrow 0}
\frac{det(I+\lambda X)-1}{\lambda}=tr(X)$ (penser au polyn\^ome
caract\'eristique). En d\'eduire $D_{Id_n}f(X)$.}
    \item \question{En remarquant que $\frac{det(A+\lambda X)-det(A)}{\lambda}$ est
\'egal \`a $det(A)\frac{det(I+\lambda A^{-1}X)-1}{\lambda}$, pour
$A$ inversible, calculer $D_Af(X)$ lorsque $A$ est inversible.}
    \item \question{Montrer que $Sl_n(\mathbb{R})$ est une sous-vari\'et\'e de
${\cal M}_n(\mathbb{R})$, de dimension $n^2-1$, dont l'espace
tangent en $Id$ est $\{X \in M_n(\mathbb{R}); tr(X)=0\}$.}
\end{enumerate}
}

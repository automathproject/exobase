\uuid{rCOT}
\exo7id{6793}
\auteur{gijs}
\organisation{exo7}
\datecreate{2011-10-16}
\isIndication{false}
\isCorrection{false}
\chapitre{Espace tangent, application linéaire tangente}
\sousChapitre{Espace tangent, application linéaire tangente}

\contenu{
\texte{
Soit $f:\Rr^n \to \Rr^p$ une fonction dérivable
et soit $M = f^{-1}(\mathbf{0})$. On suppose que $M$ est une
sous-variété de $\Rr^n$ de dimension $k$.
}
\begin{enumerate}
    \item \question{Donner la définition d'un  vecteur
tangent à $M$ au point $x \in M$. Donner la définition
d'un champ de vecteurs sur $M$.}
    \item \question{Soit $f$ une fonction sur $M$ et $\alpha$ une
1-forme sur $M$. Donner la définition de $df$ sans
utiliser une carte de $M$. Donner la définition de
$d\alpha$.}
    \item \question{Soit $X$ un vecteur tangent à $M$ au point $x\in M$.
Montrer que $\bigl(\ df|_x(X) \equiv\ \bigr)\ Xf =
\mathbf{0}$. En déduire que $X \in \operatorname{ker}(J(x))$
(où $J(x)$ est la matrice jacobienne de $f$ en $x$).}
    \item \question{Soit $X$ un vecteur tangent $\underline{ \text{à\ }
\Rr^n}$ au point $x\in M$, et supposons que $Xf  =
\mathbf{0}$, et que $\operatorname{rang}(J(x)) = p$.
Montrer que $k=n-p$ et que  $X$ est un vecteur tangent
$\underline{ \text{à }M}$.}
\end{enumerate}
}

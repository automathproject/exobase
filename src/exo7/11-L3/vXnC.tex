\uuid{vXnC}
\exo7id{6777}
\titre{exo7 6777}
\auteur{gijs}
\organisation{exo7}
\datecreate{2011-10-16}
\isIndication{false}
\isCorrection{false}
\chapitre{Champ de vecteurs}
\sousChapitre{Champ de vecteurs}
\module{Géométrie différentielle}
\niveau{L3}
\difficulte{}

\contenu{
\texte{
Soit $M\subset \Rr^n$ une sous-variété
de dimension $k$, et soit $X$ un champ de vecteurs sur
$M$. On définit $D = \{\,m\in M\mid X(m) \neq 0\,\}$ et
$S = \overline D = \text{ fermeture } D$. On vous demande
de démontrer l'énoncé : ``si $S$ est compact, alors
$X$ est complet''. Les questions suivantes peuvent vous
guider.
}
\begin{enumerate}
    \item \question{Pour $x \notin S$ trouver la courbe intégrale
maximale $\gamma : J_x \to M$ passant par $x$.}
    \item \question{Montrer que si $x\in S$, et si $\gamma : J \to M$
est une courbe intégrale passant par $x$, alors
$\forall\ t\in J\ :\ \gamma(t) \in S$.}
    \item \question{En utilisant la compacité de $S$, montrer que
$X$ est complet (sur $M$ !).}
\end{enumerate}
}

\uuid{S7z8}
\exo7id{5921}
\auteur{tumpach}
\organisation{exo7}
\datecreate{2010-11-11}
\isIndication{false}
\isCorrection{true}
\chapitre{Calcul d'intégrales}
\sousChapitre{Théorie}

\contenu{
\texte{
On dit qu'une partie $A$ de $\mathbb{R}$ est \emph{n\'egligeable}
si, pour tout nombre r\'eel $\varepsilon>0$, il existe une suite
$(I_{n})_{n\in\mathbb{N}}$ d'intervalles $I_{n} = ]a_{n},b_{n}[$
telle que~:
\begin{equation*}
A \subset \bigcup_{n\in\mathbb{N}}I_{n}\quad \text{et}\quad
\sum_{n\in\mathbb{N}}(b_{n}-a_{n})\leq\varepsilon.
\end{equation*}
}
\begin{enumerate}
    \item \question{Montrer qu'une r\'eunion d\'enombrable d'ensembles
n\'egligeables est un ensemble n\'egligeable.}
    \item \question{Montrer qu'une
fonction born\'ee $f~:[a,b]\rightarrow\mathbb{R}$ est int\'egrable
au sens de Riemann sur $[a,b]$ si et seulement si l'ensemble des
points o\`u $f$ n'est pas continue est \emph{n\'egligeable}.}
\reponse{
cf Andr\'e Gramain, \emph{Int\'egration}, p.~7, Hermann (1998).
}
\end{enumerate}
}

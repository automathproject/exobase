\uuid{cO0Q}
\exo7id{695}
\auteur{gourio}
\organisation{exo7}
\datecreate{2001-09-01}
\isIndication{false}
\isCorrection{false}
\chapitre{Continuité, limite et étude de fonctions réelles}
\sousChapitre{Fonction continue par morceaux}

\contenu{
\texte{
On dit qu'un ensemble $A$ de fonctions d\'{e}finies sur un intervalle $I=[a,b]$
de $\Rr$ est dense dans un ensemble $B$ si :
$$\forall f\in B,\forall \epsilon >0,\exists g\in A,\forall x\in I,\left|
f(x)-g(x)\right| <\epsilon . $$
Le cours dit par exemple que l'ensemble des fonctions en escaliers est dense
dans l'ensemble des fonctions continues par morceaux si $I=[a,b].$
Montrer que l'ensemble des fonctions continues affines par morceaux est
dense dans l'ensemble des fonctions continues sur un intervalle $I=[a,b]$.
}
}

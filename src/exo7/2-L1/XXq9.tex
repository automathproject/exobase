\uuid{XXq9}
\exo7id{3977}
\auteur{quercia}
\organisation{exo7}
\datecreate{2010-03-11}
\isIndication{false}
\isCorrection{true}
\chapitre{Dérivabilité des fonctions réelles}
\sousChapitre{Autre}

\contenu{
\texte{
Montrer que pour tout $x$ réel, il existe $a(x)$ unique tel que
$ \int_{t=x}^{a(x)} e^{t^2}\, d t=1$. Montrez que $a$ est indéfiniment
dérivable, et que son graphe est symétrique par rapport à la deuxième
bissectrice.
}
\reponse{
Si l'on pose $F(x)= \int_{t=0}^x e^{t^2}\, d t$, on constate que
$a(x)=F^{-1}(1+F(x))$ ce qui prouve l'existence, l'unicité et
le caractère $\mathcal{C}^\infty$ de~$a$. Pour la symétrie, il faut montrer que
$a(-a(x)) = -x$ soit $ \int_{t=-a(x)}^{-x} e^{t^2}\, d t=1$ ce qui est immédiat.
}
}

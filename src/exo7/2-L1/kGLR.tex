\uuid{kGLR}
\exo7id{4716}
\auteur{quercia}
\organisation{exo7}
\datecreate{2010-03-16}
\isIndication{false}
\isCorrection{true}
\chapitre{Suite}
\sousChapitre{Suite définie par une relation de récurrence}

\contenu{
\texte{
Soit $f : {[a,b]} \to {[a,b]}$ continue et la suite $(u_n)$ d{\'e}finie par
$u_0\in{[a,b]}$ et $u_{n+1} = f(u_n)$. Montrer que si $\lim(u_{n+1}-u_n) = 0$
alors la suite $(u_n)$ converge.
}
\reponse{
L'ensemble des valeurs d'adh{\'e}rence de la suite est un intervalle
dont tous les {\'e}l{\'e}ments sont points fixes par $f$.
S'il y a plusieurs valeurs d'adh{\'e}rence il faut passer de l'une {\`a} l'autre
avec une longueur de saut qui tend vers z{\'e}ro, on doit tomber sur point fixe
entre les deux, contradiction.
}
}

\uuid{Fx2u}
\exo7id{5922}
\titre{exo7 5922}
\auteur{tumpach}
\organisation{exo7}
\datecreate{2010-11-11}
\isIndication{false}
\isCorrection{true}
\chapitre{Calcul d'intégrales}
\sousChapitre{Théorie}
\module{Analyse}
\niveau{L1}
\difficulte{}

\contenu{
\texte{
Pour tout $n\in\mathbb{N}$, on d\'efinit
$f_{n}~:]0,1]\rightarrow\mathbb{R}$ par~: $f_{n}(x)= n e^{-nx}$.
Montrer que la suite $(f_{n})_{n\in\mathbb{R}}$ converge
simplement vers une fonction $f$ sur $]0,1]$ mais que
\begin{equation*}
\int_{0}^{1}\lim_{n\rightarrow+\infty}f_{n}(x)\,dx \quad\neq\quad
\lim_{n\rightarrow+\infty}\int_{0}^{1}f_{n}(x)\,dx.
\end{equation*}
V\'erifier que la convergence de $(f_{n})_{n\in\mathbb{N}}$ vers
$f$ n'est pas \emph{uniforme} sur $]0,1]$.
}
\reponse{
Pour tout $n\in\mathbb{N}$, on d\'efinit
$f_{n}:]0,1]\rightarrow\mathbb{R}$ par~: $f_{n}(x)= n e^{-nx}$.
Pour tout $x\in]0,1]$, on a $\lim_{n\rightarrow+\infty}f_{n}(x) =
\lim_{n\rightarrow+\infty}ne^{-nx} = 0$. On en d\'eduit que la
suite de fonctions $f_{n}$ converge ponctuellement (ou
\emph{simplement}) vers la fonction identiquement nulle
$f\equiv0$. On a $\int_{0}^{1}f(x)\,dx = 0$ mais
\begin{equation*}
\int_{0}^{1} f_{n}(x)\,dx = 1- e^{-n},
\end{equation*}
et $\lim_{n\rightarrow+\infty}\int_{0}^{1}f_{n}(x)\,dx = 1$. La
suite $(f_{n})_{n\in\mathbb{R}}$ ne converge pas uniform\'ement
vers $f$ sur $]0,1]$, car pour tout $\varepsilon>0$, et pour tout
$n\in\mathbb{N}$, on a~:
\begin{equation*}
\sup_{]0,
-\frac{1}{n}\log\left(\frac{\varepsilon}{n}\right)[}|f_{n}(x)-f(x)|
> \varepsilon.
\end{equation*}
}
}

\uuid{HrLH}
\exo7id{4463}
\titre{exo7 4463}
\auteur{quercia}
\organisation{exo7}
\datecreate{2010-03-14}
\isIndication{false}
\isCorrection{false}
\chapitre{Série numérique}
\sousChapitre{Autre}
\module{Analyse}
\niveau{L1}
\difficulte{}

\contenu{
\texte{
On considère la suite $(u_n)$ définie par : $0 < u_0 < 1$
et $\forall\ n\in\N,\ u_{n+1} = u_n - u_n^2$.
}
\begin{enumerate}
    \item \question{Montrer que la suite $(u_n)$ converge. Quelle est sa limite ?}
    \item \question{Montrer que la série de terme général $u_n^2$ converge.}
    \item \question{Montrer que les séries de termes généraux
    $\ln\left(\frac{u_{n+1}}{u_n}\right)$ et $u_n$ divergent.}
    \item \question{Montrer que $u_n < \frac 1{n+1}$ et que la suite $(nu_n)$ est croissante.
    On note $\ell$ sa limite.}
    \item \question{On pose $u_n = \frac{\ell-v_n}n$. Montrer que la série de terme général
    $v_{n+1}-v_n$ converge.}
    \item \question{En déduire que $u_n$ est équivalent à $\frac 1n$.}
\end{enumerate}
}

\uuid{TnHi}
\exo7id{1935}
\titre{exo7 1935}
\auteur{gineste}
\organisation{exo7}
\datecreate{2001-11-01}
\isIndication{false}
\isCorrection{false}
\chapitre{Série numérique}
\sousChapitre{Série à  termes positifs}
\module{Analyse}
\niveau{L1}
\difficulte{}

\contenu{
\texte{
Soit $(u_n)$ une suite de réels strictement positifs, on suppose que $ \displaystyle \lim(\frac{u_{n+1}}{u_n})=1 $ et que $$ \frac{u_{n+1}}{u_n}=1 - \frac{\alpha}{n} + O(\frac{1}{n^{\beta}}) \mbox{ , o\`u } \alpha > 0 \ \  \beta > 1.$$
On pose $v_n=n^{\alpha}u_n$. Etudier  $ \displaystyle  \frac{v_{n+1}}{v_n} $ et montrer que $(v_n)$ a une limite finie. \
Application : Etudier la série de terme général $$u_n = \sqrt{n!} \sin 1 \sin \frac{1}{\sqrt{2}} \cdots \sin \frac{1}{\sqrt{n}} . $$
}
}

\uuid{9Bvj}
\exo7id{738}
\auteur{bodin}
\organisation{exo7}
\datecreate{2001-11-01}
\video{MqjLEkOQD3w}
\isIndication{true}
\isCorrection{true}
\chapitre{Dérivabilité des fonctions réelles}
\sousChapitre{Autre}

\contenu{
\texte{
Soient $f,g :[a , b] \longrightarrow \R$ deux
fonctions continues
 sur $[a, b]$ ($a < b$) et d\'erivables sur $]a , b[.$ On suppose que
 $g' (x) \neq 0$ pour tout $x \in ]a , b[.$
}
\begin{enumerate}
    \item \question{Montrer que $g (x) \neq g (a)$ pour tout $x \in ]a , b[.$}
\reponse{Supposons par l'absurde, qu'il existe $x_{0}\in ]a,b]$  tel
que $g (x_{0}) = g (a).$ Alors en appliquant le th\'eor\`eme de
Rolle \`a la restriction de $g$ \`a l'intervalle $[a,x_{0}]$ (les
hypoth\`eses \'etant clairement v\'erifi\'ees), on en d\'eduit
qu'il existe $c \in ]a,x_{0}[$ tel que $g' (c) = 0,$ ce qui
contredit les hypoth\`eses faites sur $g.$ Par cons\'equent on a
d\'emontr\'e que $g (x) \neq g (a)$ pour tout $x \in ]a,b].$}
    \item \question{Posons $p = \frac{f (b) - f (a)}{g (b) - g (a)}$ et
consid\'erons  la fonction $h (x) = f
 (x) - p g (x)$ pour $x \in [a , b].$
 Montrer que $h$ v\'erifie les hypoth\`eses du th\'eor\`eme de Rolle
 et en d\'eduire qu'il existe un nombre r\'eel $c \in ]a , b[$ tel que
 $$ \frac{f (a) - f (b)}{g (a) - g (b)} = \frac{f' (c)}{g' (c)}.$$}
\reponse{D'apr\`es la question pr\'ec\'edente, on a en particulier  $g (b)
\neq g (a)$ et donc $p$ est un nombre r\'eel bien d\'efini et $h =
f - p \cdot g$ est alors une fonction continue sur $[a,b]$ et
d\'erivable sur $]a,b[.$ Un calcul simple montre que $h (a) = h
(b).$ D'apr\`es le th\'eor\`eme de Rolle il en r\'esulte  qu'il
existe $c \in ]a,b[$ tel que $h' (c) = 0.$ Ce qui implique la
relation requise.}
    \item \question{On suppose que $\lim_{x \to b^{-}} \frac{f' (x)}{g' (x)} =
\ell,$ o\`u $\ell $ est un nombre r\'eel.
 Montrer que  $$ \lim_{x \to b^{-}} \frac{f (x) - f (b)}{g (x) - g (b)} = \ell.$$}
\reponse{Pour chaque $x \in ]a,b[,$ on peut appliquer
la question 2. aux
 restrictions de $f$ et $g$ \`a l'intervalle $[x,b],$ on en d\'eduit qu'il
 existe un point $c (x) \in ]x,b[,$ d\'ependant de $x$ tel que
 $$\frac{f (x) - f (b)}{g (x) - g (b)} = \frac{f' (c (x))}{g' (c (x))} .
 \leqno (*)$$
 Alors, comme $\lim_{x \to b^{-}} \frac{f' (t)}{g' (t)} = \ell$ et
 $\lim_{x \to b^{-}} c (x) = b,$ (car $c(x) \in ]x,b[$) on en d\'eduit en passant \`a la
 limite dans $(*)$ que
 $$ \lim_{x \to b^{-}} \frac{f (x) - f (b)}{g (x) - g (b)} = \ell.$$
 Ce r\'esultat est connu sous le nom de ``r\`egle de
 l'H\^opital''.}
    \item \question{\emph{Application.} Calculer la limite suivante:
 $$ \lim_{x \to 1^{-}} \frac{\Arccos x}{\sqrt{1- x^{2}}}.$$}
\reponse{Consid\'erons les deux fonctions $f
 (x) = \Arccos x$ et $g (x) = \sqrt{1-x^2}$ pour $x \in [0,1].$ Ces fonctions sont continues sur  $[0,1]$ et
 d\'erivables sur $]0,1[$ et $f' (x) = - 1 \slash \sqrt{1-x^2}$,
 $g' (x) = - x \slash \sqrt{1-x^2} \neq 0$ pour tout $ x
 \in ]0,1[.$ En appliquant les r\'esultats de la question 3., on en
 d\'eduit que
 $$ \lim_{x \to 1^{-}} \frac{\Arccos x}{\sqrt{1-x^2}} = \lim_{x \to 1^{-}} \frac{\frac{-1}{\sqrt{1-x^2}}} {\frac{-x}{\sqrt{1-x^2}}}= \lim_{x \to 1^{-}} \frac{1}{x} = 1.$$}
\indication{\begin{enumerate}
 \item Raisonner par l'absurde et appliquer le th\'eor\`eme de Rolle.
 \item Calculer $h(a)$ et $h(b)$.
 \item Appliquer la question 2. sur l'intervalle $[x,b]$.
 \item Calculer $f'$ et $g'$.
\end{enumerate}}
\end{enumerate}
}

\uuid{Gs2x}
\exo7id{5150}
\titre{exo7 5150}
\auteur{rouget}
\organisation{exo7}
\datecreate{2010-06-30}
\isIndication{false}
\isCorrection{true}
\chapitre{Série numérique}
\sousChapitre{Autre}
\module{Analyse}
\niveau{L1}
\difficulte{}

\contenu{
\texte{
Soient $n\in\Nn^*$ et $a_1$, $a_2$,..., $a_n$, $b_1$, $b_2$,..., $b_n$, $2n$ réels. Montrer que

$$|\sum_{k=1}^{n}a_kb_k|\leq\sum_{k=1}^{n}|a_k|.|b_k|\leq\sqrt{\sum_{k=1}^{n}a_k^2}\sqrt{\sum_{k=1}^{n}b_k^2}.$$

(Indication. Considérer le polynôme $f(x)=\sum_{k=1}^{n}(a_k+b_kx)^2$, développer puis ordonner suivant les puissances
décroissantes puis utiliser, dans le cas général, les connaissances sur le second degré). Retrouver alors le résultat de l'exercice \ref{exo:suprou4bis}.
}
\reponse{
Pour $x$ réel, posons $f(x)=\sum_{k=1}^{n}(a_k+b_kx)^2$. On remarque que pour tout réel $x$,
$f(x)\geq0$. En développant les $n$ carrés, on obtient,
$$f(x)=\sum_{k=1}^{n}(b_k^2x^2+2a_kb_kx+a_k^2)=(\sum_{k=1}^{n}b_k^2)x^2+2(\sum_{k=1}^{n}a_kb_k)x+(\sum_{k=1}^{n}a_k^2).$$

\begin{itemize}
\item[\textbf{1er cas.}] Si $\sum_{k=1}^{n}b_k^2\neq0$, $f$ est un trinôme du second degré de signe constant sur $\Rr$.
Son discriminant réduit est alors négatif ou nul. Ceci fournit

$$0\geq\Delta'=(\sum_{k=1}^{n}a_kb_k)^2-(\sum_{k=1}^{n}b_k^2)(\sum_{k=1}^{n}a_k^2),$$

et donc

$$\left|\sum_{k=1}^{n}a_kb_k\right|\leq\sqrt{\sum_{k=1}^{n}a_k^2}\sqrt{\sum_{k=1}^{n}b_k^2}.$$

\item[\textbf{2ème cas.}] Si $\sum_{k=1}^{n}b_k^2=0$, alors tous les $b_k$ sont nuls et l'inégalité est immédiate.\\
Finalement, dans tous les cas,
\begin{center}
\shadowbox{
$\left|\sum_{k=1}^{n}a_kb_k\right|\leq\sqrt{\sum_{k=1}^{n}a_k^2}\sqrt{\sum_{k=1}^{n}b_k^2}.$
}
\end{center}
\end{itemize}

Cette inégalité est encore valable en remplaçant les $a_k$ et les $b_k$ par leurs valeurs absolues, ce qui fournit
les inégalités intermédiaires.

Retrouvons alors l'inégalité de l'exercice \ref{exo:suprou4bis}. Puisque les $a_k$ sont strictement positifs, on peut écrire~:

$$\left(\sum_{i=1}^{n}a_i\right)\left(\sum_{i=1}^{n}\frac{1}{a_i}\right)=\left(\sum_{i=1}^{n}\sqrt{a_i}^2\right)\left(
\sum_{i=1}^{n}\sqrt{\frac{1}{a_i}}^2\right)\geq\left(\sum_{i=1}^{n}\sqrt{a_i}\sqrt{\frac{1}{a_i}}\right)^2=n^2.$$
}
}

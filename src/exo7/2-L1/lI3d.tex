\uuid{lI3d}
\exo7id{5919}
\titre{exo7 5919}
\auteur{tumpach}
\organisation{exo7}
\datecreate{2010-11-11}
\isIndication{false}
\isCorrection{true}
\chapitre{Calcul d'intégrales}
\sousChapitre{Théorie}
\module{Analyse}
\niveau{L1}
\difficulte{}

\contenu{
\texte{
Montrer qu'une fonction \emph{continue} sur $[a,b]$ est
Riemann-int\'egrable sur $[a,b]$.
}
\reponse{
Une fonction $f$ continue sur $[a, b]$ est uniform\'ement continue
sur $[a, b]$. En particulier, pour tout $\varepsilon>0$, il existe
$n>0$ tel que
\begin{equation*}
|x - y| < \left(\frac{b-a}{n}\right) \Rightarrow |f(x) - f(y)|<
\varepsilon.
\end{equation*}
Soit $\sigma=\{a_{0}=a < \dots < a_{n} = b\}$ la subdivision
r\'eguli\`ere de $[a,b]$, de pas $\left(\frac{b-a}{n}\right)$. On
a~:
\begin{equation*}
\sup_{]a_{k-1}, a_{k}[} f - \inf_{]a_{k-1}, a_{k}[} f \leq
2\varepsilon.
\end{equation*}
Il vient alors~:
\begin{equation*}
\overline{S}_{f}^{\sigma} - \underline{S}_{f}^{\sigma} \leq
\left(\frac{b-a}{n}\right)\sum_{k=1}^{n} 2 \varepsilon = (b-a)
2\varepsilon,
\end{equation*}
ce qui permet de conclure gr\^ace au th\'eor\`eme de l'introduction que $f$
est Riemann-int\'egrable sur $[a,b]$.
}
}

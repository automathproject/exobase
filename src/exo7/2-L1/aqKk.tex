\uuid{aqKk}
\exo7id{609}
\titre{exo7 609}
\auteur{bodin}
\organisation{exo7}
\datecreate{1998-09-01}
\video{db5yEXsEbYc}
\isIndication{true}
\isCorrection{true}
\chapitre{Continuité, limite et étude de fonctions réelles}
\sousChapitre{Limite de fonctions}
\module{Analyse}
\niveau{L1}
\difficulte{}

\contenu{
\texte{

}
\begin{enumerate}
    \item \question{D\'emontrer que $\displaystyle{ \lim_{x\rightarrow 0}\frac{\sqrt{1+x}-\sqrt{1-x}}{ x}=1}$.}
    \item \question{Soient $m,n$ des entiers positifs. \'Etudier $\displaystyle{\lim_{x\rightarrow 0}\frac{\sqrt{1+x^m}-
\sqrt{1-x^m}}{ x^n}}$.}
    \item \question{D\'emontrer que $\displaystyle{ \lim_{x\rightarrow 0}\frac{1}{ x}(\sqrt{1+x+x^2}-1)=
\frac{1}{ 2}}$.}
\reponse{
G\'en\'eralement pour calculer des limites faisant intervenir des sommes de racines carr\'ees,  il est utile de faire intervenir ``l'expression conjugu\'ee": 
$$\sqrt a - \sqrt b = \frac{(\sqrt a - \sqrt b)(\sqrt a + \sqrt b)}{\sqrt a + \sqrt b} = \frac{a-b}{\sqrt a + \sqrt b}.$$
Les racines au num\'erateur ont ``disparu" en utilisant l'identit\'e
$(x-y)(x+y) = x^2-y^2$.

Appliquons ceci sur un exemple :
\begin{align*}
 f(x) &= 
  \frac{\sqrt{1+x^m}-\sqrt{1-x^m}}{x^n} \\
     &=  \frac{(\sqrt{1+x^m}-\sqrt{1-x^m})(\sqrt{1+x^m}+\sqrt{1-x^m})}{x^n(\sqrt{1+x^m}+\sqrt{1-x^m})} \\
   &= \frac{1+x^m-(1-x^m)}{x^n(\sqrt{1+x^m}+\sqrt{1-x^m})}  \\
   &= \frac{2x^m}{x^n(\sqrt{1+x^m}+\sqrt{1-x^m})}  \\
   &= \frac{2x^{m-n}}{\sqrt{1+x^m}+\sqrt{1-x^m}}  \\
\end{align*}
Et nous avons 
$$ \lim_{x \rightarrow 0} \frac 2 {\sqrt{1+x^m}+\sqrt{1-x^m}} = 1.$$
Donc l'\'etude de la limite de $f$ en $0$ est la m\^eme que celle de la fonction $x \mapsto x^{m-n}$.

Distinguons plusieurs cas pour la limite de $f$ en $0$.
\begin{itemize}
  \item Si $m > n$ alors $x^{m-n}$, et donc $f(x)$, tendent vers $0$.
  \item Si $m=n$ alors $x^{m-n}$ et $f(x)$ tendent vers $1$.
  \item Si $m < n$ alors $x^{m-n} = \frac 1 {x^{n-m}}  = \frac 1 {x^k}$ avec $k = n-m$ un exposant positif. Si $k$ est pair alors les limites 
\`a droite et \`a gauche de $\frac 1{x^k}$ sont $+\infty$. 
Pour $k$ impair la limite \`a droite vaut $+\infty$ et la limite \`a gauche vaut $-\infty$. Conclusion pour $k=n-m>0$ pair, la limite de $f$ en $0$ vaut $+\infty$ et pour $k = n-m>0$ impair $f$ \emph{n'a pas de limite en $0$} car les limites \`a droite et \`a gauche ne sont pas \'egales.
\end{itemize}
}
\indication{Utiliser l'expression conjugu\'ee.}
\end{enumerate}
}

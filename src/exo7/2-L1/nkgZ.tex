\uuid{nkgZ}
\exo7id{5391}
\auteur{rouget}
\organisation{exo7}
\datecreate{2010-07-06}
\isIndication{false}
\isCorrection{true}
\chapitre{Continuité, limite et étude de fonctions réelles}
\sousChapitre{Continuité : théorie}

\contenu{
\texte{
Soit $f$ une fonction continue et périodique sur $\Rr$ à valeurs dans $\Rr$, admettant une limite réelle quand $x$ tend vers $+\infty$. Montrer que $f$ est constante.
}
\reponse{
Soit $T$  une période strictement positive de $f$. On note $\ell$ la limite de $f$ en $+\infty$.

Soit $x$ un réel. $\forall n\in\Nn,\;f(x)=f(x+nT)$ et quand $n$ tend vers $+\infty$, on obtient~:
 
$$f(x)=\lim_{n\rightarrow +\infty}f(x+nT)=\ell.$$ 

Ainsi, $\forall x\in\Rr,\;f(x)=\ell$ et donc, $f$ est constante sur $\Rr$.
}
}

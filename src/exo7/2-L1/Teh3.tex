\uuid{Teh3}
\exo7id{5421}
\titre{exo7 5421}
\auteur{rouget}
\organisation{exo7}
\datecreate{2010-07-06}
\isIndication{false}
\isCorrection{true}
\chapitre{Dérivabilité des fonctions réelles}
\sousChapitre{Théorème de Rolle et accroissements finis}
\module{Analyse}
\niveau{L1}
\difficulte{}

\contenu{
\texte{
Soient $f$ et $g$ deux fonctions continues sur $[a,b]$ et dérivables sur $]a,b[$.

Soit $\begin{array}[t]{cccc}
\Delta~:&[a,b]&\rightarrow&\Rr\\
 &x&\mapsto&\left|
 \begin{array}{ccc}
 f(a)&f(b)&f(x)\\
 g(a)&g(b)&g(x)\\
 1&1&1
 \end{array}
 \right|
\end{array}$.
}
\begin{enumerate}
    \item \question{Montrer que $\Delta$ est continue sur $[a,b]$, dérivable sur $]a,b[$ et calculer sa dérivée.}
    \item \question{En déduire qu'il existe $c$ dans $]a,b[$ tel que $(g(b)-g(a))f'(c)=(f(b)-f(a))g'(c)$.}
\reponse{
En pensant à l'expression développée de $\Delta$, on voit que $\Delta$ est continue sur $[a,b]$, dérivable sur $]a,b[$ et vérifie $\Delta(a)=\Delta(b)(=0)$ (un déterminant ayant deux colonnes identiques est nul).

Donc, d'après le théorème de \textsc{Rolle}, $\exists c\in]a,b[/\;\Delta'(c)=0$.

Mais, pour $x\in]a,b[$, $\Delta'(x)=f'(x)(g(a)-g(b))-g'(x)(f(a)-f(b))$ (dérivée d'un déterminant). L'égalité $\Delta'(c)=0$ s'écrit~: $f'(c)(g(b)-g(a))=g'(c)(f(b)-f(a))$ ce qu'il fallait démontrer.

(Remarque. Ce résultat généralise le théorème des accroissements finis ($g=Id$ \og~est~\fg~le théorème des accroissements finis.))
}
\end{enumerate}
}

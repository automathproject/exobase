\uuid{4mTF}
\exo7id{4486}
\auteur{quercia}
\organisation{exo7}
\datecreate{2010-03-14}
\isIndication{false}
\isCorrection{true}
\chapitre{Série numérique}
\sousChapitre{Autre}

\contenu{
\texte{
Soit $P(n)=\max\{p \text{ premier},\, p\mid n\}$. Montrer que $\sum_{n}\frac{1}{nP(n)}$ converge.
}
\reponse{
Soit $(p_0,p_1,\dots)$ la suite croissante des nombres premiers
et $S_k = \sum_{P(n)\le k}\frac1n$.
On a $S_k = S_{k-1}\sum_{i=0}^\infty \frac1{p_k^i} = \frac{p_k}{p_k-1}S_{k-1}$,
ce qui prouve que $S_k$ est fini. La série demandée est
$\frac{S_0}{p_0}+ \sum_{k=1}^\infty \frac{S_k-S_{k-1}}{p_k} =
\frac{S_0}{p_0}+ \sum_{k=1}^\infty\frac{S_k}{p_k^2}$.

Montrons que $S_k\le2\sqrt{p_k}$, ceci prouvera la convergence. C'est vrai pour $k=0$ et $k=1$, et si
c'est vrai pour $k-1$ avec $k\ge 2$ alors on obtient $S_k\le 2\sqrt{p_k}\sqrt{\frac{p_kp_{k-1}}{(p_k-1)^2}}
\le 2\sqrt{p_k}\sqrt{\frac{p_k(p_k-2)}{(p_k-1)^2}} \le 2\sqrt{p_k}$.

Remarque~: on a en réalité $S_k\sim e^\gamma\ln(p_k)$ où $\gamma$ est la constante
d'Euler (formule de Mertens).
}
}

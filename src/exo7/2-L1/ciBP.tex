\uuid{ciBP}
\exo7id{5315}
\auteur{rouget}
\organisation{exo7}
\datecreate{2010-07-04}
\isIndication{false}
\isCorrection{true}
\chapitre{Fonctions circulaires et hyperboliques inverses}
\sousChapitre{Fonctions circulaires inverses}

\contenu{
\texte{
Montrer que $\sum_{k=0}^{n-1}\cotan^2(\frac{\pi}{2n}+\frac{k\pi}{n})= n(n-1)$. (Indication. Poser $x_k=\cotan^2(\frac{\pi}{2n}+\frac{k\pi}{n})$ puis trouver un polynôme dont les $x_k$ sont les racines.)
}
\reponse{
Il faut prendre garde au fait que les nombres $x_k=\cotan^2(\frac{\pi}{2n}+\frac{k\pi}{n})$ ne sont pas nécessairement deux à deux distincts.

\begin{itemize}
\item[1er cas.] Si $n$ est pair, posons $n=2p$, $p\in\Nn^*$.

\begin{align*}\ensuremath
S_n&=\sum_{k=0}^{p-1}\cotan^2(\frac{\pi}{4p}+\frac{k\pi}{2p})+\sum_{k=p}^{2p-1}\cotan^2(\frac{\pi}{4p}+\frac{k\pi}{2p})\\
 &=\sum_{k=0}^{p-1}\cotan^2(\frac{\pi}{4p}+\frac{k\pi}{2p})+\sum_{k=0}^{p-1}\cotan^2(\frac{\pi}{4p}+\frac{(2p-1-k)\pi}{2p})
\end{align*}
 
Or, $\cotan^2(\frac{\pi}{4p}+\frac{(2p-1-k)\pi}{2p})=\cotan^2(\pi-\frac{\pi}{4p}-\frac{k\pi}{2p})=\cotan^2(\frac{\pi}{4p}+\frac{k\pi}{2p})$ et donc $S_n=2\sum_{k=0}^{p-1}\cotan^2(\frac{\pi}{4p}+\frac{k\pi}{2p})$.

Mais cette fois ci, 

$$0\leq k\leq p-1\Rightarrow 0<\frac{\pi}{4p}+\frac{k\pi}{2p}\leq\frac{\pi}{4p}+\frac{(p-1)\pi}{2p}=\frac{(2p-1)\pi}{4p}<\frac{2p\pi}{4p}=\frac{\pi}{2}.$$

et comme, la fonction $x\mapsto\cotan^2x$ est strictement décroissante sur $]0,\frac{\pi}{2}[$, les $x_k$, $0\leq k\leq p-1$, sont deux à deux distincts.

Pour $0\leq k\leq p-1$, posons $y_k=\cotan(\frac{\pi}{4p}+\frac{k\pi}{2p})$.

\begin{align*}\ensuremath
y_k&=i\frac{e^{(2k+1)i\pi/4p}+1}{e^{(2k+1)i\pi/4p}-1}\Rightarrow e^{(2k+1)i\pi/4p}(y-k-i)=y_k+i\\
 &\Rightarrow(y_k+i)^{2p}=e^{(2k+1)i\pi}(y_k-i)^{2p}=(-1)^{2k+1}(y_k-i)^{2p}=-(y_k-i)^{2p}\\
 &\Rightarrow(y_k+i)^{2p}+(y_k-i)^{2p}=0\Rightarrow2(y_k^{2p}-C_{2p}^2y_k^{2p-2}+...+(-1)^p)=0\\
 &\Rightarrow x_k^p-C_{2p}^2x_k^{p-1}+...+(-1)^p=0.
\end{align*}

Les $p$ nombres deux à deux distincts $x_k$ sont racines de l'équation de degré $p$~:~$z^p-C_{2p}^{2}z^{p-1}+...+(-1)^p=0$ qui est de degré $p$. On en déduit que 

$$S_n=2\sum_{k=0}^{p-1}x_k=2C_{2p}^{2}=n(n-1).$$

\item[2ème cas.] Si $n$ est impair, posons $n=2p+1$, $p\in\Nn$.

\begin{align*}\ensuremath
S_n&=\sum_{k=0}^{p-1}\cotan^2(\frac{\pi}{2(2p+1)}+\frac{k\pi}{2p+1})+\cotan^2\frac{\pi}{2}+\sum_{k=p+1}^{2p}\cotan^2(\frac{\pi}{2(2p+1)}+\frac{k\pi}{2p+1})\\
 &=2\sum_{k=0}^{p-1}\cotan^2(\frac{\pi}{2(2p+1)}+\frac{k\pi}{2p+1})
\end{align*}

La même démarche amène alors à $S_n=2C_{2p+1}^{2}=n(n-1)$.
\end{itemize}

Dans tous les cas, 

$$\sum_{k=0}^{n-1}\cotan^2(\frac{\pi}{2n}+\frac{k\pi}{n})=n(n-1).$$
}
}

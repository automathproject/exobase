\uuid{dICU}
\exo7id{5697}
\titre{exo7 5697}
\auteur{rouget}
\organisation{exo7}
\datecreate{2010-10-16}
\isIndication{false}
\isCorrection{true}
\chapitre{Série numérique}
\sousChapitre{Série à  termes positifs}
\module{Analyse}
\niveau{L1}
\difficulte{}

\contenu{
\texte{
Soit $(u_n)_{n\in\Nn}$ une suite positive telle que la série de terme général $u_n$ converge. Etudier la nature de la série de terme général $\frac{\sqrt{u_n}}{n}$.
}
\reponse{
Pour $n\in\Nn^*$, on a $\left(\sqrt{u_n}-\frac{1}{n}\right)^2$ et donc $0\leqslant\frac{\sqrt{u_n}}{n}\leqslant\frac{1}{2}\left(u_n+\frac{1}{n^2}\right)$. Comme la série terme général $\frac{1}{2}\left(u_n+\frac{1}{n^2}\right)$ converge, la série de terme général $\frac{\sqrt{u_n}}{n}$  converge.
}
}

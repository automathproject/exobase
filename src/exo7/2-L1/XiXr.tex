\uuid{XiXr}
\exo7id{5094}
\titre{exo7 5094}
\auteur{rouget}
\organisation{exo7}
\datecreate{2010-06-30}
\isIndication{false}
\isCorrection{true}
\chapitre{Fonctions circulaires et hyperboliques inverses}
\sousChapitre{Fonctions hyperboliques et hyperboliques inverses}
\module{Analyse}
\niveau{L1}
\difficulte{}

\contenu{
\texte{

}
\begin{enumerate}
    \item \question{Montrer que pour tout réel $x$ non nul, on a~:~$\tanh x=\frac{2}{\tanh(2x)}-\frac{1}{\tanh x}$.}
    \item \question{En déduire la valeur de $u_n=2^0\tanh(2^0x)+2^1\tanh(2^1x)+ \cdots +2^{n}\tanh(2^{n}x)$ pour $n$ entier naturel non nul et $x$ réel
non nul donnés puis calculer la limite de $(u_n)$.}
\reponse{
On a vu au \ref{exo:suprou3} que pour tout réel $x$, $\tanh(2x)=\frac{2\tanh x}{1+\tanh^2x}$ ce qui s'écrit pour
$x$ non nul~:~$\frac{1+\tanh^2x}{\tanh x}=\frac{2}{\tanh(2x)}$ ou encore $\tanh x+\frac{1}{\tanh x}=\frac{2}{\tanh(2x)}$
ou finalement

\begin{center}
\shadowbox{
$\forall x\in\Rr^*,\;\tanh x=\frac{2}{\tanh(2x)}-\frac{1}{\tanh x}.$
}
\end{center}
Soient $n$ un entier naturel non
nul et $x$ un réel non nul. D'après ce qui précède,

$$u_n
=\sum_{k=0}^{n}2^k\tanh(2^kx)=\sum_{k=0}^{n}\left(\frac{2^{k+1}}{\tanh(2^{k+1}x)}-\frac{2^{k}}{\tanh(2^{k}x)}\right)
=\sum_{k=1}^{n+1}\frac{2^{k}}{\tanh(2^{k}x)}-\sum_{k=0}^{n}\frac{2^{k}}{\tanh(2^{k}x)}=\frac{2^{n+1}}{\tanh(2^{n+1}x)}
-\frac{1}{\tanh x}.$$
Ensuite, pour $x>0$, $\tanh(2^{n+1}x)$ tend vers $1$ quand $n$ tend vers l'infini. Donc $u_n$ tend vers $+\infty$ quand
$n$ tend vers $+\infty$ si $x>0$ et vers $-\infty$ quand $n$ tend vers $+\infty$ si $x<0$.
}
\end{enumerate}
}

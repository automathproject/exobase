\uuid{qT7v}
\exo7id{5447}
\titre{exo7 5447}
\auteur{rouget}
\organisation{exo7}
\datecreate{2010-07-10}
\isIndication{false}
\isCorrection{true}
\chapitre{Calcul d'intégrales}
\sousChapitre{Somme de Riemann}
\module{Analyse}
\niveau{L1}
\difficulte{}

\contenu{
\texte{
Soit $f$ une fonction de classe $C^2$ sur $[0,1]$. Déterminer le réel $a$ tel que :
 
$$\int_{0}^{1}f(t)\;dt-\frac{1}{n}\sum_{k=1}^{n-1}f(\frac{k}{n})\underset{n\rightarrow+\infty}{=}\frac{a}{n}+o(\frac{1}{n}).$$
}
\reponse{
Supposons $f$ de classe $C^2$ sur $[0,1]$.
Soit $F$ une primitive de $f$ sur $[0,1]$. Soit $n$ un entier naturel non nul.

$$u_n=\int_{0}^{1}f(t)\;dt-\frac{1}{n}\sum_{k=0}^{n-1}f(\frac{k}{n})=\sum_{k=0}^{n-1}(\int_{k/n}^{(k+1)/n}f(t)\;dt-\frac{1}{n}f(\frac{k}{n}))=\sum_{k=0}^{n-1}(F(\frac{k+1}{n})-F(\frac{k}{n})-\frac{1}{n}F'(\frac{k}{n})).$$

$f$ est de classe $C^2$ sur le segment $[0,1]$. Par suite, $F^{(3)}=f''$ est définie et bornée sur ce segment. En notant $M_2$ la borne supérieure de $|f''|$ sur $[0,1]$, l'inégalité de \textsc{Taylor}-\textsc{Lagrange} à l'ordre 3 appliquée à $F$ sur le segment $[0,1]$ fournit

$$\left|F(\frac{k+1}{n})-F(\frac{k}{n})-\frac{1}{n}F'(\frac{k}{n})-\frac{1}{2n^2}F''(\frac{k}{n})\right|
\leq\frac{(1/n)^3M_2}{6},$$

et donc,
 
\begin{align*}\ensuremath
\left|\sum_{k=0}^{n-1}[F(\frac{k+1}{n})-F(\frac{k}{n})-\frac{1}{n}F'(\frac{k}{n})-\frac{1}{2n^2}F''(\frac{k}{n})]
\right|&\leq\sum_{k=0}^{n-1}|F(\frac{k+1}{n})-F(\frac{k}{n})-\frac{1}{n}F'(\frac{k}{n})-\frac{1}{2n^2}F''(\frac{k}{n})|\\
 &\leq\sum_{k=0}^{n-1}\frac{(1/n)^3M_2}{6}=\frac{M_2}{6n^2}.
\end{align*}

Ainsi, $\sum_{k=0}^{n-1}[F(\frac{k+1}{n})-F(\frac{k}{n})-\frac{1}{n}F'(\frac{k}{n})-\frac{1}{2n^2}F''(\frac{k}{n})]=O(\frac{1}{n^2})$, ou encore $\sum_{k=0}^{n-1}[F(\frac{k+1}{n})-F(\frac{k}{n})-\frac{1}{n}F'(\frac{k}{n})-\frac{1}{2n^2}F''(\frac{k}{n})]=o(\frac{1}{n})$, ou enfin,

$$u_n=\sum_{k=0}^{n-1}\frac{1}{2n^2}F''(\frac{k}{n})+o(\frac{1}{n}).$$

Maintenant, 

$$\sum_{k=0}^{n-1}\frac{1}{2n^2}F''(\frac{k}{n})=\frac{1}{2n}.\frac{1}{n}\sum_{k=0}^{n-1}f'(\frac{k}{n}).$$

Or, la fonction $f'$ est continue sur le segment $[0,1]$. Par suite, la somme de \textsc{Riemann} $\frac{1}{n}\sum_{k=0}^{n-1}f'(\frac{k}{n})$ tend vers $\int_{0}^{1}f'(t)\;dt=f(1)-f(0)$ et donc 

$$\frac{1}{2n}\frac{1}{n}\sum_{k=0}^{n-1}f'(\frac{k}{n})=\frac{1}{2n}(f(1)-f(0)+o(1))=\frac{f(1)-f(0)}{2n}+o(\frac{1}{n}).$$

Finalement,

$$\int_{0}^{1}f(t)\;dt-\frac{1}{n}\sum_{k=0}^{n-1}f(\frac{k}{n})=\frac{f(1)-f(0)}{2n}+o(\frac{1}{n}).$$
}
}

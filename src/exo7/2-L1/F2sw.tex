\uuid{F2sw}
\exo7id{5471}
\auteur{rouget}
\organisation{exo7}
\datecreate{2010-07-10}
\isIndication{false}
\isCorrection{true}
\chapitre{Calcul d'intégrales}
\sousChapitre{Primitives diverses}

\contenu{
\texte{
Condition nécessaire et suffisante sur $a$, $b$, $c$ et $d$ pour que les primitives de $\frac{(x-a)(x-b)}{x-c)^2(x-d)^2}$ soient rationnelles ($a$, $b$, $c$ et $d$ réels donnés).
}
\reponse{
Si $c\neq d$, les primitives considérées sont rationnelles si et seulement si il existe $A$ et $B$ tels que

$$\frac{(x-a)(x-b)}{(x-c)^2(x-d)^2}=\frac{A}{(x-c)^2}+\frac{B}{(x-d)^2}\;(*)$$

\begin{align*}\ensuremath
(*)&\Leftrightarrow\exists(A,B)\in\Rr^2/\;\left\{
\begin{array}{l}
A+B=1\\
-2(Ad+Bc)=-(a+b)\\
Ad^2+Bc^2=ab
\end{array}
\right.\Leftrightarrow\exists(A,B)\in\Rr^2/\;\left\{
\begin{array}{l}
B=1-A\\
A(d-c)+c=\frac{1}{2}(a+b)\\
Ad^2+Bc^2=ab
\end{array}
\right.
\\
 &\Leftrightarrow\exists(A,B)\in\Rr^2/\;\left\{
\begin{array}{l}
A=\frac{a+b-2c}{2(d-c)}\\
B=\frac{2d-a-b}{2(d-c)}\\
Ad^2+Bc^2=ab
\end{array}
\right.
\Leftrightarrow\frac{a+b-2c}{2(d-c)}d^2+\frac{2d-a-b}{2(d-c)}c^2=ab\\
 &\Leftrightarrow d^2(a+b-2c)+c^2(2d-a-b)=2ab(d-c)\Leftrightarrow(a+b)(d^2-c^2)-2cd(d-c)=2ab(d-c)\\
 &\Leftrightarrow2cd+(a+b)(c+d)=2ab\Leftrightarrow(a+b)(c+d)=2(ab-cd).
\end{align*}

Si $c=d$, il existe trois nombres $A$, $B$ et $C$ tels que $(x-a)(x-b)=A(x-c)^2+B(x-c)+C$ et donc tels que

$$\frac{(x-a)(x-b)}{(x-c)^4}=\frac{A}{(x-c)^2}+\frac{B}{(x-c)^3}+\frac{C}{(x-c)^4}.$$

Dans ce cas, les primitives sont rationnelles. Finalement, les primitives considérées sont rationnelles si et seulement si $c=d$ ou ($c\neq d$ et $(a+b)(c+d)=2(ab-cd)$).
}
}

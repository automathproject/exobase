\uuid{4AIV}
\exo7id{2095}
\auteur{bodin}
\organisation{exo7}
\datecreate{2008-02-04}
\video{1uLeRF-liOk}
\isIndication{true}
\isCorrection{true}
\chapitre{Calcul d'intégrales}
\sousChapitre{Fraction rationnelle en sin, cos ou en sh, ch}

\contenu{
\texte{
Calculer les intégrales suivantes :
$$\int_0^{\frac \pi 2}\frac 1{1+\sin x}d x \quad \mbox{ et } \quad \int_0^{\frac \pi 2}\frac{\sin
x}{1+\sin x}d x.$$
}
\indication{$\int_0^{\frac \pi 2}\frac 1{1+\sin x}dx=1$ (changement de variables $t=\tan \frac x2$).

$\int_0^{\frac \pi 2}\frac{\sin x}{1+\sin x}dx=\frac \pi 2-1$ (utiliser la précédente).}
\reponse{
Notons $I = \int_0^{\frac \pi 2}\frac 1{1+\sin x} dx$.
Le changement de variable $t = \tan \frac x2$ transforme toute fraction rationnelle 
de sinus et cosinus en une fraction rationnelle en $t$ (que l'on sait résoudre !).

En posant $t=\tan \frac{x}{2}$ on a $x=\arctan \frac t2$ ainsi que les formules suivantes : 
$$\cos x = \frac {1-t^2}{1+t^2}, \quad \sin x = \frac{2t}{1+t^2}, 
\quad \tan x = \frac{2t}{1-t^2}, \quad dx=\dfrac{2dt}{1+t^2}.$$

Ici, on a seulement à remplacer $\sin x$.
Comme $x$ varie de $x=0$ à $x=\frac\pi 2$ alors $t=\tan \frac{x}{2}$ varie de $t=0$ à $t=1$.

\begin{align*}  
  I &= \int_0^{\frac \pi 2}\frac 1{1+\sin x} dx 
     = \int_0^1  \frac 1 {1+ \frac{2t}{1+t^2}}  \dfrac{2dt}{1+t^2} \\ 
    &= \int_0^1 \frac{2}{1+t^2 + 2t} dt 
    = \int_0^1 \frac{2}{(1+t)^2} dt \\
    &= \left[ \frac{-2}{1+t} \right]_0^1 
    = 1
\end{align*}
Notons $J = \int_0^{\frac \pi 2}\frac{\sin x}{1+\sin x} dx$.
Alors 
$$I+J =  \int_0^{\frac \pi 2}\frac 1{1+\sin x} dx + \int_0^{\frac \pi 2}\frac{\sin x}{1+\sin x} dx
= \int_0^{\frac \pi 2} \frac{1+\sin x}{1+\sin x} dx = \int_0^{\frac \pi 2} 1 \, dx = \big[ x \big]_0^{\frac \pi 2} = \frac \pi 2.$$
Donc $J = \frac \pi 2 - I= \frac \pi 2 - 1$.
}
}

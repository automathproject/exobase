\uuid{nXm5}
\exo7id{6973}
\titre{exo7 6973}
\auteur{blanc-centi}
\organisation{exo7}
\datecreate{2014-05-06}
\video{70pEydvJEw4}
\isIndication{true}
\isCorrection{true}
\chapitre{Fonctions circulaires et hyperboliques inverses}
\sousChapitre{Fonctions circulaires inverses}
\module{Analyse}
\niveau{L1}
\difficulte{}

\contenu{
\texte{
Montrer que pour tout $x>0$, on a
$$\Arctan\left(\frac{1}{2x^2}\right)=\Arctan\left(\frac{x}{x+1}\right)-\Arctan\left(\frac{x-1}{x}\right).$$
En déduire une expression de $\displaystyle S_n=\sum_{k=1}^n\Arctan\left(\frac{1}{2k^2}\right)$ 
et calculer $\displaystyle\lim_{n\to +\infty}S_n$.
}
\indication{Dériver la différence des deux expressions.}
\reponse{
Posons $f(x)=\Arctan\left(\frac{1}{2x^2}\right)-\Arctan\left(\frac{x}{x+1}\right)+\Arctan\left(\frac{x-1}{x}\right)$ 
pour tout $x>0$. La fonction $f$ est dérivable, et
\begin{eqnarray*}
f'(x)
 &=& \frac{-\frac{2}{2x^3}}{1+\left(\frac{1}{2x^2}\right)^2}-
\frac{\frac{1}{(1+x)^2}}{1+\left( \frac{x}{x+1} \right)^2}
+\frac{\frac{1}{x^2}}{1+\left( \frac{x-1}{x} \right)^2}\\
 &=& \frac{-4x}{4x^4+1}-\frac{1}{(1+x)^2+x^2}+\frac{1}{x^2+(x-1)^2}\\
 &=& \frac{-4x}{4x^4+1}+\frac{-\big(x^2+(x-1)^2\big)+\big((1+x)^2+x^2\big)}{\big((1+x)^2+x^2\big)\big(x^2+(x-1)^2\big)}\\
 &=& 0
\end{eqnarray*}
Ainsi $f$ est une fonction constante. 
Or $f(x)\xrightarrow[x\to +\infty]{}\Arctan 0-\Arctan 1+\Arctan 1=0$. Donc la constante vaut $0$, 
d'où l'égalité cherchée. 

Alors :
\begin{eqnarray*}
S_n
 &=&\sum_{k=1}^n\Arctan\left(\frac{1}{2k^2}\right)\\
 &=&\sum_{k=1}^n\Arctan\left(\frac{k}{k+1}\right)-\sum_{k=1}^n\Arctan\left(\frac{k-1}{k}\right)
 \quad \text{(par l'identité prouvée)}\\
 &=&\sum_{k=1}^n\Arctan\left(\frac{k}{k+1}\right)-\sum_{k'=0}^{n-1}\Arctan\left(\frac{k'}{k'+1}\right)
 \quad \text{(en posant $k'=k-1$)}\\
 &=&\Arctan\left(\frac{n}{n+1}\right)-\Arctan\left(\frac{0}{0+1}\right)
 \quad \text{(les sommes se simplifient)}\\
 &=&\Arctan\left(1-\frac{1}{n+1}\right)
 \quad \text{(car $\tfrac{n}{n+1} = 1-\tfrac{1}{n+1}$)} \\
\end{eqnarray*} 
Ainsi $S_n\xrightarrow[n\to +\infty]{}\Arctan 1=\frac{\pi}{4}$.
}
}

\uuid{Ja5x}
\exo7id{1234}
\titre{exo7 1234}
\auteur{legall}
\organisation{exo7}
\datecreate{1998-09-01}
\isIndication{false}
\isCorrection{false}
\chapitre{Dérivabilité des fonctions réelles}
\sousChapitre{Applications}
\module{Analyse}
\niveau{L1}
\difficulte{}

\contenu{
\texte{
Soit $  f   $ une fonction d\' erivable et $  a  $ un r\' eel. Soit $  h>0  $
un nombre r\' eel strictement positif fix\' e.
}
\begin{enumerate}
    \item \question{Montrer qu'il existe $  \theta \in ]0,1[  $ tel que
$$\displaystyle{\frac {f(a+h)-2f(a)+f(a-h)}{ h}=f'(a+\theta h)-f'(a-\theta h)}  .$$}
    \item \question{Pour tout $  h\not = 0  $ on note :
$  \displaystyle{ \varphi (h)=\frac{f(a+h)-2f(a)+f(a-h)}{ h^2}}  .$
Montrer que si $  f''(a)  $ existe, alors $  \displaystyle{\lim _{h\rightarrow 0}\varphi(h)=f''(a) }.$}
\end{enumerate}
}

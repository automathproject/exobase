\uuid{DlVH}
\exo7id{746}
\titre{exo7 746}
\auteur{bodin}
\organisation{exo7}
\datecreate{1998-09-01}
\isIndication{true}
\isCorrection{true}
\chapitre{Fonctions circulaires et hyperboliques inverses}
\sousChapitre{Fonctions circulaires inverses}
\module{Analyse}
\niveau{L1}
\difficulte{}

\contenu{
\texte{
D\'emontrer les in\'egalit\'es suivantes :
$$
\Arcsin a < \frac{a}{\sqrt{1-a^2}} \quad \text{ si } 0<a<1 ;
$$
$$
\Arctan a > \frac{a}{1+a^2} \quad  \text{ si } a>0.
$$
}
\indication{On pourra \'etudier les fonctions d\'efinies par la diff\'erence des deux termes de chaque in\'egalit\'e.}
\reponse{
Soit $f(a) = \Arcsin a - \frac{a}{\sqrt{1-a^2}}$ sur $]0,1[$.
Alors $f'(a) = \frac{1}{\sqrt{1-a^2}} - \frac{1}{\sqrt{1-a^2}(1-a^2)} = \frac{1}{\sqrt{1-a^2}}\cdot \frac{-a^2}{1-a^2}$
donc $f'(a) \leq 0$. Ainsi $f$ est strictement décroissante et $f(0)=0$
donc $f(a) < 0$ pout tout $a \in ]0,1[$.
Si $g(a) = \Arctan a - \frac{a}{1+a^2}$
alors $g'(a) = \frac1{1+a^2}-\frac{1-a^2}{(1+a^2)^2}=\frac{2a^2}{(1+a^2)^2} > 0$.
Donc $g$ est strictement croissante et $g(0) = 0$ donc $g$ est strictement positive sur $]0,+\infty[$.
}
}

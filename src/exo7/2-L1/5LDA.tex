\uuid{5LDA}
\exo7id{3856}
\auteur{quercia}
\organisation{exo7}
\datecreate{2010-03-11}
\isIndication{false}
\isCorrection{true}
\chapitre{Continuité, limite et étude de fonctions réelles}
\sousChapitre{Continuité : théorie}

\contenu{
\texte{

}
\begin{enumerate}
    \item \question{Soit $f : {[0,1]} \to {[0,1]}$ continue. Montrer qu'il existe $x \in {[0,1]}$
tel que $f(x) = x$.}
\reponse{$f(x)-x$ change de signe entre $0$ et $1$.}
    \item \question{Soient ${f,g} : {[0,1]} \to {[0,1]}$ continues telles que $f\circ g = g\circ f$.
Montrer qu'il existe $x \in {[0,1]}$
tel que $f(x) = g(x)$ (on pourra s'intéresser aux points fixes de~$f$).}
\reponse{Sinon $f-g$ est de signe constant, par exemple positif.
Si $a$ est le plus grand point fixe de~$f$ alors $g(a)>a$ et $g(a)$ est
aussi point fixe de~$f$, absurde.}
\end{enumerate}
}

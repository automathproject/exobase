\uuid{PmmV}
\exo7id{5696}
\titre{exo7 5696}
\auteur{rouget}
\organisation{exo7}
\datecreate{2010-10-16}
\isIndication{false}
\isCorrection{true}
\chapitre{Série numérique}
\sousChapitre{Autre}
\module{Analyse}
\niveau{L1}
\difficulte{}

\contenu{
\texte{
Nature de la série de terme général $u_n=\sin\left(\pi(2+\sqrt{3})^n\right)$.
}
\reponse{
Pour $n\in\Nn$, posons $u_n=\sin\left(\pi(2+\sqrt{3})^n\right)$. D'après la formule du binôme de \textsc{Newton}, $(2+\sqrt{3})^n=A_n +B_n\sqrt{3}$ où $A_n$ et $B_n$ sont des entiers naturels. Un calcul conjugué fournit aussi $(2-\sqrt{3})^n=A_n-B_n\sqrt{3}$. Par suite, $(2+\sqrt{3})^n+(2-\sqrt{3})^n= 2A_n$ est un entier pair. Par suite, pour $n\in\Nn$,

\begin{center}
$u_n =\sin\left(2A_n\pi-\pi(2-\sqrt{3})^n\right)=-\sin\left(\pi(2-\sqrt{3})^n\right)$.
\end{center}

Mais $0< 2-\sqrt{3}< 1$ et donc $(2-\sqrt{3})^n\underset{n\rightarrow+\infty}{\rightarrow}0$. On en déduit que $|u_n|\underset{n\rightarrow+\infty}{\sim}\pi(2-\sqrt{3})^n$ terme général d'une série géométrique convergente. Donc la série de terme général $u_n$ converge.
}
}

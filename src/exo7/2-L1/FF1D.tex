\uuid{FF1D}
\exo7id{4496}
\auteur{quercia}
\organisation{exo7}
\datecreate{2010-03-14}
\isIndication{false}
\isCorrection{true}
\chapitre{Série numérique}
\sousChapitre{Familles sommables}

\contenu{
\texte{
Soit $S(t) = \sum_{n=1}^\infty \frac{t^n}{1+t^n}$.
}
\begin{enumerate}
    \item \question{Pour quelles valeurs de~$t$ $S(t)$ a-t-elle un sens~?}
\reponse{$|t|<1$.}
    \item \question{Montrer que $S(t) = \sum_{k=1}^\infty (-1)^{k-1}\frac{t^k}{1-t^k}$.}
\reponse{$S(t) = \sum_{n=1}^\infty \frac{t^n}{1+t^n}
                   = \sum_{n=1}^\infty \sum_{k=1}^\infty (-1)^{k-1}t^{kn}$
    et on peut échanger les deux sommes car il y a convergence absolue.}
    \item \question{Soit $F_m(t) = \sum_{k=1}^m (-1)^{k-1}\frac{t^k(1-t)}{1-t^k}$.
    Montrer que $(F_m(t))$ converge uniformément vers $(1-t)S(t)$ sur $[0,1]$.
    En déduire la limite en~$1$ de $(1-t)S(t)$.
    On rappelle que $\ln2 = \sum_{m=1}^\infty \frac{(-1)^{m-1}}m$.}
\reponse{On suppose $t\in{]0,1[}$.
    $\frac{d}{d x}\Bigl(\frac{t^x}{1-t^x}\Bigr) = \frac{t^x\ln t}{(1-t^x)^2} < 0$
    donc le critère des séries alternées s'applique,
    le reste est majoré en valeur absolue par le premier terme du reste.
    $0\le\frac{t^k(1-t)}{1-t^k} = \frac t{1+\frac1t+\dots+\frac1{t^{k-1}}}\le\frac1k$
    donc le terme général converge uniformément vers~$0$.
    
    Par interversion de limite (puisqu'il y a convergence uniforme) on
    obtient $\lim_{t\to1^-}(1-t)S(t) = \sum_{k=1}^\infty\frac{(-1)^{k-1}}k=\ln 2$.}
    \item \question{Calculer le développement en série entière de~$S(t)$.
    Donner une interprétation arithmétique des coefficients de ce développement
    et préciser leur signe en fonction de~$n$.}
\reponse{$S(t) = \sum_{n=1}^\infty \sum_{k=1}^\infty (-1)^{k-1}t^{kn}
                   = \sum_{p=1}^\infty \sum_{kn=p} (-1)^{k-1}t^p
                   = \sum_{p=1}^\infty \sigma(p)t^p$

    avec $\sigma(p) = (\text{nombre de diviseurs impairs de }p)
                    -(\text{nombre de diviseurs pairs de }p) = \sigma_i(p)-\sigma_p(p)$.
   Si $p=2^\alpha q$ avec $q$ impair alors $\sigma_p(p) = \alpha\sigma_i(p)= \alpha\sigma_i(q)$
   donc $\sigma(p) > 0$ ssi $p$ est impair {\it (très joli exercice)}.}
\end{enumerate}
}

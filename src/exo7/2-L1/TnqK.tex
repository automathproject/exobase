\uuid{TnqK}
\exo7id{7183}
\titre{exo7 7183}
\auteur{megy}
\organisation{exo7}
\datecreate{2017-07-26}
\isIndication{true}
\isCorrection{true}
\chapitre{Propriétés de R}
\sousChapitre{Autre}
\module{Analyse}
\niveau{L1}
\difficulte{}

\contenu{
\texte{
%[séparation de termes]
Soient $a$, $b$ et $c$ trois réels. Montrer que
\[ a^4+b^4+c^4 \geq abc(a+b+c).\]
}
\indication{Utiliser l'inégalité arithmético-géométrique.}
\reponse{
Partir du membre de gauche et appliquer l'inégalité arithmético-géométrique à trois variables ne donne pas immédiatement le résultat. Si on développe le membre de droite, on obtient
\[ a^2bc+b^2ca+c^2ab,\]
que l'on peut essayer de minorer par trois utilisations indépendantes de l'inégalité arithmético-géométrique à quatre variables:
\[ a^2bc = \sqrt[4]{a^4a^4b^4c^4} \leq \frac14\left(a^4+a^4+b^4+c^4\right) = \frac14\left(2a^4+b^4+c^4\right). \]
De même, 
\[ b^2ac\leq \frac14\left(a^4+2b^4+c^4\right)
\text{ et }
c^2ab \leq \frac14\left(a^4+b^4+2c^4\right).\]
En sommant ces trois inégalités, on obtient le résultat.
}
}

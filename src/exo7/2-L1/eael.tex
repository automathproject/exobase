\uuid{eael}
\exo7id{457}
\auteur{bodin}
\organisation{exo7}
\datecreate{1998-09-01}
\video{KX375CPpZjU}
\isIndication{true}
\isCorrection{true}
\chapitre{Propriétés de R}
\sousChapitre{Les rationnels}

\contenu{
\texte{
Soit $p(x) = \sum_{i=0}^{n} a_i \cdot x^i$. On suppose que tous les
$a_i$ sont des entiers.
}
\begin{enumerate}
    \item \question{Montrer que si $p$ a une racine rationnelle $\frac{\alpha}{\beta}$ (avec $\alpha$ et $\beta$ premiers entre eux)
alors $\alpha$ divise $a_0$ et $\beta$ divise $a_n$.}
\reponse{Soit $\frac \alpha \beta \in \Qq$ avec $\pgcd(\alpha,\beta) = 1$.
Pour $p(\frac \alpha \beta) = 0$, alors $\sum_{i=0}^n {a_i
\left(\frac \alpha \beta \right)^i} = 0$. Apr\`es multiplication par
$\beta^n$ nous obtenons l'\'egalit\'e suivante :
$$
a_n\alpha^n+a_{n-1}\alpha^{n-1}\beta + \cdots +
a_1\alpha\beta^{n-1}+a_0\beta^n = 0.$$ 
En factorisant tous les termes de cette somme sauf le premier par $\beta$, nous \'ecrivons
$a_n\alpha^n+\beta q=0$. Ceci entra\^{\i}ne que $\beta$ divise
$a_n\alpha^n$, mais comme $\beta$ et $\alpha^n$ sont premier entre
eux alors par le lemme de Gauss
$\beta$ divise $a_n$. De m\^eme en factorisant par $\alpha$ tous les termes
de la somme ci-dessus, sauf le dernier,  nous obtenons $\alpha q'
+a_0\beta^n = 0$ et par un raisonnement similaire $\alpha$ divise
$a_0$.}
    \item \question{On consid\`ere le nombre $\sqrt 2+\sqrt 3$. En calculant son carr\'e, montrer que ce
carr\'e est racine d'un polyn\^ome de degr\'e 2. En d\'eduire, \`a
l'aide du r\'esultat pr\'ec\'edent qu'il n'est pas rationnel.}
\reponse{Notons $\gamma = \sqrt 2+\sqrt 3$.
Alors $\gamma^2 = 5 +2\sqrt 2 \sqrt 3$ Et donc
$\left(\gamma^2-5\right)^2= 4\times 2 \times 3$, Nous choisissons
$p(x) = (x^2-5)^2-24$, qui s'\'ecrit aussi $p(x)=x^4-10x^2+1$. Vu
notre choix de $p$, nous avons $p(\gamma)=0$. Si nous supposons
que $\gamma$ est rationnel, alors $\gamma = \frac \alpha \beta$ et
d'apr\`es la premi\`ere question $\alpha$ divise le terme constant de
$p$, c'est-\`a-dire $1$. Donc $\alpha=\pm 1$. De m\^eme $\beta$
divise le coefficient du terme de plus haut degr\'e de $p$, donc
$\beta$ divise $1$, soit $\beta = 1$. Ainsi $\gamma = \pm 1$, ce
qui est \'evidemment absurde !}
\indication{\begin{enumerate}
  \item Calculer $\beta^n p(\frac \alpha \beta)$ et utiliser le lemme de Gauss.
  \item Utiliser la premi\`ere question avec $p(x)=(x^2-5)^2-24$.

\end{enumerate}}
\end{enumerate}
}

\uuid{O8Cn}
\exo7id{4039}
\auteur{quercia}
\organisation{exo7}
\datecreate{2010-03-11}
\isIndication{false}
\isCorrection{true}
\chapitre{Développement limité}
\sousChapitre{Développements limités implicites}

\contenu{
\texte{
Soit $f : x  \mapsto \frac{x+1}x e^x$.
}
\begin{enumerate}
    \item \question{Tracer la courbe $\mathcal{C}$ représentative de $f$.}
    \item \question{Soit $\lambda \in \R^+$. Si $\lambda$ est assez grand, la droite d'équation
    $y=\lambda$ coupe $\mathcal{C}$ en deux points d'abscisses $a < b$.
 \begin{enumerate}}
    \item \question{Montrer que $a \sim \frac1\lambda$, et
            $e^b \sim \lambda$ pour $\lambda \to +\infty$.}
    \item \question{Chercher la limite de $b^a$ quand $\lambda$ tend vers $+\infty$.}
\reponse{
\begin{enumerate}
$a \sim e^{-b}  \Rightarrow  a\ln b \to 0  \Rightarrow  b^a \to 1$.
}
\end{enumerate}
}

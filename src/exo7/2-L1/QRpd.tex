\uuid{QRpd}
\exo7id{4469}
\titre{exo7 4469}
\auteur{quercia}
\organisation{exo7}
\datecreate{2010-03-14}
\isIndication{false}
\isCorrection{false}
\chapitre{Série numérique}
\sousChapitre{Autre}
\module{Analyse}
\niveau{L1}
\difficulte{}

\contenu{
\texte{
Pour $n \in \N$ on note $T_n$ le nombre de manières de décomposer
$n$ francs avec des pièces de 1, 2, 5 et 10 francs ($T_0 = 1$).
Montrer que :
$$\forall\ x \in {[0,1[},\
\sum_{k=0}^\infty T_kx^k = \frac1{(1-x)(1-x^2)(1-x^5)(1-x^{10})}.$$
}
}

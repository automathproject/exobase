\uuid{uFMW}
\exo7id{804}
\titre{exo7 804}
\auteur{cousquer}
\organisation{exo7}
\datecreate{2003-10-01}
\isIndication{false}
\isCorrection{false}
\chapitre{Calcul d'intégrales}
\sousChapitre{Théorie}
\module{Analyse}
\niveau{L1}
\difficulte{}

\contenu{
\texte{

}
\begin{enumerate}
    \item \question{Soit $f$ deux fois dérivable sur $[a,b]$, vérifiant
$\vert f''\vert\leq M$ sur $[a,b]$. Soit
$$\varphi(t)=f(t)-f(a)-(t-a)\frac{f(b)-f(a)}{b-a}-A(b-t)(t-a)$$
Soit $x\in\mathopen]a,b\mathclose[$~; on choisit $A=A(x)$ pour que 
$\varphi(x)=\nolinebreak 0$
(dessiner~!).
Montrer qu'il existe $c_1,c_2\in[a,b]$ tels que $c_1<c_2$ et
$\varphi'(c_1)=\varphi'(c_2)=0$, puis qu'il existe $c\in[a,b]$ tel que
$\varphi''(c)=0$. En déduire une majoration de $\vert A\vert$ pour 
$x\in[a,b]$. On convient de poser $A(a)=A(b)=0$.}
    \item \question{On note $E$ l'erreur commise en remplaçant $\int_a^bf(x)\,dx$ par
l'aire du trapèze défini par l'axe des $x$, les droites $x=a$ et
$x=b$ et la corde du graphe joignant les points $(a,f(a))$ et $(b,f(b))$
(dessiner~!).
Montrer que $E=\int_a^bA(x)(b-x)(x-a)\,dx$, et vérifier que
l'intégrale a un sens. En déduire que
$\vert E\vert\leq\frac{M(b-a)^3}{12}$ (utiliser 1)).}
    \item \question{Pour $n\geq1$ on pose $I_n=\frac{b-a}{n}\Bigl[\frac{f(a)}{2}+f(x_1)+
f(x_2)+\cdots+f(x_{n-1})+\frac{f(b)}{2}\Bigr]$ où $x_p=a+p\frac{b-a}{n}$ pour
$p=1,2,\ldots,n-1$. Montrer que $I_n$ est la somme des aires des trapèzes
construits sur les points d'abscisses $a,x_1,x_2,\ldots,x_{n-1},b$ et
les cordes correspondantes du graphe de $f$ (dessiner~!). Montrer que
$$\biggl\vert\int_a^b f(x)\,dx-In\biggr\vert\leq\frac{M(b-a)^3}{12n^2}$$}
    \item \question{On prend $[a,b]=[0,1]$ et $f(x)=e^{-x^2}$. Calculer
$M=\sup_{[0,1]}\vert f''\vert$.
Déterminer $n$ pour que la méthode des trapèzes avec $n$ intervalles
donne un nombre qui approche $\int_0^1e^{-x^2}\,dx$ à moins de $10^{-2}$
près. En déduire un encadrement de cette intégrale.}
\end{enumerate}
}

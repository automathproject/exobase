\uuid{yOrk}
\exo7id{5210}
\auteur{rouget}
\organisation{exo7}
\datecreate{2010-06-30}
\isIndication{false}
\isCorrection{true}
\chapitre{Propriétés de R}
\sousChapitre{Maximum, minimum, borne supérieure}

\contenu{
\texte{
\label{exo:suprou2bis}
Soient $A$ et $B$ deux parties de $\Rr$, non vides et bornées. Montrer que $\mbox{sup }A$, $\mbox{sup }B$, $\mbox{sup}(A+B)$, $\mbox{inf }A$, $\mbox{inf B}$, $\mbox{inf }(A+B)$ existent et que l'on a $\mbox{sup }(A+B)=\mbox{sup }A+\mbox{sup }B$ et $\mbox{inf }(A+B)=\mbox{inf }A+\mbox{inf }B$. ($A+B$ désigne l'ensemble des sommes d'un élément de $A$ et d'un élément de $B$).
}
\reponse{
$A$ et $B$ sont deux parties non vides et majorées de $\Rr$ et admettent donc des bornes supérieures notées respectivement $\alpha$ et $\beta$.
Pour tout $(a,b)\in A\times B$, on a $a+b\leq\alpha+\beta$. Ceci montre que $A+B$ est une partie non vide et majorée de $\Rr$, et donc que $\mbox{sup}(A+B)$ existe dans $\Rr$. (De plus, puisque $\alpha+\beta$ est un majorant de $A+B$, on a déjà $\mbox{sup}(A+B)\leq\alpha+\beta$).
Soit alors $\varepsilon>0$.
Il existe $a\in A$ et $b\in B$ tels que $\alpha-\frac{\varepsilon}{2}<a\leq\alpha$ et $\beta-\frac{\varepsilon}{2}<b\leq\beta$, et donc tels que $\alpha+\beta-\varepsilon<a+b\leq\alpha+\beta$.

En résumé,

\begin{center}
\textbf{(1)} $\forall(a,b)\in A\times B,\;a+b\leq\alpha+\beta$ et \textbf{(2)} $\forall\varepsilon>0,\;\exists(a,b)\in A\times B/\;a+b>\alpha+\beta-\varepsilon$.
\end{center}

On en déduit que

\begin{center}
\shadowbox{
$\mbox{sup}(A+B)=\alpha+\beta=\mbox{sup }A+\mbox{sup }B$.
}
\end{center}
Pour les bornes inférieures, on peut refaire le travail précédent en l'adaptant ou appliquer le résultat précédent aux ensembles $-A$ et $-B$ car $\text{Inf}A=-\text{sup}(-A)$.
}
}

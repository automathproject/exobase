\uuid{SxnR}
\exo7id{1299}
\auteur{gourio}
\organisation{exo7}
\datecreate{2001-09-01}
\isIndication{false}
\isCorrection{false}
\chapitre{Calcul d'intégrales}
\sousChapitre{Intégrale impropre}

\contenu{
\texte{
Soient $f$ et $g $ deux fonctions de ${\Rr}^{+}$ dans ${\Rr}$ telles que
$f\geq 0, g\geq 0, g=o(f) $ en $+\infty , $ et $ \int_{0}^{\infty }f $
    n'existe pas. Montrer alors :
$$\int_{0}^{x}g(u)du=o\left( \int_{0}^{x}f(u)du\right) $$ quand $x\rightarrow \infty .$
}
}

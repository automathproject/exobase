\uuid{TKE5}
\exo7id{7175}
\titre{exo7 7175}
\auteur{megy}
\organisation{exo7}
\datecreate{2017-07-26}
\isIndication{true}
\isCorrection{true}
\chapitre{Propriétés de R}
\sousChapitre{Autre}
\module{Analyse}
\niveau{L1}
\difficulte{}

\contenu{
\texte{
%[séparer les termes]
Soient $a, b \in \R^*$. Montrer que
\[
(1+a^2)(1+b^2) \geq 4ab.
\]
}
\indication{Utiliser l'inégalité arithmético-géométrique.}
\reponse{
On applique l'inégalité arithmético-géométrique à chacun des deux facteurs ce qui donne:
\[
(1+a^2)(1+b^2) \geq 
(2\sqrt{a^2})(2\sqrt{b^2})
= 4|ab| \geq  4ab.
\]

Remarque : on aurait également pu développer le membre de gauche et minorer par une seule utilisation de l'inégalité arithmético-géométrique à quatre variables:
\[
(1+a^2)(1+b^2) 
=1+a^2+b^2+a^2b^2
\geq  
4\sqrt[4]{a^2b^2a^2b^2} =4\sqrt[4]{a^4b^4}
= 4|ab| \geq 4ab.
\]
}
}

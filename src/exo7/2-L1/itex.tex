\uuid{itex}
\exo7id{1243}
\titre{exo7 1243}
\auteur{roussel}
\organisation{exo7}
\datecreate{2001-09-01}
\video{eGl4xUQYh6Q}
\isIndication{true}
\isCorrection{true}
\chapitre{Développement limité}
\sousChapitre{Calculs}
\module{Analyse}
\niveau{L1}
\difficulte{}

\contenu{
\texte{

}
\begin{enumerate}
    \item \question{Développement limité en $1$ à l'ordre $3$ de $f(x)=\sqrt{x}$.}
\reponse{Première méthode.
On applique la formule de Taylor (autour du point $x=1$)
$$f(x)=f(1)+f'(1)(x-1)+\frac{f''(1)}{2!}(x-1)^2 + \frac{f'''(1)}{3!}(x-1)^3 + o((x-1)^3)$$
Comme $f(x) = \sqrt x= x^{\frac12}$ alors $f'(x) = \frac12 x^{-\frac12}$ et donc $f'(1)=\frac12$.
Ensuite on calcule $f''(x)$ (puis $f''(1)$), $f'''(x)$ (et enfin $f'''(1)$).

On trouve le dl de $f(x)=\sqrt x$ au voisinage de $x=1$ :
$$\sqrt x = 1 + \frac12 (x-1) - \frac18 (x-1)^2 + \frac{1}{16} (x-1)^3 + o((x-1)^3)$$


\bigskip

Deuxième méthode.
Posons $h=x-1$ (et donc $x=h+1$). On applique la formule du dl de $\sqrt{1+h}$ autour de $h=0$.
\begin{align*}
f(x)=\sqrt x  
  & = \sqrt{1+h} \\
  & = 1 + \frac 12 h - \frac18 h^2 + \frac{1}{16} h^3 + o(h^3)   \quad \text{ c'est la formule du dl de } \sqrt{1+h} \\
  & = 1 + \frac12 (x-1) - \frac18 (x-1)^2 + \frac{1}{16} (x-1)^3 + o((x-1)^3) \\
\end{align*}}
    \item \question{Développement limité en $1$ à l'ordre $3$ de $g(x)= e^{\sqrt{x}}$.}
\reponse{La première méthode consiste à calculer $g'(x)=\frac{1}{2\sqrt x}\exp{\sqrt x}$, $g''(x)$, $g'''(x)$
 puis $g(1)$, $g'(1)$, $g''(1)$, $g'''(1)$ pour pouvoir appliquer la formule de Taylor conduisant à :
$$\exp(\sqrt x)= e + \frac{e}{2} (x-1) + \frac{e}{48} (x-1)^3 + o((x-1)^3)$$
(avec $e=\exp(1)$).

\bigskip

Autre méthode. Commencer par calculer le dl de $k(x)=\exp x$ en $x=1$ ce qui est très facile car pour tout $n$, $k^{(n)}(x)=\exp x$
et donc $k^{(n)}(1)=e$ :
$$\exp x= e + e (x-1) + \frac{e}{2!} (x-1)^2 + \frac{e}{3!} (x-1)^3 + o((x-1)^3).$$

Pour obtenir le dl $g(x)= h(\sqrt x)$ en $x=1$ on écrit d'abord :
$$\exp (\sqrt x)= e + e (\sqrt{x}-1) + \frac{e}{2!} (\sqrt{x}-1)^2 + \frac{e}{3!} (\sqrt{x}-1)^3 + o((\sqrt{x}-1)^3).$$
Il reste alors à substituer $\sqrt{x}$ par son dl obtenu dans la première question.}
    \item \question{Développement limité à l'ordre $3$ en $\frac\pi3$ de $h(x)=\ln (\sin x)$.}
\reponse{Posons $u=x-\frac\pi3$ (et donc $x=\frac\pi3+u$).
Alors 
$$\sin(x)=\sin(\frac\pi3+u) = \sin(\frac\pi3)\cos(u)+\sin(u)\cos(\frac\pi3) = \frac{\sqrt3}{2}\cos u +\frac12\sin u$$
On connaît les dl de $\sin u$ et $\cos u$ autour de $u=0$ (car on cherche un dl autour de $x=\frac\pi3$) donc

\begin{align*}
\sin x 
  & = \frac{\sqrt3}{2}\cos u +\frac12\sin u \\
  & = \frac{\sqrt3}{2} \bigg(1-\frac{1}{2!}u^2 + o(u^3) \bigg) + \frac{1}{2} \bigg( u -\frac{1}{3!}u^3 + o(u^3) \bigg) \\
  & = \frac{\sqrt3}{2} + \frac{1}{2} u -  \frac{\sqrt3}{4} u^2 -\frac{1}{12}u^3 + o(u^3) \\
  & = \frac{\sqrt3}{2} + \frac{1}{2} (x-\frac\pi3) -  \frac{\sqrt3}{4}(x-\frac\pi3)^2 -\frac{1}{12}(x-\frac\pi3)^3 + o((x-\frac\pi3)^3) \\
\end{align*}


\bigskip

Maintenant pour le dl de la forme $\ln(a+v)$ en $v=0$ on se ramène au dl de $\ln(1+v)$ ainsi :
$$\ln(a+v)=\ln\big(a(1 + \frac v a)\big) = \ln a + \ln(1 + \frac v a)
= \ln a + \frac v a - \frac12 \frac{v^2}{a^2}  + \frac13 \frac{v^3}{a^3} + o(v^3)$$

On applique ceci à $h(x)= \ln(\sin x)$ en posant toujours $u=x-\frac\pi3$ :
\begin{align*}
h(x)= \ln(\sin x)
  & = \ln\left(\frac{\sqrt3}{2} + \frac{1}{2} u -  \frac{\sqrt3}{4} u^2 -\frac{1}{12}u^3 + o(u^3)\right) \\
  & = \ln\left(\frac{\sqrt3}{2}\right) + \ln\left(1 + \frac{2}{\sqrt3}\left(\frac{1}{2} u -  \frac{\sqrt3}{4} u^2 -\frac{1}{12}u^3 + o(u^3)\right) \right) \\
  & = \ \cdots \qquad \text{ on effectue le dl du } \ln \text{ et on regroupe les termes} \\
  & = \ln\left(\frac{\sqrt3}{2}\right) + \frac{1}{\sqrt 3} u -\frac23 u^2 + \frac{4}{9\sqrt3}u^3 + o(u^3) \\
  & = \ln\left(\frac{\sqrt3}{2}\right) + \frac{1}{\sqrt 3}(x-\frac\pi3)  -\frac23 (x-\frac\pi3)^2 + \frac{4}{9\sqrt3}(x-\frac\pi3)^3 + o((x-\frac\pi3)^3) \\
\end{align*}

On trouve donc :
$$\ln(\sin x) = \ln\left(\frac{\sqrt3}{2}\right) + \frac{1}{\sqrt 3}(x-\frac\pi3) 
 -\frac23 (x-\frac\pi3)^2 + \frac{4}{9\sqrt3}(x-\frac\pi3)^3 + o((x-\frac\pi3)^3)$$


\bigskip Bien sûr une autre méthode consiste à calculer $h(1)$, $h'(1)$, $h''(1)$ et $h'''(1)$.}
\indication{Pour la première question vous pouvez appliquer la formule de Taylor ou bien 
poser $h=x-1$ et considérer un dl au voisinage de $h=0$.}
\end{enumerate}
}

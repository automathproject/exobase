\uuid{tuGf}
\exo7id{3861}
\titre{exo7 3861}
\auteur{quercia}
\organisation{exo7}
\datecreate{2010-03-11}
\isIndication{false}
\isCorrection{true}
\chapitre{Continuité, limite et étude de fonctions réelles}
\sousChapitre{Continuité : théorie}
\module{Analyse}
\niveau{L1}
\difficulte{}

\contenu{
\texte{
Soit $f : \R \to \R$. On dit que $f$ vérifie la propriété des valeurs intermédiaires
si :
$$\forall\ a,b\in\R \text{ avec } a < b, \forall\ y \text{ compris entre }
f(a) \text{ et } f(b),\ \exists\ x \in {[a,b]} \text{ tq } f(x) = y.$$
}
\begin{enumerate}
    \item \question{Montrer que si $f$ vérifie la propriété des valeurs intermédiaires et est
    injective, alors elle est continue.}
    \item \question{Trouver une fonction discontinue ayant la propriété des valeurs
    intermédiaires.}
\reponse{
Soit $a\in\R$ tel que $f$ est discontinue en $a$. Il existe une suite
         $a_n$ telle que $a_n \to a$ et $|f(a_n)-f(a)| \ge \varepsilon$.
         Alors $f(a)\pm\varepsilon$ a une infinité d'antécédents.
}
\end{enumerate}
}

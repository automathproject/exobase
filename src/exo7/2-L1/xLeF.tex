\uuid{xLeF}
\exo7id{717}
\titre{exo7 717}
\auteur{legall}
\organisation{exo7}
\datecreate{1998-09-01}
\video{w_zY_6IuJnE}
\isIndication{true}
\isCorrection{true}
\chapitre{Dérivabilité des fonctions réelles}
\sousChapitre{Théorème de Rolle et accroissements finis}
\module{Analyse}
\niveau{L1}
\difficulte{}

\contenu{
\texte{
Montrer que le polyn\^ome $  X^n+aX+b  $,  ($  a  $ et $  b  $ r\' eels) admet au plus trois racines r\' eelles.
}
\indication{On peut appliquer le th\'eor\`eme de Rolle plusieurs fois.}
\reponse{
Par l'absurde on suppose qu'il y a (au moins) quatre racines distinctes  pour 
$  P_n(X) =  X^n+aX+b  $. Notons les $x_1 <x_2<x_3<x_4$.
Par le th\'eor\`eme de Rolle appliqu\'e trois fois (entre $x_1$ et $x_2$, entre $x_2$ et $x_3$,...) il existe $x'_1<x'_2<x'_3$ des racines de $P'_n$. On applique deux fois le th\'eor\`eme Rolle entre $x'_1$ et $x'_2$ et entre $x'_2$ et $x'_3$. On obtient deux racines distinctes pour $P''_n$. Or 
$P''_n = n(n-1)X^{n-2}$ ne peut avoir que $0$ comme racines. Donc nous avons 
obtenu une contradiction.
\emph{Autre m\'ethode :} 
Le r\'esultat est \'evident si $  n\leq 3  .$ On suppose donc $
n\geq 3  .$ Soit $  P_n  $ l'application $  X\mapsto X^n+aX+b  $
de $  \R   $ dans lui-m\^eme. Alors $  P'_n(X)=nX^{n-1}+a  $
s'annule en au plus deux valeurs. Donc $P_n$ est successivement
croissante-d\'ecroissante-croissante ou bien
d\'ecroissante-croissante-d\'ecroissante. Et donc $P_n$ s'annule au plus 
trois fois.
}
}

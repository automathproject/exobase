\uuid{xsHT}
\exo7id{700}
\titre{exo7 700}
\auteur{bodin}
\organisation{exo7}
\datecreate{1998-09-01}
\video{K9yZt1R0xW8}
\isIndication{true}
\isCorrection{true}
\chapitre{Dérivabilité des fonctions réelles}
\sousChapitre{Calculs}
\module{Analyse}
\niveau{L1}
\difficulte{}

\contenu{
\texte{
Soit $f : \Rr^* \longrightarrow \Rr$ d\'efinie par
$\displaystyle{f(x)= x^2\sin \frac{1}{x} }$. }

\question{ Montrer que
$f$ est prolongeable par continuit\'e en $0$ ; on note encore
$f$ la fonction prolong\'ee. Montrer que $f$ est
d\'erivable sur $\Rr$ mais que $f'$ n'est pas continue en $0$.
}
\indication{$f$ est continue en $0$ en la prolongeant par $f(0)=0$.
$f$ est alors d\'erivable en $0$ et $f'(0)=0$.}
\reponse{
Comme $|\sin (1/x)| \leq 1$ alors
$f$ tend vers $0$ quand $x\rightarrow 0$. Donc en prolongeant $f$ par $f(0) =0$,
la fonction  $f$ prolongée est continue sur $\Rr$.
Le taux d'accroissement est
$$\frac{f(x)-f(0)}{x-0}= x \sin\frac{1}{x}.$$
Comme ci-dessus il y a une limite (qui vaut $0$) en $x=0$.
Donc $f$ est d\'erivable en $0$ et $f'(0)=0$.
Sur $\Rr^*$, $f'(x) = 2x\sin (1/x) -\cos(1/x)$,
Donc $f'(x)$ n'a pas de limite quand $x\rightarrow 0$.
Donc $f'$ n'est pas continue en $0$.
}
}

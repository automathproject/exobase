\uuid{m1Ag}
\exo7id{637}
\titre{exo7 637}
\auteur{gourio}
\organisation{exo7}
\datecreate{2001-09-01}
\isIndication{true}
\isCorrection{true}
\chapitre{Continuité, limite et étude de fonctions réelles}
\sousChapitre{Limite de fonctions}
\module{Analyse}
\niveau{L1}
\difficulte{}

\contenu{
\texte{
Trouver pour $(a,b)\in (\Rr^{+*})^{2}$ :
$$\lim\limits_{x\rightarrow +\infty}\left(\frac{a^{x}+b^{x}}{2}\right)^{\frac{1}{x}}.$$
}
\indication{R\'{e}ponse: $\max(a,b)$.}
\reponse{
Supposons $a \ge b$.
Alors 
$$\left(\frac{a^{x}+b^{x}}{2}\right)^{\frac{1}{x}} = \left( a^x \times \frac{1+ (\frac{b}{a})^x}{2}\right)^{\frac{1}{x}}
= a \left( \frac{1+ (\frac{b}{a})^x}{2}\right)^{\frac{1}{x}}.$$
Or $0 \le \frac b a \le 1$, donc $0 \le (\frac ba)^x \le 1$ pour tout $x \ge 1$.
Donc $(\frac 12)^{\frac{1}{x}} \le \left( \frac{1+ (\frac{b}{a})^x}{2}\right)^{\frac{1}{x}} \le 1^{\frac{1}{x}}$.
Les deux termes extrêmes tendent vers $1$ lorsque $x$ tend vers $+\infty$ donc le terme du milieu tend aussi vers $1$.
Conclusion : si $a \ge b$ alors  $\lim\limits_{x\rightarrow +\infty}\left(\frac{a^{x}+b^{x}}{2}\right)^{\frac{1}{x}} = a.$
Si $b \ge a$ alors cette limite vaudrait $b$.
Cela se résume dans le cas général o\`u $a,b$ sont quelconques par 
 $\lim\limits_{x\rightarrow +\infty}\left(\frac{a^{x}+b^{x}}{2}\right)^{\frac{1}{x}} = \max(a,b).$
}
}

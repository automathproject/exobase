\uuid{UG2u}
\exo7id{752}
\titre{exo7 752}
\auteur{bodin}
\organisation{exo7}
\datecreate{1998-09-01}
\video{AjgAXbaAu3E}
\isIndication{true}
\isCorrection{true}
\chapitre{Fonctions circulaires et hyperboliques inverses}
\sousChapitre{Fonctions circulaires inverses}
\module{Analyse}
\niveau{L1}
\difficulte{}

\contenu{
\texte{
Vérifier
$$
\Arcsin x + \Arccos x = \frac{\pi}{2}\qquad\text{ et } \quad
\Arctan x + \Arctan\frac{1}{x} = \text{sgn}(x)\frac{\pi}{2}.
$$
}
\indication{Faire une étude de fonction.
La fonction $\text{sgn}(x)$ est la \emph{fonction signe} : elle vaut $+1$ si $x> 0$,
$-1$ si $x < 0$ (et $0$ si $x=0$).}
\reponse{
Soit $f$ la fonction définie sur $[-1,1]$ par 
$f(x) = \Arcsin x +\Arccos x$: $f$ est continue sur l'intervalle $[-1,1]$, 
et dérivable sur $]-1,1[$. Pour tout $x\in]-1,1[$, 
$f'(x)= \frac{1}{\sqrt{1-x^2}}+\frac{-1}{\sqrt{1-x^2}}  = 0$. 
Ainsi $f$ est constante sur $]-1,1[$, donc sur $[-1,1]$ (car continue aux extrémités). 
Or $f(0) = \Arcsin 0 +\Arccos 0 = \frac \pi 2$
donc pour tout $x\in[-1,1]$, $f(x) = \frac \pi 2$.
Soit $g(x) = \Arctan x + \Arctan \frac 1x$.
Cette fonction est définie sur $]-\infty,0[$ et sur $]0,+\infty[$ (mais pas en $0$).
On a 
$$g'(x)= \frac{1}{1+x^2} + \frac{-1}{x^2} \cdot \frac{1}{1+\frac{1}{x^2}} = 0,$$
donc $g$ est constante sur chacun de ses intervalles de définition:
$g(x) = c_1$ sur $]-\infty,0[$ et $g(x) = c_2$ sur $]0,+\infty[$.
Sachant $\Arctan 1 = \frac\pi4$, on calcule $g(1)$ et $g(-1)$ on obtient $c_1 = -\frac \pi 2$
et $c_2 = +\frac \pi 2$.
}
}

\uuid{N4MR}
\exo7id{4286}
\auteur{quercia}
\organisation{exo7}
\datecreate{2010-03-12}
\isIndication{false}
\isCorrection{true}
\chapitre{Calcul d'intégrales}
\sousChapitre{Intégrale impropre}

\contenu{
\texte{
Soit $\gamma$ la constante d'Euler. Montrer que~\dots
}
\begin{enumerate}
    \item \question{$ \int_{t=0}^{+\infty} e^{-t}\ln t\,d t = -\gamma$.}
\reponse{Intégrations par parties successives,\par
\begin{align*}
 \int_{t=0}^{+\infty} e^{-t}\ln t\,d t
&=  \int_{t=0}^{+\infty} e^{-t}\frac{t^k}{k!}\left(\ln t - \frac 11 - \frac 12 - \dots - \frac 1k\right)\,d t\cr
&=  \int_{t=0}^{+\infty} e^{-t}\frac{t^k}{k!}\ln(t/k)\,d t + \left(\ln k - \frac 11 - \frac 12 - \dots - \frac 1k\right).\cr\end{align*}

Soit $I_k =  \int_{t=0}^{+\infty} e^{-t}\frac{t^k}{k!}\ln(t/k)\,d t$.
On pose $t=ku$~:
\begin{align*}I_k
          &= \frac{k^{k+1}}{k!} \int_{u=0}^{+\infty} (ue^{-u})^k\ln u\,d u\cr
          &= \frac{k^{k+1}}{k!} \int_{u=0}^{+\infty}  d\left(\frac{(ue^{-u})^{k-1}}{k-1}\right) \frac{u^2e^{-u}\ln u}{1-u}\cr
          &= -\frac{k^{k+1}}{k!(k-1)} \int_{u=0}^{+\infty} (ue^{-u})^{k-1}  d\left(\frac{u^2e^{-u}\ln u}{1-u}\right).\cr\end{align*}

Comme $0\le ue^{-u}\le e^{-1}$, il reste $\sqrt k$ au dénominateur multiplié par quelque chose de borné.}
    \item \question{$ \int_{t=0}^1 \frac{1-e^{-t}-e^{-1/t}}t\,d t = \gamma$.}
\reponse{\begin{align*}
 \int_{t=0}^1 \frac{1-e^{-t}-e^{-1/t}}t\,d t
&=  \int_{t=0}^1 \frac{1-e^{-t}}t\,d t -  \int_{t=1}^{+\infty}\frac{e^{-t}}t\,d t\cr
&= \lim_{x\to0^+}\left(-\ln x -  \int_{t=x}^{+\infty}\frac{e^{-t}}t\,d t\right)\cr
&= \lim_{x\to0^+}\left((e^{-x}-1)\ln x -  \int_{t=x}^{+\infty}e^{-t}\ln t\,d t\right)\cr
&= - \int_{t=0}^{+\infty} e^{-t}\ln t\,d t.\cr\end{align*}}
    \item \question{$ \int_{t=0}^1 \left(\frac 1t + \frac 1{\ln(1-t)}\right)d t = \gamma$.}
\reponse{$ \int_{t=x}^{+\infty}\frac{e^{-t}}t\,d t = (t=-\ln(1-u)) =
 \int_{u=1-e^{-x}}^1 \frac{-d u}{\ln(1-u)}$.}
\end{enumerate}
}

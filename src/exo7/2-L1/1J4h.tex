\uuid{1J4h}
\exo7id{6976}
\titre{exo7 6976}
\auteur{blanc-centi}
\organisation{exo7}
\datecreate{2014-05-06}
\video{pMuDpGtaNAM}
\isIndication{true}
\isCorrection{true}
\chapitre{Fonctions circulaires et hyperboliques inverses}
\sousChapitre{Fonctions hyperboliques et hyperboliques inverses}
\module{Analyse}
\niveau{L1}
\difficulte{}

\contenu{
\texte{
Soit $x$ un réel fixé. Pour $n\in\Nn^*$, on pose
$$C_n=\sum_{k=1}^n\ch(kx)\qquad\text{ et }\qquad S_n=\sum_{k=1}^n\sh(kx).$$
Calculer $C_n$ et $S_n$.
}
\indication{Commencer par calculer $C_n+S_n$ et $C_n-S_n$ à l'aide des fonctions $\ch$ et $\sh$.}
\reponse{
Puisque $\ch x+\sh x=e^x$ et $\ch x-\sh x=e^{-x}$, 
les expressions $C_n+S_n=\sum_{k=1}^ne^{kx}$ et $C_n-S_n=\sum_{k=1}^ne^{-kx}$ 
sont des sommes de termes de suites géométriques, de raison respectivement $e^x$ et $e^{-x}$.

Si $x=0$, on a directement $C_n=\sum_{k=1}^n1=n$ et $S_n=\sum_{k=1}^n0=0$.

Supposons $x\not=0$, alors $e^{x}\not=1$. 
Donc
\begin{eqnarray*}
C_n+S_n
  &=& \sum_{k=1}^ne^{kx} = \frac{e^x-e^{(n+1)x}}{1-e^x}\\
  &=& e^x\,\frac{1-e^{nx}}{1-e^x}\\
  &=& e^x\ \frac{e^{\frac{nx}{2}}(e^{-\frac{nx}{2}}-e^{\frac{nx}{2}})}{e^{\frac x2}(e^{-\frac x2}-e^{\frac x2})}\\
  &=& e^{\frac{(n+1)x}{2}}\ \frac{e^{\frac{nx}{2}}-e^{-\frac{nx}{2}}}{e^{\frac x2}-e^{-\frac x2}}\\
  &=& e^{\frac{(n+1)x}{2}}\ \frac{\sh\frac{nx}{2}}{\sh\frac{x}{2}}
\end{eqnarray*}

De même $C_n-S_n = \sum_{k=1}^ne^{-kx}$ ; c'est donc la même formule que ci-dessus en remplaçant $x$ par $-x$.
Ainsi :
$$C_n-S_n = e^{-\frac{(n+1)x}{2}}\ \frac{\sh\frac{nx}{2}}{\sh\frac{x}{2}}$$

En utilisant $C_n=\frac{(C_n+S_n)+(C_n-S_n)}{2}$ et $S_n=\frac{(C_n+S_n)-(C_n-S_n)}{2}$, on récupère donc
$$C_n=\frac{e^{\frac{(n+1)x}{2}}+e^{-\frac{(n+1)x}{2}}}{2}\,\frac{\sh\frac{nx}{2}}{\sh\frac{x}{2}}=\ch\left(\tfrac{(n+1)x}{2}\right)\,\frac{\sh\frac{nx}{2}}{\sh\frac{x}{2}}$$
$$S_n=\frac{e^{\frac{(n+1)x}{2}}-e^{-(n+1)\frac x2}}{2}\,\frac{\sh\frac{nx}{2}}{\sh\frac{x}{2}}=\sh\left(\tfrac{(n+1)x}{2}\right)\,\frac{\sh\frac{nx}{2}}{\sh\frac{x}{2}}$$
}
}

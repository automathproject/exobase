\uuid{9NNY}
\exo7id{2088}
\auteur{bodin}
\organisation{exo7}
\datecreate{2008-02-04}
\isIndication{false}
\isCorrection{true}
\chapitre{Calcul d'intégrales}
\sousChapitre{Primitives diverses}

\contenu{
\texte{
Calculer les primitives suivantes, en pr\'ecisant si n\'ecessaire les
intervalles de validit\'e des calculs:
$$
\begin{array}{llll}
\textbf{a)~}\displaystyle\int \arctan x d x & \textbf{b)~}\displaystyle\int \tan ^2x d x &
\textbf{c)~}\displaystyle \int \frac 1{x\ln x} d x & \textbf{d)~}\displaystyle \int \frac
x{\sqrt{x+1}}d x \\
\textbf{e)~} \displaystyle\int \arcsin x d x & \textbf{f)~}\displaystyle \int \frac 1{3+\exp
  \left( -x\right)}d x \quad & \textbf{g)~} \displaystyle \int \frac{-1}{\sqrt{4x-x^2}}d
x & \textbf{h)~} \displaystyle \int \frac 1{x\sqrt{1-\ln ^2x}}d x\\
\textbf{i)~} \displaystyle \int \frac 1{\sqrt{1+\exp x}}d x \quad & \textbf{j)~} \displaystyle \int
\frac{x-1}{x^2+x+1} d x & \textbf{k)~} \displaystyle \int \frac{x+2}{x^2-3x-4}d x
\quad &
\textbf{l)~} \displaystyle\int \cos x\exp xd x 
\end{array}$$
}
\reponse{
a-$\int \arctan xdx=x\arctan x-\frac 12\ln \left( 1+x^2\right) +c$ sur $\Bbb{%
R}$ (int\'egration par parties)

b-$\int \tan ^2xdx=\tan x-x+c$ sur $\left] -\frac \pi 2+k\pi ,\frac \pi
2+k\pi \right[ $

c-$\int \frac 1{x\ln x}dx=\ln \left| \ln x\right| +c$ sur $\left] 0,1\right[
\cup \left] 1,+\infty \right[ $ (changement de variable : $u=\ln x$)

d-$\int \frac x{\sqrt{x+1}}dx=\frac 23\left( x-2\right) \left( x+1\right)
^{\frac 12}+c$ sur $\left] -1,+\infty \right[ $ (changement de variable : $u=\sqrt{x+1}$ ou 
int\'egration par parties)

e-$\int \arcsin xdx=x\arcsin x+\sqrt{1-x^2}+c$ sur $\left] -1,1\right[ $
(int\'egration par parties)

f-$\int \frac 1{3+\exp \left( -x\right) }dx=\frac 13\ln \left( 3\exp
x+1\right) +c$ sur $\Bbb{R}$ (changement de variable : $u=\exp x$)

g-$\int \frac{-1}{\sqrt{4x-x^2}}dx=\arccos \left( \frac 12x-1\right) +c$ sur 
$\left] 0,4\right[ $ (changement de variable : $u=\frac 12x-1$)

h-$\int \frac 1{x\sqrt{1-\ln ^2x}}dx=\arcsin \left( \ln x\right) +c$ sur $%
\left] \frac 1e,e\right[ $ (changement de variable : $u=lnx$)

i-$\int \frac 1{\sqrt{1+\exp x}}dx=x-2\ln \left( 1+\sqrt{\exp x+1}\right) +c$
sur $\Bbb{R}$ (changement de variable : $u=\sqrt{\exp x+1}$)

j-$\int \frac{x-1}{x^2+x+1}dx=\frac 12\ln \left( x^2+x+1\right) -\sqrt{3}%
\arctan \left( \frac 2{\sqrt{3}}\left( x+\frac 12\right) \right) +c$ sur $%
\Bbb{R}$

k-$\int \frac{x+2}{x^2-3x-4}dx=-\frac 15\ln \left| x+1\right| +\frac 65\ln
\left| x-4\right| +c$ sur $\R\setminus\left\{
-1,4\right\} $ (d\'ecomposition en \'el\'ements simples)

l-$\int \cos x\exp xdx=\frac 12\left( \cos x+\sin x\right) \exp x+c$ sur $%
\R$ (deux int\'egrations par parties)
}
}

\uuid{7KpA}
\exo7id{2918}
\auteur{quercia}
\organisation{exo7}
\datecreate{2010-03-08}
\isIndication{true}
\isCorrection{true}
\chapitre{Dénombrement}
\sousChapitre{Cardinal}

\contenu{
\texte{

}
\begin{enumerate}
    \item \question{Quel est le nombre de parties {\`a} $p$ {\'e}l{\'e}ments de $\{1, \dots, n\}$
  ne contenant pas d'{\'e}l{\'e}ments cons{\'e}cutifs ?}
    \item \question{Soit $t_n$ le nombre de parties de $\{1, \dots, n\}$ de cardinal
  quelconque sans {\'e}l{\'e}ments cons{\'e}cutifs.
  \begin{enumerate}}
    \item \question{Montrer que $t_{n+2} = t_{n+1} + t_n$, $t_{2n+1} = t_n^2 +
    t_{n-1}^2$, et $t_{2n} = t_n^2 - t_{n-2}^2$.}
    \item \question{Calculer $t_{50}$.}
\reponse{
Comme $\{ x_1-1, \dots, x_p-p\}$ est une partie quelconque de $\{0,
  \dots, n-p\}$, on a $N = C_{n-p+1}^p$.
\begin{enumerate}
32951280099.
}
\indication{Si $\{x_1, \dots, x_p\}$ est une telle partie avec $x_1 <
x_2 < \dots < x_p$, consid{\'e}rer l'ensemble $\{ x_1-1, \dots, x_p-p\}$.}
\end{enumerate}
}

\uuid{cWXR}
\exo7id{967}
\auteur{legall}
\organisation{exo7}
\datecreate{1998-09-01}
\isIndication{false}
\isCorrection{false}
\chapitre{Application linéaire}
\sousChapitre{Morphismes particuliers}

\contenu{
\texte{
Soit $E$ un espace vectoriel; on note $i_E$ l'identit\'e sur $E$. Un
endomorphisme $u$ de $E$ est un \textbf{projecteur} si $u\circ u=u$.
}
\begin{enumerate}
    \item \question{Montrer que si $u$ est un projecteur alors $i_E-u$ est un
projecteur. V\'erifier aussi que  $\text{Im} u=\{x\in E;\;\;u(x)=x\}$ et que
$E=\text{Ker} u\oplus \text{Im} u$.\\
Un endomorphisme $u$ de $E$ est appel\'e \textit{involutif} si $u\circ
u=i_E$.}
    \item \question{Montrer que si $u$ est involutif alors $u$ est bijectif et
$E=\text{Im}(i_E+u)\oplus \text{Im}(i_E-u)$.\\
     Soit $E=F\oplus G$ et soit $x\in E$ qui s'\'ecrit donc de fa\c{c}on
unique $x=f+g$, $f\in F$, $g\in $G. Soit $u:E\ni x\mapsto f-g\in E$.}
    \item \question{Montrer que $u$ est involutif, $F=\{x\in E;\;u(x)=x\}$ et
$G=\{x\in E;\;u(x)=-x\}$.}
    \item \question{Montrer que si $u$ est un projecteur, $2u-i_E$ est involutif
et que tout endomorphisme involutif peut se mettre sous cette forme.}
\end{enumerate}
}

\uuid{FreE}
\exo7id{7205}
\auteur{megy}
\organisation{exo7}
\datecreate{2019-07-23}
\isIndication{false}
\isCorrection{false}
\chapitre{Logique, ensemble, raisonnement}
\sousChapitre{Relation d'équivalence, relation d'ordre}

\contenu{
\texte{
(Écrasement d'une partie d'un ensemble) 
Soit $X$ un ensemble. Pour tout sous-ensemble $A\subseteq X$, on définit la relation binaire $\sim_A$ sur $X$ comme suit:
\[ \forall (x,y)\in X^2, \: x\sim_A y \iff \left(x=y \text{ ou } \left(x\in A\text{ et } y\in A\right)\right).\]
}
\begin{enumerate}
    \item \question{Montrer que c'est une relation d'équivalence sur $X$. Quelles sont ses classes d'équivalence ?}
    \item \question{Soit $f$ une fonction de $X$ dans un ensemble $E$, constante sur $A$. Montrer qu'elle descend au quotient en une application $[f] : X/\sim_A \to E$.}
    \item \question{Montrer que pour tout ensemble $E$, l'application 
\[
\phi : \left\{f\in \mathcal F(X,E),\:\middle| \: f\text{ est constante sur } A\right\} \to \mathcal F(X/\sim_A, E),
\]
qui à $f$ associe $[f]$ est surjective.}
    \item \question{Identifier, parmi les relations d'équivalence étudiées dans le cours et les exercices du chapitre, celles qui sont des cas particuliers d'écrasements de parties.}
\end{enumerate}
}

\uuid{kkp2}
\exo7id{6962}
\auteur{blanc-centi}
\organisation{exo7}
\datecreate{2014-04-01}
\video{uMsrg-zPUko}
\isIndication{false}
\isCorrection{true}
\chapitre{Polynôme, fraction rationnelle}
\sousChapitre{Racine, décomposition en facteurs irréductibles}

\contenu{
\texte{

}
\begin{enumerate}
    \item \question{Soit $P=X^n+a_{n-1}X^{n-1}+\cdots+a_1X+a_0$ un polynôme de degré $n\ge 1$ 
à coefficients dans $\Zz$. Démontrer que si $P$ admet une racine dans $\Zz$, 
alors celle-ci divise $a_0$.}
\reponse{Si $k\in\Zz$ est racine de $P$, alors $k^n+a_{n-1}k^{n-1}+\cdots+a_1k=-a_0$ 
ce qui donne $k(k^{n-1}+\cdots+a_1)=-a_0$, donc $k$ divise $a_0$.}
    \item \question{Les polynômes $X^3-X^2-109X-11$ et $X^{10}+X^5+1$ ont-ils des racines dans $\Z$?}
\reponse{Si $X^3-X^2-109X-11$ a une racine $k\in\Zz$, nécessairement $k$ divise $11$, 
donc $k$ vaut $-1$, $1$, $-11$ ou $11$. En testant ces quatre valeurs, on trouve 
que seul $11$ est racine. 

De même, si $X^{10}+X^5+1$ admettait une racine entière $k$, celle-ci diviserait $1$ donc vaut 
$k = \pm 1$, or on vérifie que ni $+1$, ni $-1$ ne sont racines.
Ainsi $X^{10}+X^5+1$ n'a pas de racine entière.}
\end{enumerate}
}

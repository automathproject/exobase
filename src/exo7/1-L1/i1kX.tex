\uuid{i1kX}
\exo7id{305}
\titre{exo7 305}
\auteur{gourio}
\organisation{exo7}
\datecreate{2001-09-01}
\video{IXE-IGeo-ts}
\isIndication{true}
\isCorrection{true}
\chapitre{Arithmétique dans Z}
\sousChapitre{Pgcd, ppcm, algorithme d'Euclide}
\module{Algèbre}
\niveau{L1}
\difficulte{}

\contenu{
\texte{
R\'{e}soudre dans ${\Zz}:1665x+1035y=45.$
}
\indication{Commencer par simplifier l'équation !
Ensuite trouver une solution particulière $(x_0,y_0)$
à l'aide de l'algorithme d'Euclide par exemple. Ensuite trouver
un expression pour une solution générale.}
\reponse{
En divisant par $45$ (qui est le pgcd de $1665, 1035, 45$) nous obtenons l'\'equation \'equivalente :
$$37x+23y=1 \qquad (E)$$
Comme le pgcd de $37$ et $23$ est $1$, alors d'apr\`es le th\'eor\`eme de B\'ezout
cette \'equation $(E)$ a des solutions.

L'algorithme d'Euclide pour le calcul du pgcd de $37$ et $23$ fourni
les coefficients de Bézout: $37\times 5 + 23 \times (-8) = 1$.
Une solution particuli\`ere de $(E)$ est donc 
$(x_0,y_0) = (5,-8)$.

Nous allons maintenant trouver l'expression générale pour les solutions de l'équation $(E)$.
Soient $(x,y)$ une solution de l'équation $37x+23y=1$.
Comme $(x_0,y_0)$ est aussi solution, nous avons $37x_0+23y_0=1$.
Faisons la différence de ces deux égalités pour obtenir $37(x-x_0)+23(y-y_0)=0$.
Autrement dit 
$$37(x-x_0)=-23(y-y_0) \quad (*)$$
On en déduit que $37 | 23 (y-y_0)$, or $\pgcd(23,37)=1$ donc par le lemme de Gauss,
$37 | (y -y_0)$. (C'est ici qu'il est important d'avoir divisé par $45$ dès le début !)
Cela nous permet d'écrire $y-y_0 = 37 k$ pour un $k \in \Zz$.

Repartant de l'égalité $(*)$ : nous obtenons $37(x-x_0)=-23 \times 37 \times k$.
Ce qui donne $x-x_0 = -23 k$.
Donc si $(x,y)$ est solution de $(E)$ alors elle est de la forme :
$(x,y)= (x_0 - 23k,y_0+37k)$, avec $k \in \Zz$.

Réciproquement pour chaque  $k\in\Zz$, si $(x,y)$ est de cette forme alors c'est une solution de $(E)$
(vérifiez-le !).

Conclusion : les solutions sont 
 $$\big\lbrace (5 - 23k,-8+37k) \mid k\in \Zz\big\rbrace.$$
}
}

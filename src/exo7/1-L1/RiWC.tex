\uuid{RiWC}
\exo7id{3169}
\titre{exo7 3169}
\auteur{quercia}
\organisation{exo7}
\datecreate{2010-03-08}
\isIndication{false}
\isCorrection{true}
\chapitre{Polynôme, fraction rationnelle}
\sousChapitre{Autre}
\module{Algèbre}
\niveau{L1}
\difficulte{}

\contenu{
\texte{
Soit $P \in { K[X]}$ de degr{\'e} $n$.
D{\'e}montrer que la famille $\bigl(P(X), P(X+1), \dots, P(X+n)\bigr)$ est une base de~${ K_n[X]}$.

(Utiliser l'op{\'e}rateur $\Delta$ de l'exercice \ref{opdiff})
}
\reponse{
vect$(P(X), P(X+1), \dots, P(X+n))$ contient
$P$, $\Delta P$, $\Delta^2P$, ..., $\Delta^nP$ donc ${ K_n[X]}$
d'apr{\`e}s le thm des degr{\'e}s {\'e}tag{\'e}s.
}
}

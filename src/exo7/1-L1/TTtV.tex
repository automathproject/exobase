\uuid{TTtV}
\exo7id{3191}
\auteur{quercia}
\organisation{exo7}
\datecreate{2010-03-08}
\isIndication{false}
\isCorrection{true}
\chapitre{Polynôme, fraction rationnelle}
\sousChapitre{Autre}

\contenu{
\texte{
Montrer que l'ensemble des solutions de l'in{\'e}quation $\sum_{k=1}^{100}\frac k{x-k}\ge 1$
est une r{\'e}union finie d'intervalles disjoints. Calculer la somme des
longueurs de ces intervalles.
}
\reponse{
Soit~$f(x) = \sum_{k=1}^{100}\frac k{x-k}$.
$f$ est strictement d{\'e}croissante de~$0$ {\`a}~$-\infty$ sur~$]-\infty,0[$,
de~$+\infty$ {\`a} $-\infty$ sur chaque
intervalle~$]k,k+1[$,~$1\le k\le 100$ et de~$+\infty$ {\`a}~$0$ sur~$]100,+\infty[$.
Donc il existe $1<\alpha_1<2<\alpha_2<\dots<\alpha_{99}<100<\alpha_{100}$
tels que $E = \{x\in\R\text{ tq }f(x)\ge 1\} = \bigcup_{k=1}^{100}]k,\alpha_k]$.

La somme des longueurs est $L = \sum_{k=1}^{100}\alpha_k-\sum_{k=1}^{100}k$
et $\alpha_1,\dots,\alpha_{100}$ sont les racines du polyn{\^o}me :
$$P(X) = \prod_{k=1}^{100}(X-k) - \sum_{k=1}^{100}k\prod_{i\ne k}(X-i)
= X^{100}-2X^{99}\sum_{k=1}^{100}k + \dots$$
D'o{\`u} $\sum_{k=1}^{100}\alpha_k = 2\sum_{k=1}^{100}k$ et $L=\sum_{k=1}^{100}k=5050$.
}
}

\uuid{c20l}
\exo7id{7149}
\auteur{megy}
\organisation{exo7}
\datecreate{2017-05-01}
\isIndication{false}
\isCorrection{true}
\chapitre{Nombres complexes}
\sousChapitre{Géométrie}

\contenu{
\texte{
Soit $ABC$ un triangle tel que $C$ soit l'image de $B$ par la rotation de centre $A$ et d'angle $\pi/2$.
Soit $s$ une similitude envoyant $A$ sur $B$ et $B$ sur $C$.
}
\begin{enumerate}
    \item \question{Que peut valoir $s(C)$ ?}
\reponse{Il y a exactement deux similitudes envoyant le couple $(A,B)$ sur le couple $(B,C)$ : une directe et une indirecte, et l'une se déduit de l'autre en composant (à gauche) par la réflexion d'axe $(BC)$.

D'après l'énoncé, le triangle $ABC$ est rectangle isocèle en $A$. On a donc $BC = \sqrt 2 AB$ et donc $s$ est de rapport $\sqrt 2$.
De plus, une similitude conservant les angles non orientés, on a $(AB)\bot (AC) \Rightarrow (s(A)s(B)) \bot (s(A)s(C))$, c'est-à-dire que $s(C)$ appartient à la perpendiculaire à $(s(A)s(B)) = (BC)$ passant par $s(A)=B$. Comme d'autre part on sait que $Cs(C)=\sqrt 2 BC$, cela donne deux possibilités, suivant que la similitude est directe ou indirecte.}
    \item \question{On suppose que $s$ est directe. Déterminer son centre $\Omega$. On l'exprimera comme barycentre de $A$, $B$ et $C$.}
\reponse{Si la similitude est directe, son angle est $3\pi/4$ puisque $(\overrightarrow{AB},\overrightarrow{BC}) = 3\pi/4$.\\

On fixe maintenant un repère orthonormé direct du plan dont l'origine est $A$. On note $a$, $b$ et $c$ les affixes des trois points et on a donc $a=0$ et $c=ib$ d'après l'énoncé.\\
Soit $(\alpha,\beta) \in \C^*\times \C$ tel que $s$ s'écrive $z\mapsto \alpha z+\beta$ en coordonnée complexe. Par ce qui précède, on a $\alpha = \sqrt 2 e^{3i\pi/4} = -1+i$. D'autre part, comme $A$ est envoyé sur $B$, on a $b = \beta$.\\
La similitude s'écrit donc $z\mapsto (i-1)z + b$. Son unique point fixe $\Omega$ a pour affixe 
\[
\omega = \frac{b}{1-(i-1)} 
= \frac{b}{2-i} 
= \frac{2b+ib}{5} 
= \frac{2b+c}{5}. 
\]
On peut reformuler ceci sous la forme $\omega = \frac{2}{5} a + \frac{2}{5} b + \frac{1}{5} c $, ce qui montre que $\Omega$ est le barycentre  de $(A,2/5)$, $(B,2/5)$ et $(C,1/5)$.}
    \item \question{Si la similitude est indirecte, déterminer son centre et son axe.}
\reponse{On fixe maintenant un repère orthonormé direct du plan dont l'origine est $A$, et tel que $B$ soit sur l'axe des abscisses (donc son affixe est réel). Soit $(\alpha,\beta) \in \C^*\times \C$ tel que $s$ s'écrive $z\mapsto \alpha \bar z+\beta$ en coordonnée complexe. On a $s(0)=b$ donc $\beta=b$, et $s(b)=c=ib$ donc $\alpha\bar b + b = ib$ ce qui donne $\alpha = i-1$. La transformation s'écrit donc dans ce repère 
\[ z\mapsto (i-1)\bar z + b,\]
 et $c$ est envoyé sur $(i+2)b$.

Cherchons un point fixe $\omega$ pour $s$ : si $s(\omega)=\omega$, alors en particulier $s(s(\omega))=\omega$, ce qui s'écrit $(i-1)\overline{(i-1)\overline{z}+b}+b=\omega $, autrement dit $2\omega+ib = \omega$ et finalement $\omega=-ib$. Réciproquement, on vérifie que le point d'affixe $-ib$ est bien fixe sous $S$. Comme on peut écrire $\omega = -ib = -c = 2\cdot a - c$, on en déduit que $\Omega$ est le barycentre de $(A,2)$ et $(C,-1)$.

Cherchons maintenant l'axe de la similitude indirecte. C'est une droite passant par le centre de la similitude et d'après l'écriture de la similitude, elle est dirigée par un vecteur d'affixe $e^{3i\pi/8}$. Un paramétrage en coordonnée complexe de cette droite est donc :
\[\Delta = \{ -c+t.e^{3i\pi/8}\:|\: t\in \R\}
\]}
\end{enumerate}
}

\uuid{HNUJ}
\exo7id{7208}
\auteur{megy}
\organisation{exo7}
\datecreate{2019-07-23}
\isIndication{false}
\isCorrection{false}
\chapitre{Logique, ensemble, raisonnement}
\sousChapitre{Relation d'équivalence, relation d'ordre}

\contenu{
\texte{
(Union/disjonction et intersection/conjonction de deux relations)
Soient $\mathcal R$ et $\mathcal S$ deux relations sur $E$. On définit la disjonction (ou union), notée $\mathcal R \vee \mathcal S$, par : 
\[ x (\mathcal R \vee \mathcal S) y \iff (x \mathcal R y\text{ ou } x \mathcal S y)\]
De façon équivalente, le graphe de $\mathcal R \vee \mathcal S$ est l'union des graphes de $\mathcal R$ et de $\mathcal S$. De même, on définit la conjonction (ou intersection) $\mathcal R \wedge \mathcal S$ comme la relation dont le graphe est l'intersection des deux graphes de $\mathcal R$ et $\mathcal S$, c'est-à-dire 
\[ x (\mathcal R \wedge \mathcal S) y \iff (x \mathcal R y\text{ et } x \mathcal S y).\]

Si $\mathcal R$ et $\mathcal S$ sont des relations d'équivalence, montrer que $\mathcal R \wedge \mathcal S$ est une relation d'équivalence, mais pas forcément $\mathcal R \vee \mathcal S$.

(Note : on peut définir la conjonction ou la disjonction d'un nombre quelconque de relations, à l'aide de l'union ou de l'intersection des graphes associés.)
}
}

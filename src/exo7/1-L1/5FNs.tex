\uuid{5FNs}
\exo7id{5072}
\titre{exo7 5072}
\auteur{rouget}
\organisation{exo7}
\datecreate{2010-06-30}
\isIndication{false}
\isCorrection{true}
\chapitre{Nombres complexes}
\sousChapitre{Trigonométrie}
\module{Algèbre}
\niveau{L1}
\difficulte{}

\contenu{
\texte{
Soit $a$ un réel distinct de $\frac{1}{\sqrt{3}}$ et $-\frac{1}{\sqrt{3}}$.
}
\begin{enumerate}
    \item \question{Calculer $\tan(3\theta)$ en fonction de $\tan\theta$.}
\reponse{Tout d'abord, d'après la formule de \textsc{Moivre},

$$\cos(3\theta)+i\sin(3\theta)=(\cos\theta+i\sin\theta)^3=(\cos^3\theta-3\cos\theta\sin^2\theta)
+i(3\cos^2\theta\sin\theta-\sin^3\theta),$$

et par identification des parties réelles et imaginaires,

\begin{center}
\shadowbox{
$\forall\theta\in\Rr,\;\cos(3\theta)=\cos^3\theta-3\cos\theta\sin^2\theta\;\mbox{et}\;\sin(3\theta)
=3\cos^2\theta\sin\theta-\sin^3\theta.$
}
\end{center}
Ensuite, $\tan(3\theta)$ et $\tan\theta$ existent $\Leftrightarrow3\theta\notin\frac{\pi}{2}+\pi\Zz$ et
$\theta\notin\frac{\pi}{2}+\pi\Zz\Leftrightarrow3\theta\notin\frac{\pi}{2}+\pi\Zz\Leftrightarrow\theta\notin\frac{\pi}{6}+
\frac{\pi}{3}\Zz$.
Soit donc $\theta\notin\frac{\pi}{6}+\frac{\pi}{3}\Zz$.

$$\tan(3\theta)=\frac{\sin(3\theta)}{\cos(3\theta)}
=\frac{3\cos^2\theta\sin\theta-\sin^3\theta}{\cos^3\theta-3\cos\theta\sin^2\theta}
=\frac{3\tan\theta-\tan^3\theta}{1-3\tan^2\theta},$$
après division du numérateur et du dénominateur par le réel non nul $\cos^3\theta$.

\begin{center}
\shadowbox{
$\forall\theta\in\Rr\setminus\left(\frac{\pi}{6}+\frac{\pi}{3}\Zz\right),\;\tan(3\theta)
=\frac{3\tan\theta-\tan^3\theta}{1-3\tan^2\theta}.$
}\end{center}}
    \item \question{Résoudre dans $\Rr$ l'équation~:

$$\frac{3x-x^3}{1-3x^2}=\frac{3a-a^3}{1-3a^2}.$$
On trouvera deux méthodes, l'une algébrique et l'autre utilisant la formule de trigonométrie établie en 1).}
\reponse{Soit $a\neq\pm\frac{1}{\sqrt{3}}$.
\textbf{1ère méthode.} $a$ est bien sûr racine de l'équation proposée, ce qui permet d'écrire~:

\begin{align*}
\frac{3x-x^3}{1-3x^2}=\frac{3a-a^3}{1-3a^2}&\Leftrightarrow(3x-x^3)(1-3a^2)=(1-3x^2)(3a-a^3)\;(\mbox{car}\;\pm\frac{1}{\sqrt{3}}
\;\mbox{ne sont pas solution de l'équation})\\
 &\Leftrightarrow(x-a)((3a^2-1)x^2+8ax-a^2+3)=0.
\end{align*}
Le discriminant réduit du trinôme $(3a^2-1)x^2+8ax-a^2+3$ vaut~:

$$\Delta'=16a^2-(3a^2-1)(-a^2+3)=3a^4+6a^2+3=(\sqrt{3}(a^2+1))^2>0.$$
L'équation proposée a donc trois racines réelles~:

\begin{center}
\shadowbox{
$\mathcal{S}=\left\{a,\frac{4a-\sqrt{3}(a^2+1)}{1-3a^2},\frac{4a+\sqrt{3}(a^2+1)}{1-3a^2}\right\}.$
}
\end{center}
\textbf{2ème méthode.} Il existe un unique réel
$\alpha\in\left]-\frac{\pi}{2},\frac{\pi}{2}\right[\setminus\left\{-\frac{\pi}{6},\frac{\pi}{6}\right\}$ tel que $a=\tan\alpha$. De même,
si $x$ est un réel distinct de $\pm\frac{1}{\sqrt{3}}$, il existe un unique réel
$\theta\in\left]-\frac{\pi}{2},\frac{\pi}{2}\right[\setminus\left\{-\frac{\pi}{6},\frac{\pi}{6}\right\}$ tel que $x=\tan\theta$ (à savoir
$\alpha=\Arctan a$ et $\theta=\arctan x$). Comme $\pm\frac{1}{\sqrt{3}}$ ne sont pas solution de l'équation proposée,
on a~:

\begin{align*}
\frac{3x-x^3}{1-3x^2}=\frac{3a-a^3}{1-3a^2}&\Leftrightarrow\frac{3\tan\theta-\tan^3\theta}{1-3\tan^2\theta}
=\frac{3\tan\alpha-\tan^3\alpha}{1-3\tan^2\alpha}\Leftrightarrow\tan(3\theta)=\tan(3\alpha)\\
 &\Leftrightarrow3\theta\in3\alpha+\pi\Zz\Leftrightarrow\theta\in\alpha+\frac{\pi}{3}\Zz.
\end{align*}
Ceci refournit les solutions $x=\tan\alpha=a$, puis

$$x=\tan\left(\alpha+\frac{\pi}{3}\right)=\frac{\tan\alpha+\tan\frac{\pi}{3}}{1-\tan\alpha\tan\frac{\pi}{3}}=\frac{a+\sqrt{3}}{1
-\sqrt{3}a}=\frac{(a+\sqrt{3})(1+\sqrt{3}a)}{1-3a^2}=\frac{4a+\sqrt{3}(a^2+1)}{1-3a^2},$$
et $x=\tan\left(\alpha-\frac{\pi}{3}\right)=\frac{4a-\sqrt{3}(a^2+1)}{1-3a^2}$.}
\end{enumerate}
}

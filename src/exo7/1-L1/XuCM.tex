\uuid{XuCM}
\exo7id{66}
\auteur{cousquer}
\organisation{exo7}
\datecreate{2003-10-01}
\isIndication{false}
\isCorrection{true}
\chapitre{Nombres complexes}
\sousChapitre{Géométrie}

\contenu{
\texte{
D\'eterminer par le calcul et g\'eom\'etriquement les nombres complexes $z$ tels
que $\Bigl\vert\frac{z-3}{z-5}\Bigr\vert= k$ ($k>0$, $k\neq1$).
G\'en\'eraliser pour $\Bigl\vert\frac{z-a}{z-b}\Bigr\vert=k$.
}
\reponse{
M\'ethode analogue \`a celle de l'exercice \ref{exo:compl}. On trouve $z=
{a-bke^{i\theta }\over 1-ke^{i\theta }}$.
On peut v\'erifier que le point d'affixe $z$ d\'ecrit le cercle dont un diam\`etre
joint les points correspondant \`a $\theta =0$ et \`a $\theta =\pi $
(v\'erifier en cherchant le milieu $z_0$ de ce segment et en \'etudiant $\vert
z-z_0\vert$).
}
}

\uuid{xhXq}
\exo7id{5119}
\auteur{rouget}
\organisation{exo7}
\datecreate{2010-06-30}
\isIndication{false}
\isCorrection{true}
\chapitre{Nombres complexes}
\sousChapitre{Forme cartésienne, forme polaire}

\contenu{
\texte{
Calculer de deux façons les racines carrées de $1+i$ et en déduire les valeurs exactes de $\cos\left(\frac{\pi}{8}\right)$
et $\sin\left(\frac{\pi}{8}\right)$.
}
\reponse{
D'abord on a $1+i=\sqrt{2}e^{i\pi/4}$. Les racines carrées de $1+i$ dans $\Cc$ sont donc
$\sqrt[4]{2}e^{i\pi/8}$ et $-\sqrt[4]{2}e^{i\pi/8}$.
On a aussi, pour $(x,y)\in\Rr^2$,
\begin{align*}
(x+iy)^2=1+i&\Leftrightarrow\left\{
\begin{array}{l}
x^2-y^2=1\\
x^2+y^2=\sqrt{2}\\
xy>0
\end{array}
\right.
\Leftrightarrow
\left\{
\begin{array}{l}
\rule[-4mm]{0mm}{0mm}x^2=\frac{1}{2}(\sqrt{2}+1)\\
y^2=\frac{1}{2}(\sqrt{2}-1)\\
xy>0
\end{array}
\right.\Leftrightarrow
(x,y)\in\left\{\pm\left(\sqrt{\frac{\sqrt{2}+1}{2}},\sqrt{\frac{\sqrt{2}-1}{2}}\right)\right\}.
\end{align*}
Les racines carrées de $1+i$ sont donc aussi
$\pm\left(\sqrt{\frac{\sqrt{2}+1}{2}}+i\sqrt{\frac{\sqrt{2}-1}{2}}\right)$. Puisque
$\Re(e^{i\pi/8})=\cos\frac{\pi}{8}>0$, on
obtient $\sqrt[4]{2}e^{i\pi/8}=\sqrt{\frac{\sqrt{2}+1}{2}}+i\sqrt{\frac{\sqrt{2}-1}{2}}$, ou encore

$$e^{i\pi/8}=\sqrt{\frac{\sqrt{2}+1}{2\sqrt{2}}}+i\sqrt{\frac{\sqrt{2}-1}{2\sqrt{2}}}=\frac{1}{2}\left(\sqrt{2+\sqrt{2}}+i
\sqrt{2-\sqrt{2}}\right)$$
et donc, par identification des parties réelles et imaginaires,
\begin{center}
\shadowbox{
$\cos\frac{\pi}{8}=\frac{1}{2}\sqrt{2+\sqrt{2}}\;\mbox{et}\;\sin\frac{\pi}{8}
=\frac{1}{2}\sqrt{2-\sqrt{2}}.$
}
\end{center}
}
}

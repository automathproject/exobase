\uuid{0apo}
\exo7id{3344}
\auteur{quercia}
\organisation{exo7}
\datecreate{2010-03-09}
\isIndication{false}
\isCorrection{false}
\chapitre{Application linéaire}
\sousChapitre{Morphismes particuliers}

\contenu{
\texte{
Soit $E$ un $K$ espace vectoriel de dimension finie et
$f\in\mathcal{L}(E)$ tel que
$\forall\ x\in E,\ \exists p_x\in\N^*,\ f^{p_x}(x) = \vec 0$.
Montrer que $f$ est nilpotent.
Donner un contre-exemple en dimension infinie.
}
}

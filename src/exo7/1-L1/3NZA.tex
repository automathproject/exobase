\uuid{3NZA}
\exo7id{284}
\titre{exo7 284}
\auteur{ridde}
\organisation{exo7}
\datecreate{1999-11-01}
\isIndication{false}
\isCorrection{false}
\chapitre{Arithmétique dans Z}
\sousChapitre{Divisibilité, division euclidienne}
\module{Algèbre}
\niveau{L1}
\difficulte{}

\contenu{
\texte{
Montrer que $\forall (a, b) \in \Nn \times \Nn^*$ il existe un unique $r(a)
\in \left\{ 0, \ldots, b-1\right\}$ tel qu'il existe $q \in \Nn$ avec
$a = bq + r (a)$.
}
\begin{enumerate}
    \item \question{En utilisant ceci pour $b = 13$, d\'eterminer les entiers $n\in \Nn$ tels
que $13|n^2 + n + 7$.}
    \item \question{Si $a \in \Nn$ et $b = 7$, d\'eterminer les valeurs possibles de $r (a^2)$
 (on rappelle que $r (a^2)$ doit appartenir \`a $\left\{ 0, \ldots, b-1\right\}$).\\
Montrer alors que $\forall (x, y) \in \Nn^2$ $ (7|x^2 + y^2) \text{ ssi }(7|x \text{ et }
7|y)$.}
    \item \question{Montrer qu'un entier positif de la forme $8k + 7$ ne peut pas être la
somme de trois carr\'es d'entiers.}
\end{enumerate}
}

\uuid{9q25}
\exo7id{396}
\auteur{legall}
\organisation{exo7}
\datecreate{2003-10-01}
\isIndication{false}
\isCorrection{false}
\chapitre{Polynôme, fraction rationnelle}
\sousChapitre{Racine, décomposition en facteurs irréductibles}

\contenu{
\texte{

}
\begin{enumerate}
    \item \question{Montrer que le polyn\^ome $P(X)= X^5-X^2+1$ admet une unique 
racine r\'eelle et que celle-ci est irationnelle.}
    \item \question{Montrer que le polyn\^ome $Q(X)=2X^3-X^2-X-3$ a une racine 
rationnelle (qu'on calculera). En d\'eduire sa d\'ecomposition en 
produit de facteurs
irr\'eductibles dans $\Cc [X].$}
\end{enumerate}
}

\uuid{lLlP}
\exo7id{3253}
\titre{exo7 3253}
\auteur{quercia}
\organisation{exo7}
\datecreate{2010-03-08}
\isIndication{true}
\isCorrection{true}
\chapitre{Polynôme, fraction rationnelle}
\sousChapitre{Racine, décomposition en facteurs irréductibles}
\module{Algèbre}
\niveau{L1}
\difficulte{}

\contenu{
\texte{
Soit $ K$ un sous-corps de~$\C$, $a\in K$ et $p\in\N$ premier.
Montrer que le polyn{\^o}me $X^p-a$ est irr{\'e}ductible sur~$ K$ si et seulement
s'il n'a pas de racine dans~$ K$.
}
\indication{Si $X^p-a = PQ$ avec $P,Q\in K[X]$ unitaires non
constants, factoriser $P$ dans~$\C$ et consid{\'e}rer $P(0)$.}
\reponse{
On suppose $a\ne 0$ et $X^p-a = PQ$ avec $P,Q\in K[X]$ unitaires
non constants. Soit $n=\deg(P)\in{[[1,p-1]]}$ et $b=(-1)^nP(0)\in K$.
$b$ est le produit de cetraines $p$-{\`e}mes de~$a$, donc $b^p = a^n$.
De plus $n\wedge p = 1$~; soit $nu+pv=1$ une relation de B{\'e}zout.
On a alors $b^{pu} = a^{nu} = a^{1-pv}$ d'o{\`u} $a = (b^u/a^v)^p$ 
donc $b^u/a^v\in K$ est racine de $X^p-a$.
}
}

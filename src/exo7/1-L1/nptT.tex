\uuid{nptT}
\exo7id{250}
\auteur{bodin}
\organisation{exo7}
\datecreate{1998-09-01}
\video{6MywGfpcPic}
\isIndication{true}
\isCorrection{true}
\chapitre{Arithmétique dans Z}
\sousChapitre{Divisibilité, division euclidienne}

\contenu{
\texte{
Trouver le reste de la division par $13$ du nombre $100^{1000}$.
}
\indication{Il faut travailler modulo $13$, tout d'abord r\'eduire $100$ modulo $13$.
Se souvenir que si $a\equiv b \pmod{13}$ alors $a^k\equiv b^k \pmod{13}$.
Enfin calculer ce que cela donne pour les exposants $k=1,2,3,\ldots$
en essayant de trouver une r\`egle g\'en\'erale.}
\reponse{
Il sagit de calculer $100^{1000}$ modulo $13$.
Tout d'abord  $100 \equiv 9 \pmod{13}$ donc  $100^{1000} \equiv 9^{1000} \pmod{13}$.
Or  $9^{2} \equiv 81 \equiv 3 \pmod{13}$, $9^{3} \equiv 9^2.9 \equiv 3.9 \equiv 1 \pmod{13}$, Or 
 $9^{4} \equiv 9^3.9 \equiv 9 \pmod{13}$, $9^{5} \equiv 9^4.9 \equiv 9.9 \equiv 3 \pmod{13}$. 
Donc 
$100^{1000} \equiv 9^{1000} \equiv 9^{3.333+1} \equiv (9^3)^{333}.9  \equiv 1^{333}.9 \equiv 9 \pmod{13}$.
}
}

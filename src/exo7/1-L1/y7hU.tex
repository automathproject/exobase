\uuid{y7hU}
\exo7id{963}
\auteur{legall}
\organisation{exo7}
\datecreate{1998-09-01}
\video{ZIq3tKZZgfg}
\isIndication{true}
\isCorrection{true}
\chapitre{Application linéaire}
\sousChapitre{Image et noyau, théorème du rang}

\contenu{
\texte{
Soit  $E$  et  $F$  deux espaces vectoriels de dimension finie
et  $\phi $  une application lin\' eaire  de  $E$  dans  $F$.
Montrer que  $\phi $  est un isomorphisme si et seulement si l'image par  $\phi $  de
toute base de  $E$  est une base de  $F$.
}
\indication{Pour une base $\mathcal{B} =\{ e_1, \ldots , e_n \}$ de $E$ considérer la famille 
$\{ \phi (e_1), \ldots, \phi (e_n) \} $.}
\reponse{
Montrons que si  $\phi $  est un isomorphisme,
l'image de toute base de  $E$  est une base de  $F$ : soit
$\mathcal{B} =\{ e_1, \ldots , e_n \} $  une base de  $E$  et
nommons  $\mathcal{B} ' $  la famille  $\{ \phi (e_1), \ldots ,
\phi (e_n) \} $.
    \begin{enumerate}
$\mathcal{B} '$  est libre. Soient en effet  $\lambda _1 , \ldots , \lambda _n\in {\R}$  tels
que  $\lambda _1\phi (e_1)+ \cdots + \lambda _n \phi (e_n)
=0 $. Alors  $ \phi (\lambda _1e_1+ \cdots + \lambda _ne_n) =0
$  donc, comme  $\phi$  est injective,  $\lambda _1e_1+ \cdots
+ \lambda _ne_n=0$  puis, comme  $\mathcal{B} $  est libre,
$\lambda _1=\cdots =\lambda _n=0$.
$\mathcal{B} '$  est g\' en\' eratrice. Soit  $y\in F$. Comme  $\phi $  est surjective, il
existe  $x\in E$  tel que  $y=\phi (x)$. Comme  $\mathcal{B}$
est g\' en\' eratrice, on peut choisir   $\lambda _1 , \cdots ,
\lambda _n\in {\R}$  tels que  $x=\lambda _1 e_1 +\cdots + \lambda
_n e_n $. Alors  $y=\lambda _1\phi (e_1)+ \cdots + \lambda _n
\phi (e_n)  $.
}
}

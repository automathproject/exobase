\uuid{H6Wv}
\exo7id{157}
\auteur{ridde}
\organisation{exo7}
\datecreate{1999-11-01}
\video{okAsZpWWB0M}
\isIndication{true}
\isCorrection{true}
\chapitre{Logique, ensemble, raisonnement}
\sousChapitre{Récurrence}

\contenu{
\texte{
Soit $X$ un ensemble. Pour $f \in \mathcal{F} (X, X)$, on d\'efinit $f^0 = id$ et
par r\'ecurrence pour $n \in \Nn$ $f^{n + 1} = f^n \circ f$.
}
\begin{enumerate}
    \item \question{Montrer que $\forall n \in \Nn$ $f^{n + 1} = f \circ f^n$.}
\reponse{Montrons la proposition demand\'ee par r\'ecurrence:
soit $\mathcal{A}_{n}$ l'assertion $f^{n + 1} = f \circ f^n$.
Cette assertion est vraie pour $n=0$. Pour $n\in \Nn$ supposons
$\mathcal{A}_{n}$ vraie. Alors
$$f^{n + 2} = f^{n + 1} \circ f = (f \circ f^n) \circ f = f \circ (f^n \circ f) = f \circ f^{n + 1}.$$
Nous avons utiliser la definition de $f^{n + 2}$, puis la proposition $\mathcal{A}_{n}$,
puis l'associativit\'e de la composition, puis la d\'efinition de $f^{n + 1}$.
Donc $\mathcal{A}_{n+1}$ est vraie. Par le principe de r\'ecurrence
$$\forall \in \Nn \ \ f^n\circ f = f\circ f^n.$$}
    \item \question{Montrer que si $f$ est bijective alors $\forall n \in \Nn$ $ (f^{-1})^n
 = (f^n)^{-1}$.}
\reponse{On proc\`ede de m\^eme par r\'ecurrence:
soit $\mathcal{A}_{n}$ l'assertion $ (f^{-1})^n  = (f^n)^{-1}$.
Cette assertion est vraie pour $n=0$. Pour $n\in \Nn$ supposons
$\mathcal{A}_{n}$ vraie. Alors
$$(f^{-1})^{n+1} = (f^{-1})^{n} \circ f^{-1} = (f^n)^{-1} \circ f^{-1} = (f\circ f^n)^{-1} =
 ( f^n \circ f)^{-1} = ( f^{n+1})^{-1} .$$
Donc $\mathcal{A}_{n+1}$ est vraie. Par le principe de r\'ecurrence
$$\forall \in \Nn \ \ (f^{-1})^n
 = (f^n)^{-1}.$$}
\indication{Pour les deux questions, travailler par r\'ecurrence.}
\end{enumerate}
}

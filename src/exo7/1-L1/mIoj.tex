\uuid{mIoj}
\exo7id{183}
\auteur{ridde}
\organisation{exo7}
\datecreate{1999-11-01}
\isIndication{false}
\isCorrection{false}
\chapitre{Logique, ensemble, raisonnement}
\sousChapitre{Autre}

\contenu{
\texte{
Pour $p \in \left\{ 1, 2, 3\right\}$ on note $S_{p} = \sum\limits_{k = 0}^n k^p$.
}
\begin{enumerate}
    \item \question{A l'aide du changement d'indice $i = n-k$ dans $S_{1}$, calculer $S_{1}$.}
    \item \question{Faire de même avec $S_{2}$. Que se passe-t-il ?}
    \item \question{Faire de même avec $S_{3}$ pour l'exprimer en fonction de $n$ et $S_{2}$.}
    \item \question{En utilisant l'exercice \ref{ex104}, calculer $S_{3}$.}
\end{enumerate}
}

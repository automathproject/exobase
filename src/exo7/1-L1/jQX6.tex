\uuid{jQX6}
\exo7id{996}
\titre{exo7 996}
\auteur{legall}
\organisation{exo7}
\datecreate{1998-09-01}
\video{Pmhg9aR10aA}
\isIndication{true}
\isCorrection{true}
\chapitre{Espace vectoriel}
\sousChapitre{Base}
\module{Algèbre}
\niveau{L1}
\difficulte{}

\contenu{
\texte{
D\'eterminer pour quelles valeurs de $ t\in {\Rr} $ les
vecteurs  
$$\big\{(1, 0, t),  (1, 1, t), (t,0,1)\big\}$$
forment une base de  $\Rr^3$.
}
\indication{C'est une base pour $t\neq \pm 1$.}
\reponse{
Quand le nombre de vecteurs égal la dimension de l'espace  nous avons les équivalences, entre 
\emph{être une famille libre} et \emph{être une famille génératrice} et donc aussi \emph{être une base}.

Trois vecteurs dans $\Rr^3$ forment donc une base si et seulement s'ils forment une famille libre.
Vérifions quand c'est le cas.


\begin{align*}
      & a (1, 0, t) + b (1, 1, t) + c (t,0,1) = (0,0,0) \\
\iff  & (a+b+tc,b,at+bt+c)=(0,0,0) \\
\iff  & \begin{cases}
         a+b+tc = 0 \\
         b = 0 \\
         at+bt+c = 0 \\
        \end{cases} 
\iff \begin{cases}
         b = 0 \\
         a+tc = 0 \\
         at+c = 0 \\
        \end{cases} \\
\iff & \begin{cases}
         b = 0 \\
         a=-tc \\
         (-tc)t+c = 0 \\
        \end{cases} 
\iff \begin{cases}
         b = 0 \\
         a=-tc \\
         (t^2-1)c = 0 \\
        \end{cases} \\
\end{align*}

Premier cas : si $t\neq \pm 1$. Alors $t^2-1 \neq 0$ et donc
la seule solution du système est $(a=0,b=0,c=0)$.
Dans ce cas la famille est libre et est donc aussi une base.


Deuxième cas : si $t=\pm1$. Alors la dernière ligne du système disparaît
et il existe des solutions non triviales (par exemple si $t=1$, $(a=1,b=0,c=-1)$ est une solution).
La famille n'est pas libre et n'est donc pas une base.
}
}

\uuid{n08t}
\exo7id{5584}
\titre{exo7 5584}
\auteur{rouget}
\organisation{exo7}
\datecreate{2010-10-16}
\isIndication{false}
\isCorrection{true}
\chapitre{Application linéaire}
\sousChapitre{Morphismes particuliers}
\module{Algèbre}
\niveau{L1}
\difficulte{}

\contenu{
\texte{
Soient $E$ un espace de dimension finie $n$ non nulle et $f$ un endomorphisme nilpotent de $E$. Montrer que $f^n=0$.
}
\reponse{
\textbf{1ère solution.} Si $f=0$, c'est immédiat . Sinon, soit $p$ l'indice de nilpotence de $f$ ($p\geqslant2$).

Par définition de $p$, il existe un vecteur $x_0$ tel que $f^{p-1}(x_0)\neq 0$ (et $f^p(x_0)=0$).

Montrons que la famille $(f^k(x_0))_{0\leqslant k\leqslant p-1}$ est libre. Dans le cas contraire, il existe $a_0$,..., $a_{p-1}$ $p$ scalaires non tous nuls tels que $a_0x_0+...+a_{p-1}f^{p-1}(x_0)=0$.

Soit $k=\text{Min}\{i\in\llbracket0,p-1\rrbracket/\;a_i\neq0\}$.

\begin{align*}\ensuremath
\sum_{i=0}^{p-1}a_if^i(x_0)=0&\Rightarrow\sum_{i=k}^{p-1}a_if^i(x_0)=0\Rightarrow f^{p-1}\left(\sum_{i=k}^{p-1}a_if^i(x_0)\right)=0\\
 &\Rightarrow a_kf^{p-1}(x_0)=0\;(\text{car pour}\;i\geqslant p,\;f^i=0\\
 &\Rightarrow a_k=0\;(\text{car}\;f^{p-1}(x_0)\neq0).
\end{align*}

Ceci contredit la définition de $k$ et donc la famille $(f^k(x_0))_{0\leqslant k\leqslant p-1}$ est libre. Puisque le cardinal d'une famille libre est inférieur à la dimension de
l'espace, on a montré que $p\leqslant n$ ou, ce qui revient au même, $f^n=0$.

\textbf{2ème solution.} (pour les redoublants)

Soit $p\in\Nn^*$ l'indice de nilpotence de $f$. Le polynôme $X^p$ est annulateur de $f$. Son polynôme minimal est un diviseur de $X^p$ et donc égal à $X^k$ pour un certain $k\in\llbracket1,p\rrbracket$. Par définition de l'indice de nilpotence, $k=p$ puis $\mu_f=X^p$. D'après le théorème de \textsc{Cayley}-\textsc{Hamilton}, $\mu_f$ divise $\chi_f$ qui est de degré $n$ et en particulier $p\leqslant n$.
}
}

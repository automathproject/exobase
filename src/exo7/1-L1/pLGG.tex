\uuid{pLGG}
\exo7id{5320}
\titre{exo7 5320}
\auteur{rouget}
\organisation{exo7}
\datecreate{2010-07-04}
\isIndication{false}
\isCorrection{true}
\chapitre{Polynôme, fraction rationnelle}
\sousChapitre{Autre}
\module{Algèbre}
\niveau{L1}
\difficulte{}

\contenu{
\texte{
Soit $P$ un polynôme différent de $X$. Montrer que $P(X)-X$ divise $P(P(X))-X$.
}
\reponse{
Si $P$ est de degré inférieur ou égal à $0$, c'est clair.

Sinon, posons $P=\sum_{k=0}^{n}a_kX^k$ avec $n\in\Nn^*$.

\begin{align*}\ensuremath
P(P(X))-X&=P(P(X))-P(X)+P(X)-X=\sum_{k=0}^{n}a_k((P(X))^k-X^k)+(P(X)-X)\\
 &=\sum_{k=1}^{n}a_k((P(X))^k-X^k)+(P(X)-X).
\end{align*}

Mais, pour $1\leq k\leq n$, $(P(X))^k-X^k)=(P(X)-X)((P(X))^{k-1}+X(P(X))^{k-2}+...+X^{k-1})$ est divisible par $P(X)-X$ et il en est donc de même de $P(P(X))-X$.
}
}

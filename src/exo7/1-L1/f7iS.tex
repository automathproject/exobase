\uuid{f7iS}
\exo7id{3199}
\titre{exo7 3199}
\auteur{quercia}
\organisation{exo7}
\datecreate{2010-03-08}
\isIndication{false}
\isCorrection{true}
\chapitre{Polynôme, fraction rationnelle}
\sousChapitre{Division euclidienne}
\module{Algèbre}
\niveau{L1}
\difficulte{}

\contenu{
\texte{
Soient $P \in { K[X]}$, $a,b\in K$ distincts, et $\alpha = P(a)$, $\beta = P(b)$.
}
\begin{enumerate}
    \item \question{Quel est le reste de la division euclidienne de $P$ par $(X-a)(X-b)$ ?}
\reponse{$\frac {\alpha(b-X) + \beta(X-a)}{b-a}$.}
    \item \question{Trouver le reste de la division euclidienne de
    $(\cos\theta + X\sin\theta)^n$ par $X^2+1$.}
\reponse{$\cos n\theta + X\sin n\theta$.}
\end{enumerate}
}

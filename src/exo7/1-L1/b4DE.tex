\uuid{b4DE}
\exo7id{5133}
\titre{exo7 5133}
\auteur{rouget}
\organisation{exo7}
\datecreate{2010-06-30}
\isIndication{false}
\isCorrection{true}
\chapitre{Nombres complexes}
\sousChapitre{Géométrie}
\module{Algèbre}
\niveau{L1}
\difficulte{}

\contenu{
\texte{
Pour $z\in\Cc\setminus\{1\}$, on pose $Z=\frac{1+z}{1-z}$. Déterminer et construire
l'ensemble des points $M$ d'affixes $z$ tels que
}
\begin{enumerate}
    \item \question{$|Z|=1$.}
\reponse{$$|Z|=1\Leftrightarrow\frac{|1+z|^2}{|1-z|^2}=1\Leftrightarrow(1+x)^2+y^2=(1-x)^2+y^2\;\text{et}\;(x,y)\neq(1,0)\Leftrightarrow4x=0\Leftrightarrow x=0.$$
L'ensemble cherché est la droite $(Oy)$.}
    \item \question{$|Z|=2$.}
\reponse{\begin{align*}
|Z|=2&\Leftrightarrow(1+x)^2+y^2=4((1-x)^2+y^2)\Leftrightarrow3x^2+3y^2-10x+3=0\;\text{et}\;(x,y)\neq(1,0)\\
 &\Leftrightarrow x^2+y^2-\frac{10}{3}x+1=0\;\text{et}\;(x,y)\neq(1,0)\\
 &\Leftrightarrow\left(x-\frac{5}{3}\right)^2+y^2=\frac{16}{9}\;\text{et}\;(x,y)\neq(1,0)\\
 &\Leftrightarrow\left(x-\frac{5}{3}\right)^2+y^2=\frac{16}{9}.
\end{align*}
L'ensemble cherché est le cercle de centre $\Omega\left(\frac{5}{3},0\right)$ et de rayon $\frac{4}{3}$.}
    \item \question{$Z\in\Rr$.}
\reponse{\begin{align*}
Z\in\Rr&\Leftrightarrow Z=\overline{Z}\Leftrightarrow\frac{1+z}{1-z}=\frac{1+{\bar z}}{1-{\bar z}}\\
 &\Leftrightarrow(1+z)(1-{\bar z})=(1-z)(1+{\bar z})\;\text{et}\;z\neq1\Leftrightarrow z-{\bar z}={\bar z}-z\;\text{et}\;z\neq1\\
  &\Leftrightarrow z={\bar z}\;\text{et}\;z\neq1\Leftrightarrow z\in\Rr\;\text{et}\;z\neq1.
\end{align*}
L'ensemble cherché est la droite $(Ox)$ privé du point $(1,0)$.}
    \item \question{$Z\in i\Rr$.}
\reponse{\begin{align*}
Z\in i\Rr&\Leftrightarrow Z=-\overline{Z}\Leftrightarrow\frac{1+z}{1-z}=-\frac{1+{\bar z}}{1-{\bar z}}\Leftrightarrow(1+z)(1-{\bar z})=-(1-z)(1+{\bar z})\;\text{et}\;z\neq1\\
 &\Leftrightarrow 1-z{\bar z}=-1+z{\bar z}\;\text{et}\;z\neq1\Leftrightarrow|z|^2=1\;\text{et}\;z\neq1\Leftrightarrow|z|=1\;\text{et}\;z\neq1.
\end{align*}
L'ensemble cherché est donc le cercle de centre $O$ et de rayon $1$ privé du point $(1,0)$.}
\end{enumerate}
}

\uuid{iecU}
\exo7id{281}
\auteur{cousquer}
\organisation{exo7}
\datecreate{2003-10-01}
\isIndication{false}
\isCorrection{false}
\chapitre{Arithmétique dans Z}
\sousChapitre{Divisibilité, division euclidienne}

\contenu{
\texte{
Montrer que~:
}
\begin{enumerate}
    \item \question{Si un entier est de la forme $6k+5$, alors il est 
    nécessairement de la forme $3k-1$, alors que la réciproque est 
    fausse.}
    \item \question{Le carré d'un entier de la forme $5k+1$ est aussi de cette 
    forme.}
    \item \question{Le carré d'un entier est de la forme $3k$ ou $3k+1$, mais
    jamais de la forme $3k+2$.}
    \item \question{Le carré d'un entier est de la forme $4k$ ou $4k+1$, mais jamais 
de la forme $4k+2$ ni de la forme $4k+3$.}
    \item \question{Le cube de tout entier est de la forme $9k$, $9k+1$ ou $9k+8$.}
    \item \question{Si un entier est à la fois un carré et un cube, alors c'est une 
puissance sixième, et il est de la forme $7k$ ou $7k+1$.}
\end{enumerate}
}

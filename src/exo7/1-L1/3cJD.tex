\uuid{3cJD}
\exo7id{3375}
\titre{exo7 3375}
\auteur{quercia}
\organisation{exo7}
\datecreate{2010-03-09}
\isIndication{false}
\isCorrection{false}
\chapitre{Matrice}
\sousChapitre{Autre}
\module{Algèbre}
\niveau{L1}
\difficulte{}

\contenu{
\texte{

}
\begin{enumerate}
    \item \question{Soit $A \in \mathcal{M}_{n,p}(K)$ non nulle. Montrer que l'application
    ${f_A} : {\mathcal{M}_{p,n}(K)} \to { K}, X \mapsto {\mathrm{tr}(AX)}$ est une forme linéaire
    non nulle sur $\mathcal{M}_{p,n}(K)$.}
    \item \question{Réciproquement : Soit $\phi  : {\mathcal{M}_{p,n}(K)} \to { K}$ une forme linéaire
    quelconque. Montrer qu'il existe une unique matrice $A \in \mathcal{M}_{n,p}(K)$
    telle que $\phi = f_A$ (on pourra considérer l'application $A  \mapsto f_A$).}
    \item \question{Soit $\phi  : {\mathcal{M}_n(K)} \to  K$ une forme linéaire vérifiant :
    $\forall\ X,Y \in \mathcal{M}_n(K),\ \phi(XY) = \phi(YX)$.
    \par
    Montrer qu'il existe $\lambda \in  K$ tel que $\phi = \lambda\mathrm{tr}$.}
\end{enumerate}
}

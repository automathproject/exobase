\uuid{ojWx}
\exo7id{7197}
\titre{exo7 7197}
\auteur{megy}
\organisation{exo7}
\datecreate{2019-07-23}
\isIndication{false}
\isCorrection{true}
\chapitre{Logique, ensemble, raisonnement}
\sousChapitre{Relation d'équivalence, relation d'ordre}
\module{Algèbre}
\niveau{L1}
\difficulte{}

\contenu{
\texte{
Soit $f : \R\to \mathbb U, t\mapsto e^{it}$, et soit $\mathcal R$ la relation d'équivalence sur $\R$ définie par $x\mathcal R y \iff x\equiv y \pmod{2\pi}$. On note $\R/2\pi\Z$ l'ensemble quotient $\R/\mathcal R$. Montrer que l'application $f$ descend au quotient en une application $[f] :\R/2\pi\Z \to \mathbb U$ qui est une bijection.
}
\reponse{
L'application $f$ est constante sur les classes d'équivalence de $\mathcal R$, donc par définition, elle descend au quotient en une application $[f] : \R/2\pi\Z \to \mathbb U$, qui vérifie  $f= [f]\circ p$. Comme $f$ est surjective, $[f]$ aussi. Montrons l'injectivité.

Soient $\alpha$ et $\beta$ dans $\R/2\pi\Z$ tels que $[f](\alpha) = [f](\beta)$. Si $x$ et $y$ sont des représentants de $\alpha$ et $\beta$, on a donc $f(x)=f(y)$, c'est-à-dire $e^{ix}=e^{iy}$, d'où par le cours sur l'exponentielle complexe, $x\equiv y \pmod{2\pi}$, d'où $[x]=[y]$, ou encore $\alpha=\beta$.
}
}

\uuid{EJFd}
\exo7id{7145}
\auteur{megy}
\organisation{exo7}
\datecreate{2017-05-01}
\isIndication{false}
\isCorrection{true}
\chapitre{Nombres complexes}
\sousChapitre{Géométrie}

\contenu{
\texte{
Déterminer les éléments caractéristiques des transformations représentées par:
}
\begin{enumerate}
    \item \question{$z\mapsto (1-i )z + i$;}
\reponse{C'est une similitude directe de rapport $|1+i|=\sqrt 2$, d'angle $-\pi/4$. Son centre $\Omega$ est son (unique) point fixe, son affixe $\omega$ vérifie donc $\omega = (1-i)\omega + i $ c'est-à-dire $\omega=\frac{i}{i}=1$.}
    \item \question{$z\mapsto i \bar z + 1-i$;}
\reponse{C'est une similitude indirecte de rapport $|i|=1$, donc un antidéplacement. C'est donc une réflexion ou une réflexion glissée. Cherchons d'éventuels points fixes.

Un point $z=x+iy$ est fixe ssi $x+iy = i(x-iy)+1-i = y+1+i(x-1)$ autrement dit ssi $x-y=1$.



(Rédaction alternative de la recherche de points fixes, sans prendre la partie réelle et imaginaire : un point d'affixe $z$ est fixe ssi:
\begin{align*}
z=i\bar z +1-i
&\Leftrightarrow z - i\bar z -1+i=0\\
&\Leftrightarrow \overline{(1-i)}z +(1-i)\overline z -2=0
\end{align*}
On reconnaît l'équation complexe de la même droite.)

L'antidéplacement est donc la réflexion d'axe d'équation $x-y-1=0$.}
    \item \question{$z\mapsto 2i\bar z +3$;}
\reponse{C'est une similitude indirecte (notons-la $s$) de rapport $2$. Elle admet donc un point fixe d'affixe $\omega$ vérifiant $ \omega = 2i\bar \omega +3$, ce qui donne après calcul $\omega=-1-2i$.

Soit $h$ l'homothétie de centre $\Omega$ et de rapport $2$. Alors la similitude $s$ s'écrit $h\circ \sigma$, avec $\sigma$ une réflexion que l'on peut obtenir comme $h^{-1}\circ s$. En coordonnée complexe, $h^{-1}$ s'écrit $z\mapsto \frac12 (z+1+2i)-1-2i = \frac12 z -\frac{1+2i}{2}$. En composant, on trouve que $\sigma$ est représentée par 
\[z\mapsto \frac12 (2i\bar z+3) -\frac{1+2i}{2}
=i\bar z +1-i.
\]
D'après la question précédente, c'est la réflexion d'axe $y=x-1$.}
    \item \question{$ z\mapsto \bar z + 1$.}
\reponse{C'est la réflexion glissée composée de la translation d'affixe $1$ et de la réflexion suivant l'axe des abscisses.}
\end{enumerate}
}

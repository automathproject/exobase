\uuid{Teod}
\exo7id{3195}
\auteur{quercia}
\organisation{exo7}
\datecreate{2010-03-08}
\isIndication{false}
\isCorrection{true}
\chapitre{Polynôme, fraction rationnelle}
\sousChapitre{Autre}

\contenu{
\texte{
On donne un entier $n\ge 0$.

Montrer qu'il existe des polyn{\^o}mes
$P_0,\dots,P_n$ dans $\Z_n[X]$ tels que
$\forall\ i,j\in{[[0,n]]},\  \int_{t=0}^1 t^iP_j(t)\,dt = \delta_{ij}$.
}
\reponse{
Analyse~: on pose $P_j = a_0 + a_1X + \dots + a_nX^n$ et on consid{\`e}re la fraction rationnelle
$$F(X) = \frac{a_0}{X} + \frac{a_1}{X+1} + \dots + \frac{a_n}{X+n}
= \frac{P(X)}{X(X+1)\dots(X+n)}.$$
Alors $ \int_{t=0}^1 t^iP_j(t)\,dt = F(i+1) = \frac{i!\,P(i+1)}{(i+n+1)!}$
donc $P(j+1) = \frac{(j+n+1)!}{j!}$ et $P(k) = 0$ pour $k\in{[[1,n+1]]}\setminus\{j+1\}$,
soit $$
\begin{aligned}
  P(X) &= \frac{(j+n+1)!}{j!}\prod_{k\ne j+1}\frac{X-k}{j+1-k} \\&=
  (-1)^{n-j}\frac{(j+n+1)!}{(j!)^2(n-j)!}\prod_{k\ne j+1}(X-k) =
  Q_j(X).
\end{aligned}
$$

Synth{\`e}se : soit $Q_j$ le polyn{\^o}me ci-dessus et $a_0,\dots,a_n$ les
coefficients de la d{\'e}composition en {\'e}l{\'e}ments simples de $\frac{Q_j(X)}{X(X+1)\dots(X+n)}$.
On doit juste v{\'e}rifier que les $a_i$ sont entiers. Calcul~:
$$
\begin{aligned}
  a_i &= \frac{Q_j(-i)}{(-1)^ii!\,(n-i)!}  =
  (-1)^{i+j}\frac{(i+j)!\,(i+n+1)!\,(j+n+1)!}{(i+j+1)!\,(i!)^2\,(j!)^2\,(n-i)!\,(n-j)!}
  \\ &= (-1)^{i+j}C_{i+j}^iC_{i+n+1}^{i+j+1}C_{j+n+1}^{j}C_n^i(n+1) \in\Z.
\end{aligned}
$$
}
}

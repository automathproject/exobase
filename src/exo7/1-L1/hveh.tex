\uuid{hveh}
\exo7id{2752}
\titre{exo7 2752}
\auteur{tumpach}
\organisation{exo7}
\datecreate{2009-10-25}
\isIndication{false}
\isCorrection{false}
\chapitre{Matrice}
\sousChapitre{Propriétés élémentaires, généralités}
\module{Algèbre}
\niveau{L1}
\difficulte{}

\contenu{
\texte{
L'\textit{exponentielle} d'une matrice carr\'ee $M$ est, par d\'efinition, la limite de la s\'erie 
$$
e^M = 1 + M + \frac{M^2}{2!} + \dots = \lim_{n\rightarrow+\infty} \sum_{k=0}^{n}\frac{M^k}{k!}.
$$
On admet que cette limite existe en vertu d'un th\'eor\`eme d'analyse.
}
\begin{enumerate}
    \item \question{Montrer que si $AB = BA$ alors $e^{A+B} = e^A e^B$. On est autoris\'e, pour traiter cette question, \`a passer \`a la limite sans pr\'ecautions.}
    \item \question{Calculer $e^M$ pour les quatre matrices suivantes~:
$$
\left(\begin{array}{ccc}
a & 0 & 0\\0 & b & 0\\0 & 0& c
\end{array}\right),
\left(\begin{array}{ccc}
0 & a & b\\0 & 0 & c\\ 0 & 0 & 0
\end{array}\right),
\left(\begin{array}{cc}
0 & 1\\-1 & 0
\end{array}\right),
\left(\begin{array}{cc}
1 & 0 \\ 0  & 0
\end{array}\right).
$$}
    \item \question{Chercher un exemple simple o\`u $e^{A + B}\neq e^A e^B$.}
\end{enumerate}
}

\uuid{nHW1}
\exo7id{2565}
\titre{exo7 2565}
\auteur{delaunay}
\organisation{exo7}
\datecreate{2009-05-19}
\isIndication{false}
\isCorrection{true}
\chapitre{Matrice}
\sousChapitre{Matrice et application linéaire}
\module{Algèbre}
\niveau{L1}
\difficulte{}

\contenu{
\texte{
Soit $A=(a_{ij})_{1\leq i,j\leq n}$ une matrice carr\'ee $n\times n$. On veut d\'emontrer le
 r\'esultat suivant d\^u \`a Hadamard : Supposons que pour tout $i\in\{1,\cdots,n\}$, on ait
$$|a_{ii}|>\sum_{j=1,j\neq i}^{n}|a_{ij}| $$
alors $A$ est inversible.
}
\begin{enumerate}
    \item \question{Montrer le r\'esultat pour $n=2$.}
\reponse{{\it Montrons le r\'esultat pour $n=2$.}

Dans ce cas, la matrice $A$ s'\'ecrit 
$$A=\begin{pmatrix}a_{11}&a_{12} \\  a_{21}&a_{22}\end{pmatrix}$$ et les hypoth\`eses deviennent
$$|a_{11}|>|a_{12}|\ {\hbox{et}}\ |a_{22}|>|a_{21}|.$$
La matrice $A$ est inversible si et seulement si son d\'eterminant est non nul, or 
$$\det A=a_{11}a_{22}-a_{12}a_{21},$$
 et, compte tenu des hypoth\`eses,
$$|a_{11}a_{22}|=|a_{11}||a_{22}|>|a_{12}||a_{21}|=|a_{12}a_{21}|,$$
ainsi$|a_{11}a_{22}|>|a_{12}a_{21}|$ donc $a_{12}a_{21}\neq a_{12}a_{21}$ et le d\'eterminant est non nul.}
    \item \question{Soit $B$, la matrice obtenue en rempla\c cant, pour $j\geq 2$, chaque colonne $c_j$ de $A$ par la colonne
$$ c_j-{\frac{a_{1j}}{a_{11}}}c_1 ,$$ Calculer les $b_{ij}$ en fonction des $a_{ij}$. Montrer que si les
coefficients de $A$ satisfont les in\'egalit\'es ci-dessus, alors pour $i\geq 2$, on a
$$|b_{ii}|>\sum_{j=2,j\neq i}^{n}|b_{ij}| .$$}
\reponse{Soit $B$, la matrice obtenue en rempla\c cant, pour $j\geq 2$, chaque colonne $c_j$ de $A$ par la colonne
$$ c_j-{\frac{a_{1j}}{a_{11}}}c_1 ,$$ {\it Calculons les $b_{ij}$ en fonction des $a_{ij}$. Montrons que si les
coefficients de $A$ satisfont les in\'egalit\'es ci-dessus, alors pour $i\geq 2$, on a
$$|b_{ii}|>\sum_{j=2,j\neq i}^{n}|b_{ij}| .$$}

On a $$b_{ij}=a_{ij}-\frac{{a_{1j}}{a_{11}}}a_{i1}\ \ {\hbox{si}}\ \ j\geq2\ \ {\hbox{et}}\ \ b_{i1}=a_{i1}.$$
 par l'in\'egalit\'e triangulaire, on a
 \begin{align*}\sum_{j=2,j\neq i}|b_{ij}|&=\sum_{j=2,j\neq i}|a_{ij}-{\frac{a_{1j}}{a_{11}}}a_{i1}| \\ 
&\leq \sum_{j=2,j\neq i}|a_{ij}|+\frac{|{a_{1j}|}{|a_{11}|}}|a_{i1}| \\ 
&= \sum_{j=2,j\neq i}|a_{ij}|+\frac{|{a_{i1}|}{|a_{11}|}} \sum_{j=2,j\neq i}|a_{1j}|. 
\end{align*}
Mais, par hypoth\`ese, pour $i=1$, on a
$$\sum_{j=2}^{n}|a_{1j}|<|a_{11}|,$$
donc
$$\sum_{j=2,j\neq i}^{n}|a_{1j}|<|a_{11}|-|a_{1i}|.$$
D'o\`u, en rempla\c cant dans l'in\'egalit\'e pr\'ec\'edente
\begin{align*}\sum_{j=2,j\neq i}|b_{ij}|
&< \sum_{j=2,j\neq i}|a_{ij}|+ |a_{i1}|-\frac{{|a_{i1}|}{|a_{11}|}}|a_{1i}| \\ 
&= \sum_{j=1,j\neq i}|a_{ij}|-\frac{{|a_{i1}|}{|a_{11}|}}|a_{1i}| \\ 
&<|a_{ii}|-\frac{{|a_{i1}|}{|a_{11}|}}|a_{1i}| \\ 
&\leq \left|a_{ii}-\frac{{a_{i1}}{a_{11}}}a_{1i}\right|=|b_{ii}|.
\end{align*}}
    \item \question{D\'emontrer le r\'esultat de Hadamard pour $n$ quelconque.}
\reponse{{\it  D\'emontrons le r\'esultat de Hadamard pour $n$ quelconque.}

Soit $A=(a_{ij})_{1\leq i,j\leq n}$ une matrice carr\'ee $n\times n$, v\'erifiant pour tout $i\in\{1,\cdots,n\}$, 
$$|a_{ii}|>\sum_{j=1,j\neq i}^{n}|a_{ij}| $$
On veut d\'emontrer que $A$ est inversible.

Le r\'esultat est vrai pour $n=2$, d'apr\`es la question $1)$. Soit $n$ arbitrairement fix\'e, supposons le r\'esultat vrai pour
$n-1$ et d\'emontrons le pour $n$.

On a $\det A=\det B$ o\`u $B$ est la matrice construite dans la question $2)$
$$B=\begin{pmatrix}a_{11}&0&\cdots&0 \\ 
\vdots& &({b_{ij}}_{(2\leq i,j\leq n)}) \\ 
a_{n1}\end{pmatrix}$$
Or, la matrice $({b_{ij}}_{(2\leq i,j\leq n)})$ est une matrice carr\'ee d'ordre $n-1$ qui v\'erifie les hypoth\`eses de
Hadamard, d'apr\`es la question $2)$. Elle est donc inversible par hypoth\`ese de r\'ecurrence. Et, par cons\'equent, la 
matrice $A$ est inversible car $a_{11}\neq 0$.}
\end{enumerate}
}

\uuid{H2i9}
\exo7id{974}
\titre{exo7 974}
\auteur{gourio}
\organisation{exo7}
\datecreate{2001-09-01}
\video{6N7D6lPLPHc}
\isIndication{true}
\isCorrection{true}
\chapitre{Application linéaire}
\sousChapitre{Morphismes particuliers}
\module{Algèbre}
\niveau{L1}
\difficulte{}

\contenu{
\texte{
Soit $E$ l'espace vectoriel des fonctions de $\Rr$ dans $\Rr$. Soient $P$ le
sous-espace des fonctions paires et $I$ le sous-espace des fonctions
impaires. Montrer que $E=P\bigoplus I$. Donner l'expression du projecteur sur
$P$ de direction $I$.
}
\indication{Pour une fonction $f$ on peut \'ecrire 
$$f(x)= \frac{f(x)+f(-x)}{2}+\frac{f(x)-f(-x)}{2}.$$

Le projecteur sur $P$ de direction $I$ est l'application $\pi : E \longrightarrow E$
qui vérifie $\pi(f)\in P$, $\pi \circ \pi = \pi$ et $\ker \pi = I$.}
\reponse{
La seule fonction qui est \`a la fois paire et impaire est la fonction nulle : $P\cap I = \{0\}$. Montrons qu'une fonction $f:\Rr \longrightarrow \Rr$ se d\'ecompose en une fonction paire et une fonction impaire. 
En effet : 
 $$f(x)= \frac{f(x)+f(-x)}{2}+\frac{f(x)-f(-x)}{2}.$$
La fonction $x \mapsto \frac{f(x)+f(-x)}{2}$ est paire (le v\'erifier !),
la fonction $x \mapsto \frac{f(x)-f(-x)}{2}$ est impaire.
Donc $P+I=E$.
Bilan : $E=P\oplus I.$
Le projecteur sur $P$ de direction $I$ est l'application $\pi : E \longrightarrow E$
    qui \`a $f$ associe la fonction $x \mapsto \frac{f(x)+f(-x)}{2}$, c'est-à-dire à $f$ on associe la partie paire
 de $f$.
Nous avons bien 
\begin{itemize}
$\pi(f)\in P$. Par définition de $\pi$, $\pi(f)$ est bien une fonction paire.
$\pi \circ \pi = \pi$. Si $g$ est une fonction paire alors $\pi(g)=g$. 
Appliquons ceci avec $g=\pi(f)$ (qui est bien est une fonction paire) donc $\pi(\pi(f))=\pi(f)$.
$\ker \pi = I$. Si $\pi(f)=0$ alors cela signifie exactement que la fonction $x \mapsto \frac{f(x)+f(-x)}{2}$
est la fonction nulle. Donc pour tout $x$ : $\frac{f(x)+f(-x)}{2}=0$ donc $f(x)=-f(-x)$ ; 
cela implique que $f$ est une fonction impaire. Réciproquement si $f\in I$ est une fonction impaire, 
sa partie paire est nulle donc $f\in \ker f$.
\end{itemize}
}
}

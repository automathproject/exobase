\uuid{GJz3}
\exo7id{216}
\titre{exo7 216}
\auteur{ridde}
\organisation{exo7}
\datecreate{1999-11-01}
\isIndication{false}
\isCorrection{false}
\chapitre{Logique, ensemble, raisonnement}
\sousChapitre{Relation d'équivalence, relation d'ordre}
\module{Algèbre}
\niveau{L1}
\difficulte{}

\contenu{
\texte{
Un ensemble est dit bien ordonn\'e si toute partie non vide admet un plus petit
\'el\'ement.
}
\begin{enumerate}
    \item \question{Donner un exemple d'ensemble bien ordonn\'e et un exemple d'ensemble qui ne
l'est pas.}
    \item \question{Montrer que bien ordonn\'e implique totalement ordonn\'e.}
    \item \question{La r\'eciproque est-elle vraie ?}
\end{enumerate}
}

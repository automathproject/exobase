\uuid{iIik}
\exo7id{5178}
\titre{exo7 5178}
\auteur{rouget}
\organisation{exo7}
\datecreate{2010-06-30}
\isIndication{false}
\isCorrection{true}
\chapitre{Espace vectoriel}
\sousChapitre{Somme directe}
\module{Algèbre}
\niveau{L1}
\difficulte{}

\contenu{
\texte{
Soient $u=(1,1,...,1)$ et $F=\mbox{Vect}(u)$ puis $G=\{(x_1,...,x_n)\in\Rr^n/\;x_1+...+x_n=0\}$. Montrer que $G$ est
un sous-espace vectoriel de $\Rr^n$ et que $\Rr^n=F\oplus G$.
}
\reponse{
$F=\mbox{Vect}(u)$ est un sous espace vectoriel de $\Rr^n$ et $G$ est un sous espace vectoriel de $\Rr^n$, car est le
noyau de la forme linéaire $(x_1,...,x_n)\mapsto x_1+...+x_n$.

Soit $x=(x_1,...,x_n)\in\Rr^n$ et soit $\lambda\in\Rr$.

$$x-\lambda u\in G\Leftrightarrow(x_1-\lambda,...,x_n-\lambda)\in
G\Leftrightarrow\sum_{k=1}^{n}(x_k-\lambda)=0\Leftrightarrow\lambda=\frac{1}{n}\sum_{k=1}^{n}x_k.$$

Donc,

$$\forall x\in\Rr^n,\;\exists!\lambda\in\Rr/\;x-\lambda u\in G,$$

et donc,

$$\Rr^n=F\oplus G.$$

Le projeté sur $F$ parallèlement à $G$ d'un vecteur $x=(x_1,...,x_n)$ est

$$\frac{1}{n}\sum_{k=1}^{n}x_k.u=(\frac{1}{n}\sum_{k=1}^{n}x_k,...,\frac{1}{n}\sum_{k=1}^{n}x_k)$$

et le projeté du même vecteur sur $G$ parallèlement à $F$ est

$$x-(\frac{1}{n}\sum_{k=1}^{n}x_k).u =(x_1-\frac{1}{n}\sum_{k=1}^{n}x_k,...,x_n-\frac{1}{n}\sum_{k=1}^{n}x_k).$$
}
}

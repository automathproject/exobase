\uuid{kiCw}
\exo7id{410}
\titre{exo7 410}
\auteur{cousquer}
\organisation{exo7}
\datecreate{2003-10-01}
\video{ywVn-MdHV0A}
\isIndication{false}
\isCorrection{true}
\chapitre{Polynôme, fraction rationnelle}
\sousChapitre{Racine, décomposition en facteurs irréductibles}
\module{Algèbre}
\niveau{L1}
\difficulte{}

\contenu{
\texte{
Pour quelles valeurs de $a$ le polynôme $(X+1)^7-X^7-a$ admet-il une racine multiple réelle?
}
\reponse{
Soit $x\in\R$ ; $x$ est une racine multiple de $P$ si et seulement si $P(x)=0$ et $P'(x)=0$:
$$
\begin{array}{rcl}
P(x)=P'(x) 0 
&\iff& \left\{\begin{array}{l}(x+1)^7-x^7-a=0\\7(x+1)^6-7x^6=0\end{array}\right.\\
&\iff& \left\{\begin{array}{l}(x+1)x^6-x^7-a=0\qquad \text{ en utilisant la deuxième équation}\\(x+1)^6=x^6\end{array}\right.\\
&\iff& \left\{\begin{array}{l}x^6=a\\(x+1)^3=\pm x^3 \qquad \text{ en prenant la racine carrée} \end{array}\right.\\
&\iff& \left\{\begin{array}{l}x^6=a\\x+1=\pm x \qquad \text{ en prenant la racine cubique} \end{array}\right.\\
\end{array}$$
qui admet une solution ($x=-\frac{1}{2}$) si et seulement si $a=\frac{1}{64}$.
}
}

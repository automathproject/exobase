\uuid{BvV7}
\exo7id{920}
\auteur{cousquer}
\organisation{exo7}
\datecreate{2003-10-01}
\video{ByyFihncOvA}
\isIndication{true}
\isCorrection{true}
\chapitre{Espace vectoriel}
\sousChapitre{Somme directe}

\contenu{
\texte{
On consid\`ere les vecteurs
$v_1=(1,0,0,1)$, $v_2=(0,0,1,0)$, $v_3=(0,1,0,0)$, $v_4=(0,0,0,1)$,
$v_5=(0,1,0,1)$ dans $\mathbb{R}^4$.
}
\begin{enumerate}
    \item \question{$\mbox{Vect}\{v_1,v_2\}$ et $\mbox{Vect}\{v_3\}$ sont-ils
  suppl\'ementaires dans $\mathbb{R}^4$ ?}
\reponse{Non. 
Tout d'abord par définition
$\text{Vect}\{v_1,v_2\}+\text{Vect}\{v_3\} = \text{Vect}\{v_1,v_2,v_3\}$,
Nous allons trouver un vecteur de $\Rr^4$ qui n'est pas dans 
$\text{Vect}\{v_1,v_2\}+\text{Vect}\{v_3\}$.
Il faut tâtonner un peu pour le choix, par exemple faisons le calcul avec $u=(0,0,0,1)$.

$u \in \text{Vect}\{v_1,v_2,v_3\}$ si et seulement si 
il existe des réels $\alpha,\beta,\gamma$ tels que $u=\alpha v_1 + \beta v_2 + \gamma v_3$.
Si l'on écrit les vecteurs verticalement, on cherche donc $\alpha,\beta,\gamma$ tels que :
$$\begin{pmatrix} 0 \\ 0 \\ 0 \\1\end{pmatrix}= \alpha \begin{pmatrix} 1 \\ 0 \\ 0 \\1\end{pmatrix}
+ \beta \begin{pmatrix} 0 \\ 0 \\ 1 \\ 0 \end{pmatrix}+\gamma\begin{pmatrix} 0 \\ 1 \\ 0 \\0\end{pmatrix}$$
Ce qui est équivalent à trouver $\alpha,\beta,\gamma$ vérifiant le système linéaire :
$$\begin{cases}
0 = \alpha \cdot 1 + \beta \cdot 0 + \gamma \cdot 0   \\
0 = \alpha \cdot 0 + \beta \cdot 0 + \gamma \cdot 1   \\
0 = \alpha \cdot 0 + \beta \cdot 1 + \gamma \cdot 0   \\
1 = \alpha \cdot 1 + \beta \cdot 0 + \gamma \cdot 0   \\
\end{cases}
\text{ qui équivaut à } 
\begin{cases}
0 = \alpha \\
0 = \gamma \\
0 = \beta  \\
1 = \alpha \\
\end{cases}
$$
Il n'y a clairement aucune solution à ce système (les trois premières lignes impliquent $\alpha=\beta=\gamma=0$
et cela rentre alors en contradiction avec la quatrième).

\bigskip


Un autre type de raisonnement, beaucoup plus rapide, 
est de dire que ces deux espaces ne peuvent engendr\'es tout $\Rr^4$ 
car il n'y pas assez de vecteurs en effet $3$ vecteurs ne peuvent engendrer 
l'espace $\Rr^4$ de dimension $4$.}
    \item \question{$\mbox{Vect}\{v_1,v_2\}$ et $\mbox{Vect}\{v_4, v_5\}$ sont-ils
  suppl\'ementaires dans $\mathbb{R}^4$ ?}
\reponse{Oui. Notons $F=\mbox{Vect}\{v_1,v_2\}$ et $G=\mbox{Vect}\{v_4, v_5\}$.
Pour montrer $F \oplus G = \Rr^4$ il faut montrer $F\cap G =\{ (0,0,0,0) \}$ et
$F+G=\Rr^4$.

  \begin{enumerate}}
    \item \question{$\mbox{Vect}\{v_1,v_3,v_4\}$ et  $\mbox{Vect}\{v_2,v_5\}$ sont-ils
  suppl\'ementaires dans $\mathbb{R}^4$ ?}
\reponse{Montrons $F\cap G =\{ (0,0,0,0 \}$. Soit $u \in F\cap G$,
d'une part $u \in F = \mbox{Vect}\{v_1,v_2\}$ donc il existe $\alpha,\beta \in \Rr$ tels que
$u = \alpha v_1+\beta v_2$. D'autre part $u \in G = \mbox{Vect}\{v_4, v_5\}$ donc il existe
$\gamma,\delta \in \Rr$ tels que $u=\gamma v_4 + \delta v_5$. On a écrit $u$ de deux façons donc on a l'égalité
$\alpha v_1+\beta v_2 = \gamma v_4 + \delta v_5$. En écrivant les vecteurs comme des vecteurs colonnes cela donne
$$\alpha \begin{pmatrix} 1 \\ 0 \\ 0 \\1\end{pmatrix}
+ \beta \begin{pmatrix} 0 \\ 0 \\ 1 \\ 0 \end{pmatrix} 
= \gamma \begin{pmatrix} 0 \\ 0 \\ 0 \\1\end{pmatrix} + \delta \begin{pmatrix} 0 \\ 1 \\ 0 \\1\end{pmatrix}$$
Donc $(\alpha,\beta,\gamma,\delta)$ est solution du système linéaire suivant :
$$\begin{cases}
\alpha = 0   \\
0 = \delta \\
\beta = 0 \\
\alpha =  \gamma + \delta \\
\end{cases}
$$
Cela implique $\alpha=\beta=\gamma=\delta=0$ et donc $u = (0,0,0,0)$.
Ainsi le seul vecteur de $F\cap G$ est le vecteur nul.}
    \item \question{$\mbox{Vect}\{v_1,v_4\}$ et $\mbox{Vect}\{v_3, v_5\}$ sont-ils
  suppl\'ementaires dans $\mathbb{R}^4$ ?}
\reponse{Montrons $F+G=\Rr^4$.
$F+G = \mbox{Vect}\{v_1,v_2\}+\mbox{Vect}\{v_4, v_5\}= \mbox{Vect}\{v_1,v_2,v_4, v_5\}$.
Il faut donc montrer que n'importe quel vecteur $u = (x_0,y_0,z_0,t_0)$ de $\Rr^4$ s'écrit comme une combinaison linéaire
de $v_1,v_2,v_4,v_5$. Fixons $u$ et cherchons $\alpha,\beta,\gamma,\delta \in \Rr$ tels que 
$\alpha v_1+\beta v_2 + \gamma v_4 + \delta v_5=u$.
Après avoir considéré les vecteurs comme des vecteurs colonnes cela revient à résoudre le système linéaire :
$$\begin{cases}
\alpha = x_0 \\
\delta = y_0 \\
\beta = z_0 \\
\alpha + \gamma + \delta = t_0 \\    
  \end{cases}$$
Nous étant donné un vecteur $u = (x_0,y_0,z_0,t_0)$ on a calculé qu'en choisissant 
$\alpha = x_0$, $\beta=z_0$, $\gamma= t_0 -x_0-y_0$, $\delta=y_0$ on obtient bien
$\alpha v_1+\beta v_2 + \gamma v_4 + \delta v_5=u$.
Ainsi tout vecteur est engendré par $F+G$.}
\indication{\begin{enumerate}
\item Non.
\item Oui.
\item Non.
\item Non. 
\end{enumerate}}
\end{enumerate}
}

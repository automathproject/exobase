\uuid{MgN5}
\exo7id{236}
\auteur{bodin}
\organisation{exo7}
\datecreate{1998-09-01}
\video{zkb9m4IjyfM}
\isIndication{true}
\isCorrection{true}
\chapitre{Dénombrement}
\sousChapitre{Cardinal}

\contenu{
\texte{
Pour $A,B$ deux ensembles de $E$ on note $A\Delta
B= (A\cup B )\setminus(A\cap B)$. Pour $E$ un ensemble fini,
montrer :
$$\text{Card\,} A\Delta B = \text{Card\,} A + \text{Card\,} B -2\text{Card\,} A\cap B.$$
}
\indication{Tout d'abord faire un dessin (avec des patates !).

Pour $A$ et $B$ deux ensembles finis quelconques, commencer par (re)démontrer la formule : 
$\text{Card\,} A\cup B = \text{Card\,} A + \text{Card\,} B - \text{Card\,} A\cap B$.}
\reponse{
Tout d'abord si deux ensembles finis $A$ et $B$ sont disjoints
alors $\text{Card\,} A\cup B = \text{Card\,} A + \text{Card\,} B$.

Si maintenant $A$ et $B$ sont deux ensembles finis quelconques : nous décomposons $A\cup B$ en trois ensembles :
$$A \cup B = (A \setminus (A\cap B)) \cup (B \setminus (A\cap B))  \cup (A\cap B).$$
Ces trois ensembles sont disjoints deux à deux donc :
$\text{Card\,} A\cup B = \text{Card\,} A\setminus (A\cap B) + \text{Card\,} B \setminus (A\cap B) + \text{Card\,} A\cap B$.

Mais pour $R \subset S$ nous avons
$\text{Card\,} S\setminus R = \text{Card\,} S - \text{Card\,} R$.

Donc 
$\text{Card\,} A\cup B = \text{Card\,} A - \text{Card\,} A\cap B  + \text{Card\,}B - \text{Card\,} A\cap B  + \text{Card\,} A\cap B$.

Donc $\text{Card\,} A\cup B = \text{Card\,} A + \text{Card\,} B - \text{Card\,} A\cap B$.


Appliquons ceci à $A \Delta B  = (A \cup B) \setminus (A \cap B)$ :
$$ \text{Card\,} A \Delta B = \text{Card\,} A\cup B - \text{Card\,} A\cap B =
\text{Card\,} A + \text{Card\,} B - 2\text{Card\,} A\cap B.$$
}
}

\uuid{sq4d}
\exo7id{2567}
\titre{exo7 2567}
\auteur{delaunay}
\organisation{exo7}
\datecreate{2009-05-19}
\isIndication{false}
\isCorrection{true}
\chapitre{Réduction d'endomorphisme, polynôme annulateur}
\sousChapitre{Diagonalisation}
\module{Algèbre}
\niveau{L2}
\difficulte{}

\contenu{
\texte{
Soit $$A=\begin{pmatrix}1&1&-1 \\ 0&1&0 \\ 1&0&1\end{pmatrix}$$
Factoriser le polyn\^ome caract\'eristique de $A$. La matrice $A$ est-elle diagonalisable dans $\R$~? dans $\C$ ?
}
\reponse{
Soit $$A=\begin{pmatrix}1&1&-1 \\ 0&1&0 \\ 1&0&1\end{pmatrix}$$
{\it Factorisons le polyn\^ome caract\'eristique de $A$}. 
$$P_A(X)=\begin{vmatrix}1-X&1&-1 \\  0&1-X&0 \\ 1&0&1-X\end{vmatrix}=(1-X)^3+(1-X)=(1-X)((1-X)^2+1)=(1-X)(X^2-2X+2)$$
factorisons maintenant le polyn\^ome $X^2-2X+2$, le discriminant r\'eduit $\Delta'=1-2=-1$, ce polyn\^ome n'admet donc pas de racines r\'eelles, mais 
deux racines complexes conjugu\'ees qui sont : $1+i$ et $1-i$. On a $P_A(X)=(1-X)(1-i-X)(1+i-X)$.

La matrice $A$ n'est pas diagonalisable dans $\R$ car son polyn\^ome caract\'eristique n'a pas toutes ses racines dans $\R$, elle est diagonalisable dans
$\C$ car c'est une matrice $3\times 3$ qui admet trois valeurs propres distinctes.
}
}

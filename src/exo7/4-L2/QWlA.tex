\uuid{QWlA}
\exo7id{1548}
\titre{exo7 1548}
\auteur{barraud}
\organisation{exo7}
\datecreate{2003-09-01}
\isIndication{false}
\isCorrection{false}
\chapitre{Endomorphisme particulier}
\sousChapitre{Endomorphisme orthogonal}
\module{Algèbre}
\niveau{L2}
\difficulte{}

\contenu{
\texte{
Soit $\mathcal{B}=(e_{1},\ldots,e_{n})$ une base orthogonal d'un espace euclidien $E$. On
dit qu'un endomorphisme $f$ de $E$ conserve l'orthogonalité de $\mathcal{B}$ si et
seulement si $(f(e_{1}),\ldots,f(e_{n}))$ est une famille orthogonale.

Montrer que $f$ conserve l'orthogonalité de $\mathcal{B}$ si et seulement si $\mathcal{B}$
est une base de vecteurs propres de ${}^t{f}\,f$.

Montrer que pour tout endomorphisme $f$ de $E$, il existe une base orthogonale dont $f$
conserve l'orthogonalité.
}
}

\uuid{OzpF}
\exo7id{7368}
\titre{exo7 7368}
\auteur{mourougane}
\organisation{exo7}
\datecreate{2021-08-10}
\isIndication{false}
\isCorrection{false}
\chapitre{Groupe, anneau, corps}
\sousChapitre{Anneau}
\module{Algèbre}
\niveau{L2}
\difficulte{}

\contenu{
\texte{
\textit{Le but de ce problème est de montrer que l'équation $X^2+X+1=0$ peut avoir un nombre arbitrairement grand de solutions dans les anneaux $\Z/n\Z$.}


\textit{Pour un nombre premier $p$, on notera $\mathbb{F}_p$ le corps $\Z/p\Z$.
On rappelle qu'alors le groupe $(\mathbb{F}_p^\times,\times)$ des inversibles de $\mathbb{F}_p$ est cyclique.
On pourra admettre le résultat d'une question pour continuer.}
}
\begin{enumerate}
    \item \question{Si $p$ est un nombre premier, combien au maximum l'équation $X^2+X+1=0$ a-t-elle de solutions dans l'anneau $\Z/p\Z$ ?}
    \item \question{Déterminer les solutions de l'équation $X^2+X+1=0$ 
\begin{enumerate}}
    \item \question{[a] dans $\Z/2\Z$ , dans $\Z/2n\Z$.}
    \item \question{[b] dans $\Z/7\Z$.}
    \item \question{[c] dans $\Z/13\Z$.}
    \item \question{[d] \`A l'aide des questions précédentes, déterminer les solutions de l'équation\\ $X^2+X+1=0$ dans $\Z/91\Z$.}
\end{enumerate}
}

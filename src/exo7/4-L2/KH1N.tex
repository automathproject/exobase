\uuid{KH1N}
\exo7id{1650}
\titre{exo7 1650}
\auteur{legall}
\organisation{exo7}
\datecreate{1998-09-01}
\isIndication{false}
\isCorrection{false}
\chapitre{Réduction d'endomorphisme, polynôme annulateur}
\sousChapitre{Diagonalisation}
\module{Algèbre}
\niveau{L2}
\difficulte{}

\contenu{
\texte{
Soient $\ M  $ et $  N \in M_n (\mathbb{K} )  .$ On note $  \varphi _M \in \mathcal{L}
(M_n (\mathbb{K} ))  $ l'application $  N\mapsto  MN-NM   .$
}
\begin{enumerate}
    \item \question{Soient $  A= \begin{pmatrix}3 &
-4\cr
                                      2 & -3 \cr \end{pmatrix} $ et $  B= \begin{pmatrix}1 & 2\cr
                                      0 & 1 \cr \end{pmatrix} .$ Diagonaliser $  A  $ et
montrer que $  B   $ n'est pas diagonalisable.}
    \item \question{Montrer que si $  N  $ est un vecteur propre associ\'e \`a une
valeur propre non nulle $  \lambda   $ de $  \varphi _M   $ alors $  N   $ est nilpotente. (on
pourra \'etablir que pour tout $  k \in \N   :   MN^k-N^kM=k\lambda N^k  $.)}
    \item \question{Montrer que l'identit\'e n'appartient pas \`a l'image de $  \varphi _M   .$ (utiliser la trace.)}
    \item \question{Soit $  D=\begin{pmatrix}1 & 0  \cr
                                      0 & -1 \cr \end{pmatrix} .$ Diagonaliser $  \varphi _D
  $ puis $  \varphi _A  .$ Montrer que
$  \varphi _B  $ n'est pas diagonalisable.}
    \item \question{Montrer que si $  M  $ est diagonalisable, $  \varphi _M   $ est diagonalisable.}
    \item \question{Etablir la r\'eciproque lorsque $  M  $ a au moins une valeur propre.}
\end{enumerate}
}

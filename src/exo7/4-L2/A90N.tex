\uuid{A90N}
\exo7id{5616}
\titre{exo7 5616}
\auteur{rouget}
\organisation{exo7}
\datecreate{2010-10-16}
\isIndication{false}
\isCorrection{true}
\chapitre{Déterminant, système linéaire}
\sousChapitre{Autre}
\module{Algèbre}
\niveau{L2}
\difficulte{}

\contenu{
\texte{
Résoudre dans $\mathcal{M}_n(\Rr)$ l'équation $M=\text{com}M$ ($n\geqslant2$).
}
\reponse{
Si $\text{rg} M\leqslant n-1$, l'égalité $M=\text{com}M$ entraîne $M{^t}M=M{^t}(\text{com}M)=(\text{det}M)I_n=0$ et donc $M = 0$. En effet, 

\begin{align*}\ensuremath
M{^t}M=0&\Rightarrow\forall X\in\mathcal{M}_{n,1}(\Rr),\;M{^t}MX=0\Rightarrow\forall X\in\mathcal{M}_{n,1}(\Rr),\;{^t}XM{^t}MX = 0\Rightarrow\forall X\in\mathcal{M}_{n,1}(\Rr),\;\left\|{^t}MX\right\|^2=0\\
 &\Rightarrow\forall X\in\mathcal{M}_{n,1}(\Rr),\;{^t}MX = 0\Rightarrow{^t}M=0\Rightarrow M=0.
\end{align*}

En résumé, si $M$ est solution, $M=0$ ou $M$ est inversible.

Dans le deuxième cas, d'après l'exercice \ref{ex:rou17}, on doit avoir $\text{det}M=(\text{det}M)^{n-1}$ et donc, puisque $\text{det}M\neq0$, $\text{det}M\in\{-1,1\}$ (et même $\text{det}M=1$ si $n$ est impair) car $\text{det}M$ est réel.

\textbullet~Si $\text{det}M=-1$, on doit avoir $M{^t}M=-I_n$ mais ceci est impossible car le coefficient ligne $1$, colonne $1$, de la matrice $M{^t}M$ vaut $m_{1,1}^2+...+m_{1,n}^2\neq-1$.

\textbullet~Il reste le cas où $\text{det}M=1$, l'égalité $M=\text{com}M$ entraîne $M{^t}M=I_n$ c'est-à-dire $M$ est orthogonale positive.

Réciproquement, si $M$ est orthogonale positive, ${^t}M=M^{-1}=\frac{1}{\text{det}M}{^t}(\text{com}M)={^t}\text{com}M$ et donc $M=\text{com}M$.

Finalement ,

\begin{center}
\shadowbox{
$\mathcal{S}=\{0\}\cup O_n^+(\Rr)$.
}
\end{center}
}
}

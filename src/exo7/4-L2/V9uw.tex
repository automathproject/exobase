\uuid{V9uw}
\exo7id{5791}
\titre{exo7 5791}
\auteur{rouget}
\organisation{exo7}
\datecreate{2010-10-16}
\isIndication{false}
\isCorrection{true}
\chapitre{Espace euclidien, espace normé}
\sousChapitre{Problèmes matriciels}
\module{Algèbre}
\niveau{L2}
\difficulte{}

\contenu{
\texte{
Soit $A$ une matrice orthogonale. Montrer que la valeur absolue de la somme des coefficients de $A$ est inférieure ou égale à $n$. Cas d'égalité si de plus tous les coefficients de $A$ sont positifs ?
}
\reponse{
Soit $A =(a_{i,j})_{1\leqslant i,j\leqslant n}$ une matrice orthogonale. On pose $X =\left(
\begin{array}{c}
1\\
\vdots\\
1
\end{array}
\right)\in\mathcal{M}_{n,1}(\Rr)$.

D'après l'inégalité de \textsc{Cauchy}-\textsc{Schwarz}

\begin{align*}\ensuremath
\left|\sum_{1\leqslant i,j\leqslant n}^{}a_{i,j}\right|&=\left|\sum_{1\leqslant i,j\leqslant n}^{}1\times a_{i,j}\times1\right|=\left|{^t}XAX\right| =\left|\left(AX|X\right)\right|\\
 &\leqslant\|AX\|\|X\|\;(\text{d'après l'inégalité de \textsc{Cauchy}-\textsc{Schwarz}})\\
 &=\|X\|^2\;(\text{puisque la matrice}\;A\;\text{est orthogonale})\\
 &=n.
\end{align*}
	        

On a l'égalité si et seulement si la famille $(X,AX)$ est liée ce qui équivaut à $X$ vecteur propre de $A$.

On sait que les valeurs propres (réelles) de $A$ ne peuvent être que $1$ ou $-1$. Donc, 

\begin{center}
égalité $\Leftrightarrow AX = X\;\text{ou}\;AX = -X\Leftrightarrow \forall i\in\llbracket1,n\rrbracket,\;\left|\sum_{j=1}^{n}a_{i,j}\right|= 1$
\end{center}

Il paraît difficile d'améliorer ce résultat dans le cas général. Supposons de plus que $\forall(i,j)\in\llbracket1,n\rrbracket^2$, $a_{i,j}\geqslant 0$. Soit $i\in\llbracket1,n\rrbracket$. Puisque tous les $a_{i,j}$ sont éléments de $[0,1]$,

\begin{center}
$1=\sum_{j=1}^{n}a_{i,j}\geqslant\sum_{j=1}^{n}a_{i,j}^2=1$.
\end{center}

L'inégalité écrite est donc une égalité et on en déduit que chaque inégalité $a_{i,j}\geqslant a_{i,j}^2$, $1\leqslant j\leqslant n$, est une égalité. Par suite, $\forall(i,j)\in\llbracket1,n\rrbracket^2$, $a_{i,j}\in\{0,1\}$. Ceci montre que la matrice $A$ est une matrice de permutation qui réciproquement convient.
}
}

\uuid{fLIk}
\exo7id{3677}
\auteur{quercia}
\organisation{exo7}
\datecreate{2010-03-11}
\isIndication{false}
\isCorrection{false}
\chapitre{Espace euclidien, espace normé}
\sousChapitre{Produit scalaire, norme}

\contenu{
\texte{

}
\begin{enumerate}
    \item \question{Soit $M \in \mathcal{M}_n(\R)$ inversible. Montrer qu'il existe une matrice orthogonale, $P$,
    et une matrice triangulaire supérieure à coefficients diagonaux positifs, $T$,
    uniques telles que $M = PT$.}
    \item \question{Application : inégalité de Hadamard. Soit $E$ un espace vectoriel euclidien,
    $(\vec e_1,\dots,\vec e_n)$ une
    base orthonormée, et $\vec u_1,\dots,\vec u_n$ des vecteurs quelconques.

    Démontrer que $|\det_{(\vec e_i)}(\vec u_j)| \le \prod_j \|\vec u_j\|$.
    \'Etudier les cas d'égalité.}
\end{enumerate}
}

\uuid{9g5V}
\exo7id{5353}
\titre{exo7 5353}
\auteur{rouget}
\organisation{exo7}
\datecreate{2010-07-04}
\isIndication{false}
\isCorrection{true}
\chapitre{Groupe, anneau, corps}
\sousChapitre{Groupe de permutation}
\module{Algèbre}
\niveau{L2}
\difficulte{}

\contenu{
\texte{
Soit $\sigma$ l'élément de $S_{12}$~:~$\sigma=(3\;10\;7\;1\;2\;6\;4\;5\;12\;8\;9\;11)$.
}
\begin{enumerate}
    \item \question{Combien $\sigma$ possède-t-elle d'inversions~?~Que vaut sa signature~?}
\reponse{Les inversions de $\sigma$ sont :
$\sigma=(3\;10\;7\;1\;2\;6\;4\;5\;12\;8\;9\;11)$. 

$$\begin{array}{l}
\{1,4\},\;\{1,5\},\;\{2,3\},\;\{2,4\},\;\{2,5\},\;\{2,6\},\;\{2,7\},\;\{2,8\},\;\{2,10\},\;\{2,11\},\;\{3,4\},\;\{3,5\},\;\{3,6\},\;\{3,7\},\\
\{3,8\},\;\{6,7\},\;\{6,8\},\;\{9,10\},\;\{9,11\},\;\{9,12\}.
\end{array}$$

Au total, il y a $2+8+5+2+3=20$ inversions. $\sigma$ est donc une permutation paire (de signature $1$).}
    \item \question{Décomposer $\sigma$ en produit de transpositions. Retrouvez sa signature.}
\reponse{$\tau_{11,12}\circ\sigma=(3\;10\;7\;1\;2\;6\;4\;5\;11\;8\;9\;12)$. 

Puis, $\tau_{9,11}\circ\tau_{11,12}\circ\sigma=(3\;10\;7\;1\;2\;6\;4\;5\;9\;8\;11\;12)$. 

Puis, $\tau_{10,8}\circ\tau_{9,11}\circ\tau_{11,12}\circ\sigma=(3\;8\;7\;1\;2\;6\;4\;5\;9\;10\;11\;12)$. 

Puis, 
$\tau_{8,5}\circ\tau_{10,8}\circ\tau_{9,11}\circ\tau_{11,12}\circ\sigma=(3\;5\;7\;1\;2\;6\;4\;8\;9\;10\;11\;12)$. 

Puis, $\tau_{7,4}\circ\tau_{8,5}\circ\tau_{10,8}\circ\tau_{9,11}\circ\tau_{11,12}\circ\sigma=(3\;5\;4\;1\;2\;6\;7\;8\;9\;10\;11\;12)$. 

Puis, $\tau_{5,2}\circ\tau_{7,4}\circ\tau_{8,5}\circ\tau_{10,8}\circ\tau_{9,11}\circ\tau_{11,12}\circ\sigma=(3\;2\;4\;1\;5\;6\;7\;8\;9\;10\;11\;12)$.

Puis, $\tau_{1,4}\circ\tau_{5,2}\circ\tau_{7,4}\circ\tau_{8,5}\circ\tau_{10,8}\circ\tau_{9,11}\circ\tau_{11,12}\circ\sigma=(3\;2\;1\;4\;5\;6\;7\;8\;9\;10\;11\;12)=\tau_{1,3}$.

Par suite,

$$\sigma=\tau_{11,12}\circ\tau_{9,11}\circ\tau_{10,8}\circ\tau_{8,5}\circ\tau_{7,4}\circ\tau_{5,2}\circ\tau_{1,4}\circ\tau_{1,3}.$$}
    \item \question{Déterminer les orbites de $\sigma$.}
\reponse{$O(1)=\{1,3,4,7\}=O(3)=O(4)=O(7)$, puis $O(2)=\{2,5,8,10\}$ puis $O(6)=\{6\}$ et $O(9)=\{9,11,12\}=O(11)=O(12)$. $\sigma$ a $4$ orbites, deux de cardinal $4$, une de cardinal $3$ et un singleton (correspondant à un point fixe).}
    \item \question{Déterminer $\sigma^{2005}$.}
\reponse{$\sigma$ est donc le produit commutatif des cycles $c_1=\left(\begin{array}{cccc}
1&3&4&7\\
3&7&7&4
\end{array}
\right)$, $c_2=\left(\begin{array}{cccc}
2&5&8&10\\
10&2&5&8
\end{array}
\right)$ et

$c_3=\left(\begin{array}{ccc}
9&11&12\\
12&9&11
\end{array}
\right)$.

On a $c_1^4=c_2^4=Id$ et $c_3^3=Id$. Or, $2005=4.1001+1$. Donc, $c_1^{2005}=c_1(c_1^4)^{1001}=c_1$, et de même $c_2^{2005}=c_2$. Puis, $c_3^{2005}=(c_3^{3})^{668}c_3=c_3$. Puisque $c_1$, $c_2$ et $c_3$ commutent, 

$$\sigma^{2005}=c_1^{2005}c_2^{2005}c_3^{2005}=c_1c_2c_3=\sigma=(3\;10\;7\;1\;2\;6\;4\;5\;12\;8\;9\;11).$$}
\end{enumerate}
}

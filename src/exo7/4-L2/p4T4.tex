\uuid{p4T4}
\exo7id{3079}
\auteur{quercia}
\organisation{exo7}
\datecreate{2010-03-08}
\isIndication{false}
\isCorrection{false}
\chapitre{Groupe, anneau, corps}
\sousChapitre{Groupe de permutation}

\contenu{
\texte{
Soit $\sigma \in S_n$. On appelle {\it orbite de $\sigma$\/} toute partie
$X$ de $\{1,\dots,n\}$ sur laquelle $\sigma$ induit une permutation circulaire.
(Les orbites sont les supports des cycles de $\sigma$, et les singletons
constitu{\'e}s de points fixes)

On note $N(\sigma)$ le nombre d'orbites de $\sigma$.
}
\begin{enumerate}
    \item \question{Montrer que si $\tau$ est une transposition, alors $N(\tau\circ\sigma) = N(\sigma) \pm 1$.}
    \item \question{Application : Quel est le nombre minimal de transpositions n{\'e}c{\'e}ssaires
    pour obtenir un $n$-cycle ?}
\end{enumerate}
}

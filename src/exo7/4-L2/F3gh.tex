\uuid{F3gh}
\exo7id{1186}
\auteur{liousse}
\organisation{exo7}
\datecreate{2003-10-01}
\isIndication{false}
\isCorrection{false}
\chapitre{Déterminant, système linéaire}
\sousChapitre{Système linéaire, rang}

\contenu{
\texte{
L'objectif de ce probl\`eme est de r\'esoudre l'\'enigme du berger :\\
Un berger poss\`ede un troupeau de 101 moutons et remarque par hasard la 
propri\'et\'e suivante : pour chaque mouton, il peut trouver une fa\c{c}on 
de scinder 
le troupeau des 100
autres moutons en deux troupeaux de 50 moutons et de m\^eme poids total. Il en
d\'eduit que tous les moutons ont le m\^eme poids. Comment a-t-il fait ? 
On montre, 
dans un premier temps, un r\'esultat utile pour la d\'emonstration finale.\\
}
\begin{enumerate}
    \item \question{\begin{enumerate}}
    \item \question{Montrer par r\'ecurrence que le d\'eterminant de 
toute matrice carr\'ee, dont les \'el\'ements diagonaux sont des nombres impairs, 
et dont tous les autres sont des nombres pairs, est un nombre impair.\\}
    \item \question{En d\'eduire qu'une matrice de cette forme est inversible.\\}
\end{enumerate}
}

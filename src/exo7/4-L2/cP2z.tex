\uuid{cP2z}
\exo7id{2470}
\titre{exo7 2470}
\auteur{matexo1}
\organisation{exo7}
\datecreate{2002-02-01}
\isIndication{false}
\isCorrection{false}
\chapitre{Réduction d'endomorphisme, polynôme annulateur}
\sousChapitre{Valeur propre, vecteur propre}
\module{Algèbre}
\niveau{L2}
\difficulte{}

\contenu{
\texte{
Soit $M$ une matrice de $\mathcal M_n(\C)$\,; on suppose qu'il existe un
entier $p$ tel que $M^p = I$. Montrer que si $\omega$ est une
racine $p$-i\`eme de l'unit\'e, c'est une valeur propre de $M$ ou
alors $M$ v\'erifie
$$ M^{p-1} + \omega M^{p-2} +\cdots+\omega^{p-2}M +\omega^{p-1}I=0.$$
}
}

\uuid{QLUZ}
\exo7id{5373}
\titre{exo7 5373}
\auteur{rouget}
\organisation{exo7}
\datecreate{2010-07-06}
\isIndication{false}
\isCorrection{true}
\chapitre{Déterminant, système linéaire}
\sousChapitre{Calcul de déterminants}
\module{Algèbre}
\niveau{L2}
\difficulte{}

\contenu{
\texte{
Soient $a_{i,j}$ ($(i,j)$ élément de $\{1,...,n\}^2$) $n^2$ fonctions de $\Rr$ dans $\Rr$, dérivables sur $\Rr$ et $A=(ai,j)_{1\leq i,j\leq n}$. Calculer la dérivée de la fonction $x\mapsto\mbox{det}(A(x))$.

Applications. Calculer
}
\begin{enumerate}
    \item \question{$\left|\begin{array}{ccccc}
x+1&1&\ldots& &1\\
1&x+1&\ddots& &\vdots\\
\vdots&\ddots&\ddots&\ddots&\vdots\\
\vdots& &\ddots&\ddots&1\\
1&\ldots&\ldots&1&x+1
\end{array}
\right|$}
\reponse{Soit $\Delta_n(x)=\left|\begin{array}{ccccc}
x+1&1&\ldots& &1\\
1&x+1&\ddots& &\vdots\\
\vdots&\ddots&\ddots&\ddots&\vdots\\
\vdots& &\ddots&\ddots&1\\
1&\ldots&\ldots&1&x+1
\end{array}
\right|$. $\Delta_n$ est un polynôme dont la dérivée est d'après ce qui précède, $\Delta_{n}'=\sum_{k=1}^{n}\delta_k$ où $\delta_k$ est le déterminant déduit de $\Delta_n$ en remplaçant sa $k$-ème colonne par le $k$-ème vecteur de la base canonique de $M_{n,1}(\Kk)$. En développant $\delta_k$ par rapport à sa $k$-ème colonne, on obtient $\delta_k=\Delta_{n-1}$ et donc $\Delta_{n}'=n\Delta_{n-1}$.
Ensuite, on a déjà $\Delta_1=X+1$ puis $\Delta_2=(X+1)^2-1=X^2+2X$ ...
Montrons par récurrence que pour $n\geq1$, $\Delta_n=X^n+nX^{n-1}$.
C'est vrai pour $n=1$ puis, si pour $n\geq1$, $\Delta_n=X^n+nX^{n-1}$ alors $\Delta_{n+1}'=(n+1)X^n+(n+1)nX^{n-1}$ et, par intégration,  $\Delta_{n+1}=X^{n+1}+(n+1)X^n+\Delta_{n+1}(0)$. Mais, puisque $n\geq1$, on a $n+1\geq2$ et $\Delta_{n+1}(0)$ est un déterminant ayant au moins deux colonnes identiques. Par suite, $\Delta_{n+1}(0)=0$ ce qui montre que $\Delta_{n+1}=X^{n+1}+(n+1)X^n$. Le résultat est démontré par récurrence.

\begin{center}
\shadowbox{
$\forall n\in\Nn^*,\;\Delta_n=x^n+nx^{n-1}$.
}
\end{center}}
    \item \question{$\left|\begin{array}{ccccc}
x+a_1&x&\ldots& &x\\
x&x+a_2&\ddots& &\vdots\\
\vdots&\ddots&\ddots&\ddots&\vdots\\
\vdots& &\ddots&\ddots&x\\
x&\ldots&\ldots&x&x+a_n
\end{array}
\right|$}
\reponse{Soit $\Delta_n(x)=\left|\begin{array}{ccccc}
x+a_1&x&\ldots& &x\\
x&x+a_2&\ddots& &\vdots\\
\vdots&\ddots&\ddots&\ddots&\vdots\\
\vdots& &\ddots&\ddots&x\\
x&\ldots&\ldots&x&x+a_n
\end{array}
\right|$. $\Delta_n=\mbox{det}(a_1e_1+xC,...,a_ne_n+xC)$ où $e_k$ est le $k$-ème vecteur de la base canonique de $M_{n,1}(\Kk)$ et $C$ est la colonne dont toutes les composantes sont égales à $1$. Par linéarité par rapport à chaque colonne, $\Delta_n$ est somme de $2^n$ déterminants mais dès que $C$ apparait deux fois, le déterminant correspondant est nul. Donc, $\Delta_n=\mbox{det}(a_1e_1,...,a_ne_n)+\sum_{}^{}\mbox{det}(a_1e_1,...,xC,...,a_ne_n)$. Ceci montre que $\Delta_n$ est un polynôme de degré inférieur ou égal à $1$.
La formule de \textsc{Taylor} fournit alors : $\Delta_n=\Delta_n(0)+X\Delta_{n}'(0)$. Immédiatement, $\Delta_n(0)=\prod_{k=1}^{n}a_k=\sigma_n$ puis $\Delta_{n}'(0)=\sum_{k=1}^{n}\mbox{det}(a_1e_1,...,C,...,a_ne_n)=\sum_{k=1}^{n}\prod_{i\neq k}^{}a_i=\sigma_{n-1}$. Donc, $\Delta_n=\sigma_n+X\sigma_{n-1}$.}
\end{enumerate}
}

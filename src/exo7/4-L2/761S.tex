\uuid{761S}
\exo7id{1554}
\titre{exo7 1554}
\auteur{barraud}
\organisation{exo7}
\datecreate{2003-09-01}
\isIndication{false}
\isCorrection{true}
\chapitre{Endomorphisme particulier}
\sousChapitre{Endomorphisme orthogonal}
\module{Algèbre}
\niveau{L2}
\difficulte{}

\contenu{
\texte{
Dans un espace euclidien $E$, on considère un vecteur unitaire $a$, et
  à un réel $k\neq -1$ on associe l'endomorphisme $u_{k}$ de $E$ défini
  par~:
  $$
  u_{k}(x)=k<x,a>a + x
  $$
}
\begin{enumerate}
    \item \question{Montrer que $u_{k}$ est un isomorphisme. Déterminer $u_{k}^{-1}$.
    \emph{(on pourra commencer par calculer $<u_{k}(x),a>$)}}
\reponse{$<u_{k}(x),a>=k<x,a><a,a>+<x,a>=(k+1)<x,a>$ donc
    $x=\frac{-k}{k+1}<u_{k}(x),a>a+u_{k}(x)$. On en déduit que $u_{k}$
    est inversible, et que $u_{k}^{-1}=u_{\frac{-k}{k+1}}$.}
    \item \question{Rappeler la caractérisation de l'adjoint d'un endomorphisme, et 
    montrer que $u$ est auto adjoint.}
\reponse{L'adjoint d'un endomorphisme $u$ est l'unique endomorphisme $v$ qui
    satisfait~: $\forall(x,y)\in E^{2}, <u(x),y>=<x,u(y)>$. Or
    $<u_{k}(x),y>=k<x,a><y,a>+<x,y>=<x,u_{k}(y)>$. Donc $u_{k}$ est égal
    à son adjoint.}
    \item \question{Pour quelles valeurs de $k$ $u$ est-il orthogonal~? Interpréter alors
    géométriquement cette transformation.}
\reponse{Si $u_{k}$ est orthogonal, on doit avoir $\Vert u_{k}(a)\Vert=\Vert
    a\Vert=1$, soit $|k+1|=1$. Ainsi $k=0$ ou $k=-2$.

    Pour $k=0$, $u_{k}=\mathrm{id}$ est bien orthogonal. Pour $k=-2$,
    $u_{-2}^{-1}=u_{\frac{-2}{-2+1}}=u_{-2}={}^t u_{-2}$. Donc
    $u_{-2}$ est bien orthogonal. Il s'agit de la symétrie orthogonale
    par rapport à l'hyperplan $\{a\}^{\bot}$}
    \item \question{Déterminer les valeurs propres et vecteurs propres de $u_{k}$.}
\reponse{Si $k=0$, 1 est la seule valeur propre et $E_{1}=E$

    Si $k\neq 1$, $\forall x\in\{a\}^{\bot}, u_{k}(x)=x$ donc $1$ est
    valeur propre de multiplicité au moins $n-1$. De plus
    $u_{k}(a)=(k+1)a$ donc $(k+1)$ est valeur propre. Finalement, $1$
    est valeur propre de multiplicité exactement $n-1$, avec pour espace
    propre $\{a\}^{\bot}$, et $k+1$ est valeur propre simple avec espace
    propre $\R a$.}
\end{enumerate}
}

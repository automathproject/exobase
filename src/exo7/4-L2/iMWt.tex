\uuid{iMWt}
\exo7id{1307}
\auteur{cousquer}
\organisation{exo7}
\datecreate{2003-10-01}
\isIndication{false}
\isCorrection{false}
\chapitre{Groupe, anneau, corps}
\sousChapitre{Groupe, sous-groupe}

\contenu{
\texte{

}
\begin{enumerate}
    \item \question{Une permutation de l'ensemble de $n$ éléments $\{1,2,\ldots,n\}$ est
une bijection de cet ensemble dans lui-même. Il est commode de désigner une
telle permutation $s$ par le tableau de valeurs suivant~:
$s=
\left(\begin{array}{cccc}
  1   &   2&\cdots&n\\
  s(1)&s(2)&\cdots&s(n)
\end{array}\right)$. On note
$\frak{S}_n$ l'ensemble de ces permutations pour $n$ donné.}
    \item \question{Écrire les éléments de $\frak{S}_2$ et de $\frak{S}_3$.}
    \item \question{Établir les tables de composition de ces deux ensembles.}
    \item \question{De la table de $\frak{S}_3$ on peut extraire des parties \emph{stables}
ne faisant intervenir que certains éléments~; lesquelles~? Peut-on les
trouver toutes.}
    \item \question{Voyez-vous des analogies (totales ou partielles) entre ces tables et des
situations rencontrées plus haut~?}
    \item \question{On peut obtenir tous les éléments de $\frak{S}_3$ à partir de la
composition de certains d'entre-eux~; lesquels~?}
    \item \question{Combien d'éléments possède $\frak{S}_n$~? Combien de cases
contient la table de composition de $\frak{S}_4$, $\frak{S}_5,\ldots$~?
Pourrait-on étudier $\frak{S}_4$ et $\frak{S}_5$ à partir de ces tables~?}
\end{enumerate}
}

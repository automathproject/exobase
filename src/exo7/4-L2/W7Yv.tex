\uuid{W7Yv}
\exo7id{1345}
\auteur{legall}
\organisation{exo7}
\datecreate{1998-09-01}
\isIndication{false}
\isCorrection{true}
\chapitre{Groupe, anneau, corps}
\sousChapitre{Morphisme, isomorphisme}

\contenu{
\texte{
Soit  $f:{\Rr}\rightarrow {\Cc}^*$  l'application
qui \`a tout $x\in {\Rr}$  associe  $e^{ix}\in {\Cc}^*$. Montrer
que $f$ est un homomorphisme de groupes. Calculer son noyau et son
image. $f$ est-elle injective ?
}
\reponse{
\begin{align*}
f : (\Rr,+)& \longrightarrow (\Cc^*,\times) \\
x& \mapsto e^{ix}\\
\end{align*}

V\'erifions que $f$ est un morphisme de groupe. Soit $x,y \in
\Rr$, alors
$$f(x+y) = e^{i(x+y)}=e^{ix}e^{iy}=f(x)\times f(y),$$
et
$$f(x^{-1})= e^{i(-x)} = \frac{1}{e^{ix}}={f(x)}^{-1}.$$
Donc $f$ est un morphisme de groupe.

Montrons que $f$ n'est pas injective en prouvant que le noyau
n'est pas r\'eduit \`a $0$ :
$$ \text{Ker}\, f = \left\lbrace x\in\Rr \text{ tels que } f(x)=1 \right\rbrace
   = \left\lbrace x\in\Rr \text{ tels que } e^{ix}=1 \right\rbrace
= \left\lbrace x = 0+2k\pi,\ \ k\in\Zz \right\rbrace. $$

Enfin
$$\text{Im}\, f =  \left\lbrace y\in\Cc^*, y = e^{ix} \right\rbrace$$
est l'ensemble des complexes de module $1$, c'est-\`a-dire le
cercle de centre $0$ et de rayon $1$.
}
}

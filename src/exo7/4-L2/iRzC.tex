\uuid{iRzC}
\exo7id{2465}
\titre{exo7 2465}
\auteur{matexo1}
\organisation{exo7}
\datecreate{2002-02-01}
\isIndication{false}
\isCorrection{false}
\chapitre{Déterminant, système linéaire}
\sousChapitre{Système linéaire, rang}
\module{Algèbre}
\niveau{L2}
\difficulte{}

\contenu{
\texte{
Soit $A$ une matrice carr\'ee d'ordre $n$ tridiagonale,
c'est-\`a-dire telle que $a_{i,j} = 0$ si $|i-j| > 1$. Montrer
qu'il existe une matrice triangulaire inf\'erieure $L$
v\'erifiant  $l_{i,j} = 0$ si $j > i+1$ et une triangulaire
sup\'erieure $U$ v\'erifiant $u_{i,i} = 1$ et $u_{i,j} = 0$ si $i
> j+1$ telles que $A = LU$, et que ces matrices sont uniques. En
d\'eduire la solution du syst\`eme lin\'eaire $Ax = b$, o\`u $b$
est un vecteur donn\'e dans $\R^n$.
}
}

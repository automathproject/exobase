\uuid{gSmm}
\exo7id{1541}
\auteur{legall}
\organisation{exo7}
\datecreate{1998-09-01}
\isIndication{false}
\isCorrection{false}
\chapitre{Endomorphisme particulier}
\sousChapitre{Autres endomorphismes normaux}

\contenu{
\texte{
Soit $  (E , \langle   ,   \rangle )  $ un espace euclidien. Un
endomorphisme $  \varphi \in \mathcal{L} (E)  $ est dit antisym\'etrique lorsque
 $  \varphi ^*=-\varphi   .$
}
\begin{enumerate}
    \item \question{Montrer que $  \varphi  $ est antisym\'etrique si et seulement si $  \forall x \in E  ,  
\langle \varphi (x),x \rangle =0  .$ (on pourra remarquer que $  \varphi +\varphi ^*   $ est
autoadjoint.)}
    \item \question{Montrer que si $  \varphi  $ est antisym\'etrique alors
 $  (\hbox{Ker}(\varphi))^\perp =\hbox{Im}
(\varphi )  $ puis que $  \hbox{rg}(\varphi )  $ est pair.}
\end{enumerate}
}

\uuid{E8QP}
\exo7id{7346}
\auteur{mourougane}
\organisation{exo7}
\datecreate{2021-08-10}
\isIndication{false}
\isCorrection{false}
\chapitre{Groupe, anneau, corps}
\sousChapitre{Anneau}

\contenu{
\texte{

}
\begin{enumerate}
    \item \question{Montrer que le polynôme $X^9-1$ de $\mathbb{F}_3[X]$ vaut $(X-1)^9$. On considère le code ternaire $C$ de longueur $9$ associé au polynôme $g=(X-1)^5$}
    \item \question{Déterminer l'alphabet, la longueur des mots, la dimension du code, le nombre de mots de code. Le code est-il cyclique ?}
    \item \question{Donner une matrice génératrice de $C$.}
    \item \question{Déterminer un élément de poids $3$ du code.}
    \item \question{Déterminer une matrice de contrôle $H$ de ce code.}
    \item \question{Déterminer la distance de ce code. Combien d'erreurs ce code peut-il détecter ? combien d'erreurs peut-il corriger ?}
    \item \question{On a reçu le mot $r=121102210$. Calculer son image par $H$. Le mot $r$ est-il un mot du code ?}
    \item \question{Corriger le mot $r$ en supposant qu'il n'y a eu au plus qu'une seule erreur de transmission.}
\end{enumerate}
}

\uuid{Knh5}
\exo7id{3703}
\auteur{quercia}
\organisation{exo7}
\datecreate{2010-03-11}
\isIndication{false}
\isCorrection{true}
\chapitre{Espace euclidien, espace normé}
\sousChapitre{Espace vectoriel euclidien de dimension 3}

\contenu{
\texte{
Soient $f,g \in {\cal O}(\R^3)$ ayant même polynôme caractéristique.

Montrer qu'il existe $h \in {\cal O}(\R^3)$ tel que $f = h^{-1}\circ g \circ h$.

Si $f$ et $g$ sont positifs, a-t-on $h$ positif ?
}
\reponse{
Pour $f,g \in {\cal O}^+(\R^3)$, $f$ et $g$ ont la même matrice
         réduite dans une base orthonormée convenable, donc sont conjugués
         dans ${\cal O}(\R^3)$. $h$ n'est pas toujours positif car les bases
         peuvent ne pas avoir même orientation (ex~: deux rotations inverses).
         Pour $f,g \in {\cal O}^-(\R^3)$, considérer $-f$, $-g$.
}
}

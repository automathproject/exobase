\uuid{M4z1}
\exo7id{1412}
\auteur{legall}
\organisation{exo7}
\datecreate{1998-09-01}
\isIndication{false}
\isCorrection{true}
\chapitre{Groupe, anneau, corps}
\sousChapitre{Groupe de permutation}

\contenu{
\texte{

}
\begin{enumerate}
    \item \question{Rappeler $  \vert S_3\vert   .$ Montrer que $  S_3  $ ne contient pas d'\'el\'ement d'ordre $  6  .$}
\reponse{$|S_n| = n!$ donc $|S_3| = 3! =6$. Montrons plus g\'en\'eralement
qu'il n'existe pas d'\'el\'ement d'ordre $n!$ dans $S_n$ ($n\ge
3$). Par l'absurde soit $\alpha$ un tel \'el\'ement. Alors par
hypoth\`ese $S_n$ est engendr\'e par $\alpha$ et donc $S_n$ est un
groupe commutatif. Mais $(1,2)(2,3) \not= (2,3)(1,2)$ ce qui est
absurde. En conclusion il n'existe pas d'\'el\'ements d'ordre $6$.}
    \item \question{Montrer que $  S_3  $ contient un unique sous-groupe d'ordre $  3  .$ D\'eterminer tous
les sous-groupes d'ordre $  2  $ de $  S_3  .$}
\reponse{Explicitons $S_3$ :
$$S_3 = \left\lbrace  id ; \tau_1=(1,2) ; \tau_2=(2,3) ;
\tau_3=(1,3) ; \sigma_1 = (1,2,3) ; \sigma_2 = \sigma_1^{-1}=
(3,2,1) \right\rbrace.$$ Remarquons

Les sous-groupes d'ordre $2$ sont de la forme $\{ id ; \tau \}$
avec $\tau^2 = id$. Les seuls \'el\'ements d'ordre $2$ sont les
transpositions et donc se sont les groupes $\{ id ; (1,2) \}, $\{
id ; (1,3) \}, $\{ id ; (2,3) \}$.

Les sous-groupes d'ordre trois sont de la forme $\{
id,\sigma,\sigma^2 \}$ avec $\sigma^2=\sigma^{-1}$. Et donc le
seul sous-groupe d'ordre $3$ est $\{ id ; (1,2,3) ; (3,2,1) \}$.}
    \item \question{D\'eduire de ce qui pr\'ec\`ede tous
les sous-groupes de $  S_3  .$}
\reponse{Les sous-groupes de $S_3$ ont un ordre qui divise
$|S_3| = 6$. Donc un sous-groupe peut-\^etre  d'ordre $1,2,3$ ou
$6$. L'unique sous-groupe d'ordre $1$ est $\{ id \}$, et l'unique
sous-groupe d'ordre $6$ est $S_3$. Les sous-groupes d'ordre $2$ et
$3$ ont \'et\'es donn\'es \`a la question pr\'ec\'edente.}
\end{enumerate}
}

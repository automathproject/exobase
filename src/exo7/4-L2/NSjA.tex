\uuid{NSjA}
\exo7id{5488}
\titre{exo7 5488}
\auteur{rouget}
\organisation{exo7}
\datecreate{2010-07-10}
\isIndication{false}
\isCorrection{true}
\chapitre{Espace euclidien, espace normé}
\sousChapitre{Projection, symétrie}
\module{Algèbre}
\niveau{L2}
\difficulte{}

\contenu{
\texte{
Matrice de la projection orthogonale sur la droite d'équations $3x=6y=2z$ dans la base canonique orthonormée de $\Rr^3$ ainsi que de la symétrie orthogonale par rapport à cette même droite.
De manière générale, matrice de la projection orthogonale sur le vecteur unitaire $u=(a,b,c)$ et de la projection orthogonale sur le plan d'équation $ax+by+cz=0$ dans la base canonique orthonormée de $\Rr^3$.
}
\reponse{
Un vecteur engendrant $D$ est $\overrightarrow{u}=(2,1,3)$. Pour $(x,y,z)\in\Rr^3$,

$$p((x,y,z))=\frac{(x,y,z)|(2,1,3)}{||(2,1,3)||^2}(2,1,3)=\frac{2x+y+3z}{14}(2,1,3).$$ 
On en déduit que $\mbox{Mat}_{\mathcal{B}}p=P=\frac{1}{14}\left(
\begin{array}{ccc}
4&2&6\\
2&1&3\\
6&3&9
\end{array}
\right)$, puis $\mbox{Mat}_{\mathcal{B}}s=2P-I=\frac{1}{7}\left(
\begin{array}{ccc}
-3&2&6\\
2&-6&3\\
6&3&2
\end{array}
\right)$.
Plus généralement, la matrice de la projection orthogonale sur le vecteur unitaire $(a,b,c)$ dans la base canonique orthonormée est $P=\left(
\begin{array}{ccc}
a^2&ab&ac\\
ab&b^2&bc\\
ac&bc&c^2
\end{array}
\right)$ et la matrice de la projection orthogonale sur le plan $ax+by+cz=0$ dans la base canonique orthonormée est $I-P=
\left(
\begin{array}{ccc}
1-a^2&-ab&-ac\\
-ab&1-b^2&-bc\\
-ac&-bc&1-c^2
\end{array}
\right)$.
}
}

\uuid{LS8p}
\exo7id{5789}
\titre{exo7 5789}
\auteur{rouget}
\organisation{exo7}
\datecreate{2010-10-16}
\isIndication{false}
\isCorrection{true}
\chapitre{Espace euclidien, espace normé}
\sousChapitre{Orthonormalisation}
\module{Algèbre}
\niveau{L2}
\difficulte{}

\contenu{
\texte{
\label{exo:sperou4}
Soit $E$ un espace euclidien de dimension $n\geqslant1$ et $\mathcal{B}$ une base orthonormée de $E$.

Montrer que pour tout $n$-uplet de vecteurs $(x_1,...x_n)$, on a : $\left|\text{det}_{\mathcal{B}}(x_1,...,x_n)\right|\leqslant\|x_1\|...\|x_n\|$. Cas d'égalité ?
}
\reponse{
Si la famille $(x_1,...,x_n)$ est liée, l'inégalité est vraie.

Si la famille $(x_1,...,x_n)$ est libre, on peut considérer $B_0 =(e_1,...,e_n)$ l'orthonormalisée de \textsc{Schmidt} de la famille $(x_1,...,x_n)$. Les bases $B_0$ et $B$ sont des bases orthonormées de $E$ et donc

\begin{align*}\ensuremath
\left|\text{det}_B(x_1,...,x_n)\right|&=\left|\text{det}_{B_0}(x_1,...,x_n)\right|
=\text{abs}\left(\left|\begin{array}{cccc}(x_1|e_1)&\times&\ldots&\times\\
 0&\ddots&\ddots&\vdots\\
 \vdots&\ddots&\ddots&\times\\
 0&\ldots&0&(x_n|e_n)\end{array}\right|\right)\\
 &=\prod_{k=1}^{n}|(x_k|e_k)|\leqslant\prod_{k=1}^{n}\|x_k\|\|e_k\|\;(\text{d'après l'inégalité de \textsc{Cauchy}-\textsc{Schwarz}})\\
 &=\prod_{k=1}^{n}\|x_k\|.
\end{align*}

\begin{center}
\shadowbox{
$\forall(x_1,\ldots,x_n)\in E^n$, $\left|\text{det}_B(x_1,...,x_n)\right|\leqslant\prod_{k=1}^{n}\|x_k\|$ (inégalité de \textsc{Hadamard}).
}
\end{center}

Ensuite, 

- si la famille $(x_1,...,x_n)$ est liée, on a l'égalité si et seulement si l'un des vecteurs $x_k$ est nul 

- si la famille $(x_1,...,x_n)$ est libre, on a l'égalité si et seulement si $\forall k\in\llbracket1,n\rrbracket$, $|(x_k|e_k)|=\|x_k\|\|e_k\|$. Les cas d'égalité

de l'inégalité de \textsc{Cauchy}-\textsc{Schwarz} étant connus, on a l'égalité si et seulement si $\forall k\in\llbracket1,n\rrbracket$, $x_k$ est colinéaire à $e_k$ 

ou encore si et seulement si la famille $(x_1,...,x_n)$ est orthogonale.

En résumé, l'inégalité de \textsc{Hadamard} est une égalité si et seulement si la famille $(x_1,...,x_n)$ est orthogonale libre ou si l'un des vecteurs est nul.
}
}

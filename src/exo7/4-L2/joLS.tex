\uuid{joLS}
\exo7id{3162}
\auteur{quercia}
\organisation{exo7}
\datecreate{2010-03-08}
\isIndication{false}
\isCorrection{true}
\chapitre{Arithmétique}
\sousChapitre{Anneau Z/nZ, théorème chinois}

\contenu{
\texte{
Soit $p$ un nombre premier impair.
}
\begin{enumerate}
    \item \question{Montrer qu'une {\'e}quation du second degr{\'e}~: $x^2 + ax + b = \dot 0$ admet
    une solution dans $\Z/p\Z$ si et seulement si son discriminant~: $a^2 - 4b$
    est un carr{\'e} dans $\Z/p\Z$.}
    \item \question{On suppose que $p\equiv 1 (\mathrm{mod}\, 3)$~: $p=3q+1$.
  \begin{enumerate}}
    \item \question{Montrer qu'il existe $a\in(\Z/p\Z)^*$ tel que $a^q\ne \dot 1$.}
    \item \question{En d{\'e}duire que $-\dot 3$ est un carr{\'e}.}
\reponse{
\begin{enumerate}
Le nombre de solutions de l'{\'e}quation $x^q = \dot 1$ est inf{\'e}rieur
    ou {\'e}gal {\`a}~$q < p-1$.
$\dot 0 = a^{3q}-\dot 1 = (a^q - \dot 1)(a^{2q} + a^q + \dot 1)$
    donc $a^{2q}$ est racine de $x^2 + x + \dot 1 = \dot 0$, de discriminant
    $-\dot 3$.
}
\end{enumerate}
}

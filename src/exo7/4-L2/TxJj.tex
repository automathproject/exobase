\uuid{TxJj}
\exo7id{3088}
\auteur{quercia}
\organisation{exo7}
\datecreate{2010-03-08}
\isIndication{false}
\isCorrection{false}
\chapitre{Groupe, anneau, corps}
\sousChapitre{Groupe de permutation}

\contenu{
\texte{
Soit $H$ un sous-groupe de ${\cal S}_n$ d'ordre $\frac {n!}2$.
On note $K = {\cal S}_n \setminus H$.
}
\begin{enumerate}
    \item \question{Pour $\sigma \in H$, montrer que $\sigma H = H$ et $\sigma K = K$.}
    \item \question{Soit $\sigma \in {\cal S}_n$. D{\'e}terminer les ensembles $\sigma H$,
    $\sigma K$, $H\sigma$, $K\sigma$ suivant que $\sigma \in H$ ou
    $\sigma \in K$.}
    \item \question{En d{\'e}duire que si deux permutations sont conjugu{\'e}es, alors elles sont toutes deux
    dans $H$ ou toutes deux dans $K$.}
    \item \question{Montrer enfin que $H = {\cal A}_n$.}
\end{enumerate}
}

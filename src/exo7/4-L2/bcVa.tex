\uuid{bcVa}
\exo7id{2589}
\auteur{delaunay}
\organisation{exo7}
\datecreate{2009-05-19}
\isIndication{false}
\isCorrection{true}
\chapitre{Réduction d'endomorphisme, polynôme annulateur}
\sousChapitre{Réduction de Jordan}

\contenu{
\texte{
On consid\`ere la matrice suivante
$$A=\begin{pmatrix}1&-1&0 \\  1&0&-1 \\  -1&0&2\end{pmatrix}$$
et $f$ l'endomorphisme de $\R^3$ associ\'e.
}
\begin{enumerate}
    \item \question{Factoriser le polyn\^ome caract\'eristique de $A$.}
\reponse{{\it Factorisons le polyn\^ome caract\'eristique de $A$.}

On a 
\begin{align*}P_A(X)&=\begin{vmatrix}1-X&-1&0 \\  1&-X&-1 \\  -1&0&2-X\end{vmatrix} \\ 
&=(1-X)\begin{vmatrix}-X&-1 \\  0&2-X\end{vmatrix}+\begin{vmatrix}1&-1 \\  -1&2-X\end{vmatrix} \\ 
&=(1-X)(X^2-2X)+(1-X) \\ 
&=(1-X)(X^2-2X+1)=(1-X)^3.
\end{align*}
Le polyn\^ome caract\'eristique de $A$ admet une valeur propre triple $\lambda=1$.}
    \item \question{D\'eterminer les sous-espaces propres et caract\'eristiques de $A$.}
\reponse{{\it D\'eterminons les sous-espaces propres et caract\'eristiques de $A$.}

La matrice $A$ admet une unique valeur propre $\lambda=1$ de multiplicit\'e $3$, le sous-espace propre associ\'e est l'espace $E_{1}=\ker(A-I)$, et on a
$$(x,y,z)\in E_{1}\iff\left\{\begin{align*}x-y&=x \\  x-z&=y \\  -x+2z&=z\end{align*}\right.
\iff\left\{\begin{align*}y&=0 \\  x&=z\end{align*}\right.$$
Le sous-espace $E_{1}$ est la droite vectorielle engendr\'ee par le vecteur $(1,0,1)$.

Le sous-espace caract\'eristique de $A$, associ\'e \`a l'unique valeur propre $\lambda=1$, est le sous-espace $N_{1}=\ker(A-I)^3$, or, compte tenu du th\'eor\`eme de Hamilton-Cayley, on sait que $P_A(A)=0$, ainsi, la matrice $(A-I)^3$ est la matrice nulle, ce qui implique $N_{1}=\R^3$, c'est donc l'espace tout entier.}
    \item \question{D\'emontrer qu'il existe une base de $\R^3$ dans laquelle la matrice de $f$ s'\'ecrit
$$B=\begin{pmatrix}1&1&0 \\  0&1&1 \\  0&0&1\end{pmatrix}.$$}
\reponse{{\it D\'emontrons qu'il existe une base de $\R^3$ dans laquelle la matrice de $f$ s'\'ecrit}
$$B=\begin{pmatrix}1&1&0 \\  0&1&1 \\  0&0&1\end{pmatrix}.$$

Nous cherchons des vecteurs $e_1,e_2,e_3$ tels que $Ae_1=e_1$, $Ae_2=e_1+e_2$ et $Ae_3=e_2+e_3$.
Le vecteur $e_1$ appartient \`a $E_1=\ker(A-I)$, et $\ker(A-I)$ est la droite d'\'equations :

$\{y=0,x=z\}$. On d\'etermine $e_2=(x,y,z)$ tel que
$Ae_2=e_1+e_2$, on obtient le syst\`eme 
$$\begin{pmatrix}1 & -1 & 0 \\  1&0&-1 \\ -1&0&2\end{pmatrix}\begin{pmatrix}x \\  y \\  z\end{pmatrix}=\begin{pmatrix}1+x \\  1+y \\  z\end{pmatrix}\iff
\left\{\begin{align*}x-y&=1+x \\  x-z&=y \\  -x+2z&=1+z\end{align*}\right.
\iff\left\{\begin{align*}y&=-1 \\  x-z&=-1\end{align*}\right.$$
Ainsi, les vecteurs $e_1=(1,0,1)$ et $e_2=(-1,-1,0)$ conviennent. Il nous reste \`a chercher un vecteur $e_3$ tel que $Ae_3=e_2+e_3$, c'est-\`a-dire
$$\begin{pmatrix}1 & -1 & 0 \\  1&0&-1 \\ -1&0&2\end{pmatrix}\begin{pmatrix}x \\  y \\  z\end{pmatrix}=\begin{pmatrix}-1+x \\  -1+y \\  z\end{pmatrix}\iff
\left\{\begin{align*}x-y&=x-1 \\  x-z&=y-1 \\  -x+2z&=z\end{align*}\right.
\iff\left\{\begin{align*}y&=1 \\  x&=z\end{align*}\right.$$
Le vecteur $e_3=(0,1,0)$ convient. On obtient alors la matrice $P$ suivante qui est inversible et v\'erifie $A=PBP^{-1}$,
$$P=\begin{pmatrix}1 & -1 & 0 \\  0&-1&1 \\ 1&0&0\end{pmatrix}$$}
    \item \question{Ecrire la d\'ecomposition de Dunford de $B$ (justifier).}
\reponse{{\it D\'ecomposition de Dunford de $B$}

On a 
$$B=\begin{pmatrix}1 & 1 & 0 \\  0&1&1 \\ 0&0&1\end{pmatrix}=
\begin{pmatrix}1 & 0& 0 \\  0&1&0 \\ 0&0&1\end{pmatrix}+
\begin{pmatrix}0& 1 & 0 \\  0&0&1 \\ 0&0&0\end{pmatrix}$$
et il est clair que les deux matrices commutent car l'une est \'egale \`a $I$. 
Or, il existe un unique couple de matrices $D$ et $N$, $D$ diagonalisable
et $N$ nilpotente, telles que $B=D+N$ et $DN=ND$. Or si
$$D=\begin{pmatrix}1 & 0& 0 \\  0&1&0 \\ 0&0&1\end{pmatrix}\ \ {\hbox{et}}\ \ 
N=\begin{pmatrix}0& 1 & 0 \\  0&0&1 \\ 0&0&0\end{pmatrix},$$
On a 
$$N^2=\begin{pmatrix}0& 1 & 0 \\  0&0&1 \\ 0&0&0\end{pmatrix}\begin{pmatrix}0& 1 & 0 \\  0&0&1 \\ 0&0&0\end{pmatrix}=\begin{pmatrix}0& 0 & 1 \\  0&0&0 \\ 0&0&0\end{pmatrix}$$
et $N^3=0$. La d\'ecomposition $B=D+N$ est donc bien la d\'ecomposition de Dunford de la matrice $B$.}
\end{enumerate}
}

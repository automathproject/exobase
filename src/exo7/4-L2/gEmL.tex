\uuid{gEmL}
\exo7id{1506}
\titre{exo7 1506}
\auteur{barraud}
\organisation{exo7}
\datecreate{2003-09-01}
\isIndication{false}
\isCorrection{false}
\chapitre{Espace euclidien, espace normé}
\sousChapitre{Orthonormalisation}
\module{Algèbre}
\niveau{L2}
\difficulte{}

\contenu{
\texte{
On considère un espace euclidien $(E,<>)$.
}
\begin{enumerate}
    \item \question{\textbf{Théorème de Pythagore :}

Soient $u$ et $v$ deux vecteurs orthogonaux de $E$. Calculer $||u+v||^{2}$. Illustrer le
résultat obtenu à l'aide d'un dessin.}
    \item \question{\textbf{Projection orthogonale et distance à un sous-espace :}

Soit $F$ un sous-espace de $E$. On rappelle que $E=F\oplus F^{\bot}$, et donc que tout
vecteur $x$ de $E$ se décompose de manière unique en une somme $x=x_{1}+_{2}$ avec
$x_{1}\in F$ et $x_{2}\in F^{\bot}$. Le vecteur $x_{1}$ s'appelle alors la projection
orthogonale de $x$ sur $F$.
\begin{enumerate}}
    \item \question{Montrer que l'application $p$ qui à un vecteur asocie sa projection orthogonale sur $E$
est une application linéaire. Vérifier que : $\forall y\in F, <x-p(x),y>=0$.}
    \item \question{\label{q: proj-realise-d}
On considère maintenant un vecteur $x$ de $E$. On appelle distance de $x$ à $F$ le nombre
$\mathrm{dist}(x,F)=\inf_{y\in F}\Vert x-y\Vert$.

Pour $y\in F$, vérifier que $x-p(x)$ et $y-p(x)$ sont orthogonaux. Utiliser alors la
question 1 pour montrer que $||x-y||^{2}\geq ||x-p(x)||^{2}$. Illustrer sur un dessin.

En déduire que $\mathrm{dist}(x,F)=||x-p(x)||$.}
    \item \question{\label{q: proj-en-coord}
Soit $(e_{1},\ldots,e_{r})$ une base orthonormée de $F$. Montrer que
$p(x)=\sum_{i=1}^{r}<x,e_{i}>\;e_{i}$.}
\end{enumerate}
}

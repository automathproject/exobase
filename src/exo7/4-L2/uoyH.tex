\uuid{uoyH}
\exo7id{7370}
\auteur{mourougane}
\organisation{exo7}
\datecreate{2021-08-10}
\isIndication{false}
\isCorrection{true}
\chapitre{Groupe, anneau, corps}
\sousChapitre{Anneau}

\contenu{
\texte{

}
\begin{enumerate}
    \item \question{L'entier $-1601$ est-il un représentant de la classe $[-7387]_{ 2893}$ de $\Z/2893\Z$?}
\reponse{$(-7387)-(-1601)=-5786=(-2)2893$ donc, l'entier $-1601$ est un représentant de la classe $[-7387]_{ 2893}$ de $\Z/2893\Z$.}
    \item \question{Calculer l'élément $11 ^{329}$ dans $\Z/13\Z$. Le résultat doit être représenté par un nombre compris entre $0$ et $12$.}
\reponse{Par le petit théorème de Fermat, $11^{12}=1[13]$.
 On effectue la division euclidienne de $329$ par $12$.
 $329-27\times 12=5$.
 Par conséquent, $11 ^{329}=11^5[13]$.
 $11^2=121=4[13]$, $11^4=4^2=3[13]$, d'où 
 $11 ^{329}=11^5=33=7[13]$}
    \item \question{La classe $[51]$ est-elle inversible dans l'anneau $\Z/131\Z$. Si oui, calculer son inverse dans $\Z/131\Z$. Le résultat doit être représenté par un nombre compris entre $0$ et $130$.}
\reponse{Puisque $\text{pgcd}(51,131)=1$, la classe $[51]$ est inversible dans $\Z/131\Z$. En utilisant l'algorithme d'Euclide on trouve que $1=131\times (-7)+51\times 18$. Alors, dans $\Z/131\Z$ on a $51^{-1}=18$.}
\end{enumerate}
}

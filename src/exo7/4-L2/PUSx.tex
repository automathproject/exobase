\uuid{PUSx}
\exo7id{5614}
\titre{exo7 5614}
\auteur{rouget}
\organisation{exo7}
\datecreate{2010-10-16}
\isIndication{false}
\isCorrection{true}
\chapitre{Déterminant, système linéaire}
\sousChapitre{Autre}
\module{Algèbre}
\niveau{L2}
\difficulte{}

\contenu{
\texte{
\label{ex:rou17}
Soit $A$ une matrice carrée de format $n$. Calculer le déterminant de sa comatrice.
}
\reponse{
On a toujours $A{^t}(\text{com}A)=(\text{det}A)I_n$. Par passage au déterminant et puisqu'une matrice a même déterminant que sa transposée, on obtient

\begin{center}
$(\text{det}A)(\text{det}(\text{com}A)) = (\text{det}A)^n$.
\end{center}

\textbullet~Si $\text{det}A$ n'est pas nul, on en déduit $\text{det}(\text{com}A)=(\text{det}A)^{n-1}$.

\textbullet~Si $\text{det}A$ est nul, on a $A{^t}(\text{com}A) = 0$ et donc ${^t}\text{com}A$ est soit nulle, soit diviseur de zéro, et donc dans tous les cas non inversible. Il en est de même de $\text{com}A$ et donc $\text{det}(\text{com}A) = 0=(\text{det}A)^{n-1}$. Finalement 

\begin{center}
\shadowbox{
$\forall A\in\mathcal{M}_(\Rr),\;\text{det}(\text{com}A)=(\text{det}A)^{n-1}$.
}
\end{center}
}
}

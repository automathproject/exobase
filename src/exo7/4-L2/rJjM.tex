\uuid{rJjM}
\exo7id{3509}
\titre{exo7 3509}
\auteur{quercia}
\organisation{exo7}
\datecreate{2010-03-10}
\isIndication{false}
\isCorrection{false}
\chapitre{Réduction d'endomorphisme, polynôme annulateur}
\sousChapitre{Valeur propre, vecteur propre}
\module{Algèbre}
\niveau{L2}
\difficulte{}

\contenu{
\texte{
Soit $M=(m_{ij}) \in \mathcal{M}_n(\R)$ telle que :
$\begin{cases} \forall\ i,j,\ m_{ij} \ge 0 \cr
         \forall\ i,\ m_{i,1} + m_{i,2} + \dots + m_{i,n} = 1.\cr \end{cases}$
({\it matrice stochastique\/})
}
\begin{enumerate}
    \item \question{Montrer que 1 est valeur propre de $M$.}
    \item \question{Soit $\lambda$ une valeur propre complexe de $M$.
    Montrer que $|\lambda| \le 1$
    (si $(x_1,\dots,x_n)\in \C^n$ est un vecteur propre associé, considérer le
    coefficient $x_k$ de plus grand module).
    Montrer que si tous les
    coefficients $m_{ij}$ sont strictement positifs alors $|\lambda| = 1  \Rightarrow  \lambda = 1$.}
\end{enumerate}
}

\uuid{4usa}
\exo7id{5520}
\titre{exo7 5520}
\auteur{rouget}
\organisation{exo7}
\datecreate{2010-07-15}
\isIndication{false}
\isCorrection{true}
\chapitre{Espace euclidien, espace normé}
\sousChapitre{Projection, symétrie}
\module{Algèbre}
\niveau{L2}
\difficulte{}

\contenu{
\texte{
Soient $(D)$ la droite dont un système d'équations cartésiennes est 
$\left\{
\begin{array}{l}
x+y+z=1\\
x-2y-z=0
\end{array}
\right.$ et $(P)$ le plan d'équation cartésienne $x+3y+2z=6$. Déterminer la projetée (orthogonale) de $(D)$ sur $(P)$.
}
\reponse{
Notons $p$ la projection orthogonale sur $(P)$.
Un repère de $(D)$ est $\left(A,\overrightarrow{u}\right)$ où $A(0,-1,2)$ et $\overrightarrow{u}(1,2,-3)$. Un vecteur normal à $(P)$ est $\overrightarrow{n}(1,3,2)$. $\overrightarrow{u}$ et $\overrightarrow{n}$ ne sont pas colinéaires et donc $p(D)$ est une droite du plan $(P)$.
Plus précisément, $p(D)$ est l'intersection du plan $(P)$ et du plan $(P')$ contenant $(D)$ et perpendiculaire à $(P)$. Un repère de $(P')$ est $\left(A,\overrightarrow{u},\overrightarrow{n}\right)$. Donc

\begin{align*}\ensuremath
M(x,y,z)\in(P')&\Leftrightarrow\left|
\begin{array}{ccc}
x&1&1\\
y+1&2&3\\
z-2&-3&2
\end{array}
\right|=0\Leftrightarrow13x-5(y+1)+(z-2)=0\Leftrightarrow13x-5y+z=7.
\end{align*}

\begin{center}
\shadowbox{
La projetée orthogonale de $(D)$ sur $(P)$ est la droite d'équations $\left\{
\begin{array}{l}
13x-5y+z=7\\
x+3y+2z=6
\end{array}
\right.$.
}
\end{center}
}
}

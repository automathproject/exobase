\uuid{UTBC}
\exo7id{2186}
\titre{exo7 2186}
\auteur{debes}
\organisation{exo7}
\datecreate{2008-02-12}
\isIndication{true}
\isCorrection{false}
\chapitre{Action de groupe}
\sousChapitre{Action de groupe}
\module{Théorie des groupes}
\niveau{L3}
\difficulte{}

\contenu{
\texte{
\label{ex:deb86}
Soit $G$ un groupe fini et $X$ un $G$-ensemble. Si $k$ est
un entier ($1\leq k$),
on dit que $X$ est $k$-transitif, si pour tout couple de $k$-uplets 
$(x_1, \dots ,x_k)$ et
$(y_1, \dots , y_k)$ d'\'el\'ements de $X$ distincts deux \`a deux,
il existe au moins un \'el\'ement $g$
de
$G$ tel que pour tout $i$, $1\leq i \leq k$, $g.x_i =y_i$. Un $G$-ensemble
$1$-transitif est donc
simplement un
$G$-ensemble transitif.
\smallskip

(a) Montrer que si $X$ est $k$-transitif, il est aussi $l$-transitif pour
tout $l$, $1\leq l \leq k$.
\smallskip

(b) Montrer que $X$ est $2$-transitif si et seulement si le fixateur d'un \'el\'ement $x$
de $X$ agit transitivement sur $X\setminus\{x\}$.

(c) Montrer que si $X$ est imprimitif, il n'est pas $2$-transitif.
\smallskip

(d) Montrer qu'un groupe cyclique $C$ d'ordre premier consid\'er\'e comme
$C$-ensemble par l'action de
translation de $C$ sur lui-m\^eme, est primitif mais n'est pas $2$-transitif.
\smallskip

(e) Montrer que l'ensemble $\{ 1, \dots ,n \} $ muni de l'action du groupe
$S_n$ est $k$-transitif
pour tout
$k$,
$1\leq k\leq n$. En d\'eduire que
l'ensemble $\{ 1, \dots ,n \} $ muni de l'action du groupe $S_n$ est
primitif.
\smallskip

(f) Montrer que le fixateur de $1$ dans $S_n$ est isomorphe \`a $S_{n-1}$.
Dans la suite on
identifie $S_{n-1}$ \`a ce fixateur. D\'eduire de l'exercice
19 que $S_{n-1}$ est un
sous-groupe propre maximal de $S_n$.
}
\indication{(a) est trivial.
\smallskip

(b): Noter d'abord que la condition sur le fixateur de $x$ est ind\'ependante de $x\in X$: en effet si $g$ est un \'el\'ement de $G$ envoyant $x$ sur un autre \'el\'ement $x^\prime\in X$ (qui existe par transitivit\'e de $G$), alors $G(x^\prime) = g G(x) g^{-1}$  et la correspondance $h\rightarrow g h g^{-1}$ permet d'identifier les actions de $G(x^\prime)$ sur $X\setminus \{x^\prime\}$ et celle de $G(x)$ sur $X\setminus \{x\}$. Supposons maintenant v\'erifi\'ee la condition sur le fixateur de $x$. Si $(x,y)$ et $(x^\prime,y^\prime)$ sont deux couples d'\'el\'ements distincts de $X$, il existe $\sigma\in G$ tel que $\sigma (x)=x^\prime$ (transitivit\'e de $G$) et il existe $\tau \in G$ tel que $\tau (x^\prime)=x^\prime$ et $\tau(\sigma(y)) = y^\prime$ (transitivit\'e de $G(x^\prime)$ sur $X\setminus \{x^\prime\}$ (noter que $\sigma
(y) \not=x^\prime$ car $\sigma (x)=x^\prime$)). La permutation $\tau \sigma$ v\'erifie
$\tau \sigma (x)=x^\prime$ et $\tau \sigma (y)=y^\prime$. Cela montre que $X$ est $2$-transitif. La r\'eciproque est triviale.

\smallskip

(c) Si l'action de $G$ sur $X$ est imprimitive et $X=\bigcup_{i=1}^r X_i$ est une partition de
$X$ comme dans la d\'efinition, alors il n'existe pas d'\'el\'ement $g\in G$ envoyant un
premier \'el\'ement $x_1\in X_1$ dans $X_1$ et un second \'el\'ement $x_1^\prime \in X_1$
dans $X_2$.
\smallskip

(d) L'action par translation d'un groupe cyclique $C$ sur lui-m\^eme est transitive, elle est
primitive si $|C|$ est premier (toute partition de $C$ en sous-ensembles de m\^eme cardinal
est forc\'ement triviale) mais elle n'est pas $2$-transitive (le fixateur de
tout \'el\'ement est trivial, ce qui contredit le (c) de l'exercice \ref{ex:deb84}).
\smallskip

(e) et (f) ne pr\'esentent aucune difficult\'e.}
}

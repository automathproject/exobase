\uuid{UY85}
\exo7id{2193}
\titre{exo7 2193}
\auteur{debes}
\organisation{exo7}
\datecreate{2008-02-12}
\isIndication{false}
\isCorrection{true}
\chapitre{Théorème de Sylow}
\sousChapitre{Théorème de Sylow}
\module{Théorie des groupes}
\niveau{L3}
\difficulte{}

\contenu{
\texte{
\label{ex:deb93}
Soit $G$ un groupe d'ordre $2p$, o\`u $p$ est un nombre premier  
sup\'erieur ou \'egal \`a $3$. Montrer que $G$ contient un unique sous-groupe $H$
d'ordre $p$ et que ce sous-groupe est distingu\'e. V\'erifier que les seuls
automorphismes d'ordre $2$ d'un groupe cyclique d'ordre $p$ sont l'identit\'e et le
passage \`a l'inverse. En d\'eduire que le groupe $G$ est soit cyclique, soit
non commutatif, auquel cas il poss\`ede deux g\'en\'erateurs $s$ et $t$
v\'erifiant les relations $s^p=1$, $t^2=1$ et $tst^{-1} =s^{-1}$.
}
\reponse{
Comme $p$ divise $|G|$, il existe dans $G$ un \'el\'ement $s$ d'ordre $p$. Le
sous-groupe $H=\hskip 2pt <s>$, d'indice $2$, est n\'ecessairement distingu\'e dans
$G$. Il est de plus le seul sous-groupe d'ordre $p$ (cf l'exercice \ref{ex:le19}).

\hskip 5mm De fa\c con g\'en\'erale, un automorphisme $\chi$ d'un groupe cyclique
$<\zeta>$ d'ordre $p$ est d\'etermin\'e par $\chi(\zeta) = \zeta^{i_\chi}$ et cet
automorphisme est d'ordre $2$ si et seulement si $i_\chi^2 \equiv 1\ [\hbox{\rm
mod}\ p]$, c'est-\`a-dire si $\chi(\zeta) = \zeta$ ou $\chi(\zeta) = \zeta^{-1}$ ce
qui correspond aux deux automorphismes ``identit\'e'' et ``passage \`a l'inverse''
(que $p$ soit premier n'intervient pas ici; le r\'esultat est valable pour tout
entier $p\geq 1$).

\hskip 5mm Soit $t\in G$ d'ordre $2$ (qui existe car $2$ divise $|G|$). La
conjugaison par $t$ induit un automorphisme du sous-groupe distingu\'e $H$.
D'ap\`es ce qui pr\'ec\`ede, on a $tst^{-1}=s$ ou bien $tst^{-1}=s^{-1}$. Dans
le premier cas, la correspondance $(s^i,t^\varepsilon) \rightarrow
s^i \cdot t^\varepsilon$ ($i=0,1,2$ et $\varepsilon = \pm 1$) induit un morphisme
entre le produit direct $<s> \times <t>$ et $G$, lequel est injectif (car $<s>
\cap <t>=\{1\}$) et donc est bijectif (puisque les groupes de d\'epart et
d'arriv\'ee ont m\^eme ordre $2p$). Dans ce cas on a donc $G\simeq \Z/p\Z
\times \Z/2\Z \simeq \Z/2p\Z$ cyclique. Dans l'autre cas, $G$ est non
commutatif (puisque $tst^{-1}=s^{-1}\not=s$); il est engendr\'e par
$s$ et $t$ qui v\'erifient les relations $s^p=1$, $t^2=1$ et $tst^{-1} =s^{-1}$.
Dans ce cas $G$ est isomorphe au groupe di\'edral $\Z/p\Z \times \hskip -6pt
{\raise 1.4pt\hbox{${\scriptscriptstyle |}$}} \Z/2\Z$ d'ordre $2p$.
}
}

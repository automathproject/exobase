\uuid{TQBb}
\exo7id{2127}
\auteur{debes}
\organisation{exo7}
\datecreate{2008-02-12}
\isIndication{false}
\isCorrection{true}
\chapitre{Ordre d'un élément}
\sousChapitre{Ordre d'un élément}

\contenu{
\texte{
Soient $G$ un groupe fini et commutatif et $\{ G_i\} _{i\in I }$ la famille des sous-groupes
propres maximaux de $G$. On pose $F=\bigcap _{i\in I} G_i$. Montrer que $F$ est
l'ensemble des \'el\'ements $a$ de $G$ qui sont tels que, pour toute partie $S$ de
$G$ contenant $a$ et engendrant $G$, $S-\{ a \}$ engendre encore $G$.
}
\reponse{
Etant donn\'e $a\in F$, soit $S$ une partie de $G$ contenant $a$
et engendrant $G$. Si $<S-\{ a \}> \not= G$, alors il existe un sous-groupe propre
maximal $G_i$ tel que $<S-\{ a \}> \subset G_i$. Mais alors $<S> \hskip 3pt \subset
\hskip 3pt <S-\{ a \}> <a> \hskip 3pt \subset \hskip 3pt G_i$. Contradiction, donc
$<S-\{ a \}> \not= G$.

Inversement, supposons que $a\notin F$, c'est-\`a-dire, il existe $i\in I$ tel que
$a\notin G_i$. Alors pour $S=G_i\cup \{a\}$, on a $<S>=G$ (par maximalit\'e de
$G_i$) mais $<S-\{ a \}> = G_i \not= G$.
}
}

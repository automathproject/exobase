\uuid{2aeU}
\exo7id{1437}
\titre{exo7 1437}
\auteur{ortiz}
\organisation{exo7}
\datecreate{1999-04-01}
\isIndication{false}
\isCorrection{false}
\chapitre{Groupe quotient, théorème de Lagrange}
\sousChapitre{Groupe quotient, théorème de Lagrange}
\module{Théorie des groupes}
\niveau{L3}
\difficulte{}

\contenu{
\texte{
Soit $G$ le groupe $\Qq/\Zz$. Si $q\in\Q$, on note
$\text{cl}(q)$ la classe de $q$ modulo $~\Zz.$
}
\begin{enumerate}
    \item \question{Montrer que $\text{cl}(\frac {35}{6})=\text{cl}(\frac 56)$ et
d\'eterminer l'ordre de $\text{cl}(\frac
{35}{6})$.}
    \item \question{Montrer que si $x\in G$ il existe un unique $\alpha \in \Qq\cap
\left[ 0,1\right[ $ tel que $x=\text{cl}(\alpha ).$}
    \item \question{Montrer que tout \'el\'ement de $G$ est d'ordre fini et qu'il existe
des \'el\'ements d'ordre arbitraire.}
\end{enumerate}
}

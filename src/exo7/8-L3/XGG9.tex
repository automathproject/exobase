\uuid{XGG9}
\exo7id{2108}
\titre{exo7 2108}
\auteur{debes}
\organisation{exo7}
\datecreate{2008-02-12}
\isIndication{false}
\isCorrection{true}
\chapitre{Ordre d'un élément}
\sousChapitre{Ordre d'un élément}
\module{Théorie des groupes}
\niveau{L3}
\difficulte{}

\contenu{
\texte{
Soit $E$ un mono\" \i de unitaire. On dit qu'un \'el\'ement $a$
de $E$ admet un {\it inverse \`a gauche}  (resp. {\it inverse \`a droite}) s'il existe $b \in
E$ tel que $ba=e$ (resp. $ab=e$). \smallskip

(a) Supposons qu'un \'el\'ement $a$ admette un inverse \`a gauche $b$  qui lui-m\^eme admet un
inverse \`a gauche. Montrer que $a$ est inversible. \smallskip

(b) Supposons que tout \'el\'ement de $E$ admette un inverse \`a gauche. Montrer que $E$ est
un groupe.
}
\reponse{
(a) D\'esignant par $b$ l'inverse \`a gauche de $a$ et par $c$ l'inverse \`a gauche de $b$,
on a $ab= (cb)(ab)= c(ba)b=cb=e$. L'\'el\'ement $b$ est donc l'inverse de $a$.
\smallskip

(b) d\'ecoule imm\'ediatement de (a).
\smallskip
}
}

\uuid{C7Rk}
\exo7id{6396}
\titre{exo7 6396}
\auteur{potyag}
\organisation{exo7}
\datecreate{2011-10-16}
\isIndication{false}
\isCorrection{false}
\chapitre{Isométrie euclidienne}
\sousChapitre{Isométrie euclidienne}
\module{Algèbre et géométrie}
\niveau{L3}
\difficulte{}

\contenu{
\texte{
Soit $A B C$ un triangle isocèle en $A$ non equilatéral, le
but de cet exercice est  d'étudier l'ensemble des isométries
de $\mathcal{P}$ qui préservent globalement $A B C$.
}
\begin{enumerate}
    \item \question{Montrer que cet ensemble est groupe.}
    \item \question{Montrer que si $f$ préserve $A B C $ alors $f$ fixe le
barycentre $G$ de $A B C $.}
    \item \question{En étudiant les distances $GA$, $GB$, $GC$ montrer que
$f(A)=A$}
    \item \question{En déduire (en utilisant la classification des isométries
de $\Rr^2$) le groupe de symétries de $ABC$.}
\end{enumerate}
}

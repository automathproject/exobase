\uuid{BX6P}
\exo7id{6418}
\titre{exo7 6418}
\auteur{potyag}
\organisation{exo7}
\datecreate{2011-10-16}
\isIndication{false}
\isCorrection{false}
\chapitre{Géométrie projective}
\sousChapitre{Géométrie projective}
\module{Algèbre et géométrie}
\niveau{L3}
\difficulte{}

\contenu{
\texte{
Soit $L$ un espace vectoriel de dimension $n+1.$
}
\begin{enumerate}
    \item \question{Montrer que si $M_i\ (i\in I)$ sont des sous-espaces vectoriels de
  $L$ alors  $\displaystyle P(\bigcap_{i\in I} M_i)=\bigcap_{i\in
  I} P(M_i).$}
    \item \question{Soient $M_i\ (i\in \{1,...,k\}$ des sous-espaces linéaires de $L,$
  montrer que $$<P(M_1),..., P(M_k)>
 =P(M_1+...+M_k).$$}
    \item \question{Soient $p:L^*\to P(L)=L^{*} / _{\bf\tilde{}}\ $ l'application de la
  projectivisation (qui associe à chaque $x\in L^*$ sa classe $[x]\in P(L)$)
  et $S$ un sous-ensemble de l'espace $P(L).$ Alors montrer que $<S>= P(D),$
  où $D$ est le sous-espace de $L$ engendré par $p^{-1}(S).$}
\end{enumerate}
}

\uuid{VZvv}
\exo7id{6417}
\auteur{potyag}
\organisation{exo7}
\datecreate{2011-10-16}
\isIndication{false}
\isCorrection{false}
\chapitre{Géométrie projective}
\sousChapitre{Géométrie projective}

\contenu{
\texte{
Trouver la formule explicite suivante pour la projection stéréographique $\pi
: \R^n\cup\{\infty\}\to  S^n:$

$$\displaystyle \pi(x)=\Big(\frac{2x_1}{1+\vert\vert x\vert\vert^2},\ ... ,\
\frac{2x_n}{1+\vert\vert x\vert\vert^2},\ \frac{\vert\vert
x\vert\vert^2 -1}{\vert\vert x\vert\vert^2+1}\Big),$$
où $x=(x_1,...,x_n,0)\in \R^n\subset \Rr^{n+1}.$

{\it Indication:} \'Ecrire $\pi(x)-e_{n+1}=t(x-e_{n+1}),\ t\in\R.$
}
}

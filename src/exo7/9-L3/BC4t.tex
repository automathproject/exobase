\uuid{BC4t}
\exo7id{6424}
\titre{exo7 6424}
\auteur{potyag}
\organisation{exo7}
\datecreate{2011-10-16}
\isIndication{false}
\isCorrection{false}
\chapitre{Géométrie projective}
\sousChapitre{Géométrie projective}
\module{Algèbre et géométrie}
\niveau{L3}
\difficulte{}

\contenu{
\texte{
Cet exercice ne concerne pas directement la géométrie
projective mais sera utilisé par la suite.
}
\begin{enumerate}
    \item \question{Montrer que chaque réflexion par rapport à un hyperplan dans
$\R^n$
  est une application conforme dans  $\R^n.$ En déduire que chaque
  isométrie euclidienne et chaque isométrie sphérique sont
  conformes dans $\R^n$ et sur $ S^n$ respectivement.}
    \item \question{Montrer qu'une application linéaire $A:E\mapsto E$
  conserve les angles non-orientés  entre les vecteurs non-nuls ssi $A$
  est une matrice conforme.}
    \item \question{Montrer qu'une application $f:D\mapsto \R^n$ d'un
  ouvert $D\subset \R^n$ est conforme dans $D$ ssi elle conserve les
  angles dans $D$.}
\end{enumerate}
}

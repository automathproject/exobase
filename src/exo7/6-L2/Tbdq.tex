\uuid{Tbdq}
\exo7id{7069}
\auteur{megy}
\organisation{exo7}
\datecreate{2017-01-11}
\isIndication{true}
\isCorrection{false}
\chapitre{Géométrie affine dans le plan et dans l'espace}
\sousChapitre{Propriétés des triangles}

\contenu{
\texte{
% triangles rectangles
À l'extérieur d'un triangle $BOA$ on construit deux triangles rectangles :
\begin{itemize}
\item Le triangle $OAC$, ayant pour hypoténuse le côté $[OA]$, tel que le sommet $C$ de l'angle droit soit situé sur la bissectrice extérieure de $OAB$.
\item Le triangle $OBE$, ayant pour hypoténuse le côté $[OB]$, tel que le sommet $E$ de l'angle droit soit situé sur la bissectrice extérieure de $OBA$.
\end{itemize}
Que dire de $[EC]$ et de sa longueur ?
}
\indication{Le segment $[EC]$ est parallèle à $(AC)$, et sa longueur vaut la moitié du périmètre de $OAC$.}
}

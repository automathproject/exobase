\uuid{7rqT}
\exo7id{7508}
\titre{exo7 7508}
\auteur{mourougane}
\organisation{exo7}
\datecreate{2021-08-10}
\isIndication{false}
\isCorrection{true}
\chapitre{Géométrie affine euclidienne}
\sousChapitre{Géométrie affine euclidienne du plan}
\module{Géométrie}
\niveau{L2}
\difficulte{}

\contenu{
\texte{
Soit $\R^2$ le plan affine euclidien muni du produit scalaire standard et de la base canonique.
}
\begin{enumerate}
    \item \question{Ecrire la matrice $A$ de la forme bilinéaire symétrique donnée en coordonnées par
    $$f(\begin{pmatrix}x\\y\end{pmatrix},\begin{pmatrix}x'\\y'\end{pmatrix})
    =xx'+19yy'+12xy'+12x'y.$$}
\reponse{La matrice $A$ de la forme bilinéaire symétrique donnée en coordonnées par
    $$f(\begin{pmatrix}x\\y\end{pmatrix},\begin{pmatrix}x'\\y'\end{pmatrix})
    =xx'+19yy'+12xy'+12x'y$$
    est $$\begin{pmatrix}
    1&12\\12&19
    \end{pmatrix}$$}
    \item \question{Diagonaliser $A$ dans une base orthonormée.}
\reponse{On trouve que $e_1+2e_2$ est vecteur propre de $A$ de valeur propre $25$ et $2e_1-e_2$ est vecteur propre de $A$ de valeur propre $-5$.}
    \item \question{Déterminer la nature de la conique d'équation
    $$x^2+19y^2+24xy+5y=0.$$}
\reponse{Comme la matrice $A$ est inversible, la conique d'équation
    $$x^2+19y^2+24xy+5y=0$$
    est une conique à centre. Comme les valeurs propres de $A$ sont de signe opposé, il s'agit d'une hyperbole.}
\end{enumerate}
}

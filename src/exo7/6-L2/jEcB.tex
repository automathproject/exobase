\uuid{jEcB}
\exo7id{7067}
\auteur{megy}
\organisation{exo7}
\datecreate{2017-01-11}
\isIndication{true}
\isCorrection{true}
\chapitre{Géométrie affine dans le plan et dans l'espace}
\sousChapitre{Propriétés des triangles}

\contenu{
\texte{
% collège
% triangles isocèles, médiatrices, cercle circonscrit
Trois cercles sont tangents extérieurement deux à deux. Montrer que les tangentes communes sont concourantes.
}
\indication{Utiliser des triangles isocèles.}
\reponse{
L'exercice se résout assez simplement en utilisant trois triangles isocèles, mais on peut remarquer que les trois tangentes sont les trois axes radicaux, qui s'intersectent tous trois  au centre radical des trois cercles.
}
}

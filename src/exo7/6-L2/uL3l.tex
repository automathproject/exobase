\uuid{uL3l}
\exo7id{5035}
\titre{exo7 5035}
\auteur{quercia}
\organisation{exo7}
\datecreate{2010-03-17}
\isIndication{false}
\isCorrection{true}
\chapitre{Courbes planes}
\sousChapitre{Propriétés métriques : longueur, courbure,...}
\module{Géométrie}
\niveau{L2}
\difficulte{}

\contenu{
\texte{
Soit $\mathcal{C}$ la courbe d'équation cartésienne $ x^4+y^4+x^3+y^3=2$.
En utilisant le théorème des fonctions implicites, calculer la courbure de $\mathcal{C}$
en $A = (-1,1)$.
}
\reponse{
$-\frac{156}{125\sqrt2}$.
}
}

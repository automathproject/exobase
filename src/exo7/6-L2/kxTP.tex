\uuid{kxTP}
\exo7id{2698}
\auteur{matexo1}
\organisation{exo7}
\datecreate{2002-02-01}
\isIndication{false}
\isCorrection{false}
\chapitre{Géométrie affine dans le plan et dans l'espace}
\sousChapitre{Géométrie affine dans le plan et dans l'espace}

\contenu{
\texte{
Soit $ABC$ un triangle \'equilat\'eral de
c\^ot\'e unit\'e et $T_0$ son int\'erieur. On consid\`ere les figures
g\'eom\'etriques $T_n$ obtenues par r\'ecurrence de la mani\`ere
suivante\,:
sur chaque c\^ot\'e $MN$ de $T_{n-1}$, on ajoute l'int\'erieur d'un
triangle \'equilat\'eral
$PQR$, o\`u $P$ et $Q$ sont sur le segment $[MN]$, aux tiers de sa
longueur, et $R$ est
ext\'erieur \`a $T_{n-1}$. Finalement on d\'efinit le sous-ensemble du
plan $K$ par
$$ K = \bigcup_{n\geq 0} T_n.$$
}
\begin{enumerate}
    \item \question{Faire un dessin respr\'esentant $T_0$, $T_1$, $T_2$...}
    \item \question{Donner l'aire de $T_n$ sous forme de s\'erie. Quelle est l'aire de
$K$\,?}
    \item \question{M\^emes questions avec le p\'erim\`etre, puis le diam\`etre de
$T_n$ et $K$.}
\end{enumerate}
}

\uuid{mzGA}
\exo7id{5825}
\auteur{rouget}
\organisation{exo7}
\datecreate{2010-10-16}
\isIndication{false}
\isCorrection{true}
\chapitre{Conique}
\sousChapitre{Quadrique}

\contenu{
\texte{
Nature et \og éléments caractéristiques \fg~de la quadrique $(\mathcal{S})$ dont une équation dans un repère orthonormé donné $\mathcal{R}=(O,i,j,k)$ de l'espace de dimension $3$ est :
}
\begin{enumerate}
    \item \question{$x^2+y^2+z^2-2yz-4x+4y-1=0$.}
\reponse{Pour $(x,y,z)\in\Rr^3$, on pose $Q(x,y,z)=x^2+y^2+z^2-2yz$.

Pour tout $(x,y,z)\in\Rr^3$, $x^2+y^2+z^2-2yz = x^2 + (y-z)^2 = X^2 + 2Y^2$ en posant $X = x$, $Y = \frac{1}{\sqrt{2}}(y-z)$ et $Z =\frac{1}{\sqrt{2}}(y+z)$ correspondant au changement de bases orthonormées de matrice $P =\left(
\begin{array}{ccc}
1&0&0\\
\rule[-5mm]{0mm}{11mm}0&\frac{1}{\sqrt{2}}&\frac{1}{\sqrt{2}}\\
0&-\frac{1}{\sqrt{2}}&\frac{1}{\sqrt{2}}
\end{array}
\right)$.

Notons $\mathcal{R}'= (O,e_1,e_2,e_3)$ le repère orthonormé ainsi défini. La surface $(\mathcal{S})$ admet pour équation dans $\mathcal{R}?$ $X^2+2Y^2-4X+2\sqrt{2}(Y+Z) -1 = 0$ ou encore 

\begin{center}
$(X-2)^2 +2\left(Y+\frac{1}{\sqrt{2}}\right)^2=-2\sqrt{2}\left(Z-\frac{3}{\sqrt{2}}\right)$.
\end{center}

La surface $(\mathcal{S})$ est un paraboloïde elliptique de sommet $S$ de coordonnées $\left(2,\frac{1}{\sqrt{2}},\frac{3}{\sqrt{2}}\right)$ dans $\mathcal{R}'$ et donc $\left(2,2,1\right)$ dans $\mathcal{R}$.}
    \item \question{$x^2+y^2+z^2+2xy-1=0$.}
\reponse{En posant $X=\frac{1}{\sqrt{2}}(x+y)$, $Y=\frac{1}{\sqrt{2}}(-x+y)$ et $Z = z$, on obtient : $2X^2 + Z^2 = 1$.

La surface $(\mathcal{S})$ est un cylindre elliptique d'axe $(OY)$ ou encore d'axe la droite d'équations $\left\{
\begin{array}{l}
y=-x\\
z=0
\end{array}
\right.$.}
    \item \question{$x^2+y^2+z^2-2xy+2xz+3x-y+z+1=0$.}
\reponse{Pour $(x,y,z)\in\Rr^3$, posons $Q(x,y,z)=x^2+y^2+z^2-2xy+2xz$.

La matrice de $Q$ dans la base canonique $(i,j,k)$ de $\Rr^3$ est $A=\left(
\begin{array}{ccc}
1&-1&1\\
-1&1&0\\
1&0&1
\end{array}
\right)$.

\begin{align*}\ensuremath
\chi_A&=\left|
\begin{array}{ccc}
1-X&-1&1\\
-1&1-X&0\\
1&0&1-X
\end{array}
\right|= (1-X)^3+(X-1)+(X-1) = (1-X)((1-X)^2 - 2)\\
 &= (1-X)(1+\sqrt{2}-X)(1+\sqrt{2}-X).
\end{align*}

$Q$ est de rang $3$ et de signature $(2,1)$. La surface $(\mathcal{S})$ peut être un hyperboloïde à une ou deux nappes ou un cône de révolution.

$\text{Ker}(A-I_3) =\text{Vect}(e_1)$ où $e_1=\frac{1}{\sqrt{2}}(0,1,1)$.
  

$\text{Ker}(A-(1+\sqrt{2})I_3)=\text{Vect}(e_2)$ où $e_2=\frac{1}{2}(\sqrt{2},-1,1)$ et  $\text{Ker}(A-(1-\sqrt{2})I_3)=\text{Vect}(e_3)$ où $e_3=\frac{1}{2}(\sqrt{2},1,-1)$

 
La matrice de passage correspondante est la matrice $P=\left(
\begin{array}{ccc}
0&\frac{1}{\sqrt{2}}&\frac{1}{\sqrt{2}}\\
\rule[-4mm]{0mm}{10mm}\frac{1}{\sqrt{2}}&-\frac{1}{2}&\frac{1}{2}\\
\frac{1}{\sqrt{2}}&\frac{1}{2}&-\frac{1}{2}
\end{array}
\right)$.

Déterminons une équation réduite de la surface $(\mathcal{S})$ dans le repère $(O,e_1,e_2,e_3)$.

\begin{align*}\ensuremath
X^2+&(1+\sqrt{2})Y^2+(1-\sqrt{2})Z^2+\frac{3}{\sqrt{2}}(Y+Z)-\frac{1}{2}(\sqrt{2}X-Y+Z)+\frac{1}{2}(\sqrt{2}X+Y-Z)+1 = 0\\
 &\Leftrightarrow X^2+(1+\sqrt{2})\left(Y^2+\frac{3+\sqrt{2}}{2+\sqrt{2}}Y\right)+(1-\sqrt{2})\left(Z^2-\frac{3-\sqrt{2}}{2-\sqrt{2}}Z\right)+1 = 0\\
 &\Leftrightarrow X^2+(1+\sqrt{2})\left(Y+\frac{3+\sqrt{2}}{2(2+\sqrt{2})}\right)^2-\frac{(3+\sqrt{2})^2}{8(1+\sqrt{2})}+(1-\sqrt{2})\left(Z-\frac{3-\sqrt{2}}{2(2-\sqrt{2})}\right)^2-\frac{(3-\sqrt{2})^2}{8(1-\sqrt{2})}+1 = 0\\
 &\Leftrightarrow X^2+(1+\sqrt{2})\left(Y+1-\frac{\sqrt{2}}{4}\right)^2+(1-\sqrt{2})\left(Z-1-\frac{\sqrt{2}}{4}\right)^2 =\frac{11+6\sqrt{2}}{8(1+\sqrt{2})}+\frac{11-6\sqrt{2}}{8(1-\sqrt{2})}-1\\
 &\Leftrightarrow X^2+(1+\sqrt{2})\left(Y+1-\frac{\sqrt{2}}{4}\right)^2+(1-\sqrt{2})\left(Z-1-\frac{\sqrt{2}}{4}\right)^2 =-\frac{3}{4}\\
  &\Leftrightarrow-\frac{4}{3}X^2-\frac{4(1+\sqrt{2})}{3}\left(Y+1-\frac{\sqrt{2}}{4}\right)^2+\frac{4(\sqrt{2}-1)}{3}\left(Z-1-\frac{\sqrt{2}}{4}\right)^2 =1.
\end{align*}

La surface $(\mathcal{S})$ est un hyperboloïde à deux nappes de centre de coordonnées $\left(0,-1+\frac{\sqrt{2}}{4} ,1+\frac{\sqrt{2}}{4}\right)$ dans le repère $\mathcal{R}'$.}
    \item \question{$x^2+4y^2+5z^2-4xy-2x+4y=0$.}
\reponse{On pose $X =\frac{1}{\sqrt{5}}(x-2y)$, $Y =\frac{1}{\sqrt{5}}(2x+y)$ et $Z = z$. Dans le repère $\mathcal{R}'$ ainsi défini, la surface $(\mathcal{S})$ admet pour équation $5X^2 + 5Z^2-\frac{2}{\sqrt{5}}(X+2Y) +\frac{4}{\sqrt{5}}(-2X+Y) = 0$ ou encore 
$5\left(X-\frac{1}{\sqrt{5}}\right)^2+5Z^2 = 1$.

La surface $(\mathcal{S})$ est un cylindre de révolution d'axe la droite d'équations $\left\{
\begin{array}{l}
X=\frac{1}{\sqrt{5}}\\
Z=0
\end{array}
\right.$  dans $\mathcal{R}'$ et de rayon $\frac{1}{\sqrt{5}}$.}
    \item \question{$x^2-4x-3y-2=0$.}
\reponse{$x^2-4x-3y-2 = 0\Leftrightarrow(x-2)^2 =3(y+2)$. La surface $(\mathcal{S})$ est un cylindre parabolique de direction $(Oz)$.}
    \item \question{$7x^2-2y^2+4z^2+4xy+20xz+16yz-36x+72y-108z+36=0$.}
\reponse{Pour $(x,y,z)\in\Rr^3$, posons $Q(x,y,z)= 7x^2-2y^2+4z^2+4xy+20xz+16yz$. La matrice de $Q$ dans la base canonique $(i,j,k)$ de $\Rr^3$ est $A =\left(
\begin{array}{ccc}
7&2&10\\
2&-2&8\\
10&8&4
\end{array}
\right)$.

\begin{align*}\ensuremath
\chi_A&=\left|
\begin{array}{ccc}
7-X&2&10\\
2&-2-X&8\\
10&8&4-X
\end{array}
\right|= (7-X)(X^2-2X-72) - 2(-2X-72) + 10(10X+36)\\
 &= -X^3+9X^2+162X= -X(X+9)(X-18) .
\end{align*}

Donc $Q$ est de rang $2$ et de signature $(1,1)$.

$(x,y,z)\in\text{Ker}(A)\Leftrightarrow\left\{
\begin{array}{l}
7x+2y+10z=0\\
2x-2y+4z=0\\
10x+8y+4z=0
\end{array}
\right.\Leftrightarrow\left\{
\begin{array}{l}
x-y=-4z\\
5x+4y=-2z
\end{array}
\right.\Leftrightarrow\left\{
\begin{array}{l}
x=-2z\\
y=2z
\end{array}
\right.$ et $\text{Ker}(A)=\text{Vect}(e_1)$ où $e_1=\frac{1}{3}(-2,2,1)$.
  

$(x,y,z)\in\text{Ker}(A+9I_3)\Leftrightarrow\left\{
\begin{array}{l}
16x+2y+10z=0\\
2x+7y+8z=0\\
10x+8y+13z=0
\end{array}
\right.\Leftrightarrow\left\{
\begin{array}{l}
y=-8x-5z\\
z=-2x
\end{array}
\right.\Leftrightarrow\left\{
\begin{array}{l}
y=2x\\
z=-2x
\end{array}
\right.$  et $\text{Ker}(A+9I_3)=\text{Vect}(e_2)$ où $e_2 =\frac{1}{3}(1,2,-2)$.

$\text{Ker}(A-18I_3) =\text{Vect}(e_3)$ où $e3 =-e_1\wedge e_2 =\frac{1}{3}(2,1,2)$.

La matrice de passage du changement de bases ainsi défini est $P =\frac{1}{3}\left(
\begin{array}{ccc}
-2&1&2\\
2&2&1\\
1&-2&2
\end{array}
\right)$.

Déterminons une équation réduite de la surface $(\mathcal{S})$ dans le repère $\mathcal{R}'= (O,e_1,e_2,e_3)$

\begin{align*}\ensuremath
7x^2-&2y^2+4z^2+4xy+20xz+16yz-36x+72y-108z+36=0\\
 &\Leftrightarrow-9Y^2+18Z^2 -12(-2X+Y+2Z)+24(2X+2Y+Z)-36(X-2Y+2Z)+36 = 0\\
 &\Leftrightarrow-9Y^2+18Z^2+36X+108Y-72Z+36 = 0\Leftrightarrow-Y^2+2Z^2+4X+12Y-8Z+4=0\\
 &\Leftrightarrow4(X+8)=(Y-6)^2 - 2(Z-2)^2.
\end{align*}

La surface $(\mathcal{S})$ est un paraboloïde hyperbolique. Son point selle est le point de cordonnées $(-8,6,2)$ dans le repère $\mathcal{R}'$.}
    \item \question{$(x-y)(y-z)+(y-z)(z-x)+(z-x)(x-y)+(x-y) = 0$.}
\reponse{La surface $(\mathcal{S})$ admet pour équation cartésienne : $x^2+y^2+z^2-xy-xz-zx-x+y = 0$.

Pour $(x,y,z)\in\Rr^3$, posons $Q(x,y,z)= x^2+y^2+z^2-xy-xz-zx$. La matrice de $Q$ dans la base canonique $(i,j,k)$ de $\Rr^3$ est $A=\left(
\begin{array}{ccc}
1&-\frac{1}{2}&-\frac{1}{2}\\
\rule[-4mm]{0mm}{10mm}-\frac{1}{2}&1&-\frac{1}{2}\\
-\frac{1}{2}&-\frac{1}{2}&1
\end{array}
\right)$. $\text{Sp}(A)=\left(\frac{3}{2},\frac{3}{2},0\right)$. Un base orthonormée $(e_1,e_2,e_3)$ de vecteurs propres est la famille de matrice $P=\left(
\begin{array}{ccc}
\frac{1}{\sqrt{2}}&\frac{1}{\sqrt{6}}&\frac{1}{\sqrt{3}}\\
\rule[-4mm]{0mm}{10mm}-\frac{1}{\sqrt{2}}&\frac{1}{\sqrt{6}}&\frac{1}{\sqrt{3}}\\
0&-\frac{2}{\sqrt{6}}&\frac{1}{\sqrt{3}}
\end{array}
\right)$.

Dans le repère $\mathcal{R}'=(O,e_1,e_2,e_3)$, la surface $(\mathcal{S})$ admet pour équation cartésienne 
$\frac{3}{2}(X^2+Y^2) -\left(\frac{1}{\sqrt{2}}X+\frac{1}{\sqrt{6}}Y+\frac{1}{\sqrt{3}}Z\right)+\left(-\frac{1}{\sqrt{2}}X+\frac{1}{\sqrt{6}}Y+\frac{1}{\sqrt{3}}Z\right) = 0$ ou encore $\frac{3}{2}(X^2+Y^2) -\sqrt{2}X = 0$ ou enfin $\left(X-\frac{\sqrt{2}}{3}\right)^2 + Y^2 =\frac{2}{9}$.

La surface $(\mathcal{S})$ est un cylindre de révolution d'axe la droite d'équation $\left\{
\begin{array}{l}
X=\frac{\sqrt{2}}{3}\\
Y=0
\end{array}
\right.$ dans le repère $\mathcal{R}'$ et de rayon $\frac{\sqrt{2}}{3}$.}
    \item \question{$xy+yz =1$.}
\reponse{En posant $X=y$, $Y=\frac{1}{\sqrt{2}}(y+z)$ (et $Z=\frac{1}{\sqrt{2}}(-y+z)$, $xy + yz=0\Leftrightarrow XY=\sqrt{2}$.

La surface $(\mathcal{S})$ est un cylindre hyperbolique.}
    \item \question{$xy+yz+zx+2y+1=0$.}
\reponse{Pour $(x,y,z)\in\Rr^3$, posons $Q(x,y,z)=xy+yz+zx$.

La matrice de $Q$ dans la base canonique $(i,j,k)$ de $\Rr^3$ est $\frac{1}{2}\left(
\begin{array}{ccc}
0&1&1\\
1&0&1\\
1&1&0
\end{array}
\right)$. $\text{Sp}(A)=(-\frac{1}{2},-\frac{1}{2},1)$ et donc la surface $(\mathcal{S})$ est soit un hyperboloïde à une ou deux nappes, soit un cône du second degré et dans tous les cas une surface de révolution (puisque les deux valeurs propres négatives sont égales) d'axe de direction 
$\text{Ker}(A-I_3) =\text{Vect}(1,1,1)$ et passant par le point critique $\Omega(-1,1,-1)$.

Quand on se place dans le repère $(\Omega,i,j,k)$, la surface $(\mathcal{S}$) admet pour équation $XY+YZ+ZX +2 = 0$ (car $f(-1,1,-1)=2$) puis dans le repère $(\Omega,e_1,e_2,e_3)$, $-\frac{1}{2}X^2-\frac{1}{2}Y^2+Z^2 +2 = 0$ ou encore $\frac{1}{4}X^2+\frac{1}{4}Y^2-\frac{1}{2}Z^2=1$.

La surface $(\mathcal{S})$ est un hyperboloïde de révolution à une nappe.}
\end{enumerate}
}

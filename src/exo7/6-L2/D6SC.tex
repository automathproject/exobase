\uuid{D6SC}
\exo7id{7075}
\titre{exo7 7075}
\auteur{megy}
\organisation{exo7}
\datecreate{2017-01-11}
\isIndication{true}
\isCorrection{false}
\chapitre{Géométrie affine dans le plan et dans l'espace}
\sousChapitre{Propriétés des triangles}
\module{Géométrie}
\niveau{L2}
\difficulte{}

\contenu{
\texte{
Soit $ABC$ un triangle dont on note $a$, $b$ et $c$ les longueurs des côtés.
}
\begin{enumerate}
    \item \question{Exprimer l'aire $S$ du triangle en fonction du périmètre $a+b+c=2p$ et du rayon $r$ du cercle inscrit.}
    \item \question{Exprimer également $S$ en fonction de $a$ et du rayon $r_A$ du cercle exinscrit en $A$.}
    \item \question{En déduire $\frac{1}{r} = \frac{1}{r_A}+\frac{1}{r_B}+\frac{1}{r_C}$.}
\indication{Partitionner le triangle en plusieurs triangles pour calculer l'aire.}
\end{enumerate}
}

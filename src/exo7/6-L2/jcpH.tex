\uuid{jcpH}
\exo7id{5548}
\titre{exo7 5548}
\auteur{rouget}
\organisation{exo7}
\datecreate{2010-07-15}
\isIndication{false}
\isCorrection{true}
\chapitre{Conique}
\sousChapitre{Parabole}
\module{Géométrie}
\niveau{L2}
\difficulte{}

\contenu{
\texte{
Soit, dans $\Rr^3$ rapporté à un repère orthonormé $(O,\overrightarrow{i},\overrightarrow{j},\overrightarrow{k})$, la courbe
$(\Gamma)$ d'équations $\left\{
\begin{array}{l}
y=x^2+x+1\\
x+y+z=1
\end{array}\right.$. Montrer que $(\Gamma)$ est une parabole dont on déterminera le sommet, l'axe, le foyer et la 
directrice.
}
\reponse{
On choisit un repère orthonormé $\mathcal{R}_1=\left(O',\overrightarrow{I},\overrightarrow{J},\overrightarrow{K}\right)$ tel que le plan d'équation $x+y+z=1$ dans $\mathcal{R}$ soit le plan d'équation $Z=0$ dans $\mathcal{R}_1$.
On prend $O'=(1,0,0)$ puis $\overrightarrow{K}=\left(\frac{1}{\sqrt{3}},\frac{1}{\sqrt{3}},\frac{1}{\sqrt{3}}\right)$, $\overrightarrow{I}=\left(\frac{1}{\sqrt{2}},-\frac{1}{\sqrt{2}},0\right)$ et enfin $\overrightarrow{J}=\overrightarrow{K}\wedge\overrightarrow{I}=\left(\frac{1}{\sqrt{6}},\frac{1}{\sqrt{6}},-\frac{2}{\sqrt{6}}\right)$. Les formules de changement de repère s'écrivent

\begin{center}
$\left\{
\begin{array}{l}
x=\frac{X}{\sqrt{2}}+\frac{Y}{\sqrt{6}}+\frac{Z}{\sqrt{3}}+1\\
y=-\frac{X}{\sqrt{2}}+\frac{Y}{\sqrt{6}}+\frac{Z}{\sqrt{3}}\\
z=-\frac{2Y}{\sqrt{6}}+\frac{Z}{\sqrt{3}}
\end{array}
\right.
$
\end{center}
Ensuite, soit $M$ un point de l'espace dont les coordonnées dans $\mathcal{R}$ sont notées $(x,y,z)$ et les coordonnées dans $\mathcal{R}_1$ sont notées $(X,Y,Z)$.

\begin{align*}\ensuremath
M\in(\Gamma)&\Leftrightarrow\left\{
\begin{array}{l}
y=x^2+x+1\\
x+y+z=1
\end{array}\right.\Leftrightarrow\left\{
\begin{array}{l}
-\frac{X}{\sqrt{2}}+\frac{Y}{\sqrt{6}}+\frac{Z}{\sqrt{3}}=\left(\frac{X}{\sqrt{2}}+\frac{Y}{\sqrt{6}}+\frac{Z}{\sqrt{3}}+1\right)^2+\left(\frac{X}{\sqrt{2}}+\frac{Y}{\sqrt{6}}+\frac{Z}{\sqrt{3}}+1\right)+1\\
Z=0
\end{array}\right.
\\
 &\Leftrightarrow\left\{
\begin{array}{l}
Z=0\\
-\frac{X}{\sqrt{2}}+\frac{Y}{\sqrt{6}}=\left(\frac{X}{\sqrt{2}}+\frac{Y}{\sqrt{6}}+1\right)^2+\frac{X}{\sqrt{2}}+\frac{Y}{\sqrt{6}}+2\end{array}\right.
\\
 &\Leftrightarrow\left\{
\begin{array}{l}
Z=0\\
\left(\frac{X}{\sqrt{2}}+\frac{Y}{\sqrt{6}}\right)^2+\frac{4X}{\sqrt{2}}+\frac{2Y}{\sqrt{6}}+3=0\end{array}\right..
\end{align*}
On travaille maintenant en dimension $2$ et on note encore $\mathcal{R}_1$ le repère $\left(O',\overrightarrow{I},\overrightarrow{J}\right)$. Une équation de $(\Gamma)$ dans $\mathcal{R}_1$ est 
$\left(\frac{X}{\sqrt{2}}+\frac{Y}{\sqrt{6}}\right)^2+\frac{4X}{\sqrt{2}}+\frac{2Y}{\sqrt{6}}+3=0$ ou encore $\frac{2}{3}\left(\frac{\sqrt{3}X}{2}+\frac{Y}{2}\right)^2+\frac{4X}{\sqrt{2}}+\frac{2Y}{\sqrt{6}}+3=0$.
On pose $\left\{
\begin{array}{l}
x'=\frac{\sqrt{3}X}{2}+\frac{Y}{2}\\
\rule{0mm}{7mm}y'=-\frac{X}{2}+\frac{\sqrt{3}Y}{2}
\end{array}
\right.$ ou encore $\left\{
\begin{array}{l}
X=\frac{\sqrt{3}x'}{2}-\frac{y'}{2}\\
\rule{0mm}{7mm}Y=\frac{x'}{2}+\frac{\sqrt{3}y'}{2}
\end{array}
\right.$ et on note $\mathcal{R}'=\left(O',\overrightarrow{i'},\overrightarrow{j'}\right)$ le nouveau repère défini par ces formules.
\begin{align*}\ensuremath
M\in(\Gamma)&\Leftrightarrow\frac{2}{3}x'^2+\frac{4}{\sqrt{2}}\left(\frac{\sqrt{3}x'}{2}-\frac{y'}{2}\right)+\frac{2}{\sqrt{6}}\left(\frac{x'}{2}+\frac{\sqrt{3}y'}{2}\right)+3=0\Leftrightarrow\frac{2}{3}x'^2+\frac{7x'}{\sqrt{6}}-\frac{y'}{\sqrt{2}}+3=0\\
 &\Leftrightarrow\frac{2}{3}\left(x'+\frac{21}{4\sqrt{6}}\right)^2=\frac{y'}{\sqrt{2}}+\frac{1}{16}\Leftrightarrow\left(x'+\frac{21}{4\sqrt{6}}\right)^2=\frac{3}{2\sqrt{2}}\left(y'+\frac{1}{8\sqrt{2}}\right).
\end{align*}
$\left(\Gamma\right)$ est une parabole de paramètre $p=\frac{3}{4\sqrt{2}}$.
Eléments de $(\Gamma)$ dans $\mathcal{R}'$ : sommet $S\left(-\frac{21}{4\sqrt{6}},-\frac{1}{8\sqrt{2}}\right)_{\mathcal{R}'}$, axe : $x'=-\frac{21}{4\sqrt{6}}$, foyer $F\left(-\frac{21}{4\sqrt{6}},\frac{1}{4\sqrt{2}}\right)_{\mathcal{R}'}$, directrice : $y'=-\frac{1}{2\sqrt{2}}$.
Eléments de $(\Gamma)$ dans $\mathcal{R}_1$ en repassant à trois coordonnées : sommet $S\left(-\frac{41}{16\sqrt{2}},-\frac{45}{16\sqrt{6}},0\right)_{\mathcal{R}_1}$, axe : $\left\{
\begin{array}{l}
\sqrt{3}X+Y=-\frac{21}{2\sqrt{6}}\\
Z=0\\
\end{array}
\right.$, foyer $F\left(-\frac{11}{4\sqrt{2}},-\frac{3}{\sqrt{6}},0\right)_{\mathcal{R}_1}$, directrice : $\left\{
\begin{array}{l}
-X+\sqrt{3}Y=-\frac{1}{\sqrt{2}}\\
Z=0
\end{array}
\right.$.
Eléments de $(\Gamma)$ dans $\mathcal{R}$ : sommet $S\left(-\frac{3}{4},\frac{13}{16},\frac{15}{16}\right)_{\mathcal{R}}$, axe : $\left\{
\begin{array}{l}
8x-4y-4z+21=0\\
x+y+z=1\\
\end{array}
\right.$, foyer $F\left(-\frac{7}{8},\frac{7}{8},10\right)_{\mathcal{R}}$, directrice : $\left\{
\begin{array}{l}
2y-2z+1=0\\
x+y+z=1
\end{array}
\right.$.
}
}

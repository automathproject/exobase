\uuid{m8Tv}
\exo7id{7090}
\auteur{megy}
\organisation{exo7}
\datecreate{2017-01-21}
\isIndication{true}
\isCorrection{false}
\chapitre{Géométrie affine euclidienne}
\sousChapitre{Géométrie affine euclidienne du plan}

\contenu{
\texte{
% source : Leitchnam
% se fait facilement en coisissant des coordonnées affines, en fait.

Soit $ABCD$ un parallélogramme, et $M$ un point sur la diagonale $(BD)$. Soit $I$ le symétrique de $C$ par rapport à $M$. Soit $E$ la projection de $I$ sur $(AB)$ parallèlement à $(AD)$, et $F$ la projection de $I$ sur $(AD)$ parallèlement à $(AB)$. Montrer que $E$, $M$ et $F$ sont alignés.
}
\indication{Utiliser une homothétie et une symétrie centrale.}
}

\uuid{adeq}
\exo7id{7497}
\auteur{mourougane}
\organisation{exo7}
\datecreate{2021-08-10}
\isIndication{false}
\isCorrection{false}
\chapitre{Géométrie affine euclidienne}
\sousChapitre{Géométrie affine euclidienne du plan}

\contenu{
\texte{
On reprend les notations de l'exercice précédent. 
Soit $E$ un plan affine euclidien. Soit $\mathcal{R}=(O,i,j)$ un repère 
cartésien orthonormé de $E$. On considère la translation $t$ de vecteur $\vec{u}=3i+j$ et la symétrie orthogonale $s$ d'axe la droite $d$ d'équation $(x+y=1)$.
}
\begin{enumerate}
    \item \question{Décomposer le vecteur $\vec{u}$ dans la somme directe orthogonale
    $\vec{E}=\vec{d}^\perp\oplus\vec{d}$ en $\vec{u}=\vec{v}+\vec{w}$.}
    \item \question{Déterminer géométriquement la nature et les éléments caractéristiques de la composée 
    $t_{\vec{v}}\circ s$. On pourra décomposer chaque isométrie en produits de réflexions.}
    \item \question{Déterminer géométriquement la nature et les éléments caractéristiques de la composée 
    $t\circ s$.}
\end{enumerate}
}

\uuid{VrvQ}
\exo7id{7274}
\auteur{mourougane}
\organisation{exo7}
\datecreate{2021-08-10}
\isIndication{false}
\isCorrection{false}
\chapitre{Géométrie affine euclidienne}
\sousChapitre{Géométrie affine euclidienne du plan}

\contenu{
\texte{
On rappelle que l'\emph{homothétie} de centre \(O\) et de 
rapport \(\lambda\), pour \(O\) un point du plan \(\mathcal{P}\) et 
\(\lambda \in \Rr \smallsetminus \{0\}\), est l'application 
de \(\mathcal{P}\) dans lui-même qui à un point \(P \in \mathcal{P}\) 
associe l'unique point \(Q\) vérifiant
\(\overrightarrow{OQ} = \lambda \overrightarrow{OP}\).

Dans cet exercice, on considère une homothétie \(h\) de centre \(O\) 
et de rapport \(\lambda\).
}
\begin{enumerate}
    \item \question{Montrer que si \(\lambda \neq 1\), alors \(O\) est l'unique 
point du plan dont l'image par \(h\) est lui-même.}
    \item \question{La propriété précédente est-elle vraie si \(\lambda = 1\)? 
(Rappel: votre réponse doit être accompagnée d'une démonstration.)}
    \item \question{Si \(P\) est un point du plan autre que \(O\), montrer que 
\(h(P)\) est un point de la droite \((OP)\) (\emph{indication :} on pourra 
considérer les vecteurs \(\overrightarrow{OP}\) et 
\(\overrightarrow{O \; h(P)}\)).}
    \item \question{En déduire que si \(\mathcal{D}\) est une droite passant par 
le point \(O\), alors \(h(\mathcal{D}) \subseteq \mathcal{D}\). 
(Rappel: on note \(h(\mathcal{D})\) l'ensemble des points \(h(P)\) 
pour \(P \in \mathcal{D}\).)}
    \item \question{En considérant l'homothétie de centre \(O\) et de rapport 
\(\frac{1}{\lambda}\), en déduire que \(h(\mathcal{D}) = \mathcal{D}\) 
(avec les mêmes notations qu'à la question précédente).}
    \item \question{Soint \(\Delta\) une droite. On admet dans cet exercice que 
\(h(\Delta)\) est aussi une droite. Montrer que si \(\Delta\) et 
\(h(\Delta)\) ont un point commun et si \(\lambda \neq 1\) alors 
ce point commun est \(O\).}
    \item \question{En déduire que si \(\Delta\) est une droite ne passant pas 
par \(O\) et si \(\lambda \neq 1\) alors les droites \(\Delta\) et 
\(h(\Delta)\) sont parallèles et sont deux droites distinctes.}
    \item \question{Montrer que l'image d'une droite par une homothétie est une 
droite parallèle. (Attention, dans cette question, il n'y a plus 
d'hypothèse particulière sur la droite considérée, ni sur le rapport 
de l'homothétie.)}
\end{enumerate}
}

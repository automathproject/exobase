\uuid{0Z9r}
\exo7id{2044}
\titre{exo7 2044}
\auteur{liousse}
\organisation{exo7}
\datecreate{2003-10-01}
\isIndication{false}
\isCorrection{false}
\chapitre{Géométrie affine euclidienne}
\sousChapitre{Géométrie affine euclidienne de l'espace}
\module{Géométrie}
\niveau{L2}
\difficulte{}

\contenu{
\texte{
L'espace est rapport\'e \`a un rep\`ere orthonorm\'e direct 
$(0,\vec{\strut\imath},\vec{\strut\jmath},\vec{\strut k})$. On d\'efinit 
les points
$$A :\, (1,2,3) \; ; \quad B :\, (2,3,1) \; ; \quad C :\, (3,1,2) \; ; 
\quad D :\; (1,1,1)$$
et le plan
$$\Pi : 2x-3y+4z=0.$$
}
\begin{enumerate}
    \item \question{Montrer que les points $A,B,C$ ne sont pas align\'es.}
    \item \question{Montrer que les points $A,B,C,D$ ne sont pas coplanaires.}
    \item \question{Donner une \'equation cart\'esienne du plan $P$ passant par $A,B,C$.}
    \item \question{Calculer la distance de $D$ au plan $P$.}
    \item \question{Donner une repr\'esentation param\'etrique de la droite $d=P\cap\Pi$.}
\end{enumerate}
}

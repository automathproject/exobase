\uuid{MZby}
\exo7id{7517}
\auteur{mourougane}
\organisation{exo7}
\datecreate{2021-08-10}
\isIndication{false}
\isCorrection{true}
\chapitre{Géométrie affine euclidienne}
\sousChapitre{Géométrie affine euclidienne de l'espace}

\contenu{
\texte{
Soit $(\R_{ev}^3, standard)$ l'espace euclidien standard muni de la base canonique.

Déterminer une base orthonormée du sous-espace $vect(e_1, e_1+e_2+e_3)$.
La compléter en une base orthonormée de $\R^3$.
}
\reponse{
Une base orthonormée du sous-espace $vect(e_1, e_1+e_2+e_3)$ est 
$f_1=e_1,f_2=\frac{e_2+e_3}{\sqrt{2}}$.
Pour la compléter en une base orthonormée de $\R^3$, il suffit de considérer $f_3=f_1\wedge f_2=\frac{e_3-e_2}{\sqrt{2}}$.
}
}

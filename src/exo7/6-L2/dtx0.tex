\uuid{dtx0}
\exo7id{7056}
\auteur{megy}
\organisation{exo7}
\datecreate{2017-01-08}
\isIndication{true}
\isCorrection{false}
\chapitre{Géométrie affine euclidienne}
\sousChapitre{Géométrie affine euclidienne du plan}

\contenu{
\texte{

}
\begin{enumerate}
    \item \question{On donne une droite $\mathcal D$, un point $H$ sur $\mathcal D$ et un point $A$ en-dehors. Tracer le cercle passant par $A$ et tangent à la droite en $H$.% cas particulier de PPP ou PPD}
    \item \question{On donne trois droites dont deux parallèles. Dénombrer et construire les cercles tangents aux trois droites.}
\indication{\begin{enumerate}
\item Test de méthodologie : quelles droites peut-on tracer à partir de ce qui est donné ?
\item Construire la droite équidistante (à distance $r$) des deux parallèles, puis les deux droites parallèles à la troisième et à distance $r$.
\end{enumerate}}
\end{enumerate}
}

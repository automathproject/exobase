\uuid{171N}
\exo7id{7083}
\titre{exo7 7083}
\auteur{megy}
\organisation{exo7}
\datecreate{2017-01-21}
\isIndication{true}
\isCorrection{true}
\chapitre{Géométrie affine euclidienne}
\sousChapitre{Géométrie affine euclidienne du plan}
\module{Géométrie}
\niveau{L2}
\difficulte{}

\contenu{
\texte{
%tags : homothéties
% source : http://debart.pagesperso-orange.fr/geoplan/construc_probleme_classique.html#ch13
Soient $\mathcal C$ et $\mathcal C'$ deux cercles de rayons distincts. Montrer qu'il existe des homothéties transformant l'un en l'autre. Suivant la position et la taille des cercles, combien y a-t-il de telles homothéties ? Tracer leurs centres.
}
\indication{Si $\phi$ est une homothétie envoyant $\mathcal C$ sur $\mathcal C'$, alors $O'=\phi(O)$. Il suffit d'avoir un deuxième couple $(M, \phi(M))$ pour pouvoir tracer le centre de l'homothétie $\phi$.}
\reponse{
Tracer des rayons de $\mathcal C$ et $\mathcal C'$ parallèles entre eux.
}
}

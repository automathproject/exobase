\uuid{BdaZ}
\exo7id{4894}
\titre{exo7 4894}
\auteur{quercia}
\organisation{exo7}
\datecreate{2010-03-17}
\isIndication{false}
\isCorrection{true}
\chapitre{Géométrie affine dans le plan et dans l'espace}
\sousChapitre{Propriétés des triangles}
\module{Géométrie}
\niveau{L2}
\difficulte{}

\contenu{
\texte{
Dans le plan, on considère :

--  trois points non alignés $A,B,C$.\par
--  trois points alignés $P,Q,R$ avec $P \in (AB)$, $Q \in (AC)$, $R \in (BC)$.

On construit les points $I,J,K$ de sorte que $BPIR$, $APJQ$, $CQKR$ soient
des parallélogrammes.

Montrer que $I,J,K$ sont alignés.
}
\reponse{
Repère $(A,\vec{AB},\vec{AC})$.
}
}

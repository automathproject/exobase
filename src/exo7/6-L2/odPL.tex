\uuid{odPL}
\exo7id{5544}
\titre{exo7 5544}
\auteur{rouget}
\organisation{exo7}
\datecreate{2010-07-15}
\isIndication{false}
\isCorrection{true}
\chapitre{Conique}
\sousChapitre{Conique}
\module{Géométrie}
\niveau{L2}
\difficulte{}

\contenu{
\texte{
Déterminer l'image du cercle trigonométrique par la fonction
$\begin{array}[t]{cccc}f~:&\Cc&\rightarrow&\Cc\\
 &z&\mapsto&\frac{1}{1+z+z^2}
\end{array}$.
}
\reponse{
Un point du plan est sur le cercle de centre $O$ et de rayon $1$ si et seulement si son affixe $z$
est de module $1$ ou encore si et seulement si il existe un réel $\theta$ tel que $z=e^{i\theta}$. Or, pour $\theta$
réel,

$$f(e^{i\theta})=\frac{1}{1+e^{i\theta}+e^{2i\theta}}=\frac{e^{-i\theta}}{e^{i\theta}+1+e^{-i\theta}}
=\overline{\left(\frac{e^{i\theta}}{1+2\cos\theta}\right)}.$$
L'ensemble cherché est donc la symétrique par rapport à $(Ox)$ de la courbe d'équation polaire
$r=\frac{1}{1+2\cos\theta}$. Cette dernière est une ellipse, symétrique par rapport à $(0x)$. Donc l'ensemble cherché
est l'ellipse d'équation polaire $r=\frac{1}{1+2\cos\theta}$ (voir l'exercice \ref{exo:routhe4}, 1)).
}
}

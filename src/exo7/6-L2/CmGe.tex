\uuid{CmGe}
\exo7id{7071}
\auteur{megy}
\organisation{exo7}
\datecreate{2017-01-11}
\isIndication{true}
\isCorrection{true}
\chapitre{Géométrie affine dans le plan et dans l'espace}
\sousChapitre{Propriétés des triangles}

\contenu{
\texte{
% triangle isocèle, cercles
Soit $A$  un point quelconque du diamètre d'un cercle $\mathcal C$ et $B$ l'extrémité d'un rayon perpendiculaire à ce diamètre. On mène une droite $(BA)$ qui coupe le cercle en $P$, puis la tangente au point $P$ qui coupe en $C$ le diamètre prolongé. Démontrer que $CA = CP$.
}
\indication{Utiliser la caractérisation des triangles isocèles à l'aide d'angles.
% Utiliser des angles opposés par le sommet.}
\reponse{
%%%%%%%%%%%%
%% FIGURE %%
%%%%%%%%%%%%

Pour montrer $CA=CP$, on va montrer que le triangle $CAP$ est donc isocèle en $C$.

On a les égalités d'angles:
\begin{align*}
\widehat{CAP} &= \widehat{OAB} \text{ car les angles sont opposés par le sommet}\\
&= \pi/2 - \widehat{ABO} \\
&= \pi/2 -\widehat{OPB}  \text{ car $POB$ est isocèle en $O$}\\
&= \widehat{APC}
\end{align*}

Le triangle $CAP$ est donc isocèle en $C$, et donc $CA=CP$.
}
}

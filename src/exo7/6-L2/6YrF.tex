\uuid{6YrF}
\exo7id{4989}
\titre{exo7 4989}
\auteur{quercia}
\organisation{exo7}
\datecreate{2010-03-17}
\isIndication{false}
\isCorrection{false}
\chapitre{Courbes planes}
\sousChapitre{Courbes paramétrées}
\module{Géométrie}
\niveau{L2}
\difficulte{}

\contenu{
\texte{
Soit $\Gamma$ un cercle de centre $O$ et de rayon 1, $A \in \Gamma$, et $D$ le
diamètre de $\Gamma$ perpendiculaire à $(OA)$.

Pour $M \in \Gamma \setminus\{A\}$, on construit le point $N$ intersection de
$D$ et $(AM)$, puis le point $P$ tel que $\vec {AP} = \vec {MN}$.
}
}

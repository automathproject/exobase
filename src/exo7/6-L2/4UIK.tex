\uuid{4UIK}
\exo7id{4976}
\titre{exo7 4976}
\auteur{quercia}
\organisation{exo7}
\datecreate{2010-03-17}
\isIndication{false}
\isCorrection{true}
\chapitre{Géométrie affine euclidienne}
\sousChapitre{Géométrie affine euclidienne de l'espace}
\module{Géométrie}
\niveau{L2}
\difficulte{}

\contenu{
\texte{
Soient $f,g$ deux vissages d'angles $\ne \pi$. Trouver une CNS pour que
$f\circ g = g \circ f$.
(On étudiera $f \circ g \circ f^{-1}$)
}
\reponse{
Soient $D,\vec u, \alpha$ l'axe, le vecteur, et l'angle de $g$.
Alors $f \circ g \circ f^{-1}$ est le vissage d'axe $f(D)$, de vecteur
$\vec f(\vec u)$, et d'angle $\alpha$.

On veut que ce soit $g$, donc $D$ est invariant par $f$, ce qui implique
que $D$ soit l'axe de $f$.

Réciproquement, si $f$ et $g$ ont même axe, alors ils commutent.
}
}

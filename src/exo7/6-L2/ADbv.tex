\uuid{ADbv}
\exo7id{6884}
\titre{exo7 6884}
\auteur{rouget}
\organisation{exo7}
\datecreate{2012-09-05}
\video{CGIhcrD5Jtg}
\isIndication{false}
\isCorrection{true}
\chapitre{Géométrie affine dans le plan et dans l'espace}
\sousChapitre{Géométrie affine dans le plan et dans l'espace}
\module{Géométrie}
\niveau{L2}
\difficulte{}

\contenu{
\texte{
% Exos de J.-L. Rouget #5197 légèrement modifié
Déterminer le projeté orthogonal du point $M_0(x_0,y_0)$ sur la droite $(D)$ 
d'équation $2x-3y=5$ ainsi que son symétrique orthogonal.
}
\reponse{
$(D)$ est une droite de vecteur normal $\vec n = (2,-3)$. Le projeté orthogonal 
$p(M_0)$ de $M_0$ sur $(D)$ est de la forme $M_0+\lambda.\vec{n}$ 
où $\lambda$ est un réel à déterminer.
Le point $M_0+\lambda.\vec{n}$ a pour coordonnées $(x_0+2\lambda,y_0-3\lambda)$.

$$M_0+\lambda.\vec{n}\in(D)\iff 2(x_0+2\lambda)-3(y_0-3\lambda)=5
\iff\lambda=\frac{-2x_0+3y_0+5}{13}.$$

$p(M_0)$ a pour coordonnées $\big(x_0+2\frac{-2x_0+3y_0+5}{13},y_0-3\frac{-2x_0+3y_0+5}{13}\big)$ ou encore
$p(M_0)= \big(\frac{9x_0+6y_0+10}{13},\frac{6x_0+4y_0-15}{13}\big)$.

Le symétrique orthogonal $s(M_0)$ vérifie~:~$s(M_0)=M_0+2\overrightarrow{M_0p(M_0)}$
 (car $p(M_0)$ est le milieu du segment $[M_0,s(M_0)]$)
autrement dit $s(M_0)=M_0+2\lambda . \vec n$ (pour le $\lambda$ obtenu ci-dessus).

Ses coordonnées sont donc 
$s(M_0) = \big(x_0+4\frac{-2x_0+3y_0+5}{13},y_0-6\frac{-2x_0+3y_0+5}{13}\big)$ ou encore
$\big(\frac{5x_0+12y_0+20}{13},\frac{12x_0-5y_0-30}{13}\big)$.
}
}

\uuid{kfej}
\exo7id{6874}
\titre{exo7 6874}
\auteur{gammella}
\organisation{exo7}
\datecreate{2012-05-29}
\isIndication{false}
\isCorrection{true}
\chapitre{Analyse vectorielle}
\sousChapitre{Forme différentielle, champ de vecteurs, circulation}
\module{Géométrie}
\niveau{L2}
\difficulte{}

\contenu{
\texte{
On considère le changement de variables en coordonnées sphériques suivant :
$$  \left\{ \begin{array}{lll}
x& = & r\cos \varphi\cos \theta\\
y&= & r\cos \varphi\sin \theta \\
z & = & r \sin \varphi \\
\end{array} \right .$$
}
\begin{enumerate}
    \item \question{Calculer $dx$, $dy$, $dz$.}
    \item \question{Vérifier que $x dx+ydy+zdz=rdr.$ En déduire 
$ \frac{\partial r}{\partial x}$, $ \frac{\partial r}{\partial y}$ 
et $ \frac{\partial r}{\partial z}$.}
\reponse{
On vérifie que :
\begin{enumerate}
$dx= \cos \varphi \cos \theta dr-r\sin \varphi\cos \theta d\varphi-r\sin \theta\cos \varphi 
d\theta$
$dy = \cos \varphi \sin \theta dr-r\sin \varphi\sin \theta d\varphi+r\cos \theta\cos\varphi d\theta $
$dz=\sin \varphi dr+r\cos \varphi d\varphi.$
}
\end{enumerate}
}

\uuid{O11B}
\exo7id{7065}
\titre{exo7 7065}
\auteur{megy}
\organisation{exo7}
\datecreate{2017-01-11}
\isIndication{true}
\isCorrection{false}
\chapitre{Géométrie affine dans le plan et dans l'espace}
\sousChapitre{Propriétés des triangles}
\module{Géométrie}
\niveau{L2}
\difficulte{}

\contenu{
\texte{
%  aire, hauteur,
Un quadrilatère est dit \emph{orthodiagonal} si ses diagonales sont perpendiculaires.

Soit $ABCD$ un quadrilatère orthodiagonal non croisé. Montrer que son aire vaut $\frac12 AC\cdot BD$.
}
\indication{Décomposer l'aire comme la somme des aires de deux triangles.}
}

\uuid{trSk}
\exo7id{7060}
\titre{exo7 7060}
\auteur{megy}
\organisation{exo7}
\datecreate{2017-01-11}
\isIndication{true}
\isCorrection{true}
\chapitre{Géométrie affine dans le plan et dans l'espace}
\sousChapitre{Propriétés des triangles}
\module{Géométrie}
\niveau{L2}
\difficulte{}

\contenu{
\texte{
% collège
% orthocentre, triangle rectangle inscrit 
% cercle circonscrit, hauteurs

Soit $[AB]$ un segment et $M$, $N$ deux points appartenant au cercle  $\mathcal C$  de diamètre $[AB]$. On suppose que les droites $(MB)$ et $(AN)$ (respectivement $(NB)$ et $(AM)$ ) s'intersectent en $P$ (respectivement en $Q$). Déterminer l'angle formé par  les droites $(AB)$ et $(PQ)$.
}
\indication{Déterminer les angles $\widehat{AMB}$ et $\widehat{ANB}$.}
\reponse{
Le point $A$ est l'orthocentre de $PQB$. L'angle est donc droit.
}
}

\uuid{ZflC}
\exo7id{5512}
\titre{exo7 5512}
\auteur{rouget}
\organisation{exo7}
\datecreate{2010-07-15}
\isIndication{false}
\isCorrection{true}
\chapitre{Géométrie affine dans le plan et dans l'espace}
\sousChapitre{Géométrie affine dans le plan et dans l'espace}
\module{Géométrie}
\niveau{L2}
\difficulte{}

\contenu{
\texte{
Trouver toutes les droites sécantes aux quatre droites $(D_1)$~$x-1=y=0$, $(D_2)$~:~$y-1=z=0$, $(D_3)$~:~$z-1=x=0$ et $(D_4)$~:~$x=y=-6z$.
}
\reponse{
Notons $(\Delta)$ une éventuelle droite solution.
\textbullet~$(\Delta)$ est sécante à $(D_1)$ et $(D_2)$ si et seulement si $(\Delta)$ passe par un point de la forme $(1,0,a)$ et par un point de la forme $(b,1,0)$ ou encore si et seulement si $(\Delta)$ passe par un point de la forme $(1,0,a)$ et est dirigée par un vecteur de la forme $(b-1,1,-a)$.
Ainsi, $(\Delta)$ est sécante à $(D_1)$ et $(D_2)$ si et seulement si $(\Delta)$ admet un système d'équations paramétriques de la forme $\left\{
\begin{array}{l}
x=1+\lambda(b-1)\\
y=\lambda\\
z=a-\lambda a
\end{array}
\right.$ ou encore un système d'équations cartésiennes de la forme $\left\{
\begin{array}{l}
x-(b-1)y=1\\
ay+z=a
\end{array}
\right.$.

\textbullet~Ensuite, $(\Delta)$ et $(D_3)$ sécantes $\Leftrightarrow\exists y\in\Rr/\;\left\{
\begin{array}{l}
-(b-1)y=1\\
ay+1=a
\end{array}
\right.\Leftrightarrow b\neq 1\;\text{et}\;-\frac{a}{b-1}+1=a\Leftrightarrow b\neq0\;\text{et}\;b\neq1\;\text{et}\;a=1-\frac{1}{b}$.
En résumé, les droites sécantes à $(D_1)$, $(D_2)$ et $(D_3)$ sont les droites dont un système d'équations cartésiennes est 

\begin{center}
$\left\{
\begin{array}{l}
x-(b-1)y=1\\
\left(1-\frac{1}{b}\right)y+z=1-\frac{1}{b}
\end{array}
\right.$, $b\notin\{0,1\}$.
\end{center}
Enfin,

\begin{align*}\ensuremath
(\Delta)\;\text{et}\;(D)\;\text{sécantes}&\Leftrightarrow\exists(x,y,z)\in\Rr^3/\;\left\{
\begin{array}{l}
x-(b-1)y=1\\
\left(1-\frac{1}{b}\right)y+z=1-\frac{1}{b}\\
x=y=-6z
\end{array}
\right.\\
&\Leftrightarrow\exists(x,y,z)\in\Rr^3/\;\left\{
\begin{array}{l}
-6z+6(b-1)z=1\\
-6\left(1-\frac{1}{b}\right)z+z=1-\frac{1}{b}\\
x=y=-6z
\end{array}
\right.\\
 &\Leftrightarrow b\notin\{0,1,2\}\;\text{et}\;-6\left(1-\frac{1}{b}\right)\frac{1}{6(b-2)}+\frac{1}{6(b-2)}=1-\frac{1}{b}\\
 &\Leftrightarrow b\notin\{0,1,2\}\;\text{et}\;-6(b-1)+b=6(b-1)(b-2)\Leftrightarrow b\notin\{0,1,2\}\;\text{et}\;6b^2-13b+6=0\\
 &\Leftrightarrow b\in\left\{\frac{2}{3},\frac{3}{2}\right\}.
\end{align*}

\begin{center}
\shadowbox{
Les droites solutions sont $(\Delta_1)$ : $\left\{
\begin{array}{l}
3x+y=3\\
y-2z=1
\end{array}
\right.$ et $(\Delta_2)$ : $\left\{
\begin{array}{l}
2x-y=2\\
y+3z=1
\end{array}
\right.$.
}
\end{center}
}
}

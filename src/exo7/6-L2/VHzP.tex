\uuid{VHzP}
\exo7id{7250}
\auteur{mourougane}
\organisation{exo7}
\datecreate{2021-08-10}
\isIndication{false}
\isCorrection{false}
\chapitre{Géométrie affine euclidienne}
\sousChapitre{Géométrie affine euclidienne du plan}

\contenu{
\texte{
% cf. 978-0-387-69216-6.pdf page 529

Un \emph{diagramme de Voronoï} est une famille de parties du plan 
(ou de l'espace, mais dans cet exercice on se limitera au plan) et 
de points associés telle que:
\begin{itemize}
\item chaque partie du plan a un unique point associé, qui est 
contenu dedans;
\item chaque partie est exactement égale à l'ensemble des points 
du plan qui sont plus proches du point associé à cette partie que 
des points associés aux autres parties.
\end{itemize}
Autrement dit, c'est une famille \((A_i, P_i)_{i \in I}\), où:
\begin{itemize}
\item \(I\) est un ensemble;
\item pour tout \(i \in I\), \(A_i\) est une partie (i.e. un 
sous-ensemble) du plan et \(P_i \in A_i\);
\item pour tout \(i \in I\), on~a (en notant \( \mathcal{P}\) le plan):
\[
A_i = \left\{ \; Q \in \mathcal{P} \; \middle/ \; \forall j \in I \setminus \{i\} \; P_iQ \leq P_jQ \; \right\}.
\]
\end{itemize}
Les parties \(A_i\) sont appelées les \emph{cellules} du diagramme 
de Voronoï. Le point \(P_i\) associé à la cellule \(A_i\) est appelé 
le \emph{germe} de la cellule.

Les diagrammes de Voronoï sont un outil utile pour représenter les 
zones de couverture d'antennes radio, ou pour étudier l'implantation 
d'écoles, d'hôpitaux, de bureaux de poste, etc, dans une région.
}
\begin{enumerate}
    \item \question{Soient \(A\) et \(B\) deux points du plan. Montrer que 
l'ensemble des points équidistants de \(A\) et \(B\) (autrement 
dit, l'ensemble des points \(P\) du plan tels que \(PA = PB\)) 
est une droite (qu'on notera \( \Delta\), et qui est appelée la 
\emph{médiatrice} du segment \([AB]\)).}
    \item \question{Montrer que la droite \( \Delta\) est orthogonale à la 
droite \((AB)\).}
    \item \question{Montrer que l'ensemble des points \(P\) du plan tels 
que \(PA \leq PB\) est le demi-plan de frontière \( \Delta\) contenant \(A\).}
    \item \question{Quel est le diagramme de Voronoï d'un ensemble de deux 
points distincts?}
\end{enumerate}
}

\uuid{oS9Z}
\exo7id{7510}
\auteur{mourougane}
\organisation{exo7}
\datecreate{2021-08-10}
\isIndication{false}
\isCorrection{false}
\chapitre{Géométrie affine euclidienne}
\sousChapitre{Géométrie affine euclidienne de l'espace}

\contenu{
\texte{
Dans l'espace affine $\mathcal{E}$ muni d'un repère orthonormé 
$(O,\vec{\imath},\vec{\jmath},\vec{k})$, on considère le cône $\mathcal{C}$ d'équation
$y^2+z^2=3(x-2)^2$.
}
\begin{enumerate}
    \item \question{Déterminer un plan dont l'intersection avec le cône $\mathcal{C}$ soit un cercle.}
    \item \question{Déterminer la nature de l'intersection de $\mathcal{C}$ avec le plan d'équation $z=1$.
On précisera (s'ils existent) le centre, les axes de symétrie et les asymptotes .}
\end{enumerate}
}

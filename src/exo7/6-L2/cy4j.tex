\uuid{cy4j}
\exo7id{5203}
\auteur{rouget}
\organisation{exo7}
\datecreate{2010-06-30}
\isIndication{false}
\isCorrection{true}
\chapitre{Géométrie affine dans le plan et dans l'espace}
\sousChapitre{Géométrie affine dans le plan et dans l'espace}

\contenu{
\texte{
Soit $\mathcal{C}$ la courbe d'équation $x^2+y^2-2x+4y+1=0$.
}
\begin{enumerate}
    \item \question{Déterminer une équation de la tangente au point de $\mathcal{C}$ de coordonnées $(2,-2+\sqrt{3})$.}
\reponse{Le point $A(2,-2+\sqrt{3})$ est effectivement sur $\mathcal{C}$ car $(2-1)^2+(-2+\sqrt{3}+2)^2=1+3=4$. La tangente $(T)$ en $A$ à $\mathcal{C}$ est la droite passant par $A$ et de vecteur normal $\overrightarrow{A\Omega}$.

$$M(x,y)\in(T)\Leftrightarrow\overrightarrow{AM}.\overrightarrow{A\Omega}=0\Leftrightarrow(x-2)+\sqrt{3}(y+2-\sqrt{3})=0
\Leftrightarrow x+\sqrt{3}y-5+2\sqrt{3}=0.$$}
    \item \question{Déterminer l'intersection de $\mathcal{C}$ et du cercle de centre $(1,0)$ et de rayon $2$.}
\reponse{Soit $\mathcal{C}'$ le cercle de centre $(1,0)$ et de rayon $2$. Une équation de ce cercle est $x^2+y^2-2x-3=0$. Par suite,

\begin{align*}
M(x,y)\in\mathcal{C}\cap\mathcal{C}'&\Leftrightarrow
\left\{
\begin{array}{l}
x^2+y^2-2x+4y+1=0\\
x^2+y^2-2x-3=0
\end{array}
\right.\Leftrightarrow
\left\{
\begin{array}{l}
4y+4=0\quad((1)-(2))\\
x^2+y^2-2x-3=0
\end{array}
\right.
\Leftrightarrow
\left\{
\begin{array}{l}
y=-1\\
x^2-2x-2=0
\end{array}
\right.
\\
 &\Leftrightarrow\left\{
\begin{array}{l}
y=-1\\
x=1+\sqrt{3}\;\mbox{ou}\;x=1-\sqrt{3}
\end{array}
\right.
\end{align*}

Il ya donc deux points d'intersection~:~$(1+\sqrt{3},-1)$ et $(1-\sqrt{3},-1)$.}
\end{enumerate}
}

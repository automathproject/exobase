\uuid{Qe5y}
\exo7id{5515}
\auteur{rouget}
\organisation{exo7}
\datecreate{2010-07-15}
\isIndication{false}
\isCorrection{true}
\chapitre{Géométrie affine dans le plan et dans l'espace}
\sousChapitre{Géométrie affine dans le plan et dans l'espace}

\contenu{
\texte{
Dans $\Rr^3$ rapporté à un repère orthonormé, soient $(D)$ $\left\{
\begin{array}{l}
x+y+z+1=0\\
2x+y+5z=2
\end{array}
\right.$ et $(D')$ $\left\{
\begin{array}{l}
x+y+z=2\\
2x+y-5z=3
\end{array}
\right.$. Déterminer la distance de $(D)$ à $(D')$ puis la perpendiculaire commune à ces deux droites.
}
\reponse{
\textbullet~Déterminons un repère de $(D)$.

\begin{center}
$\left\{
\begin{array}{l}
x+y+z+1=0\\
2x+y+5z=2
\end{array}
\right.\Leftrightarrow\left\{
\begin{array}{l}
x+y=-z-1\\
2x+y=-5z+2
\end{array}
\right.\Leftrightarrow\left\{
\begin{array}{l}
x=-4z+3\\
y=3z-4
\end{array}
\right.$
\end{center}
Un repère de $(D)$ est $\left(A,\overrightarrow{u}\right)$ où $A(3,-4,0)$ et $\overrightarrow{u}(-4,3,1)$.
\textbullet~Déterminons un repère de $(D')$.

\begin{center}
$\left\{
\begin{array}{l}
x+y+z=2\\
2x+y-5z=3
\end{array}
\right.\Leftrightarrow\left\{
\begin{array}{l}
x+y=-z+2\\
2x+y=5z+3
\end{array}
\right.\Leftrightarrow\left\{
\begin{array}{l}
x=6z+1\\
y=-7z+1
\end{array}
\right.$
\end{center}
Un repère de $(D')$ est $\left(A',\overrightarrow{u'}\right)$ où $A'(1,1,0)$ et $\overrightarrow{u'}(6,-7,1)$.
\textbullet~$\overrightarrow{u}\wedge\overrightarrow{u'}=\left(
\begin{array}{c}
-4\\
3\\
1
\end{array}
\right)\wedge\left(
\begin{array}{c}
6\\
-7\\
1
\end{array}
\right)=\left(
\begin{array}{c}
10\\
10\\
10
\end{array}
\right)\neq\overrightarrow{0}$.
\rule{0mm}{5mm}Puisque $\overrightarrow{u}$ et $\overrightarrow{u'}$ ne sont pas colinéaires, les droites $(D)$ et $(D')$ ne sont parallèles. Ceci assure l'unicité de la perpendiculaire commune  à $(D)$ et $(D')$.
\textbullet~On sait que la distance $d$ de $(D)$ à $(D')$ est donnée par

\begin{center}
$d=\frac{\text{abs}\left(\left[\overrightarrow{AA'},\overrightarrow{u},\overrightarrow{u'}\right]\right)}{\|\overrightarrow{u}\wedge\overrightarrow{u'}\|}$,
\end{center}
avec $[\overrightarrow{AA'},\overrightarrow{u},\overrightarrow{u'}]=\left|
\begin{array}{ccc}
-2&-4&6\\
5&3&-7\\
0&1&1
\end{array}
\right|=10\times(-2)+10\times5=30$ et donc $d=\frac{30}{10\sqrt{3}}=\sqrt{3}$.

\begin{center}
\shadowbox{
$d((D),(D'))=\sqrt{3}$.
}
\end{center}

\textbullet~Un système d'équations de la perpendiculaire commune est 
$\left\{
\begin{array}{l}
\left[\overrightarrow{AM},\overrightarrow{u},\overrightarrow{u}\wedge\overrightarrow{u'}\right]=0\\
\rule{0mm}{6mm}\left[\overrightarrow{A'M},\overrightarrow{u'},\overrightarrow{u}\wedge\overrightarrow{u'}\right]=0
\end{array}
\right.$. Or,

\begin{center}
$\frac{1}{10}\left[\overrightarrow{AM},\overrightarrow{u},\overrightarrow{u}\wedge\overrightarrow{u'}\right]=\left|
\begin{array}{ccc}
x-3&-4&1\\
y+4&3&1\\
z&1&1
\end{array}
\right|=2(x-3)+5(y+4)-7z=2x+5y-7z+14$,
\end{center}
et

\begin{center}
$\frac{1}{10}\left[\overrightarrow{A'M},\overrightarrow{u'},\overrightarrow{u}\wedge\overrightarrow{u'}\right]=\left|
\begin{array}{ccc}
x-1&6&1\\
y-1&-7&1\\
z&1&1
\end{array}
\right|=-8(x-1)-5(y-1)+13z=-8x-5y+13z+13$.
\end{center}
Donc

\begin{center}
\shadowbox{
\begin{tabular}{c}
un système d'équations cartésienne de la perpendiculaire commune à $(D)$ et $(D')$ est\\
$\left\{
\begin{array}{l}
2x+5y-7z=-14\\
8x+5y-13z=13
\end{array}
\right.$.
\end{tabular}
}
\end{center}
}
}

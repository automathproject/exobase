\uuid{TySC}
\exo7id{4863}
\titre{exo7 4863}
\auteur{quercia}
\organisation{exo7}
\datecreate{2010-03-17}
\isIndication{false}
\isCorrection{true}
\chapitre{Géométrie affine dans le plan et dans l'espace}
\sousChapitre{Sous-espaces affines}
\module{Géométrie}
\niveau{L2}
\difficulte{}

\contenu{
\texte{
Soit $\cal E$ un espace affine de dimension 3, et $D,D',D''$ trois droites
parallèles à un même plan $\cal P$, mais deux à deux non coplanaires.
}
\begin{enumerate}
    \item \question{Montrer que par tout point $A$ de $D$, il passe une unique droite $\Delta_A$
    rencontrant $D'$ et $D''$.}
\reponse{Le plan passant par $A$ et $D'$ n'est pas parallèle à $D''$.}
    \item \question{Montrer que les droites $\Delta_A$ sont toutes parallèles à un même plan
    $\cal Q$.}
\reponse{On doit pouvoir s'en sortir avec un repère adéquat \dots}
\end{enumerate}
}

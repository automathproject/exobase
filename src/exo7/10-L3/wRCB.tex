\uuid{wRCB}
\exo7id{2254}
\titre{exo7 2254}
\auteur{barraud}
\organisation{exo7}
\datecreate{2008-04-24}
\isIndication{false}
\isCorrection{true}
\chapitre{Anneau, corps}
\sousChapitre{Anneau, corps}
\module{Algèbre et théorie des nombres}
\niveau{L3}
\difficulte{}

\contenu{
\texte{
\label{exodessus}
Lesquels de ces sous-ensembles donn\'es de $\Cc$
sont des anneaux ? Lesquels sont des corps ?
}
\begin{enumerate}
    \item \question{$\bigcup\limits_{n\in\Nn}10^{-n}\Zz$ ;}
    \item \question{$\{\frac{m}{n}\mid m\in\Zz,n\in\Nn^*, (m,n)=1, p\nmid n\}$ ($p$ est
un nombre premier fix\'e) ;}
    \item \question{$\Zz[\sqrt{-1}]=\Zz +\Zz\sqrt{-1}$, 
$\ \ \Zz[\sqrt{2}]=\Zz +\Zz\sqrt{2}$;}
    \item \question{$\Qq[\sqrt{-1}]=\Qq +\Qq\sqrt{-1}$, 
$\ \ \Qq[\sqrt{2}]=\Qq +\Qq\sqrt{2}$.}
\reponse{
$A$ est l'ensemble des nombres dont le développement décimal s'arrête
    (``nombre fini de chiffres après la virgule'').

    Stabilité par addition~: Soit $x=10^{-n}a$ et $y=10^{-m}b$. Supposons
    par exemple que $n\geq m$. Alors $x+y=10^{-n}(a+10^{n-m}b)$ et
    $a+10^{n-m}b\in\Zz$ donc $x+y\in A$. Les autres vérifications sont
    analogues.

    Ce n'est pas un corps~: $3$ n'est pas inversible, puisque si
    $3\cdot10^{-n}a=1$, alors $3a=10^{n}$ donc $3|10^{n}$ ce qui est
    impossible. Un élément est inversible ssi il est de la forme
    $10^{-n}2^{\alpha}5^{\beta}$, $\alpha,\beta\in\Nn$.
Stabilité par addition~: Soit $x=\frac{a}{b}\in A$ et
    $y=\frac{c}{d}\in A$, avec $\pgcd(a,b)=\pgcd(c,d)=\pgcd(p,b)=\pgcd(p,d)=1$. Alors $x+y=\frac{ad+bc}{bd}$. 

    Ce n'est pas un corps~: $p$ n'est pas inversible. Un élément est
    inversible ssi ce n'est pas un multiple de $p$.
N'est pas un corps~: $2$ n'est pas inversible. Les seuls éléments
    inversibles sont $1,-1,i,-i$. En effet, si $z\in A^{\times}$, alors
    $|z|\geq 1$ et $|z^{-1}|\geq 1$. Donc $|z|=1$ et $z\in\{\pm1,\pm
    i\}$. Réciproquement, ces éléments sont bien tous inversibles.
}
\end{enumerate}
}

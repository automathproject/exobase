\uuid{KMzu}
\exo7id{2269}
\titre{exo7 2269}
\auteur{barraud}
\organisation{exo7}
\datecreate{2008-04-24}
\isIndication{false}
\isCorrection{true}
\chapitre{Polynôme}
\sousChapitre{Polynôme}
\module{Algèbre et théorie des nombres}
\niveau{L3}
\difficulte{}

\contenu{
\texte{
\label{ex:bar9}
Soient $f,g\in \Qq[x]$. Supposons que
$f$ soit  irr\'eductible et qu'il existe $\alpha \in \Cc$ tel que
$f(\alpha)=g(\alpha)=0$. Alors $f$ divise $g$.
}
\reponse{
$f$ est irréductible, donc si $f$, ne divise pas $g$, alors $f$ et $g$
 sont premiers entre eux. Ainsi,$\exists u,v\in\Qq[X], uf+vg=1$. En
 évaluant en $\alpha$, on obtient $u(\alpha)\cdot0+v(\alpha)\cdot0=1$ ce
 qui est impossible!
}
}

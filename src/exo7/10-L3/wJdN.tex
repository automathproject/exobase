\uuid{wJdN}
\exo7id{2294}
\titre{exo7 2294}
\auteur{barraud}
\organisation{exo7}
\datecreate{2008-04-24}
\isIndication{false}
\isCorrection{true}
\chapitre{Polynôme}
\sousChapitre{Polynôme}
\module{Algèbre et théorie des nombres}
\niveau{L3}
\difficulte{}

\contenu{
\texte{
Les polynômes suivants sont-ils irréductibles ?
}
\begin{enumerate}
    \item \question{$X^5+121X^4+1221X^3+12221X^2+122221X+222222$ dans $\Qq [X]$.}
\reponse{Ce polynôme est unitaire donc primitif. $11$ est nombre premier qui
    divise tous les coefficients sauf le dominant. $11^{2}=121$ ne divise
    pas le coefficient de degré $0$, donc, d'après le critère
    d'Eisenstein, c'est un polynôme irréductible de $\Qq[X]$.}
    \item \question{$f(X,Y)=X^2Y^3+X^2Y^2+Y^3-2XY^2+Y^2+X-1$ dans $\Cc [X,Y]$ et $\mathbb{F}_2[X,Y]$.}
\reponse{$f(X,Y)=(X^2+1)Y^3+(X-1)^{2}Y^2+(X-1)$. Regardons $f$ comme un
    polynôme de $A[Y]$ avec $A=\Cc[X]$. Alors, $f$ est primitif sur $A$, et
    $(X-1)$ est un irréductible de $A$ qui divise tous les
    coefficients de $f$ sauf le dominant, et dont le carré ne divise pas
    le terme constant. D'après le critère d'Eisenstein, on en déduit que
    $f$ est irréductible dans $A[Y]=\Cc[X,Y]$.

    Dans $\Zz_{2}[X,Y]$, on a $(X^{2}+1)=(X+1)^{2}$ et
    $f=(X+1)((X+1)(Y^3+Y^2)+1)$, donc $f$ n'est pas irréductible..}
    \item \question{$f(X,Y)=Y^7+Y^6+7Y^4+XY^3+3X^2Y^2-5Y+X^2+X+1$ dans $\Qq[X,Y]$.}
\reponse{$f(X,Y)=Y^7+Y^6+7Y^4+XY^3+3X^2Y^2-5Y+X^2+X+1$. Considérons $f$ comme
    un polynôme de $A[X]$ où $A=\Qq[Y]$. Alors $f$ est primitif sur $A$.
    Soit $\pi=Y\in A$. $\pi$ est irréductible, $\pi$ ne divise pas le
    coefficient dominant de $f$, et la réduction $\bar{f}$ modulo $\pi$
    est $\bar{f}=X^{2}+X+1\in A/(\pi)[X]=\Qq[X,Y]/(Y)\simeq\Qq[X]$.
    $\bar{f}$ est donc irréductible dans $A/(\pi)$, donc d'après
    l'exercice précédent, $f$ est irréductible dans $\Qq[X,Y]$.}
\end{enumerate}
}

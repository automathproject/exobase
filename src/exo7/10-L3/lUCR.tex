\uuid{lUCR}
\exo7id{7821}
\titre{exo7 7821}
\auteur{mourougane}
\organisation{exo7}
\datecreate{2021-08-11}
\isIndication{false}
\isCorrection{false}
\chapitre{Forme bilinéaire}
\sousChapitre{Forme bilinéaire}
\module{Algèbre et théorie des nombres}
\niveau{L3}
\difficulte{}

\contenu{
\texte{
On appelle espace d'Artin (ou espace hyperbolique) tout espace
vectoriel $E$ muni d'une forme quadratique $q$ équivalente à 
$$x=(x_i)_{1\leq i\leq 2p}\ \ \ \ \ \ \ 
 q(x)=2\sum_{i=1}^p x_ix_{i+p}.$$
}
\begin{enumerate}
    \item \question{Montrer que sur $\Cc^{2p}$, toute forme quadratique non dégénérée
 définit un espace d'Artin. Caractériser à l'aide de la signature les
 espaces d'Artin sur $\Rr^{2p}$. Les caractériser à l'aide de
 l'indice en supposant la forme non dégénérée.}
    \item \question{Montrer que tout espace d'Artin est somme directe orthogonale de
 plans d'Artin orthogonaux.}
    \item \question{Soit $(E,q)$ quelconque, $x$ un vecteur isotrope mais pas dans
 le noyau de $q$. Montrer qu'il existe un plan $P$ qui contient $x$
 et tel que $(P,q_{|P})$ soit un plan d'Artin.}
\end{enumerate}
}

\uuid{VeJ3}
\exo7id{6377}
\auteur{potyag}
\organisation{exo7}
\datecreate{2011-10-16}
\isIndication{false}
\isCorrection{false}
\chapitre{Groupe}
\sousChapitre{Groupe}

\contenu{
\texte{
\label{pot:exo9}
 Le but de cet exercice
est de donner la construction d'un groupe libre et d'introduire la notion de
 présentation d'un groupe.

Soit $S=\{s_i\}_{i\in I}$ un ensemble quelconque qu'on appellera
alphabet. Un mot dans l'alphabet $S$ est par définition une
succession finie (ou vide) :

$$w=s^{\epsilon_1}_{i_1}\cdot\cdot\cdot s^{\epsilon_k}_{i_k},\
o\grave u\ \epsilon_i=\pm 1\ et\ s_{i_j}\in S,\
k\in\N\hfill\eqno(1)$$


Notons $W$ l'ensemble de tous les mots. Un mot $w\in W$ est dit
réduit si son écriture (1)  ne contient pas deux lettres
consécutives du type $ s^{\epsilon}_{i}$ et $ s^{-\epsilon}_{i}$.
Les mots $w_1$ et $w_2$ sont dits voisins si $w_2=g s_i^{\epsilon}
s_i^{-\epsilon} h$ et $w_1=g h$. Deux mots $f$ et $g$ s'appellent
équivalents (on note $f\sim g$) s'il existe une succession finie
de mots : $f=w_0, w_1, ..., w_n=g$ où les mots $w_i$ et
$w_{i-1}$ sont voisins ($i\in\{1,...,n\}$).
}
\begin{enumerate}
    \item \question{Montrer que $\sim$ est une relation
d'équivalence.



Etant donné un mot $f=a_1 ... a_t$   (où $a_j=
s_{i_j}^{\epsilon_j}$) définissons une suite de transformations
appelée $R$-procédé :
$R_0=e$ (le mot vide), $R_1=a_1$
et 
$$R_{i+1}= 
\begin{cases}
  R_i a_{i+1} & \text{ si } R_i \text{ n'est  pas un  mot réduit du type } Xa^{-1}_{i+1} \cr
  X           & \text{ si } R_i \text{ est un mot réduit du type } Xa^{-1}_{i+1} \cr
\end{cases}.$$


Autrement dit un $R$-procédé consiste à faire
toutes les simplifications de droite à gauche.}
    \item \question{On suppose que $w_1=a_1 ... a_r a_{r+1} ... a_t$ et $w_1=a_1
... a_r s_j^{\epsilon} s_j^{-\epsilon} a_{r+1} ... a_t$ sont deux
mots et que $R^i$ désigne le $R$-procédé appliqué au mot
$w_i$. Montrer que $R^1_t=R^2_{t+2}$ c.-à.-d.
$R^1(w_1)=R^2(w_2)$.
En déduire que chaque classe de $W/\sim$
contient un mot réduit et un seul.

Pour deux classes $[w_i]\in W/\sim\ (i=1,2)$ définissons
maintenant leur produit  comme suit (de gauche à droite) :

$$[w_1] [w_2] = [w_1 w_2]\hfill\eqno (2).$$}
    \item \question{Démontrer que (2) ne dépend pas du choix des
représentants des classes $[w_i]$. Montrer que l'ensemble
$\displaystyle F=W/\sim$ muni de l'opération (2) est un
groupe.


Ce groupe s'appelle groupe libre engendré par $S$, on appelle
les éléments de $S$ géné\-ra\-teurs libres de $F$.

Soit maintenant $G$
un groupe quelconque engendré par un système $X$ où
$X=\{x_i\}_{i\in I}$ et $F$ est le groupe libre engendré par
$S$. Supposons qu'il existe une bijection $f:S\mapsto X$ telle que
$f(s_i)=x_i\ (i\in I)$.}
    \item \question{\begin{enumerate}}
    \item \question{Montrer que $f$ se prolonge en un homomorphisme $f:F\mapsto
  G$.}
    \item \question{En particulier, en déduire que si ${\text { Card } (S)}= \text { Card } (S') $, alors
   le groupe libre engendré par $S$ est isomorphe au le groupe libre engendré par
  $S'$.}
\end{enumerate}
}

\uuid{v6sX}
\exo7id{2311}
\auteur{barraud}
\organisation{exo7}
\datecreate{2008-04-24}
\isIndication{false}
\isCorrection{true}
\chapitre{Anneau, corps}
\sousChapitre{Anneau, corps}

\contenu{
\texte{
Soit $\Zz_{36}=\Zz/36\Zz\ $ l'anneau des entiers modulo $36$.
}
\begin{enumerate}
    \item \question{D\'ecrire tous les \'el\'ements inversibles, tous les diviseurs
de z\'ero et tous les \'el\'ements nilpotents de l'anneau $\Zz_{36}$.
({\it Un \'el\'ement $a$ d'un anneau $A$ est dit nilpotent
si il existe $n$ tel que $a^n=0$.})}
    \item \question{Trouver tous les id\'eaux de l'anneau $\Zz_{36}$.}
    \item \question{Soit $A$ un anneau arbitraire. Montrer que
  $$
  (a\in A^{\times} \hbox{ et } b\in A^{\times}) \Longleftrightarrow
  (a\cdot b)\in A^{\times}.
  $$}
    \item \question{Donner un exemple d'un polyn\^ome inversible de degr\'e $1$ sur
  $\Zz_{36}$.}
    \item \question{D\'ecrire tous les \'el\'ements inversibles de l'anneau
  $\Zz_{36}[x]$.}
\reponse{
$\bar{n}$ est inversible ssi $\pgcd(n,36)=1$ (Bezout~!), i.e.
$\bar{n}\in\{\pm1, \pm5, \pm7, \pm11, \pm13, \pm17\}$. Les autres
éléments sont tous des diviseurs de $0$ puisque $\bar{n}$ divise $0$ ssi
$\pgcd(n,36)\neq1$. Enfin, $\bar{n}$ est nilpotent ssi $2|n$ et $3|n$,
donc ssi $6|n$, soit $\bar{n}\in\{0,\pm6,\pm12,18\}$.
Montrons que l'ensemble $\mathcal{I}$ des idéaux de $\Zz/36\Zz$ est en
bijection avec l'ensemble $\mathcal{D}=\{1,2,3,4,6,9,12,18,36\}$  des
diviseurs (positifs) de $36$.

Considérons l'application $\phi:\mathcal{D}\to\mathcal{I}$ définie par
$\phi(d)=(\bar{d})$.

\textsl{Injectivité~:} Si $\phi(d)=\phi(d')$, alors $\exists a,b\in\Zz,
d=d'a+36b$. Comme $d|36$, on en déduit que $d|d'$. De même, on a $d'|d$,
et donc $d=d'$.

\textsl{Surjectivité~:}Soit $I\in\mathcal{I}$. $\Zz/36\Zz$ est principal,
donc $\exists a\in\Zz, I=(\bar{a})$. Soit $d=\pgcd(a,36)$. Notons
$a=da'$~: $\pgcd(a',36)=1$. On en déduit que $\bar{a}'$ est inversible
dans $\Zz/36\Zz$. Alors $\bar{d}\sim\bar{a}$ dans $\Zz/36\Zz$. On en
déduit que $I=(\bar{d})=\phi(d)$.

Finalement, il y a donc 9 idéaux dans $\Zz_{36}$~:
\begin{itemize}
$(\overline{ 1})=\Zz_{36}$,
$(\overline{ 2})=\{0,\pm2,\pm4,\pm6,\pm8,\pm10,\pm12,\pm14,\pm16,18\}$,
$(\overline{ 3})=\{0,\pm3,\pm6,\pm9,\pm12,\pm15,18\}$,
$(\overline{ 4})=\{0,\pm4,\pm8,\pm12,\pm16\}$,
$(\overline{ 6})=\{0,\pm6,\pm12\}$
$(\overline{ 9})=\{0,\pm9,18\}$
$(\overline{12})=\{0,\pm12\}$
$(\overline{18})=\{0,18\}$
$(\overline{36})=\{0\}$,
\end{itemize}
Si $a,b\in A^{\times}$, alors $(a b)(b^{-1} a^{-1})=1$ donc $ab\in
A^{\times}$.

Si $ab\in A^{\times}$, soit $c=(ab)^{-1}$. Alors $a(bc)=1$ donc $a\in
A^{\times}$ et $b(ac)=1$ donc $b\in A^{\times}$.
On a $(6x+1)(-6x+1)=1$ dans $\Zz_{36}[x]$, donc $18x+1$ y est inversible.
Soit $f$ un inversible de $\Zz_{36}[x]$. Choisissons $P\in\Zz[x]$ tel que
$\bar{P}=f$ et $Q\in\Zz[x]$ tel que $\bar{Q}=f^{-1}$.

La projection $\Zz\to\Zz_{2}$ se factorise par
$\Zz\to\Zz_{36}\to\Zz_{2}$. Ces projections sont bien définies, et sont
des morphismes d'anneaux. Notons $P_{[2]}$ la réduction de $P$ modulo
$2$~: on a alors $P_{[2]}Q_{[2]}=(PQ)_{[2]}=1$, et comme $\Zz_{2}$ est un
corps, $P_{[2]}=1$, $Q_{[2]}=1$. On en déduit que $2$ divise tous les
coefficients de $P$, sauf celui de degré $0$. De même, en considérant la
réduction modulo $3$, on obtient que $3$ divise tous les coefficients de
$P$, sauf celui de degré $0$. Finalement, $6$ divise tous les
coefficients de $P$ sauf celui de degré $0$, qui est inversible modulo
$36$~: à association (dans $\Zz_{36}$) près, $f$ est donc de la forme~:
$$
f=\sum_{i=1}^{d} 6a_{i} x^{i}+1,\qquad (a_{i})\in\Zz_{36}.
$$

Réciproquement, si $f$ est de cette forme, c'est à dire  $f=1+6xf_{1}$,
avec $f_{1}\in\Zz_{36}[x]$, alors~:
$$
 (1+6xf_{1})(1-6xf_{1})=1
$$
donc $f$ est inversible.
}
\end{enumerate}
}

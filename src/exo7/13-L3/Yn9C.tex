\uuid{Yn9C}
\exo7id{6332}
\auteur{queffelec}
\organisation{exo7}
\datecreate{2011-10-16}
\isIndication{false}
\isCorrection{false}
\chapitre{Théorème de Cauchy-Lipschitz}
\sousChapitre{Théorème de Cauchy-Lipschitz}

\contenu{
\texte{
Soit $f$ une application
$K$-lipschitzienne sur un ouvert $U\subset \Rr^n$. On va démon\-trer que
le flot de solutions de $x'=f(x)$, supposé défini sur un
intervalle $[t_0,t_1]$, dépend continument de la condition initiale $
x(t_0)=x_0$.
}
\begin{enumerate}
    \item \question{Soit $x_1,\ x_2$ deux telles solutions; montrer que si $t\in[t_0,t_1]$, 
$$||x_1(t)-x_2(t)||\leq ||x_1(t_0)-x_2(t_0)||e^{K(t-t_0)}$$}
    \item \question{En déduire le résultat et le vérifier sur l'exemple : $x'=x^2$ sur $\Rr$.}
\end{enumerate}
}

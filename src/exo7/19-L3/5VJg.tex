\uuid{5VJg}
\exo7id{6943}
\titre{exo7 6943}
\auteur{ruette}
\organisation{exo7}
\datecreate{2013-01-24}
\isIndication{false}
\isCorrection{true}
\chapitre{Loi, indépendance, loi conditionnelle}
\sousChapitre{Loi, indépendance, loi conditionnelle}
\module{Probabilité et statistique}
\niveau{L3}
\difficulte{}

\contenu{
\texte{

}
\begin{enumerate}
    \item \question{Soit $X$ une variable aléatoire de loi  $\mathcal{E}(\lambda)$. Montrer que
$P(X> t+s\mid X> t)=P(X> s)$ pour tous $s,t\geq 0$.}
\reponse{$P(A\mid B):= \frac{P(A\cap B)}{P(B)}$ donc
$P(X> t+s\mid X> t)=\frac{P(X> t+s)}{P(X> t)}$.
Or $\displaystyle P(X> y)=\displaystyle\int_y^{+\infty}\lambda e^{-\lambda x}dx=
e^{-\lambda y}$ si $y\geq 0.$
Donc $\displaystyle
P(X> t+s\mid X> t)=\frac{e^{-\lambda(t+s)}}{e^{-\lambda t}}=
e^{-\lambda s}=P(X> s).$}
    \item \question{Soit $X$ une variable aléatoire positive avec une densité continue sur
$\Rr_+$. Si
$P(X> t+s\mid X> t)=P(X> s)$  pour tous $s,t\geq 0$, montrer
que $X$ suit une loi exponentielle.

\medskip
\textit{
Ce résultat montre que la loi exponentielle est une loi sans mémoire,
et que c'est la seule sous l'hypothèse du 2.
En fait, cette hypothèse n'est pas nécessaire
mais le résultat est alors plus difficile à montrer.}}
\reponse{Soit $H(t)=1-F_X(t)=P(X>t)$. Par hypothèse, on a $H(t+s)=H(t)H(s)$.
Comme la densité de $X$ est continue sur $\Rr_+$, $F$ est dérivable sur
$\Rr_+$, donc $H$
aussi. Si on dérive par rapport à $s$, on trouve : $H'(t+s)=H(t)H'(s)$.
Si $s=0$, on a $H'(t)=H(t)H'(0)$. Posons $\lambda=-H'(0)$. La fonction $H$
est solution sur $[0,+\infty[$ de l'équation $y'+\lambda y=0$, 
donc il existe $K$ tel que
$H(t)=Ke^{-\lambda t}$, et $F_X(t)=1-Ke^{-\lambda t}$ pour tout $t\geq 0$. 
Le cas $K=0$ est exclu, sinon
$P_X=\delta_0$, ce qui est impossible car $X$ est une variable aléatoire à densité.
On a
$$
\lim_{t\to+\infty}F_X(t)=1\quad \mbox{et}\quad
F_X(0)=0\quad \mbox{car}\quad X\mbox{ est une variable aléatoire positive},
$$ 
donc nécessairement $\lambda>0$ et $K=1$. 
En dérivant $F_X(t)=1-e^{-\lambda t}$,
on trouve que la densité de $X$ sur $\Rr_+$ 
est $\lambda e^{-\lambda t}$, c'est-à-dire
que $X$ est de loi exponentielle de paramètre $\lambda$.}
\end{enumerate}
}

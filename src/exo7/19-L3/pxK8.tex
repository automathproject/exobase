\uuid{pxK8}
\exo7id{6948}
\titre{exo7 6948}
\auteur{ruette}
\organisation{exo7}
\datecreate{2013-01-24}
\isIndication{false}
\isCorrection{true}
\chapitre{Loi, indépendance, loi conditionnelle}
\sousChapitre{Loi, indépendance, loi conditionnelle}
\module{Probabilité et statistique}
\niveau{L3}
\difficulte{}

\contenu{
\texte{
Soit $X$ une variable aléatoire de loi exponentielle $\mathcal{E}(\lambda)$. Calculer sa
fonction caractéristique.
}
\reponse{
\def\I1{{ \rm 1\:\!\!\! l}}
La densité de $X$ étant  $\I1_{\Rr^+}(x)\lambda e^{-\lambda x}$, on doit
calculer
$\displaystyle\varphi_X(t)=\int_0^{+\infty}e^{itx}\lambda e^{-\lambda x}dx$.
On a
$$
\int_0^Me^{itx}\lambda e^{-\lambda x}dx
=
\lambda \left[\frac{1}{it-\lambda}e^{(it-\lambda)x}\right]_0^M
=\frac{\lambda}{it-\lambda}\left(e^{(it-\lambda)M}-1\right)
$$
(remarquer que $it-\lambda\not=0$ car $t\in\Rr$ et $\lambda>0$).

\medskip
On a $|e^{(it-\lambda)M}|=|e^{it}|e^{-\lambda M}=e^{-\lambda M}\to 0$ quand
$M\to +\infty$ donc l'intégrale est bien convergente et
$\displaystyle\varphi_X(t)=\frac{\lambda}{\lambda-it}$.
}
}

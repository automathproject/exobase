\uuid{Npla}
\exo7id{5983}
\auteur{quinio}
\organisation{exo7}
\datecreate{2011-05-18}
\isIndication{false}
\isCorrection{true}
\chapitre{Probabilité discrète}
\sousChapitre{Probabilité et dénombrement}

\contenu{
\texte{
Une entreprise décide de classer $20$ personnes
susceptibles d'être embauchées; leurs CV étant très proches,
le patron décide de recourir au hasard : combien y-a-il de classements
possibles : sans ex-aequo; avec exactement $2$ ex-aequo ?
}
\reponse{
Choix des deux ex-aequo : $\binom{20}{2}=$ $190$ choix;
Place des ex-aequo : il y a $19$ possibilités;
Classements des $18$ autres personnes, une fois les ex-aequo placés : il y a $18!$ choix.
}
}

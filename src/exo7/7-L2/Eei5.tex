\uuid{Eei5}
\exo7id{6931}
\titre{exo7 6931}
\auteur{ruette}
\organisation{exo7}
\datecreate{2013-01-24}
\isIndication{false}
\isCorrection{true}
\chapitre{Statistique}
\sousChapitre{Tests d'hypothèses, intervalle de confiance}
\module{Probabilité et statistique}
\niveau{L2}
\difficulte{}

\contenu{
\texte{
La firme Comtec vient de développer un 
nouveau dispositif électronique. Avant de le mettre en production, on 
veut en estimer la fiabilité en termes de durée de vie. D'après le 
bureau de Recherche et Développement de l'entreprise, l'écart-type 
de la durée de vie de ce dispositif serait de l'ordre de  100 heures.

Déterminer le nombre d'essais requis pour estimer, avec un niveau de 
confiance de 95\%, la durée de vie moyenne d'une grande production 
de sorte que la marge d'erreur dans l'estimation n'excède pas  $\pm$ 50 heures.
 Même question pour une marge d'erreur n'excédant pas $\pm$ 20 heures.
}
\reponse{
On suppose que la variable $X=$ ``durée de vie d'un appareil'' suit 
une loi normale d'espérance $m_{pop}$ et d'écart-type $\sigma_{pop}=100$h. 
Si on fait $n$ essais indépendants, alors $T=\sqrt{n}\frac{\bar{X}-m_{pop}}{\sigma_{pop}}$ 
suit une loi normale standard, donc $p(|T|>1,96)<0,05$. Au niveau de confiance 95\%,
on peut affirmer que l'intervalle 
$[\bar{x}-1,96\frac{100}{\sqrt{n}},\bar{x}+1,96\frac{100}{\sqrt{n}}]$ 
contient la durée de vie moyenne cherchée $m_{pop}$. 

La marge d'erreur n'excède pas 50h dès que $1,96\frac{100}{\sqrt{n}}<50$, i.e. si $n\geq 16$.

La marge d'erreur n'excède pas 20h dès que $1,96\frac{100}{\sqrt{n}}<20$, i.e. si $n\geq 97$.
}
}

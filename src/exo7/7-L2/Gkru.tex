\uuid{Gkru}
\exo7id{5999}
\auteur{quinio}
\organisation{exo7}
\datecreate{2011-05-20}
\isIndication{false}
\isCorrection{true}
\chapitre{Probabilité discrète}
\sousChapitre{Probabilité conditionnelle}

\contenu{
\texte{
Un constructeur aéronautique équipe ses avions trimoteurs d'un
moteur central de type A et de deux moteurs, un par aile, de type B; chaque
moteur tombe en panne indépendamment d'un autre, et on estime à $p$ la
probabilité pour un moteur de type A de tomber en panne et à $q$ la
probabilité pour un moteur de type B de tomber en panne.

Le trimoteur peut voler si le moteur central \emph{ou} les deux moteurs d'ailes
fonctionnent : quelle est la probabilité pour l'avion de voler?
Application numérique : $p = 0.001\%$, $q = 0.02\%$.
}
\reponse{
On obtient par calcul direct ou par événement contraire
la probabilité de voler : $1-p+p(1-q)^{2}$.
}
}

\uuid{ElFZ}
\exo7id{6900}
\titre{exo7 6900}
\auteur{ruette}
\organisation{exo7}
\datecreate{2013-01-24}
\isIndication{false}
\isCorrection{true}
\chapitre{Probabilité discrète}
\sousChapitre{Probabilité conditionnelle}
\module{Probabilité et statistique}
\niveau{L2}
\difficulte{}

\contenu{
\texte{
Il y a $5\%$ de daltoniens chez les hommes et $0,25\%$
chez les femmes. Il y a $48\%$ d'hommes et
$52\%$ de femmes dans la population. Quelle est la probabilité pour
qu'un daltonien soit un homme ?

\medskip
\textit{Remarque : la forme la plus courante du daltonisme est génétique, due
à un gène récessif porté par le chromosome X. Un homme (XY) est daltonien
dès que le chromosome X porte ce gène. Une femme (XX) n'est daltoniene
que si les 2 chromosomes X portent ce gène. Ceci explique les taux
très différents chez les hommes et les femmes.}
}
\reponse{
La méthode la plus simple consiste à introduire la population
totale $N$ et à compter les daltoniens. Soit $d_H=5\%$, $d_F=0,25\%$
(taux de daltoniens chez les hommes et les femmes),
$p_H=48\%$, $p_F=52\%$ (proportions d'hommes et de femmes dans la population).
Le nombre d'hommes est $Np_H$, le nombre d'hommes daltoniens
est $Np_Hd_H$. De même, le nombre de femmes daltoniennes est
$Np_Fd_F$. La proportion de daltoniens hommes parmi les daltoniens est
donc
$$
\frac{\mbox{nombre de daltoniens hommes}}{\mbox{nombre de daltoniens}}=
\frac{N p_Hd_H}{Np_Hd_H+Np_Fd_F}=\frac{p_Hd_H}{p_Hd_H+p_Fd_F}\approx 0,95.
$$
La probabilité pour qu'un daltonien soit un homme est d'environ 95\%.

\medskip
Une formulation plus élaborée (mais strictement équivalente) 
consiste à utiliser la formule de Bayes.
Soit $H$ l'événement ``être un homme'' et $D$ l'événement ``être daltonien''.
On veut calculer $P(H|D)$. Selon la formule de Bayes,
$P(H|D)=\frac{P(D|H)P(H)}{P(D)}.$ En utilisant que
$P(D)=P(D|H)P(H)+P(D|F)P(F)$ (formule des probabilités totales), on
obtient
$$P(H|D)=\frac{P(D|H)P(H)}{P(D|H)P(H)+P(D|F)P(F)}=\frac{p_Hd_H}{p_Hd_H+p_Fd_F}
.$$
}
}

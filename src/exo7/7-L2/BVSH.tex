\uuid{BVSH}
\exo7id{6021}
\auteur{quinio}
\organisation{exo7}
\datecreate{2011-05-20}
\isIndication{false}
\isCorrection{true}
\chapitre{Probabilité discrète}
\sousChapitre{Lois de distributions}

\contenu{
\texte{
On effectue un contrôle sur des pièces de un euro
dont une proportion $p=0,05$ est fausse et sur des pièces 
de 2 euros dont une proportion $p'=0,02$ est fausse.
Il y a dans un lot $500$ pièces dont $150$ pièces de 
un euro et $350$ pièces de 2 euros.
}
\begin{enumerate}
    \item \question{On prend une pièce au hasard dans ce lot: quelle est la probabilité qu'elle soit fausse?}
    \item \question{Sachant que cette pièce est fausse, quelle est la probabilité
qu'elle soit de un euro?}
    \item \question{On contrôle à présent un lot de 1000 pièces de un euro.
Soit $X$ la variable aléatoire: <<nombre de pièces fausses parmi 1000>>.
Quelle est la vraie loi de $X$ ? (on ne donnera que la forme générale);
quelle est son espérance, son écart-type?}
    \item \question{En approchant cette loi par celle d'une loi normale adaptée, calculez
la probabilité pour que $X$ soit compris entre 48 et 52.}
\reponse{
Soit $F$ l'événement <<la pièce est fausse>>; soit $U$ l'événement <<la pièce est un euro>>; 
soit $D$ l'événement <<la pièce est deux euros>>. Alors
$P(F)=P(F/U)P(U)+P(F/D)P(D)=2.9\%$.
On cherche $P(U/F)=$($P(F/U)P(U))/P(F)=51.7\%$.
$X$ la variable aléatoire <<nombre de pièces 
fausses parmi 1000>> obéit à une loi binomiale$B(1000;5\%)$.
Espérance: 50; écart-type: $\sigma =\sqrt{47.5}$.
En approchant cette loi par une loi normale $N(50; \sigma)$,
la probabilité pour que $X$ soit compris entre 48 et 52 est :
$P[(47.5-50)/\sigma \leq (X-50)/\sigma \leq (52.5-50)/\sigma ]\simeq 28.3\%$.
}
\end{enumerate}
}

\uuid{0aA0}
\exo7id{6891}
\auteur{ruette}
\organisation{exo7}
\datecreate{2013-01-24}
\isIndication{false}
\isCorrection{true}
\chapitre{Probabilité discrète}
\sousChapitre{Probabilité et dénombrement}

\contenu{
\texte{
Une urne contient une boule rouge, trois boules vertes et seize boules blanches. 
La boule rouge permet de gagner 10 euros, chaque boule verte permet de gagner 5 euros 
et les boules blanches ne rapportent rien. Un joueur tire simultanément cinq boules. 
Quelle est la probabilité pour que ce joueur gagne exactement 10 euros ?
}
\reponse{
Pour gagner 10 euros, il faut avoir tiré exactement une boule rouge et 4 boules blanches, 
ou 2 boules vertes et 3 boules blanches. Ces deux événements, notés $A$ et $B$, sont 
incompatibles. Ici l'univers $\Omega$ est l'ensemble des combinaisons de 5 boules, 
c'est-à-dire l'ensemble des parties à 5 éléments d'un ensemble de 20 boules. 
Les ${20\choose{5}}=\frac{20.19.18.17.16}{5!}=15504$ combinaisons sont équiprobables. 
Les éléments de l'événement $A$ sont les combinaisons formées de la boule rouge et 
d'une combinaison de 4 boules blanches. Il y en a ${16\choose4}=\frac{16.15.14.13}{4!}=1820$. 
Par conséquent, $p(A)=\frac{1820}{15504}$. Les éléments de l'événement $B$ sont les 
combinaisons formées d'une combinaison de deux boules vertes et d'une combinaison de 3 
boules blanches. Il y en a ${3\choose{2}}{16\choose{3}}=3.\frac{16.15.14}{6!}=1680$. 
Par conséquent, $p(B)=\frac{1680}{15504}$.  La probabilité de gagner 10 euros est donc 
égale à $p(A)+p(B)= \frac{3500}{
15504}\simeq 0.225...$.
}
}

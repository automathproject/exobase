\uuid{DnqT}
\exo7id{2790}
\auteur{burnol}
\organisation{exo7}
\datecreate{2009-12-15}
\isIndication{false}
\isCorrection{true}
\chapitre{Fonction holomorphe}
\sousChapitre{Fonction holomorphe}

\contenu{
\texte{
En quels points la fonction $z\mapsto \overline{z}$
  est-elle dérivable au sens complexe, et/ou holomorphe?
  Même question pour les fonctions $z\mapsto x$ et $z\mapsto y$.
}
\reponse{
La fonction $f(z)=\overline{z}$ n'est nulle part dérivable au sens complexe
(et donc nulle part holomorphe): car
$$\frac{1}{h} (f(z+h)-f(z))=\frac{\overline{h}}{h}$$
et la limite de cette expression n'existe pas lorsque $h\to 0$.
\medskip
\emph{Remarque.} Plus g\'en\'eralement, une application $\R$--lin\'eaire de $\C$ dans $\C$ est de la forme
\begin{equation}\label{eq::1}
w\mapsto \alpha w +\beta \overline{w} 
\end{equation}
(ce n'est qu'une \'ecriture complexe des applications lin\'eaires de $\R^2$ dans $\R^2$) ; une telle application
est holomorphe si et seulement si $\beta =0$.
C'est exactement la diff\'erence entre diff\'erentiabilit\'e
(donc r\'eelle) et holomorphie (d\'erivabilit\'e au sens complexe). En effet, les \'equations de Cauchy-Riemann
sont \'equivalentes \`a l'\'equation $\frac{\partial}{\partial \overline{z}} f(z) =0$ o\`u
$\frac{\partial}{\partial \overline{z}}=\frac{1}{2} \left( \frac{\partial}{\partial x}+i \frac{\partial}{\partial y}\right)$.
C'est une r\'e\'ecriture complexe des \'equations de Cauchy-Riemann. Si vous avez une fonction $f$ diff\'erentiable,
alors sa diff\'erentielle $D f(z)$ est une application lin\'eaire de la forme \eqref{eq::1}. Un calcul simple montre que dans
ce cas
$$\beta = \frac{\partial f}{\partial \overline{z}} (z) \quad et \quad \alpha = \frac{\partial f}{\partial z} (z) $$
avec $\frac{\partial}{\partial z}=\frac{1}{2} \left( \frac{\partial}{\partial x}-i \frac{\partial}{\partial y}\right)$.
De nouveau, $f$ est complexe diff\'erentiable en $z$ si et seulement si $\beta =\frac{\partial f}{\partial \overline{z}} (z)=0$.
Dans ce cas $f'(z) = \frac{\partial f}{\partial z} (z)$.
\bigskip
Revenons \`a l'exercice. Si vous \^etes d'accord avec ma remarque, alors nous sommes aussi d'accord sur le fait que :
$$z\mapsto x =\frac{z+\overline{z}}{2}$$
n'est pas holomorphe. Ce raisonnement s'applique aussi \`a $z\mapsto y$. Nous reviendrons \`a ce
genre d'applications dans l'exercice \ref{ex:burnol1.1.10}.
}
}

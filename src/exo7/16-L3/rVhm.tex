\uuid{rVhm}
\exo7id{2857}
\titre{exo7 2857}
\auteur{burnol}
\organisation{exo7}
\datecreate{2009-12-15}
\isIndication{false}
\isCorrection{true}
\chapitre{Théorème des résidus}
\sousChapitre{Théorème des résidus}
\module{Analyse complexe}
\niveau{L3}
\difficulte{}

\contenu{
\texte{
Soit $f$ une fonction holomorphe sur
$\overline\Omega\setminus\{z_1, \dots, z_N\}$, avec $\Omega$
le domaine intérieur à une courbe de Jordan $\gamma$. Soit
$g_n(z)$ la partie principale (partie singulière) de $f$ en
la singularité isolée $z_n$. Prouver \emph{la formule
intégrale générale de Cauchy}:
\[\forall z\in
\Omega\setminus\{z_1,\dots,z_N\}\qquad f(z) = \sum_{1\leq
n\leq N} g_n(z) + \frac1{2\pi i
}\int_{\partial\Omega}\frac{f(w)}{w-z}dw\]
Pour cela, remarquer d'abord $\mathrm{Res}(\frac{f(w)}{w-z},z_n)=
 \mathrm{Res}(\frac{g_n(w)}{w-z},z_n)$; puis montrer que le résidu à
 l'infini de la fonction $\frac{g_n(w)}{w-z}$  de $w \in
 \Cc\setminus\{z_n\}$, 
 est nul. On pourra utiliser l'exercice \ref{exo:residuinfini}.
}
\reponse{
Rappelons la formule de Cauchy pour $f$ holomorphe sur $\overline{\Omega}$
(donc sans singularit\'es):
$$f(z)=\frac{1}{2i\pi}\int _{\partial \Omega} \frac{f(w)}{w-z}dw.$$
Il s'agit ici d'obtenir une version g\'en\'eralis\'ee pour des fonctions $f$ ayant des singularit\'es
$z_1,...,z_N\in \Omega$. Fixons $z\in \Omega \setminus \{z_1,...,z_N\}$ et consid\'erons $G(w)=\frac{f(w)}{w-z}$.
Cette fonction a un p\^ole simple en $w=z$ et :
$$\mathrm{Res} (G,z)=\lim_{w\to z} (w-z)G(w)=f(z).$$
Les autres singularit\'es de $G$ dans $\Omega$ sont $z_1,...,z_N$.
Par d\'efinition, le r\'esidu de $G$ en $z_j$ est le \og coefficient
$a_{-1}$\fg{} de la s\'erie de Laurent de $G$ en $z_j$. Or
$$G(w)= \frac{1}{w-z} f(w) = \sum _{k\geq 0} b_k (w-z_j)^k \sum _{l\in \Z} c_l (w-z_j)^l$$
puisque $w\mapsto \frac{1}{w-z}$ est holomorphe au voisinage de $z_j$ (et bien s\^ur on peut calculer les $b_k$,
mais ce n'est pas utile). On remarque que pour calculer $a_{-1}$ interviennent seulement les indices $(k,l)$
qui v\'erifient $k+l=-1$. Comme $k\geq 0$ on a $l=-1-k<0$. D'o\`u :
$$\mathrm{Res} (G,z_j)= \mathrm{Res} \left(\frac{g_j(w)}{w-z},z_j\right).$$
On peut maintenant utiliser l'exercice \ref{exo:residuinfini} ou alors conclure directement: si $R_0=2\max \{|z| , |z_j|\}$,
alors pour tout $R>R_0$,
$$\frac{1}{2i\pi}\int_{|w|=R}\frac{g_j(w)}{w-z}dw = \sum_{\xi \in \{z,z_j \}} \mathrm{Res} \left(\frac{g_j(w)}{w-z},\xi \right).$$
Par cons\'equent, l'int\'egrale $\int_{|w|=R}\frac{g_j(w)}{w-z}dw$ ne d\'epend pas de $R> R_0$. Or, il existe $C>0$ tel que
$$\left|\frac{g_j(w)}{w-z} \right| \leq \frac{C}{|w|^2} \quad , \quad |w| > R_0 \,,$$
ce qui entra\^ine
$$\left|\int_{|w|=R_0}\frac{g_j(w)}{w-z}dw \right|\leq \lim_{|w|=R} \int_{|w|=R_0}\frac{C}{|w|^2} |dw| =0.$$
D'o\`u
$$0= \frac{1}{2i\pi} \int_{|w|=R_0}\frac{g_j(w)}{w-z}dw = \mathrm{Res} \left(\frac{g_j(w)}{w-z},z_j\right)+g_j(z).$$
Il suffit alors d'appliquer le th\'eor\`eme des r\'esidus \`a $G$ pour conclure:
$$\frac{1}{2i\pi} \int_{\partial \Omega } G(w) \, dw = f(z) -g_1(z) -...-g_N(z).$$
}
}

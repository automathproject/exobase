\uuid{DDSi}
\exo7id{6686}
\auteur{queffelec}
\organisation{exo7}
\datecreate{2011-10-16}
\isIndication{false}
\isCorrection{false}
\chapitre{Formule de Cauchy}
\sousChapitre{Formule de Cauchy}

\contenu{
\texte{
On considère la série entière
$$L(z)=\sum_{n\ge 1}{z^n\over n^2}$$
Soit $f(z)={-\mathrm{Log}{(1-z)}\over z}$ où $\mathrm{Log}$ désigne la détermination
principale du logarithme complexe.
}
\begin{enumerate}
    \item \question{On note $U={\Cc}\setminus [1,+\infty\mathclose[$. Montrer que $f$ est définie 
dans $U$.}
    \item \question{Vérifier que si $D=\{ z\in {\Cc}\vert\ \vert z\vert<1\}$, 
on a $L'(z)=f(z)$.}
    \item \question{En déduire qu'il existe une primitive de $f$, définie dans $U$ tout
entier, dont la restriction à $D$ est égale à $L$. On note cette primitive 
$L$ par abus de langage.}
    \item \question{Soit $x\in {\Rr}$, $x>1$. Calculer $\lim_{y\to 0}L(x+iy)-L(x-iy)$
comme fonction de $x$.}
\end{enumerate}
}

\uuid{2IK8}
\exo7id{6668}
\titre{exo7 6668}
\auteur{queffelec}
\organisation{exo7}
\datecreate{2011-10-16}
\isIndication{false}
\isCorrection{false}
\chapitre{Fonction logarithme et fonction puissance}
\sousChapitre{Fonction logarithme et fonction puissance}
\module{Analyse complexe}
\niveau{L3}
\difficulte{}

\contenu{
\texte{
On veut démontrer qu'il existe une détermination continue $f$ de 
$\sqrt{1-z^2}$ sur $U=\C\setminus [-1,1]$ telle que $f(i)=\sqrt 2$.
}
\begin{enumerate}
    \item \question{Définir $f$ sur $\C\setminus \mathopen]-\infty,1]$ au moyen des fonctions
$\mathrm{Arg}{(z+1)}$ et $\mathrm{Arg}{(z-1)}$.}
    \item \question{Soit $x$ un réel strictement inférieur à 1. Etudier $\lim_{y\to 0}
f(x+iy)$ quand $y$ tend vers 0 par valeurs positives puis négatives.
Conclure.}
    \item \question{Montrer qu'on obtient ainsi une application $f$ telle que
$f(U)\subset U$ et $f\circ f=-{\rm Id_{\vert U}}$. En déduire que $f$ est
une bijection de $U$ sur lui-même.}
\end{enumerate}
}

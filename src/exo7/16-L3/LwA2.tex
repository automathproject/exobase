\uuid{LwA2}
\exo7id{6713}
\titre{exo7 6713}
\auteur{queffelec}
\organisation{exo7}
\datecreate{2011-10-16}
\isIndication{false}
\isCorrection{false}
\chapitre{Formule de Cauchy}
\sousChapitre{Formule de Cauchy}
\module{Analyse complexe}
\niveau{L3}
\difficulte{}

\contenu{
\texte{
Soit $f$ une fonction holomorphe non identiquement nulle dans un
ouvert $\Omega $ connexe con\-te\-nant le disque fermé
$\overline{D(z_0,r)}$. Soit $\gamma $ le cercle $\{z\vert\ \vert
z-z_0\vert =r\}$ orienté positivement.
}
\begin{enumerate}
    \item \question{Montrer que $f$  a un nombre fini
de zéros dans $D(z_0,r)$.}
    \item \question{On suppose que $f$ n'a pas de zéros sur $\gamma ^*$. Calculer
$\displaystyle {1\over 2i\pi}\int_\gamma {f'(z)\over f(z)}z^pdz$ pour
$p=0,1,\dots$ (utiliser les zéros de $f$ dans $D(z_0,r)$).}
    \item \question{On suppose toujours que $f$ n'a pas de zéros sur $\gamma ^*$. On prend
$p=0$. Montrer que l'intégrale précédente est égale au nombre total de
zéros de $f$ dans $D(z_0,r)$. Montrer que ce nombre est aussi l'indice du
point $0$ par rapport à la courbe fermée $\Gamma =f(\gamma )$.}
    \item \question{Soit $w_0=f(z_0)$ et $n$ la multiplicité du zéro $z_0$ pour la
fonction $f(z)-w_0$. Montrer que l'on peut choisir $\varepsilon >0$ tel
que $f'(z)$ ne s'annule pas pour $0<\vert z-z_0\vert <\varepsilon $, et
qu'il existe un $\delta >0$ tel que pour tout $a$ avec $\vert a-w_0\vert
<\delta $, l'équation $f(z)=a$ a exactement $n$ racines dans le disque
$\vert z-z_0\vert <\varepsilon $.}
    \item \question{En déduire le principe du maximum : si $f$ est holomorphe et non
constante dans $\Omega $, alors $\vert f\vert$ n'a pas de maximum
dans $\Omega $.}
\end{enumerate}
}

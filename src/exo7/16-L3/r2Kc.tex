\uuid{r2Kc}
\exo7id{2797}
\auteur{burnol}
\organisation{exo7}
\datecreate{2009-12-15}
\isIndication{false}
\isCorrection{true}
\chapitre{Fonction holomorphe}
\sousChapitre{Fonction holomorphe}

\contenu{
\texte{
Déterminer en  tout $z_0\neq 1,\;2$ la série de Taylor et son rayon
de convergence pour la fonction analytique
$\frac1{(z-1)(z-2)}$. On aura intérêt à réduire en éléments
simples. De plus on demande d'indiquer le rayon de
convergence \emph{avant} de déterminer explicitement la série
de Taylor.
}
\reponse{
La fonction $f(z)=\frac{1}{(z-1)(z-2)}$ est holomorphe dans $U=\C\setminus\{1,2\}$.
Par ce que l'on vient de dire \`a l'exercice pr\'ec\'edent, 
le rayon de convergence demand\'e est $R=\min \{|z_0-1|,|z_0-2| \}$,
o\`u $z_0\in U$ est un point quelconque fix\'e.
On a $\frac{1}{(z-1)(z-2)}=\frac{1}{z-2}-\frac{1}{z-1}$ et:
$$\frac{1}{z-2}=\frac{1}{z-z_0+z_0-2}=\frac{1}{z_0-2}\frac{1}{1-\left( \frac{z_0-z}{z_0-2}\right)}
=\frac{1}{z_0-2}\sum_{k\geq0} \left( \frac{z_0-z}{z_0-2}\right)^k$$
pour $|z-z_0|<|z_0-2|=R_2$. La s\'erie demand\'ee est alors la diff\'erence entre celle-ci et celle de
l'exercice pr\'ec\'edent. Notons aussi que le rayon de convergence est exactement le minimum des rayons
$R_1$ et $R_2$.
}
}

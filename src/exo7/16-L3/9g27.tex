\uuid{9g27}
\exo7id{6752}
\titre{exo7 6752}
\auteur{queffelec}
\organisation{exo7}
\datecreate{2011-10-16}
\isIndication{false}
\isCorrection{false}
\chapitre{Théorème des résidus}
\sousChapitre{Théorème des résidus}
\module{Analyse complexe}
\niveau{L3}
\difficulte{}

\contenu{
\texte{
Dans cet exercice, on justifiera soigneusement chaque passage à la
limite. Calculer les intégrales :
}
\begin{enumerate}
    \item \question{$\displaystyle \int_{-\infty}^{+\infty}{1\over 1+x^6}\ dx$}
    \item \question{$\displaystyle \int_0^{+\infty}{1\over x^\alpha (1+x)}\ dx\ ,\ 0<\alpha <1$}
    \item \question{$\displaystyle \int_0^{+\infty}{\sin^2x\over x^2}\ dx$
(on pourra considérer la fonction $\displaystyle {1-e^{2ix}\over x^2}$)}
    \item \question{$\displaystyle \int_{-\infty}^{+\infty}{\cos x\over e^x+e^{-x}}\ dx$
(on pourra utiliser le rectangle de sommets $-R$, $R$, $R+i\pi$,
$-R+i\pi$)}
    \item \question{$\displaystyle \int_0^{+\infty}{\mathrm{Log}^2 x\over 1+x^2}\ dx$}
    \item \question{$\displaystyle \int_0^{+\infty}{\mathrm{Log} x\over (1+x)^3}\ dx$}
\end{enumerate}
}

\uuid{k9tP}
\exo7id{6656}
\titre{exo7 6656}
\auteur{queffelec}
\organisation{exo7}
\datecreate{2011-10-16}
\isIndication{false}
\isCorrection{false}
\chapitre{Fonction logarithme et fonction puissance}
\sousChapitre{Fonction logarithme et fonction puissance}
\module{Analyse complexe}
\niveau{L3}
\difficulte{}

\contenu{
\texte{
Soit $\Omega$ un ouvert connexe de $\Cc$ et $f$ une fonction complexe sans
zéro sur $\Omega$. 
On rappelle que
$f$ admet un logarithme continu (resp. holomorphe) sur $\Omega$ s'il existe une
fonction
$g$
 continue (resp. holomorphe) sur $\Omega$ telle que $e^{g(z)}=f(z)$. 

Montrer
que deux déterminations continues du logarithme de $f$ sur $\Omega$ diffèrent
d'une constante $2ki\pi$.

 En reproduisant la démonstration du théorème d'inversion locale, montrer que
si
$f$ admet sur $\Omega$ un logarithme continu, elle y admet un logarithme
holomorphe.
}
}

\uuid{pZeD}
\exo7id{2808}
\titre{exo7 2808}
\auteur{burnol}
\organisation{exo7}
\datecreate{2009-12-15}
\isIndication{false}
\isCorrection{true}
\chapitre{Formule de Cauchy}
\sousChapitre{Formule de Cauchy}
\module{Analyse complexe}
\niveau{L3}
\difficulte{}

\contenu{
\texte{
Il est intéressant que l'équation de l'exercice précédent
 $\frac{\partial F}{\partial\theta} = i\, r\frac{\partial 
 F}{\partial r}$, peut se réécrire dans le système de coordonnées $(a,b) =
 (\log(r),\theta)$ sous la forme: 
\[ \frac{\partial F}{\partial b} = i\, \frac{\partial
 F}{\partial a}\;,\]
autrement dit exactement sous la même forme qu'ont les
 équations de Cauchy-Riemann originelles dans les
 coordonnées cartésiennes $(x,y)$.\footnote{$\frac{\partial
F}{\partial y} = i \frac{\partial
 F}{\partial x}$, ou, plus mnémotechnique: $\frac{\phantom{i}\,\partial
F}{i\,\partial y} = \frac{\partial
 F}{\partial x}$ qui dit ``holomorphe $\Leftrightarrow$ $iy$
 est comme $x$''.} 
 Or $a$ et $b$ sont les parties réelles et imaginaires de la
 combinaison $a+ib$ qui est holomorphe comme fonction de
 $x+iy$: $a+ib = \log(x+iy)$. Montrer que cela est général:
 dans un système de coordonnées $(a,b)$ telles que $w= a+ib$
 est une fonction holomorphe de $z = x+iy$ les équations de
 Cauchy-Riemann pour l'holomorphie (par rapport à $(x,y)$)
 d'une fonction $F$ sont $\frac{\partial F}{\partial
 b} = i \frac{\partial F}{\partial a}$ (ce qui équivaut à
 l'holomorphie de $F$ comme fonction ``sur le plan de
 $w=a+ib$''\footnote{autrement dit pour qu'une fonction
 soit holomorphe comme fonction de $x+iy$ il est nécessaire
 et suffisant qu'elle soit holomorphe comme fonction de
 $a+ib$. En particulier $x+iy$ est une fonction holomorphe
 de $a+ib$: on a donc prouvé que la réciproque d'une
 bijection holomorphe est aussi holomorphe. Nous reviendrons
 là-dessus avec d'autres méthodes (dont celle très concrète de
 l'``inversion'' d'une série entière).}). Indication: prouver
 l'identité :
\[ \frac\partial{\partial x} +
 i\frac\partial{\partial y} = \left(\frac{\partial
 a}{\partial x} - i \frac{\partial
 b}{\partial x}\right)\left(\frac\partial{\partial a} +
 i\frac\partial{\partial b}\right)\;,\]
en exploitant les équations de Cauchy-Riemann
$\frac{\partial
 b}{\partial x} = -\frac{\partial a}{\partial y}$,
 $\frac{\partial a}{\partial x} = +\frac{\partial
 b}{\partial y}$ pour $a+ib = g(x+iy)$.
}
\reponse{
On a $w=g(z)$ avec $g(z)=a(z)+ib(z)$ une fonction holomorphe et avec $z=x+iy$. Utilisons de nouveau le changement
de coordonn\'ees:
$$\frac{\partial }{\partial x}= \frac{\partial a(z)}{\partial x}\frac{\partial }{\partial a}+\frac{\partial b(z)}{\partial x}\frac{\partial }{\partial b}$$
$$\frac{\partial }{\partial y}= \frac{\partial a(z)}{\partial y}\frac{\partial }{\partial a}+\frac{\partial b(z)}{\partial y}\frac{\partial }{\partial b}.$$
En utilisant les \'equations de Cauchy-Riemann on en d\'eduit
$$\begin{aligned}
\frac{\partial }{\partial x}+i\frac{\partial }{\partial y}&= \left(\frac{\partial a(z)}{\partial x}+i \frac{\partial a(z)}{\partial y} \right)\frac{\partial }{\partial a}
+ \left(\frac{\partial b(z)}{\partial x}+i \frac{\partial b(z)}{\partial y} \right)\frac{\partial }{\partial b}\\
&= \left(\frac{\partial a(z)}{\partial x}-i \frac{\partial b(z)}{\partial x} \right)\frac{\partial }{\partial a}
+ \left(\frac{\partial b(z)}{\partial x}+i \frac{\partial a(z)}{\partial x} \right)\frac{\partial }{\partial b}\\
&=\left(\frac{\partial a(z)}{\partial x}-i \frac{\partial b(z)}{\partial x} \right)\left(\frac{\partial }{\partial a}+i\frac{\partial }{\partial b}\right).
\end{aligned}$$
}
}

\uuid{ktFi}
\exo7id{2787}
\auteur{burnol}
\organisation{exo7}
\datecreate{2009-12-15}
\isIndication{false}
\isCorrection{true}
\chapitre{Fonction holomorphe}
\sousChapitre{Fonction holomorphe}

\contenu{
\texte{
\label{ex:burnol1.1.5}
Soit $f$ et $g$ deux fonctions $n$-fois dérivables au sens
complexe sur un ouvert non vide $U$ (remarque: d'après le
cours il suffit qu'elles soient dérivables une fois sur $U$
pour qu'elles le soient un nombre quelconque de
     fois). Montrer la formule de Leibniz généralisée: 
\[ \forall z\in U\qquad (fg)^{(n)}(z) = \sum_{j=0}^n \binom{n}{j}
f^{(j)}(z)g^{(n-j)}(z)\]
}
\reponse{
La formule de Leibniz se montre par r\'ecurrence. Le cas $n=1$, c'est-\`a-dire $(fg)'=fg'+f'g$,
a \'et\'e d\'emontr\'e dans l'exercice \ref{ex:burnol1.1.2}. Supposons alors que cette formule soit vraie au rang $n\geq 1$.
Dans ce cas,
$$\begin{aligned}
(fg)^{(n+1)}(z) &=\frac{d}{dz} \left( (fg)^{(n)}\right)(z) =
\sum_{j=0}^n  \binom{n}{j} \left\{ f^{(j+1)}(z)g^{(n-j)}(z)+f^{(j)}(z)g^{(n-j+1)}(z) \right\}\\
&=\sum_{j=1}^{n+1}  \binom{n}{j-1}f^{(j)}(z)g^{(n+1-j)}(z) +\sum_{j=0}^{n}  \binom{n}{j}f^{(j)}(z)g^{(n+1-j)}(z) .
\end{aligned}$$
La conclusion vient du fait : $\binom{n}{j-1}+\binom{n}{j}=\binom{n+1}{j}$ qui est simple \`a v\'erifier.
}
}

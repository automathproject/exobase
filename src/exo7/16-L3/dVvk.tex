\uuid{dVvk}
\exo7id{6652}
\auteur{queffelec}
\organisation{exo7}
\datecreate{2011-10-16}
\isIndication{false}
\isCorrection{false}
\chapitre{Fonction holomorphe}
\sousChapitre{Fonction holomorphe}

\contenu{
\texte{

}
\begin{enumerate}
    \item \question{Soient $\rho $ un réel strictement positif, $z$ et $w$ des
nombres complexes tels que $\vert z\vert >\rho $ et $\vert w\vert
>\rho $, et $n$ un entier naturel.

Montrer que 
$$\left\vert {1\over w^n}-{1\over z^n}\right\vert\le \vert z-w\vert
{n\over \rho ^{n+1}}$$
et que
$$\left\vert {1\over z-w}\left({1\over w^n}-{1\over
z^n}\right)-{n\over z^{n+1}}\right\vert =
\left\vert\sum_{k=1}^n\left({1\over w^n}-{1\over z^n}\right){1\over
z^{n-k+1}}\right\vert \le \vert z-w\vert {n^2\over \rho ^{n+2}}$$

Soient maintenant $\sigma $ et $\phi $ deux fonctions continues à
valeurs complexes définies sur un intervalle $I=[a,b]$. On fixe un point
$z\in \C\setminus \sigma (I)$ et on pose $\rho ={1\over 2}\inf_{a\le
t\le b}\vert \sigma (t)-z\vert $.}
    \item \question{Soit $\displaystyle g(z)=\int_a^b{\phi (t)\over (\sigma (t)-z)^n}dt$.
En remplaçant dans a) $z$ par $\sigma (t)-z$ et $w$ par $\sigma
(t)-z-h$ avec $\vert h\vert <\rho $, montrer que
$$\left\vert {g(z+h)-g(z)\over h}-n\int_a^b{\phi (t)\over (\sigma
(t)-z)^{n+1}}dt\right\vert\le {\vert h\vert n^2\over \rho
^{n+2}}\int_a^b\vert \phi (t)\vert dt$$

En déduire que $g(z)$ est holomorphe sur $\C\setminus \sigma (I)$ et
que $g'$ est donnée par
$$g'(z)=n\int_a^b{\phi (t)\over (\sigma (t)-z)^{n+1}}dt$$}
\end{enumerate}
}

\uuid{su10}
\exo7id{6758}
\auteur{queffelec}
\organisation{exo7}
\datecreate{2011-10-16}
\isIndication{false}
\isCorrection{false}
\chapitre{Théorème des résidus}
\sousChapitre{Théorème des résidus}

\contenu{
\texte{
En utilisant le théorème de Rouché, on se propose de
donner une preuve du théorème d'inversion local ``holomorphe'' n'utilisant
pas le théorème d'inversion local dans $\Rr^2$. 

Soit donc $f$ une fonction holomorphe au voisinage d'un point $z_0$ de $\Cc$ telle que $f'(z_0)\ne0$. On suppose sans restreindre la généralité que
$z_0=0$, $f(z_0)=0$, et $f'(z_0)=1$.
}
\begin{enumerate}
    \item \question{Montrer qu'il existe un voisinage $V$ de $z_0$ et une constante $K>0$
tel que l'on ait, pour tout $z$ dans $V$, $|f(z) - z|\le K\,|z^2|$.}
    \item \question{Montrer qu'il existe $r>0$ tel que, si l'on a $|\alpha|<r/2$,
l'équation $f(z) = \alpha$ a une solution unique dans le disque $D(0,r)$.}
    \item \question{Montrer enfin qu'il existe un ouvert $U$ de $\Cc$ tel que $f$
soit une bijection de $U$ sur le disque $D(0, r/2)$.}
    \item \question{En déduire que $g = f^{-1}$ est continue sur le disque $D(0,r/2)$
et, de là, que $g$ est holomorphe dans ce disque.}
\end{enumerate}
}

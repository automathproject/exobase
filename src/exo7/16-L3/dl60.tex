\uuid{dl60}
\exo7id{6603}
\titre{exo7 6603}
\auteur{hueb}
\organisation{exo7}
\datecreate{2011-10-16}
\isIndication{false}
\isCorrection{false}
\chapitre{Tranformée de Laplace et de Fourier}
\sousChapitre{Tranformée de Laplace et de Fourier}
\module{Analyse complexe}
\niveau{L3}
\difficulte{}

\contenu{
\texte{
La transformée de
Laplace de la \lq\lq fonction\rq\rq\ impulsion de Dirac: Pour
$\varepsilon >0$, on pose
$$
F_\varepsilon (t) = \begin{cases}
  \frac 1 \varepsilon,\quad & 0 \leq t \leq \varepsilon\\
  0,\quad & t > \varepsilon \phantom{ \leq 0}
                    \end{cases}.
                    $$
}
\begin{enumerate}
    \item \question{Trouver $\mathcal{L}(F_\varepsilon)$.}
    \item \question{Montrer que $\lim_{\varepsilon \to 0}\mathcal{L}(F_\varepsilon)
  = 1$.}
\end{enumerate}
}

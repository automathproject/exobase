\uuid{Fkjn}
\exo7id{2673}
\titre{exo7 2673}
\auteur{matexo1}
\organisation{exo7}
\datecreate{2002-02-01}
\isIndication{false}
\isCorrection{true}
\chapitre{Théorème des résidus}
\sousChapitre{Théorème des résidus}
\module{Analyse complexe}
\niveau{L3}
\difficulte{}

\contenu{
\texte{
Calculer les int{\'e}grales
$$ I = \int_{-\infty }^{+\infty } {dx\over x^4+x^2+1}
\qquad J = \int_{-\infty }^{+\infty } {\sin^2 x\over x^4+x^2+1}\,dx.$$
}
\reponse{
Le polyn{\^o}me  $P(z) = z^4+z^2+1$ a pour racines les racines carr{\'e}es de $j$ et $j^2$,
soit $\pm j^2$ et $\pm j$. Seuls $j$ et $-j^2$ ont une partie imaginaire positive, donc
$$ I = 2i\pi  (\mbox{\rm Res}(1/P, j) + \mbox{\rm Res}(1/P, -j^2))
 = 2i\pi  \left({1\over P'(j)} + {1\over P'(-j^2)}\right) = {\pi \over \sqrt 3}.$$

Puis on a
$$ J = {1\over 2} \int_{-\infty }^{+\infty } {1-\cos 2x\over x^4+x^2+1}\,dx
 = {1\over 2} I -{1\over 2} \int_{-\infty }^{+\infty } {\cos 2x\over x^4+x^2+1}\,dx
 = (I-K)/2 $$
On a par la m{\^e}me m{\'e}thode
$$ K = 2i\pi  \left({e^{2i j}\over P'(j)} + {e^{2i (-j^2)}\over P'(-j^2)}\right) $$
Or
$$ e^{2ij} = \exp\left(2i \left(1 +i\sqrt 3 \over 2\right)\right) = e^{-\sqrt3 -i}$$
et de m{\^e}me $e^{2i(-j^2)}= e^{-\sqrt 3+i}$. Finalement
$$ J = {\pi \over 2\sqrt 3} \left[ 1 -e^{-\sqrt3} \left(\sqrt 3\sin 1 +\cos1\right)\right].$$
}
}

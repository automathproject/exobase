\uuid{R9FF}
\exo7id{7629}
\titre{exo7 7629}
\auteur{mourougane}
\organisation{exo7}
\datecreate{2021-08-10}
\isIndication{false}
\isCorrection{true}
\chapitre{Autre}
\sousChapitre{Autre}
\module{Analyse complexe}
\niveau{L3}
\difficulte{}

\contenu{
\texte{
Soit $f :\Cc\to\Cc$ une application holomorphe non constante.
}
\begin{enumerate}
    \item \question{Montrer que le point $0$ est dans l'adhérence de l'image de $f$.}
\reponse{Supposons que $0$ n'est pas dans l'adhérence de l'image de $f$. Il existe alors $r>0$ tel que $f(\Cc)\cap \Delta_r =\emptyset$. Autrement dit,
$$\forall z\in\Delta, \ |f(z)|\geq r.$$
L'application $f$ ne s'annule donc pas dans $\Cc$ et la fonction holomorphe $1/f$ est majorée par $1/r$ sur $\Cc$. Par le théorème de Liouville, elle est donc constante, ainsi que $f$.}
    \item \question{Déterminer l'adhérence de l'image de $f$.}
\reponse{Soit $c\in \Cc$. En appliquant le résultat précédent à $f-c$ on obtient que $c$ est dans l'adhérence de l'image de $f$.
         Par conséquent, l'adhérence de l'image de $f$ est $\Cc$.}
\end{enumerate}
}

\uuid{dvtf}
\exo7id{7243}
\auteur{megy}
\organisation{exo7}
\datecreate{2021-03-06}
\isIndication{false}
\isCorrection{false}
\chapitre{Formule de Cauchy}
\sousChapitre{Formule de Cauchy}

\contenu{
\texte{
(Raffinement de Goursat.)
L'objectif de cet exercice est de démontrer la version suivante du lemme de Goursat :\\
\textbf{Lemme. (Raffinement de Goursat)}\\
Soit \(U\) un ouvert. Soit  \(\Delta\) un triangle dans \(U\). Soit \(z_0\in \Delta\). Soit \(f:U\to \C\) une fonction continue sur \(U\) holomorphe sur \(U\setminus\{z_0\}\). Alors,
\[\int_{\partial \Delta}f(z)dz=0.\]
}
\begin{enumerate}
    \item \question{On traite d'abord le cas où \(z_0\) est un sommet du triangle. On dénote les deux autres sommets par \(z_1\) et \(z_2\).
% AJOUTER FIGURE ?
\begin{enumerate}}
    \item \question{Montrer que pour tout \(z_1'\) sur le segment \([z_0,z_1]\) et pour tout \(z_2'\) sur le segment \([z_0,z_2]\), on a 
\[\int_{\partial \Delta_{z_0z_1z_2}}f(z)dz=\int_{\partial \Delta_{z_0z'_1z'_2}}f(z)dz.\]}
    \item \question{Montrer que quand \(z_1'\to z_0\) et \(z_2'\to z_0\), alors \(\displaystyle{\int_{\partial \Delta_{z_0z'_1z'_2}}f(z)dz\to 0}\).}
    \item \question{Démontrer le lemme dans ce cas.}
\end{enumerate}
}

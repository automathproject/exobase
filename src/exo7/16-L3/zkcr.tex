\uuid{zkcr}
\exo7id{6604}
\titre{exo7 6604}
\auteur{hueb}
\organisation{exo7}
\datecreate{2011-10-16}
\isIndication{false}
\isCorrection{false}
\chapitre{Tranformée de Laplace et de Fourier}
\sousChapitre{Tranformée de Laplace et de Fourier}
\module{Analyse complexe}
\niveau{L3}
\difficulte{}

\contenu{
\texte{
Calculer la transformée
de Laplace de la fonction $F$ où $a,b, \omega, k \in \Rr$, $a,b >
0$:
}
\begin{enumerate}
    \item \question{$\quad F(t) = a \sin \omega t,$}
    \item \question{$\quad F(t) = a (1-e^{- bt}),$}
    \item \question{$\quad F(t) = a \cos (bt-k).$
  N.B. Ici la formule qui exprime la transformée de Laplace de la fonction \\
  $G(t) = \begin{cases}
    F(t-a),&t \geq a,\\
    0,\quad &t < a,
                \end{cases}$
                en fonction de celle de $F$ ne s'applique pas à (3).
                Pourquoi pas?}
\end{enumerate}
}

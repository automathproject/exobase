\uuid{JMVg}
\exo7id{6657}
\auteur{queffelec}
\organisation{exo7}
\datecreate{2011-10-16}
\isIndication{false}
\isCorrection{false}
\chapitre{Fonction logarithme et fonction puissance}
\sousChapitre{Fonction logarithme et fonction puissance}

\contenu{
\texte{
On rappelle qu'une fonction complexe $f$ a une
racine
$n$ième holomorphe dans un ouvert connexe $\Omega$ s'il existe $g\in H(\Omega)$
telle que $g^n(z)=f(z)$.
}
\begin{enumerate}
    \item \question{Montrer que si $f$ admet un logarithme holomorphe dans $\Omega$, elle y
admet des racines de tous ordres; montrer, sur un exemple, qu'une fonction
holomorphe $f$ peut admettre une racine sans admettre de logarithme
(holomorphe).}
    \item \question{Si $g_1,g_2$ sont deux fonctions continues de 
$\Omega$ connexe dans ${\Cc}\backslash\{0\}$, telles que $g_1^n=g_2^n$ pour un entier $n\geq1$,
montrer que  $g_1=e^{2i\pi k\over n}g_2$ où $k$ est un entier et $g_1=g_2$ dès
que les fonctions coincident en un point.}
\end{enumerate}
}

\uuid{4fC6}
\exo7id{6719}
\auteur{queffelec}
\organisation{exo7}
\datecreate{2011-10-16}
\isIndication{false}
\isCorrection{false}
\chapitre{Formule de Cauchy}
\sousChapitre{Formule de Cauchy}

\contenu{
\texte{
Soit $f$ une fonction holomorphe dans un domaine contenant la couronne
fermée constituée par les $z\in\C$ tels que $r_1\le\vert z\vert \le
r_2$ (où $0<r_1<r_2$). On pose $M(r)=\max_{\vert z\vert =r}\vert
f(z)\vert$ pour $r_1\le r\le r_2$.
}
\begin{enumerate}
    \item \question{Montrer qu'il existe un nombre réel $\alpha $ tel que
$r_1^\alpha M(r_1)=r_2^\alpha M(r_2)$.}
    \item \question{Montrer que
$$M(r)\le M(r_1)^{\ln{r_2}-\ln{r}\over \ln{r_2}-\ln{r_1}}
M(r_2)^{\ln{r}-\ln{r_1}\over \ln{r_2}-\ln{r_1}}$$
(on appliquera le principe du maximum à la fonction $z^pf(z)^q$ où
$p\in{\Zz}$ et $q\in {\Nn}^*$, puis on considèrera une suite
$(p_n,q_n)$, $p_n\in{\Zz}$ et $q_n\in {\Nn}^*$, telle que
$\lim_{n\to\infty}p_n/q_n=\alpha $).}
\end{enumerate}
}

\uuid{qZCd}
\exo7id{2884}
\titre{exo7 2884}
\auteur{burnol}
\organisation{exo7}
\datecreate{2009-12-15}
\isIndication{false}
\isCorrection{true}
\chapitre{Autre}
\sousChapitre{Autre}
\module{Analyse complexe}
\niveau{L3}
\difficulte{}

\contenu{
\texte{
Déterminer l'image par $z\mapsto \frac{3z+5}{z+2}$ du cercle
unité, du cercle de rayon $2$ centré en $1$, du cercle de
rayon $2$ centré en l'origine; de la droite imaginaire, de
la droite d'équation $x=y$, de la droite verticale passant
en $3$, de la droite verticale passant en $-2$.
}
\reponse{
L'application $\Phi (z)=\frac{3z+5}{z+2}$ est une homographie. L'image d'un cercle est alors de nouveau
un cercle ou une droite. De plus on remarque que
(1) $\Phi (x)\in \R$ pour tout r\'eel $x\neq -2$.
(2) $\Phi (\overline{z}) = \overline {\Phi (z)}$ pour tout $z\in \C\setminus \{-2\}$.
\noindent Comme $\Phi (-1) =2$ et $\Phi (1) =\frac{8}{3}$, l'image du cercle unit\'e est un cercle sym\'etrique
par rapport \`a l'axe r\'eel (cf. (2)) avec centre $\big( \frac{8}{3} +2\big) \frac{1}{2}=\frac{7}{3}$
et de rayon $\frac{8}{3}-\frac{7}{3}=\frac{1}{3}$.
Le cercle de rayon $2$ centr\'e \`a l'origine contient $-2$. C'est l'unique point dont l'image est
$\Phi (-2)=\infty$. L'image de ce cercle est alors une droite et c'est
$$\Phi (2)+i\R = \frac{11}{4} + i\R .$$
}
}

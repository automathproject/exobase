\uuid{tpKI}
\exo7id{7524}
\auteur{mourougane}
\organisation{exo7}
\datecreate{2021-08-10}
\isIndication{false}
\isCorrection{false}
\chapitre{Fonction holomorphe}
\sousChapitre{Fonction holomorphe}

\contenu{
\texte{

}
\begin{enumerate}
    \item \question{\'Ecrire en termes d'$\varepsilon$ et $\delta$ la définition de la $\Cc$-dérivabilité d'une fonction $f : D\to \Cc$
 d'un ouvert $D$ de $\Cc$ en un point $a$ de $D$.}
    \item \question{Soit $n$ un entier naturel non nul. Soit $a$ un point de $\Cc$.
 Déterminer une fonction $f_1 : \Cc\to\Cc$ telle que sur $\Cc$,
 $$z^n=a^n+(z-a)f_1(z).$$
 En déduire que l'application $\Cc\to\Cc$, $z\mapsto z^n$ est $\Cc$-dérivable en $a$ et déterminer sa dérivée en $a$.}
    \item \question{Démontrer que si $f$ et $g$ sont deux applications d'un ouvert $D$ de $\Cc$ dans $\Cc$ qui sont $\Cc$-dérivables en $a$,
alors leur produit est $\Cc$-dérivable en $a$. Déterminer alors la dérivée du produit en $a$ à l'aide des dérivées en $a$ de $f$ et de $g$.}
    \item \question{Donner un exemple d'application $f :\Cc\to\Cc$ continue mais nulle part $\Cc$-dérivable.}
\end{enumerate}
}

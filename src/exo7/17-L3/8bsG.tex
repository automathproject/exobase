\uuid{8bsG}
\exo7id{2218}
\auteur{matos}
\organisation{exo7}
\datecreate{2008-04-23}
\isIndication{false}
\isCorrection{true}
\chapitre{Autre}
\sousChapitre{Autre}

\contenu{
\texte{
Montrer que, pour $A\in\Cc^{n\times m}$
}
\begin{enumerate}
    \item \question{$\|A\|_2=\sigma_1$, la plus grande valeur singuli\`ere de $A$}
\reponse{$\|A\|_2=\|U\Sigma V^*\|_2=\|\Sigma\|_2=\max |\sigma_j|=\sigma_1$}
    \item \question{$\|A\|_F=\sqrt{\sigma_1^2 +\sigma_2^2 +\cdots +\sigma_r^2}$ o\`u les $\sigma_i$ sont les valeurs singuli\`eres de $A$.}
\reponse{$\|A\|^2_F =$tr$(A^*A)=$tr$(U^*A^*AU)=\|AU\|^2_F=$tr$(A^*U^*UA)=\|UA\|^2_F$ et donc
$$\|A\|_F=\|U\Sigma V^*\|_F =\|\Sigma\|_F=\sqrt{\sigma_1^2 + \cdots +\sigma_r^2}$$}
    \item \question{Les valeurs singuli\`eres non nulles de $A$ sont les racines carr\'es des valeurs propres non nulles de $A^*A$ et $AA^*$.}
\reponse{$A^*A=(V\Sigma^*U^*) (U\Sigma V^*)=V(\Sigma^*\Sigma )V^*$ et donc $A^*A$ est semblable \`a $\Sigma^*\Sigma$, les deux matrices ont donc les m\^emes valeurs propres. Les valeurs propres de $\Sigma^*\Sigma$ sont $\sigma_1^2, \cdots , \sigma_r^2$ plus $n-r$ valeurs propres nulles si $n>r$.}
    \item \question{pour $A\in\Cc^{m\times m}$ , $|\det (A)|=\prod_{i=1}^m \sigma_i$.}
\reponse{$|\det A|=|\det (U\Sigma V^*)|= |\det U|\det|\Sigma||\det V^*|= |\det \Sigma|=\prod_{i=1}^r \sigma_i$}
    \item \question{Si $A=A^*$ alors les valeurs singuli\`eres de $A$ sont les valeurs absolues des valeurs propres de $A$}
\reponse{Une matrice hermitienne \'etant diagonalisable a une base orthonormale de vecteurs propres
$$A=Q\Lambda Q^* = Q |\Lambda|\mbox{sign} (\Lambda)Q^*$$
or $U=$sign$(\Lambda )Q^*$ est une matrice unitaire: $U^*U=Q$sign$(\Lambda) $sign$(\Lambda) Q^*= QQ^*=I$. Donc $Q |\Lambda|U$ est une d\'ecomposition en valeurs singuli\`eres de $A$, les valeurs singuli\`eres \'etant $|\lambda_1|,\cdots , |\lambda_n|$.}
\end{enumerate}
}

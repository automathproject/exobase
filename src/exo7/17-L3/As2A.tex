\uuid{As2A}
\exo7id{2235}
\titre{exo7 2235}
\auteur{matos}
\organisation{exo7}
\datecreate{2008-04-23}
\isIndication{false}
\isCorrection{false}
\chapitre{Autre}
\sousChapitre{Autre}
\module{Analyse numérique}
\niveau{L3}
\difficulte{}

\contenu{
\texte{
Soit $a\in\Rr$ et $A=\left( \begin{array}{ccc}
1&a&a\\ a&1&a\\ a&a&1\end{array}\right)$
}
\begin{enumerate}
    \item \question{Pour qu'elles valeurs de $a$ $A$ est--elle d\'efinie positive?}
    \item \question{Pour qu'elles valeurs de $a$ la m\'ethode de Gauss--Seidel est--elle convergente?}
    \item \question{Ecrire la matrice $J$ de l'it\'eration de Jacobi.}
    \item \question{Pour qu'elles valeurs de $a$ la m\'ethode de Jacobi converge--t--elle?}
    \item \question{Ecrire la matrice ${\cal L}_1$ de l'it\'eration de Gauss--Seidel. Calculer $\rho ({\cal L}_1)$.}
    \item \question{Pour quelles valeurs de $a$ la m\'ethode de Gauss--Seidel converge--t--elle plus vite que celle de Jacobi?}
\end{enumerate}
}

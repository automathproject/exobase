\uuid{toom}
\exo7id{2225}
\titre{exo7 2225}
\auteur{matos}
\organisation{exo7}
\datecreate{2008-04-23}
\isIndication{false}
\isCorrection{true}
\chapitre{Autre}
\sousChapitre{Autre}
\module{Analyse numérique}
\niveau{L3}
\difficulte{}

\contenu{
\texte{
Soit $A$ une matrice sym\'etrique inversible admettant une factorisation LU. Montrer que l'on peut \'ecrire $A$ sous la forme 
$$A=B\tilde{B}^T  \mbox{ o\`u }$$ 
\begin{itemize} 
\item $B$ est une matrice triangulaire inf\'erieure ;
\item $\tilde{B}$ est une matrice o\`u chaque colonne est soit \'egale \`a la colonne correspondante de $B$, soit \'egale \`a la colonne correspondante de $B$ chang\'ee de signe. 
\end{itemize} 
\emph{Application num\'erique} 
$$A=\left(\begin{array}{cccc} 
1&2&1&1\\ 
2&3&4&3\\ 
1&4&-4&0\\ 
1&3&0&0 
\end{array}\right).$$
}
\reponse{
Soit $LU$ la factorisation $LU$ de $A$. On va intercaler dans cette factorisation la matrice r\'eelle $\Lambda=$diag$(\sqrt{|u_{ii}|})$.


$A=(L\Lambda ) (\Lambda^{-1}U) =BC$. La sym\'etrie de $A$ entraine $BC=C^TB^T. $ On a 

$C(B^T)^{-1}$ matrice triangulaire sup\'erieure, $B^{-1}C^T$ matrice triangulaire inf\'erieure et $C(B^T)^{-1}=B^{-1}C^T$ et donc

$C(B^T)^{-1}=B^{-1}C^=$diag(sign$(u_{ii})=S$ $\Rightarrow C(B^T)^{-1}S^{-1}=I=S^{-1}B^{-1}C^T \Leftrightarrow C^T=BS=\tilde{B}$. Donc $A$ peut \^etre mise sous la forme
$$A=B\tilde{B}^T \mbox{ avec } \tilde{B}=BS$$
i.e. la $i$-\`eme colonne de $\tilde{B}$ est \'egale \`a la $i$-\`eme colonne de $B$ affect\'ee du signe de $u_{ii}$

Application num\'erique:
$$\tilde{B}=\left(\begin{array}{cccc}
1&2&1&1\\
&-1&2&1\\
&&-1&-1\\&&&1\end{array}\right).$$
}
}

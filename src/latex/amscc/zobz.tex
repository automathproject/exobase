\uuid{zobz}


\titre{Maximum de vraisemblance et tests}
\theme{Statistique}
\auteur{}
\organisation{AMSCC}
\contenu{


\texte{


On observe un échantillon \( X_1, \dots, X_n \) dont la loi admet
la densité
$$
f_\theta(x) = \exp(- (x - \theta)) \mathrm{1}_{[\theta, +\infty[}(x),
$$
où \( \theta \) est un paramètre réel inconnu.}

\begin{enumerate}
  \item \question{Quels estimateurs de \( \theta \) pouvez-vous proposer en utilisant les méthodes usuelles ?}
  \item \question{Déterminer la loi de \( n(\hat{\theta}_n - \theta) \), où \( \hat{\theta}_n \) est l’estimateur du maximum de vraisemblance. En déduire un intervalle de confiance pour \( \theta \) de niveau \( 1 - \alpha \).}
  \item \question{Soit \( \alpha \in ]0, 1[ \). On souhaite tester au niveau \( \alpha \):
    $$ H_0:\,\, \theta \geq 0\,\,\, \text{contre} \,\,\, H_1:\,\, \theta < 0 .$$
    
    \begin{enumerate}
      \item \question{Construire un test à partir de l’intervalle de confiance de la question 2, calculer sa puissance et donner son allure (pour \( n \) et \( \alpha \) fixés). Quelle est sa taille \( \alpha^* \) ?}
      \item \question{Proposer un autre test qui soit, lui, de taille \( \alpha \).}
      \item \question{Calculer la fonction puissance du test. La représenter en fonction de \( \theta \) pour \( n \) et \( \alpha \) fixés.}
      \item \question{Comment varie la puissance en fonction de \( \alpha \) ? en fonction de \( n \) ?}
    \end{enumerate}
  }
  \item \question{Proposer un test de niveau \( \alpha \) de \( H_0 : \) « La loi de \( X_1 \) est une loi exponentielle » contre \( H_1 : \) « La loi de \( X_1 \) n’est pas une loi exponentielle ». Calculer la puissance de ce test.}
\end{enumerate}
}
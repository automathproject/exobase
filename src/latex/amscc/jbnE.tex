\uuid{jbnE}
\chapitre{Probabilité continue}
\niveau{L2}
\module{Probabilité et statistique}
\sousChapitre{Densité de probabilité}
\titre{Loi d'une somme de variables aléatoires, question de cours}
\theme{variables aléatoires à densité}
\auteur{}
\datecreate{2022-11-15}
\organisation{AMSCC}
\difficulte{}
\contenu{


\texte{ Soient $X$ et $Y$ deux variables aléatoires indépendantes suivant chacune une loi uniforme sur $[0;1]$.  } 

\question{ Déterminer la loi de $S=X+Y$. }

\reponse{ Si $X$ et $Y$ suivent chacune une loi uniforme sur $[0;1]$, alors $S=X+Y$ admet une densité $h$ définie par 
	$$h(s) = \int_0^1 \textbf{1}_{[0;1]}(s-x) dx = \int_{s-1}^s \textbf{1}_{[0;1]}(u)du$$
	\begin{itemize}
		\item si $s < 0$ alors $s-x <0$ et $h(s) = 0$
		\item si $0<s<1$ alors $h(s) = \int_{0}^s \textbf{1}_{[0;1]}(u)du = s$
		\item si $1<s<2$ alors $h(s) = \int_{s-1}^1 \textbf{1}_{[0;1]}(u)du = 2-s$
		\item si $2 < s$ alors $h(s) = 0$
	\end{itemize}
	En traçant cette densité, on voit apparaître une loi triangulaire. }}

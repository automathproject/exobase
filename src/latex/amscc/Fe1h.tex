\uuid{Fe1h}
\chapitre{Déterminant, système linéaire}
\sousChapitre{Calcul de déterminants}
\titre{Déterminant de Vandermonde}
\theme{calcul déterminant}
\auteur{}
\datecreate{2023-01-14}
\organisation{AMSCC}
\contenu{

\question{ Montrer que :
$$
\left|\begin{array}{lll}
	1 & 1 & 1 \\
	a & b & c \\
	a^2 & b^2 & c^2
\end{array}\right|=(c-a)(b-a)(c-b)
$$ }

\reponse{ 

$$
\begin{aligned}
	& \left|\begin{array}{ccc}
		1 & 1 & 1 \\
		a & b & c \\
		a^2 & b^2 & c^2
	\end{array}\right|=\left|\begin{array}{ccc}
		\mathcal{c}_1 & \mathcal{c}_2-\mathcal{c}_1 & \mathcal{c}_3-\mathcal{c}_1 \\
		a & 0 & 0 \\
		a^2 & b^2-a^2 & c^2-a^2
	\end{array}\right| \\
	& =(b-a)(c-a)\left|\begin{array}{ccc}
		c_1 & c_2 & \mathcal{c}_3-c_2 \\
		1 & 0 & 0 \\
		a & 1 & 0 \\
		a^2 & b+a & c-b
	\end{array}\right| \\
	& =(b-a)(c-a)(c-b) \\
	&
\end{aligned}
$$ }}

\uuid{b1T2}
\chapitre{Résolution de systèmes linéaires : méthode itérative}
\sousChapitre{Résolution de systèmes linéaires : méthode itérative}
\titre{Etude d'une matrice tridiagonale}
\theme{analyse numérique}
\auteur{}
\datecreate{2023-03-02}
\organisation{AMSCC}
\contenu{

\texte{ 
Soit $A \in \mathcal{M}_{n}(\mathbb{R})$ la matrice tridiagonale d'ordre $n > 2$ définie par 
$$A = \begin{pmatrix}
	2 & -1 &  &  & & \\
	-1 & 2 & -1 &      & &\\
	&   \ddots & \ddots & \ddots  && \\
	& & -1 &2 &-1 \\
	& & & -1 & 2 \\
\end{pmatrix}
$$

On admet que l'ensemble des valeurs propres de $A$ est $$sp(A) = \left\{ \lambda_k = 4 \sin^2\left(\frac{k\pi}{2(n+1)} \right) \, , \, k \in \{1,...,n \} \right\}$$

On souhaite résoudre un système linéaire $AX=b$ à l'aide d'une méthode itérative et on note $X$ sa solution. }

\begin{enumerate}
	\item  \question{ Exprimer la suite des itérés de la méthode de Jacobi sous la forme $X^{(k+1)}=BX^{(k)}+C$ en exprimant la matrice $B$ en fonction de la matrice identité $I_n$ et de la matrice $A$. La matrice $A$ est-elle à diagonale strictement dominante ? }
	\reponse{Avec les notations du cours, $A=M-N$ avec $M=2I$ d'où la suite de Jacobi $x^{(k+1)}=M^{-1}Nx^{(k)}+M^{-1}b = (I-\frac{1}{2}A) x^{(k)} + \frac{1}{2}b$.
		La matrice $A$ n'est pas à diagonale strictement dominante donc la convergence de la méthode de Jacobi n'est pas acquise.	
	}	
	
	\item  \question{ On définit l'erreur $e^{(k)}=X^{(k)}-X$ à la $k$-ème itération. Exprimer $e^{(k)}$ en fonction de $e^{(k-1)}$ et en déduire que $\Vert e^{(k)}\Vert \leq \vvvert B\vvvert^k\Vert e^{(0)}\Vert$ où $\Vert.\Vert$ est une norme quelconque sur $\mathbb{R}^{n}$ et $\vvvert.\vvvert$ la norme induite sur $\mathcal{M}_{n}(\mathbb{R})$. }
	\reponse{
		\begin{align*}
			e^{(k)} &= X^{k}-X\\
			&=(BX^{k-1}+ C) - (BX+C) \\
			&= B(X^{(k-1)}-X) \\
			&=B e^{(k-1)} 
		\end{align*}
		donc par récurrence $e^{(k)}=B^ke^{(0)}$.	En passant à la norme et par inégalité des normes induites, $|| e^{(k)}|| \leq \vvvert B\vvvert^k || e^{(0)}||$
	}
	\item \question{ Calculer $\vvvert B\vvvert_{\infty}$. Qu'en conclure ? }
	\reponse{On calcule $B = \begin{pmatrix}
			0 & \frac{1}{2} &  &  & & \\
			\frac{1}{2} & 0 & \frac{1}{2} &      & &\\
			&   \ddots & \ddots & \ddots  && \\
			& & \frac{1}{2} &0 &\frac{1}{2} \\
			& & & \frac{1}{2} & 0 \\
		\end{pmatrix}
		$ D'après le cours,   $\displaystyle\vvvert B\vvvert_{\infty} =  \sup\limits_{1 \leq i \leq N} \sum_{j=1}^N|b_{i,j}| = \frac{1}{2}+\frac{1}{2} = 1$. Donc ce choix de norme ne permet pas de conclure que l'erreur tend vers $0$.
	}
	
	\item \question{ Vérifier que le rayon spectral de $B$ est  $\rho(B)=\cos\left(\frac{\pi}{n+1}\right)$. }
	\reponse{Les valeurs propres de $B=I-\frac{1}{2}A$ sont les valeurs $\mu_k = 1-2 \sin^2\left(\frac{k\pi}{2(n+1)}\right) = \cos\left(\frac{k\pi}{n+1}\right)$ donc $\rho(B) = \max\{|\mu_k|\} = \cos\left(\frac{\pi}{n+1}\right)$.} 		
	
	\item \question{ En déduire que la méthode de Jacobi converge pour la matrice $A$ quelque soit l'initialisation. }
	\reponse{
		On remarque que $A$ est une matrice symétrique donc $B$ est une matrice symétrique. D'après la propriété 1 admise, $||B||_2 =\rho(B) =  \cos\left(\frac{\pi}{n+1}\right)$ donc $0<||B||_2<1$. Donc d'après la question 2, $|| e^{(k)}||_2 \to 0$ quelque soit l'erreur $e^{(0)}$.
		commise au départ, autrement dit la méthode converge.
	}	
	
\end{enumerate}}

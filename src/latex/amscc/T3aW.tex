\uuid{T3aW}
\titre{ Séries : vrai ou faux }
\theme{séries}
\auteur{ Quentin Liard }
\datecreate{ 2025-05-05 }
\organisation{AMSCC}

\contenu{
	
	\texte{ Vrai ou Faux ? Justifier. 
	}
	
	\begin{enumerate}
		\item \question{ Si pour tout $n \in \N^*$, $u_n = \frac{4}{5^n}$ alors la série $\displaystyle \sum_{n \geq 1} u_n$ converge et sa somme vaut $\displaystyle \sum_{n = 1}^{+\infty} u_n = 5$.   }
		\reponse{ FAUX. Il s'agit d'une série géométrique de raison $q=\frac{1}{5}$ convergente, en revanche sa somme vaut $\displaystyle \sum_{n = 1}^{+\infty} u_n = \sum_{n = 0}^{+\infty} u_n - u_0 = 5 - 4 = 1$.  }
%		\item \question{Si pour tout $n \in \N^*$, $u_n = \ln\left(1+\frac{1}{n}\right)^{\frac{1}{n}}$ alors la série $\displaystyle \sum_{n \geq 1} u_n$ converge. }
%		\indication{}
%		\reponse{}
		\item \question{Soit $(u_n)$ une suite strictement positive telle que $u_n = \frac{1}{n} + \underset{+\infty}{o}\left(\frac{1}{n}\right)$ alors la série $\displaystyle \sum_{n \geq 1} u_n$ diverge. }
		\indication{}
		\reponse{VRAI. Le terme général $u_n$ est positif et équivalent $\frac{1}{n}$ quand $n \to +\infty$. Par comparaison à une série de Riemann divergente, la série $\sum u_n$ diverge. }
%		\item \question{ Si pour tout $n \in \N^*$, $u_n = \frac{1}{n^2}$ alors la série $\displaystyle \sum_{n \geq 1} u_n$ converge et sa somme vaut $\frac{5}{4}$. }
%		\item \question{ Si pour tout $n \in \N$, $u_n>0$ et   $\displaystyle \sum_{n \geq 0} u_n$ converge alors la série $\displaystyle \sum_{n \geq 0} \frac{u_n}{1+n^2u_n}$ converge. }
	\end{enumerate}
	
}
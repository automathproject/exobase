\uuid{64pg}
\chapitre{Statistique}
\niveau{L2}
\module{Probabilité et statistique}
\sousChapitre{Estimation}
\titre{Sondage et confidentialité}
\theme{variables aléatoires, estimateurs, intervalle de confiance}
\auteur{}
\datecreate{2023-01-11}
\organisation{AMSCC}
\difficulte{}
\contenu{


\texte{ Pour préserver la confidentialité des opinions individuelles, un sondage est effectué avec le protocole suivant. Chaque personne sondée doit, avant de répondre <<oui>> ou <<non>> à la question posée, réaliser confidentiellement (elle seule connait le résultat) une variable de Bernoulli de paramètre $a$, $a \in ]0;1[$ donné et connu du sondeur. Si le résultat est $1$, la personne doit répondre à la question selon son avis. Si le résultat est $0$, la personne doit répondre à la question selon le contraire de son avis. 
	
	On suppose que les personnes sondées suivent parfaitement ce protocole et l'on note $X_i$ la variable aléatoire qui vaut $1$ si la $i$-ème personne sondée répond <<oui>> et $0$ si elle répond <<non>>. 
	
	On note $n$ le nombre de personnes sondées et $p$ la proportion de personnes dont l'opinion personnelle est <<oui>> dans la population sondée. Soit $q$ la probabilité qu'une personne sondée réponde <<oui>>. Enfin, on suppose que $a \neq \frac{1}{2}$. }
	
	\begin{enumerate}
		\item \question{ Exprimer $q$ en fonction de $p$ et $a$. }
		\reponse{D'après l'énoncé de la situation et le théorème des probabilités totales, $q=ap+(1-a)(1-p)$.}
		\item Vérifier que $\overline{X} = \frac{1}{n} \sum_{i=1}^n X_i$ est un estimateur de $q$ sans biais et convergent.
		\reponse{ Chaque variable $X_i$ suit une loi $\mathcal{B}(q)$ donc $\mathbb{E}(\overline{X})=q$ : $\overline{X}$ est un estimateur sais biais de $q$. Et $V(\overline{X}) = \frac{q(1-q)}{n} \to 0$ donc $\overline{X}$ est un estimateur convergent.}
		\item \question{ On pose $$T_n = \frac{\overline{X}-1+a}{2a-1}$$
		Calculer l'espérance et la variance de $T_n$. }
		\reponse{Par linéarité et d'après le calcul précédent, on trouve $\mathbb{E}(T_n)=\frac{q-1+a}{2a-1}=p$. D'après les règles de calcul pour la variance, 
			$$V(T_n) = \frac{1}{(a-1)^2}V(\overline{X}-1+a) = \frac{1}{(2a-1)^2}V(\overline{X}) = \frac{q(q-1)}{n(2a-1)^2}$$
			Or en développant le calcul, on trouve $q(q-1) = p(1-p)(2a-1)^2+a(1-a)$ donc $V(T_n) = \frac{p(1-p)}{n}+\frac{a(1-a)}{n(2a-1)^2}$}
		\item \question{ Démontrer que $T_n$ est un estimateur de $p$ sans biais et convergent. }
		\reponse{Cela permet de voir que $B(T_n) = p-p=0$ et $V(T_n) \to 0$ quand $n \to +\infty$.}
		\item \question{ Justifier que la variable $\frac{T_n-p}{\sigma(T_n)}$ converge en loi vers la loi normale centrée réduite. }
		\reponse{Pour pouvoir appliquer le théorème central limite, il suffit de vérifier que $T_n$ s'écrit bien sous la forme d'une somme de variables aléatoires indépendantes et identiquement distribuées. En effet, on a 
			$$T_n = \frac{\frac{1}{n} \left( \sum_{i=1}^n  X_i \right) -1+a}{2a-1} = \frac{ \frac{1}{n} \sum_{i=1}^n  \left( X_i  -1+a \right)}{2a-1} = \sum_{i=1}^n \frac{X_i  -1+a}{n(2a-1)} $$
			Les variables $X_i$ étant i.i.d, les hypothèses sont bien vérifiées et d'après le théorème central limite, la variable $\frac{T_n-\mathbb{E}(T_n)}{\sigma(T_n)}$ converge bien en loi vers une variable suivant une loi normale centrée réduite.}
		\item \question{ Donner une estimation de $p$ par intervalle de confiance au niveau de confiance $1-\alpha$ que l'on notera $I_\alpha$. }
		\reponse{D'après la question précédente, on peut (en supposant $n$ grand) approcher la variable $\frac{T_n-\mathbb{E}(T_n)}{\sigma(T_n)}$ par une variable $Z$ qui suit une loi $\mathcal{N}(0,1)$. Ensuite, on construit facilement un intervalle de confiance pour $p=\mathbb{E}(T_n)$ au niveau $1-\alpha$ en choisissant dans la table la valeur $u_{\alpha/2}$ permettant d'avoir 
			$$\PP(-u_{\alpha/2} < Z < u_{\alpha/2}) = 1-\alpha$$
			puis en redéployant l'inégalité autour de $p$ :
			$$\PP(T_n-u_{\alpha/2} \sigma(T_n) < \mathbb{E}(T_n) < T_n + u_{\alpha/2} \sigma(T_n)) = 1-\alpha $$
			D'où l'intervalle de confiance pour $p = \mathbb{E}(T_n)$ : 
			$$I_conf = [ T_n - u_{\alpha/2} \sigma(T_n) \,;\,  T_n + u_{\alpha/2} \sigma(T_n)]$$
			Pour réaliser cet intervalle, il suffit de réaliser l'échantillon et de remplacer $ T_n$ et $\sigma( T_n)$ par leurs réalisations. On obtiendra un intervalle par excès en replaçant $\sigma(T_n)$ par un majorant $\frac{1/2}{\sqrt{n(2a-1)^2}}$}
		\item \question{  On fixe $a = \frac{1}{6}$, $n=1000$ et on considère une réalisation de la variable aléatoire $\overline{X}$ égale à $0{,}425$. Déterminer une réalisation de l'intervalle de confiance utilisé pour l'estimation de $p$ au risque $\alpha = 0{,}05$ (on pourra utiliser la majoration $q(1-q) \leq \frac{1}{4}$).  }
		\reponse{Application numérique : on trouve $\alpha = 0.05$, on utilise $u_{\alpha/2} = 1.96$ et on trouve 
			$$I = [0.566 \,;\, 0.658]$$ }
	\end{enumerate}
}

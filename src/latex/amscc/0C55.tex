\uuid{0C55}
\titre{Étude des troubles musculo-squelettiques (TMS) chez les militaires}
\theme{statistiques, tests d'hypothèses}
\auteur{Maxime NGUYEN}
\datecreate{2024-12-10}
\organisation{AMSCC}

\contenu{

\texte{
On s’intéresse aux troubles musculo-squelettiques (TMS) chez le combattant de quatre régiments de l’Armée de Terre (sources : \url{https://hal.univ-lorraine.fr/hal-01734435}).
}

\begin{enumerate}
	\item \question{Dans cette étude, 19 femmes sur 21 (soit 90,5\%) et 191 hommes sur 227 (84,1\%) déclarent avoir eu des TMS. En réalisant un test d’indépendance du $\chi^2$ avec un risque de première espèce de 5\%, dire si la différence de prévalence entre les femmes et les hommes est significative.}
	\indication{Utiliser la formule du $\chi^2$ pour comparer les fréquences observées et attendues.}
    \reponse{À compléter.}

    \item \question{Parmi les personnes ayant déclaré avoir eu des TMS, 1 n’a pas souhaité répondre et : 
    \begin{itemize}
        \item 10 femmes sur 19 ont été gênées par les TMS durant les 12 derniers mois (52,6\%),
        \item 142 hommes sur 190 (74,7\%) ont été gênés par les TMS durant la même période.
    \end{itemize}
    En réalisant un nouveau test d’indépendance du $\chi^2$ avec un risque de première espèce de 5\%, dire si les hommes sont significativement plus sensibles que les femmes aux TMS dans l’exercice de leurs fonctions.}
    \indication{Calculez le $\chi^2$ en tenant compte des effectifs observés et attendus pour chaque groupe.}
    \reponse{À compléter.}

    \item \question{Soit $f_1$ la proportion de femmes gênées par les TMS les 12 derniers mois et $f_2$ la proportion de femmes gênées par les TMS les 12 derniers mois. On souhaite tester l’hypothèse $H_0 : f_1 = f_2$ à partir de l’échantillon décrit à la question précédente et la variable de décision : 
    \[
    Z = \frac{F_1 - F_2}{\sqrt{P(1-P)\left(\frac{1}{n_1} + \frac{1}{n_2}\right)}}
    \]
    où $n_1$ est l’effectif de l’échantillon de femmes, $n_2$ est l’effectif de l’échantillon d’hommes, $F_1$ est la fréquence de femmes gênées, $F_2$ est la fréquence d’hommes gênés, et $P = \frac{n_1F_1 + n_2F_2}{n_1 + n_2}$. On admet que $Z$ suit une loi normale centrée réduite.

    En faisant un test de comparaison de proportions avec un risque de première espèce de 5\%, dire de nouveau si les hommes sont significativement plus sensibles que les femmes aux TMS dans l’exercice de leurs fonctions.}
    \indication{Utiliser la statistique $Z$ pour calculer la valeur critique et la comparer à la valeur obtenue.}
    \reponse{À compléter.}
\end{enumerate}

}

\uuid{r3ax}
\chapitre{Probabilité discrète}
\niveau{L2}
\module{Probabilité et statistique}
\sousChapitre{Variable aléatoire discrète}
\titre{Identification de lois usuelles}
\theme{variables aléatoires discrètes}
\auteur{ }
\datecreate{2023-08-30}
\organisation{AMSCC}
%Proba360

\difficulte{2}
\contenu{
	\texte{ Pour chacune des situations suivantes, déterminer la loi de la variable aléatoire $X$ et donner son espérance.
	}
\begin{enumerate}
	\item \question{ On lance un dé équilibré 100 fois de suite et on note $X$ le nombre d'apparitions du chiffre \og 2 \fg{}.  }
	\reponse{ La variable aléatoire $X$ suit une loi binomiale de paramètres $n=100$ et $p=\frac{1}{6}$. Son espérance est $\E(X)=n\times p = \frac{50}{3}$.  }
	\item \texte{ Deux personnes lancent chacune une pièce équilibrée. On dit que l'expérience est un succès si elles obtiennent toutes les deux \og face \fg{}.  }
	\begin{enumerate}
		\item \question{ Ces personnes répètent l'expérience 8 fois. On note $X$ le nombre de succès. }
		\reponse{ La variable aléatoire $X$ suit une loi binomiale de paramètres $n=8$ et $p=\frac{1}{4}$. Son espérance est $\E(X)=n\times p = 2$. }
		\item \question{ Ces personnes répètent l'expérience jusqu'à ce qu'elles obtiennent un succès. On note $X$ le nombre de lancers. } 
		\reponse{ La variable aléatoire $X$ suit une loi géométrique de paramètre $p=\frac{1}{4}$. Son espérance est $\E(X) = \frac{1}{p} = {4}$. }
	\end{enumerate}
   \item \texte{ Une agence de location de voitures propose à ses clients 3 catégories de voitures : $A$, $B$ et $C$. Elle a constaté que dans une journée, $40\%$ des demandes sont pour $A$, $50\%$ des demandes sont pour $B$, $10\%$ des demandes sont pour $C$. Les demandes de location sont supposées indépendantes.  } 
   \question{ Un jour donné, l'agence a reçu 12 demandes de location. On note $X$ le nombre de catégories $A$ demandées. }
   \reponse{ La variable aléatoire $X$ suit une loi binomiale de paramètres $n=12$ et $p=0.4$. Son espérance est $\E(X) = 4{,}8$. }
\end{enumerate}
}

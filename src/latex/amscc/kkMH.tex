\chapitre{Probabilité continue}
\sousChapitre{Loi normale}
\uuid{kkMH}
\titre{Densité d'une somme}
\theme{variables aléatoires à densité, loi normale}
\auteur{}
\datecreate{2022-11-15}
\organisation{AMSCC}
\contenu{
	

\texte{ 	Soient $X$ et $Y$ deux variables aléatoires indépendantes suivant chacune une loi $\mathcal{N}(0,1)$. } 
 \question{ Déterminer  la loi de $S=X+Y$. }

\reponse{	Si $X$ et $Y$ suivent chacune une loi $\mathcal{N}(0,1)$, alors $S=X+Y$ admet une densité $h$ définie par
	$$h(s) = \frac{1}{2\pi}\int_{\R}^{} e^{-\frac{(s-x)^2}{2}} e^{-\frac{x^2}{2}} dx$$
	
	Or $-(s-x)^2-x^2 = -\frac{s^2}{2} - 2 (x-\frac{s}{2})^2$ donc 
	
	$$h(s) = \frac{1}{2\pi} e^{-\frac{s^2}{4}} \int_{\R}^{} e^{-(x-\frac{s}{2})^2 } dx =  \frac{1}{2\pi} e^{-\frac{s^2}{4}} \times \sqrt{\pi} = \frac{1}{\sqrt{2}\sqrt{2\pi}}e^{-\frac{s^2}{2\sqrt{2}^2}}$$
	
	Donc $S$ suit une loi normale de moyenne $\mu=0$ et d'écart-type $\sigma=\sqrt{2}$.
}
}
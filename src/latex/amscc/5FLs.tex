\chapitre{Statistique}
\sousChapitre{Tests d'hypothèses, intervalle de confiance}
\uuid{5FLs}
\titre{Test d'indépendance}
\theme{statistiques, tests d'hypothèses}	
\auteur{}
\datecreate{2022-09-28}
\organisation{AMSCC}
\contenu{

\texte{ On veut étudier la liaison entre les caractères : 
	<<être fumeur>> (plus de 20 cigarettes par
	jour, pendant 10 ans) et <<avoir un cancer de la gorge>>, 
	sur une population de 1000 personnes, dont 500 sont
	atteintes d'un cancer de la gorge. Voici les résultats observés:
	
	
	\begin{center}
		\begin{tabular}{c|cc|c}
			\emph{Observé} & cancer & non cancer & TOTAL \\ 
			\hline
			\text{fumeur} & 342 & 258 & 600 \\ 
			\text{non fumeur} & 158 & 242 & 400 \\ 
			\hline
			\text{TOTAL} & 500 & 500 & 1000
		\end{tabular}
	\end{center} }
\question{ 	Faire un test d'indépendance pour établir la liaison entre ces caractères. }

\reponse{
	Mise en oeuvre du test:
	\begin{enumerate}
		\item On définit un risque: 5\%.
		Pour étudier la dépendance de ces caractères
		faisons l'hypothèse $H_{0}$ : <<les deux caractères sont indépendants 
		>> et voyons ce qui se passerait sous cette hypothèse.
		Notons les événements:
		\begin{itemize}
			\item $C$ : <<avoir un cancer dans la population observée>>
			\item $F$: <<être fumeur dans la population observée>>
		\end{itemize}
		Si les événements $F$ et $C$ sont indépendants, alors: $P(F\cap
		C)=P(F)\cdot P(C)$ et de même pour les trois autres possibilités: 
		$P(\overline{C}\cap F),$ $P(\overline{C}\cap \overline{F}),P(C\cap \overline{F})$, 
		quantités que l'on peut donc calculer sous $H_{0}$:
		
		$P(F)=\frac{600}{1000}$, $P(C)=\frac{500}{1000}$, $P(F) \cdot P(C)=\frac{3}{10}$,
		alors l'effectif théorique correspondant à la catégorie 
		<<fumeur et cancéreux>> est de $300$.
		\item On en déduit le tableau théorique sous $H_{0}$ :
		
		\begin{center}
			\begin{tabular}{cccc}
				\emph{Théorique} & cancer & non cancer & marge \\ 
				\text{fumeur} & 300 & 300 & 600 \\ 
				\text{non fumeur} & 200 & 200 & 400 \\ 
				\text{marge} & 500 & 500 & 1000
			\end{tabular}
		\end{center}
		
		\item On calcule alors la valeur de 
		$s=\sum_{i=1}^{i=4} \frac{(O_i-T_i)^{2}}{T_i}$ : on obtient : $s=34.73$.
		On a précisé le risque de \%, mais pour $\alpha =0,001$, on lit
		dans la table du khi-deux à un degré de liberté :
		$P[\chi ^{2}\geq 10.83]=0.001$ et le $\chi^{2}$ calculé est $34.73$!
		\item On décide de rejeter $H_{0}$.
		Ainsi, en rejetant l'hypothèse de l'indépendance des caractères
		<<être fumeur>> et <<avoir un cancer de la gorge>>, on a moins de
		une chance sur $1000$ de se tromper, puisque moins de un tableau possible
		sur mille conduit à un calcul de $\chi ^{2}$ plus grand que $10.83$ ;
		beaucoup moins sans doute, conduiraient à un calcul de $\chi ^{2}$ plus
		grand que $34.73$.
	\end{enumerate}
	
	
}}

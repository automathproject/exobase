\uuid{7JTd}
\chapitre{Probabilité continue}
\sousChapitre{Densité de probabilité}
\titre{Simulation et fonction de répartition}
\theme{variables aléatoires à densité, simulation de loi}
\auteur{}
\datecreate{2022-09-22}
\organisation{AMSCC}
\contenu{

\texte{ 	Soit $X$ une variable aléatoire et soit $F$ sa fonction de répartition. On suppose que $F$ est continue et strictement croissante sur $\mathbb{R}$. Soit la variable aléatoire $Y=F(X)$. } 

\question{ Démontrer que $Y$ suit une loi uniforme sur $[0;1]$. }
\reponse{
	On cherche la fonction de répartition de la variable aléatoire $Y$. Soit $t\in\mathbb{R}$. Par définition :
	\[ F_Y(t)=\mathbb{P}(Y\leq t)
	=\mathbb{P}(F(X)\leq t).
	\]
	Or la fonction de répartition $F$ prend ses valeurs dans $[0;1]$. On en déduit
	\begin{align*}
		F_Y(t)=\begin{cases}
			0 & \text{ si } t< 0 \\
			1 & \text{ si } t\geq 1
		\end{cases}
	\end{align*}
	Soit $t\in[0;1[$. Comme la fonction de répartition $F$ est croissante et supposée continue sur $\mathbb{R}$, il existe au moins un réel $t_0$ tel que $F(t_0)=t$. Alors on a
	\begin{align*}
		F_Y(t)&=\mathbb{P}(F(X)\leq t) \\
		&=\mathbb{P}(F(X)\leq F(t_0)) \\
		&=\mathbb{P}(X\leq t_0) \quad \text{ car $F$ est croissante sur $\mathbb{R}$} \\
		&=F(t_0) \\
		&=t.
	\end{align*}
	On a ainsi obtenu 
	\begin{align*}
		F_Y(t)=\begin{cases}
			0 & \text{ si } t< 0 \\
			t & \text{ si } t\in[0;1[ \\
			1 & \text{ si } t\geq 1
		\end{cases}
	\end{align*}
	qui correspond à la fonction de répartition de la loi uniforme sur $[0;1]$. Donc $Y\sim \mathcal{U}([0;1])$.
}}

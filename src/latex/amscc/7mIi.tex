\uuid{7mIi}
\titre{ Résolution d'un système linéaire }
\theme{matrices, systèmes linéaires}
\auteur{}
\datecreate{2024-01-31}
\organisation{AMSCC}

\contenu{
    \texte{
        Soit la matrice $M = \begin{pmatrix}
            0&1&2\\
            1&1&2\\
            0&2&3
            \end{pmatrix}$.
    }
\begin{enumerate}
    \item \question{ Montrer que $M$ est inversible et calculer $M^{-1}$. }
    \indication{ On pourra calculer le déterminant de $M$. }
    \reponse{
        On calcule le déterminant de $M$ :
        \begin{align*}
            \det(M) &= 0 \times \begin{vmatrix} 1&2\\2&3 \end{vmatrix} - 1 \times \begin{vmatrix} 1&2\\0&3 \end{vmatrix} + 2 \times \begin{vmatrix} 1&1\\0&2 \end{vmatrix} \\
            &= -1 \times (3-0) + 2 \times (2-0) = -3+4 = 1.
        \end{align*}
        Donc $M$ est inversible. On calcule $M^{-1}$ :
        \begin{align*}
            M^{-1} &= \frac{1}{\det(M)} \begin{pmatrix} 3&-2&1\\-2&1&0\\1&0&-1 \end{pmatrix} = \begin{pmatrix} 3&-2&1\\-2&1&0\\1&0&-1 \end{pmatrix}.
        \end{align*}
    }
    \item \question{ Résoudre le système linéaire suivant : 
        $$
        \left\{\begin{aligned}
        y+2z & =1 \\
        x+y+2z & =2 \\
        2y+3z & =3
        \end{aligned}\right.
        $$
    }
\end{enumerate}
}
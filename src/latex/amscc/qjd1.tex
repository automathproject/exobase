\uuid{qjd1}
\chapitre{Déterminant, système linéaire}
\sousChapitre{Calcul de déterminants}
\titre{Factoriser dans un déterminant}
\theme{calcul déterminant}
\auteur{}
\datecreate{2023-01-11}
\organisation{AMSCC}
\contenu{

\question{ Soit $P=\left(\begin{array}{ccc}a^2 & a b & a c \\ a b & b^2 & b c \\ a c & b c & c^2\end{array}\right)$. Justifier que $\operatorname{det} P=0$.  }

\reponse{ On remarque que l'on peut mettre :
- $a$ en facteur de la première ligne ;
- $b$ en facteur de la deuxième ligne ;
- $c$ en facteur de la troisième ligne ;
Ainsi :
$$
\operatorname{det} P=\left|\begin{array}{lll}
	a^2 & a b & a c \\
	a b & b^2 & b c \\
	a c & b c & c^2
\end{array}\right|=a\left|\begin{array}{ccc}
	a & b & c \\
	a b & b^2 & b c \\
	a c & b c & c^2
\end{array}\right|=a b\left|\begin{array}{ccc}
	a & b & c \\
	a & b & c \\
	a c & b c & c^2
\end{array}\right|=a b c\left|\begin{array}{lll}
	a & b & c \\
	a & b & c \\
	a & b & c
\end{array}\right|=0
$$
$\left|\begin{array}{lll}a & b & c \\ a & b & c \\ a & b & c\end{array}\right|=0$ }
}

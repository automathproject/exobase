\uuid{w4TW}
\titre{Maximum de vraisemblance}
\datecreate{2024-11-05}
\theme{maximum de vraisemblance, variables aléatoires à densité}
\contenu{
\texte{On considère une variable aléatoire $X$ suivant une loi exponentielle de paramètre $\theta > 0$. On rappelle que sa densité est donnée par :
$$f_\theta(x) = \begin{cases} \theta e^{-\theta x} & \text{ si } x \geq 0 \\ 0 & \text{ sinon. } \end{cases}$$
On dispose d'un échantillon $(X_1,...,X_n)$ de $n$ variables aléatoires indépendantes et de même loi que $X$. On cherche à estimer le paramètre $\theta$ par la méthode du maximum de vraisemblance.
}

\begin{enumerate}
    \item \question{Écrire la vraisemblance $L(\theta)$ de l'échantillon en fonction de $\theta$ et des observations $(x_1,...,x_n)$.}
    \reponse{
        Par indépendance des variables, la vraisemblance est le produit des densités :
        \begin{align*}
            L(\theta) &= \prod_{i=1}^n f_\theta(x_i) \\
            &= \prod_{i=1}^n \theta e^{-\theta x_i}\mathbf{1}_{x_i \geq 0} \\
            &= \theta^n e^{-\theta \sum_{i=1}^n x_i}\mathbf{1}_{\min(x_i) \geq 0}
        \end{align*}
    }

    \item \question{En déduire la log-vraisemblance $\ell(\theta)$ puis calculer sa dérivée $\ell'(\theta)$.}
    \reponse{
        La log-vraisemblance est :
        \begin{align*}
            \ell(\theta) &= \ln(L(\theta)) \\
            &= n\ln(\theta) - \theta \sum_{i=1}^n x_i
        \end{align*}
        
        Sa dérivée est :
        $$\ell'(\theta) = \frac{n}{\theta} - \sum_{i=1}^n x_i$$
    }

    \item \question{En déduire  l'estimateur du maximum de vraisemblance $\widehat{\theta}_n$ de $\theta$.}
    \reponse{
        L'équation $\ell'(\theta)=0$ donne :
        \begin{align*}
            \frac{n}{\theta} - \sum_{i=1}^n x_i &= 0 \\
            \frac{n}{\theta} &= \sum_{i=1}^n x_i \\
            \theta &= \frac{n}{\sum_{i=1}^n x_i}
        \end{align*}

        Donc $\widehat{\theta}_n = \frac{n}{\sum_{i=1}^n x_i} = \frac{1}{\overline{X_n}}$ où $\overline{X_n}$ est la moyenne empirique.
    }

    \item \question{Vérifier que $\widehat{\theta}_n$ est bien un maximum en étudiant le signe de $\ell''(\theta)$.}
    \reponse{
        On calcule la dérivée seconde :
        $$\ell''(\theta) = -\frac{n}{\theta^2}$$
        
        Cette dérivée seconde est toujours négative pour $\theta > 0$, donc $\widehat{\theta}_n$ correspond bien à un maximum.
    }

    \item \question{Montrer que $\frac{1}{\widehat{\theta}_n}$ est un estimateur sans biais de $\frac{1}{\theta}$.}
    \reponse{
        On a :
        \begin{align*}
            \E\left(\frac{1}{\widehat{\theta}_n}\right) &= \E\left(\overline{X_n}\right) \\
            &= \E\left(\frac{1}{n}\sum_{i=1}^n X_i\right) \\
            &= \frac{1}{n}\sum_{i=1}^n \E(X_i) \\
            &= \frac{1}{n} \times n \times \frac{1}{\theta} \\
            &= \frac{1}{\theta}
        \end{align*}
        
        Donc $\frac{1}{\widehat{\theta}_n}$ est un estimateur sans biais de $\frac{1}{\theta}$.
    }
\end{enumerate}
}

\uuid{dC7P}
\titre{ Calcul d'une somme de série entière}
\theme {séries entières}
\auteur{ }
\datecreate{2023-06-01}
\organisation{ AMSCC }


\contenu{
	\begin{enumerate}
	\item \question{ Déterminer le domaine de convergence de la série entière $\displaystyle \sum_{n\geq 0} \frac{(n+2)2^n}{(n+1)!} z^n$ à variable complexe. }
	\reponse{ 
		$R=+\infty$ et $D=\mathbb{C}$.
	}
	
	\item \question{ Rappeler ce que vaut $\displaystyle \sum_{n=0}^{+\infty} \frac{z^n}{n!}$. En déduire $\displaystyle \sum_{n=0}^{+\infty} \frac{z^n}{(n+1)!}$, puis $\displaystyle \sum_{n=0}^{+\infty} \frac{(n+2)2^n}{(n+1)!}z^n$. }
	\reponse{ 
		$$\forall z \in \R, \quad \sum_{n=0}^{+\infty} \frac{z^n}{n!}=e^z.$$
		Pour la première somme:
		$$ \forall z \in \R, \qquad \sum_{n=0}^{+\infty} \frac{z^n}{(n+1)!}
		= \begin{cases} \frac{e^z-1}{z} & \text{ si } z \neq 0 \\
		0 & \text{ sinon }
		\end{cases}
		$$
		Pour la deuxième somme:
		$$ \forall z \in \R, \qquad \sum_{n=0}^{+\infty} \frac{(n+2)2^n}{(n+1)!}z^n
		= \begin{cases} \frac{e^{2z}-1}{2z}+e^{2z} & \text{ si } z \neq 0 \\
		0 & \text{ sinon }
		\end{cases}
		$$	
	}	
\end{enumerate}
}
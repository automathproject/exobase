\uuid{vnJs}
\chapitre{Déterminant, système linéaire}
\niveau{L2}
\module{Algèbre}
\sousChapitre{Calcul de déterminants}
\titre{Calcul de déterminants de matrices triangulaires}
\theme{calcul déterminant}
\auteur{}
\datecreate{2023-01-11}
\organisation{AMSCC}
\difficulte{}
\contenu{

\question{ Calculer les déterminants suivants :
	$$
	\begin{vmatrix}
		1 & 2 \\
		2 & 3
	\end{vmatrix} \quad \begin{vmatrix}
		1 & 0 & 2 \\
		3 & -1 & 2 \\
		1 & -1 & 0
	\end{vmatrix} \quad \begin{vmatrix}
		1 & 1 & 5 \\
		2 & 1 & -1 \\
		3 & 2 & 5
	\end{vmatrix}$$
	$$\begin{vmatrix}
		1 & -1 & 2 \\
		-2 & 3 & 0 \\
		2 & 4 & 8
	\end{vmatrix} \quad \begin{vmatrix}
		2 & -54 & 120 & -40 \\
		0 & 3 & -56 & 132 \\
		0 & 0 & -5 & 212 \\
		0 & 0 & 0 & 4
	\end{vmatrix}
	$$
 }

\indication{ On peut réaliser des combinaisons invariantes pour le déterminant afin d'obtenir une matrice triangulaire, ou bien développer par rapport à une ligne ou une colonne. }

\reponse{ 
	$$
	\left|\begin{array}{ll}
		1 & 2 \\
		2 & 3
	\end{array}\right|=1 \times 3-2 \times 2=-1
	$$
	Développement par rapport à la première ligne :
	$$
	\left|\begin{array}{ccc}
		1 & 0 & 2 \\
		3 & -1 & 2 \\
		1 & -1 & 0
	\end{array}\right|=1 \times\left|\begin{array}{cc}
		-1 & 2 \\
		-1 & 0
	\end{array}\right|-0+2 \times\left|\begin{array}{cc}
		3 & -1 \\
		1 & -1
	\end{array}\right|=1 \times(2)-0+2 \times(-2)=-2
	$$
	On fait apparaitre des 0 dans la première colonne :
	$$
	\left|\begin{array}{ccc}
		1 & 1 & 5 \\
		2 & 1 & -1 \\
		3 & 2 & 5
	\end{array}\right|=\begin{gathered}
		\ell_1 \\
		\ell_3-3 \ell_1
	\end{gathered}\left|\begin{array}{ccc}
		1 & 1 & 5 \\
		0 & -1 & -11 \\
		0 & -1 & -10
	\end{array}\right|=\begin{gathered}
		\ell_1 \\
		\ell_2 \\
		\ell_3+\ell_2
	\end{gathered}\left|\begin{array}{ccc}
		1 & 1 & 5 \\
		0 & -1 & -11 \\
		0 & 0 & 1
	\end{array}\right|=1 \times(-1) \times 1=-1
	$$
	On fait apparaître des 0 dans la première colonne :
	$$
	\left|\begin{array}{ccc}
		1 & -1 & 2 \\
		-2 & 3 & 0 \\
		2 & 4 & 8
	\end{array}\right|=\begin{gathered}
		\ell_1 \\
		\ell_3-2 \ell_1 \\
		2
	\end{gathered}\left|\begin{array}{ccc}
		1 & -1 & 2 \\
		0 & 1 & 4 \\
		0 & 6 & 4
	\end{array}\right|=\begin{gathered}
		\ell_1 \\
		\ell_2 \\
		\ell_3-6 \ell_2
	\end{gathered}\left|\begin{array}{ccc}
		1 & -1 & 2 \\
		0 & 1 & 4 \\
		0 & 0 & -20
	\end{array}\right|=1 \times 1 \times(-20)=-20
	$$
	On fait apparaitre des 0 dans la deuxième ligne :
	$$
	\begin{aligned}
		\left|\begin{array}{cccc}
			\mathcal{c}_1 & \mathcal{c}_2 & c_3 & \mathcal{c}_4 \\
			-1 & 3 & 1 & -4 \\
			0 & 2 & 1 & 0 \\
			-5 & 1 & 3 & -1 \\
			-2 & 1
		\end{array}\right| & =\left|\begin{array}{cccc}
			\mathcal{c}_1+\mathcal{C}_3 & c_2-2 \mathcal{C}_3 & \mathcal{c}_3 & \mathcal{c}_4 \\
			0 & 1 & 1 & -4 \\
			0 & 0 & 1 & 0 \\
			3 & -2 & 3 & -1 \\
			-7 & 5 & -2 & 1
		\end{array}\right|=-0+0-1 \times\left|\begin{array}{ccc}
			3 & 1 & -4 \\
			3 & -2 & -1 \\
			-7 & 5 & 1
		\end{array}\right|+0 \\
		& =(-1) \times\left[3 \times\left|\begin{array}{cc}
			-2 & -1 \\
			5 & 1
		\end{array}\right|-1 \times\left|\begin{array}{cc}
			3 & -1 \\
			-7 & 1
		\end{array}\right|+(-4) \times\left|\begin{array}{cc}
			3 & -2 \\
			-7 & 5
		\end{array}\right|\right] \\
		& =-[3 \times 3-1 \times(-4)+(4) \times 1] \\
		& =-9
	\end{aligned}
	$$
	Le déterminant est triangulaire. Il est donc égal au produit des termes de la diagonale :
	
	$\left|\begin{array}{cccc}2 & -54 & 120 & -40 \\ 0 & 3 & -56 & 132 \\ 0 & 0 & -5 & 212 \\ 0 & 0 & 0 & 4\end{array}\right|=2 \times 3 \times(-5) \times 4=-120$
 }}

\uuid{WCGV}
\chapitre{Probabilité discrète}
\sousChapitre{Lois de distributions}
\titre{Fiabilité}
\theme{variables aléatoires discrètes, loi binomiale}
\auteur{}
\datecreate{2023-09-04}
\organisation{AMSCC}
\contenu{


\texte{ Une usine produit des circuits imprimés dont $1$\% sont défectueux. }
\begin{enumerate}
	\item \question{ On fait un prélèvement de $100$ circuits imprimés. Quelle est la probabilité qu'il n'y ait pas de circuit imprimé défectueux dans ce prélèvement ? }
	\reponse{Soit $X$ le nombre de circuits défectueux dans ce prélèvement.
		Alors $X\sim \mathcal{B}(100,0.01)$. La probabilité qu'il n'y ait pas de circuit défectueux est:
		\[ \prob(X=0)=\binom{100}{0}\times 0.01^{0}\times 0.99^{100}=0.99^{100} \simeq 0.366.\]
	}
	
	\item \question{ Quelle est la probabilité qu'il y ait deux circuits imprimés défectueux dans un prélèvement de $100$ circuits imprimés ? }
	\reponse{$\prob(X=2)=\binom{100}{2}\times 0.01^{2}\times 0.99^{98}\simeq 0.185$
	}
	
	\item \question{ On prélève les circuits imprimés un par un et on les teste immédiatement. Quelle est la probabilité que le premier circuit imprimé défectueux apparaisse au $100$\up{ème} tirage ? }
	\reponse{Soit $Y$ le rang du premier circuit imprimé défectueux. Alors $Y\sim \mathcal{G}(0.01)$ et on a:
		\[ \prob(Y=100)=0.99^{99}\times 0.01 \simeq 0.0037,\]
		soit environ $0.37$\% de chance de rencontrer le premier circuit défectueux au dernier tirage.
	}
	
	\item \question{ Sur $N=10\ 000$ circuits imprimés en stock dont $d=100$ sont défectueux, on en choisit $500$. Quelle est la probabilité que sur les $500$ prélevés il y en ait cinq défectueux ? }
	\reponse{ Soit $Z$ le nombre de circuits défectueux sur les $500$ prélevés. Alors $Z\sim \mathcal{H}(500,0.01,10\ 000)$ et on a
		\[ \prob(Z=5)=\frac{\binom{100}{5}\binom{9900}{495}}{\binom{10\ 000}{500}}.
		\]
		Quand le paramètre $N$ est grand (dans les faits, quand $\frac n N < 0,1$), on peut approcher une loi hypergéométrique $\mathcal{H}(n,p,N)$ par une loi binomiale $\mathcal{B}(n,p)$. Ici, on obtient une approximation de $\prob(X=5)$ par $\binom{500}{5}\times 0.01^5 \times 0.99^{495} \simeq 0.18$.
	}
	
\end{enumerate}
}
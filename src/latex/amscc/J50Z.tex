\uuid{J50Z}
\chapitre{Analyse numérique}
\sousChapitre{Autre}
\titre{Evolution d'une concentration chimique}
\theme{analyse numérique}
\auteur{}
\datecreate{2023-03-20}
\organisation{AMSCC}
\contenu{

\texte{ 
	L'évolution de la concentration de certaines réactions chimiques au cours du temps peut être décrite par l'équation différentielle :
	$$y'(t) = - \frac{1}{1+t^2}y(t)$$ }

\begin{enumerate}
	\item \question{ Sachant qu'à l'instant $t=0$, la concentration vaut $y(0)=5$, déterminer la concentration en $t=2$ à l'aide de la méthode d'Euler implicite avec un pas $h=0.5$. }
	\reponse{ 		Le schéma implicite s'écrit 
		$$y_{n+1} = y_n + hF(t_{n+1},y_{n+1})$$
		où $F(t,y) = - \frac{1}{1+t^2}y$.
		
		Après réécriture sous forme explicite, on obtient 
		$$y_{n+1}=\frac{y_n}{1+ \frac{h}{1+t_{n+1}^2} }$$
		Avec $h=0.5$, on a $t_n = \frac{n}{2}$.
		
		Après calculs, on trouve $y_4 = \frac{520}{231} \approx 2.25$ soit $y(2) \approx 2.25$. }
	\item \question{ Faire de même pour un pas de $h=10^{-3}$ et commenter la différence.  }
%	\begin{minted}{python}
%	y0 = 5
%	T = 2
%	h = 0.5
%	X = arange(0, T + h, h)
%	\end{minted}
	\item \question{ Résoudre analytiquement l'équation différentielle, puis proposer sur un même graphique la solution approchée de la méthode d'Euler implicite et la solution exacte pour différentes valeurs de $h$. }
	\reponse{ \insertnotebook{J50Z} }
\end{enumerate}
}

\uuid{9oDf}
\chapitre{Série entière}
\sousChapitre{Développement en série entière}
\titre{Développement en série entière d'une fonction}
\theme{séries entières}
\auteur{}
\datecreate{2024-06-17}
\organisation{AMSCC}	

\contenu{

\texte{ Soit la fonction d'une variable réelle $f \colon x \mapsto e^{x^2}-1$.  }
%\begin{enumerate}
%	\item \question{ Rappeler le développement en série entière de la fonction $x \mapsto e^x$ et donner, sans justifier, son rayon de convergence. }
%	\reponse {On sait que $R=+\infty$ et que pour tout $x \in \R$, 
%		$$e^x = \sum_{n=0}^{+\infty} \frac{x^n}{n!}$$}
	%\item 
	\question{ Déterminer un développement en série entière de la fonction $f$ et préciser le rayon de convergence. }
	\reponse{On en déduit en subtistuant $x$ par $x^2$ que pour tout $x \in \R$,
		$$f(x) = \sum_{n=0}^{+\infty} \frac{x^{2n}}{n!}-1 = \sum_{n=1}^{+\infty} \frac{x^{2n}}{n!}$$}
%\end{enumerate}
}

\uuid{c4R1}
\titre{Probabilités conditionnelles et prise de décision}
\niveau{L1}

\module{Probabilité et statistique}
\chapitre{Probabilité discrète}
\sousChapitre{Probabilité conditionnelle}
\theme{Probabilité conditionnelle, théorème de Bayes, décision}
\auteur{}
\datecreate{2025-09-24}
\organisation{}
\difficulte{3}
\contenu{
	\texte{
		Vous êtes un mage perdu dans les marais maudits et une créature des ténèbres arrive. Il vous faut un certain temps pour invoquer un sort d'attaque et celui-ci dépend de la créature. Vous savez que 80\% des créatures maléfiques peuplant le marais sont des trolls et 20\% des Leprechauns. Pour savoir quel sort préparer, vous postez une vigie un peu myope (elle a raison 90\% du temps) qui vous signale un Leprechaun.

		On définit les événements suivants :
		\begin{itemize}
			\item \(C\) : « La créature est un troll. »
			\item \(V\) : « La vigie signale un troll. »
		\end{itemize}
	}
	\begin{enumerate}
		\item \question{Donner les probabilités \(\mathbb{P}(C)\), \(\mathbb{P}(\bar{C})\), \(\mathbb{P}(V \mid C)\) et \(\mathbb{P}(V \mid \bar{C})\).}
		\item \question{Calculer la probabilité \(\mathbb{P}(C \mid V)\) et l'interpréter.}
		\item \question{En déduire \(\mathbb{P}(\bar{C} \mid V)\).}
		\item \question{Que doit faire le mage ?}
	\end{enumerate}
}
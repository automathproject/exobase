\uuid{XKwm}
\chapitre{Polynôme, fraction rationnelle}
\sousChapitre{Racine, décomposition en facteurs irréductibles}
\titre{Multiplicité d'une racine}
\theme{polynômes}
\auteur{}
\datecreate{2023-01-23}
\organisation{AMSCC}
\contenu{

\question{ Déterminer l'ordre de multiplicité de la racine 1 du polynôme $$P(X)=X^5-5 X^4+14 X^3-22 X^2+17 X-5$$
	
En déduire la factorisation de $P(X)$. }

\indication{
	Revenir à la définition ou utiliser la caractérisation d'une racine multiple avec les dérivées du polynôme.
	
	Première méthode : effectuer la division euclidienne de $P$ par $(X-1)$, puis successivement du quotient par $(X-1)$ tant que le reste est nul. 

Deuxième méthode : évaluer $P(1)$, puis $P'(1)$, puis $P^{(2)}(1)$... tant que le résultat est nul. 
}


\reponse{ Méthode 1: Par divisions successives.

Le reste de la quatrième division euclidienne n'est pas nul, donc 1 est racine de multiplicité 3 .
En reprenant les calculs de division euclidienne, il vient :

\begin{align*}
X^5-X^4-X^3-X^2+4 X-2 & =(X-1)\left(X^4-4 X^3+10 X^2-12 X+5\right) \\
& =(X-1)(X-1)\left(X^3-3 X^2+7 X-5\right) \\
& =(X-1)(X-1)(X-1)\left(X^2-2 X+5\right) \\
& =(X-1)^3\left(X^2-2 X+5\right)
\end{align*}
}

\reponse{ Méthode 2 : par les dérivées successives.

$$
\begin{array}{rlrl}
P(X) & =X^5-5 X^4+14 X^3-22 X^2+17 X-5 & P(1) & =0 \\
P^{\prime}(X) & =5 X^4-20 X^3+42 X^2-44 X+17 & P^{\prime}(1) & =0 \\
P^{\prime \prime}(X) & =20 X^3-60 X^2+84 X-44 & P^{\prime \prime}(1) & =0 \\
P^{(3)}(X) & =60 X^2-120 X+84 & P^{(3)}(1) & =4 \neq 0
\end{array}
$$
Donc 1 est racine de multiplicité 3 de $P(X)$. Autrement dit, $P(X)$ est divisible par $(X-1)^3$
On effectue la division euclidienne de $P(X)$ par $(X-1)^3=X^3-3 X^2+3 X-1$, il vient :
$$
\begin{array}{r|r}
X^5-5 X^4+14 X^3-22 X^2+17 X-5 & X^3-3 X^2+3 X-1 \\
-\left(X^5-3 X^4+3 X^3-X^2\right) & X^2-2 X+5 \\
\hline-2 X^4+11 X^3-21 X^2+17 X-5 & \\
-\left(-2 X^4+6 X^3-6 X^2+2 X\right) & \\
\hline 5 X^3-15 X^2+15 X-5 & \\
-\left(5 X^3-15 X^2+15 X-5\right) & \\
\hline X^5-X^4-X^3-X^2+4 X-2=\left(X^3-3 X^2+3 X-1\right)\left(X^2-2 X+5\right) \\
=(X-1)^3\left(X^2-2 X+5\right)
\end{array}
$$
Il reste à factoriser $X^2-2 X+5$, polynôme du second degré dont le discriminant est: $\Delta=(-2)^2-4 \times 1 \times 5=-16<0$. La factorisation en produit de polynômes irréductibles sur $\mathbb{R}[X]:$
$$
P(X)=(X-1)^3\left(X^2-2 X+5\right)
$$
Les racines complexes conjuguées de $X^2-2 X+5$ sont $z=\frac{2 \pm(4 i)}{2}=1 \pm 2 i$. La factorisation en produit de polynômes irréductibles sur $\mathbb{C}[X]$ :
$$
P(X)=(X-1)^3(X-1-2 i)(X-1+2 i)
$$ }}

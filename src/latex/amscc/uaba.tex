\uuid{uaba}
\titre{Loi de Gumbel}
\chapitre{Probabilité continue}
\niveau{L2}
\module{Probabilité et statistique}
\sousChapitre{Théorème Central Limite}
\theme{Probabilité}
\auteur{Quentin Liard}
\organisation{AMSCC}

\difficulte{}
\contenu{

\texte{
Soit $X_1,\,\cdots,X_n$ des v.a. indépendantes et de même loi $X$, de densité:
$$f(x)=\exp(-(x-\theta)-e^{-(x-\theta)}),$$
où $\theta$ est un nombre réel positif donné. 
}
\begin{enumerate}
\item \question{Vérifier que $f$ est bien une densité de probabilité.}
\reponse{}
\item \question{On pose la v.a. $Y=e^{-(X-\theta)}$. Déterminer sa loi et donner son espérance et sa variance.}
\reponse{}
\item \question{Déterminer, en justifiant, une approximation de loi de la variable aléatoire $$T_n:=\frac{1}{n}\displaystyle\sum_{i=1}^{n}e^{-(X_i-\theta)}$$.}
\reponse{}
\end{enumerate}






}





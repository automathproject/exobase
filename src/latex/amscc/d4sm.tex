\uuid{d4sm}
\titre{Loi de Pareto}
\theme{variables aléatoires à densité}
\auteur{}
\datecreate{2023-11-29}
\organisation{AMSCC}

\contenu{
    Soient trois réels $\alpha,a,x_0$ avec $a>0$. On dit qu'une variable aléatoire $ X $ suit une loi de Pareto de paramètre $(\alpha, a, x_0)$ si elle admet pour densité la fonction $ f $ définie par : 
    $$ f(x) = \mathbf{1}_{]x_0 + a ;+\infty[}(x) \frac{\alpha}{a} \left( \frac{a}{x-x_0} \right)^{\alpha+1} $$
    
    Dans la suite, on fixe $a=1$ et $x_0=0$.
    
    \begin{enumerate}
        \item \question{Vérifier que la fonction $f$ définit bien une densité de probabilité. }
        \reponse{Pour que $ f $ soit une densité, l'intégrale de $ f $ doit être égale à 1. On a : 
        \begin{align*}
            \int_1^{+\infty} \alpha \left( \frac{1}{x} \right)^{\alpha+1} dx &= \alpha \int_1^{+\infty} x^{-(\alpha+1)} dx \\
            &= \alpha \left[ \frac{x^{-\alpha}}{-\alpha} \right]_1^{+\infty} \\
            &= \frac{\alpha}{\alpha} = 1
        \end{align*}
        Donc $f$ est positive et $\int f = 1$.}
        
        \item \question{Déterminer la fonction de répartition de $ X $.}
        \reponse{
            Soit $t \in \R$. Si $t \leq 1$, alors $F_X(t) = 0$. Si $t > 1$, alors :
            \begin{align*}
                F_X(t) &= \int_1^t \alpha \left( \frac{1}{x} \right)^{\alpha+1} dx \\
                &= \alpha \left[ \frac{x^{-\alpha}}{-\alpha} \right]_1^t \\
                &= 1 - \frac{1}{t^\alpha}
            \end{align*}
        }
        
        \item \question{Étudier l'existence de $ \E(X)$ en fonction de la valeur du paramètre $\alpha$.}
        \reponse{
            Si elle existe, l'espérance de $ X $ est $ \int_1^{+\infty} x \alpha \left( \frac{1}{x} \right)^{\alpha+1} dx$. Or cette intégrale existe si et seulement si $ \alpha > 1 $.
        }
        
        %\item \question{Étudier l'existence de la valeur éventuelle de $ V(X) $.}
        %\reponse{La variance de $X$ existe si $X$ admet un moment d'ordre 1 et un moment d'ordre 2. Or l'intégrale $ \int_1^{+\infty} x^2 \alpha \left( \frac{1}{x} \right)^{\alpha+1} dx $ existe si et seulement si $ \alpha > 2 $. Donc, la variance de $X$ existe si et seulement si $ \alpha > 2 $.}    
    Soit $ X $ une variable aléatoire suivant une loi de Pareto de paramètre $(\alpha,1 ,0)$ et soient deux réels $r,s$ tels que  $r > 0$. On pose alors la variable aléatoire $ Y = rX + s $.

        \item \question{Déterminer la loi de la variable aléatoire $Y$. }
        \reponse{
            Soit $ t \in \R $. Si $ t \leq s+1 $, alors $ F_Y(t) = 0 $. Si $ t > s+1 $, alors :
            \begin{align*}
                F_Y(t) &= \prob(Y \leq t) \\
                &= \prob(rX + s \leq t) \\
                &= \prob(X \leq \frac{t-s}{r}) \\
                &= F_X\left( \frac{t-s}{r} \right) \\
                &= 1 - \frac{1}{\left( \frac{t-s}{r} \right)^\alpha} \\
                &= 1 - \left( \frac{r}{t-s} \right)^\alpha
            \end{align*}
            Par dérivation, on trouve que $Y$ est une variable aléatoire absolument continue et que la densité de $Y$ est $ f_Y(x) = \mathbf{1}_{]s+1;+\infty[}(x) \frac{\alpha}{r} \left( \frac{r}{x-s} \right)^{\alpha+1} $. On trouve une loi de Pareto de paramètre $(\alpha, r, s+1)$.
        
        }
    \end{enumerate}
    
%    \begin{enumerate}
%        \setcounter{enumi}{5} % Commencer à partir de la question 6
%        \item \question{Soient $ X $ une variable aléatoire suivant une loi exponentielle de paramètre strictement positif $ \alpha $, et $ c $ un réel strictement positif. Quelle est la loi de la variable aléatoire $ Y = \frac{1}{cX^2} $?}
%        \reponse{En utilisant la transformation de variables aléatoires, nous trouvons que la loi de $ Y $ est ...}
%        
%        \item \question{Étudier la réciproque de la propriété ainsi démontrée.}
%        \reponse{Pour étudier la réciproque, nous considérons ... [Analyse de la réciproque] ...}
%    \end{enumerate}
%    
%    \begin{enumerate}
%        \setcounter{enumi}{7} % Commencer à partir de la question 8
%        \item \question{Soient $ Z $ une variable aléatoire qui suit une loi de Pareto de paramètre $(\alpha, 0)$, et $ c $ un réel strictement positif. Quelle est la loi de la variable aléatoire $ Z^c $?}
%        \reponse{Pour déterminer la loi de $ Z^c $, nous utilisons ... [Dérivation de la loi] ... La loi résultante est ...}
%    \end{enumerate}
}
\uuid{SxZb}
\chapitre{Probabilité continue}
\niveau{L2}
\module{Probabilité et statistique}
\sousChapitre{Densité de probabilité}
\titre{Densité}
\theme{variables aléatoires à densité, loi conjointe}
\auteur{}
\datecreate{2023-09-06}
\organisation{AMSCC}
\difficulte{}
\contenu{

\texte{ Soit la fonction $f$ définie sur $\R^2$ par : $$f(x,y)=2e^{-x}e^{-2y}\textbf{1}_{(\R^+)^2}(x,y).$$  
}
\begin{enumerate}
	\item \question{ Vérifier que $f$ définit une densité de probabilité. }
	\reponse{ On vérifie que pour tout $(x,y) \in \R^2$, $f(x,y) \geq 0$. De plus, par le théorème de Fubini,
	\begin{align*}
	\int_{-\infty}^{+\infty}\int_{-\infty}^{+\infty} f(x,y) dxdy &= \int_{0}^{+\infty}\int_{0}^{+\infty} 2e^{-x}e^{-2y} dxdy \\
	&= \int_{0}^{+\infty} e^{-x} dx \int_{0}^{+\infty} 2e^{-2y} dy \\
	&= 1
	\end{align*}
	}
	\item \question{ Calculer $\prob(X>1,Y<1)$, $\prob(X<Y)$ et $\prob(X<a)$. }
	\reponse{ On applique la définition et on trouve $\prob(X>1,Y<1) = e^{-1}(1-e^{-2})$, $\prob(X<Y)=\frac{1}{3}$ et $\prob(X<a)=1-e^{-a}$. }
\end{enumerate}
}
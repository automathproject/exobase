\uuid{G5Jt}
\chapitre{Probabilité discrète}
\niveau{L2}
\module{Probabilité et statistique}
\sousChapitre{Variable aléatoire discrète}
\titre{Tirage complet d'une urne}
\theme{variables aléatoires discrètes}
\auteur{}
\datecreate{2023-09-04}
\organisation{AMSCC}
\difficulte{4}
\contenu{

%proba268

\texte{ Soit $n \in \N^*$. Une urne contient $n$ boules blanches numérotées de $1$ à $n$ et $2$ boules noires numérotées $1$ et $2$. On effectue le tirage de toutes les boules de l'urne, une à une, et sans remise. On appelle $X$ le rang d'apparition de la première boule blanche et $Y$ celui du premier numéro $1$. }

\begin{enumerate}
	\item \question{ Déterminer la loi de $X$. }
	\indication{Voir les probabilités $\prob(X=3)$ et $\prob(X=2)$ comme des probabilités conditionnelles. }
	\reponse{ L'ensemble des valeurs prises par $X$ est $X(\Omega) = \{1,2,3\}$. 
Les boules sont a priori indiscernables au toucher donc le tirage d'un boule parmi les $n+2$ suit une loi uniforme, autrement dit la probabilité de tirer une boule en particulier est $\frac{1}{n+2}$. On en déduit que la probabilité que la première boule tirée soit blanche est $$\prob(X=1) = \frac{n}{n+2}$$.

D'après la formule de Bayes, on a $$\prob(X=2) = \prob(\overline{N_1})\prob(N_2 \mid \overline{N_1}) = \frac{2}{n+2}\frac{n}{n+1} = \frac{2n}{(n+2)(n+1)}$$

De même, on a $$\prob(X=3) = \frac{2}{n+2}\frac{1}{n+1}\frac{n}{n} = \frac{2}{(n+2)(n+1)}$$
 }
	\item \question{ Montrer que les événements $\{X=1\}$ et $\{Y=1\}$ sont indépendants si et seulement si $n=2$.  }
	\reponse{ On a $\prob(X=1,Y=1) = \frac{1}{n+2}$, c'est la probabilité que la première boule tirée soit blanche et qu'elle porte le numéro 1. 
	
On a $\prob(Y=1) = \frac{2}{n+2}$ car au premier tirage, $2$ boules portent le numéro $1$. 

Les événements  $\{X=1\}$ et $\{Y=1\}$ sont indépendants si et seulement si : 
\begin{align*}
\prob(X=1,Y=1) = \prob(X=1)\prob(Y=1) &\iff \frac{1}{n+2} = \frac{n}{n+2} \times \frac{2}{n+2} \\
&\iff 2n = n+2 \\
&\iff n = 2
\end{align*}
  }
	\item \question{ Montrer que les variables aléatoires $X$ et $Y$ ne sont pas indépendantes. }
	\reponse{ On compare par exemple $\prob(X=3,Y=3)$ et $\prob(X=3)\times \prob(Y=3)$. Or $\prob(X=3,Y=3)=0$ car pour que la première boule blanche apparaisse au rang $3$, il faut que les deux premières tirées soient noires, dont l'une d'elle porte le numéro $1$. 
	
Or il est clair que $\prob(X=3) \neq 0$ et $\prob(Y=3)\neq 0$. On en déduit que  $\prob(X=3,Y=3) \neq \prob(X=3)\times \prob(Y=3)$ ce qui permet de conclure que  les variables aléatoires $X$ et $Y$ ne sont pas indépendantes. }
	\item \texte{ On suppose maintenant que $n=2$. }
	\begin{enumerate}
		\item \question{ Montrer que $X$ et $Y$ ont la même loi de probabilité. }
\reponse{
	Pour $n=2$, l'urne contient 4 boules au total : deux blanches numérotées $\{1, 2\}$ et deux noires numérotées $\{1, 2\}$. Les ensembles des valeurs pour $X$ et $Y$ sont $X(\Omega) = Y(\Omega) = \{1, 2, 3\}$.
	
	La loi de probabilité de $X$ a été établie à la question 1. Pour $n=2$, nous avons :
	$$\prob(X=1) = \frac{1}{2}, \quad \prob(X=2) = \frac{1}{3}, \quad \prob(X=3) = \frac{1}{6}$$
	
	Pour déterminer la loi de $Y$, notons $N_i$ l'événement "la $i$-ème boule tirée porte le numéro 1" et $\overline{N_i}$ son événement contraire.
	
	\begin{itemize}
		\item L'événement $\{Y=1\}$ est réalisé si la première boule tirée porte le numéro 1. Cet événement est $N_1$. Sur les 4 boules initiales, 2 portent le numéro 1.
		$$\prob(Y=1) = \prob(N_1) = \frac{2}{4} = \frac{1}{2}$$
		
		\item L'événement $\{Y=2\}$ est réalisé si la première boule tirée ne porte pas le numéro 1, et la seconde si. Cet événement correspond à l'intersection $\overline{N_1} \cap N_2$. En utilisant la formule des probabilités composées :
		$$\prob(Y=2) = \prob(\overline{N_1} \cap N_2) = \prob(\overline{N_1}) \times \prob(N_2 \mid \overline{N_1})$$
		La probabilité $\prob(\overline{N_1})$ est la probabilité de tirer une boule numérotée 2, soit $\frac{2}{4}$. Sachant $\overline{N_1}$, il reste 3 boules dans l'urne, dont les 2 boules portant le numéro 1.
		$$\prob(Y=2) = \frac{2}{4} \times \frac{2}{3} = \frac{4}{12} = \frac{1}{3}$$
		
		\item L'événement $\{Y=3\}$ correspond à l'intersection $\overline{N_1} \cap \overline{N_2} \cap N_3$. On applique à nouveau la formule des probabilités composées :
		$$\prob(Y=3) = \prob(\overline{N_1}) \times \prob(\overline{N_2} \mid \overline{N_1}) \times \prob(N_3 \mid \overline{N_1} \cap \overline{N_2})$$
		Nous avons $\prob(\overline{N_1}) = \frac{2}{4}$. Sachant $\overline{N_1}$, il reste 3 boules dont une seule porte le numéro 2, donc $\prob(\overline{N_2} \mid \overline{N_1}) = \frac{1}{3}$. Sachant $\overline{N_1} \cap \overline{N_2}$, il ne reste que les 2 boules portant le numéro 1, donc $\prob(N_3 \mid \overline{N_1} \cap \overline{N_2}) = \frac{2}{2} = 1$.
		$$\prob(Y=3) = \frac{2}{4} \times \frac{1}{3} \times 1 = \frac{2}{12} = \frac{1}{6}$$
	\end{itemize}
	En comparant les résultats, on observe bien que pour tout $k \in \{1, 2, 3\}$, $\prob(X=k) = \prob(Y=k)$. Les variables aléatoires $X$ et $Y$ suivent donc la même loi de probabilité.
}
 		\item \question{ Déterminer la loi du couple $(X,Y)$.  }
 		\reponse{ En utilisant les questions précédentes et en calculant de manière similaire $\prob(X=1,Y=2)$, $\prob(X=1,Y=3)$, $\prob(X=3,Y=1)$
\begin{center}
	 	\begin{tabular}{|c|c|c|c|}
 		\hline
 	$X \backslash Y$	& $1$ & $2$ & $3$ \\
 		\hline
 	$1$	& $\frac{1}{4}$ & $\frac{1}{6}$  & $\frac{1}{12}$ \\
 		\hline
 	$2$	& $\frac{1}{6}$ & $\frac{1}{12}$ & $\frac{1}{12}$ \\
 		\hline
 	$3$	& $\frac{1}{12}$ & $\frac{1}{12}$ & $0$ \\
 		\hline
 	\end{tabular}	
\end{center}
En sommant les lignes et les colonnes, on retrouve les résultats des lois marginales calculées précédemment. 
 	 }
	\end{enumerate}
\end{enumerate}


}
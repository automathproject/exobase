\chapitre{Dérivabilité des fonctions réelles}
\sousChapitre{Applications}
\uuid{lH75}
\titre{Extremum d'un polynôme}
\theme{calcul différentiel, optimisation}
\auteur{}
\datecreate{2023-03-20}
\organisation{AMSCC}
\contenu{


\texte{ 	Soit une fonction $f$ définie sur $\mathbb{R}^2$ par : 
	$$f \colon (x,y) \mapsto x^2+xy+y^2+2x+3y$$ }

\begin{enumerate}
	\item \question{ La fonction $f$ admet-elle des points critiques sur $\mathbb{R}^2$ ? }
	\reponse{ La fonction $f$ est de classe $\mathcal{C}^1$ sur $\mathbb{R}^2$. On étudie d'abord l'éventualité d'un extremum local en cherchant les points critiques :
			$$\frac{\partial f}{\partial x}(x,y) = \frac{\partial f}{\partial y}(x,y) = 0 \iff 2x+y+2 = x+2y+3 = 0 \iff (x,y)=\left(-\frac{1}{3},-\frac{4}{3}\right)$$ }
	\item \question{ Vérifier que $-\frac{7}{3}$ constitue un minimum local et global de $f$ sur $\R^2$. }
	\reponse{ Or $f(x,y) = (x+y/2+1)^2 + \frac{3}{4}(y+4/3)^2-\frac{7}{3} \geq -\frac{7}{3} = f\left(-\frac{1}{3},-\frac{4}{3}\right)$
		donc $-\frac{7}{3}$ est un minimum local et global de $f$.  }
	\item \question{ La fonction $f$ admet-elle un maximum global sur $\R^2$ ? }
	\reponse{ 	On constate facilement que $f$ n'admet pas de maximum global car $f(x,0) \xrightarrow[]{x \to +\infty} +\infty$.  }
\end{enumerate}	
	}

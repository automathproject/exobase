\chapitre{Résolution d'équation différentielle}
\sousChapitre{Résolution d'équation différentielle}
\uuid{XQqR}
\titre{Méthode du point milieu (Euler améliorée)}
\theme{analyse numérique}
\auteur{}
\datecreate{2023-03-20}
\organisation{AMSCC}
\contenu{



\texte{ La méthode d'Euler explicite est convergente et consistante d'ordre 1. On considère l'approximation suivante :
	 $$\int_{t_n}^{t_{n+1}}f(s,y(s))ds \approx h\cdot f\left( t_n+\frac{h}{2},y\left( t_n+\frac{h}{2} \right)  \right).$$

La méthode qui en découle s'écrit donc sous la forme 
$$y_{n+1} = y_n + h\cdot f\left( t_n + \frac{h}{2},y_{n+1/2} \right)$$
où $y_{n+1/2}$ reste à définir sous la forme 
$$y_{n+1/2} = y_n + \frac{h}{2}G(t_n,y_n,h)$$ }

\question{ Déterminer une fonction $G$ permettant à ce schéma d'être consistant d'ordre 2. }

\reponse{
La méthode s'écrit donc sous la forme $y_{n+1} = y_n + H(t_n,y_n,h)$ avec $H(t,y,h) = f\left(t+\frac{h}{2},y+\frac{h}{2}G(t,y,h)  \right)$

Quel que soit le choix de $G$, on obtient $H(t,y,0) = f(t,y)$ de sorte que par théorème, la méthode est consistante au moins d'ordre 1.

On dérive $H$ par rapport à $h$ :
...

et on trouve que $\frac{\partial H}{\partial h}(t,y,0) = \frac{1}{2} f^{[1]}(t,y)$ si et seulement si $G(t,y,0)=f(t,y)$. On prend donc $G(t,y,h)=f(t,y)$.
}}

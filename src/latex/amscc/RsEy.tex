\chapitre{Probabilité continue}
\sousChapitre{Loi normale}
\uuid{RsEy}
\titre{Fonction caractéristique et loi normale}
\theme{loi normale, fonction caractéristique}
\auteur{}
\datecreate{2024-10-24}
\organisation{AMSCC}
\contenu{

\texte{
Soit $Z$ une variable aléatoire suivant une loi normale centrée réduite. On rappelle que la fonction caractéristique de $ Z $ est définie pour tout réel $t$ par :
$$\phi_Z(t)=E(e^{itZ})=e^{-\frac{t^2}{2}}.$$
}

\begin{enumerate}
    \item \question{ Soit $\mu$ un réel et $\sigma$ un réel strictement positif. Sans justifier, donner la loi de $X=\sigma Z+\mu$, puis calculer la fonction caractéristique de $ X$. }
    
    \item \question{ Soit $(X_1,\dots{},X_5)$ une suite de 5 variables aléatoires indépendantes et équidistribuées selon une loi normale de moyenne $\mu=70$ et d’écart type $\sigma=15$. Soit $S=\displaystyle\sum_{i=1}^{5}X_i$. }
    \begin{enumerate}
        \item \question{ Calculer la fonction caractéristique de $S$ et en déduire la loi de $S$. }
        \item \question{ Calculer la valeur de $P(S>450)$ à $10^{-2}$ près. }
    \end{enumerate}
\end{enumerate}
}
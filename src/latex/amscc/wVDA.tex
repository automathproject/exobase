\uuid{wVDA}
\chapitre{Analyse numérique}
\sousChapitre{Autre}
\titre{Erreur relative et consistance}
\theme{analyse numérique}
\auteur{}
\datecreate{2023-04-21}
\organisation{AMSCC}
\difficulte{}
\contenu{
	%	Les questions suivantes sont indépendantes.
	
	%	\begin{enumerate}
	%		\item 
\texte{ 	Soient $b$ et $\delta b$ deux vecteurs de $\mathbb{R}^n$ avec $b\neq 0$ et $A=(a_{ij})\in\mathbb{R}^{n\times n}$ une matrice inversible. 
	La matrice $A$ étant inversible, soient $X$ l'unique solution de $AX=b$ et $X+\delta X$ l'unique solution de $A(X+\delta X)=b+\delta b$. }
	\begin{enumerate}
	\item 	
\question{ Démontrer que : $$\frac{||\delta X||}{||X||}\leq \text{cond}(A) \frac{||\delta b||}{||b||}$$ }
		\item \question{ En déduire l'encadrement 
		$$
		\frac{1}{\text{cond}(A)} \frac{||\delta b||}{||b||}
		\leq 
		\frac{||\delta X||}{||X||}
		\leq 
		\mathrm{cond}(A) \frac{||\delta b||}{||b||}$$ }
	\end{enumerate}}

\uuid{9Acu}
\chapitre{Continuité, limite et étude de fonctions réelles}
\sousChapitre{Fonctions équivalentes, fonctions négligeables}
\titre{Equivalents simples}
\theme{limite}
\auteur{Maxime NGUYEN}
\datecreate{2025-04-12}
\organisation{AMSCC}
\contenu{
	\texte{Trouver un équivalent simple pour chaque fonction suivante :}
	\colonnes{\solution}{2}{1}
	%\begin{multicols}{3}
	\begin{enumerate}
		\item \question{$\frac{3x^2-x+4}{2x^3+5x+7} \ \text{ en } 0$}
		\reponse{Réponse : $\frac{4}{7}$ \\ 
			Lorsque $x \to 0$, le numérateur tend vers $3(0)^2-(0)+4 = 4$ et le dénominateur vers $2(0)^3+5(0)+7 = 7$. Donc la fonction est équivalente à $\frac{4}{7}$ en $0$.}	
		
		\item \question{$ \sqrt{\frac{x^2+1}{x^2-1}} \ \text{ en } +\infty$}
		\reponse{Réponse : $1$ \\
			Quand $x \to +\infty$, $\frac{x^2+1}{x^2-1} = \frac{1+\frac{1}{x^2}}{1-\frac{1}{x^2}} \sim 1$. Par continuité de la racine carrée, $\sqrt{\frac{x^2+1}{x^2-1}} \sim \sqrt{1} = 1$.}
		
		\item \question{$e^{x}-1-x \ \text{ en } 0$}
		\reponse{Réponse : $\frac{x^2}{2}$ \\
			En développant $e^x$ en $0$ : $e^x = 1 + x + \frac{x^2}{2} + o(x^2)$. Donc $e^x-1-x = \frac{x^2}{2} + o(x^2)$.}
		
		\item \question{$\ln(1+\frac{1}{x^4}) \ \text{ en } +\infty$}
		\reponse{Réponse : $\frac{1}{x^4}$ \\
			Quand $x \to +\infty$, $\frac{1}{x^4} \to 0$. En posant $t = \frac{1}{x}$, on a $\ln(1+t) \sim t$ pour $t$ proche de $0$. Donc $\ln(1+\frac{1}{x^4}) \sim \frac{1}{x^4}$.}
		
		\item \question{$ \frac{\sin x - x\cos x}{x^3} \ \text{ en } 0$}
		\reponse{Réponse : $\frac{1}{6}$ \\
			Développons $\sin x = x - \frac{x^3}{6} + o(x^3)$ et $\cos x = 1 - \frac{x^2}{2} + o(x^3)$ en $0$. Donc $\sin x - x\cos x = x - \frac{x^3}{6} - x(1 - \frac{x^2}{2}) + o(x^3) = x - \frac{x^3}{6} - x + \frac{x^3}{2} + o(x^3)= \frac{x^3}{3} + o(x^3)$. Ainsi, $\frac{\sin x - x\cos x}{x^3} \sim \frac{1}{3}$.}
		
		\item \question{$ \frac{x^3+2\sqrt{x}}{3x^3-x^2} \ \text{ en } +\infty$}
		\reponse{Réponse : $\frac{1}{3}$ \\
			Pour $x \to +\infty$, les termes dominants sont $x^3$ au numérateur et $3x^3$ au dénominateur. Donc $\frac{x^3+2\sqrt{x}}{3x^3-x^2} \sim \frac{x^3}{3x^3} = \frac{1}{3}$.}


	\end{enumerate}
	\fincolonnes{\solution}{2}{1}
	%\endcolumn
}
\chapitre{Probabilité continue}
\sousChapitre{Loi normale}
\uuid{Ac8Y}
\titre{ Loi log-normale }
\theme{variables aléatoires à densité, loi normale}
\auteur{Maxime NGUYEN}
\datecreate{2022-11-07}
\organisation{AMSCC}
\contenu{


\texte{ Soit $X$ une variable aléatoire suivant une loi normale quelconque $\mathcal{N}(\mu,\sigma)$ avec $\mu \in \R$ et $\sigma >0$. }

\begin{enumerate} 
  \item \question{ On pose $Y = e^X$. Calculer l'espérance et la variance de $Y$. }
  \reponse{
      On a $Y = e^X = e^{\mu + \sigma Z} = e^\mu e^{\sigma Z}$. D'après le théorème de transfert, 
\begin{align*}
    \E(e^{\sigma Z}) & = \int_{-\infty}^{+\infty} e^{\sigma z} \frac{1}{\sqrt{2\pi}} e^{-\frac{z^2}{2}} dz \\
    & = \frac{1}{\sqrt{2\pi}} \int_{-\infty}^{+\infty} e^{-\frac{z^2 - 2\sigma z + \sigma^2 - \sigma^2}{2}} dz \\
    & = \frac{1}{\sqrt{2\pi}} \int_{-\infty}^{+\infty} e^{-\frac{(z-\sigma)^2}{2}} e^{\frac{\sigma^2}{2}} dz \\
    & = e^{\frac{\sigma^2}{2}} \frac{1}{\sqrt{2\pi}} \int_{-\infty}^{+\infty} e^{-\frac{(z-\sigma)^2}{2}} dz \\
    & = e^{\frac{\sigma^2}{2}} \\ 
\end{align*}
Donc par linéarité, $\E(Y) = e^{\mu + \frac{\sigma^2}{2}}$. 

De même, on s'intéresse à $\E(Y^2) = \E(e^{2X}) = \E(e^{2\mu + 2\sigma Z}) = e^{2\mu} \E(e^{2\sigma Z})$. D'après le calcul précédent, $\E(e^{2\sigma Z}) = e^{2\sigma^2}$. Donc $\E(Y^2) = e^{2\mu + 2\sigma^2}$ et par théorème de Koenig-Huygens, 
\begin{align*}
  \var(Y) & = \E(Y^2) - \E(Y)^2 \\
  & = e^{2\mu + 2\sigma^2} - e^{2\mu + \sigma^2} \\
  & = e^{2\mu + \sigma^2} (e^{\sigma^2} - 1) 
\end{align*}
  }
  \item \question{ On suppose que $m=0$ et $\sigma = 1$. Déterminer une fonction densité de la variable $Y$. }
  \reponse{
      On a $Y = e^X = e^{\sigma Z} = e^Z$. Donc $Y$ est une variable aléatoire positive. Si $t >0$, on a :
      \begin{align*}
          \prob(Y \leq t) & = \prob(e^Z \leq t) \\
          & = \prob(Z \leq \ln(t)) \\
          &= F_Z(\ln(t))
      \end{align*}
      Donc la fonction de répartition de $Y$ est $F_Y(t) = \begin{cases}
          0 & \text{ si } t \leq 0 \\
          F_Z(\ln(t)) & \text{ si } t > 0 \\
      \end{cases}$. 
      Par dérivation, on obtient la densité de $Y$ : 
         $$ f_Y(x)  = \begin{cases}
              0 & \text{ si } x \leq 0 \\
              \frac{1}{x\sqrt{2\pi} } e^{-\frac{(\ln(x))^2}{2}} & \text{ si } x > 0 \\
          \end{cases} $$ 
  }
\end{enumerate}}

\uuid{ZTjz}
\titre{Développement limité}
\theme{analyse asymptotique}
\auteur{}
\datecreate{2023-05-12}
\organisation{AMSCC}
\contenu{

\question{ Donner le développement limité de $\ln(1+x)$ au voisinage de $0$ à l'ordre $3$. En déduire le  développement limité de $\left(\ln(1+x)\right)^2$ à l'ordre $3$.  }

\reponse{
	$$
	\begin{aligned}
		\ln (1+x) &=\underbrace{x-\frac{x^{2}}{2}+\frac{x^{3}}{3}}_{P_{1}(x)}+x^{3} \cdot \varepsilon(x) \\
		\left(P_{3}(x)\right)^{2} &=\left(x-\frac{x^{2}}{2}+\frac{x^{3}}{3}\right)^{2} \\
		&=\left(x-\frac{x^{2}}{2}+\frac{x^{3}}{3}\right) \times\left(x-\frac{x^{2}}{2}+\frac{x^{3}}{3}\right) \\
		&=x \times\left(x-\frac{x^{2}}{2}\right)-\frac{x^{2}}{2} \times x+ o(x^3) \\
		&=x^{2}-x^{3}+o(x^3)
	\end{aligned}
	$$
	Ainsi, $(\ln (1+x))^{2}=x^{2}-x^{3}+x^{3} \cdot \varepsilon(x)= x^2-x^3 + o(x^3)$ }}

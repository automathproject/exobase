\uuid{Hubl}
\chapitre{Série numérique}
\sousChapitre{Série à termes positifs}
\titre{ Variations autour de la série harmonique}
\theme{séries}
\auteur{}
\datecreate{2023-05-30}
\organisation{AMSCC}
\contenu{

	\begin{enumerate}
		\item \texte{   }
		\begin{enumerate}		

		\item \question{ Calculer $\displaystyle \sum_{n=1}^{k} \left(\ln(n+1)-\ln(n)\right)$ pour tout $k \geq 1$ et en déduire la nature de la série : $$\displaystyle \sum_{n \geq 1} \left(\ln(n+1)-\ln(n)\right)$$ }
		\reponse{C'est une série télescopique : 
			\begin{align*}
			 \sum_{n=1}^{k} \left(\ln(n+1)-\ln(n)\right) &= \ln(k+1)-\ln(k)+\ln(k)-\ln(k-1)+...-\ln(2)+\ln(2)-\ln(1) \\
			  &= \ln(k+1)-\ln(1) \\
			  &= \ln(k+1) \xrightarrow[k\to+\infty]{}+\infty
		 \end{align*}  
Donc par définition d'une série convergente, la série 	$\displaystyle \sum_{n \geq 1} \left(\ln(n+1)-\ln(n)\right)$ diverge car la suite des sommes partielles diverge.
 }
		\item \question{ On admet que pour tout $x \geq 0$, $\ln(1+x)\leq x $. Montrer que pour tout $n \in \mathbb{N}^*$, $$0 \leq \ln(n+1)-\ln(n) \leq \frac{1}{n}$$. }
\reponse{On remarque que $\ln(n+1)-\ln(n) = \ln\left(\frac{n+1}{n}\right) = \ln\left(1+\frac{1}{n}\right) $ or d'après la question précédente, $\ln\left(1+\frac{1}{n}\right) \leq \frac{1}{n}$.}
		\item \question{ En déduire que la série $\sum\limits_{n \geq 1} \frac{1}{n}$ diverge. }
\reponse{D'après la question 1.b), on a 
$$ \sum_{n=1}^{k} \left(\ln(n+1)-\ln(n)\right) \leq \sum_{n=1}^{k} \frac{1}{n}$$
donc par théorème des gendarmes, $\sum\limits_{n=1}^{k} \frac{1}{n} \xrightarrow[k\to+\infty]{
} +\infty$ ce qui permet de redémontrer le résultat du cours : la série $\sum_{n \geq 1} \frac{1}{n}$ est une série divergente.}
	\end{enumerate}	
	\item \texte{ Pour tout $n \geq 1$, on pose : $$u_n = \left(\frac{1}{n} \right)^{1+\frac{1}{n}}$$ }
	\begin{enumerate}
		\item \question{ Déterminer $\lim\limits_{n\to +\infty} e^{-\frac{\ln(n)}{n}}$. }
		\reponse{Par théorème de croissances comparées, on sait que $\frac{\ln(n)}{n} \xrightarrow[n\to+\infty]{} 0$ donc par composition de limites, $e^{-\frac{\ln(n)}{n}} \xrightarrow[n\to+\infty]{}e^{-0} = 1$. }
		\item \question{ Démontrer que $u_n \underset{n\to+\infty}\sim \frac{1}{n}$. }
		\reponse{Il suffit de calculer la limite du quotient et montrer qu'elle est égale à 1 : 
			\begin{align*}
			\frac{u_n}{\frac{1}{n}} &= n \times u_n = n \times \left(\frac{1}{n} \right)^{1+\frac{1}{n}}\\
			 &= n \times \frac{1}{n} \times \left(\frac{1}{n} \right)^{\frac{1}{n}} \\
			 &= \left(\frac{1}{n} \right)^{\frac{1}{n}} \\
			 &= e^{\frac{1}{n}\ln\left(\frac{1}{n}\right)} \\
			 &= e^{-\frac{\ln(n)}{n}} \xrightarrow[n\to+\infty]{} 1
			 \end{align*}
			}
		\item \question{ La série $\displaystyle \sum_{n \geq 1} u_n$ est-elle convergente ? }
		\reponse{La série $\sum u_n$ est une série à termes positifs, le terme général est équivalent au terme général d'une série de Riemann divergente donc la série $\sum u_n$ est divergente. }
		\item \question{ La série $\displaystyle \sum\limits_{n \geq 1} \dfrac{u_n}{n^{\frac{1}{10}}+\frac{1}{10}}$ est-elle convergente ? }
		\reponse{Il est clair que $n^{\frac{1}{10}}+\frac{1}{10} \underset{n\to+\infty}{\sim} n^{\frac{1}{10}}$ donc par quotient d'équivalents : 
			$$ \dfrac{u_n}{n^{\frac{1}{10}}+\frac{1}{10}} \underset{n\to+\infty}{\sim} \dfrac{\frac{1}{n}}{n^{\frac{1}{10}}} = \dfrac{1}{n^{\frac{11}{10}}}$$.
	On reconnaît le terme général d'une série de Riemann convergente. Par équivalence, la série à termes positifs 	$\sum\limits \dfrac{u_n}{n^{\frac{1}{10}}+\frac{1}{10}}$ est convergente.
	 }
	\end{enumerate}
	\end{enumerate}	
}

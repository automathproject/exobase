\uuid{dmDL}
\chapitre{Continuité, limite et étude de fonctions réelles}
\sousChapitre{Fonctions équivalentes, fonctions négligeables}
\titre{Equivalents : vrai ou faux}
\theme{limite}
\auteur{}
\datecreate{2023-05-11}
\organisation{AMSCC}
\contenu{

\texte{ Quels sont les équivalents corrects parmi les propositions suivantes ? Justifier (en revenant à la définition si nécessaire).
}
    \begin{multicols}{3}
    \begin{enumerate}
        \item \question{$x+1 \underset{x\to+\infty}\sim x$}
        \item \question{$x^2-x \underset{x\to+\infty}\sim x$}
        \item \question{$ \ln(x) \underset{x\to+\infty}\sim \ln(10^6x)$}
        \item \question{$x^2+2x+1 \underset{x\to+\infty}\sim x^2+2x$}
        \item \question{$\sqrt{x+1} \underset{x\to 0}\sim 1$}
        \item \question{$ x \underset{x\to 0}\sim 2x$}
        \item \question{$e^x \underset{x\to+\infty}\sim e^{x+10^6}$}
        \item \question{$e^x \underset{x\to 0}\sim e^{2x}$}
        \item \question{$\frac{6x^3+2x}{2x+1} \underset{x\to+\infty}\sim 3x^2$}
        \item \question{$\frac{6x^3+2x}{2x+1} \underset{x\to 0}\sim 2x$}
    \end{enumerate}
    \end{multicols}
    \href{https://moodle.st-cyr.terre.defense.gouv.fr/moodle/mod/quiz/view.php?id=31595}{A faire sur Moodle}
}

\uuid{PPoO}
\chapitre{Probabilité discrète}
\sousChapitre{Probabilité conditionnelle}
\titre{Urne et variable aléatoire}
\theme{probabilités conditionnelles}
\auteur{}
\datecreate{2023-02-07}
\organisation{AMSCC}
\contenu{

\texte{ 	Une urne contient sept boules : une rouge, deux jaunes et quatre vertes. Un joueur tire au hasard une boule.
	Si elle est rouge, il gagne 10\euro{}, si elle est jaune, il perd 5\euro{}, si elle est verte, il tire une deuxième boule de l'urne
	sans avoir replacé la première boule tirée.
	Si cette deuxième boule est rouge, il gagne 8\euro{}, sinon il perd 4\euro{}. }
	\begin{enumerate}
		\item \question{ Construire un arbre pondéré représentant l'ensemble des éventualités de ce jeu. }
		\item  \texte{ Soit X la variable aléatoire associant à chaque tirage le gain algébrique du joueur (une perte est comptée négativement). }
		\begin{enumerate}
			\item \question{ Établir la loi de probabilité de la variable $X$. }
			\item \question{ Calculer l'espérance de $X$. }
		\end{enumerate}
		\item \question{ Les conditions de jeu restent identiques. Indiquer le montant du gain algébrique qu'il faut attribuer à un joueur 	lorsque la boule tirée au deuxième tirage est rouge, pour que l'espérance de X soit nulle. }
	\end{enumerate}}

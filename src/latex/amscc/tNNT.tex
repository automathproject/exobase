\uuid{tNNT}
\chapitre{Probabilité continue}
\sousChapitre{Densité de probabilité}
\titre{Fonctions de répartition et changement de variable}
\theme{variables aléatoires à densité}
\auteur{}
\datecreate{2022-10-07}
\organisation{AMSCC}
\contenu{


\texte{ Soit $f$ une fonction réelle définie par 
 $$f(x)=\begin{cases}
 xe^{-\frac{x^2}{2}}& \text{ si } x \geq 0 \\
 0 & \text{ si } x < 0
 \end{cases}$$ }
 	\begin{enumerate}
 	\item \question{ Montrer que $f$ est une densité de probabilité. On définit $X$ une variable aléatoire admettant $f$ comme densité. }
 	\reponse{Il suffit de vérifier que $f(x) \geq 0$ pour tout $x \in \R$ puis de calculer :
 		\begin{align*}
 		\int_{-\infty}^{+\infty} f(x)dx &= \int_0^{+\infty} xe^{-\frac{x^2}{2}} dx \\
 		&= \left[-e^{-\frac{x^2}{2}}\right]_0^{+\infty} \\
 		&= 1
 		\end{align*}
 	}
 	\item \question{ Déterminer la fonction de répartition $F$ de la variable aléatoire $Y=X^2$ en fonction de celle de $X$. En déduire la densité de la variable $Y$. }
 	\reponse{Soit $t \in \R$ et $F_{X^2}$ la fonction de répartition de la variable aléatoire $X^2$ : alors 
 \begin{align*}
 F_{X^2}(t) &= \PP(X^2 \leq t) \\
 &= \begin{cases}
  \PP\left(-\sqrt{t} \leq X \leq \sqrt{t}\right) & \text{ si } t \geq 0 \\
  0 & \text{ si } t < 0
 \end{cases} \\
 &= \begin{cases}
 \int_{-\sqrt{t}}^{\sqrt{t}} f(x) dx & \text{ si } t \geq 0 \\
 0 & \text{ si } t < 0
 \end{cases} \\
 &= \begin{cases}
\int_{0}^{\sqrt{t}} xe^{-\frac{x^2}{2}} dx & \text{ si } t \geq 0 \\
0 & \text{ si } t < 0
\end{cases} \\
 &= \begin{cases}
\left[-e^{-\frac{x^2}{2}}\right]_0^{\sqrt{t}} & \text{ si } t \geq 0 \\
0 & \text{ si } t < 0
\end{cases} \\
&= \left(1-e^{-\frac{t}{2}}\right)\textbf{1}_{[0;+\infty[}(t)
 \end{align*}	
 
La fonction $F_{X^2}$ est dérivable presque partout (sauf en ${0}$). Sa dérivée coïncide donc presque partout avec une fonction densité $g$ de la variable $X^2$. On peut donc poser 
$$g(x)=\frac{1}{2}e^{-\frac{t}{2}}\textbf{1}_{[0;+\infty[}(x)$$
et on conclut que $X^2$ suit une loi exponentielle de paramètre $\lambda = \frac{1}{2}$.		
 		}
\end{enumerate}}

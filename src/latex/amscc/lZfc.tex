\uuid{lZfc}
\chapitre{Matrice}
\niveau{L1}
\module{Algèbre}
\sousChapitre{Inverse, méthode de Gauss}
\titre{Produit matriciel et inverse}
\theme{calcul matriciel}
\auteur{Maxime NGUYEN}
\datecreate{2022-12-15}
\organisation{AMSCC}
\difficulte{}
\contenu{


\texte{ Soient $A=\begin{pmatrix}a & 0 & 0 \\ 0 & b & 0 \\ 0 & 0 & c\end{pmatrix}$ et $B=\begin{pmatrix}d & 0 & 0 \\ 0 & e & 0 \\ 0 & 0 & f\end{pmatrix}$.  } où $a,b,c,d,e,f$ sont des réels quelconques.

\begin{enumerate}
	\item \question{ Calculer $A \times B$ et $B \times A$. Que remarque-t-on ? }
	\reponse{ On remarque que $AB = BA = \begin{pmatrix}ad & 0 & 0 \\ 0 & be & 0 \\ 0 & 0 & cf\end{pmatrix}$ }
	\item \question{ Déterminer $A^k$ pour tout entier $k \geq 1$.  }
	\reponse{ On déduit de la question précédente que $A^2 = \begin{pmatrix}a^2 & 0 & 0 \\ 0 & b^2 & 0 \\ 0 & 0 & c^2\end{pmatrix}$ puis par récurrence que :
		$$A^k = \begin{pmatrix}a^k & 0 & 0 \\ 0 & b^k & 0 \\ 0 & 0 & c^k\end{pmatrix}.$$  }
	\item \question{ \`A quelles conditions sur $a,b,c \in \R$ la matrice $A$ est-elle inversible ? Déterminer dans ce cas $A^{-1}$ puis $A^{-k}$ pour tout entier $k \geq 1$. }
	\reponse{ Si $a \neq 0$, $b \neq 0$ et $c \neq 0$ alors la matrice $A$ est inversible et 
$$A^{-1} = \begin{pmatrix}a^{-1} & 0 & 0 \\ 0 & b^{-1} & 0 \\ 0 & 0 & c^{-1}\end{pmatrix}$$	
  }
\end{enumerate}
}

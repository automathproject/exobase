\uuid{DITh}
\chapitre{Fonction de plusieurs variables}
\sousChapitre{Autre}
\titre{Fonctions périodiques}
\theme{fonctions de plusieurs variables}
\auteur{}
\datecreate{2024-04-18}
\organisation{AMSCC}
\contenu{


\texte{ 	Soit $f \colon \mathbb{R}^2 \to \mathbb{R}$ une fonction différentiable telle que pour tout $t \in \mathbb{R}$ et pour tout $(x,y) \in \mathbb{R}^2$, 
	$$f(x+t,y+t) = f(x,y)$$ }

\textbf{Partie A}
	\begin{enumerate}
		\item \question{ On fixe $(x,y) \in \mathbb{R}^2$ et pour tout $t \in \mathbb{R}$ on pose $g(t) = f(x+t,y+t)$. Calculer $g'(t)$ pour tout $t \in \mathbb{R}$, en l'exprimant en fonction de $\dfrac{\partial f}{\partial x}(x,y)$ et $\dfrac{\partial f}{\partial y}(x,y)$. }
		\reponse{Par la règle des chaînes, $g$ est dérivable sur $\mathbb{R}$ et $g'(t) = 1 \times \dfrac{\partial f}{\partial x}(x,y) + 1 \times \dfrac{\partial f}{\partial y}(x,y)$. }
		\item \question{ En déduire que $\dfrac{\partial f}{\partial x}(x,y) + \dfrac{\partial f}{\partial y}(x,y) = 0$. }
		\reponse{Par hypothèse sur la fonction $f$, $g(t) = f(x,y)$ donc $g$ est constante par rapport à $t$ donc $g'(t) = 0$. D'où l'égalité demandée.}
	\end{enumerate}
\textbf{Partie B}

%$$\left{ \begin{array}{c} u = x+y \\ v = x-y \end{array}\right. \Rightarrow  \left{ \begin{array}{c} x = \dfrac{u+v}{2} \\ y = \dfrac{u-v}{2} \end{array}\right. $$
		
		On définit la fonction $h : \mathbb{R}^{2} \to \mathbb{R}$ par  $ h(u,v) = f \left(\dfrac{u+v}{2}, \dfrac{u-v}{2} \right) $.
	
	\begin{enumerate}	
		\item \question{ En utilisant la règle des chaînes, exprimer la dérivée partielle $\frac{\partial h}{\partial u}(u,v)$, en fonction des dérivées partielles de $f$. }
		\reponse{
			Par la règle des chaînes, $h$ admet des dérivées partielles en tout point $(u,v) \in \mathbb{R}^2$ et on a :
			\begin{align*}
				\frac{\partial h}{\partial u}(u,v) & = \frac{\partial f}{\partial x}\left(\dfrac{u+v}{2}, \dfrac{u-v}{2}\right) \times \frac{1}{2} + \frac{\partial f}{\partial y}\left(\dfrac{u+v}{2}, \dfrac{u-v}{2}\right) \times \frac{1}{2} \\
				& = \frac{1}{2}\left(\frac{\partial f}{\partial x}\left(\dfrac{u+v}{2}, \dfrac{u-v}{2}\right) + \frac{\partial f}{\partial y}\left(\dfrac{u+v}{2}, \dfrac{u-v}{2}\right)\right).
			\end{align*}
		}
		
		\item \question{ En utilisant une question précédente, en déduire que $\frac{\partial h}{\partial u}(u,v) = 0$ }
		\reponse{
			D'après la question 2 de la partie A, on a $\dfrac{\partial f}{\partial x}\left(\dfrac{u+v}{2}, \dfrac{u-v}{2}\right) + \dfrac{\partial f}{\partial y}\left(\dfrac{u+v}{2}, \dfrac{u-v}{2}\right) = 0$. Donc, $\frac{\partial h}{\partial u}(u,v) = 0$.
		}
		
		\item \question{ En déduire qu'il existe une fonction $k \colon  \mathbb{R} \to \mathbb{R}$ telle que pour tout $(x,y) \in \R^2$, $$f(x,y) = k(x-y).$$ }
		\reponse{
			On a démontré à la question précédente que $\frac{\partial h}{\partial u}(u,v) = 0$. Donc, $h$ est constante par rapport à $u = x+y$. Il existe donc une fonction $k \colon \mathbb{R} \to \mathbb{R}$ telle que pour tout $(x,y) \in \R^2$, $f(x,y) = k(x-y)$.
		}
	\end{enumerate}
}
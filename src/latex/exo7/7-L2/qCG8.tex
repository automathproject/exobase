\uuid{qCG8}
\exo7id{6010}
\titre{exo7 6010}
\auteur{quinio}
\organisation{exo7}
\datecreate{2011-05-20}
\isIndication{false}
\isCorrection{true}
\chapitre{Probabilité discrète}
\sousChapitre{Variable aléatoire discrète}

\contenu{
\texte{
Dans une poste d'un petit village, on remarque qu'entre 10 heures
et 11 heures, la probabilité pour que deux personnes entrent durant 
la même minute est considérée comme nulle et que l'arrivée des
personnes est indépendante de la minute considérée. 
On a observé que la probabilité pour qu'une personne se présente entre la
minute $n$ et la minute $n+1$ est: $p = 0.1$. On veut calculer la probabilité 
pour que : 3,4,5,6,7,8... personnes se présentent au guichet entre 10h et 11h.
}
\begin{enumerate}
    \item \question{Définir une variable aléatoire adaptée, puis répondre au problème considéré.}
    \item \question{Quelle est la probabilité pour que au moins 10 personnes
se présentent au guichet entre 10h et 11h?}
\reponse{
Une variable aléatoire adaptée à ce problème est le nombre $X$ de personnes 
se présentant au guichet entre 10h et 11h. Compte tenu
des hypothèses, on partage l'heure en $60$ minutes. Alors $X$ suit une
loi binomiale de paramètres $n=60$ et $p=0.1$. On est dans le cas de
processus poissonnien : on peut approcher la loi de $X$ par la loi de
Poisson de paramètre $\lambda =60\times 0.1=6$.
L'espérance de $X$ est donc $E(X)=6$;

On peut alors calculer les probabilités demandées:
$P[X=k]=\frac{6^{k}e^{-6}}{k!}$. Valeurs lues dans une table ou calculées :
$P[X=3]\simeq 0.9\%;$ $P[X=4]\simeq 13.4\%;$ $P[X=5]=P[X=6]\simeq
\,16.1\%;$
$P[X=7]\,\simeq 13.8\%;P[X=8]\simeq 10.3\,\%.$

Remarque : de façon générale si le paramètre 
$\lambda$ d'une loi de Poisson est un entier $K$, on a:
$P[X=K-1]=\frac{K^{K-1}e^{-K}}{(K-1)!}=\frac{K^{K}e^{-K}}{K!}=P[X=K]\,.$

Calculons maintenant la probabilité pour que au moins $10$
personnes se présentent au guichet entre $10$h et $11$h: C'est $P[X\geq
10]=1-\sum_{k=0}^{9}\frac{6^{k}e^{-6}}{k!}\simeq 8.392\times
10^{-2}.$
}
\end{enumerate}
}

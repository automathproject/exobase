\uuid{Hcw5}
\exo7id{6936}
\auteur{ruette}
\organisation{exo7}
\datecreate{2013-01-24}
\isIndication{false}
\isCorrection{false}
\chapitre{Statistique}
\sousChapitre{Tests d'hypothèses, intervalle de confiance}

\contenu{
\texte{
Le couvert végétal du domaine vital d'un élan d'Amérique se
composait de  feuillus ($25,8\%$ de la superficie), de forêts mixtes 
($38\%$  de la superficie), de résineux ($25,8\%$ de la superficie) et d'un marécage ($10,4\%$ de la
superficie). Dans ce domaine, l'élan fut localisé à $511$ reprises
au cours de l'année : $118$ fois dans les feuillus, $201$ fois dans les forêts mixtes, $110$ fois dans les
résineux et $82$ fois dans le marécage. [Source~: B. Scherrer
``Biostatistique'', éditeur Gaetan Morin, 1984, page 556]

L'élan fréquente-t-il indifféremment les quatre types de
végétation de son domaine vital~?
}
}

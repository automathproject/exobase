\uuid{obWg}
\exo7id{5989}
\titre{exo7 5989}
\auteur{quinio}
\organisation{exo7}
\datecreate{2011-05-18}
\isIndication{false}
\isCorrection{true}
\chapitre{Probabilité discrète}
\sousChapitre{Probabilité et dénombrement}
\module{Probabilité et statistique}
\niveau{L2}
\difficulte{}

\contenu{
\texte{
La probabilité pour une population d'être atteinte
d'une maladie $A$ est $p$ donné; dans cette même population, un individu
peut être atteint par une maladie $B$ avec une probabilité $q$ donnée aussi; 
on suppose que les maladies sont indépendantes : quelle est la
probabilité d'être atteint par l'une et l'autre de ces maladies?
Quelle est la probabilité d'être atteint par l'une ou l'autre de ces
maladies?
}
\reponse{
$P(A\cap B)=pq$ car les maladies sont indépendantes.
$P(A\cup B)=P(A)+P(B)-P(A\cap B)=p+q-pq$
}
}

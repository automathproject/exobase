\uuid{1xXH}
\exo7id{6015}
\titre{exo7 6015}
\auteur{quinio}
\organisation{exo7}
\datecreate{2011-05-20}
\isIndication{false}
\isCorrection{true}
\chapitre{Probabilité discrète}
\sousChapitre{Lois de distributions}
\module{Probabilité et statistique}
\niveau{L2}
\difficulte{}

\contenu{
\texte{
Des machines fabriquent des plaques de tôle destinées à être
empilées.
}
\begin{enumerate}
    \item \question{Soit $X$ la variable aléatoire <<épaisseur de la plaque en mm>> ; 
on suppose que $X$ suit une loi normale de paramètres $m=0.3$ et $\sigma =0.1$. 
Calculez la probabilité pour que $X$ soit inférieur à 0.36mm et la
probabilité pour que $X$ soit compris entre 0.25 et 0.35mm.}
\reponse{La probabilité pour que $X$ soit inférieur à 0.36mm est :
$P[X\leq 0.36]=P[\frac{X-0.3}{0.1}\leq 0.6]=0.726,$ soit $72.6$\%.

 La probabilité pour que $X$ soit compris entre $0.25$ et $0.35\text{mm}$ est
$P[0.25\leq X\leq 0.35]=2 F(0.5)-1= 0.383$, soit $38.3$\%.}
    \item \question{L'utilisation de ces plaques consiste à en empiler $n$, numérotées de $1$ à $n$ 
en les prenant au hasard : soit $X_{i}$ la variable aléatoire <<épaisseur de la plaque numéro $i$ en mm>> 
et $Z$ la variable aléatoire <<épaisseur des $n$ plaques en mm>>.
Pour $n = 20$, quelle est la loi de $Z$, son espérance et sa variance?}
\reponse{Pour $n = 20$, la loi de $Z=\sum X_{i}$ est une loi normale de paramètres: d'espérance $E(Z)=20 m = 6$
et de variance $\text{Var}\,Z=20 \sigma^2 = 0.2$.}
\end{enumerate}
}

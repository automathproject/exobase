\uuid{CZRp}
\exo7id{6001}
\titre{exo7 6001}
\auteur{quinio}
\organisation{exo7}
\datecreate{2011-05-20}
\isIndication{false}
\isCorrection{true}
\chapitre{Probabilité discrète}
\sousChapitre{Probabilité conditionnelle}
\module{Probabilité et statistique}
\niveau{L2}
\difficulte{}

\contenu{
\texte{
Dans mon trousseau de clés il y a $8$ clés; elles sont toutes
semblables. Pour rentrer chez moi je mets une clé au hasard; je
fais ainsi des essais jusqu'à ce que je trouve la bonne; j'écarte au
fur et à mesure les mauvaises clés. Quelle est la probabilité
pour que j'ouvre la porte :
}
\begin{enumerate}
    \item \question{du premier coup ?}
    \item \question{au troisième essai ?}
    \item \question{au cinquième essai ?}
    \item \question{au huitième essai?}
\reponse{
Les permutations (fictives) qui traduisent le cas (1) sont celles qui
peuvent être représentées par une suite : 
$BMMMMMMM$, la lettre $B$ désigne la bonne, $M$ désigne une
mauvaise. Il y a $7!$ permutations de ce type.
Donc $P(A)=\frac{7!}{8!}=\frac{1}{8},$ on s'en doutait!
De même, les permutations (fictives) sont
celles qui peuvent être représentées par une suite:
$MBMMMMMM$: il y en a encore $7$!, et la probabilité est la même.
Le raisonnement permet en fait de conclure que la probabilité, avant
de commencer, d'ouvrir la porte est la même pour le premier, deuxième,..., huitième essai.
}
\end{enumerate}
}

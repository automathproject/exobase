\uuid{YRLt}
\exo7id{6019}
\titre{exo7 6019}
\auteur{quinio}
\organisation{exo7}
\datecreate{2011-05-20}
\isIndication{false}
\isCorrection{true}
\chapitre{Probabilité discrète}
\sousChapitre{Lois de distributions}
\module{Probabilité et statistique}
\niveau{L2}
\difficulte{}

\contenu{
\texte{
Pour chacune des variables aléatoires qui sont décrites ci-dessous, indiquez quelle est la loi exacte 
avec les paramètres éventuels (espérance, variance) et indiquez éventuellement une loi approchée.
}
\begin{enumerate}
    \item \question{Nombre annuel d'accidents à un carrefour donné où la
probabilité d'accident par jour est estimée à $\frac{4}{365}$.}
\reponse{Loi binomiale $B(365;\frac{4}{365}),$ approchée par la loi de Poisson
de paramètre 4, d'espérance et variance 4.}
    \item \question{Nombre de garçons dans une famille de 6 enfants; nombre de filles par
jour dans une maternité où naissent en moyenne 30 enfants par jour.}
\reponse{Loi binomiale $B(6;\frac{1}{2})$, d'espérance 3 et variance $\frac{3}{2}$.}
    \item \question{Dans un groupe de 21 personnes dont 7 femmes, le nombre de femmes dans
une délégation de 6 personnes tirées au hasard.}
\reponse{Loi hypergéométrique.}
\end{enumerate}
}

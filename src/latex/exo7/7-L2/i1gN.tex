\uuid{i1gN}
\exo7id{6914}
\auteur{ruette}
\organisation{exo7}
\datecreate{2013-01-24}
\isIndication{false}
\isCorrection{true}
\chapitre{Probabilité discrète}
\sousChapitre{Variable aléatoire discrète}

\contenu{
\texte{
Charles ne supporte pas les chats et Sophie déteste les chiens. 
Charles n'élève pas plus d'un chien et Sophie pas plus d'un chat. 
La probabilité pour que Charles ait un chien est de $0,2$. Si 
Charles n'a pas de chien, la probabilité pour que Sophie ait un chat est de $0,1$.
On note $X$ le nombre de chiens de Charles, $Y$ le nombre de chats de Sophie
et $Z$ le nombre d'animaux du couple.
}
\begin{enumerate}
    \item \question{Calculer la probabilité pour qu'ils n'aient pas d'animaux.}
\reponse{Soit $A$ l'événement ``Charles n'a pas de chien'' et $B$ l'événement 
``Sophie n'a pas de chat''. L'énoncé donne $P(B|A)=0,9$. Par conséquent, 
la probabilité pour que le ménage n'ait aucun animal est $P(A\cap B)=P(B|A)P(A)=0,9\times 0,8=0,72$.}
    \item \question{On suppose de plus que la probabilité que $Z$ soit égal à 1 est de $0,1$.
\begin{enumerate}}
\reponse{\begin{enumerate}}
    \item \question{Calculer la probabilité pour que $Z$ soit égal à 2.}
\reponse{$Z$ ne peut prendre que les valeurs 0, 1 et 2. L'événement $\{Z=0\}$ 
coïncide avec $A\cap B$, donc $P(Z=0)=0,72$. Il vient $P(Z=2)=1-P(Z=0)-P(Z=1)=1-0,72-0,1=0,18$.}
    \item \question{Déterminer l'espérance et l'écart-type de $Z$.}
\reponse{$E(Z)=0 \cdot P(Z=0)+1 \cdot P(Z=1)+2 \cdot P(Z=2)=0,1+0,36=0,46$. $E(Z^2 )=0^2 P(Z=0)+1^2 
P(Z=1)+2^2 P(Z=2)=0,1+0,72=0,82$, d'où $\text{Var}(Z)=E(Z^2 )-E(Z)^2 =0,6084$, $\sigma(Z)=\sqrt{\text{Var} (Z)}=0,78$.}
    \item \question{\'Etablir la loi de probabilité du couple $(X,Y)$.
Quelle est la loi de probabilité de $Y$ ?}
\reponse{On calcule $P(X=0\mbox{ et } Y=0)=P(Z=0)=0,72$, $P(X=1\mbox{ et } Y=1)=P(Z=2)=0,18$, 
$
P(X=0\mbox{ et }Y=1)=P(A\cap B^c)=P(B^c|A)P(A)=(1-P(B|A))P(A)=0,1\times 0,8=0,08. 
$
On complète le tableau en utilisant le fait que le somme des probabilités des événements élémentaires vaut 1.
\begin{center}
\begin{tabular}{|c|c|c|c|}
\hline
&$Y=0$&$Y=1$&Total\\\hline
$X=0$&0,72&0,08&0,8\\\hline
$X=1$&0,02&0,18&0,2\\\hline
Total&0,74&0,26&1\\
\hline
\end{tabular}
\end{center}
La dernière ligne du tableau donne la loi de $Y$, $P(Y=0)=0,74$ et $P(Y=1)=0,26$.}
    \item \question{Les variables $X$ et $Y$  sont-elles indépendantes ?}
\reponse{On constate que $P(X=1\mbox{ et } Y=1)=0,18$ n'est pas égal à 
$P(X=1)P(Y=1)=0,2\times 0,26$, donc $X$ et $Y$ ne sont pas indépendantes.}
\end{enumerate}
}

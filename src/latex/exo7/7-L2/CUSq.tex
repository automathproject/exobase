\uuid{CUSq}
\exo7id{5997}
\titre{exo7 5997}
\auteur{quinio}
\organisation{exo7}
\datecreate{2011-05-20}
\isIndication{false}
\isCorrection{true}
\chapitre{Probabilité discrète}
\sousChapitre{Probabilité conditionnelle}
\module{Probabilité et statistique}
\niveau{L2}
\difficulte{}

\contenu{
\texte{
En cas de migraine trois patients sur cinq prennent de l'aspirine
(ou équivalent), deux sur cinq prennent un médicament M présentant des effets secondaires :

Avec l'aspirine, 75\% des patients sont soulagés.

Avec le médicament M, 90\% des patients sont soulagés.
}
\begin{enumerate}
    \item \question{Quel est le taux global de personnes soulagées?}
\reponse{Le taux global de personnes soulagées :
$P(S)=\frac{3}{5}0.75+\frac{2}{5}0.90=0.81$.}
    \item \question{Quel est la probabilité pour un patient d'avoir pris de l'aspirine
sachant qu'il est soulagé?}
\reponse{Probabilité pour un patient d'avoir pris de l'aspirine sachant qu'il
est soulagé :
$P(A/S)=P(A\cap S)/P(S)=P(A)P(S/A)/P(S)=\frac{\frac{3}{5}0.75}{0.81}=55.6\%$.}
\end{enumerate}
}

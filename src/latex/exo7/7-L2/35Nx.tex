\uuid{35Nx}
\exo7id{6905}
\titre{exo7 6905}
\auteur{ruette}
\organisation{exo7}
\datecreate{2013-01-24}
\isIndication{false}
\isCorrection{true}
\chapitre{Probabilité discrète}
\sousChapitre{Variable aléatoire discrète}
\module{Probabilité et statistique}
\niveau{L2}
\difficulte{}

\contenu{
\texte{
Un transporteur aérien a observé que  25\% en moyenne des personnes ayant réservé 
un siège pour un vol ne se présentent pas au départ. Il décide d'accepter jusqu'à  
23  réservations alors qu'il ne dispose que de  20  sièges pour ce vol.
}
\begin{enumerate}
    \item \question{Soit $X$ la variable aléatoire ``nombre de clients qui viennent après
réservation quand 23 places ont été réservées''.
Quelle est la loi de $X$ (précisez les hypothèses que vous faites pour modéliser la situation) ?
Quelle est son espérance ?}
\reponse{La loi de $X$ est une loi binomiale de paramètres $n=23$, $p=0,75$ : 
$P(X=k)={k\choose{n}}p^k(1-p)^{n-k}$ si $0\le k\le n$.
Son espérance est $np=17,25$.}
    \item \question{Si 23 personnes ont réservé, quelle est la probabilité que toutes les personnes qui se présentent au départ aient un siège ?}
\reponse{$P(X\le 20)=1-P(X\in\{21, 22, 23\})\simeq 0,951$.}
\end{enumerate}
}

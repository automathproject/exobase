\uuid{7Zpw}
\exo7id{6016}
\auteur{quinio}
\organisation{exo7}
\datecreate{2011-05-20}
\isIndication{false}
\isCorrection{true}
\chapitre{Probabilité discrète}
\sousChapitre{Lois de distributions}

\contenu{
\texte{
Des machines fabriquent des plaques de tôle destinées à être
empilées; on estime à 0.1\% la proportion de plaques inutilisables.
L'utilisation de ces plaques consiste à en empiler $n$, numérotées
de $1$ à $n$ en les prenant au hasard.
Pour $n = 2000$, quelle est la loi suivie par la variable aléatoire $N$
<<nombre de plaques inutilisables parmi les 2000>> ? (on utilisera une loi de probabilité adaptée); 
quelle est la probabilité pour que $N$ soit inférieure ou égal à 3 ? 
Quelle est la probabilité pour que $N$ soit strictement inférieure à 3?
}
\reponse{
Pour $n = 2000$, la loi suivie par la variable aléatoire $N$
<<nombre de plaques inutilisables parmi les 2000>> est une loi de Poisson de paramètre 2:
alors $P[N\leq 3]=0.86$.

Remarquons qu'en faisant l'approximation par une loi normale et en employant
le théorème central limite, on obtient: $P[N\leq 3]\simeq 0.76,$ et
avec correction de continuité on obtient $P[N\leq 3]\simeq 0.85.$
}
}

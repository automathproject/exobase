\uuid{zXLh}
\exo7id{2271}
\titre{exo7 2271}
\auteur{barraud}
\organisation{exo7}
\datecreate{2008-04-24}
\isIndication{false}
\isCorrection{true}
\chapitre{Polynôme}
\sousChapitre{Polynôme}
\module{Algèbre et théorie des nombres}
\niveau{L3}
\difficulte{}

\contenu{
\texte{
Trouver le $\pgcd(x^n-1,\ x^m-1)$ dans
$\Zz[x]$.
}
\reponse{
Soit $d=\pgcd(m,n)$. Notons $n=dn'$ et $m=dm'$. Alors
  $X^{n}-1=(X^{d})^{n'}-1$. Or $(Y-1)|Y^{n'}-1$ donc
  $(X^{d}-1)|(X^{n}-1)$. De même, $(X^{d}-1)|(X^{m}-1)$, et donc
  $(X^{d}-1)|\pgcd(X^{n}-1,X^{m}-1)$.

  Par ailleurs, soit $D=\pgcd(X^{n}-1,X^{m}-1)$. Les racines de $D$ dans
  $\Cc$ sont des racines à la fois n-iéme et m-ième de $1$, qui sont
  touts simples~: elles sont donc de la forme $\omega=e^{i2\pi\alpha}$ où
  $\alpha=\frac{k}{n}=\frac{k'}{m}$. Ainsi $km'=k'n'$. On a
  $\pgcd(m',n')=1$, donc par le théorème de Gauss, on en déduit que $k'$
  est un multiple de $m'$, soit $\frac{k'}{m}=\frac{k''}{d}$, et $\omega$
  est donc une racine $d$-ième de 1. On en déduit que $D|X^{d}-1$, et
  finalement~:
  $$\pgcd(X^{n}-1,X^{m}-1)=X^{\pgcd(m,n)}-1.$$
}
}

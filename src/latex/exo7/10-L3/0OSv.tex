\uuid{0OSv}
\exo7id{2312}
\auteur{barraud}
\organisation{exo7}
\datecreate{2008-04-24}
\isIndication{false}
\isCorrection{true}
\chapitre{Anneau, corps}
\sousChapitre{Anneau, corps}

\contenu{
\texte{
Montrer que les polyn\^omes suivantes sont irr\'eductibles dans
  $\Zz[x]$ :
}
\begin{enumerate}
    \item \question{$P=x^{2004}+4x^{2002}+2000x^4+2002$;}
\reponse{Le critère d'Eisenstein avec $2$ pour module donne directement le
résultat.}
    \item \question{$Q=x^6+6x^5+12x^4+12x^3+3x^2+6x+25$.}
\reponse{La réduction modulo $2$ de $Q$ est $Q_{[2]}=x^{6}+x^{2}+1$, qui n'a pas
de racine, et n'est pas divisible par $x^{2}+x+1$, le seul irréductible
de degré $2$ de $\Zz_{2}[x]$. Ainsi, $Q_{[2]}$ est soit irréductible,
auquel cas $Q$ l'est aussi sur $\Zz$, soit le produit de deux
irréductibles de degré $3$.

Si $Q_{[2]}$ n'est pas irréductible, on considère la réduction modulo $3$
de $Q$~: $Q_{[3]}=x^{6}+1=(x^{2}+1)^{3}$. $x^{2}+1$ est irréductible sur
$\Zz_{3}$, car il est de degré $2$ et n'a pas de racine. Soit $Q=RS$ une
factorisation non triviale de $Q$ sur $\Zz$. On peut supposer $R$ et $S$
unitaires. Alors, en considérant la réduction modulo $2$, on obtient que
$R_{[2]}$ et $S_{[2]}$ sont deux irréductibles de degré $3$ de
$\Zz_{2}[x]$. En particulier $\deg(R)=\deg(R_{[2]})=3$ (car $R$ est
unitaire) et $\deg(S)=\deg(S_{[2]})=3$. Cependant, la réduction modulo
$3$ de $Q$ n'admet pas de factorisation suivant deux polynômes de degré
3. C'est une contradiction~: on en déduit que $Q$ n'a pas de factorisation
non triviale.}
\end{enumerate}
}

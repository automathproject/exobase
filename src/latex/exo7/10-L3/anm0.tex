\uuid{anm0}
\exo7id{6454}
\auteur{drutu}
\organisation{exo7}
\datecreate{2011-10-16}
\isIndication{false}
\isCorrection{false}
\chapitre{Groupe}
\sousChapitre{Groupe}

\contenu{
\texte{

}
\begin{enumerate}
    \item \question{Vérifier que pour tout cycle $(i_1\, i_2\dots i_p),\;
p\geq 2$, dans $S_n$ et toute permutation $\sigma \in S_n$, 
$$
\sigma (i_1\, i_2\dots i_p)\sigma^{-1}=(\sigma(i_1)\, \sigma(i_2)\dots
\sigma(i_p))\, .
$$}
    \item \question{Vérifier que pour tous entiers distincts $i,j\in \lbrace 1,2,\dots
n\rbrace$ on a $(i\, j)=(i\, k)(j\, k)(i\, k)$.}
    \item \question{En déduire que les familles de transpositions $\lbrace (1\, i)\mid i\in
\lbrace 1,2,\dots n\rbrace \rbrace$ et $\lbrace (i\, i+1)\mid i\in
\lbrace 1,2,\dots n-1\rbrace \rbrace$ engendrent $S_n$.}
    \item \question{Soient $\tau =(1\, 2)$ et $c=(1\, 2\, \dots n)$. Calculer $c^{i-1}\tau
c^{1-i}$ pour tout $i\in \lbrace 1,2,\dots n-1\rbrace$. Montrer que toute
permutation de $S_n$ s'écrit comme un produit de puissances de $\tau$ et
$c$.}
\end{enumerate}
}

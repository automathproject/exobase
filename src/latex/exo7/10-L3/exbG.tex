\uuid{exbG}
\exo7id{2306}
\titre{exo7 2306}
\auteur{barraud}
\organisation{exo7}
\datecreate{2008-04-24}
\isIndication{false}
\isCorrection{true}
\chapitre{Anneau, corps}
\sousChapitre{Anneau, corps}

\contenu{
\texte{

}
\begin{enumerate}
    \item \question{D\'etermin\'er la structure des anneaux 
quotients suivants:
$$ \Zz_2[x]/(x^3+x^2+x+1),\quad\Zz[x]/(x^2-1),\quad  \Qq[x]/(x^8-1). $$}
    \item \question{Consid\'erons l'anneau quotient $K[x]/(f^ng^m)$
o\`u $f$ et  $g$ sont deux polyn\^omes  distincts irr\'eductibles sur
le corps $K$. D\'ecrirer  les diviseurs  de z\'ero et 
les \'el\'ements nilpotents de l'anneau  $K[x]/(f^ng^m)$.}
    \item \question{Quels id\'eaux a-t-il cet anneau ?}
    \item \question{Soit $K$ le corps fini \`a $p$ \'el\'ements.
Trouver le nombre des \'el\'ements du groupe multiplicatif
de l'anneau $K[x]/(f^mg^l)$.}
    \item \question{Donner une g\'en\'eralisation de la question 4) dans le cas
du produit de  $n$ polyn\^omes irr\'eductibles sur un corps fini $K$
\`a $q$ \'el\'ements.}
\reponse{
En posant $y=x+1$, on a
$\Zz_{2}[x]/(x^{3}+x^{2}+x+1)=\{0,1,x,y,x^{2},y^{2},xy,xy+1\}$.
Les tables des opérations sont les suivantes (elles sont symétriques)~:
$$
\begin{array}{c|c|c|c|c|c|c|c|c|c|c|}
\oplus    &   0 &   1 &   x &   y & x^2  & y^2  & xy   & xy+1\\\hline
0         &   0 &   1 &   x &   y & x^2  & y^2  & xy   & xy+1\\\hline
1         &     &   0 &   y &   x & y^2  & x^2  & xy+1 & xy  \\\hline
x         &     &     &   0 &   1 & xy   & xy+1 & x^2  & y^2 \\\hline
y         &     &     &     &   0 & xy+1 & xy   & y^2  & x^2 \\\hline
x^2       &     &     &     &     & 0    & 1    & x    & y   \\\hline
y^2       &     &     &     &     &      & 0    & y    & x   \\\hline
xy        &     &     &     &     &      &      & 0    & 1   \\\hline
xy+1      &     &     &     &     &      &      &      & 0   \\\hline
\end{array}
$$

$$
\begin{array}{c|c|c|c|c|c|c|c|c|c|c|}
\otimes   &   0 &   1 &   x &   y & x^2  & y^2  & xy   & xy+1\\\hline
0         &   0 &   0 &   0 &   0 & 0    & 0    & 0    & 0   \\\hline
1         &     &   1 &   x &   y & x^2  & y^2  & xy   & xy+1\\\hline
x         &     &     & x^2 &  xy & xy+1 & y^2  & y    & 1   \\\hline
y         &     &     &     & y^2 &  y   & 0    & y^2  & xy  \\\hline
x^2       &     &     &     &     &  1   & y^2  & xy   & x   \\\hline
y^2       &     &     &     &     &      & 0    & 0    & y^2 \\\hline
xy        &     &     &     &     &      &      & y^2  & y   \\\hline
xy+1      &     &     &     &     &      &      &      & x^2 \\\hline
\end{array}
$$
 
Pour $\Zz[x]/(x^{2}-1)$, $(x-1)$ et $(x+1)$ sont deux idéaux étrangers,
et le lemme chinois nous donne $\Zz[x]/(x^{2}-1)\simeq
\Zz[x]/(x-1)\times\Zz[x]/(x+1)$. Or $\Zz[x]/(x+1)\simeq\Zz$ et
$\Zz[x]/(x-1)\simeq\Zz$ donc $\Zz[x]/(x^{2}-1)\simeq\Zz\times\Zz$.


La factorisation de $(x^{8}-1)$ sur $\Qq$ est
$(x^{8}-1)=(x-1)(x+1)(x^{2}+1)(x^{4}+1)$. En utilisant le lemme chinois,
on obtient que $\Qq[x]/(x^{8}-1)\simeq \Qq[x]/(x+1)\times
\Qq[x]/(x^{2}+1)\times \Qq[x]/(x^{4}+1)$ soit~:

$$
\Qq[x]/(x^{8}-1)\simeq
\Qq\times\Qq\times\Qq[i]\times\Qq[e^{i\pi/4}].
$$

Montrons en effet que $\Qq[x]/(x^{2}+1)\simeq\Qq[i]$~: l'application
$\phi:\Qq[x]/(x^{2}+1)\to \Qq[i]$ définie par $\bar P\mapsto P(i)$ est un
morphisme d'anneau.
\begin{itemize}
injectivité~:
  Soit $\bar{P}\in\ker\phi$. Alors $P(i)=0$. Comme $P$ est à coefficient
  rationnels donc réels, $-i$ est aussi raine de $P$. Donc $x^{2}+1|P$.
surjectivité~:
  Soit $z=a+ib\in\Qq[i]$. Alors $z=\phi(ax+b)$.
\end{itemize}

\smallskip

De même pour $\Qq[x]/(x^{4}+1)\simeq\Qq[e^{i\pi/4}]$. Considérons le
morphisme $\phi~:\Qq[x]/(x^{4}+1)\to\Qq[e^{i\pi/4}]$ défini par
$\phi(\bar P)=P(e^{i\pi/4})$. $\phi$ est bien définie, c'est un morphisme
d'anneau.
\begin{itemize}
injectivité~:
  Soit $\bar{P}\in\ker\phi$. Alors $P(e^{i\pi/4})=0$. Par ailleurs
  $X^{4}+1$ est \emph{irréductible} dans $\Qq$~: sa factorisation sur
  $\Rr$ est $(x^{2}+\sqrt{2}x+1)(x^{2}-\sqrt{2}x+1)$, et aucun de ces
  deux polynômes, même à inversible réel près, n'est rationnel. On en
  déduit que si $(x^{4}+1)$ ne divise pas $P$, alors $\pgcd(X^{4}+1,P)=1$.
  Il existerait donc $U,V\in\Qq[x]$, $UP+V(X^{4}+1)=1$. En évaluant en
  $x=e^{i\pi/4}$, on obtient une contradiction. Donc $X^{4}+1|P$. (cf.
  exexercice \ref{ex:bar9}).
surjectivité~:
  Soit $z=a+be^{i\pi/4}\in\Qq[e^{i\pi/4}]$. Alors $z=\phi(ax+b)$.
\end{itemize}
On a $K[x]/(f^{n}g^{m})\simeq K[x]/(f^{m})\times K[x]/(g^{m})$. On en
  déduit que les diviseurs de $0$ sont les polynômes de la forme
  $\bar{P}$ où $P$ satisfait l'une des conditions suivantes~:
$$
\left|
\begin{array}{ll}
  f^{n}|P \text{ et } g^{m}\!\!\!\not| P & \qquad(\{0\}\times K[x]/(g^{m})\setminus\{0\})\\
  g^{m}|P \text{ et } f^{n}\!\!\!\not| P & \qquad(K[x]/(f^{n})\setminus\{0\}\times \{0\})\\
  f|P \text{ et } f^{n}\!\!\!\not| P & \qquad (\mathcal{D}_{K[x]/(f^{n})}\times K[x]/(g^{m}))\\
  g|P \text{ et } g^{m}\!\!\!\not| P & \qquad (K[x]/(f^{n})\times\mathcal{D}_{ K[x]/(g^{m})})
\end{array}\right.
$$
Les nilpotents sont donnés par les conditions
$$
\left\{
\begin{array}{ll}
  fg|P\\
  (f^{n}g^{m}\!\!\!\not| P \text{ si on veut exclure }0)
\end{array}\right.
$$
Les idéaux de $K[x]/(f^{n})$ sont les idéaux engendrés par les
   diviseurs de $f^{n}$ soit les $f^{k}$ pour $0\leq k\leq n$.

   \smallskip

   La démonstration peut se faire en toute généralité exactement de la
   même manière que dans $\Zz/n\Zz$~: Soit $\mathcal{D}$ l'ensemble des
   diviseurs de $f^{n}$ (modulo $K^{*}$). Ici, $\mathcal{D}=\{f^{k},0\leq
   k\leq n\}$. Soit $\mathcal{I}$ l'ensemble de idéaux de $K[x]/(f^{n})$.

   On a une flèche de $\mathcal{D}\to\mathcal{I}$, donnée par
   $d\mapsto(\bar{d})$.
   \begin{itemize}
surjectivité
     Soit $I\in\mathcal{I}$. $I$ est principal~: notons $I=(\bar{h})$.
     Soit $d=\pgcd(f,h)$, et $h_{1}$ le polynôme déterminé par
     $h=dh_{1}$. Alors $\pgcd(f,h_{1})=0$ et $h_{1}$ est inversible dans
     le quotient. On en déduit que $(\bar{h})=(\bar{d})=I$ (or $d\in\mathcal{D}$).
injectivité
     Soit $d,d'\in\mathcal{D}$ tels que $(\bar{d})=(\bar{d'})$. On a
     alors $d=h_{1}d'+h_{2}f$ donc $d'|d$. De même, $d|d'$. On en déduit
     que $d\sim d'$.
   \end{itemize}

   \smallskip
   
   Revenons à notre exercice~: les idéaux de $K[x]/(f^{n})\times
   K[x]/g^{m}$ sont donc de la forme $(f^{\alpha})\times(g^{\beta})$. En
   revenant à $K[x]/(f^{n}g^{m})$, on obtient que l'ensemble des idéaux
   est $$\{(f^{\alpha}g^{\beta}), 0\leq\alpha,\beta\leq n\}$$
Les inversibles de $K[x]/(f^{n})$ sont les (classes des) polynômes
   premiers avec $f$. Le complémentaire est donc formé des multiples de
   $f$, il y en a donc autant que de polynômes de degré $(nd-1)-d$ où $d$
   est le degré de $f$, soit $p^{(n-1)d}$. Il y a donc $p^{(n-1)d}(p-1)$
   inversibles dans $K[x]/(f^{n})$.

   On en déduit qu'il y en a $p^{(n-1)d_{f}+(m-1)d_{g}}(p-1)^{2}$ dans
   $K[x]/(f^{n}g^{m})$, où $d_{f}$ et $d_{g}$ sont les degrés respectifs
   de $f$ et $g$.
Plus généralement, si les $f_{i}$ sont des polynômes irréductibles
   distincts, dans $K[x]/(f_{1}^{n_{1}}\cdots f_{k}^{n_{k}})$ il y a
   $p^{\sum (n_{i}-1)d_{i}}(p-1)^{k}$ inversibles, où $d_{i}$ est le
   degré de $f_{i}$.
}
\end{enumerate}
}

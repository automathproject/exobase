\uuid{dYnM}
\exo7id{2314}
\titre{exo7 2314}
\auteur{barraud}
\organisation{exo7}
\datecreate{2008-04-24}
\isIndication{false}
\isCorrection{true}
\chapitre{Anneau, corps}
\sousChapitre{Anneau, corps}

\contenu{
\texte{
Soient $f=x^6+x^5+x^4+x^3+1\in \Zz_2[x]\ $, $g=x^3+x^2+1 \in
  \Zz_2[x]$ deux polyn\^omes sur le corps $\Zz_2$.
}
\begin{enumerate}
    \item \question{En utilisant l'algorithme d'Euclide trouver le p.g.c.d. de $f$ et
  $g$ et sa repr\'esentation lin\'eaire.}
\reponse{\begin{align*}
%(x^{6}+x^{5}+x^{4}+x^{3}+1)&=(x^{3}+x^{2}+1)(x^{3}+x+1)+(x^{2}+x)\\
%(x^{3}+x^{2}+1)&=(x^{2}+x)x+1
f&=g(x^{3}+x+1)+(x^{2}+x)\\
g&=(x^{2}+x)x+1
\end{align*}
donc $\pgcd(f,g)=1$ et
$$
1=g-(x^{2}+x)x=g-(f-g(x^{3}+x+1))x=(x^{4}+x^{2}+x+1)g-xf
$$}
    \item \question{Les polyn\^omes $f$ et $g$ sont-ils irr\'eductibles ?}
\reponse{$f=(x^{4}+x+1)(x^{2}+x+1)$ donc $f$ n'est pas irréductible.

$g$ est de degré $3$ et n'a pas de racine, donc $g$ est irréductible.}
    \item \question{Soit $(g)$ l'id\'eal principal engendr\'e par $g$.  Combien
  d'\'el\'ements contient l'anneau quotient $A=\Zz_2[x]/(g)$ ?}
\reponse{Les éléments de $A$ sont en bijection avec les polynômes de $\Zz_{2}[x]$
de degré $<\deg(g)=3$. Il y a $8$ polynômes de degré au plus $2$ sur
$\Zz_{2}$, donc $A$ a $8$ éléments.}
    \item \question{Soit $\pi\,:\,\Zz_2[x]\to A$ la projection canonique.  On pose
  $\pi(x)=\bar x\in A$. Trouver l'inverse de l'\'el\'ement $\pi(f)$
  dans l'anneau quotient $A$.}
\reponse{On utilise la représentation linéaire $uf+vg=1$ de $\pgcd(f,g)$ obtenue
plus haut. $uf=1+vg$, donc $\bar{u} \bar{f}=\bar{1}+\bar{0}=\bar{1}$.
Donc $(\bar{f})^{-1}=\bar{u}=\bar{x}$.}
    \item \question{L'anneau quotient $B=\Zz_2[x]/(f)$ est-il un corps ?  Justifier
  la r\'eponse, i.e. donner une d\'emonstration si $B$ est un corps ou
  trouver un \'el\'ement non-inversible dans $B$ dans le cas
  contraire.}
\reponse{Soit $f_{1}=x^{2}+x+1$ et $f_{2}=x^{4}+x+1$. Alors $f_{1}f_{2}=f$ donc
$\bar{f_{1}}\bar{f_{2}}=\bar{0}$. Pourtant, $f$ ne divise ni $f_{1}$ ni
$f_{2}$, donc $\bar{f_{1}}\neq \bar{0}$ et $\bar{f_{2}}\neq \bar{0}$~:
 $B$ n'est pas intègre, donc $B$ n'est pas un corps.}
\end{enumerate}
}

\uuid{jk0G}
\exo7id{2313}
\auteur{barraud}
\organisation{exo7}
\datecreate{2008-04-24}
\isIndication{false}
\isCorrection{true}
\chapitre{Anneau, corps}
\sousChapitre{Anneau, corps}

\contenu{
\texte{
Soit $p$ un nombre premier impair. Montrer que la congruence
  \newline\noindent $x^2\equiv-1 \mod p\ $ a une solution si et
  seulement si $p\equiv 1 \mod 4$.
}
\reponse{
Soit $p$ un nombre premier impair. Notons $p=2m+1$. On a
$$
 (m!)^{2}\equiv(-1)^{m+1} [p]
$$
en effet, (modulo $p$)~:
\begin{align*}
(p-1)!
&=\prod_{k=1}^{2m}k=m!\prod_{k=1}^{m}(m+k)\\
&=m!\prod_{k=1}^{m}(m+k-p)=m!\prod_{k=1}^{m}(-k)\\
&=(-1)^{m}(m!)^{2}
\end{align*}
Or, dans $\Zz_{p}[x]$, $1^{-1}=1$ et $(p-1)^{-1}=p-1$,  donc $\forall
k\in\{2,...,p-2\}$, $k^{-1}\in\{2,...,p-2\}$. Ainsi,
$\prod_{k=2}^{p-1}k\equiv1 [p]$, et donc $(p-1)!\equiv -1 [p]$. D'où le
résultat.

\begin{itemize}
\item 
  Si $p\equiv 1[4]$, $(-1)^{m+1}=-1$, et donc $m!$ est une solution de
  $x^{2}\equiv-1[p]$.
  
\item
  Si cette équation a une solution, alors $x^{2m}\equiv 1[p]$, et comme
  $x^{p-1}\equiv 1[p]$, $1\equiv (-1)^{m}[p]$. On en déduit que $m$ est
  pair, donc $p\equiv1[4]$.
\end{itemize}
}
}

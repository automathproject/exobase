\uuid{ZSpB}
\exo7id{7818}
\auteur{mourougane}
\organisation{exo7}
\datecreate{2021-08-11}
\isIndication{false}
\isCorrection{false}
\chapitre{Forme bilinéaire}
\sousChapitre{Forme bilinéaire}

\contenu{
\texte{
Soit $k$ un corps de caractéristique différente
de $2$.
Soit $E$ un $k$-espace vectoriel de dimension $n\geq 2$ et $\phi$ une
forme sesquilinéaire
non dégénérée sur $E$ symétrique, hermitienne ou alternée. Soit
$\tau$ une transformation de $E$ donnée à l'aide d'une forme linéaire
non nulle 
$f$ sur $E$ et un vecteur $a$ de $Ker f$ par $\forall x\in E, \tau
(x)=x+f(x)a$.
}
\begin{enumerate}
    \item \question{Déterminer la nature de $\tau$.}
    \item \question{On suppose désormais que $\tau$ est une isométrie relativement à
 $\phi$. Montrer que $a$ est isotrope.}
    \item \question{Montrer que $f$ et $\phi (\cdot , a)$ sont proportionnelles. On
 notera $\lambda\in k^\star$ tel que $f=\lambda\phi (\cdot , a)$.}
    \item \question{Montrer que si $\sigma\not =Id$ et $\phi$ est hermitienne ou
 symétrique, alors $\lambda +\sigma (\lambda)=0$.}
    \item \question{Montrer qu'il n'existe pas de transvections orthogonales, qu'il
 existe des transvections unitaires si et seulement si l'indice est
 plus grand que $1$ et qu'il existe toujours des transvections
 symplectiques.}
\end{enumerate}
}

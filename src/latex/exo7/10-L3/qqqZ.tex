\uuid{qqqZ}
\exo7id{2276}
\titre{exo7 2276}
\auteur{barraud}
\organisation{exo7}
\datecreate{2008-04-24}
\isIndication{false}
\isCorrection{true}
\chapitre{Polynôme}
\sousChapitre{Polynôme}
\module{Algèbre et théorie des nombres}
\niveau{L3}
\difficulte{}

\contenu{
\texte{
Soient $A=\Zz [\sqrt{-3}]$ et $K$ son corps de fractions. 
Montrer que $x^2-x+1$ est irr\'eductible dans $A[x]$ sans 
pour autant \^etre irr\'eductible dans $K[x]$. Expliquer la contradiction
apparente avec le corollaire du lemme de Gauss.
}
\reponse{
Soit $P=x^{2}-x+1$. Si $P$ a une factorisation non triviale, $P$ est
  divisible par un polynôme de degré $1$, et comme $P$ est unitaire, ce
  diviseur peut être choisi unitaire~: on en déduit que $P$ a une racine.
  On calcule $P(a+bi\sqrt{3})=(a^{2}-3b^{2}-a+1)+(2ab-b)i\sqrt{3}$. Comme
  $1/2\notin A=\Zz[i\sqrt{3}]$, $2a-1\neq 0$, donc si
  $P(a+bi\sqrt{3})=0$, alors $b=0$, et $P(a)=0$. Mais $x^{2}-x+1$ est
  primitif et se réduction modulo $2$ est irréductible, donc il est
  irréductible sur $\Zz[x]$. En particulier il n'a pas de racine dans
  $\Zz$. On en déduit que $P$ n'a pas de racine sur $A$, et est donc
  irréductible.

  \medskip
  Soit $K=\mathrm{frac}(A)=\Qq[i\sqrt{3}]$. On a
  $P(\frac{1+i\sqrt{3}}{2})=0$ donc $P$ a une racine dans $K$, donc $P$
  est réductible sur $K$.
}
}

\uuid{x2t9}
\exo7id{2278}
\titre{exo7 2278}
\auteur{barraud}
\organisation{exo7}
\datecreate{2008-04-24}
\isIndication{false}
\isCorrection{true}
\chapitre{Polynôme}
\sousChapitre{Polynôme}

\contenu{
\texte{

}
\begin{enumerate}
    \item \question{Soit $P\in \Zz[x]$. Soit $\dfrac{a}{b}$
 sa racine rationnelle : $P(\dfrac{a}{b})=0$, $\pgcd(a,b)=1$.
Montrer que $\forall\,k\in \Zz$ $\ (a-bk)$ divise $P(k)$.}
    \item \question{Quelles racines rationnelles ont les polyn\^omes
$f(x)=x^3-6x^2+15x-14$ et $g(x)=2x^3+3x^2+6x-4$ ?}
\reponse{
$(X-\frac{a}{b})|P$ donc $\exists Q\in\Qq[x], \,
    P=(x-\frac{a}{b})Q=(bx-a)\frac{Q}{b}$. En réduisant tous les
    coefficients de $Q$ au même dénominateur, on peut mettre $Q$ sous la
    forme~: $Q=\frac{1}{m}Q_{1}$, avec $Q_{1}\in\Zz[X]$ primitif. Alors
    $bdP=(bx-a)Q_{1}$. En considérant les contenus de ces polynômes, on a
    $c(bx-a)=\pgcd(a,b)=1$, $c(Q_{1})=1$ donc $c(bdP)=bd\,c(P)=1$. Ainsi
    $bd=\pm1$, et $(bx-a)|P$.
On considère par exemple les cas $k=0,\dots,3$. (Pour $k=2$, on
    constate que $P(2)=0$~: on peut diviser $P$ par $(X-2)$ et déterminer
    les trois racines complexes de $P$...). On obtient que
    \begin{align*}
    (*)&&\qquad       a&|14       &\quad   (k=0),      \\
    (**)&&\qquad   (a-b)&|4       &\quad   (k=1),      \\
    (***)&&\qquad  (a-3b)&|2^{3}5 &\quad   (k=3).      \\
    \end{align*}
    Au passage On peut remarquer que si $\alpha\leq0$, $P(\alpha)<0$, donc on
    peut supposer $a>0$ et $b>0$.
    \begin{itemize}
Si $a=1$~: $(**)\Rightarrow b\in\{2,3,5\}$. Aucune de ces
      possibilités n'est compatible avec $(***)$.
Si $a=2$~: $(**)\Rightarrow b\in\{1,3,4,6\}$. Comme $\pgcd(a,b)=1$,
      $4$et $6$ sont exclus. $3$ n'est pas compatible avec $(***)$. Pour
      $2$, on vérifie que $P(2)=0$.
Si $a=7$~: $(**)\Rightarrow b\in\{3,5,9,11\}$. Mais aucune de ces
      solution ne convient.
Si $a=14$~: $(**)\Rightarrow b\in\{10,12,16,18\}$ mais
      $\pgcd(a,b)=1$ exclu toutes ces possibilités.
    \end{itemize}
    Finalement, $2$ est la seule racine rationnelle de $P$.
}
\end{enumerate}
}

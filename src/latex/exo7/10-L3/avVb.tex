\uuid{avVb}
\exo7id{2253}
\titre{exo7 2253}
\auteur{barraud}
\organisation{exo7}
\datecreate{2008-04-24}
\isIndication{true}
\isCorrection{true}
\chapitre{Anneau, corps}
\sousChapitre{Anneau, corps}

\contenu{
\texte{
\label{ex:bar5}
D\'emontrer que tout anneau int\`egre fini est un corps.
}
\indication{Voir la solution de l'exercice \ref{ptfermat}, deuxième question.}
\reponse{
Soit $a\in A\setminus\{0\}$. Soit $\phi_{a}:A\to A, x\mapsto ax$. Si
  $\phi_{a}(x)=\phi_{a}(y)$, alors $ax=ay$. 
Mais $ax=ay $ ssi $a(x-y)=0$, or $a\neq 0$ et $A$ est int\`egre, donc $x=y$.
%donc $a^{-1}ax=a^{-1}ay$ et $x=y$. 
Ainsi $\phi_{a}$ est injective de $A$ dans $A$. Comme $A$ est
  fini, elle est donc aussi surjective~: $\exists x\in A, \phi_{a}(x)=1$.
}
}

\uuid{M2ug}
\exo7id{4770}
\auteur{quercia}
\organisation{exo7}
\datecreate{2010-03-16}
\isIndication{false}
\isCorrection{false}
\chapitre{Topologie}
\sousChapitre{Topologie des espaces vectoriels normés}

\contenu{
\texte{
$E=C([0,1],\R)$. Soit $g\in E$. Pour tout $f\in E$ on pose
$N(f)=\sup\limits_{x\in [0,1]}\{|f(x)g(x)|\}$.
}
\begin{enumerate}
    \item \question{Donner une condition n{\'e}cessaire et suffisante sur $g$ pour que $N$ soit une
norme sur $E$.}
    \item \question{Si pour tout $x\in [0,1]$, $g(x)\ne 0$, montrer qu'alors $N$ et $\|\ \|_{\infty}$
sont des normes sur $E$ {\'e}quivalentes.}
    \item \question{D{\'e}montrer la r{\'e}ciproque de la proposition pr{\'e}c{\'e}dente.}
\end{enumerate}
}

\uuid{Waz7}
\exo7id{1751}
\titre{exo7 1751}
\auteur{maillot}
\organisation{exo7}
\datecreate{2001-09-01}
\isIndication{false}
\isCorrection{false}
\chapitre{Topologie}
\sousChapitre{Borne supérieure}
\module{Analyse}
\niveau{L2}
\difficulte{}

\contenu{
\texte{
Soient $A$ et $B$ deux parties non vides et majorées de $\R$. On définit
:
$$
A+B=\{c\in\R\ | \ \exists a\in A,\exists b\in B, c=a+B\}.
$$
}
\begin{enumerate}
    \item \question{Montrer que $A+B$ admet une borne supérieure, puis que
$\sup(A+B)=\sup A+\sup B$.}
    \item \question{Montrer l'implication :
$$
\exists M\in\R\;\forall x\in A,\forall y\in B,\;x+y<M\Rightarrow\sup A+\sup B\leq M.
$$}
\end{enumerate}
}

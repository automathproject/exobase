\uuid{RPTM}
\exo7id{4513}
\titre{exo7 4513}
\auteur{quercia}
\organisation{exo7}
\datecreate{2010-03-14}
\isIndication{false}
\isCorrection{true}
\chapitre{Suite et série de fonctions}
\sousChapitre{Convergence simple, uniforme, normale}

\contenu{
\texte{
On pose $f_0(t) = 0$, $f_{n+1}(t) = \sqrt{t+f_n(t)}$, pour $t \ge 0$.
}
\begin{enumerate}
    \item \question{Déterminer la limite simple, $\ell$, des fonctions $f_n$.}
    \item \question{Y a-t-il convergence uniforme sur $\R^+$ ?}
    \item \question{Démontrer que : $\forall\ t > 0,\
    |f_{n+1}(t) - \ell(t)| \le \frac {|f_n(t) - \ell(t)|}{2f_{n+1}(t)}$.}
    \item \question{En déduire que la suite $(f_n)$ converge uniformément sur tout intervalle
    $[a,+\infty[$, avec $a>0$.
    (Remarquer que $f_n - \ell$ est bornée pour $n \ge 1$)}
\reponse{
$\ell(t) = \begin{cases} 0 \text{ si } t=0 \cr
                       \frac {1+\sqrt{1+4t}}2 \text{ si } t >0.\cr\end{cases}$
Accroissements finis.
}
\end{enumerate}
}

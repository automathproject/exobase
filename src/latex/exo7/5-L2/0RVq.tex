\uuid{0RVq}
\exo7id{4736}
\titre{exo7 4736}
\auteur{quercia}
\organisation{exo7}
\datecreate{2010-03-16}
\isIndication{false}
\isCorrection{true}
\chapitre{Topologie}
\sousChapitre{Topologie des espaces métriques}
\module{Analyse}
\niveau{L2}
\difficulte{}

\contenu{
\texte{
Soient $E$, $F$ deux espaces vectoriels norm{\'e}s et $f : E \to {F.}$

Montrer que $f$ est continue 
si et seulement si : $\forall\ A \subset E,\ f(\overline A) \subset \overline{f(A)}$\par
si et seulement si : $\forall\ B \subset F,\ f^{-1}(\mathring B) \subset f^{-1}(B)^\circ$.\par
}
\reponse{
Si $f$ est continue : 
soit $x \in \overline A$ : $x = \lim a_n  \Rightarrow  f(x) = \lim f(a_n) \in \overline{f(A)}$.\par
soit $x \in f^{-1}(B)^\circ$ : $f(x) \in \mathring B  \Rightarrow  \exists\ B(f(x),r) \subset B$,
$\exists\ \delta > 0 \text{ tq } f(B(x,\delta)) \subset B(f(x),r)$\par
$ \Rightarrow  B(x,\delta) \subset f^{-1}(B)$.

si $f(\overline A) \subset \overline{f(A)}$ : 
soit $B \subset F$ ferm{\'e} et $A = f^{-1}(B)$ :
$f(\overline A) \subset B$ donc $\overline A \subset A$.

si $f^{-1}(\mathring B) \subset f^{-1}(B)^\circ$ : 
soit $B \subset F$ ouvert et $A = f^{-1}(B)$ :
$\mathring A \supset f^{-1}(\mathring B) = A$.
}
}

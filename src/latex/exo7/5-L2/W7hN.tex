\uuid{W7hN}
\exo7id{4145}
\titre{exo7 4145}
\auteur{quercia}
\organisation{exo7}
\datecreate{2010-03-11}
\isIndication{false}
\isCorrection{true}
\chapitre{Fonction de plusieurs variables}
\sousChapitre{Dérivée partielle}
\module{Analyse}
\niveau{L2}
\difficulte{}

\contenu{
\texte{
Soit $f$ une application de classe $\mathcal{C}^2$ de $\R^{+*}$ dans $\R$.

On définit une application $F$ de $\R^n\setminus\{\vec 0\}$ dans $\R$ par :
$F(x_1,\dots,x_n) = f(\sqrt{x_1^2 + \dots + x_n^2}\,)$.

Calculer le laplacien de $F$ en fonction de $f$.
}
\reponse{
$\Delta F = \frac{n-1}r f'(r) + f''(r)$.
}
}

\uuid{HuGe}
\exo7id{4192}
\titre{exo7 4192}
\auteur{quercia}
\organisation{exo7}
\datecreate{2010-03-11}
\isIndication{false}
\isCorrection{true}
\chapitre{Fonction de plusieurs variables}
\sousChapitre{Extremums locaux}
\module{Analyse}
\niveau{L2}
\difficulte{}

\contenu{
\texte{
Soit $A \in \R^p$ fixé et $f : {\R^p} \to \R, M \mapsto {AM^2}$
$g : {\R^p} \to \R, M \mapsto {AM}$
(distance euclidienne)
}
\begin{enumerate}
    \item \question{Calculer les gradients de $f$ et $g$ en un point $M$.}
    \item \question{Soient $A,B,C$ trois points non alignés du plan. Trouver les points $M$
    du plan réalisant le minimum de :
  \begin{enumerate}}
    \item \question{$MA^2 + MB^2 + MC^2$.}
    \item \question{$MA   + MB   + MC  $.}
    \item \question{$MA\times MB\times MC$.}
\reponse{
\begin{enumerate}
isobarycentre de $ABC$.
Point de Fermat ou $A,B,C$.
$A,B,C$.
}
\end{enumerate}
}

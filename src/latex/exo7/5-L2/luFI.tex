\uuid{luFI}
\exo7id{4760}
\titre{exo7 4760}
\auteur{quercia}
\organisation{exo7}
\datecreate{2010-03-16}
\isIndication{false}
\isCorrection{true}
\chapitre{Topologie}
\sousChapitre{Topologie des espaces vectoriels normés}

\contenu{
\texte{
Soit $a\in\R$. On pose pour $P\in\R[X]$~: $N_a(P) = |P(a)| +  \int_{t=0}^1 |P'(t)|\,d t$.
Montrer que\dots
}
\begin{enumerate}
    \item \question{$N_a$ est une norme.}
    \item \question{$N_0$ et $N_1$ sont {\'e}quivalentes.}
    \item \question{Si $a,b\in{[0,1]}$, alors $N_a$ et $N_b$ sont {\'e}quivalentes.}
    \item \question{Soit $P_n = (X/2)^n$. D{\'e}terminer pour quelles normes $N_a$ la suite
    $(P_n)$ est convergente et quelle est sa limite.}
    \item \question{Si $0 \le a < b$ et $b > 1$
    alors aucune des normes $N_a$, $N_b$ n'est plus fine que l'autre.}
\reponse{
$(P_n)$ converge vers $0$ pour $a\in{]-2,2[}$ et vers $1$ pour $a=2$.
    La suite est non born{\'e}e si $|a| > 2$~; elle est born{\'e}e divergente pour $a=-2$.
$(X/b)^n$ converge vers $1$ pour $N_b$ et vers $0$ pour $N_a$.
}
\end{enumerate}
}

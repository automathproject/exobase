\uuid{jNul}
\exo7id{4553}
\titre{exo7 4553}
\auteur{quercia}
\organisation{exo7}
\datecreate{2010-03-14}
\isIndication{false}
\isCorrection{true}
\chapitre{Suite et série de fonctions}
\sousChapitre{Autre}

\contenu{
\texte{
Soit $S(t) = \sum_{n=1}^\infty \frac{t^n}{1-t^n}$.
}
\begin{enumerate}
    \item \question{Pour quelles valeurs de $t$, $S$ est-elle définie~? Est-elle continue~?}
\reponse{$-1<t<1$.}
    \item \question{Montrer qu'au voisinage de~$1^-$ on a $S(t) = -\frac{\ln(1-t)}{1-t} + O\Bigl(\frac1{1-t}\Bigr)$.
On pourra développer $\ln(1-t)$ en série entière.}
\reponse{Pour $0\le t<1$ et $n\ge 2$ on a~:
\begin{align*}
(1-t)\frac{t^n}{1-t^n}
&= \frac{t^n}{1+t+\dots+t^{n-1}}\cr
&= \frac{t^n}{n} + \frac{t^n((1-t) + (1-t^2) + \dots + (1-t^{n-1}))}{n(1+t+\dots+t^{n-1})}\cr
&= \frac{t^n}{n} + \frac{(t^n-t^{n+1})((n-1) + (n-2)t + \dots + t^{n-2})}{n(1+t+\dots+t^{n-1})}\cr
\end{align*}
d'où $0 \le (1-t)\frac{t^n}{1-t^n} - \frac{t^n}{n} \le \frac{n-1}{n}(t^n-t^{n+1}) \le t^n-t^{n+1}$
(vrai aussi si $n=1$) et en sommant~:
$$0\le (1-t)S(t) + \ln(1-t) \le 1.$$}
\end{enumerate}
}

\uuid{anrw}
\exo7id{4186}
\titre{exo7 4186}
\auteur{quercia}
\organisation{exo7}
\datecreate{2010-03-11}
\isIndication{false}
\isCorrection{true}
\chapitre{Fonction de plusieurs variables}
\sousChapitre{Dérivée partielle}

\contenu{
\texte{
Soit une fonction $f$ de classe $\mathcal{C}^2$ sur le disque unité du plan, telle que son laplacien
$\frac{\partial^2 f}{\partial x^2} + \frac{\partial^2 f}{\partial y^2}$ soit nul.
}
\begin{enumerate}
    \item \question{Montrer $ \int_{\theta=0}^{2\pi}f(r\cos\theta,r\sin\theta)\,d\theta$
    ne dépend pas de $r\in[0,1]$.}
\reponse{On pose $g(r,\theta) = f(r\cos\theta,r\sin\theta)$,
    $h(r) =  \int_{\theta=0}^{2\pi}f(r\cos\theta,r\sin\theta)\,d\theta$ et l'on a
    $0=\Delta f = \frac{\partial^2 g}{\partial r^2} + \frac1r\frac{\partial g}{\partial r} + \frac1{r^2}\frac{\partial^2 g}{\partial \theta^2}$
    d'où~:
    $$0=h''(r) + \frac1rh'(r) + \frac1{r^2}\Bigl[\frac{\partial g}{\partial \theta}(r,\theta)\Bigr]_{\theta=0}^{2\pi}$$
    Le crochet est nul par $2\pi$-périodicité de $g$ donc $h''(r) + \frac1rh'(r) = 0$
    soit $h'(r) = \frac Kr$ et $K=0$ par continuité de $h'$ en $0$.}
    \item \question{Calculer alors $\iint_{D_r} f(x,y)\,d xd y$ $D_r$ étant le disque fermé
    de centre $0$ et de rayon $r$.}
\reponse{$\pi r^2f(0,0)$.}
\end{enumerate}
}

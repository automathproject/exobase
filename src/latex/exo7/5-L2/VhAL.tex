\uuid{VhAL}
\exo7id{4822}
\auteur{quercia}
\organisation{exo7}
\datecreate{2010-03-16}
\isIndication{false}
\isCorrection{true}
\chapitre{Topologie}
\sousChapitre{Topologie des espaces vectoriels normés}

\contenu{
\texte{
Soit~$u$ une application lin{\'e}aire de~$\R^n$ dans~$\R^m$. Prouver que
$u$ est surjective si et seulement si elle transforme tout ouvert
de~$\R^n$ en ouvert de~$\R^m$.
}
\reponse{
Si $u(\mathring B(0,1))$ est ouvert alors il engendre $\R^m$ donc
$u$ est surjective.

Si $u$ est surjective, soit $A=u(\mathring B(0,1))$. $A$ est convexe,
born{\'e}, sym{\'e}trique par rapport {\`a}~$0$ et la r{\'e}union des homoth{\'e}tiques
de~$A$ est {\'e}gale {\`a}~$\R^m$~; la jauge associ{\'e}e {\`a}~$A$ est une norme
sur~$\R^m$ {\'e}quivalente {\`a} l'une des normes usuelles donc
$A$ contient une boule de centre $0$ et, par homoth{\'e}tie-translation, tout
ouvert de~$\R^n$ a une image ouverte dans~$\R^m$.
}
}

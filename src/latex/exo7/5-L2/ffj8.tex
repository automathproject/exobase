\uuid{ffj8}
\exo7id{5741}
\titre{exo7 5741}
\auteur{rouget}
\organisation{exo7}
\datecreate{2010-10-16}
\isIndication{false}
\isCorrection{true}
\chapitre{Suite et série de fonctions}
\sousChapitre{Suites et séries d'intégrales}
\module{Analyse}
\niveau{L2}
\difficulte{}

\contenu{
\texte{
Calculer $\int_{0}^{+\infty}\frac{x}{\sh x}\;dx$ en écrivant cette intégrale comme somme d'une série.
}
\reponse{
C'est presque le même exercice que l'exercice \ref{ex:rou3bis}. Pour tout réel $x>0$,

\begin{center}
$\frac{x}{\sh x}=\frac{2xe^{-x}}{1-e^{-2x}}=2xe^{-x}\sum_{n=0}^{+\infty}e^{-2nx}=\sum_{n=0}^{+\infty}2xe^{-(2n+1)x}$,
\end{center}

puis avec la même démarche que dans l'exercice précédent

\begin{align*}\ensuremath
\int_{0}^{+\infty}\frac{x}{\sh x}\;dx&=\sum_{n=0}^{+\infty}\int_{0}^{+\infty}2xe^{-(2n+1)x}\;dx=\sum_{n=0}^{+\infty}\frac{2}{(2n+1)^2}\int_{0}^{+\infty}ue^{-u}\;du=\sum_{n=0}^{+\infty}\frac{2\Gamma(2)}{(2n+1)^2}\\
 &=2\sum_{n=0}^{+\infty}\frac{1}{(2n+1)^2}=2\left(\sum_{n=1}^{+\infty}\frac{1}{n^2}-\sum_{n=1}^{+\infty}\frac{1}{(2n)^2}\right)=2\left(1-\frac{1}{4}\right)\frac{\pi^2}{6}=\frac{\pi^2}{4}.
\end{align*}

\begin{center}
\shadowbox{
$\int_{0}^{+\infty}\frac{x}{\sh x}\;dx=\frac{\pi^2}{4}$.
}
\end{center}
}
}

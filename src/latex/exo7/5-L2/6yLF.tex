\uuid{6yLF}
\exo7id{2627}
\titre{exo7 2627}
\auteur{debievre}
\organisation{exo7}
\datecreate{2009-05-19}
\isIndication{true}
\isCorrection{true}
\chapitre{Fonction de plusieurs variables}
\sousChapitre{Dérivée partielle}

\contenu{
\texte{
On consid\`ere les fonctions $f\colon \R^2\longrightarrow \R^3$ et
$g\colon \R^3\longrightarrow \R$ d\'efinies par
\[
f(x,y)=(\sin (xy), y\cos x, xy\sin(xy)\exp(y^2)),\quad 
g(u,v,w)= uvw .
\]
}
\begin{enumerate}
    \item \question{Calculer explicitement $g\circ f$.}
\reponse{$g(f(x,y))= xy^2 \sin^2 (xy) \cos x \exp(y^2)$}
    \item \question{En utilisant l'expression trouv\'ee en (1), calculer les d\'eriv\'ees partielles de $g\circ f$.}
\reponse{\begin{align*}
\frac{\partial (g \circ f)}{\partial x}
&= 
y^2 \sin (xy) \exp(y^2)
(
2xy \cos x  \cos (xy) 
-
x \sin x\sin (xy)
+
\cos x \sin (xy)
)
\\
\frac{\partial (g\circ f)}{\partial y}
&=
2xy \cos x   \sin (xy)\exp(y^2)
(xy  \cos (xy) +(1+y^2)  \sin (xy) )
\end{align*}}
    \item \question{D\'eterminer les matrices jacobiennes $J_f(x,y)$ et $J_g(u,v,w)$ de $f$ et de $g$.}
\reponse{Calculons d'abord
\begin{align*}
\frac{\partial w}{\partial x}
&=y\sin(xy)\exp(y^2) +xy^2\cos(xy)\exp(y^2)
\\
&=y\exp(y^2)(\sin(xy) +xy\cos(xy))
\\
\frac{\partial w}{\partial y}
&=x\sin(xy)\exp(y^2) +x^2y\cos(xy)\exp(y^2) +2xy^2\sin(xy)\exp(y^2)
\\
&=
x\exp(y^2)(\sin(xy) +xy\cos(xy) +2y^2\sin(xy))
\\
&=
x\exp(y^2)((1+2y^2)\sin(xy) +xy\cos(xy)) .
\end{align*}
Ainsi la matrice jacobienne  $\mathrm J_f$ de $f$ s'\'ecrit
\begin{align*}
\mathrm{J_f}&=
\left[
\begin{matrix} 
\frac{\partial u}{\partial x}
&
\frac{\partial u}{\partial y}
\\
\frac{\partial v}{\partial x}
&
\frac{\partial v}{\partial y}
\\
\frac{\partial w}{\partial x}
&
\frac{\partial w}{\partial y}
\end{matrix}
\right]
\\
&=
\left[
\begin{matrix} 
y \cos (xy)
&
x \cos (xy)
\\
-y\sin x
&
\cos x
\\
y\exp(y^2)(\sin(xy) +xy\cos(xy))
&
x\exp(y^2)((1+2y^2)\sin(xy) +xy\cos(xy))
\end{matrix}
\right] .
\end{align*}
De m\^eme, la matrice jacobienne $\mathrm J_g$ de $g$ est :
\begin{align*}
\mathrm J_g
&=
\left[
\frac{\partial g}{\partial u}, \frac{\partial g}{\partial v},\frac{\partial g}{\partial w}
\right]=[vw,uw,uv]
\\
&=\left [
xy^2 \sin (xy) \cos x \exp(y^2),xy \sin^2 (xy) \exp(y^2),y \sin (xy) \cos x 
\right]
\end{align*}}
    \item \question{Retrouver le r\'esultat sous (2.) en utilisant un produit appropri\'e de matrices jacobiennes.}
\reponse{La matrice jacobienne $\mathrm J_{g\circ f}$ de la fonction
compos\'ee  $g\circ f$ s'\'ecrit comme produit matricielle
\begin{align*}
\mathrm J_{g\circ f}&=\mathrm J_g \circ \mathrm J_f =\left[
\frac{\partial g}{\partial u}, \frac{\partial g}{\partial v},\frac{\partial g}{\partial w}
\right]
\left[
\begin{matrix} 
\frac{\partial u}{\partial x}
&
\frac{\partial u}{\partial y}
\\
\frac{\partial v}{\partial x}
&
\frac{\partial v}{\partial y}
\\
\frac{\partial w}{\partial x}
&
\frac{\partial w}{\partial y}
\end{matrix}
\right]
\end{align*}
d'o\`u
\begin{align*}
\frac{\partial (g \circ f)}{\partial x}
&= 
(xy^2 \sin (xy) \cos x \exp(y^2)) y \cos (xy)
\\
&\quad
-(xy \sin^2 (xy) \exp(y^2)) y\sin x
\\
&\quad
+(y \sin (xy) \cos x)
y\exp(y^2)(\sin(xy) +xy\cos(xy))
\\
&=
xy^3 \cos x \sin (xy) \cos (xy)  \exp(y^2))
-
xy^2 \sin x\sin^2 (xy) \exp(y^2)
\\
&\quad
+y^2 \cos x \sin^2 (xy) \exp(y^2)
+xy^3 \cos x \sin (xy) \cos(xy)\exp(y^2)
\\&=
y^2 \sin (xy) \exp(y^2)
(
2xy \cos x  \cos (xy) 
-
x \sin x\sin (xy)
+
\cos x \sin (xy)
)
\\
\frac{\partial (g\circ f)}{\partial y}
&=
(xy^2 \sin (xy) \cos x \exp(y^2))
x\cos (xy)
\\
&\quad
+
(xy \sin^2 (xy) \exp(y^2))
\cos x
\\
&\quad
+(y \sin (xy) \cos x)
x\exp(y^2)((1+2y^2)\sin(xy) +xy\cos(xy))
\\
&=
x^2y^2  \cos x\sin (xy) 
\cos (xy)\exp(y^2) +xy \cos x\sin^2 (xy) \exp(y^2)
\\
&\quad
+xy(1+2y^2)  \cos x \sin^2 (xy)
\exp(y^2)
+x^2y^2 \cos x\sin (xy)\cos(xy)\exp(y^2)
\\
&=
2 x^2y^2  \cos x\sin (xy) 
\cos (xy)\exp(y^2) +2xy(1+y^2) \cos x\sin^2 (xy) \exp(y^2)
\\
&=
2xy \cos x   \sin (xy)\exp(y^2)
(xy  \cos (xy) +(1+y^2)  \sin (xy) ) .
\end{align*}}
\indication{\'Ecrire
$f(x,y)=(\sin (xy), y\cos x, xy\sin(xy)\exp(y^2))=(u,v,w)$.}
\end{enumerate}
}

\uuid{I8Kv}
\exo7id{5783}
\auteur{rouget}
\organisation{exo7}
\datecreate{2010-10-16}
\isIndication{false}
\isCorrection{true}
\chapitre{Série de Fourier}
\sousChapitre{Calcul de coefficients}

\contenu{
\texte{
Soit $a\in\Cc\setminus[-1,1]$.
}
\begin{enumerate}
    \item \question{\begin{enumerate}}
\reponse{\begin{enumerate}}
    \item \question{Développer en série trigonométrique la fonction $f~:~t\mapsto\frac{1}{a-\cos t}$ (utiliser la racine de plus petit module, notée $b$,  de l'équation $z^2-az+1=0$).}
\reponse{Soit $a\in\Cc\setminus[-1,1]$. Pour tout réel $t$, $a-\cos t\neq0$ et

\begin{center}
$\frac{1}{a-\cos t}=\frac{2}{2a-e^{it}-e^{-it}}=\frac{-2e^{it}}{(e^{it})^2-2ae^{it}+1}$.
\end{center}

L'équation $z^2-2az+1=0$ admet deux solutions non  nulles inverses l'une de l'autre. On note $b$ la solution de plus petit module de sorte que $|b|\leqslant1$.

On ne peut avoir $|b|=1$ car alors il existe $\theta\in\Rr$ tel que $b=e^{i\theta}$. On en déduit que $2a=b+\frac{1}{b}=2\cos\theta\in[-2,2]$ puis que $a\in[-1,1]$ ce qui n'est pas. Donc $|b|\neq1$. Plus précisément, puisque $|b|\leqslant\left|\frac{1}{b}\right|$, on a $|b|<1$ et $\left|\frac{1}{b}\right|$. En particulier, $b\neq\frac{1}{b}$.

Ensuite, pour $|t|<|b|$, on a

\begin{align*}\ensuremath
\frac{1}{a-\cos t}&=\frac{-2e^{it}}{(e^{it}-b)\left(e^{it}-\frac{1}{b}\right)}=\frac{2}{\frac{1}{b}-b}\left(\frac{b}{e^{it}-b}-\frac{1/b}{e^{it}-\frac{1}{b}}\right)=\frac{2b}{1-b^2}\left(\frac{be^{-it}}{1-be^{-it}}+\frac{1}{1-be^{it}}\right)\\
 &=\frac{2b}{1-b^2}\left(be^{-it}\sum_{n=0}^{+\infty}b^ne^{-int}+\sum_{n=0}^{+\infty}b^ne^{int}\right)\;(\text{car}\;|be^{it}|=|be^{-it}|=|b|<1)\\
 &=\frac{2b}{1-b^2}\left(\sum_{n=0}^{+\infty}b^{n+1}e^{-i(n+1)t}+\sum_{n=0}^{+\infty}b^ne^{int}\right)=\frac{2b}{1-b^2}\left(1+\sum_{n=1}^{+\infty}b^ne^{int}+\sum_{n=1}^{+\infty}b^{n}e^{-int}\right)\\
 &=\frac{2b}{1-b^2}\left(1+2\sum_{n=1}^{+\infty}b^n\cos(nt)\right).
\end{align*}

\begin{center}
\shadowbox{
$\forall t\in\Rr$, $\frac{1}{a-\cos t}=\frac{2b}{1-b^2}\left(1+2\sum_{n=1}^{+\infty}b^n\cos(nt)\right)$.
}
\end{center}}
    \item \question{La série obtenue est-elle la série de \textsc{Fourier} de $f$ ?}
\reponse{Pour tout réel $t\in[-\pi,\pi]$ et tout entier naturel non nul $n$, on a $|b^n\cos(nt)|\leqslant|b|^n$. Comme la série numérique de terme général $|b|^n$ converge, on en déduit que la série de fonctions de terme général $t\mapsto b^n\cos(nt)$, $n\in\Nn$, converge normalement et donc uniformément sur le segment $[-\pi,\pi]$.

On sait alors que la série obtenue est la série de \textsc{Fourier} de $f$.}
\end{enumerate}
}

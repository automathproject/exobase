\uuid{eDHs}
\exo7id{4836}
\titre{exo7 4836}
\auteur{quercia}
\organisation{exo7}
\datecreate{2010-03-16}
\isIndication{false}
\isCorrection{true}
\chapitre{Topologie}
\sousChapitre{Compacité}
\module{Analyse}
\niveau{L2}
\difficulte{}

\contenu{
\texte{
Soit $E = {\cal B}(\N,\R) = \{ \text{suites } u = (u_n) \text{ born{\'e}es}\}$.
On munit $E$ de la norme :
$\|u\| = \sum_{n=0}^\infty \frac{|u_n|}{2^n}$.
Montrer que $A = \{u\in E \text{ tq } \forall\ n\in\N,\ 0\le u_n\le 1\}$
est compact.
}
\reponse{
Soit $(u^n)$ une suite de suites {\'e}l{\'e}ments de $A$ : $u^n = (u^n_k)$.
On peut trouver une sous-suite $(u^{n_{p_0}})$ telle que $(u^{n_{p_0}}_0)$ converge
vers $u_0 \in {[0,1]}$, puis une sous-suite $(u^{n_{p_1}})$ telle que
$(u^{n_{p_1}}_0,u^{n_{p_1}}_1)$ converge vers $(u_0,u_1) \in {[0,1]}^2$, etc.
Alors la suite $(u^{n_{p_k}})_k$ converge dans $A$ vers $(u_0,u_1,\dots)$.
}
}

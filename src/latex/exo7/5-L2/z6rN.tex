\uuid{z6rN}
\exo7id{5752}
\titre{exo7 5752}
\auteur{rouget}
\organisation{exo7}
\datecreate{2010-10-16}
\isIndication{false}
\isCorrection{true}
\chapitre{Série entière}
\sousChapitre{Calcul de la somme d'une série entière}

\contenu{
\texte{
Calculer $\sum_{n=1}^{+\infty}\frac{1}{n}\cos\left(\frac{2n\pi}{3}\right)x^n$ pour $x$ dans $]-1,1[$.
}
\reponse{
Pour tout entier naturel non nul, $|a_n|\leqslant\frac{1}{n}$ et donc $R\geqslant1$. Mais si $x > 1$, la suite $\left(\frac{1}{n}\cos\left(\frac{2n\pi}{3}\right)x^n\right)_{n\geqslant1}$ n'est pas bornée comme on le voit en considérant la suite extraite des termes d'indices multiples de $3$ et donc $R = 1$. Pour $x$ dans $]-1,1[$, $f(x)=\text{Re}\left(\sum_{n=1}^{+\infty}\frac{(jx)^n}{n}\right)$. Le problème est alors de ne pouvoir écrire $-\ln(1-jx)$. Il faut s'y prendre autrement.

$f$ est donc dérivable sur $]-1,1[$ et pour $x$ dans $]-1;1[$,

\begin{center}
$f'(x)=\sum_{n=1}^{+\infty}\cos\left(\frac{2n\pi}{3}\right)x^{n-1}=\text{Re}\left(\sum_{n=0}^{+\infty}j^nx^{n-1}\right) =\text{Re}\left(\frac{j}{1-jx}\right)=\text{Re}\left(\frac{j(1-j^2x)}{x^2+x+1}\right) = -\frac{1}{2}\frac{2x+1}{x^2+x+1}$.
\end{center}

Par suite, pour $x\in]-1,1[$, $f(x) = f(0)+\int_{0}^{x}f'(t)\;dt= -\frac{1}{2}\ln(x^2+x+1)$.

\begin{center}
\shadowbox{
$\forall x\in]-1,1[$, $\sum_{n=1}^{+\infty}\frac{1}{n}\cos\left(\frac{2n\pi}{3}\right)x^{n}=-\frac{1}{2}\ln(x^2+x+1)$.
}
\end{center}
}
}

\uuid{sBA3}
\exo7id{4143}
\titre{exo7 4143}
\auteur{quercia}
\organisation{exo7}
\datecreate{2010-03-11}
\isIndication{false}
\isCorrection{true}
\chapitre{Fonction de plusieurs variables}
\sousChapitre{Dérivée partielle}
\module{Analyse}
\niveau{L2}
\difficulte{}

\contenu{
\texte{
Soit $f : {\R^2} \to \R$ de classe $\mathcal{C}^2$, et
$g(\rho,\theta) = f(\rho\cos\theta,\rho\sin\theta)$.
On pose $\Delta f = \frac{\partial^2 f}{\partial x^2} + \frac{\partial^2 f}{\partial y^2}$ (laplacien de $f$).
}
\begin{enumerate}
    \item \question{Calculer $\frac{\partial g}{\partial \rho}$, $\frac{\partial g}{\partial \theta}$, $\frac{\partial^2 g}{\partial \rho^2}$, $\frac{\partial^2 g}{\partial \theta^2}$ en fonction des dérivées
    partielles de $f$.}
    \item \question{Exprimer $\Delta f$ en fonction des dérivées de $g$.}
\reponse{
$\Delta f = \frac{\partial^2 g}{\partial \rho^2} + \frac 1\rho \frac{\partial g}{\partial \rho} + \frac 1{\rho^2} \frac{\partial^2 g}{\partial \theta^2}$.
}
\end{enumerate}
}

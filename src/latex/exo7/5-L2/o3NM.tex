\uuid{o3NM}
\exo7id{4097}
\titre{exo7 4097}
\auteur{quercia}
\organisation{exo7}
\datecreate{2010-03-11}
\isIndication{false}
\isCorrection{true}
\chapitre{Equation différentielle}
\sousChapitre{Equations différentielles linéaires}

\contenu{
\texte{
On considère l'équation $(*) \iff y'' + a(t)y' + b(t)y = 0$, avec $a,b$ continues.
}
\begin{enumerate}
    \item \question{Soit $y$ une solution non nulle de $(*)$. Montrer que les zéros de $y$ sont isolés.}
    \item \question{Soient $y,z$ deux solutions de $(*)$ non proportionelles.
  \begin{enumerate}}
    \item \question{Montrer que $y$ et $z$ n'ont pas de zéros commun.}
    \item \question{Montrer que si $u,v$ sont deux zéros consécutifs de $y$, alors $z$ possède un
    unique zéro dans l'intervalle $]u,v[$
    (étudier $\frac zy$).}
\reponse{
\begin{enumerate}
Wronskien.
$\left(\frac zy\right)' = \frac {z'y-zy'}{y^2}$ est de signe constant
$ \Rightarrow  \frac zy$ est monotone.

$\frac zy$ admet des limites infinies en $u$ et $v$. TVI
}
\end{enumerate}
}

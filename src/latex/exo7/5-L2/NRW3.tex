\uuid{NRW3}
\exo7id{4787}
\titre{exo7 4787}
\auteur{quercia}
\organisation{exo7}
\datecreate{2010-03-16}
\isIndication{false}
\isCorrection{true}
\chapitre{Topologie}
\sousChapitre{Topologie des espaces vectoriels normés}

\contenu{
\texte{
On consid{\`e}re l'espace $\mathcal{M}_n(\C)$ muni d'une norme quelconque.
}
\begin{enumerate}
    \item \question{Montrer que $GL_n(\C)$ est ouvert dense de $\mathcal{M}_n(\C)$.}
\reponse{$GL_n(\C) = \det^{-1}(\C\setminus\{0\})$ est ouvert.
    Il est dense car $A\in\mathcal{M}_n(\C)$ quelconque est limite des matrices
    $A - \frac1p I$ inversibles pour presque tout entier $p$ ($A$ a un
    nombre fini de valeurs propres).}
    \item \question{Soit $D_n(\C)$ l'ensemble des matrices diagonalisables de $\mathcal{M}_n(\C)$.
    Montrer que $D_n(\C)$ est dense dans $\mathcal{M}_n(\C)$.}
\reponse{Toute matrice triangulaire est limite de matrices triangulaires
    {\`a} coefficients diagonaux distincts.}
    \item \question{Quel est l'int{\'e}rieur de $D_n(\C)$~?}
\reponse{$\begin{pmatrix}\lambda &0\cr 0&\lambda\cr\end{pmatrix}
             = \lim_{p\to\infty}\begin{pmatrix}\lambda &1/p\cr 0&\lambda\cr\end{pmatrix}$
    donc une matrice triangulaire {\`a} valeurs propres non distinctes est
    limite de matrices non diagonalisables.
    Par conjugaison, la fronti{\`e}re de $D_n(\C)$ contient
    l'ensemble des matrices ayant au moins une valeur propre multiple.
    
    R{\'e}ciproquement, soit $(A_k)$ une suite de matrices non diagonalisables
    convergeant vers une matrice~$A$. Les matrices $A_k$ ont toutes au moins une valeur
    propre multiple, et ces valeurs propres sont born{\'e}es
    (car si $\lambda$ est une valeur propre de~$M$ alors $|\lambda| \le \| M\|$ en prenant une norme
    sur $\mathcal{M}_n(\C)$ subordonn{\'e}e {\`a} une norme sur $\C^n$) donc on peut trouver
    une suite $(z_k)$ de complexes convergeant vers un complexe $z$ telle
    que $\chi'_{A_k}(z_k) = 0$. A la limite on a $\chi_A(z) = 0$ ce qui prouve
    que $A$ a au moins une valeur propre multiple.
    
    Conclusion~: la fronti{\`e}re de~$D_n(\C)$ est exactement l'ensemble des
    matrices diagonalisables ayant au moins une valeur propre multiple et l'int{\'e}rieur
    de~$D_n(\C)$ est l'ensemble des matrices {\`a} valeurs propres distinctes.}
\end{enumerate}
}

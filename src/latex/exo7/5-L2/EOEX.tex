\uuid{EOEX}
\exo7id{5903}
\titre{exo7 5903}
\auteur{rouget}
\organisation{exo7}
\datecreate{2010-10-16}
\isIndication{false}
\isCorrection{true}
\chapitre{Fonction de plusieurs variables}
\sousChapitre{Extremums locaux}
\module{Analyse}
\niveau{L2}
\difficulte{}

\contenu{
\texte{
Minimum de $f(x,y) =\sqrt{x^2+(y-a)^2}+\sqrt{(x-a)^2+y^2}$, $a$ réel donné.
}
\reponse{
Soient $A$ et $B$ les points du plan de coordonnées respectives $(0,a)$ et $(a,0)$ dans un certain repère $\mathcal{R}$ orthonormé. Soit $M$ un point du plan de coordonnées $(x,y)$ dans $\mathcal{R}$. Pour $(x,y)\in\Rr^2$,

\begin{center}
$f(x,y)=\left\|\overrightarrow{MA}\right\|_2+\left\|\overrightarrow{MB}\right\|_2=MA+MB\geqslant AB$ avec égalité si et seulement si $M\in[AB]$.
\end{center}

Donc $f$ admet un minimum global égal à $AB=a\sqrt{2}$ atteint en tout couple $(x,y)$ de la forme $(\lambda a,(1-\lambda)a)$, $\lambda\in[0,1]$.
}
}

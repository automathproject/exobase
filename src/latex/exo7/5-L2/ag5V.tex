\uuid{ag5V}
\exo7id{4530}
\titre{exo7 4530}
\auteur{quercia}
\organisation{exo7}
\datecreate{2010-03-14}
\isIndication{false}
\isCorrection{false}
\chapitre{Suite et série de fonctions}
\sousChapitre{Autre}
\module{Analyse}
\niveau{L2}
\difficulte{}

\contenu{
\texte{
Soit $\zeta(x) = \sum_{n=1}^\infty \frac1{n^x}$.
}
\begin{enumerate}
    \item \question{Déterminer le domaine de définition de $\zeta$.
    Montrer que $\zeta$ est de classe $\mathcal{C}^\infty$ sur ce domaine.}
    \item \question{Prouver que $\zeta(x) \to 1$ lorsque $x\to+\infty$ $\Bigl($majorer
    $\sum_{n=2}^\infty \frac1{n^x}$ par comparaison à une intégrale$\Bigr)$.}
    \item \question{Prouver que $\zeta(x) \to +\infty$ lorsque $x\to1^+$.}
\end{enumerate}
}

\uuid{vm0y}
\exo7id{5899}
\titre{exo7 5899}
\auteur{rouget}
\organisation{exo7}
\datecreate{2010-10-16}
\isIndication{false}
\isCorrection{true}
\chapitre{Fonction de plusieurs variables}
\sousChapitre{Différentiabilité}
\module{Analyse}
\niveau{L2}
\difficulte{}

\contenu{
\texte{
Déterminer la différentielle en tout point de $\begin{array}[t]{cccc}
f~:&\Rr^3\times\Rr^3&\rightarrow&\Rr\\
 &(x,y)&\mapsto&x.y
\end{array}$ et $\begin{array}[t]{cccc}
g~:&\Rr^3\times\Rr^3&\rightarrow&\Rr\\
 &(x,y)&\mapsto&x\wedge y
\end{array}$.
}
\reponse{
On munit $(\Rr^3)^2$ de la norme définie par $\forall(x,y)\in(\Rr^3)^2$, $\|(x,y)\|=\text{Max}\{\|h\|_2,\|k\|_2\}$.

\textbullet~Soit $(a,b)\in(\Rr^3)^2$. Pour $(h,k)\in(\Rr^3)^2$,

\begin{center}
$f((a,b)+(h,h))=(a+h).(b+k)=a.b+a.h+b.k+h.k$,
\end{center}

et donc $f((a,b)+(h,h))-f((a,b))=(a.h+b.k)+h.k$. Maintenant l'application $L~:~(h,k)\mapsto a.h+b.k$ est linéaire et de plus, pour $(h,k)\neq(0,0)$,

\begin{center}
$|f((a,b)+(h,h))-f((a,b))-L((h,k))|=|h.k|\leqslant\|h\|_2\|k\|_2\leqslant\|(h,k)\|^2$,
\end{center}

et donc $ \frac{1}{\|(h,k)\|}|f((a,b)+(h,h))-f((a,b))-L((h,k))|\leqslant\|(h,k)\|$ puis

\begin{center}
$\lim_{(h,k) \rightarrow (0,0)} \frac{1}{\|(h,k)\|}|f((a,b)+(h,h))-f((a,b))-L((h,k))|=0$.
\end{center}

Puisque l'application $(h,k)\mapsto a.h+b.k$ est linéaire, on en déduit que $f$ est différentiable en $(a,b)$ et que $\forall(h,k)\in(\Rr^3)^2$, $df_{(a,b)}(h,k)=a.h+b.k$.

La démarche est analogue pour le produit vectoriel :

\begin{center}
$ \frac{1}{\|(h,k)\|}\|(a+h)\wedge(b+k)-a\wedge b-a\wedge h-b\wedge k\|_2= \frac{\|h\wedge k\|_2}{\|(h,k)\|}\leqslant \frac{\|h\|_2\|k\|_2}{\|(h,k)\|}\leqslant\|(h,k)\|$.
\end{center}

Puisque l'application $(h,k)\mapsto a\wedge h+b\wedge k$ est linéaire, on en déduit que $g$ est différentiable en $(a,b)$ et que $\forall(h,k)\in(\Rr^3)^2$, $dg_{(a,b)}(h,k)=a\wedge h+b\wedge k$.
}
}

\uuid{In48}
\exo7id{5864}
\titre{exo7 5864}
\auteur{rouget}
\organisation{exo7}
\datecreate{2010-10-16}
\isIndication{false}
\isCorrection{true}
\chapitre{Suite et série de fonctions}
\sousChapitre{Suite et série de matrices}
\module{Analyse}
\niveau{L2}
\difficulte{}

\contenu{
\texte{
Déterminer $\lim_{n \rightarrow +\infty}\left(
\begin{array}{cc}
1&- \frac{a}{n}\\
 \frac{a}{n}&1
\end{array}
\right)^n$ ($a$ réel strictement positif donné).
}
\reponse{
Soit $a\in\Rr$. Pour $n\in\Nn^*$, on pose $A_n=\left(
\begin{array}{cc}
1&- \frac{a}{n}\\
 \frac{a}{n}&1
\end{array}
\right)$.

Soit $n\in\Nn^*$. On peut écrire $A_n=\sqrt{1+ \frac{a^2}{n^2}}\left(
\begin{array}{cc}
 \frac{1}{\sqrt{1+ \frac{a^2}{n^2}}}&- \frac{a/n}{\sqrt{1+ \frac{a^2}{n^2}}}\\
 \frac{a/n}{\sqrt{1+ \frac{a^2}{n^2}}}& \frac{1}{\sqrt{1+ \frac{a^2}{n^2}}}
\end{array}
\right)$. Les sommes des carrés des deux nombres $ \frac{1}{\sqrt{1+ \frac{a^2}{n^2}}}$ et $ \frac{a/n}{\sqrt{1+ \frac{a^2}{n^2}}}$ est égale à $1$. Donc il existe un réel $\theta_n\in]-\pi,\pi]$ tel que $\cos(\theta_n)= \frac{1}{\sqrt{1+ \frac{a^2}{n^2}}}$ et $\sin(\theta_n)= \frac{a/n}{\sqrt{1+ \frac{a^2}{n^2}}}$. De plus, $\cos(\theta_n)>0$ et $\sin(\theta_n)>0$ et donc on peut prendre 

\begin{center}
$\theta_n=\Arctan\left( \frac{a}{n}\right)\in\left]0, \frac{\pi}{2}\right[$.
\end{center}

Pour $n\in\Nn^*$, on a alors

\begin{center}
$A_n^n=\left(\sqrt{1+ \frac{a^2}{n^2}}\right)^n\left(
\begin{array}{cc}
\cos(\theta_n)&-\sin(\theta_n)\\
\sin(\theta_n)&\cos(\theta_n)
\end{array}
\right)^n=\left(1+ \frac{a^2}{n^2}\right)^{n/2}\left(
\begin{array}{cc}
\cos(n\theta_n)&-\sin(n\theta_n)\\
\sin(n\theta_n)&\cos(n\theta_n)
\end{array}
\right)$.
\end{center}

Maintenant, $\left(1+ \frac{a^2}{n^2}\right)^{n/2}=\text{exp}\left( \frac{n}{2}\ln\left(1+ \frac{a^2}{n^2}\right)\right)\underset{n\rightarrow+\infty}{=}\text{exp}\left( \frac{a^2}{2n}+o\left( \frac{1}{n}\right)\right)\underset{n\rightarrow+\infty}{=}1+o(1)$.

D'autre part, $n\theta_n=n\Arctan\left( \frac{a}{n}\right)\underset{n\rightarrow+\infty}{=}n\times \frac{a}{n}=a$. Donc

\begin{center}
$\lim_{n \rightarrow +\infty}A_n^n=1.\left(
\begin{array}{cc}
\cos(a)&-\sin(a)\\
\sin(a)&\cos(a)
\end{array}
\right)=\left(
\begin{array}{cc}
\cos(a)&-\sin(a)\\
\sin(a)&\cos(a)
\end{array}
\right)$.
\end{center}

\begin{center}
\shadowbox{
$\forall a>0$, $\lim_{n \rightarrow +\infty}\left(
\begin{array}{cc}
1&- \frac{a}{n}\\
 \frac{a}{n}&1
\end{array}
\right)^n=\left(
\begin{array}{cc}
\cos(a)&-\sin(a)\\
\sin(a)&\cos(a)
\end{array}
\right)$.
}
\end{center}
}
}

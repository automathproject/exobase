\uuid{RyaI}
\exo7id{5870}
\auteur{rouget}
\organisation{exo7}
\datecreate{2010-10-16}
\isIndication{false}
\isCorrection{true}
\chapitre{Suite et série de fonctions}
\sousChapitre{Suite et série de matrices}

\contenu{
\texte{
Soit $A=\left(
\begin{array}{ccc}
0&1/2&-2\\
1/2&0&0\\
0&0&-1/2
\end{array}
\right)$. Calculer $\ln(I_3+tA)=\sum_{n=1}^{+\infty} \frac{(-1)^{n-1}t^n}{n}A^n$ en précisant les valeurs de $t$ pour lesquelles la série converge.
}
\reponse{
$\;$

$\chi_A=\left|
\begin{array}{ccc}
-X&1/2&-2\\
1/2&-X&0\\
0&0&-1/2-X
\end{array}
\right|=-X\left(X^2+ \frac{1}{2}X\right)- \frac{1}{2}\left(- \frac{1}{2}X- \frac{1}{4}\right)=-X^2\left(X+ \frac{1}{2}\right)+ \frac{1}{4}\left(X+ \frac{1}{2}\right)=-\left(X+ \frac{1}{2}\right)^2\left(X- \frac{1}{2}\right)$. 

Soit $n\in\Nn$. La division euclidienne de $X^n$ par $\chi_A$ s'écrit $X^n=Q_n\chi_A+a_nX^2+b_nX+c_n$ où $Q_n\in\Rr[X]$ et $(a_n,b_n,c_n)\in\Rr^3$.

On évalue les deux membres de cette égalité en $ \frac{1}{2}$ et $- \frac{1}{2}$ et on obtient $ \frac{a_n}{4}+ \frac{b_n}{2}+c_n=\left( \frac{1}{2}\right)^n$ et $ \frac{a_n}{4}- \frac{b_n}{2}+c_n=\left(- \frac{1}{2}\right)^n$.

Puis en dérivant les deux membres de l'égalité et en évaluant en $- \frac{1}{2}$, on obtient $-a_n+b_n=n\left(- \frac{1}{2}\right)^{n-1}=-2n\left(- \frac{1}{2}\right)^{n}$. Maintenant,

\begin{align*}\ensuremath
\left\{
\begin{array}{l}
 \frac{a_n}{4}+ \frac{b_n}{2}+c_n=\left( \frac{1}{2}\right)^n\\
 \frac{a_n}{4}- \frac{b_n}{2}+c_n=\left(- \frac{1}{2}\right)^n\\
-a_n+b_n=-2n\left(- \frac{1}{2}\right)^{n}
\end{array}
\right.&\Leftrightarrow\left\{
\begin{array}{l}
b_n=\left( \frac{1}{2}\right)^n-\left(- \frac{1}{2}\right)^n\\
 \frac{a_n}{2}+2c_n=\left( \frac{1}{2}\right)^n+\left(- \frac{1}{2}\right)^n\\
-a_n+\left( \frac{1}{2}\right)^n-\left(- \frac{1}{2}\right)^n=-2n\left(- \frac{1}{2}\right)^{n}
\end{array}
\right.
\\
 &\Leftrightarrow\left\{
\begin{array}{l}
a_n=\left( \frac{1}{2}\right)^n+(2n-1)\left(- \frac{1}{2}\right)^n\\
b_n=\left( \frac{1}{2}\right)^n-\left(- \frac{1}{2}\right)^n\\
c_n= \frac{1}{4}\left( \frac{1}{2}\right)^n- \frac{2n-3}{4}\left(- \frac{1}{2}\right)^n
\end{array}
\right.
\end{align*}

Donc $\forall n\in\Nn$, $A^n=\left(\left( \frac{1}{2}\right)^n+(2n-1)\left(- \frac{1}{2}\right)^n\right)A^2+\left(\left( \frac{1}{2}\right)^n-\left(- \frac{1}{2}\right)^n\right)A+\left( \frac{1}{4}\left( \frac{1}{2}\right)^n- \frac{2n-3}{4}\left(- \frac{1}{2}\right)^n\right)I_3$ avec $A^2=\left(
\begin{array}{ccc}
0&1/2&-2\\
1/2&0&0\\
0&0&-1/2
\end{array}
\right)\left(
\begin{array}{ccc}
0&1/2&-2\\
1/2&0&0\\
0&0&-1/2
\end{array}
\right)=\left(
\begin{array}{ccc}
1/4&0&1\\
0&1/4&-1\\
0&0&1/4
\end{array}
\right)$. On en déduit que pour $|t|<2$,

\begin{align*}\ensuremath
\ln(I_3+tA)&=\sum_{n=1}^{+\infty} \frac{(-1)^{n-1}t^n}{n}A^n\\
 &=\left(\sum_{n=1}^{+\infty} \frac{(-1)^{n-1}t^n}{n}\left(\left( \frac{1}{2}\right)^n+(2n-1)\left(- \frac{1}{2}\right)^n\right)\right)A^2+\left(\sum_{n=1}^{+\infty} \frac{(-1)^{n-1}t^n}{n}\left(\left( \frac{1}{2}\right)^n-\left(- \frac{1}{2}\right)^n\right)\right)A\\
  &+\left(\sum_{n=1}^{+\infty} \frac{(-1)^{n-1}t^n}{n} \frac{1}{4}\left( \frac{1}{2}\right)^n- \frac{2n-3}{4}\left(- \frac{1}{2}\right)^n\right)I_3.
\end{align*}

et donc

\begin{align*}\ensuremath
\ln(I_3+tA)&=\left(\sum_{n=1}^{+\infty} \frac{(-1)^{n-1}(t/2)^n}{n}-2\sum_{n=1}^{+\infty}(t/2)^n-\sum_{n=1}^{+\infty} \frac{(-1)^{n-1}(-t/2)^n}{n}\right)A^2\\
 &+\left(\sum_{n=1}^{+\infty} \frac{(-1)^{n-1}(t/2)^n}{n}-\sum_{n=1}^{+\infty} \frac{(-1)^{n-1}(-t/2)^n}{n}\right)A\\
  &+ \frac{1}{4}\left(\sum_{n=1}^{+\infty} \frac{(-1)^{n-1}(t/2)^n}{n}+2\sum_{n=1}^{+\infty}(t/2)^n+3\sum_{n=1}^{+\infty} \frac{(-1)^{n-1}(-t/2)^n}{n}\right)I_3\\
  &=\left(\ln\left(1+ \frac{t}{2}\right)-2\left( \frac{1}{1- \frac{t}{2}}-1\right)-\ln\left(1- \frac{t}{2}\right)\right)A^2+\left(\ln\left(1+ \frac{t}{2}\right)-\ln\left(1- \frac{t}{2}\right)\right)A\\
  &+ \frac{1}{4}\left(\ln\left(1+ \frac{t}{2}\right)+2\left( \frac{1}{1- \frac{t}{2}}-1\right)+3\ln\left(1- \frac{t}{2}\right)\right)I_3\\
  &=\left(\ln\left( \frac{2+t}{2-t}\right)- \frac{2t}{2-t}\right)\left(
\begin{array}{ccc}
1/4&0&1\\
0&1/4&-1\\
0&0&1/4
\end{array}
\right)+\ln\left( \frac{2+t}{2-t}\right)\left(
\begin{array}{ccc}
0&1/2&-2\\
1/2&0&0\\
0&0&-1/2
\end{array}
\right)\\
  &+ \frac{1}{4}\left(\ln\left(1+ \frac{t}{2}\right)+ \frac{2t}{2-t}+3\ln\left(1- \frac{t}{2}\right)\right)\left(
\begin{array}{ccc}
1&0&0\\
0&1&0\\
0&0&1
\end{array}
\right)\\
 &=\left(
 \begin{array}{ccc}
 \frac{1}{2}\ln\left(1- \frac{t^2}{4}\right)& \frac{1}{2}\ln\left( \frac{2+t}{2-t}\right)&-\ln\left( \frac{2+t}{2-t}\right)- \frac{2t}{2-t}\\
 \frac{1}{2}\ln\left( \frac{2+t}{2-t}\right)& \frac{1}{2}\ln\left(1- \frac{t^2}{4}\right)&-\ln\left( \frac{2+t}{2-t}\right)+ \frac{2t}{2-t}\\
 0&0&\ln\left(1- \frac{t}{2}\right)
 \end{array}
 \right).
\end{align*}

\begin{center}
\shadowbox{
$\forall t\in]-2,2[$, $\ln(I_3+tA)=\left(
 \begin{array}{ccc}
 \frac{1}{2}\ln\left(1- \frac{t^2}{4}\right)& \frac{1}{2}\ln\left( \frac{2+t}{2-t}\right)&-\ln\left( \frac{2+t}{2-t}\right)- \frac{2t}{2-t}\\
 \frac{1}{2}\ln\left( \frac{2+t}{2-t}\right)& \frac{1}{2}\ln\left(1- \frac{t^2}{4}\right)&-\ln\left( \frac{2+t}{2-t}\right)+ \frac{2t}{2-t}\\
 0&0&\ln\left(1- \frac{t}{2}\right)
 \end{array}
 \right)$.
}
\end{center}
}
}

\uuid{PIZI}
\exo7id{4353}
\titre{exo7 4353}
\auteur{quercia}
\organisation{exo7}
\datecreate{2010-03-12}
\isIndication{false}
\isCorrection{true}
\chapitre{Intégration}
\sousChapitre{Intégrale de Riemann dépendant d'un paramètre}
\module{Analyse}
\niveau{L2}
\difficulte{}

\contenu{
\texte{
Soit $I(\alpha) =  \int_{x=0}^{+\infty} \frac{\sin\alpha x}{e^x-1}\,d x$.
}
\begin{enumerate}
    \item \question{Justifier l'existence de $I(\alpha)$.}
    \item \question{Déterminer les réels $a$ et~$b$ tels que~:
    $I(\alpha) = \sum_{n=1}^\infty \frac a{b+n^2}$.}
    \item \question{Donner un équivalent de $I(\alpha)$ quand $\alpha\to+\infty$.}
\reponse{
$a=\alpha$, $b=\alpha^2$.
comparaison série-intégrale $ \Rightarrow  I(\alpha)\to\frac\pi2$ lorsque $\alpha\to+\infty$.
}
\end{enumerate}
}

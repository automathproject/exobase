\uuid{QFLD}
\exo7id{5839}
\auteur{rouget}
\organisation{exo7}
\datecreate{2010-10-16}
\isIndication{false}
\isCorrection{true}
\chapitre{Topologie}
\sousChapitre{Topologie des espaces vectoriels normés}

\contenu{
\texte{
\label{ex:rou1bis}
Montrer que la boule unité d'un espace vectoriel normé est un convexe de cet espace.
}
\reponse{
\textbf{Cas de la boule fermée.}  Soit $B=\{u\in E/\;\|u\|\leqslant 1\}$. Soient $(x,y)\in B^2$ et $\lambda\in[0,1]$.

\begin{center}
$\|\lambda x+(1-\lambda)y\|\leqslant \lambda\|x\| + (1-\lambda)\|y\|\leqslant\lambda+1-\lambda= 1$.
\end{center}

Ainsi, $\forall(x,y)\in B^2$, $\forall\lambda\in[0,1]$, $\lambda x+(1-\lambda)y\in B$ et donc $B$ est convexe.

\textbf{Cas de la boule ouverte.} Soit $B=\{u\in E/\;\|u\|< 1\}$. Soient $(x,y)\in B^2$ et $\lambda\in[0,1]$.

Puisque $0\leqslant \lambda\leqslant 1$ et $0\leqslant\|x\|<1$, on en déduit que $\lambda\|x\|<1$. Comme $(1-\lambda)\|y\|\leqslant1$ (et même $<1$) et donc

\begin{center}
$\|\lambda x+(1-\lambda)y\|\leqslant \lambda\|x\| + (1-\lambda)\|y\|<1$.
\end{center} 

\begin{center}
\shadowbox{
La boule unité fermée (ou ouverte) de l'espace vectoriel normé $(E,\|\;\|)$ est un convexe de l'espace vectoriel $E$.
}
\end{center}
}
}

\uuid{l5ix}
\exo7id{6999}
\titre{exo7 6999}
\auteur{blanc-centi}
\organisation{exo7}
\datecreate{2015-07-04}
\video{cNVcKnV6F2E}
\isIndication{true}
\isCorrection{true}
\chapitre{Equation différentielle}
\sousChapitre{Résolution d'équation différentielle du deuxième ordre}

\contenu{
\texte{
Résoudre sur $]0;\pi[$ l'équation différentielle
$y''+y=\mathrm{cotan}\, x$, où $\mathrm{cotan}\, x = \frac{\cos x}{\sin x}$.
}
\indication{Utiliser la méthode de variation de la constante.}
\reponse{
Les solutions de l'équation homogène sont les $\lambda\cos x+\mu\sin x$. 
En posant $y_1(x)=\cos x$ et $y_2(x)=\sin x$, on va chercher les solutions sous 
la forme $\lambda y_1+\mu y_2$, vérifiant
\begin{eqnarray*}
\left\{\begin{array}{l}
\lambda'y_1+\mu'y_2=0\\
\lambda'y'_1+\mu'y'_2=\mathrm{cotan}\, x
\end{array}\right.
&\Longleftrightarrow&
\left\{\begin{array}{l}
\lambda'\cos x+\mu'\sin x=0\\
\lambda'(-\sin x)+\mu'\cos x=\mathrm{cotan}\, x
\end{array}\right.\\
\ &\Longleftrightarrow&
\left\{\begin{array}{l}
\lambda'(x)=\frac{\begin{array}{|cc|}0&\sin x\\\mathrm{cotan}\, x&\cos x \end{array}}{\begin{array}{|cc|}\cos x&\sin x\\-\sin x&\cos x \end{array}}\\
\ \\
\mu'(x)=\frac{\begin{array}{|cc|}\cos x&0\\-\sin x&\mathrm{cotan}\, x \end{array}}{\begin{array}{|cc|}\cos x&\sin x\\-\sin x&\cos x \end{array}}
\end{array}\right.
\end{eqnarray*}
d'après les formules de Cramer, où $\begin{array}{|cc|}\cos x&\sin x\\-\sin x&\cos x \end{array}=1$. On obtient donc 
$$\left\{\begin{array}{l}
\lambda'(x)=-\cos x\\
\mu'(x)=\frac{\cos^2x}{\sin x}
\end{array}\right.
$$
ce qui donne une primitive $\lambda(x)=-\sin x$. 

Pour $\mu$, on cherche à primitiver 
$\frac{\cos^2x}{\sin x}$ à l'aide du changement de variable $t=\cos x$ (et donc $\dd t= -\sin t \,\dd x$), 
on calcule une primitive
\begin{eqnarray*}
\int \frac{\cos^2x}{\sin x}\,\dd x&=&-\int\frac{t^2}{1-t^2}\,\dd t=t-\int\frac{1}{1-t^2}\,\dd t\\ 
 &=&t+\frac{1}{2}\ln(1-t) -\frac{1}{2}\ln(1-t)= \cos x + \frac{1}{2}\ln(1-\cos x) - \frac{1}{2}\ln(1-\cos x)
\end{eqnarray*}
En remplaçant, les solutions générales sont les 
$$y(x)=c_1\cos x+c_2\sin x +  (-\sin x)\cos x+\left(\cos x + \frac{1}{2}\ln(1-\cos x) - \frac{1}{2}\ln(1-\cos x)\right)\sin x$$
qui se simplifie $y(x)=c_1\cos x+c_2\sin x + \frac{1}{2}\sin x\ln\frac{1-\cos x}{1+\cos x}$, $c_1,c_2\in\Rr$.
}
}

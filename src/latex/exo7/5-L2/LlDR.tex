\uuid{LlDR}
\exo7id{4617}
\titre{exo7 4617}
\auteur{quercia}
\organisation{exo7}
\datecreate{2010-03-14}
\isIndication{false}
\isCorrection{true}
\chapitre{Série entière}
\sousChapitre{Analycité}
\module{Analyse}
\niveau{L2}
\difficulte{}

\contenu{
\texte{
Soit $(a_n)$ une suite réelle avec $a_0=0$ et $a_1=1$. On note 
$f(z)=\sum_{n=0}^{\infty}a_nz^n$. On suppose que $f$ est injective et que le rayon de convergence
de la série entière vaut 1. On considère $\Omega ^+=\{z\in \C\,|\, \Im z >0\}$ et
$D=\{z\in \C\,|\, |z|<1\}$.
}
\begin{enumerate}
    \item \question{Montrer que, pour tout $z\in D$, $f(z)\in \R$ si et seulement si $z\in \R$.}
\reponse{$f(\overline z) = \overline{f(z)}$.}
    \item \question{Montrer que $f(D\cap \Omega ^+)\subset \Omega ^+$.}
\reponse{$\Im(f)$ est de signe constant sur le connexe $D\cap \Omega ^+$
    et $f(z)\sim z$ au voisinage de~$0$.}
    \item \question{Montrer que, pour tout $|r|<1$, 
    $a_n=\frac{2}{\strut\pi r^n} \int_{\theta=0}^{\pi}\Im (f(re^{i\theta})) \sin (n\theta)\, d \theta$.}
\reponse{Intégrer terme à terme.}
    \item \question{Montrer que $|\sin (n\theta)|\le n|\sin \theta |$. En déduire que $|a_n|\le n$.}
\reponse{$\Im (f(re^{i\theta})) \sin\theta \ge 0$ par la question {2.}.}
\end{enumerate}
}

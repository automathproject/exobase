\uuid{l6ix}
\exo7id{5840}
\titre{exo7 5840}
\auteur{rouget}
\organisation{exo7}
\datecreate{2010-10-16}
\isIndication{false}
\isCorrection{true}
\chapitre{Topologie}
\sousChapitre{Topologie des espaces vectoriels normés}
\module{Analyse}
\niveau{L2}
\difficulte{}

\contenu{
\texte{

}
\begin{enumerate}
    \item \question{Inégalités de \textsc{Hölder} et de \textsc{Minkowski}. Soit $(p,q)\in]0,+\infty[^2$  tel que $ \frac{1}{p}+ \frac{1}{q}=1$. 
   \begin{enumerate}}
\reponse{Puisque $p>0$ et $q>0$, $1= \frac{1}{p}+ \frac{1}{q}> \frac{1}{p}$ et donc $p>1$. De même, $q>1$. D'autre part, $q= \frac{p}{p-1}$.

  \begin{enumerate}}
    \item \question{Montrer que pour $(x,y)\in[0,+\infty[^2$, $xy\leqslant \frac{x^p}{p}+ \frac{x^q}{q}$.}
\reponse{L'inégalité est immédiate quand $y=0$. Soit $y > 0$ fixé.

Pour $x\geqslant0$, on pose $f(x)= \frac{x^p}{p}+ \frac{y^q}{q} -xy$. Puisque $p > 1$, la fonction $f$ est dérivable sur $[0,+\infty[$ et $\forall x\geqslant0$, $f'(x)=x^{p-1}-y$. $f$ admet donc un minimum en $x_0=y^{1/(p-1)}$ égal à 

\begin{center}
$f\left(y^{1/(p-1)}\right)= \frac{y^{p(p-1)}}{p}+ \frac{y^{p/(p-1)}}{q}-y^{1/(p-1)}y=y^{p/(p-1}\left( \frac{1}{p}+ \frac{1}{q}-1\right)= 0$.
\end{center}

Finalement, $f$ est positive sur $[0,+\infty[$ et donc

\begin{center}
\shadowbox{
$\forall x\geqslant0$, $\forall y\geqslant0$, $xy\leqslant \frac{x^p}{p}+ \frac{y^q}{q}$.
}
\end{center}}
    \item \question{En déduire que $\forall((a_1,...,a_n),(b_1,...,b_n))\in(\Rr^n)^2$, $\left|\sum_{k=1}^{n}a_kb_k\right|\leqslant\left(\sum_{k=1}^{n}|a_k|^p\right)^{1/p}\left(\sum_{k=1}^{n}|b_k|^q\right)^{1/q}$.}
\reponse{Posons $A=\sum_{k=1}^{n}|a_k|^p$ et $B =\sum_{k=1}^{n}|b_k|^q$.

Si $A$ (ou $B$) est nul, tous les $a_k$ (ou tous les $b_k$) sont nuls et l'inégalité est vraie.

On suppose dorénavant que $A > 0$ et $B > 0$. D'après la question a),

\begin{center}
$\sum_{k=1}^{n} \frac{|a_k|}{A^{1/p}}\times \frac{|b_k|}{B^{1/q}}\leqslant\sum_{k=1}^{n}\left( \frac{|a_k|^p}{pA}+ \frac{|b_k|^q}{qB}\right) = \frac{1}{pA}\sum_{k=1}^{n}|a_k|^p+ \frac{1}{qB}\sum_{k=1}^{n}|b_k|^q= \frac{1}{pA}\times A+ \frac{1}{qB}\times B= \frac{1}{p}+ \frac{1}{q}=1$,
\end{center}

et donc $\sum_{k=1}^{n}|a_k||b_k|\leqslant A^{1/p}B^{1/q}=\left(\sum_{k=1}^{n}|a_k|^p\right)^{1/p}\left(\sum_{k=1}^{n}|b_k|^q\right)^{1/q}$. Comme $\left|\sum_{k=1}^{n}a_kb_k\right|\leqslant\sum_{k=1}^{n}|a_k||b_k|$, on a montré que

\begin{center}
\shadowbox{
$\forall((a_k)_{1\leqslant k\leqslant n},(b_k)_{1\leqslant k\leqslant n})\in(\Rr^n)^2$, $\sum_{k=1}^{n}|a_kb_k|\leqslant\left(\sum_{k=1}^{n}|a_k|^p\right)^{1/p}\left(\sum_{k=1}^{n}|b_k|^q\right)^{1/q}$  (Inégalité de \textsc{Hölder}).
}
\end{center}

\textbf{Remarque.} Quand $p=q=2$, on a bien $ \frac{1}{p}+ \frac{1}{q}=1$ et l'inégalité de \textsc{Hölder} s'écrit 

\begin{center}
$\sum_{k=1}^{n}|a_kb_k|\leqslant\left(\sum_{k=1}^{n}|a_k|^2\right)^{1/2}\left(\sum_{k=1}^{n}|b_k|^2\right)^{1/2}$ (inégalité de \textsc{Cauchy}-\textsc{Schwarz}).
\end{center}}
    \item \question{En déduire que $\forall((a_1,...,a_n),(b_1,...,b_n))\in(\Rr^n)^2$,  $\left(\sum_{k=1}^{n}\left|a_k+b_k\right|^p\right)^{1/p}\leqslant\left(\sum_{k=1}^{n}\left|a_k\right|^p\right)^{1/p}+\left(\sum_{k=1}^{n}\left|b_k\right|^p\right)^{1/p}$.}
\reponse{Soit $((a_k)_{1\leqslant k\leqslant n},(b_k)_{1\leqslant k\leqslant n})\in(\Rr^n)^2$. D'après l'inégalité de \textsc{Hölder}, on a

\begin{align*}
\sum_{k=1}^{n}(|a_k|+|b_k|)^p&=\sum_{k=1}^{n}|a_k|(|a_k|+|b_k|)^{p-1}+\sum_{k=1}^{n}|b_k|(|a_k|+|b_k|)^{p-1}\\
 &\Leftrightarrow\left(\sum_{k=1}^{n}|a_k|^p\right)^{1/p}\left(\sum_{k=1}^{n}(|a_k|+|b_k|)^{(p-1)q}\right)^{1/q}\left(\sum_{k=1}^{n}|b_k|^p\right)^{1/p}\left(\sum_{k=1}^{n}(|a_k|+|b_k|)^{(p-1)q}\right)^{1/q}\\
 &=\left(\left(\sum_{k=1}^{n}|a_k|^p\right)^{1/p}+\left(\sum_{k=1}^{n}|b_k|^p\right)^{1/p}\right)\left(\sum_{k=1}^{n}(|a_k|+|b_k|)^{p}\right)^{1-\frac{1}{p}}.
\end{align*}

Si  $\sum_{k=1}^{n}(|a_k|+|b_k|)^p = 0$, tous les $a_k$ et les $b_k$ sont nuls et l'inégalité est claire.

Sinon $\sum_{k=1}^{n}(|a_k|+|b_k|)^p > 0$ et après simplification des deux membres de l'inégalité précédente par le réel strictement positif $\sum{k=1}^{n}(|a_k|+|b_k|)^p$,  on obtient $\left(\sum_{k=1}^{n}|a_k+b_k|^p\right)^{1/p}\leqslant\left(\sum_{k=1}^{n}|a_k|^p\right)^{1/p}+\left(\sum_{k=1}^{n}|b_k|^p\right)^{1/p}$

\begin{center}
\shadowbox{
$\forall((a_k)_{1\leqslant k\leqslant n},(b_k)_{1\leqslant k\leqslant n})\in(\Rr^n)^2$, $\left(\sum_{k=1}^{n}|a_k+b_k|^p\right)^{1/p}\leqslant\left(\sum_{k=1}^{n}|a_k|^p\right)^{1/p}+\left(\sum_{k=1}^{n}|b_k|^p\right)^{1/p}$  (Inégalité de \textsc{Minkowski}).
}
\end{center}}
\end{enumerate}
}

\uuid{SPfq}
\exo7id{1808}
\auteur{drutu}
\organisation{exo7}
\datecreate{2003-10-01}
\isIndication{false}
\isCorrection{false}
\chapitre{Fonction de plusieurs variables}
\sousChapitre{Différentiabilité}

\contenu{
\texte{

}
\begin{enumerate}
    \item \question{Calculer la d\'eriv\'ee de la fonction $F(x,y)=e^{x^2+y^2}$ au point 
$P(1,0)$ suivant la bissectrice du premier quadrant.}
    \item \question{Calculer la d\'eriv\'ee de la fonction $F(x,y,z)=x^2-3yz+5$ au point 
$P(1,2,1)$ dans une direction formant des angles \'egaux avec les trois axes de 
coordonn\'ees.}
    \item \question{Calculer la d\'eriv\'ee de la fonction 
$F(x,y,z)=xy+yz+zx$ au point 
$M(2,1,3)$ dans la direction joignant ce point au point $N(5,5,15)$.}
\end{enumerate}
}

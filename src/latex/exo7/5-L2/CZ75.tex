\uuid{CZ75}
\exo7id{4577}
\titre{exo7 4577}
\auteur{quercia}
\organisation{exo7}
\datecreate{2010-03-14}
\isIndication{false}
\isCorrection{true}
\chapitre{Série entière}
\sousChapitre{Développement en série entière}
\module{Analyse}
\niveau{L2}
\difficulte{}

\contenu{
\texte{
Développer en série entière $f(x) = \sqrt{x+\sqrt{1+x^2}}$.
}
\reponse{
$f(\sh y) = e^{y/2}$ d'où l'équation différentielle~:
$(1+x^2)f''(x) + xf'(x) = \frac14f(x)$.

En posant $f(x) = \sum_{n=0}^\infty a_n x^n$ on obtient
$4(k+1)(k+2)a_{k+2} = -(2k+1)(2k-1)a_k$ avec $a_0 = f(0) = 1$ et $a_1 = f'(0) = \frac12$,
d'où $a_{2p} = \frac{(-1)^{p+1}C_{4p-2}^{2p-1}}{p2^{4p}}$ si $p\ge 1$
et $a_{2p+1} = \frac{(-1)^pC_{4p}^{2p}}{2^{4p+1}(2p+1)}$ si $p\ge 0$.

Le rayon de convergence de la série correspondante est 1, ce qui
valide la méthode (avec le théorème d'unicité de Cauchy-Lipschitz).
}
}

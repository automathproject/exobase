\uuid{1owk}
\exo7id{4092}
\titre{exo7 4092}
\auteur{quercia}
\organisation{exo7}
\datecreate{2010-03-11}
\isIndication{false}
\isCorrection{true}
\chapitre{Equation différentielle}
\sousChapitre{Equations différentielles linéaires}

\contenu{
\texte{
Soit ${a,b} : \R \to \R$ continues telles que : $\forall\ x\in\R,\ a(x)\ge 1$
et $b(x)\to 0$ lorsque $x\to+\infty$.
}
\begin{enumerate}
    \item \question{Montrer que toute solution de l'équation : $y' + ay = b$ tend vers $0$ en $+\infty$.}
\reponse{$y =  \int_{t=\alpha}^x b(t)e^{A(t)-A(x)}\,d t + y(\alpha)e^{-A(x)}$
             avec $A' = a$ et $A(\alpha) = 0$.\par
             Comme $a \ge 1$, on a $A(x) \ge x-\alpha$ et $A(t)-A(x) \le t-x$ pour $t \le x$.\par
             Donc $|y| \le  \int_{t=\alpha}^z |b(t)|e^{t-x}\,d t +
                            \int_{t=z}^x |b(t)|e^{t-x}\,d t + |y(\alpha)|e^{\alpha-x}
                       \le \|b\|_{\infty} e^{z-x} +
                           \sup\limits_{[z,+\infty[}|b| + |y(\alpha)|e^{\alpha-x}$.\par
             On choisit $z$ tel que $z\to+\infty$ et $x-z\to+\infty  \Rightarrow $ cqfd.}
    \item \question{On suppose $b(x)\to 0$ lorsque $x\to-\infty$. Montrer qu'il y a une unique solution $y$
    qui tend vers $0$ en $-\infty$.}
\reponse{Comme $A(t)-A(x) \le t-x$ pour $t \le x$, l'intégrale
             $ \int_{t=-\infty}^x b(t)e^{A(t)-A(x)}\,d t$ converge et fournit
             une solution nulle en $-\infty$.
             Comme $e^{-A(x)}\to +\infty$ lorsque $x\to-\infty$, c'est la seule.}
\end{enumerate}
}

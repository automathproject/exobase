\uuid{9yW6}
\exo7id{4623}
\titre{exo7 4623}
\auteur{quercia}
\organisation{exo7}
\datecreate{2010-03-14}
\isIndication{false}
\isCorrection{true}
\chapitre{Série de Fourier}
\sousChapitre{Calcul de coefficients}

\contenu{
\texte{
Établir la convergence puis calculer
$ \int_{t=0}^{\pi/2} \frac{\sin(nt)}{\sin t}\,d t$.

En déduire les coefficients de Fourier de $f$~: $f(t) = \ln|\tan(t/2)|$.
}
\reponse{
$I_{n+1}-I_{n-1} =  \int_{t=0}^{\pi/2} 2\cos(nt)\,d t
                          = \frac2n\sin(n\pi/2)$.

Donc $I_{2p} = 2\Bigl(1 - \frac13 + \dots+\frac{(-1)^{p-1}}{2p-1}\Bigr)$,
     $I_{2p+1} = \frac\pi2$.

$b_n = 0$ (parité), $a_{2p} = 0$ (symétrie par rapport à $(\pi/2,0)$),
$a_{2p+1} = -\frac{4}{(2p+1)\pi}I_{2p+1} = -\frac{2}{2p+1}$.
}
}

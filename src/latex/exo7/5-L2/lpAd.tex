\uuid{lpAd}
\exo7id{4080}
\titre{exo7 4080}
\auteur{quercia}
\organisation{exo7}
\datecreate{2010-03-11}
\isIndication{false}
\isCorrection{true}
\chapitre{Equation différentielle}
\sousChapitre{Equations différentielles linéaires}
\module{Analyse}
\niveau{L2}
\difficulte{}

\contenu{
\texte{
Soient $u,v,w$ trois applications bornées et de classe~$\mathcal{C}^1$ sur~$\R$,
à valeurs dans~$\R^3$, vérifiant~:
$u'+v' = w$; $w'=-v$; $ \int_{0}^{\infty}\|u'\|^2<+\infty$.
On suppose qu'il existe une suite de terme général $t_n$ tendant vers~$+\infty$
telle que $u(t_n)$ tend vers~$a\in\R^3$.
}
\begin{enumerate}
    \item \question{Montrer que la suite de terme général $u_n=\frac1{2\pi} \int_{t=t_n}^{t_n+2\pi} u(t)\,d t$
    tend vers~$a$.}
\reponse{$u_n-u(t_n) =  \frac1{2\pi} \int_{t=t_n}^{t_n+2\pi} \int_{x=t_n}^t u'(x)\,d x\,d t$
et on majore l'intégrale interne par Cauchy-Schwarz.}
    \item \question{Montrer que les suites de termes généraux $v_n=\frac1{2\pi} \int_{t=t_n}^{t_n+2\pi} v(t)\,d t$
    et $w_n=\frac1{2\pi} \int_{t=t_n}^{t_n+2\pi} w(t)\,d t$ tendent vers~$0$.}
\reponse{$w+w''=u'$ donc $w(t) =  \int_{x=t_n}^t\sin(t-x)u'(x)\,d x + \alpha\cos t + \beta\sin t$
puis 
\begin{align*} \int_{t=t_n}^{t_n+2\pi}w(t)\,d t\ 
&=  \int_{t=t_n}^{t_n+2\pi} \int_{x=t_n}^t\sin(t-x)u'(x)\,d x\,d t\\
&=  \int_{x=t_n}^{t_n+2\pi} \int_{t=x}^{t_n+2\pi}\sin(t-x)u'(x)\,d t\,d x\\
&=  \int_{x=t_n}^{t_n+2\pi} u'(x)(\cos(t_n-t)-1)\,d x\\
&\to 0,\\ \text{ lorsque }n\to\infty\\ \end{align*}
et
$ \int_{t=t_n}^{t_n+2\pi}v(t)\,d t
= w(t_n)-w(t_n+2\pi)
= - \int_{x=t_n}^{t_n+2\pi}\sin(t-x)u'(x)\,d x$.}
\end{enumerate}
}

\uuid{6Igv}
\exo7id{4200}
\titre{exo7 4200}
\auteur{quercia}
\organisation{exo7}
\datecreate{2010-03-11}
\isIndication{false}
\isCorrection{true}
\chapitre{Fonction de plusieurs variables}
\sousChapitre{Extremums locaux}

\contenu{
\texte{
Soit $B$ la boule unit{\'e} de $\R^{n}$, $f$ de classe $\mathcal{C}^{1}$ sur $B$ et $x\in B$ tel que $f(x)=\max\{f(y),\, y\in B\}$.

Montrer que $\nabla f(x)=\lambda x$ avec $\lambda \ge 0$.
}
\reponse{
Soit $y\in B$ orthogonal {\`a} $x$. La fonction $g : \theta \mapsto f(x\cos \theta +y\sin \theta)$ admet un extr{\'e}mum en $0$, donc
$g'(0)=0$, soit $\nabla f(x) \perp y$. Si $x=0$ on a donc $\nabla f(0) = 0$.
Sinon, $\nabla f(x) = \lambda x$ avec $\lambda\in\R$,
et en faisant un d{\'e}veloppement limit{\'e} de $f(x-tx)$ on voit
que $\lambda \ge 0$.
}
}

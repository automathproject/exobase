\uuid{ruyx}
\exo7id{5847}
\titre{exo7 5847}
\auteur{rouget}
\organisation{exo7}
\datecreate{2010-10-16}
\isIndication{false}
\isCorrection{true}
\chapitre{Topologie}
\sousChapitre{Topologie des espaces vectoriels normés}
\module{Analyse}
\niveau{L2}
\difficulte{}

\contenu{
\texte{
Soit $A$ une partie non vide d'un espace vectoriel normé $(E,\|\;\|)$.

Pour $x\in E$, on pose $d_A(x) =d(x,A)$ où $d(x,A) =\text{Inf}\left\{\|x-a\|,\;a\in A\right\}$.
}
\begin{enumerate}
    \item \question{Justifier l'existence de $d_A(x)$ pour chaque $x$ de $E$.}
\reponse{Soit $x\in E$. $\left\{\|x-a\|,\;a\in A\right\}$ est une partie non vide et minorée (par $0$) de $\Rr$. $\left\{\|x-a\|,\;a\in A\right\}$ admet donc une borne inférieure dans $\Rr$. On en déduit l'existence de $d_A(x)$.}
    \item \question{\begin{enumerate}}
\reponse{\begin{enumerate}}
    \item \question{Montrer que si $A$ est fermée, $\forall x\in E$, $d_A(x) = 0\Leftrightarrow x\in A$.}
\reponse{Soit $A$ une partie fermée et non vide de $E$. Soit $x\in E$.

\textbullet~Supposons que $x\in A$. Alors $0\leqslant f(x)=\text{Inf}\{\|x-a\|,\;a\in A\}\leqslant\|x-x\|=0$ et donc $d_A(x) = 0$.

\textbullet~Supposons que $d_A(x)=0$. Par définition d'une borne inférieure, $\forall\varepsilon>0$ $\exists a_\varepsilon\in A/\;\|x-a_\varepsilon\| <\varepsilon$.

Soit $V$ un voisinage de $x$. $V$ contient une boule ouverte de centre $x$ et de rayon $\varepsilon>0$ puis d'après ce qui précède, $V$ contient un élément de $A$. Finalement, $\forall V\in\mathcal{V}(x)$, $V\cap A\neq\varnothing$ et donc $x\in\overline{A}=A$.

\begin{center}
\shadowbox{
Si $A$ est fermée, $\forall x\in E,\;(d_A(x) = 0\Leftrightarrow x\in A)$.
}
\end{center}}
    \item \question{Montrer que si $A$ est fermée et $E$ est de dimension finie, $\forall x\in E$, $\exists a\in A/\;d_A(x)=\|x-a\|$.}
\reponse{Posons $d=d_A(x)$. Pour chaque entier naturel $n$, il existe $a_n\in A$ tel que $d\leqslant \|x-a_n\|\leqslant d+ \frac{1}{n}$.

La suite $(a_n)_{n\in\Nn}$ est bornée. En effet, $\forall n\in\Nn^*$ $\|a_n\|\leqslant \|a_n-x\|+\|x\|\leqslant d+ \frac{1}{n}+\| x\|\leqslant d+\|x\|+1$.

Puisque $E$ est de dimension finie, d'après le théorème de \textsc{Bolzano}-\textsc{Weierstrass},  on peut extraire de la suite $(a_n)_{n\geqslant1}$ une suite $(a_{\varphi(n)})_{n\geqslant1}$ convergeant vers un certain élément $a$ de $E$.

Ensuite, puisque $A$ est fermée, on en déduit que $a\in A$. Puis, comme

\begin{center}
$\forall n\in\Nn^*$, $d\leqslant\|x-a_{\varphi(n)}\| \leqslant d+ \frac{1}{\varphi(n)}$,
\end{center}

et puisque $\varphi(n)$ tend vers l'infini quand $n$ tend vers $+\infty$, on obtient quand $n$ tend vers l'infini, $d =\lim_{n \rightarrow +\infty}\|x-a_{\varphi(n)}\|$. Maintenant on sait que l'application $y\mapsto\|y\|$ est continue sur l'espace normé $(E,\|\;\|)$ et donc 

\begin{center}
$\lim_{n \rightarrow +\infty}\left\|x-a_{\varphi(n)}\right\|=\left\|x-\lim_{n \rightarrow +\infty}a_{\varphi(n)}\right\|=\|x-a\|$.
\end{center}

On a montré qu'il existe $a\in A$ tel que $d_A(x)=\|x-a\|$.}
\end{enumerate}
}

\uuid{iui9}
\exo7id{4190}
\titre{exo7 4190}
\auteur{quercia}
\organisation{exo7}
\datecreate{2010-03-11}
\isIndication{false}
\isCorrection{true}
\chapitre{Fonction de plusieurs variables}
\sousChapitre{Dérivée partielle}

\contenu{
\texte{
Soit~$E$ l'ensemble des fonctions continues de~$[0,1]$ dans~$\R$.
On y définit une norme par~: $\|f\|=\sqrt{ \int_{t=0}^1 f^2(t)\,d t}$.
Soit $\varphi : \R \to \R$ de classe $\mathcal{C}^2$ telle que $\varphi''$ est bornée.
Pour~$f\in E$ on pose $T(f) =  \int_{t=0}^1\varphi(f(t))\,d t$.
}
\begin{enumerate}
    \item \question{Montrer que l'application ainsi définie~$T : E \to \R$ est continue.}
\reponse{???}
    \item \question{Montrer que~$T$ est différentiable en tout point.}
\reponse{On a pour $a,b\in\R$~: $|\varphi(a+b) - \varphi(a) - b\varphi'(a)| \le \frac12\|\varphi''\|_\infty b^2$.

Donc pour $f,h\in E$~: $|T(f+h) - T(f) - (h\mid \varphi'\circ f)| \le \frac12\|\varphi''\|_\infty \|h\|^2$,
ce qui prouve que $T$ est différentiable en~$f$ de différentielle $h \mapsto(h\mid\varphi'\circ f)$.
On en déduit alors que~$T$ est continue en~$f$.}
\end{enumerate}
}

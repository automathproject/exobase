\uuid{IF57}
\exo7id{2618}
\titre{exo7 2618}
\auteur{debievre}
\organisation{exo7}
\datecreate{2009-05-19}
\isIndication{true}
\isCorrection{true}
\chapitre{Topologie}
\sousChapitre{Ouvert, fermé, intérieur, adhérence}
\module{Analyse}
\niveau{L2}
\difficulte{}

\contenu{
\texte{

}
\begin{enumerate}
    \item \question{Soient $B_1\subset \R^n$ et $B_2\subset \R^m$ des boules 
ouvertes. Montrer
que $B_1\times B_2\subset \R^{n+m}$ est un ouvert.}
\reponse{Soient $q_1$ un point de $B_1$ et $q_2$ un point de $B_2$,  soient
$d_1$ resp. $d_2$ la distance
de $q_1$ au bord de $B_1$ resp. la distance
de $q_2$ au bord de $B_2$, et soit $0<d\leq \min(d_1,d_2)$.
Alors la boule ouverte dans $\R^{n+m}$  centr\'ee en $(q_1,q_2)$
et de rayon $d$ est dans 
$B_1\times B_2$.}
    \item \question{Soit $A$ un ouvert de $\R^2$ et $B$ un ouvert de $\R$. Montrer 
que $A\times B$
est un ouvert de $\R^3$.}
\reponse{Soient $p$ un point de $A$,  $q$ un point de $B$, soit
$B_1$ un disque ouvert dans $A$ contenant $p$, et
soit $B_2$ un intervalle ouvert dans $B$ contenant $q$.
D'apr\`es (1.), $B_1\times B_2$ 
est un ouvert de $\R^3$ tel que 
$B_1\times B_2 \subseteq A \times B$
et $(p,q)$ appartient \`a $B_1\times B_2$.
Par cons\'equent, $A\times B$
est un ouvert de $\R^3$.}
\indication{Raisonner \`a partir de la d\'efinition d'un ouvert dans le plan.}
\end{enumerate}
}

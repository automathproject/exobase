\uuid{hEnR}
\exo7id{6942}
\auteur{ruette}
\organisation{exo7}
\datecreate{2013-01-24}
\isIndication{false}
\isCorrection{true}
\chapitre{Loi, indépendance, loi conditionnelle}
\sousChapitre{Loi, indépendance, loi conditionnelle}

\contenu{
\texte{
Soient $A$ et $B$ deux événements indépendants. Montrer que
$A\bot B^c$. En déduire que $A^c\bot B$ et $A^c\bot B^c$.
}
\reponse{
$A=(A\cap B)\cup(A\cap B^c)$ et cette union est disjointe, donc
$P(A\cap B^c)=P(A)-P(A\cap B)$. Par indépendance, $P(A\cap B)=P(A)P(B)$
donc $P(A\cap B^c)=P(A)(1-P(B))=P(A)P(B^c)$, autrement dit $A\bot B^c$.
On applique le résultat à $A'=B, B'=A$ et on trouve $A^c\bot B$.
On réapplique le résultat à $A^c$ et $B$ pour trouver $A^c\bot B^c$.
}
}

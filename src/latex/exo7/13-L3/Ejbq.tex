\uuid{Ejbq}
\exo7id{6324}
\titre{exo7 6324}
\auteur{queffelec}
\organisation{exo7}
\datecreate{2011-10-16}
\isIndication{false}
\isCorrection{false}
\chapitre{Solution maximale}
\sousChapitre{Solution maximale}

\contenu{
\texte{
Soit $f: \Rr ^2 \to \Rr$ donnée par $f(t,x)=
4\frac{t^3x}{t^4+x^2}$ si $(t,x) \neq (0,0)$ et $f(0,0)=0$. On
s'intéresse à l'équation différentielle
\begin{equation} \label{eq 1}
x'(t) = f(t,x(t)) \;\; .
\end{equation}
}
\begin{enumerate}
    \item \question{L'application $f$, est-elle continue et/ou localement
lipschitzienne par rapport à sa seconde variable? Que peut-on en
déduire pour l'équation (\ref{eq 1})?}
    \item \question{Soit $\varphi $ une solution de (\ref{eq 1}) qui est définie sur un
intervalle $I$ ne contenant pas $0$. On définit une application
$\psi$  par $\varphi (t) = t^2 \psi (t)$, $t\in I$. Déterminer une
équation différentielle (\ref{eq 1}') telle que $\psi$ soit solution de
cette équation, puis résoudre cette équation (\ref{eq 1}').}
    \item \question{Que peut-on en déduire pour l'existence et
  l'unicité des solutions de l'équation différentielle
  (\ref{eq 1}) avec donnée initiale $(t_0,x_0)) = (0,0)$?}
\end{enumerate}
}

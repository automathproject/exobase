\uuid{zvcK}
\exo7id{6337}
\titre{exo7 6337}
\auteur{queffelec}
\organisation{exo7}
\datecreate{2011-10-16}
\isIndication{false}
\isCorrection{false}
\chapitre{Système linéaire à  coefficients constants}
\sousChapitre{Système linéaire à  coefficients constants}

\contenu{
\texte{
On rappelle les différentes méthodes pour résoudre un système différentiel
linéaire à coefficients constants $X'(t)= A.X(t)$ sur $E$ de dimension finie:
}
\begin{enumerate}
    \item \question{On met $A$ sous forme triangulaire et on résout de proche en proche
le nou\-veau système obtenu par changement de base avant de revenir au système
initial.}
    \item \question{On met $A$ sous forme de Dunford, $P^{-1}AP=D+N$, où $D$ semi-simple
et
$N$ nilpotente, qui commutent. On calcule ainsi $e^{tA}.X_0$, la solution
valant
$X_0$ au temps $t=0$.}
    \item \question{On utilise le théorème de Cayley-Hamilton pour établir des
relations entre les puissances de $A$ et calculer ainsi $e^{tA}$.}
    \item \question{(cf. Cartan) On décompose $E=\oplus_i E_i$ en sous-espaces
caractéris\-tiques, on calcule $e^{tA_i}$ où $A_i=A_{|E_i}$, puis
$X(t)=\sum_ie^{tA_i}v_i$, si $X_0=\sum_iv_i$.}
    \item \question{On cherche une
base de solutions par identification sous la forme de polyn\^omes-exponentielles,
suivant le résultat du cours.}
\end{enumerate}
}

\uuid{TkSh}
\exo7id{6358}
\titre{exo7 6358}
\auteur{queffelec}
\organisation{exo7}
\datecreate{2011-10-16}
\isIndication{false}
\isCorrection{false}
\chapitre{Autre}
\sousChapitre{Autre}

\contenu{
\texte{
On considère l'équation différentielle linéaire sur $\Rr^n$\quad
$$(1)\quad y'=A(x).y$$
où $A(x)$ est continue sur un intervalle $I$.
}
\begin{enumerate}
    \item \question{Montrer que si l'on suppose $A(x)A(x')=A(x')A(x)$ pour tous $x,x'\in I$, la
résolvante de $(1)$ est 
$$R(x,x_0)=\exp\Big(\int_{x_0}^x A(s)\ ds\Big)=:\exp B(x).$$

(Indic. : remarquer que $B(x)B(x')=B(x')B(x)$.)}
    \item \question{Montrer que si $A(x)=\begin{pmatrix}a(x) & b(x)\cr -b(x) & a(x)\cr\end{pmatrix}$, $A$ vérifie
l'hypothèse de a) et $B(x)$ est de la forme $\begin{pmatrix}\alpha(x) & \beta(x)\cr
-\beta(x) & \alpha(x)\cr\end{pmatrix}$.}
    \item \question{Résoudre l'équation $y'=A(x).y$ lorsque $a(x)=-{x\over2(1+x^2)}$ et $b(x)=
{1\over2(1+x^2)}$.}
    \item \question{Résoudre l'équation $y'=A(x).y + C(x)$ lorsque $A(x)=\begin{pmatrix}\sh(x) &
1\cr -1 & \sh(x)\cr\end{pmatrix}$ et $C(x)=\begin{pmatrix}\sin x\ \sh(x) 
\cr \cos x\ \sh(x) \cr\end{pmatrix}$.}
\end{enumerate}
}

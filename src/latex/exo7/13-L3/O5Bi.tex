\uuid{O5Bi}
\exo7id{6329}
\titre{exo7 6329}
\auteur{queffelec}
\organisation{exo7}
\datecreate{2011-10-16}
\isIndication{false}
\isCorrection{false}
\chapitre{Théorème de Cauchy-Lipschitz}
\sousChapitre{Théorème de Cauchy-Lipschitz}

\contenu{
\texte{
Soit $f$ un champ de vecteurs de classe $C^1$ de $\Rr^n$ dans $\Rr^n$, et
on suppose (pour simplifier) que, pour toute donnée initiale $x$ de $\Rr^n$, il
existe une unique solution passant par $x$ au temps $t=0$, définie sur $\Rr$
tout entier. On note $\phi(t, x)$ cette solution (ou
$\phi$ le flot du champ).
}
\begin{enumerate}
    \item \question{Montrer qu'on a $\phi(s,\phi(t,x))=\phi(s+t,x)$ pour tous $s,t\in
\Rr$.
et la vérifier sur l'équation  $x'=x^2,
x(0)=\alpha\geq0$ aprés avoir précisé le domaine de définition du
flot.}
    \item \question{On fait $n=2$; décrire le flot lorsque $f(x,y)=(-x,y);\quad
(y,x);\quad (-y,x)$, et vérifier la relation précédente.}
\end{enumerate}
}

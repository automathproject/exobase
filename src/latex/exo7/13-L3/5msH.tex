\uuid{5msH}
\exo7id{2559}
\auteur{tahani}
\organisation{exo7}
\datecreate{2009-04-01}
\isIndication{false}
\isCorrection{true}
\chapitre{Solution maximale}
\sousChapitre{Solution maximale}

\contenu{
\texte{
On considère l'\'equation diff\'erentielle $x'=|x|+|t|$. 
\begin {enumerate} 
\item Montrez
que pour tout r\'eel $x_0$, il existe une solution maximale
$(\varphi,J)$ telle que $\varphi(0)=x_0$. 
\item D\'etermner la
solution maximale correspondant à $x_0=1$, en distinguant les cas
$t \geq 0$ et $t<0$, et v\'erifiez qu'elle est d\'efinie sur
$\mathbb{R}$ tout entier. Combien de fois est-elle d\'erivable?
\end{enumerate}
}
\reponse{
Soit $f:\mathbb{R}\times \mathbb{R} \rightarrow \mathbb{R}$
telle que $f(t,x)=|x|+|t|$. $f$ est continue et Lipschitzienne par
rapport \`a la seconde variable. En effet,
$$|f(t,x)-f(t,y)|=||x|-|y||\leq |x-y|.$$
Remarquons que $x' \geq 0$ pour tout $t$ et que pour tout point
$(0,x_0$ passe une solution maximale unique $(\varphi,J)$.
Prenons $x_0=1$, lorque $t\geq 0$; l'\'equation devient
$$x't()=x(t)+t$$ car $|t|=t$ et $x(t) \geq x(0) > 0, x(0)=1$.
Elle admet comme solution sur $[0,+\infty[$ avec $\varphi(0)=1$
$$\varphi(t)=2e^t-t-1.$$
Lorsque $t<0$, on distingue deux cas: premier cas $x(t)\geq0$;
$x'=-t+x(t)$ et alors $x(t)=ce^t+t+1$ avec $x(0)=1$ d'o\`u $c=0$
et $\varphi(t)=t+1$. Cela n'est valable que lorsque $\varphi(t)
\geq 0$, c'est \`a dire $t \geq 1$. Donc $\varphi(t)=t+1$ sur
$[-1,1]$. Deuxi\`eme cas: $x(t) \leq 0$, ceci a lieu lorsque $t
\leq -1$ car $\varphi$ croissante et $\varphi(-1)=0$. Nous avons
alors $\varphi'(t)=-t-\varphi(t)$. D'o\`u $\varphi(t)=ce^{-t}-t+1$
or $\varphi(-1)=ce+2=0$ d'o\`u $c=-2e^{-1}$ et
$\varphi(t)=-2e^{-t+1}-t+1$ sur $]-\infty,-1]$. La solution
maximale v\'erifiant $\varphi(0)=1$ est la suivante:
$$\varphi(t)=
\left( \begin{array}{c} 2e^t-t-1 \mbox{ sur } [0,+\infty[ \\
t+1 \mbox{ sur } [-1,0] \\
-2e^{-(t+1)}-t+1 \mbox{ sur } ]-\infty,-1]
\end{array} \right)$$
$$\varphi'(t)=
\left( \begin{array}{c} 2e^t-1 \mbox{ sur } ]0,+\infty[ \\
1 \mbox{ sur } ]-1,0[ \\
2e^{-(t+1)}-1 \mbox{ sur } ]-\infty,-1[
\end{array} \right)$$
En \'etudiant les limites de $\varphi'$ aux point $0$ et $-1$, on
voit que $\varphi'$ est continue sur $\mathbb{R}$.
$$\varphi''(t)=
\left( \begin{array}{c} 2e^t \mbox{ sur } ]0,+\infty[ \\
0 \mbox{ sur } ]-1,0[ \\
-2e^{-(t+1)} \mbox{ sur } ]-\infty,-1[
\end{array} \right)$$ $\varphi$ n'est donc pas deux fois
d\'erivable en $0$ et $-1$.
}
}

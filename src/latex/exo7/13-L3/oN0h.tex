\uuid{oN0h}
\exo7id{6357}
\titre{exo7 6357}
\auteur{queffelec}
\organisation{exo7}
\datecreate{2011-10-16}
\isIndication{false}
\isCorrection{false}
\chapitre{Autre}
\sousChapitre{Autre}
\module{Equation différentielle}
\niveau{L3}
\difficulte{}

\contenu{
\texte{
On considère les équations $$\quad (1)\quad y''+p(x)y'+q(x)y=0,\quad (2)\quad
y''+p(x)y'+q(x)y=r(x)$$ où $p$, $q$ et
$r$ sont des fonctions continues d'un intervalle $I\subset\Rr$ dans $\Rr$.
Etablir ce qui suit :
}
\begin{enumerate}
    \item \question{Pour tout $x_0\in I$, et tout $(a,b)\in \Rr^2$, $(1)$ admet une solution
maximale définie sur $I$ tout entier, telle que $y(x_0)=a, y'(x_0)=b$.}
    \item \question{Soit $x_0\in I$; les solutions de $(1)$ forment un espace vectoriel $V$ de
dimension 2 dont une base est $(y_1,y_2)$ avec $y_1(x_0)=1, y_1'(x_0)=0$,  
$y_2(x_0)=0, y_2'(x_0)=1$.}
    \item \question{Soit $u$ et $v$ deux solutions de $(1)$ et $W=u'v-uv'$ leur wronskien;
trouver une équation différentielle satisfaite par $W$; en déduire que $W$ est
soit identiquement nul, soit jamais nul, et que $W\not=0\Longleftrightarrow
(u,v)$ est une base de $V$. Quel est le rapport entre $W$ et la résolvante du
système associé ?}
    \item \question{La solution $y$ de $(2)$ vérifiant $y(x_0)=0,y'(x_0)=1$, où $x_0$ fixé dans
$I$, est
$$ y(x)=y(x_0)y_1(x)+y'(x_0)y_2(x)+\int_{x_0}^x
{r(t)\Big(y_1(t)y_2(x)-y_1(x)y_2(t)\Big)\over W(t)}\ dt.
$$}
    \item \question{Exemple : Résoudre  $y''+4y=\tan x$.}
\end{enumerate}
}

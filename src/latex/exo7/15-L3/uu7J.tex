\uuid{uu7J}
\exo7id{6238}
\titre{exo7 6238}
\auteur{queffelec}
\organisation{exo7}
\datecreate{2011-10-16}
\isIndication{false}
\isCorrection{false}
\chapitre{Théorème du point fixe}
\sousChapitre{Théorème du point fixe}
\module{Topologie}
\niveau{L3}
\difficulte{}

\contenu{
\texte{
On va montrer qu'il existe une et une seule $h$ continue sur $[0,1]$ vérifiant
$h(0)=0$ et $h'(t)=\cos (th(t))$ pour tout $t\in[0,1]$. On note $E$ l'espace des
fonctions continues sur $[0,1]$ muni de la métrique uniforme.
}
\begin{enumerate}
    \item \question{$h$ est solution si et seulement si $h$ est continue et $h(s)=\int_0^s \cos
(th(t))\ dt$.}
    \item \question{L'opérateur $T:E\to E$ défini par $Tf(s)=\int_0^s \cos
(tf(t))\ dt$ est $1/2$-contrac\-tant. Conclure.}
\end{enumerate}
}

\uuid{ITVL}
\exo7id{2347}
\titre{exo7 2347}
\auteur{queffelec}
\organisation{exo7}
\datecreate{2003-10-01}
\isIndication{true}
\isCorrection{true}
\chapitre{Espace topologique, espace métrique}
\sousChapitre{Espace topologique, espace métrique}
\module{Topologie}
\niveau{L3}
\difficulte{}

\contenu{
\texte{
Soit $E=\{f\in C^1([0,1],{\Rr})\ ;\ f(0)=0\}$. On pose 
$$||f||=\sup_{0\leq x\leq 1}|f(x)+f'(x)|,\ \hbox{et}\ N(f)=\sup_{0\leq x\leq
1}|f(x)|+\sup_{0\leq x\leq 1}|f'(x)|.$$
Montrer que ce sont deux normes \'equivalentes sur $E$.
}
\indication{Montrer
\begin{itemize}
  \item $\| f \| \le N(f)$ ;
  \item $\|f'\|_\infty \le \| f \|_\infty + \|f\|$ ;
  \item $\|f\|_\infty \le \|f\|$.
\end{itemize}}
\reponse{
Par l'inégalité triangulaire $|f(x)+f'(x)| \le |f(x)|+|f'(x)|$ on obtient $\| f \| \le N(f)$.
Pour une inégalité dans l'autre sens décomposons le travail :
\begin{itemize}
  \item $\|f'\|_\infty \le \| f \|_\infty + \|f\|$ : en effet
par l'inégalité triangulaire $|f'(x)| \le |f(x)|+|f'(x)+f(x)|$.

  \item $\|f\|_\infty \le \|f\|$ : en effet $f$ est continue sur $[0,1]$
donc elle est bornée et atteint ses bornes. Soit $x_0 \in [0,1]$ ce point du maximum. Si $x_0 \in ]0,1[$ alors $f'(x_0)=0$ donc
$\|f\|_\infty = |f(x_0)| = |f(x_0)+f'(x_0)| \le \|f\|$.
Si $x_0 = 1$ alors $f$ et $f'$ ont m\^eme signe sur un intervalle 
$[1-\epsilon,1]$ donc sur cet intervalle $|f(x)| \le |f(x)+f'(x)|$
et donc $\|f\|_\infty = |f(1)|  \le \|f\|$. (Enfin $f(0)=0$ donc si $x_0=0$
alors $f$ est nulle et l'inégalité est triviale.)

  \item Il reste à rassembler les expressions :
$$ N(f) = \|f'\|_\infty  + \|f\|_\infty \le \|f\|_\infty + \|f\| + \|f\|_\infty \le 3\|f\|.$$
(La première inégalité vient du premier point et la deuxième du second.)
\end{itemize}
Les normes $\| f \|$ et $N(f)$ sont équivalentes :
$$ \frac13 N(f) \le \| f \| \le N(f).$$
}
}

\uuid{QKI8}
\exo7id{6129}
\titre{exo7 6129}
\auteur{queffelec}
\organisation{exo7}
\datecreate{2011-10-16}
\isIndication{false}
\isCorrection{false}
\chapitre{Dualité, topologie faible}
\sousChapitre{Dualité, topologie faible}
\module{Topologie}
\niveau{L3}
\difficulte{}

\contenu{
\texte{
\label{gijsexodual}
On va montrer que le dual de $l^2$ est isométriquement isomorphe à $l^2$.
On note comme d'habitude $e_n$ l'élément de $l^2$ dans toutes les composantes
sont nulles, sauf la $n$-ième qui vaut $1$.
}
\begin{enumerate}
    \item \question{Soit $x\in l^2$. Montrer que la suite d'éléments de $l^2$ $x_n=\sum_1^n
x(k)e_k$ converge vers $x$ dans $l^2$ (autrement dit, les suites nulles 
à partir d'un certain rang sont denses dans $l^2$.)
En déduire que si $f\in(l^2)'$, $f(x)=\sum_1^\infty x(n)f(e_n).$}
    \item \question{Montrer que $\Vert f\Vert \geq (\sum_1^n\vert f(e_k)\vert^2)^{1\over2}$, et
que $(f(e_n))_n$ est un élément de $l^2$.}
    \item \question{Montrer alors que pour tout $x\in l^2$, $\vert f(x)\vert\leq \Vert
x\Vert_2\Vert (f(e_n))\Vert_2$, et que  $\Vert f\Vert=\Vert (f(e_n))\Vert_2$.

En déduire que l'application $f\to(f(e_n))$ est un isomorphisme isométrique du
dual de $l^2$ sur $l^2$.}
\end{enumerate}
}

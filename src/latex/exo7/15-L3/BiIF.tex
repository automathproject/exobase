\uuid{BiIF}
\exo7id{6135}
\titre{exo7 6135}
\auteur{queffelec}
\organisation{exo7}
\datecreate{2011-10-16}
\isIndication{false}
\isCorrection{false}
\chapitre{Continuité, uniforme continuité}
\sousChapitre{Continuité, uniforme continuité}
\module{Topologie}
\niveau{L3}
\difficulte{}

\contenu{
\texte{
Soit $(X,d)$ un espace métrique, et $Y$ un sous-espace non vide de $X$. On va
montrer que toute fonction $f:Y\to \Rr$, $k$-lipschitzienne, admet un
prolongement $g:X\to\Rr$ qui est aussi $k$-lipschitzien. Soit donc $f$ ainsi;
pour tout $x\in X$ et $y\in Y$ , on pose
$$f_y(x)=f(y)+kd(x,y).$$
}
\begin{enumerate}
    \item \question{Montrer que pour $x$ fixé, l'ensemble $\{f_y(x)\}$ lorsque $y$ parcourt $Y$
est minoré. On pose $g(x) = \inf_{y\in Y}\{f_y(x)\}$.}
    \item \question{Montrer que l'application $g$ ainsi définie sur $X$, réalise un prolongement
$k$-lipschitzien de $f$.}
    \item \question{Donner une condition suffisante pour que ce prolongement soit unique.}
\end{enumerate}
}

\uuid{0MAi}
\exo7id{6094}
\titre{exo7 6094}
\auteur{queffelec}
\organisation{exo7}
\datecreate{2011-10-16}
\isIndication{false}
\isCorrection{false}
\chapitre{Espace topologique, espace métrique}
\sousChapitre{Espace topologique, espace métrique}
\module{Topologie}
\niveau{L3}
\difficulte{}

\contenu{
\texte{
On note $X$ l'espace des suites réelles $x=(x(n))$ et on le munit de
la topologie dont les ouverts élémentaires sont
$$V(x;n_1,n_2,\cdots n_k;\varepsilon)=\{y\in X /\vert
x(n_i)-y(n_i)\vert<\varepsilon,\  i=1\cdots k\}.$$
Vérifier qu'on a bien défini une base de topologie.

Comparer la topologie qu'elle engendre sur $l^{\infty}$ et $c_0$ avec la
topologie métrique de l'exercice précédent.
}
}

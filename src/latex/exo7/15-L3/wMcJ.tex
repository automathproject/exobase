\uuid{wMcJ}
\exo7id{2361}
\titre{exo7 2361}
\auteur{mayer}
\organisation{exo7}
\datecreate{2003-10-01}
\isIndication{true}
\isCorrection{true}
\chapitre{Application linéaire bornée}
\sousChapitre{Application linéaire bornée}
\module{Topologie}
\niveau{L3}
\difficulte{}

\contenu{
\texte{
Soient $E_1, E_2$ et $F$ des espaces norm\'es sur $\Rr$
et soit $B: E_1 \times E_2 \to F$ une application bilin\'eaire.
Montrer que $B$ est continue si et seulement s'il existe
$M>0$ tel que
$$ \|B(x) \|\leq M \| x_1\| \| x_2\|\quad  \text{pour tout } x=(x_1,x_2) \in E_1\times E_2 \; .$$
}
\indication{Si la relation est vérifiée montrer que $B$ est continue en $x$ en calculant $B(x+y)-B(x)$.
Si $B$ est continue alors en particulier $B$ est continue en $(0,0)$, fixer 
le $\epsilon$ de cette continuité,...}
\reponse{
Sens $\Leftarrow$.  Soit $M>0$ tel que $ \|B(x) \|\leq M \| x_1\| \| x_2\|$.
Montrons que $B$ en continue au point $x=(x_1,x_2)$ fixé. Soit $y=(y_1,y_2)$ alors
$$B(x+y)-B(x)=B(x_1+y_1,x_2+y_2)-B(x_1,x_2)= B(x_1,y_2)+B(x_2,y_1)+B(y_1,y_2).$$
Donc $$\|B(x+y)-B(x)\| \le  M\| x_1\| \| y_2\|+M\| x_2\| \| y_1\|+M\| y_1\| \| y_2\|.$$
Pour $\|y_1\| \le \frac{\epsilon}{M\| x_1\|}$ on a  $M\| x_1\| \| y_2\| \le \epsilon$
(si $x_1=0$ il n'y a rien à choisir ici). 
Pour $\|y_2\| \le \frac{\epsilon}{M\| x_2\|}$ on a  $M\| x_2\| \| y_1\| \le \epsilon$
(si $x_2=0$ il n'y a rien à choisir ici).
Enfin pour $\|y_1\|\le \sqrt {\frac \epsilon M}$ et $\|y_2\|\le \sqrt {\frac \epsilon M}$
on a $M\| y_1\| \| y_2\|\le \epsilon$.
Donc en prenant $\eta = \min (\frac{\epsilon}{M\| x_1\|},\frac{\epsilon}{M\| x_2\|},\sqrt {\frac \epsilon M})$, on obtient que pour $\|y\| = \max (\|y_1\|,\|y_2\|) \le \eta$ on a $\|B(x+y)-B(x)\| \le 3\epsilon$.
Ce qui prouve la continuité.
Donc $B$ est continue sur $E_1\times E_2$.
Sens $\Rightarrow$. Si $B$ est continue partout, en particulier 
elle est continue en $0$. Je choisis $\epsilon =1$, il existe $\eta >0$
tel que $\|x\| \le \eta$ alors $\|B(x)\|\le 1$. Donc pour $\|x_1\|\le \eta$
et $\|x_2\|\le \eta$ on a $\|B(x_1,x_2)\|\le 1$.
Soit maintenant $y = (y_1,y_2) \in E_1\times E_2$, ($y_1\neq 0, y_2\neq 0$) on a $(\eta \frac{y_1}{\| y_1 \|},\eta \frac{y_2}{\| y_2 \|})$ de norme $\le \eta$ donc
$B(\eta \frac{y_1}{\| y_1 \|},\eta \frac{y_2}{\| y_2 \|}) \le 1$ et par bilinéarité cela fournit :
$B(y_1,y_2)\le \frac{1}{\eta^2}{\| y_1 \|}{\| y_2 \|}$, et ce pour tout $(y_1,y_2)$. La constante cherchée étant $\frac 1{\eta^2}$.
}
}

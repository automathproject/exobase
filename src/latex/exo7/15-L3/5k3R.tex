\uuid{5k3R}
\exo7id{2393}
\titre{exo7 2393}
\auteur{queffelec}
\organisation{exo7}
\datecreate{2003-10-01}
\isIndication{true}
\isCorrection{true}
\chapitre{Théorème de Baire}
\sousChapitre{Théorème de Baire}
\module{Topologie}
\niveau{L3}
\difficulte{}

\contenu{
\texte{
Soit $f$ une application d\'efinie sur un espace m\'etrique complet $(X,d)$,
\`a valeurs r\'eelles et semi-continue inf\'erieurement. Montrer qu'il existe un
ouvert non vide $O$ sur lequel $f$ est major\'ee.

Application : soit $(f_n)$ une suite de formes lin\'eaires continues sur un Banach $B$,
v\'erifiant
$$\forall {x\in B},\;  \sup_n\vert f_n(x)\vert<\infty.$$

En utilisant ce qui pr\'ec\`ede, montrer que $ \sup_n\Vert f_n\Vert<\infty$.
}
\indication{\begin{enumerate}
  \item Une application $f :  X \to \Rr$ est \emph{semi-continue inf\'erieurement}
si 
$$\forall \lambda \in \Rr \qquad  \{ x\in X \mid f(x) > \lambda \} \qquad \text{est un ouvert.}$$
De façon équivalente  $f$ est \emph{semi-continue inf\'erieurement} si pour tout $x \in X$
$$\forall \epsilon >0 \quad \exists \delta >0 \quad \forall y \in X \quad (d(x,y)< \delta \Rightarrow f(x)-f(y) < \epsilon).$$
Attention il n'y a pas de valeur absolue autour de $f(x)-f(y)$.

  \item Pour la première question considérer $O_n =  \{ x\in X \mid f(x) > n \}$ et utiliser le théorème de Baire.

  \item Pour l'application utiliser la première question avec la fonction 
$$\phi : B \to \Rr, \text{ définie par } \phi(x) = \sup_{n\in\Nn} |f_n(x)|.$$
\end{enumerate}}
\reponse{
Par l'absurde supposons que sur aucun ouvert $f$ n'est majorée.
$f :  X \to \Rr$ est {semi-continue inf\'erieurement} donc 
$$\forall \lambda \in \Rr \qquad  O_\lambda := \{ x\in X \mid f(x) > \lambda \} \qquad \text{est un ouvert.}$$
De plus $O_\lambda$ est dense, en effet pour $x \in X$ et pour $V_x$ un voisinage ouvert de $x$,
alors par hypothèse $f$ n'est pas majorée sur $V_x$ donc en particulier il existe $y \in V_x$ tel que $f(y) > \lambda$ donc $y\in V_x \cap O_\lambda$. Ceci prouve  que $O_\lambda$ est dense dans $X$ ($V_x$ étant aussi petit que l'on veut).

Maintenant pour $n = 0,1,2,\ldots$, les $O_n$ sont un ensemble dénombrable d'ouverts denses. Comme $X$ est complet il vérifie le théorème de Baire donc l'intersection des $O_n$ est encore un ensemble dense. Mais il est facile de voir par la définition des $O_n$ que
$$\bigcap_{n\in\Nn}O_n = \varnothing.$$
Ce qui donne la contradiction cherchée.
On note $\phi : B \to \Rr$ la fonction définie par 
$$\phi(x) = \sup_{n\in\Nn} |f_n(x)|.$$
Il n'est pas difficile de montrer que $\phi$ est semi-continue inf\'erieurement :
en effet soit $F_\lambda := \{ x\in X \mid \phi(x) \le \lambda \}$.
Soit $\lambda$  fixé et soit $(x_k)$ une suite d'éléments de $F_\lambda$.
Pour $n$ fixé et pour tout $k$ on a $f_n(x_k) \le k$, donc par continuité de $f_n$,
on a $f_n(x) \le k$, ceci étant vrai pour tout $n$ on a $x\in F_\lambda$.
Donc $F_\lambda$ est un fermé donc $O_\lambda := \{ x\in X \mid f(x) > \lambda \}$ est un ouvert.
Donc $\phi$ est semi-continue inf\'erieurement.

Par la première question il existe un ouvert non vide $O$ et une constante $M>0$ tel que
$\phi$ soit majorée par $M$ sur $O$.
C'est-à-dire
$$\forall n\in \Nn \quad \forall x \in O \quad |f_n(x)| \le M.$$
Par translation on peut supposer que l'origine $o$ est inclus dans $O$. Donc il existe $\epsilon >0$ tel que $\bar B(o,\epsilon) \subset O$. Donc
$$\forall n\in \Nn \quad \forall x \in \bar B(o,\epsilon) \quad |f_n(x)| \le M$$
ce qui est équivalent à 
$$\forall n\in \Nn \quad \forall x \in \bar B(o,1) \quad |f_n(x)| \le \frac M \epsilon$$
Donc $$\forall n\in \Nn \quad  \|f_n\| \le \frac M \epsilon.$$
}
}

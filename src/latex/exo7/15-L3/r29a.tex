\uuid{r29a}
\exo7id{6134}
\titre{exo7 6134}
\auteur{queffelec}
\organisation{exo7}
\datecreate{2011-10-16}
\isIndication{false}
\isCorrection{false}
\chapitre{Continuité, uniforme continuité}
\sousChapitre{Continuité, uniforme continuité}
\module{Topologie}
\niveau{L3}
\difficulte{}

\contenu{
\texte{
Soit $(X,d)$ un espace métrique; on rappelle tout d'abord les propriétés de la
fonction
$d_A : x\to d(x,A)$ où $A$ est une partie de $X$ :
}
\begin{enumerate}
    \item \question{$d_A$ est $1$-lipschitzienne, et $d_A(x)=0$ si et seulement si
$x\in\overline A$. On en déduit que tout fermé est un $G_\delta$ et que tout
ouvert est un $F_\sigma$.}
    \item \question{Montrer qu'un espace métrique possède une propriété forte de séparation, à
savoir : deux fermés disjoints $F_1$ et $F_2$ peuvent être séparés par deux
ouverts disjoints, en considérant $\{x/ d(x,F_1)>d(x,F_2) \}$.}
    \item \question{Montrer que la propriété précédente est équivalente à l'existence d'une
fonction continue $f$ valant $0$ sur $F_1$ et $1$ sur $F_2$.}
    \item \question{Soit $F_1$, $F_2$,...,$F_n$, $n$ fermés disjoints dans $X$, et $c_1$,
$c_2$,...$c_n$, $n$ nombres réels. Montrer que la fonction $f$ valant $c_i$
sur $F_i$ peut se prolonger en une fonction continue à $X$ tout entier.}
\end{enumerate}
}

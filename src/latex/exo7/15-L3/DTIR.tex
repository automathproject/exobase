\uuid{DTIR}
\exo7id{1896}
\titre{exo7 1896}
\auteur{legall}
\organisation{exo7}
\datecreate{2003-10-01}
\isIndication{false}
\isCorrection{false}
\chapitre{Espace vectoriel normé}
\sousChapitre{Espace vectoriel normé}
\module{Topologie}
\niveau{L3}
\difficulte{}

\contenu{
\texte{
On munit $E$, 
l'espace vectoriel des fonctions continues sur $[0,1]$ \`a valeurs 
r\'eelles
telles que $f(0)=0$ de la norme $\Vert f\Vert _{\infty }= 
\displaystyle{ \sup _{x\in [0,1]}\vert f(x)\vert }.$
}
\begin{enumerate}
    \item \question{Soit $\varphi : E\rightarrow \Rr $
une application lin\'eaire. On pose $N(\varphi ) =\displaystyle{ \sup 
_{f\in E; \Vert f \Vert _{\infty}=1}\vert \varphi (f)\vert } .$
Montrer que $\varphi $ est continue si et seulement si $N(\varphi ) $ 
est fini. Montrer que $\varphi \mapsto N(\varphi )$ est une norme
sur l'espace vectoriel des formes lin\'eaires continues sur $E$.}
    \item \question{Calculer $\mu =N(\psi )$ lorsque $\psi $ est d\'efinie en 
posant, pour toute fonction $f\in E$ : $\psi (f) = \displaystyle{\int 
_0 ^1f(t)dt }.$}
    \item \question{Peut-on trouver une fonction $f\in E$ telle que  $\vert \psi 
(f)\vert =\mu $ et $ \Vert f \Vert _{\infty}=1$~?}
\end{enumerate}
}

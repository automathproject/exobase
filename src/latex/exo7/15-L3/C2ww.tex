\uuid{C2ww}
\exo7id{6100}
\titre{exo7 6100}
\auteur{queffelec}
\organisation{exo7}
\datecreate{2011-10-16}
\isIndication{false}
\isCorrection{false}
\chapitre{Espace topologique, espace métrique}
\sousChapitre{Espace topologique, espace métrique}

\contenu{
\texte{
On se donne une application $f:\Rr\to\Rr^n$, et on note $d$ la distance
euclidienne sur $\Rr^n$. A quelles conditions sur
$f$, $\delta(x,y)=d(f(x),f(y))$ définit-elle une distance sur $\Rr$
équivalente topologiquement à la distance usuelle (ie définissant la même
topologie.)?
}
}

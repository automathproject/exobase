\uuid{5m2q}
\exo7id{6111}
\titre{exo7 6111}
\auteur{queffelec}
\organisation{exo7}
\datecreate{2011-10-16}
\isIndication{false}
\isCorrection{false}
\chapitre{Espace topologique, espace métrique}
\sousChapitre{Espace topologique, espace métrique}

\contenu{
\texte{
\label{gijsexopol}
On va montrer que les polyn\^omes sont denses dans les fonctions continues sur
$[-1,1]$. Pour commencer, on approche la fonction $\vert t\vert$.
}
\begin{enumerate}
    \item \question{Montrer que la suite de polyn\^omes définis par récurrence : 
$$p_{n+1}(t) = p_n(t) + {1\over 2} (t^2-p_n^2(t)),\ \ p_0(t)=0,$$
converge vers $\vert t\vert$.}
    \item \question{En déduire que toute fonction affine par morceaux sur
$[-1,1]$ est limite d'une suite de polyn\^omes.}
    \item \question{Montrer que les polyn\^omes sont denses dans les fonctions continues sur
$[-1,1]$.}
\end{enumerate}
}

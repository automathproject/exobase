\uuid{qGXD}
\exo7id{2413}
\titre{exo7 2413}
\auteur{mayer}
\organisation{exo7}
\datecreate{2003-10-01}
\isIndication{true}
\isCorrection{true}
\chapitre{Théorème de Stone-Weirstrass, théorème d'Ascoli}
\sousChapitre{Théorème de Stone-Weirstrass, théorème d'Ascoli}

\contenu{
\texte{
Soient $E,F$ des espaces norm\'es et $(f_n)$ une suite d'applications  de $E$ dans $F$
\'equicontinue en $a\in E$. Montrer que, si la suite $(f_n(a))$ converge vers $b$,
alors $(f_n(x_n))$ converge \'egalement vers $b$, si $(x_n)$ est une suite de $E$
telle que $\lim_{n\to \infty} x_n =a$.

L'\'equicontinuit\'e est-elle n\'ecessaire ici?
}
\indication{Démarrer avec l'inégalité : 
$$|f_n(x_n)-b| \le |f_n(x_n)-f_n(a)|+|f_n(a)-b|.$$

Si $(f_n)$ n'est pas équicontinue le résultat peut \^etre faux. Prendre $f_n(x) =  (1+x)^n$
et $x_n= \frac 1 n$.}
\reponse{
\begin{enumerate}
Soit $(x_n)$ une suite convergeant vers $a$, alors
$$|f_n(x_n)-b| \le |f_n(x_n)-f_n(a)|+|f_n(a)-b|.$$
Soit $\epsilon >0$, il existe $N_1$ tel que pour $n\ge N_1$ on ait
$|f_n(a)-b| < \frac \epsilon 2$.
$(f_n)$ est équicontinue en $a$, donc il existe $\eta >0$ tel que
pour tout $n\in \Nn$ et tout $x\in E$, $(|x-a| < \eta \Rightarrow |f_n(x)-f_n(a)| < \frac \epsilon 2)$.
Comme $x_n\rightarrow a$ alors il existe $N_2$ tel que pour $n\ge N_2$  on ait $|x_n-a| < \eta$.
Donc pour $n \ge \max(N_1,N_2)$ on a $|f_n(x_n)-b| \le |f_n(x_n)-f_n(a)|+|f_n(a)-b| < \frac \epsilon 2 + \frac \epsilon 2 = \epsilon$. Donc
$(f_n(x_n))$ converge vers $b$.
}
}

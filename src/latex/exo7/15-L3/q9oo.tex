\uuid{q9oo}
\exo7id{2405}
\auteur{mayer}
\organisation{exo7}
\datecreate{2003-10-01}
\isIndication{true}
\isCorrection{true}
\chapitre{Théorème du point fixe}
\sousChapitre{Théorème du point fixe}

\contenu{
\texte{
\label{exoiter}
Soit $(X,d)$ un espace m\'etrique complet et soit $f: X\to X$ une
application telle que l'une de ces it\'er\'ees $f^n$ est
strictement contractante, i.e. il existe $\rho <1$ tel que
$$ d(f^n(x),f^n(y)) \leq \rho d(x,y) \quad \mbox{pour tout} \quad
x,y\in X \; .$$ Montrer que $f$ poss\`ede un unique point fixe.
Faire le rapprochement avec l'exercice \ref{exn13}.
}
\indication{Montrer que l'unique point fixe $x$ de $f^n$, est un point fixe de $f$.
Pour cela écrire l'égalité $f^n(x)=x$ et composée habilement cette égalité.
Pour conclure utiliser l'unicité du point fixe de $f^n$.}
\reponse{
Notons $g=f^n$. Alors $g$ est une application strictement contractante dans $X$ complet donc $g$ possède un unique point fixe que nous notons $x$.
Montrons l'unicité d'un point fixe pour $f$. Soit
$y\in X$ tel que $f(y)=y$ alors $g(y)=f^n(y)=y$. Donc $y$ est aussi un point fixe pour $g$, donc $y=x$. 

Il reste à montrer que $f$ possède effectivement bien un tel point fixe.
Nous avons 
\begin{align*}
 & f^n(x)=x \\
 \Rightarrow \qquad & f(f^n(x))=f(x) \\
\Rightarrow \qquad & f^n(f(x))=f(x) \\
\Rightarrow \qquad & g(f(x))=f(x) \\
\end{align*}
Nous venons de prouver que $f(x)$ est un point fixe de $g$. 
Comme $g$ possède un unique point fixe $x$ alors $f(x)=x$ !!
Donc $x$ est bien un point fixe pour $f$.
}
}

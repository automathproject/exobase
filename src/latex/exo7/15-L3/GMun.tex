\uuid{GMun}
\exo7id{2348}
\auteur{queffelec}
\organisation{exo7}
\datecreate{2003-10-01}
\isIndication{false}
\isCorrection{false}
\chapitre{Espace topologique, espace métrique}
\sousChapitre{Espace topologique, espace métrique}

\contenu{
\texte{
On d\'esigne par $d(a,b)$ la distance euclidienne usuelle de $a,b\in\Rr^2$ et
on pose

$$ \delta(a,b)= \left \{\begin{array}{ccc}
&d(a,b)\enskip &\hbox{si}\enskip
a,b\enskip
\hbox{sont align\'es avec l'origine}\enskip O\\
&d(0,a)+d(0,b)\enskip & \hbox{sinon}
\end{array}\right.$$
}
\begin{enumerate}
    \item \question{Montrer que $\delta$ est une distance sur $\Rr^2$ (``distance SNCF") plus
fine que la distance usuelle. 

 Dans
la suite, on suppose $\Rr^2$ muni de la topologie associ\'ee \`a $\delta$.}
    \item \question{Soit $H$ le demi-plan $\{(x,y)\ ;\ y>0\}$~; montrer que $H$ est 
un ouvert ; d\'eterminer $\overline H$.}
    \item \question{Quelle est la topologie induite sur une droite vectorielle; sur
le cercle unit\'e
$\Gamma
$~?}
    \item \question{Lesquelles des transformations suivantes sont continues~: homoth\'eties de centre $O$~;
rotations de centre $O$~; translations~?}
\end{enumerate}
}

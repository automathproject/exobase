\uuid{Nh1y}
\exo7id{2369}
\auteur{mayer}
\organisation{exo7}
\datecreate{2003-10-01}
\isIndication{true}
\isCorrection{true}
\chapitre{Application linéaire bornée}
\sousChapitre{Application linéaire bornée}

\contenu{
\texte{
Soit $X = \{f\in {\cal C}(\Rr ) \; ; \; (1+x^ 2 ) |f(x)| \text{ soit born\'ee}\}$. On pose
$N(f) = \sup_{x\in \Rr}(1+x^ 2 ) |f(x)|$. V\'erifier que $N$ est une norme, puis
montrer que la forme lin\'eaire suivante $L$ est continue et calculer sa norme:
$$ L:X\to \Rr \quad \mbox{d\'efinie par} \quad L(f) = \int _\Rr f(x) \; dx \; .$$
}
\indication{Montrer que $\|L\|=\pi$.}
\reponse{
$N$ est bien une norme. Et on a pour tout $x$, $(1+x^2)|f(x)| \le N(f)$.
$$|L(f)| = |\int_\Rr f | \le \int_\Rr |f| \le  \int_\Rr \frac{N(f)}{1+x^2}dx 
\le N(f) \int_\Rr \frac{1}{1+x^2} = N(f)[\Arctan x]_{-\infty}^{+\infty} = N(f)\pi.$$
Donc pour tout $f$ on a 
$$\frac{\int f}{N(f)} \le \pi.$$
De plus pour $f(x) = \frac{1}{1+x^2}$ on obtient l'égalité. Donc la norme $\|L\|$ de l'application $L$ est $\pi$.
}
}

\uuid{I9u1}
\exo7id{2416}
\titre{exo7 2416}
\auteur{mayer}
\organisation{exo7}
\datecreate{2003-10-01}
\isIndication{true}
\isCorrection{true}
\chapitre{Théorème de Stone-Weirstrass, théorème d'Ascoli}
\sousChapitre{Théorème de Stone-Weirstrass, théorème d'Ascoli}
\module{Topologie}
\niveau{L3}
\difficulte{}

\contenu{
\texte{
On consid\`ere la suite de fonctions $f_n(t) = \sin (\sqrt{t+ 4(n\pi )^2})$,
$t\in [0, \infty [$.
}
\begin{enumerate}
    \item \question{Montrer qu'il s'agit d'une suite de fonctions \'equicontinues convergent
simplement vers $f\equiv 0$.}
    \item \question{La suite $(f_n)$ est elle relativement compacte
dans $({\cal C}_b ([0, \infty [), \|.\|_\infty )$, l'ensemble des fonctions continues et bornées ?
Que dit le th\'eor\`eme d'Ascoli?}
\reponse{
\begin{enumerate}
Pour $t \ge 0$ fixé, alors
\begin{align*}
f_n(t) 
  &= \sin \sqrt{t+4(n\pi)^2} \\
  &= \sin 2n\pi\sqrt{1+\frac{t}{4n^2\pi^2}} \\
  &= \sin 2n\pi(1+\frac12\frac{t}{4n^2\pi^2}+o(\frac{1}{n^2})) \\
  &= \sin (2n\pi + \frac{t}{4n\pi}+o(\frac1n)) \\
  &= \sin (\frac{t}{4n\pi})+o(\frac1n) \\
\end{align*}
Donc quand $n\rightarrow +\infty$ alors $f_n(t) \rightarrow 0$.
Donc $(f_n)$ converge simplement vers $0$.
Pour $n\ge 1$, 
$$|f_n'(t)| = \frac 12 \frac{1}{\sqrt{t+4n^2\pi^2}}\cos\sqrt{t+4n^2\pi^2}
\le \frac 12 \frac{1}{\sqrt{t+4\pi^2}} \le \frac 1{4\pi}. $$
Pour $t\ge0$ fixé et $\epsilon >0$ donné, on pose $\eta = 4\pi\epsilon$, alors
par l'inégalité des accroissement finis
$$\forall n \ge 1 \quad |t-t'| < \eta \Rightarrow |f_n(t)-f_n(t')| \le \frac{1}{4\pi}|t-t'| < \epsilon.$$
Donc $(f_n)$ est une famille équicontinue.
}
\indication{\begin{enumerate}
  \item Pour l'équicontinuité utiliser le théorème des  accroissement finis.
Pour la convergence simple montrer que pour $t$ fixé :
$f_n(t)= \sin (\frac{t}{4n\pi})+o(\frac1n)$.

  \item Montrer que $(f_n)$ ne converge par vers la fonction nulle pour la norme $\|.\|_\infty$
(c'est-\`a-dire il y a convergence simple mais pas convergence uniforme).
Le théorème d'Acoli serait-il faux ? 
\end{enumerate}}
\end{enumerate}
}

\uuid{07D2}
\exo7id{2395}
\auteur{mayer}
\organisation{exo7}
\datecreate{2003-10-01}
\isIndication{true}
\isCorrection{true}
\chapitre{Espace métrique complet, espace de Banach}
\sousChapitre{Espace métrique complet, espace de Banach}

\contenu{
\texte{
L'espace $(\Rr , d)$ est-il complet si $d$ est l'une des
m\'etriques suivantes?
}
\begin{enumerate}
    \item \question{$d(x,y) = |x^3-y^3|$.}
\reponse{Soit $(u_n)$ une suite de Cauchy pour $d$.
Donc
$$\forall \epsilon >0 \quad \exists N\in \Nn \quad \forall p,q \ge N \quad
d(u_p,u_q) = |u_p^3-u_q^3| \le \epsilon.$$
Donc la suite $(u_n^3)$ est une suite de Cauchy pour la distance usuelle $|.|$.
Comme  $(\Rr,|.|)$ est complet alors $(u_n^3)$ converge pour la valeur absolue, notons $v$ la limite,
 nous avons $|u_n^3-v|$ qui tend vers $0$. Donc pour $u=v^{\frac13}$
nous avons $d(u_n,u)=|u_n^3-u^3|=|u_n^3-v|$ qui tend vers $0$, donc
$u_n$ converge vers $u$ pour la distance $d$.
Donc $\Rr$ est complet pour $d$.}
    \item \question{$d(x,y) = |\exp (x) -\exp (y)|$.}
\reponse{Montrons que $d$ ne définit pas une distance complète.
Soit $(u_n)$ la suite définie par $u_n=-n$, $n\in \Nn$.
Alors $d(u_p,u_q) = |e^{-p}-e^{-q}|$.
Donc pour $\epsilon>0$ fixé, soit $N$ tel que $e^{-N}<\frac\epsilon 2$,
alors pour $p,q\ge N$ on a $d(u_p,u_q) = |e^{-p}-e^{-q}| \le e^{-p}+
 e^{-q} \le 2e^{-N} \le \epsilon$. Donc $(u_n)$ est de Cauchy.
Supposons que $(u_n)$ converge, notons $u\in \Rr$ sa limite.
Alors $d(u_n,u)=|e^{-n}-e^u|$ tend vers $0$ d'une part
et vers $e^u$ d'autre part. Donc $e^u=0$ ce qui est absurde pour
$u\in \Rr$.}
    \item \question{$d(x,y) = \log (1+ |x-y|)$.}
\reponse{La fonction $\ln(1+u)$ est continue et ne s'annule qu'en $u=0$. 
Donc pour $\ln(1+u)$ suffisamment petit nous avons $u$ suffisamment petit et donc
(par la relation $\ln(1+u) = u +o(u)$) nous avons 
$$\frac12 u \le \ln(1+u) \le 2u.$$
Donc pour $(u_n)$ une suite de Cauchy pour $d$, la première inégalité prouve 
que $(u_n)$ est une suite de Cauchy pour $|.|$. Donc elle converge pour $|.|$
La deuxième inégalité montre que $(u_n)$ converge pour 
$d$. Donc $d$ définit une distance complète.}
\indication{\begin{enumerate}
  \item C'est une suite de Cauchy. Essayer de se ramener à une suite de Cauchy de $(\Rr,|.|)$.

  \item Regarder la suite définie par $u_n=-n$.

  \item Comme la première question.
\end{enumerate}}
\end{enumerate}
}

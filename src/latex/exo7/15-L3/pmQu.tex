\uuid{pmQu}
\exo7id{2365}
\titre{exo7 2365}
\auteur{mayer}
\organisation{exo7}
\datecreate{2003-10-01}
\isIndication{true}
\isCorrection{true}
\chapitre{Application linéaire bornée}
\sousChapitre{Application linéaire bornée}
\module{Topologie}
\niveau{L3}
\difficulte{}

\contenu{
\texte{
\label{exopol}
Soit $X = \Rr[x]$ l'ensemble des polyn\^omes. Pour $P(x) = \sum _{k=0}^p a_k x^k$
on pose $\|P\|= \sup_k |a_k|$, $U(P)(x) = \sum _{k=1}^n \frac{1}{k} a_k x^k$ et
$V(P)(x) = \sum _{k=1}^n k a_k x^k$.
}
\begin{enumerate}
    \item \question{Montrer que $\|.\|$ d\'efinit une norme et que $U$ et $V$ d\'efinissent des applications
lin\'eaires de $X$ dans $X$.}
    \item \question{Examiner si $U$ et $V$ sont continues?}
\reponse{
Il suffit de l'écrire...
Calculons la norme de $U$ : $\| U(P) \| = \sup_k  |\frac 1k a_k\| \le \sup_k |a_k| \le \| P \|$. Donc pour tout $P$, $\frac{\| U(P) \|}{\| P \|} \le 1$.
Et pour $P(x) = x$ on a égalité donc $\| U \|=1$.
Pour $V$, prenons $P_k(x) = x^k$, alors $\| P_k \|=1$, mais $\|V(P_k)\| = k$. Donc $V$ n'est pas bornée sur la boule unité donc $V$ n'est pas continue.
}
\indication{$U$ est continue et $\| U \|=1$, $V$ n'est pas continue.}
\end{enumerate}
}

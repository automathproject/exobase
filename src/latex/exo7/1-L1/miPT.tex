\uuid{miPT}
\exo7id{73}
\titre{exo7 73}
\auteur{liousse}
\organisation{exo7}
\datecreate{2003-10-01}
\isIndication{false}
\isCorrection{false}
\chapitre{Nombres complexes}
\sousChapitre{Géométrie}
\module{Algèbre}
\niveau{L1}
\difficulte{}

\contenu{
\texte{

}
\begin{enumerate}
    \item \question{Soit $A$ un point du plan 
d'affixe $\alpha = a+ib$. D\'eterminer l'ensemble 
des points $M$ du plan dont l'affixe $z$ v\'erifie 
$|z|^2 = \alpha \bar{z} + \bar{\alpha} z.$}
    \item \question{Quelles conditions doivent v\'erifier les points $M_1$ 
et $M_2$ d'affixes $z_1$ et $z_2$ pour que $\frac{z_1}{z_2}$ soit 
r\'eel ?}
    \item \question{D\'eterminer les nombres 
complexes $z$ tels que les points du plan complexe d'affixes 
$z,$ $iz,$ $i$ forment un triangle \'equilat\'eral.}
    \item \question{Soit $z=a+ib$, mettre l'expression $\frac{z-1}{z+1}$ sous forme $A+iB$, 
. D\'eterminer l'ensemble des points du plan complexe d'affixe 
$z$  telle que l'argument de $\frac{z-1}{z+1}$ soit $\pi\over 2$.}
\end{enumerate}
}

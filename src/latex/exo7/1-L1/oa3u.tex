\uuid{oa3u}
\exo7id{5064}
\auteur{rouget}
\organisation{exo7}
\datecreate{2010-06-30}
\isIndication{false}
\isCorrection{true}
\chapitre{Nombres complexes}
\sousChapitre{Trigonométrie}

\contenu{
\texte{
Résoudre dans $\Rr$ puis dans $[0,2\pi]$ les équations suivantes~:
}
\begin{enumerate}
    \item \question{$\sin x=\frac{1}{2}$,}
\reponse{$\sin x=\frac{1}{2} \Leftrightarrow x\in\left(\frac{\pi}{6}+2\pi\Zz\right)\cup\left(\frac{5\pi}{6}+2\pi\Zz\right)$. De plus,
$\mathcal{S}_{[0,2\pi]}=\left\{\frac{\pi}{6},\frac{5\pi}{6}\right\}$.}
    \item \question{$\sin x=-\frac{1}{\sqrt{2}}$,}
\reponse{$\sin x=-\frac{1}{\sqrt{2}}\Leftrightarrow x\in\left(-\frac{\pi}{4}+2\pi\Zz\right)\cup\left(-\frac{3\pi}{4}+2\pi\Zz\right)$. De plus,
$\mathcal{S}_{[0,2\pi]}=\left\{-\frac{\pi}{4},-\frac{3\pi}{4}\right\}$.}
    \item \question{$\tan x=-1$,}
\reponse{$\tan x=-1\Leftrightarrow x\in-\frac{\pi}{4}+\pi\Zz$. De plus, $\mathcal{S}_{[0,\pi]}=\left\{\frac{3\pi}{4}\right\}$.}
    \item \question{$\tan x=\frac{1}{\sqrt{3}}$,}
\reponse{$\tan x=\frac{1}{\sqrt{3}}\Leftrightarrow x\in\frac{\pi}{6}+\pi\Zz$. De plus,
$\mathcal{S}_{[0,\pi]}=\left\{\frac{\pi}{6}\right\}$.}
    \item \question{$\cos x=\frac{\sqrt{3}}{2}$,}
\reponse{$\cos x=\frac{\sqrt{3}}{2}\Leftrightarrow x\in\left(-\frac{\pi}{6}+\pi\Zz\right)\cup\left(\frac{\pi}{6}+\pi\Zz\right)$. De
plus, $\mathcal{S}_{[0,2\pi]}=\left\{\frac{\pi}{6},\frac{11\pi}{6}\right\}$.}
    \item \question{$\cos x=-\frac{1}{\sqrt{2}}$.}
\reponse{$\cos x=-\frac{1}{\sqrt{2}}\Leftrightarrow x\in\left(-\frac{3\pi}{4}+\pi\Zz\right)\cup\left(\frac{3\pi}{4}+\pi\Zz\right)$. De
plus, $\mathcal{S}_{[0,2\pi]}=\left\{\frac{3\pi}{4},\frac{5\pi}{4}\right\}$.}
\end{enumerate}
}

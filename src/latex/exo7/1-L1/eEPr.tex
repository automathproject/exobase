\uuid{eEPr}
\exo7id{935}
\auteur{cousquer}
\organisation{exo7}
\datecreate{2003-10-01}
\isIndication{false}
\isCorrection{false}
\chapitre{Application linéaire}
\sousChapitre{Image et noyau, théorème du rang}

\contenu{
\texte{
Soit $E$ l'espace vectoriel des polynômes de degré inférieur ou
égal à $n$. Pour $p\leq n$ on note $e_p$ le polynôme
$x\mapsto x^p$. Soit $f$ l'application définie sur $E$ par $f(P)=Q$
avec $Q(x)=P(x+1)+P(x-1)-2P(x)$.
}
\begin{enumerate}
    \item \question{Montrer que $f$ est une application linéaire de $E$ dans $E$.}
    \item \question{Calculer $f(e_p)$~; quel est son degré~? En déduire $\ker f$,
$\mbox{Im }f$ et le rang de $f$.}
    \item \question{Soit $Q$ un polynôme de $\mbox{Im }f$~; montrer qu'il existe un polynôme
unique $P$ tel que~: $f(P)=Q$ et $P(0)=P'(0)=0$.}
\end{enumerate}
}

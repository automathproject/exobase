\uuid{1wBY}
\exo7id{5287}
\auteur{rouget}
\organisation{exo7}
\datecreate{2010-07-04}
\isIndication{false}
\isCorrection{true}
\chapitre{Dénombrement}
\sousChapitre{Cardinal}

\contenu{
\texte{
Combien y a-t-il de surjections de $\{1,...,n+1\}$ sur $\{1,...,n\}$~?
}
\reponse{
Soit $n$ un naturel non nul. Dire que $f$ est une surjection de $\{1,...,n+1\}$ sur $\{1,...,n\}$ équivaut à dire que deux des entiers de $\{1,...,n+1\}$ ont même image $k$ par $f$ et que les autres ont des images deux à deux distinctes et distinctes de $k$. On choisit ces deux entiers : $C_{n+1}^2$ choix et leur image commune : $n$ images possibles ce qui fournit $nC_{n+1}^2$ choix d'une paire de $\{1,...,n+1\}$ et de leur image commune. Puis il y a $(n-1)!$ choix des images des $n-1$ éléments restants. Au total, il y a $n!\frac{n(n+1)}{2}=\frac{n.(n+1)!}{2}$ surjections de $\{1,...,n+1\}$ sur $\{1,...,n\}$.
}
}

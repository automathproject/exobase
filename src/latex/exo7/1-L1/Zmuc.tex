\uuid{Zmuc}
\exo7id{5324}
\titre{exo7 5324}
\auteur{rouget}
\organisation{exo7}
\datecreate{2010-07-04}
\isIndication{false}
\isCorrection{true}
\chapitre{Polynôme, fraction rationnelle}
\sousChapitre{Racine, décomposition en facteurs irréductibles}
\module{Algèbre}
\niveau{L1}
\difficulte{}

\contenu{
\texte{
Soit $P\in\Cc[X]$ de degré supérieur ou égal à $1$.
Montrer que les racines de $P'$ sont barycentres à coefficients positifs des racines de $P$ (on dit que les racines de $P'$ sont dans l'enveloppe convexe des racines de $P$). Indication~:~calculer $\frac{P'}{P}$.
}
\reponse{
Soit $P$ un polynôme de degré $n$ supèrieur ou égal à $2$.

Posons $P=\lambda(X-z_1)(X-z_2)...(X-z_n)$ où $\lambda$ est un complexe non nuls et les $z_k$ des complexes pas nécessairement deux à deux distincts.

$$P'=\lambda\sum_{i=1}^{n}(\prod_{j\neq i}^{}(X-z_j))=\sum_{i=1}^{n}\frac{P}{X-z_i},$$

et donc 
 
$$\frac{P'}{P}=\sum_{i=1}^{n}\frac{1}{X-z_i}.$$

Soit alors $z$ une racine de $P'$ dans $\Cc$. Si z est racine de $P$ (et donc racine de $P$ d'ordre au moins $2$) le résultat est clair. Sinon,

$$0=\frac{P'(z)}{P(z)}=\sum_{i=1}^{n}\frac{1}{z-z_i}=\sum_{i=1}^{n}\frac{\overline{z-z_i}}{|z-z_i|^2}.$$

En posant $\lambda_i=\frac{1}{|z-z_i|^2}$, ($\lambda_i$ est un réel strictement positif) et en conjugant, on obtient
$\sum_{i=1}^{n}\lambda_i(z-z_i)=0$ et donc 
$$z=\frac{\sum_{i=1}^{n}\lambda_iz_i}{\sum_{i=1}^{n}\lambda_i}=\mbox{bar}(z_1(\lambda_1),...,z_n(\lambda_n)).$$
}
}

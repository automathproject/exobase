\uuid{fM8O}
\exo7id{3033}
\titre{exo7 3033}
\auteur{quercia}
\organisation{exo7}
\datecreate{2010-03-08}
\isIndication{false}
\isCorrection{false}
\chapitre{Logique, ensemble, raisonnement}
\sousChapitre{Relation d'équivalence, relation d'ordre}
\module{Algèbre}
\niveau{L1}
\difficulte{}

\contenu{
\texte{
Soit $E$ un ensemble non vide. On consid{\`e}re les relations sur $F = E^E$:
\begin{align*}
  f \sim g &\iff \exists\ n \in \N^* \text{ tq } f^n = g^n,\cr f
  \approx g &\iff \exists\ m,n \in \N^* \text{ tq } f^n = g^m,\cr f
  \equiv g &\iff f(E) = g(E). 
\end{align*}
}
\begin{enumerate}
    \item \question{Montrer que $\sim$, $\approx$, $\equiv$ sont des relations d'{\'e}quivalence.}
    \item \question{Pour $f \in F$, on note $f^\sim$, $f^\approx$, $f^\equiv$ les classes
    d'{\'e}quivalence de $f$ modulo $\sim$, $\approx$, $\equiv$.
  \begin{enumerate}}
    \item \question{Comparer $f^\sim$, $f^\approx$.}
    \item \question{Montrer que toute classe d'{\'e}quivalence pour $\approx$ est r{\'e}union de
        classes d'{\'e}quivalence pour $\sim$.}
    \item \question{Que pouvez-vous dire de $f$ s'il existe $g \in f^\approx$ injective ?
        surjective ?}
    \item \question{M{\^e}me question avec $f^\equiv$.}
\end{enumerate}
}

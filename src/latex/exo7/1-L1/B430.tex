\uuid{B430}
\exo7id{3170}
\titre{exo7 3170}
\auteur{quercia}
\organisation{exo7}
\datecreate{2010-03-08}
\isIndication{false}
\isCorrection{true}
\chapitre{Polynôme, fraction rationnelle}
\sousChapitre{Autre}
\module{Algèbre}
\niveau{L1}
\difficulte{}

\contenu{
\texte{
Soit $k\in\N^*$ et $z_0,\dots,z_k$ des complexes.
Soient les polyn{\^o}mes $P_0 = (X+z_0)^n,\dots,P_k=(X+z_k)^n$.
Donner une condition n{\'e}cessaire et suffisante pour que~$(P_0,\dots,P_k)$ soit une
base de~$C_n[X]$.
}
\reponse{
D{\'e}j{\`a} il est n{\'e}cessaire que $k=n$. Supposant ceci r{\'e}alis{\'e}, la matrice de~$(P_0,\dots,P_k)$
dans la base canonique de~$\C_n[X]$ est {\'e}quivalente {\`a} la matrice de Vandermonde de $z_0,\dots,z_k$.
Donc une CNS est~: $k=n$ et $z_0,\dots,z_k$ sont distincts.
}
}

\uuid{2h7R}
\exo7id{7192}
\titre{exo7 7192}
\auteur{megy}
\organisation{exo7}
\datecreate{2019-07-23}
\isIndication{true}
\isCorrection{false}
\chapitre{Logique, ensemble, raisonnement}
\sousChapitre{Relation d'équivalence, relation d'ordre}
\module{Algèbre}
\niveau{L1}
\difficulte{}

\contenu{
\texte{
Soit $E$ un ensemble fini et $f : E\to E$ une involution, c'est-à-dire une application vérifiant $f\circ f=\operatorname{Id}$. Montrer que si $f$ n'a pas de points fixes, alors $|E|$ est pair. Plus généralement, montrer que la parité de $|E|$ est celle du nombre de points fixes de $f$.
}
\indication{Considérer une certaine relation d'équivalence sur $E$ et écrire que $E$ est réunion disjointe des classes d'équivalence.}
}

\uuid{h2p9}
\exo7id{5166}
\titre{exo7 5166}
\auteur{rouget}
\organisation{exo7}
\datecreate{2010-06-30}
\isIndication{false}
\isCorrection{true}
\chapitre{Espace vectoriel}
\sousChapitre{Définition, sous-espace}
\module{Algèbre}
\niveau{L1}
\difficulte{}

\contenu{
\texte{
Soit $E$ un $\Kk$-espace vectoriel. Soient $A$, $B$ et $C$ trois sous-espaces vectoriels de $E$ vérifiant $A\cap
B=A\cap C$, $A+B=A+C$ et $B\subset C$. Montrer que $B=C$.
}
\reponse{
Il suffit de montrer que $C\subset B$.

Soit $x$ un élément de $C$. Alors $x\in A+C=A+B$ et il existe $(y,z)\in A\times B$ tel que $x=y+z$. Mais $z\in
B\subset C$ et donc, puisque $C$ est un sous-espace vectoriel de $E$, $y=x-z$ est dans $C$. Donc, $y\in A\cap C=A\cap B$
et en particulier $y$ est dans $B$. Finalement, $x=y+z$ est dans $B$. On a montré que tout élément de $C$ est dans $B$
et donc que, $C\subset B$. Puisque d'autre part $B\subset C$, on a $B=C$.
}
}

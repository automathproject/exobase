\uuid{Nmei}
\exo7id{5273}
\titre{exo7 5273}
\auteur{rouget}
\organisation{exo7}
\datecreate{2010-07-04}
\isIndication{false}
\isCorrection{true}
\chapitre{Matrice}
\sousChapitre{Autre}
\module{Algèbre}
\niveau{L1}
\difficulte{}

\contenu{
\texte{
Calculs par blocs.
}
\begin{enumerate}
    \item \question{Soit $M=
\left(
\begin{array}{cc}
A&B\\
C&D
\end{array}
\right)
$ et $N=\left(
\begin{array}{cc}
A'&B'\\
C'&D'
\end{array}
\right)$ avec $(A,A')\in(\mathcal{M}_{p,r}(\Kk))^2$, $(B,B')\in(\mathcal{M}_{p,s}(\Kk))^2$, $(C,C')\in(\mathcal{M}_{q,r}(\Kk))^2$ et $(D,D')\in(\mathcal{M}_{q,s}(\Kk))^2$. Calculer $M+N$ en fonction de $A$, $B$, $C$, $D$, $A'$, $B'$, $C'$ et $D'$.}
\reponse{Soit $(i,j)\in\{1,...,p\}\times\{1,...,r\}$. Le coefficient ligne $i$, colonne $j$, de la matrice $M+N$ est la somme du coefficient ligne $i$, colonne $j$, de la matrice $M$ et du coefficient ligne $i$, colonne $j$, de la matrice $N$ ou encore la somme du coefficient ligne $i$, colonne $j$, de la matrice $A$ et du coefficient ligne $i$, colonne $j$, de la matrice $A'$. On a des résultats analogues pour les autres valeurs du couple $(i,j)$ et donc

$$M+N=\left(
\begin{array}{cc}
A+A'&B+B'\\
C+C'&D+D'
\end{array}
\right).$$}
    \item \question{Question analogue pour $MN$ en analysant précisément les formats de chaque matrice.}
\reponse{Posons $M=\left(
\begin{array}{cc}
A&B\\
C&D
\end{array}
\right)$ et $N=\left(
\begin{array}{cc}
A'&B'\\
C'&D'
\end{array}
\right)$ où $A\in\mathcal{M}_{p,r}(\Kk)$, $B\in\mathcal{M}_{q,r}(\Kk)$, $C\in\mathcal{M}_{p,s}(\Kk)$, $D\in\mathcal{M}_{q,s}(\Kk)$, puis $A'\in\mathcal{M}_{t,p}(\Kk)$, $B'\in\mathcal{M}_{u,p}(\Kk)$, $C'\in\mathcal{M}_{t,q}(\Kk)$, $D'\in\mathcal{M}_{u,q}(\Kk)$ (le découpage de $M$ en colonne est le même que le découpage de $N$ en lignes).

Soit alors $(i,j)\in\{1,...,r\}\times\{1,...,t\}$. Le coefficient ligne $i$, colonne $j$ de la matrice $MN$ vaut 

$$\sum_{k=1}^{p+q}m_{i,k}n_{k,j}=\sum_{k=1}^{p}m_{i,k}n_{k,j}+\sum_{k=p+1}^{p+q}m_{i,k}n_{k,j}.$$

Mais, $\sum_{k=1}^{p}m_{i,k}n_{k,j}$ est le coefficient ligne $i$, colonne $j$ du produit $AA'$ et $\sum_{k=p+1}^{p+q}m_{i,k}n_{k,j}$ est le coefficient ligne $i$, colonne $j$ du produit $BC'$. Finalement, $\sum_{k=1}^{p+q}m_{i,k}n_{k,j}$ est le coefficient ligne $i$, colonne $j$ du produit $AA'+BC'$. On a des résultats analogues pour les autres valeurs du couple $(i,j)$ et donc

$$MN=\left(
\begin{array}{cc}
AA'+BC'&AB'+BD'\\
CA'+DC'&CB'+DD'
\end{array}
\right).$$}
\end{enumerate}
}

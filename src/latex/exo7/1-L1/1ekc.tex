\uuid{1ekc}
\exo7id{106}
\titre{exo7 106}
\auteur{bodin}
\organisation{exo7}
\datecreate{1998-09-01}
\video{nfJizOU7DbA}
\isIndication{true}
\isCorrection{true}
\chapitre{Logique, ensemble, raisonnement}
\sousChapitre{Logique}

\contenu{
\texte{
Soient les quatre assertions suivantes :
$$ (a) \ \exists x\in \Rr \quad \forall y\in \Rr \quad x+y > 0 \quad ; \quad
 (b) \ \forall x\in \Rr \quad \exists y\in \Rr \quad x+y > 0  \ ;$$
$$ (c) \ \forall x\in \Rr \quad \forall y\in \Rr \quad x+y > 0  \quad ; \quad
(d) \ \exists x\in \Rr \quad \forall y\in \Rr \quad y^2 > x .$$
}
\begin{enumerate}
    \item \question{Les assertions $a$, $b$, $c$, $d$ sont-elles vraies ou fausses ?}
    \item \question{Donner leur n\'egation.}
\reponse{
(a) est fausse. Car sa n\'egation qui est 
 $\forall x\in \Rr \ \exists y\in \Rr \quad x+y \leq 0$
est vraie. \'Etant donn\'e $x\in \Rr$ il existe toujours un $y\in\Rr$ tel que
$x+y \leq 0$, par exemple on peut prendre $y=-(x+1)$ et alors $x+y=x-x-1=-1 \leq 0$.
(b) est vraie, pour un $x$ donn\'e, on peut prendre (par exemple) $y=-x+1$
et alors $x+y=1>0$.
La n\'egation de (b) est 
 $\exists x\in \Rr \ \forall y\in \Rr \quad x+y \leq 0$.
(c) : $\forall x\in \Rr \ \forall y\in \Rr \quad x+y > 0$
est fausse, par exemple $x=-1$, $y=0$. La n\'egation est 
$\exists x\in \Rr\ \exists y\in \Rr\ x+y \leq 0$.
(d) est vraie, on peut prendre $x=-1$. La n\'egation est:
$\forall x\in \Rr \ \exists y\in \Rr \quad y^2 \leq x$.
}
\indication{Attention : la n\'egation d'une in\'egalit\'e stricte est une in\'egalit\'e large
(et r\'eci\-pro\-quement).}
\end{enumerate}
}

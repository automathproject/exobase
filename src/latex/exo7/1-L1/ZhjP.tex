\uuid{ZhjP}
\exo7id{225}
\titre{exo7 225}
\auteur{cousquer}
\organisation{exo7}
\datecreate{2003-10-01}
\isIndication{false}
\isCorrection{false}
\chapitre{Dénombrement}
\sousChapitre{Binôme de Newton et combinaison}
\module{Algèbre}
\niveau{L1}
\difficulte{}

\contenu{
\texte{

}
\begin{enumerate}
    \item \question{Montrer que:
    $$\sum_{k=0}^{p}   C_n^kC_{n-k}^{p-k}= 2^pC_n^p , $$ où 
    $p$ et $n$  sont des entiers naturels avec $0\le p\le n$.}
    \item \question{Avec les m\^emes notations, montrer que
    $$\sum_{k=0}^{p} (-1)^{k} C_n^k C_{n-k}^{p-k}= 0.$$}
\end{enumerate}
}

\uuid{EvZ8}
\exo7id{7195}
\titre{exo7 7195}
\auteur{megy}
\organisation{exo7}
\datecreate{2019-07-23}
\isIndication{false}
\isCorrection{false}
\chapitre{Logique, ensemble, raisonnement}
\sousChapitre{Relation d'équivalence, relation d'ordre}
\module{Algèbre}
\niveau{L1}
\difficulte{}

\contenu{
\texte{
Soit $l>0$ un réel et $X=[0,l]\times [-1,1]$, et $\sim$ la relation d'équivalence la plus fine sur $X$ telle que $(0,y)\sim(l,-y)$ pour tout $y\in [-1,1]$.  Décrire le graphe et les classes d'équivalence de la relation. 


Note : l'ensemble quotient $\mathcal M = X/\sim$ est donc l'ensemble obtenu en recollant le rectangle $X=[0,l]\times [-1,1]$ le long de deux bords opposés, en suivant une orientation opposée. On l'appelle le \emph{ruban de Möbius} (de longueur $l$).
}
}

\uuid{v12I}
\exo7id{5593}
\auteur{rouget}
\organisation{exo7}
\datecreate{2010-10-16}
\isIndication{false}
\isCorrection{true}
\chapitre{Application linéaire}
\sousChapitre{Morphismes particuliers}

\contenu{
\texte{
Soit $E$ un espace vectoriel. Soit $G$ un sous-groupe fini de $\mathcal{GL}(E)$ de cardinal $n$. Soit $F$ un sous-espace de $E$ stable par tous les éléments de $G$ et $p$ un projecteur d'image $F$. Montrer que $\frac{1}{n}\sum_{g\in G}^{}g\circ p\circ g^{-1}$ est un projecteur d'image $F$.
}
\reponse{
Soit $q=\frac{1}{n}\sum_{g\in G}^{}g\circ p\circ g^{-1}$.

\begin{center}
$q^2=\frac{1}{n}\sum_{(g,h)\in G^2}^{}h\circ p\circ h^{-1}\circ g\circ p\circ g^{-1}$.
\end{center}

Mais si $g$ et $h$ sont deux éléments de $G$ et $x$ est un vecteur quelconque de $E$, $p(g^{-1}(x))$ est dans $F$ et donc par hypothèse $h^{-1}\circ g\circ p\circ g^{-1}(x)$ est encore dans $F$ ($h^{-1}$ est dans $G$ puisque $G$ est un groupe). On en déduit que

\begin{center}
$h\circ p\circ h^{-1}\circ g\circ p\circ g^{-1}=h\circ h^{-1}\circ g\circ p\circ g^{-1}=g\circ p\circ g^{-1}$.
\end{center}

Mais alors

\begin{center}
$q^2=\frac{1}{n^2}\sum_{(g,h)\in G^2}^{}g\circ p\circ g^{-1}=\frac{1}{n^2}\times n\sum_{g\in G}^{}g\circ p\circ g^{-1}=\frac{1}{n}\sum_{g\in G}^{}g\circ p\circ g^{-1}= q$
\end{center}

et $q$ est un projecteur.

Montrons que $F\subset\text{Im}q$. Soit $x$ un élément de $F$. Pour chaque $g\in G$, $g^{-1}(x)$ est encore dans $F$ et donc $p(g^{-1}(x))=g^{-1}(x)$ puis $g(p(g^{-1}(x)))=x$. Mais alors

\begin{center}
$q(x)=\frac{1}{n}\sum_{g\in G}^{}x=x$,
\end{center}

ou encore $x$ est dans $\text{Im}q$. On a montré que $F\subset\text{Im}q$.

Montrons que $\text{Im}q\subset F$. Soit $x$ un élément de $\text{Im}q$. 

\begin{align*}\ensuremath
p(x)&=p(q(x))=\frac{1}{n}\sum_{g\in G}^{}p\circ g\circ p\circ g^{-1}(x)\\
 &=\frac{1}{n}\sum_{g\in G}^{}g\circ p\circ g^{-1}(x)\;(\text{car}\;p\circ g^{-1}(x)\in F\;\text{et donc}\;g\circ p\circ g^{-1}(x)\in F)\\
 &=q(x) = x,
\end{align*}

et $x$ est dans $F$. On a montré que $\text{Im}q\subset F$ et finalement que $\text{Im}q=F$.

\begin{center}
\shadowbox{
$q$ est un projecteur d'image $F$.
}
\end{center}
}
}

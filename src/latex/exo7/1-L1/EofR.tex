\uuid{EofR}
\exo7id{5628}
\titre{exo7 5628}
\auteur{rouget}
\organisation{exo7}
\datecreate{2010-10-16}
\isIndication{false}
\isCorrection{true}
\chapitre{Matrice}
\sousChapitre{Autre}

\contenu{
\texte{
Pour $A$ matrice nilpotente donnée, on pose $\text{exp}A=\sum_{k=0}^{+\infty}\frac{A^k}{k!}$.
}
\begin{enumerate}
    \item \question{Montrer que si $A$ et $B$ commutent et sont nilpotentes alors $A+B$ est nilpotente et $\text{exp}(A+B) =\text{exp}A\times\text{exp}B$.}
\reponse{Soient $p$ l'indice de nilpotence de $A$ et $q$ l'indice de nilpotence de $B$. Puisque $A$ et $B$ commutent, la formule du binôme de \textsc{Newton} fournit

\begin{center}
$(A+B)^{p+q-1}=\sum_{k=0}^{p+q-1}\dbinom{p+q-1}{k}A^kB^{p+q-1-k}$ 
\end{center}

Dans cette somme, 

\textbullet~si $k\geqslant p$, $A^k=0$ et donc $A^kB^{p+q-1-k}= 0$

\textbullet~si $k\leqslant p-1$ alors $p+q-1-k\geqslant q$ et encore une fois $B^{p+q-1-k}= 0$.

Finalement, $(A+B)^{p+q-1}=\sum_{k=0}^{p+q-1}\dbinom{p+q-1}{k}A^kB^{p+q-1-k}=0$ et $A+B$ est nilpotente d'indice inférieur ou égal à $p+q-1$. 

Les sommes définissant $\text{exp}A$, $\text{exp}B$ et $\text{exp}(A+B)$ sont finies car $A$, $B$ et $A+B$ sont nilpotentes et

\begin{align*}\ensuremath
\text{exp}(A+B)&=\sum_{k=0}^{+\infty}\frac{1}{k!}(A+B)^k=\sum_{k=0}^{+\infty}\sum_{i+j=k}^{}\frac{1}{i!j!}A^iB^j\\
 &\left(\sum_{i=0}^{+\infty}\frac{1}{i!}A^i\right)\left(\sum_{j=0}^{+\infty}\frac{1}{j!}B^j\right)\;(\text{toutes les sommes sont finies})\\
 &=\text{exp}A\times\text{exp}B.
\end{align*}}
    \item \question{Montrer que $\text{exp}A$ est inversible.}
\reponse{Si $A$ est nilpotente, $-A$ l'est aussi et commute avec $A$. Donc $\text{exp}A\times\text{exp}(-A)=\text{exp}(A-A)=\text{exp}(0)=I_n$.

$\text{exp}A$ est inversible à gauche et donc inversible et $(\text{exp}A)^{-1}=\text{exp}(-A)$.}
    \item \question{Calculer $\text{exp}A$ où $A=\left(
\begin{array}{ccccc}
0&1&0&\ldots&0\\
\vdots&\ddots&\ddots&\ddots&\vdots\\
 & & &\ddots&0\\
\vdots& & &\ddots&1\\
0&\ldots& &\ldots&0
\end{array}
\right)$.}
\reponse{Les puissances de $A$ sont bien connues et on trouve immédiatement

\begin{center}
$\text{exp}A=\left(
\begin{array}{ccccc}
1&\frac{1}{1!}&\frac{1}{2!}&\ldots&\frac{1}{(n-1)!}\\
0&\ddots&\ddots&\ddots&\vdots\\
\vdots&\ddots& & &\frac{1}{2!}\\
\vdots& &\ddots&\ddots&\rule[-4mm]{0mm}{11mm}\frac{1}{1!}\\
0&\ldots&\ldots&0&1
\end{array}
\right)$.
\end{center}}
\end{enumerate}
}

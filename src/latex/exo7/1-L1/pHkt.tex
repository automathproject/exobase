\uuid{pHkt}
\exo7id{98}
\titre{exo7 98}
\auteur{ridde}
\organisation{exo7}
\datecreate{1999-11-01}
\isIndication{false}
\isCorrection{false}
\chapitre{Nombres complexes}
\sousChapitre{Autre}
\module{Algèbre}
\niveau{L1}
\difficulte{}

\contenu{
\texte{
Soit $ (a, b, c, d) \in \Rr^4$ tel que $ad-bc = 1$ et $c \neq 0$. Montrer que
si $z \neq -\dfrac dc$ alors $\Im (\dfrac{az + b}{cz + d}) = \dfrac{\Im (z)}
{ \left|(cz + d)\right|^2}$.
}
}

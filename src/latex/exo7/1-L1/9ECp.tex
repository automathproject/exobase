\uuid{9ECp}
\exo7id{440}
\titre{exo7 440}
\auteur{ortiz}
\organisation{exo7}
\datecreate{1999-04-01}
\isIndication{false}
\isCorrection{false}
\chapitre{Polynôme, fraction rationnelle}
\sousChapitre{Autre}

\contenu{
\texte{
Soient $m,n\in [1,+\infty[$, $d=\mathrm{pgcd} (m,n)$ et $P=X^m-1, Q=X^n-1, D=X^d-1 \in\Cc[X]$.
}
\begin{enumerate}
    \item \question{\begin{enumerate}}
    \item \question{Montrer que si $x\in\Cc$ est racine commune de $P$ et $Q$ alors $x$ est racine de $D$
 (on pourra utiliser l'\'egalit\'e de B\'ezout dans $\Zz$).}
    \item \question{Montrer que si $y\in\Cc$ est racine de $D$ alors $y$ est racine commune de $P$ et $Q$
(utiliser la d\'efinition de $d$).}
\end{enumerate}
}

\uuid{oMNG}
\exo7id{1100}
\titre{exo7 1100}
\auteur{legall}
\organisation{exo7}
\datecreate{1998-09-01}
\isIndication{false}
\isCorrection{true}
\chapitre{Matrice}
\sousChapitre{Matrice et application linéaire}

\contenu{
\texte{
Soit  $E$  un espace vectoriel de dimension  $n$  et
$\varphi $  une
application lin\' eaire de  $E$  dans  $E$. Montrer qu'il existe un
polyn\^ome  $P\in {\Rr} [X]$  tel que  $P(f)=0$. (On pourra utiliser le
fait que  $\mathcal{L} (E)$  est isomorphe \`a  $M_n({\Rr})$.)
}
\reponse{
$\mathcal{L} (E)$  est isomorphe \`a  $M_n ({\R})$  donc est de
dimension finie  $n^2$. La famille  $\{ id_E ,  \varphi  , \ldots
,  \varphi ^{n^2}\} $  compte  $n^2+1$  vecteurs donc est li\' ee
c'est \`a dire : il existe  $\lambda _0 , \ldots , \lambda _{n^2}
$  dans  ${\R}$, non tous nuls et tels que  $\lambda _0id_E +
\lambda _1 \varphi  + \cdots +\lambda _{n^2} \varphi ^{n^2}=0$.
 Le polyn\^ome  $P(X)=\lambda _0 + \lambda _1 X  + \cdots
+\lambda _{n^2} X^{n^2}$  r\' epond donc \`a la question.
}
}

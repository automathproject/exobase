\uuid{reLh}
\exo7id{3135}
\titre{exo7 3135}
\auteur{quercia}
\organisation{exo7}
\datecreate{2010-03-08}
\isIndication{false}
\isCorrection{true}
\chapitre{Arithmétique dans Z}
\sousChapitre{Nombres premiers, nombres premiers entre eux}
\module{Algèbre}
\niveau{L1}
\difficulte{}

\contenu{
\texte{
On suppose que $a^r-1$ est un nombre premier. Montrez
que $r$ est premier, puis que $a$ vaut 2. R{\'e}ciproque ?
}
\reponse{
On suppose $a,r$ entiers sup{\'e}rieurs ou {\'e}gaux {\`a}~$2$.

$a-1\mid a^r-1$ donc $a=2$. Si $r=pq$ alors $2^p-1\mid 2^r-1$ donc $r$ est premier.

La r{\'e}ciproque est fausse, $2^{11}-1 = 23\times 89$.
}
}

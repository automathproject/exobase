\uuid{najl}
\exo7id{917}
\titre{exo7 917}
\auteur{ridde}
\organisation{exo7}
\datecreate{1999-11-01}
\video{x8fXA1UP7LU}
\isIndication{true}
\isCorrection{true}
\chapitre{Espace vectoriel}
\sousChapitre{Système de vecteurs}

\contenu{
\texte{
Soit $\alpha \in \Rr$ et $f_\alpha : \Rr \to \Rr$, $x\mapsto e^{\alpha x}$.
Montrer que la famille $(f_\alpha)_{\alpha \in \Rr}$  est libre.
}
\indication{Supposer qu'il existe des r\'eels $\lambda_1,\ldots, \lambda_n$
et des indices  $\alpha_1 > \alpha_2 > \cdots > \alpha_n$ (tout cela en nombre fini !)
tels que 
$$\lambda_1f_{\alpha_1}+\cdots+\lambda_nf_{\alpha_n} = 0.$$
Ici le $0$ est la fonction constante \'egale \`a $0$. 
Regarder quel terme est dominant et factoriser.}
\reponse{
\`A partir de la famille $(f_\alpha)_{\alpha\in \Rr}$ nous
  consid\'erons une combinaison lin\'eaire (qui ne correspond qu'\`a un
  nombre \emph{fini} de termes).

  
  Soient $\alpha_1 > \alpha_2 > \ldots > \alpha_n$ des r\'eels distincts que nous avons ordonnés, consid\'erons la
  famille (finie) : $(f_{\alpha_i})_{i=1,\ldots,n}$. Supposons qu'il
  existe des r\'eels $\lambda_1,\ldots, \lambda_n$ tels que
  $\sum_{i=1}^n \lambda_i f_{\alpha_i}=0$. Cela signifie que, quelque
  soit $x \in \Rr$, alors $\sum_{i=1}^n \lambda_i f_{\alpha_i}(x) = 0$,
autrement dit pour tout $x\in \Rr$ :
$$\lambda_1 e^{\alpha_1 x} + \lambda_2 e^{\alpha_2 x} + \cdots + \lambda_n e^{\alpha_n x}=0.$$
Le terme qui domine est $e^{\alpha_1 x}$ (car $\alpha_1>\alpha_2>\cdots$).
Factorisons par $e^{\alpha_1 x}$ :
$$e^{\alpha_1 x} \Big( \lambda_1  + \lambda_2 e^{(\alpha_2-\alpha_1) x} + \cdots 
+ \lambda_n e^{(\alpha_n-\alpha_1) x} \Big) =0.$$

Mais $e^{\alpha_1 x}\neq 0$ donc :
$$\lambda_1  + \lambda_2 e^{(\alpha_2-\alpha_1) x} + \cdots 
+ \lambda_n e^{(\alpha_n-\alpha_1) x} =0.$$
Lorsque $x\to +\infty$ alors $e^{(\alpha_i-\alpha_1) x} \to 0$ 
(pour tout $i\ge 2$, car $\alpha_i-\alpha_1<0$).
Donc pour $i\ge 2$, $\lambda_i e^{(\alpha_i-\alpha_1) x} \to 0$ et
en passant à la limite dans l'égalité ci-dessus on trouve :
$$\lambda_1=0.$$

Le premier coefficients est donc nul. On repart de la combinaison linéaire qui est maintenant
$\lambda_2 f_2+\cdots + \lambda_n f_n=0$ et en appliquant le raisonnement ci-dessus 
on prouve par récurrence $\lambda_1=\lambda_2=\cdots=\lambda_n=0$.
 Donc la famille $(f_\alpha)_{\alpha\in\Rr}$ est une famille libre.
}
}

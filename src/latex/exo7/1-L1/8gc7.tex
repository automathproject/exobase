\uuid{8gc7}
\exo7id{7188}
\titre{exo7 7188}
\auteur{megy}
\organisation{exo7}
\datecreate{2019-07-17}
\isIndication{false}
\isCorrection{false}
\chapitre{Logique, ensemble, raisonnement}
\sousChapitre{Ensemble}
\module{Algèbre}
\niveau{L1}
\difficulte{}

\contenu{
\texte{
Soit $E$ un ensemble et  $\mathcal A$ une partie de $\mathcal P(E)$. On dit que $\mathcal A$ est une \emph{algèbre de parties $E$} si les conditions suivantes sont vérifiées:
\begin{itemize}
\item $\mathcal A$ n'est pas vide.
\item Si $X\in \mathcal A$, alors $E\setminus X$ aussi.
\item $\mathcal A$ est stable par union finie, autrement dit : pour tout $n\in \N^*$ et toute famille $U_1, \cdots U_n$ d'éléments de $\mathcal A$, on a $\bigcup_{i=1}^n U_i\in \mathcal A$.
\end{itemize}
}
\begin{enumerate}
    \item \question{Montrer que $\mathcal P(E)$ est une algèbre de parties de $E$.}
    \item \question{Montrer  qu'une algèbre de parties de $E$ est stable par intersection finie.}
    \item \question{Combien d'algèbres de parties y a-t-il si $E$ a (exactement) un, deux, ou trois éléments ?}
\end{enumerate}
}

\uuid{eI7J}
\exo7id{3239}
\titre{exo7 3239}
\auteur{quercia}
\organisation{exo7}
\datecreate{2010-03-08}
\isIndication{false}
\isCorrection{true}
\chapitre{Polynôme, fraction rationnelle}
\sousChapitre{Racine, décomposition en facteurs irréductibles}
\module{Algèbre}
\niveau{L1}
\difficulte{}

\contenu{
\texte{
Soit $P = \sum_{k=0}^n a_kX^k \in {\R[X]}$ dont les racines sont r{\'e}elles simples.
}
\begin{enumerate}
    \item \question{D{\'e}montrer que : $\forall\ x \in \R$, on a $P(x)P''(x) \le P'^2(x)$.}
\reponse{Soit $P(x) = a_n\prod_{k=1}^n(x-x_k)$.
      On a : $\sum_{k=0}^n \frac 1{(x-x_k)^2}
               = -\frac d{dx}\Bigl(\frac {P'}P\Bigr)(x) = \frac {P'^2 - PP''}{P^2}(x)$.}
    \item \question{D{\'e}montrer que : $\forall\ k \in \{1,\dots, n-1\},\ a_{k-1}a_{k+1} \le a_k^2$.}
\reponse{Pour $k=1,x=0$, on a : $a_0a_2 \le \frac 12a_1^2$.\\
      Pour $k$ quelconque : on applique le cas pr{\'e}c{\'e}dent {\`a} $P^{(k-1)}$ dont les
      racines sont encore r{\'e}elles simples : \\
      $(k-1)!a_{k-1} \times \frac {(k+1)!}2a_{k+1} \le \frac 12(k!a_k)^2
        \Rightarrow 
       a_{k-1}a_{k+1} \le \frac {k}{k+1}a_k^2$.}
\end{enumerate}
}

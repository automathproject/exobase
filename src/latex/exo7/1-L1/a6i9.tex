\uuid{a6i9}
\exo7id{197}
\titre{exo7 197}
\auteur{bodin}
\organisation{exo7}
\datecreate{1998-09-01}
\video{h9jrsWR1bYw}
\isIndication{false}
\isCorrection{true}
\chapitre{Injection, surjection, bijection}
\sousChapitre{Injection, surjection}
\module{Algèbre}
\niveau{L1}
\difficulte{}

\contenu{
\texte{
Si $z=x+iy$, $(x,y)\in \Rr^2$, on pose $e^z=e^x \times e^{iy}$.
}
\begin{enumerate}
    \item \question{D\'eterminer le module et l'argument de $e^z$.}
\reponse{Pour $z=x+iy$, le  module de $e^z=e^{x+iy}=e^xe^{iy}$ est $e^x$ et
son argument est $y$.}
    \item \question{Calculer $e^{z+z'}, e^{\overline{z}}, e^{-z}, \left( e^z \right)^n \text{ pour } n \in \Zz$.}
\reponse{Les r\'esultats : $e^{z+z'} = e^ze^{z'}$, $e^{\overline{z}} =
\overline{e^z}$, $e^{-z}= \left( e^{z} \right)^{-1}$,
$(e^z)^n=e^{nz}$.}
    \item \question{L'application $\exp : \Cc \rightarrow \Cc, z \mapsto e^z$, est-elle
injective ?, surjective ?}
\reponse{La fonction $\exp$ n'est pas surjective car $|e^z| = e^x >0$ et
donc $e^z$ ne vaut jamais $0$. La fonction $\exp$ n'est pas non
plus injective car pour $z\in\Cc$, $e^z=e^{z+2i\pi}$.}
\end{enumerate}
}

\uuid{A7ot}
\exo7id{3047}
\titre{exo7 3047}
\auteur{quercia}
\organisation{exo7}
\datecreate{2010-03-08}
\isIndication{false}
\isCorrection{false}
\chapitre{Logique, ensemble, raisonnement}
\sousChapitre{Relation d'équivalence, relation d'ordre}
\module{Algèbre}
\niveau{L1}
\difficulte{}

\contenu{
\texte{
Soit $E$ un ensemble et
${\cal E} = \{ (A,f)$ tq $A \subset E,\ A\ne\varnothing,$ et $f\in E^A \}$.
On ordonne $\cal E$ par :
$$(A,f) \preceq (B,g) \iff \begin{cases}A \subset B \cr
                                       \forall\ x \in A,\ f(x) = g(x)\end{cases}$$

(c'est-{\`a}-dire que la fonction $g$, d{\'e}finie sur $B$, prolonge la fonction $f$,
d{\'e}finie seulement sur $A$).
}
\begin{enumerate}
    \item \question{Montrer que $\preceq$ est une relation d'ordre. L'ordre est-il total ?}
    \item \question{Soient $(A,f)$ et $(B,g)$ deux {\'e}l{\'e}ments de $\cal E$. Trouver une CNS
     pour que la partie $\{ (A,f), (B,g) \}$ soit major{\'e}e.
     Quelle est alors sa borne sup{\'e}rieure ?}
    \item \question{M{\^e}me question avec minor{\'e}e.}
\end{enumerate}
}

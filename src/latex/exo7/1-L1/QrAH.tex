\uuid{QrAH}
\exo7id{943}
\auteur{legall}
\organisation{exo7}
\datecreate{1998-09-01}
\video{yGZYazY1EhM}
\isIndication{true}
\isCorrection{true}
\chapitre{Application linéaire}
\sousChapitre{Image et noyau, théorème du rang}

\contenu{
\texte{
Soit $E$  un espace vectoriel de dimension  $n$
et $f$ une application lin\' eaire de  $E$  dans lui-m\^eme.
Montrer que les deux assertions qui suivent sont \' equivalentes :
}
\begin{enumerate}
    \item \question{[(i)] $\ker f = \Im f$}
    \item \question{[(ii)] $f^2=0 \ \text{ et } \  n=2\cdot \text{rg}(f)$}
\reponse{
\begin{itemize}
  \item[(i) $\Rightarrow$ (ii)] Supposons
$\ker f = \Im f$. Soit $x\in E$, alors $f(x) \in \Im f$ donc $f(x)
\in \ker f$, cela entraîne $f(f(x)) = 0$ ; donc $f^2=0$. De plus
d'apr\`es la formule du rang $\dim \ker f + \text{rg}  (f) = n$, mais $\dim
\ker f = \dim \Im f = \text{rg}  f$, ainsi $2\text{rg}  (f)=n$.
  \item[(ii) $\Rightarrow$ (i)] Si $f^2 = 0$ alors
$\Im f \subset \ker f$ car pour $y\in \Im f$ il existe $x$ tel que
$y=f(x)$ et $f(y)=f^2(x)=0$. De plus si $2 \text{rg}  (f) = n$ alors la
formule du rang donne $\dim \ker f = \text{rg}  (f)$ c'est-\`a-dire $\dim \ker f =
\dim \Im f$. Nous savons donc que $\Im f$ est inclus dans $\ker f$
mais ces espaces sont de m\^eme dimension donc sont \'egaux :
$\ker f = \Im f$.
 \end{itemize}
}
\indication{Pour chacune des implications utiliser la formule du rang.}
\end{enumerate}
}

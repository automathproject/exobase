\uuid{IOuX}
\exo7id{303}
\titre{exo7 303}
\auteur{bodin}
\organisation{exo7}
\datecreate{1998-09-01}
\video{atC2uUlIQ64}
\isIndication{false}
\isCorrection{true}
\chapitre{Arithmétique dans Z}
\sousChapitre{Pgcd, ppcm, algorithme d'Euclide}

\contenu{
\texte{
Notons $a=1\;111\;111\;111$ et $b=123\;456\;789$.
}
\begin{enumerate}
    \item \question{Calculer le quotient et le reste de la division euclidienne de $a$ par $b$.}
\reponse{$a=9b+10$.}
    \item \question{Calculer $p=\, \text{pgcd}(a,b)$.}
\reponse{Calculons le pgcd par l'algorithme d'Euclide. 
$a = 9b+10$, $b = 12345678 \times 10 + 9$, $10 = 1 \times 9 +1$.
Donc le pgcd vaut $1$;}
    \item \question{D\'eterminer deux entiers relatifs $u$ et $v$ tels que $au+bv=p$.}
\reponse{Nous reprenons les \'equations pr\'ec\'edentes en partant de la fin:
$1 = 10 - 9$, puis nous rempla\c{c}ons $9$ gr\^{a}ce \`a la deuxi\`eme \'equation 
de l'algorithme d'Euclide:
$1 = 10 - (b - 12345678 \times 10) = -b + 1234679 \times 10$. Maintenant nous 
rempla\c{c}ons $10$ gr\^{a}ce \`a la premi\`ere \'equation:
$1 = -b+12345679 (a-9b) = 12345679a - 111111112b$.}
\end{enumerate}
}

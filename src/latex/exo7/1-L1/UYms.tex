\uuid{UYms}
\exo7id{5179}
\titre{exo7 5179}
\auteur{rouget}
\organisation{exo7}
\datecreate{2010-06-30}
\isIndication{false}
\isCorrection{true}
\chapitre{Espace vectoriel}
\sousChapitre{Base}

\contenu{
\texte{

}
\begin{enumerate}
    \item \question{Soit $n$ un entier naturel. Montrer que si $n$ n'est pas un carré parfait alors $\sqrt{n}\notin\Qq$.}
\reponse{Soit $n$ un entier naturel supèrieur ou égal à $2$.

Si $\sqrt{n}\in\Qq$, il existe $(a,b)\in(\Nn^*)^2$ tel que $\sqrt{n}=\frac{a}{b}$ ou encore tel que $n.b^2=a^2$. Mais
alors, par unicité de la décomposition d'un entier naturel supèrieur ou égal à 2 en facteurs premiers, tous les
facteurs premiers de $n$ ont un exposant pair ce qui signifie exactement que $n$ est un carré parfait.

Si $n=0$ ou $n=1$, $\sqrt{n}\in\Qq$ et $n$ est d'autre part un carré parfait. On a montré que~:

$$\forall n\in\Nn,\;(\sqrt{n}\in\Qq\Rightarrow n\;\mbox{est un carré parfait})$$

ou encore par contraposition

$$\forall n\in\Nn,\;(n\;\mbox{n'est pas un carré parfait}\Rightarrow\sqrt{n}\notin\Qq).$$}
    \item \question{Soit $E=\{a+b\sqrt{2}+c\sqrt{3}+d\sqrt{6},\;(a,b,c,d)\in\Qq^4\}$. Vérifier que $E$ est un $\Qq$-espace
vectoriel puis déterminer une base de $E$.}
\reponse{D'après 1), $\sqrt{2}$, $\sqrt{3}$ et $\sqrt{6}$ sont irrationnels.

$E=\mbox{Vect}_\Qq(1,\sqrt{2},\sqrt{3},\sqrt{6})$ et donc, $E$ est un $\Qq$-espace
vectoriel et $(1,\sqrt{2},\sqrt{3},\sqrt{6})$ en est une famille génératrice.

Montrons que cette famille est $\Qq$-libre.

Soit $(a,b,c,d)\in\Qq^4$.

\begin{align*}
a+b\sqrt{2}+c\sqrt{3}+d\sqrt{6}=0&\Rightarrow(a+d\sqrt{6})^2=(-b\sqrt{2}-c\sqrt{3})^2
\Rightarrow a^2+2ad\sqrt{6}+6d^2=2b^2+2bc\sqrt{6}+3c^2\\
 &\Rightarrow a^2-2b^2-3c^2+6d^2=2(-ad+bc)\sqrt{6}
\end{align*}

Puisque $\sqrt{6}\notin\Qq$, on obtient $a^2-2b^2-3c^2+6d^2=2(-ad+bc)=0$ (car si $bc-ad\neq0$,
$\sqrt{6}=\frac{a^2-2b^2-3c^2+6d^2}{2(-ad+bc)}\in\Qq$) ou encore,

$$\left\{
\begin{array}{l}
a^2-3c^2=2b^2-6d^2\quad(1)\\
ad=bc\quad(2)
\end{array}
\right..$$

De même,

\begin{align*}
a+b\sqrt{2}+c\sqrt{3}+d\sqrt{6}=0&\Rightarrow(a+c\sqrt{3})^2=(-b\sqrt{2}-d\sqrt{6})^2
\Rightarrow(a^2+2ac\sqrt{3}+3c^2=2b^2+4bd\sqrt{3}+6d^2\\
 &\Rightarrow\left\{
\begin{array}{l}
a^2+3c^2=2b^2+6d^2\quad(3)\\
ac=2bd\quad(4)
\end{array}
\right..
\end{align*}

(puisque $\sqrt{3}$ est irrationnel). En additionnant et en retranchant (1) et (3), on obtient $a^2=2b^2$ et
$c^2=2d^2$.Puisque $\sqrt{2}$ est irrationnel, on ne peut avoir $b\neq0$ (car alors $\sqrt{2}=\pm\frac{a}{b}\in\Qq$) 
ou $d\neq0$. Donc, $b=d=0$ puis  $a=c=0$. Finalement, la famille $(1,\sqrt{2},\sqrt{3},\sqrt{6})$ est $\Qq$-libre et
est donc une base de E.}
\end{enumerate}
}

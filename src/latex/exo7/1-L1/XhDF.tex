\uuid{XhDF}
\exo7id{2441}
\titre{exo7 2441}
\auteur{matexo1}
\organisation{exo7}
\datecreate{2002-02-01}
\isIndication{false}
\isCorrection{false}
\chapitre{Matrice}
\sousChapitre{Matrice et application linéaire}
\module{Algèbre}
\niveau{L1}
\difficulte{}

\contenu{
\texte{
Soit $E$ un espace vectoriel de dimension
finie $n$, et $u, v$ deux endomorphismes de $E$.
}
\begin{enumerate}
    \item \question{Montrer que $u \circ v =0$ si et seulement si l'image de
$v$ est contenue dans le noyau de $u$.}
    \item \question{Soit $(e_1, \ldots, e_n)$ une base de $E$. On suppose
dans cette question que $u$ et $v$ s'expriment dans cette base par
$$u(e_1) = e_1, \qquad u(e_i) = 0\quad\hbox{si}\quad i \neq 1,$$
$$v(e_2) = e_2, \qquad v(e_i) = 0\quad\hbox{si}\quad i \neq 2.$$
Trouver les matrices de $u$, $v$ et $u\circ v$ dans cette base.}
    \item \question{Si $u$ est un endomorphisme quelconque non nul de $E$, quelle
condition doit v\'erifier le noyau de $u$ pour qu'il existe un
endomorphisme non nul $v$ tel que $u\circ v =0$\,? Dans ce cas, $u$
est-il bijectif\,?}
\end{enumerate}
}

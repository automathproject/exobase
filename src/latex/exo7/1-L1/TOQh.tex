\uuid{TOQh}
\exo7id{2897}
\titre{exo7 2897}
\auteur{quercia}
\organisation{exo7}
\datecreate{2010-03-08}
\isIndication{false}
\isCorrection{false}
\chapitre{Injection, surjection, bijection}
\sousChapitre{Application}
\module{Algèbre}
\niveau{L1}
\difficulte{}

\contenu{
\texte{
Soit $E$ un ensemble et $f : E \to E$ bijective.

La conjugaison par $f$ est l'application
${\Phi_f} : {E^E} \to  {E^E},  \phi \mapsto {f\circ \phi\circ f^{-1}}$
}
\begin{enumerate}
    \item \question{Montrer que $\Phi_f$ est une bijection de $E^E$.}
    \item \question{Simplifier $\Phi_f \circ \Phi_g$.}
    \item \question{Simplifier $\Phi_f(\phi) \circ \Phi_f(\psi)$.}
    \item \question{Soient $\cal I$, $\cal S$, les sous-ensembles de $E^E$ constitu{\'e}s
    des injections et des surjections.
    Montrer que $\cal I$ et $\cal S$ sont invariants par $\Phi_f$.}
    \item \question{Lorsque $\phi$ est bijective, qu'est-ce que $\Bigl(\Phi_f(\phi)\Bigr)^{-1}$ ?}
\end{enumerate}
}

\uuid{DKsl}
\exo7id{5111}
\auteur{rouget}
\organisation{exo7}
\datecreate{2010-06-30}
\isIndication{false}
\isCorrection{true}
\chapitre{Injection, surjection, bijection}
\sousChapitre{Bijection}

\contenu{
\texte{
Parmi $f\circ g\circ h$, $g\circ h\circ f$ et $h\circ f\circ g$ deux sont injectives et une est surjective. Montrer que
$f$, $g$ et $h$ sont bijectives.
}
\reponse{
On peut supposer sans perte de généralité que $f\circ g\circ h$ et $g\circ h\circ f$ sont injectives et que $h\circ
f\circ g$ est surjective. D'après l'exercice \ref{exo:suprou8}, puisque $f\circ g\circ h=(f\circ g)\circ h$ est injective, $h$ est injective
et puisque $h\circ f\circ g=h\circ(f\circ g)$ est surjective, $h$ est surjective. Déjà $h$ est bijective. Mais alors,
$h^{-1}$ est surjective et donc $f\circ g=h^{-1}\circ(h\circ f\circ g)$ est surjective en tant que composée de
surjections. Puis $h^{-1}$ est injective et donc $f\circ g=(f\circ g\circ h)\circ h^{-1}$ est injective. $f\circ g$ est
donc bijective. $f\circ g$ est surjective donc $f$ est surjective. $g\circ h\circ f$ est injective donc $f$ est
injective. Donc $f$ est bijective. Enfin $g=f^{-1}\circ(f\circ g)$ est bijective en tant que composée de bijections.
}
}

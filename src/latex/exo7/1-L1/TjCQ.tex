\uuid{TjCQ}
\exo7id{2898}
\titre{exo7 2898}
\auteur{quercia}
\organisation{exo7}
\datecreate{2010-03-08}
\isIndication{false}
\isCorrection{false}
\chapitre{Injection, surjection, bijection}
\sousChapitre{Application}
\module{Algèbre}
\niveau{L1}
\difficulte{}

\contenu{
\texte{
Soient $E,F$ deux ensembles. On dit que :
\begin{tabular}{lll}
$E$ est moins puissant que $F$
         &s'il existe une injection
         &$f:\ E \to F$\\
$E$ est plus puissant que  $F$
         &s'il existe une surjection
         &$f:\ E \to F$\\
$E$ et $F$ sont {\'e}quipotents
         &s'il existe une bijection
         &$f:\ E \to F$.      
    \end{tabular}
}
\begin{enumerate}
    \item \question{D{\'e}montrer que : ($E$ est moins puissant que $F$) $\iff$
                     ($F$ est plus puissant que $E$).}
    \item \question{Montrer que $\N$, $\N^*$,
     $\{n \in \N$ tq $n$ est divisible par $3 \}$, et $\Z$ sont
     deux {\`a} deux {\'e}quipotents.}
    \item \question{D{\'e}montrer que $E$ est moins puissant que ${\cal P}(E)$.}
    \item \question{Soit $f : E \to {{\cal P}(E)}$ quelconque et
     $A = \{ x \in E$ tq $x \notin f(x) \}$. Prouver que $A \notin f(E)$.}
    \item \question{Est-ce que $E$ et ${\cal P}(E)$ peuvent {\^e}tre {\'e}quipotents ?}
    \item \question{Soit $G$ un troisi{\`e}me ensemble. Si $E$ est moins puissant que $F$,
     d{\'e}montrer que $E^G$ est moins puissant que~$F^G$.}
\end{enumerate}
}

\uuid{1qqf}
\exo7id{3335}
\titre{exo7 3335}
\auteur{quercia}
\organisation{exo7}
\datecreate{2010-03-09}
\isIndication{false}
\isCorrection{true}
\chapitre{Application linéaire}
\sousChapitre{Image et noyau, théorème du rang}

\contenu{
\texte{
Soit $E$ de dimension finie et $f,g \in \mathcal{L}(E)$ tels que :
$\begin{cases} f\circ g = 0 \cr f+g \in GL(E).\end{cases}$

Montrer que $\mathrm{rg} f + \mathrm{rg} g = \dim E$.
}
\reponse{
$\Im f \subset \mathrm{Ker} g  \Rightarrow  \mathrm{rg} f + \mathrm{rg} g \le \dim E$.\par
         $f+g$ est surjective $ \Rightarrow  \Im f + \Im g = E
                                \Rightarrow  \mathrm{rg} f + \mathrm{rg} g \ge \dim E$.
}
}

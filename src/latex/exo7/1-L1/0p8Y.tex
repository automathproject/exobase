\uuid{0p8Y}
\exo7id{3291}
\auteur{quercia}
\organisation{exo7}
\datecreate{2010-03-08}
\isIndication{false}
\isCorrection{true}
\chapitre{Polynôme, fraction rationnelle}
\sousChapitre{Fraction rationnelle}

\contenu{
\texte{
Soit $P\in\R[X]$ unitaire de degr{\'e}~$n$ et $Q(X) = X(X-1)\dots(X-n)$.

Calculer $\sum_{k=0}^n\frac{P(k)}{\prod_{i\ne k}(k-i)}$ et en d{\'e}duire
l'existence de $k\in{[[0,n]]}$ tel que $|P(k)|\ge \frac{n!}{2^n}$.
}
\reponse{
$\frac PQ = \sum_{k=0}^n\frac{P(k)}{(X-k)\prod_{i\ne k}(k-i)}$
donc $\sum_{k=0}^n\frac{P(k)}{\prod_{i\ne k}(k-i)} = \lim_{x\to\infty}\frac{xP(x)}{Q(x)}=1$.

Si l'on suppose $|P(k)|< \frac{n!}{2^n}$ pour tout $k\in{[[0,n]]}$ alors
$\left|\sum_{k=0}^n\frac{P(k)}{\prod_{i\ne k}(k-i)}\right| <
\frac{1}{2^n}\sum_{k=0}^n\frac{n!}{k!\,(n-k)\!} = 1$, contradiction.
}
}

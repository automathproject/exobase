\uuid{MfRw}
\exo7id{3060}
\titre{exo7 3060}
\auteur{quercia}
\organisation{exo7}
\datecreate{2010-03-08}
\isIndication{false}
\isCorrection{true}
\chapitre{Dénombrement}
\sousChapitre{Autre}
\module{Algèbre}
\niveau{L1}
\difficulte{}

\contenu{
\texte{
Soit $f : {\N} \to {\N}$ telle que :
$\forall\ n \in \N,\ f(f(n)) < f(n+1)$.
On veut montrer que $f = \text{id}_{\N}$.
\par {\it (Olympiades 1977)}
}
\begin{enumerate}
    \item \question{Montrer que $\forall\ n\in \N,\ \forall\ x \ge n,\ f(x) \ge n$.}
    \item \question{Soit $n \in \N$ et $a \ge n$ tel que $f(a) = \min\{f(x)$ tq $x\ge n\}$.
    Montrer que $a=n$.}
    \item \question{En d{\'e}duire que $f$ est strictement croissante, puis conclure.}
\reponse{
R{\'e}currence sur $n$.
Si $a > n$, alors $a = b+1$ avec $b \ge n$, donc $f(b) \ge n$,
    donc $f(f(b)) \ge f(a)$. Contradiction.
}
\end{enumerate}
}

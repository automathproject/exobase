\uuid{5ma0}
\exo7id{21}
\titre{exo7 21}
\auteur{bodin}
\organisation{exo7}
\datecreate{1998-09-01}
\isIndication{false}
\isCorrection{true}
\chapitre{Nombres complexes}
\sousChapitre{Forme cartésienne, forme polaire}
\module{Algèbre}
\niveau{L1}
\difficulte{}

\contenu{
\texte{
Soient $\alpha$ et $\beta$ deux nombres r\'eels. Mettre le nombre complexe $z  =
e^{i\alpha}+e^{i\beta}$ sous forme trigonom\'etrique $z = \rho e^{i\gamma}$ (indication :
poser $u = \frac {\alpha + \beta}{2}$, $v = \frac {\alpha - \beta}{2}$).

En d\'eduire la valeur de
$$ \sum_{p=0}^{n}C_n^p \cos  [p\alpha + (n-p)\beta ].$$
}
\reponse{
Soit $(\alpha,\beta)\in\R^2$ et $z$ le nombre complexe
$z=e^{i\alpha}+e^{i\beta}$. Soit $u=\frac{\alpha+\beta}{2}$ et
$v=\frac{\alpha-\beta}{2}$. Alors, $\alpha=u+v$ et $\beta=u-v$ et
:
\begin{align*}
   z &= e^{i\alpha}+e^{i\beta}\\
     &= e^{iu+iv} + e^{iu-iv} \\
     &= e^{iu} (e^{iv}+e^{-iv}) \\
     &= 2 \cos(v)e^{iu} \\
     &= 2 \cos(\frac{\alpha-\beta}{2})e^{i\frac{\alpha+\beta}{2}}
\end{align*}
On en d\'eduit la forme trigonom\'etrique de $z$ :
$$
   |z|=2|\cos(\frac{\alpha-\beta}{2})|
 \quad\text{ et, lorsque $\cos(\frac{\alpha-\beta}{2})\neq0$ :}\quad
 $$
$$ Arg(z)=
   \begin{cases}
      \frac{\alpha+\beta}{2} [2\pi] & \text{ si $\cos\frac{\alpha-\beta}{2}> 0$}\\
      \pi+\frac{\alpha+\beta}{2} [2\pi] & \text{ si $\cos\frac{\alpha-\beta}{2}< 0$}\\
   \end{cases}
$$
(Attention, si $\cos\frac{\alpha-\beta}{2}<0$, $z=2\cos ve^{iu}$
n'est pas la forme trigonom\'etrique de $z$ !).

Soit $n\in\N$. Calculons $z^n$ de deux façons diff\'erentes :
d'une part
$$z^n=(e^{i\alpha}+e^{i\beta})^n = \sum_{p=0}^n{C_{n}^p e^{ip\alpha}e^{i(n-p)\beta}},$$ et d'autre
part, en utilisant la forme obtenue plus haut: $z^n=2^n
\cos^n{v}\,e^{inu}$. En comparant les parties r\'eelles des
expressions obtenues on obtient :
$$
\sum_{p=0}^n C_{n}^p \cos[p\alpha+(n-p)\beta] =
   2^n\cos^n\frac{\alpha-\beta}{2}\cos(n\frac{\alpha+\beta}{2}).
$$
}
}

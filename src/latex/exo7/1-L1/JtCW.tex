\uuid{JtCW}
\exo7id{231}
\titre{exo7 231}
\auteur{bodin}
\organisation{exo7}
\datecreate{1998-09-01}
\isIndication{false}
\isCorrection{false}
\chapitre{Dénombrement}
\sousChapitre{Binôme de Newton et combinaison}
\module{Algèbre}
\niveau{L1}
\difficulte{}

\contenu{
\texte{
Soient $E$ un ensemble non vide et $X ,Y$ une partition de $E$.
}
\begin{enumerate}
    \item \question{Montrer que l'application suivante est une bijection :
    $$\mathcal{P}(E)\longrightarrow \mathcal{P}(X)\times \mathcal{P}(Y)$$
    $$A\mapsto(A \cap X, A \cap Y)$$}
    \item \question{Montrer que pour $p,q,r\in \Nn$ tel que $r \le p+q$ on a :
    $$\sum_{i+j=r}C_p^iC_q^j=C_{p+q}^r.$$}
    \item \question{En d\'eduire que :
    $$\sum_{k=0}^{n}(C_n^k)^2=C_{2n}^n.$$}
\end{enumerate}
}

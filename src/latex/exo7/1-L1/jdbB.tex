\uuid{jdbB}
\exo7id{5577}
\titre{exo7 5577}
\auteur{rouget}
\organisation{exo7}
\datecreate{2010-10-16}
\isIndication{false}
\isCorrection{true}
\chapitre{Espace vectoriel}
\sousChapitre{Dimension}
\module{Algèbre}
\niveau{L1}
\difficulte{}

\contenu{
\texte{
\label{ex:rou15}
Soient $F_1$, $F_2$,..., $F_n$ $n$ sous-espaces vectoriels d'un espace $E$ de dimension finie sur $\Kk$ $(n\geqslant2)$.

Montrer que $\text{dim}(F_1+...+F_n)\leqslant\text{dim}F_1+ ... +\text{dim}F_n$ avec égalité si et seulement si la somme est directe.
}
\reponse{
Montrons par récurrence que $\forall n\geqslant2$, $\text{dim}(F_1+\ldots+F_n)\leqslant\text{dim}(F_1)+\ldots+\text{dim}(F_n)$.

\textbullet~Pour $n=2$, $\text{dim}(F_1+F_2)=\text{dim}(F_1)+\text{dim}(F_2)-\text{dim}(F_1\cap F_2)\leqslant\text{dim}(F_1)+\text{dim}(F_2)$.

\textbullet~Soit $n\geqslant2$. Supposons que si $F_1$,\ldots, $F_n$ sont $n$ sous-espaces de $E$, $\text{dim}(F_1+\ldots+F_n)\leqslant\text{dim}(F_1)+\text{dim}(F_n)$.

Soient $F_1$,\ldots, $F_{n+1}$ $n+1$ sous-espaces de $E$.

\begin{align*}\ensuremath
\text{dim}(F_1+F_2+...+F_{n+1})&\leqslant\text{dim}(F_1+...+F_n) +\text{dim}(F_{n+1})\;(\text{d'après le cas}\;n=2)\\
 &\leqslant \text{dim}(F_1)+ ... +\text{dim}(F_{n+1})\;(\text{par hypothèse de récurrence}).
\end{align*} 

Le résultat est démontré par récurrence.

On sait que si la somme $F_1+\ldots+F_n$ est directe, on a $\text{dim}(F_1+\ldots+F_n)=\text{dim}(F_1)+\ldots+\text{dim}(F_n)$.

Montrons par récurrence que $\forall n\geqslant,\;2\;[\text{dim}(F_1+...+F_n)=\text{dim}F_1 + ... + \text{dim}F_n]\Rightarrow\text{la somme}\;F_1+\ldots+F_n\;\text{est directe}]$.

\textbullet~Pour n=2 , d'après le \ref{ex:rou14}, $\text{dim}(F_1+F_2)=\text{dim}(F_1)+\text{dim}(F_2)\Rightarrow\text{dim}(F_1\cap F_2)=0\Rightarrow F_1\cap F_2=\{0\}$.

\textbullet~Soit $n\geqslant2$. Soient $F_1$,...,$F_{n+1}$ $n+1$ sous-espaces de $E$ tels que $\text{dim}(F_1+...+F_{n+1})=\text{dim}(F_1)+...+\text{dim}(F_{n+1})$.

On sait que

\begin{align*}\ensuremath
\text{dim}(F_1)+...+\text{dim}(F_{n+1})&=\text{dim}(F_1+...+F_{n+1})\\ 
 &=\text{dim}(F_1+...+F_n) +\text{dim}(F_{n+1})-\text{dim}((F_1+...+F_n)\cap F_{n+1})\\
 &\leqslant\text{dim}(F_1)+...+\text{dim}(F_{n+1})-\text{dim}((F_1+...+F_n)\cap F_{n+1}),
\end{align*}

 
et donc $\text{dim}((F_1+...+F_n)\cap F_{n+1})\leqslant0$ puis $\text{dim}((F_1+...+F_n)\cap F_{n+1})= 0$. Par suite $(F_1+...+F_n)\cap F_{n+1}=\{0\}$ et aussi $\text{dim}(F_1)+...+\text{dim}(F_{n+1})=\text{dim}(F_1+...+F_n) +\text{dim}(F_{n+1})$ et donc $\text{dim}(F_1+...+F_n)=\text{dim}(F_1)+ ... +\text{dim}(F_n)$.

Mais alors, par hypothèse de récurrence, la somme $F_1+...+F_n$ est directe et si l'on rappelle que 
$(F_1+...+F_n)\cap F_{n+1}=\{0\}$, on a montré que la somme $F_1+...+F_{n+1}$ est directe.

Le résultat est démontré par récurrence.
}
}

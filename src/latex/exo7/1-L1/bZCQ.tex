\uuid{bZCQ}
\exo7id{5165}
\auteur{rouget}
\organisation{exo7}
\datecreate{2010-06-30}
\isIndication{false}
\isCorrection{true}
\chapitre{Espace vectoriel}
\sousChapitre{Définition, sous-espace}

\contenu{
\texte{
On munit $\Rr^n$ des lois produit usuelles. Parmi les sous-ensembles suivants $F$ de $\Rr^n$, lesquels
sont des sous-espaces vectoriels~?

$$\begin{array}{lll}
1)\;F=\{(x_1,...,x_n)\in\Rr^n/\;x_1=0\}&2)\;F=\{(x_1,...,x_n)\in\Rr^n/\;x_1=1\}\\
3)\;F=\{(x_1,...,x_n)\in\Rr^n/\;x_1=x_2\}&4)\;F=\{(x_1,...,x_n)\in\Rr^n/\;x_1+...+x_n=0\}\\
5)\;F=\{(x_1,...,x_n)\in\Rr^n/\;x_1.x_2=0\}
\end{array}$$
}
\reponse{
\begin{itemize}
[\textbf{1ère démarche.}] $F$ contient le vecteur nul $(0,...,0)$ et donc $F\neq\varnothing$.
Soient alors $((x_1,...,x_n),(x_1' ,...,x_n'))\in F^2$ et $(\lambda,\mu)\in\Rr^2$. On a

$$\lambda(x_1,...,x_n)+\mu(x_1',...,x_n')=(\lambda x_1+\mu x_1',...,\lambda x_n+\mu x_n')$$

avec $\lambda x_1+\mu x_1'=0$. Donc, $\lambda(x_1,...,x_n)+\mu(x_1',...,x_n')\in F$. $F$ est un sous-espace
vectoriel de $\Rr^n$.
[\textbf{2ème démarche.}] L'application $(x_1,...,x_n)\mapsto x_1$ est une forme linéaire sur $\Rr^n$ et $F$ en
est lenoyau. $F$ est donc un sous-espace vectoriel de $\Rr^n$.
[\textbf{3ème démarche.}]

\begin{align*}
F&=\{(0,x_2,...,x_n),\;(x_2,...,x_n)\in\Rr^{n-1}\}
=\{x_2(0,1,0,...,0)+...+x_n(0,...,0,1),\;(x_2,...,x_n)\in\Rr^{n-1}\}\\
 &=\mbox{Vect}((0,1,0,...,0),...,(0,...,0,1)).
\end{align*}

$F$ est donc un sous-espace vectoriel de $\Rr^n$.

\end{itemize}
$F$ ne contient pas le vecteur nul et n'est donc pas un sous-espace vectoriel de $\Rr^n$.
(Ici, $n\geq2$). L'application $(x_1,...,x_n)\mapsto x_1-x_2$ est une forme linéaire sur $\Rr^n$ et $F$ en
est le noyau. $F$ est donc un sous-espace vectoriel de $\Rr^n$.
L'application $(x_1,...,x_n)\mapsto x_1+...+x_n$ est une forme linéaire sur $\Rr^n$ et $F$ en est le
noyau. $F$ est donc un sous-espace vectoriel de $\Rr^n$.
(Ici, $n\geq2$). Les vecteurs $e_1=(1,0,...,0)$ et $e_2=(0,1,0...,0)$ sont dans $F$ mais $e_1+e_2=
(1,1,0...0)$ n'y est pas. $F$ n'est donc pas un sous espace vectoriel de $E$.

\textbf{Remarque.} $F$ est la réunion des sous-espaces $\{(x_1,...,x_n)\in\Rr^n/\;x_1=0\}$ et
$\{(x_1,...,x_n)\in\Rr^n/\;x_2=0\}$.
}
}

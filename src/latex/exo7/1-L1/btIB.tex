\uuid{btIB}
\exo7id{7151}
\titre{exo7 7151}
\auteur{megy}
\organisation{exo7}
\datecreate{2017-05-01}
\isIndication{false}
\isCorrection{true}
\chapitre{Nombres complexes}
\sousChapitre{Géométrie}
\module{Algèbre}
\niveau{L1}
\difficulte{}

\contenu{
\texte{
%[d'après bac France, juin 2008]
Le plan est muni d'un repère orthonormé direct $(O,\vec u, \vec v)$. On note $A$ et $B$ les points d'affixes $z_A=1-i$ et $z_B = 7+\frac72 i$.
}
\begin{enumerate}
    \item \question{Soit $s$ l'application du plan dans lui-même qui envoie un point d'affixe $z$ sur celui d'affixe 
\[  \frac23 iz +\frac13-\frac53 i.\]
Déterminer la nature et les éléments caractéristiques de $s$. % centre $A$...}
\reponse{L'application $s$ est une similitude directe, de rapport $2/3$, d'angle $arg(i)=\pi/2$, et de centre d'affixe $\frac{\frac{1-5i}{3}}{1-\frac{2}{3i}} = \frac{13-13i}{13}=1-i=z_A$.}
    \item \question{On note $B_0=B$ et pour tout $n\in \N$, 
on note $B_{n+1} = s(B_n)$.
\begin{enumerate}}
\reponse{\begin{enumerate}}
    \item \question{Calculer la distance $AB_{n+1}$ en fonction de $AB_n$.}
\reponse{On a $AB_{n+1}=s(A)s(B_n) = \frac{2}{3}AB_n$.}
    \item \question{Déterminer le plus petit entier $N$ vérifiant la propriété suivante :  pour tout $n\geq N$, le point $B_n$ appartient au disque de centre $A$ et de rayon $10^{-2}$. On demande une formule exacte pour cet entier mais pas son écriture explicite en base $10$.}
\reponse{D'après la question précédente, la suite réelle $AB_n$ est géométrique de raison $2/3$ et de terme initial $AB$, donc :
\[ \forall n \in \N, AB_{n} = \left(\frac{2}{3}\right)^n AB.\]

Soit maintenant $n\in \N$. Le point $B_n$ appartient au disque de centre $A$ et de rayon $10^{-2}$ ssi $AB_n \leq 10^{-2}$.
Or, on a 
\begin{align*}
AB_n \leq 10^{-2} & \Leftrightarrow \left(\frac{2}{3}\right)^n AB \leq 10^{-2} \\
& \Leftrightarrow n \ln\left(\frac{2}{3}\right)+\ln(AB) \leq -\ln(100)\\
& \Leftrightarrow n\ln\left(\frac{2}{3}\right) \leq -\ln(AB)-\ln(100)\\
& \Leftrightarrow n \geq \frac{\ln(AB)+\ln(100)}{\ln(3)-\ln(2)}
\end{align*}

On en déduit que le plus petit entier $N$ tel que $\forall n\geq n, AB_n \leq 10^{-2}$ est $\left \lceil \frac{\ln(AB)+\ln(100)}{\ln(3)-\ln(2)} \right \rceil$ (partie entière supérieure du réel).

Comme $AB = 15/2$, on peut calculer explicitement $n$ à l'aide d'une calculatrice et on trouve $N= 17$, mais ce n'était pas demandé.}
    \item \question{Déterminer l'ensemble des entiers $n$ tels que les points $A$, $B$ et $B_n$ soient alignés.}
\reponse{Pour tout $n$, les points $A$, $B$ et $B_n$ sont distincts. Ils sont alignés ssi
\[ 2\cdot arg\left(\frac{b_n-a}{b-a}\right)=0 \:(\in \R/2\pi\Z).\]
Or on a :
\begin{align*}
b_{n+1}-a
&= s(b_n)-s(a)\\
&= \frac23i b_{n}+\frac13 - \frac53i - \left( \frac23i a+\frac13 - \frac53i\right)\\
&= \frac23 i (b_n-a).
\end{align*}
La suite complexe $(b_n-a)_{n\in\N}$ est donc une suite géométrique de raison $\frac23i$, et donc pour tout $n\in \N$ on a 

\[ b_n-a = \left(\frac23i\right)^n \cdot (b-a).\]

On en déduit que  dans $\R/2\pi\Z$, on a :
\[ 2\cdot arg\left[\frac{b_n-a}{b-a}\right]
= 2\cdot arg \left[\left(\frac23i\right)^n \right]
= 2n\cdot arg \left(\frac23i\right)
= 2n\cdot \frac{\pi}{2} 
= n\pi,
\] et donc que $A$, $B$ et $B_n$ sont alignés ssi $n\pi\equiv 0 [2\pi] \Leftrightarrow n\equiv 0 [2]$, autrement dit ssi $n$ est pair.}
\end{enumerate}
}

\uuid{s3ot}
\exo7id{246}
\titre{exo7 246}
\auteur{gourio}
\organisation{exo7}
\datecreate{2001-09-01}
\isIndication{false}
\isCorrection{false}
\chapitre{Dénombrement}
\sousChapitre{Autre}
\module{Algèbre}
\niveau{L1}
\difficulte{}

\contenu{
\texte{
Soit $E$ un ensemble de cardinal $nm\in \Nn^{*}$, o\`{u} $(n,m)\in (\Nn^{*})^{2}$,
 et $P_{n,m} $l'ensemble des partitions de $E$ en $n$ parties \`{a} $m $
\'{e}l\'{e}ments chacune. Montrer que :
$$N_{n,m}=card(P_{n,m})=\frac{(nm)!}{n!(m!)^{n}}.$$
(\emph{Indication} : on peut proc\'{e}der par r\'{e}currence.)
}
}

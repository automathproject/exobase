\uuid{wnRk}
\exo7id{5351}
\auteur{rouget}
\organisation{exo7}
\datecreate{2010-07-04}
\isIndication{false}
\isCorrection{true}
\chapitre{Polynôme, fraction rationnelle}
\sousChapitre{Racine, décomposition en facteurs irréductibles}

\contenu{
\texte{
Soient $x_1$,..., $x_8$ les zéros de $X^8+X^7-X+3$. Calculer $\sum_{}^{}\frac{x_1}{x_2x_3}$ (168 termes).
}
\reponse{
Pour chacun des $8$ numérateurs possibles, il y a $C_7^2=21$ dénominateurs et donc au total, $8\times21=168$ termes.

$$\sum_{}^{}\frac{x_1}{x_2x_3}=\sum_{}^{}\frac{x_1^2x_4x_5x_6x_7x_8}{x_1x_2...x_8}=\frac{1}{\sigma_8}\sum_{}^{}x_1^2x_2x_3x_4x_5x_6=\frac{1}{3}\sum_{}^{}x_1^2x_2x_3x_4x_5x_6.$$

Ensuite,

$$\sigma_1\sigma_6=(\sum_{}^{}x_1)(\sum_{}^{}x_1x_2x_3x_4x_5x_6)=\sum_{}^{}x_1^2x_2x_3x_4x_5x_6+\sum_{}^{}x_1x_2x_3x_4x_5x_6x_7,$$
 
et donc, 

$$\sum_{}^{}x_1^2x_2x_3x_4x_5x_6=\sigma_1\sigma_6-\sigma_7=(-1)(0)-1=-1.$$

Donc, $\sum_{}^{}\frac{x_1}{x_2x_3}=-\frac{1}{3}$.
}
}

\uuid{bk4J}
\exo7id{3353}
\titre{exo7 3353}
\auteur{quercia}
\organisation{exo7}
\datecreate{2010-03-09}
\isIndication{false}
\isCorrection{false}
\chapitre{Application linéaire}
\sousChapitre{Image et noyau, théorème du rang}
\module{Algèbre}
\niveau{L1}
\difficulte{}

\contenu{
\texte{
Un idéal à gauche de $\mathcal{L}(E)$ est un sev ${\cal I}$ de $\mathcal{L}(E)$ tel que :
$\forall\ f \in {\cal I},\ \forall\ g \in \mathcal{L}(E),\ f\circ g \in {\cal I}$.

Soit ${\cal I}$ un idéal à gauche.
}
\begin{enumerate}
    \item \question{Montrer que si $f \in {\cal I}$ et $\Im g \subset \Im f$,
     alors $g \in {\cal I}$.}
    \item \question{Soient $f_1, f_2 \in {\cal I}$. Montrer qu'il existe $g_1, g_2 \in \mathcal{L}(E)$
     tels que $\Im(f_1\circ g_1 + f_2\circ g_2) = \Im f_1 + \Im f_2$.}
    \item \question{Soit $f \in {\cal I}$ tel que rg($f$) soit maximal.
     Montrer que ${\cal I} = \{g \in \mathcal{L}(E) \text{ tq } \Im g \subset \Im f\}
                   = \{ f\circ g \text{ tq } g \in \mathcal{L}(E) \}$.}
\end{enumerate}
}

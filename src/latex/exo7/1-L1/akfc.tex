\uuid{akfc}
\exo7id{5566}
\titre{exo7 5566}
\auteur{rouget}
\organisation{exo7}
\datecreate{2010-10-16}
\isIndication{false}
\isCorrection{true}
\chapitre{Espace vectoriel}
\sousChapitre{Système de vecteurs}

\contenu{
\texte{
Les familles suivantes de $\Rr^4$ sont-elles libres ou liées ? Fournir des relations de dépendance linéaire quand ces relations existent.
}
\begin{enumerate}
    \item \question{$(e_1,e_2,e_3)$ où $e_1=(3,0,1,-2)$, $e_2=(1,5,0,-1)$ et $e_3=(7,5,2,1)$.}
\reponse{La matrice de la famille $(e_1,e_2,e_3)$ dans la base canonique de $\Rr^4$ est $\left(
\begin{array}{ccc}
3&1&7\\
0&5&5\\
1&0&2\\
-2&-1&1
\end{array}
\right)$. Les trois dernières équations du système $\lambda e_1+\mu e_2+\nu e_3=0$ d'inconnues $\lambda$, $\mu$ et $\nu$ forment un sous-système de matrice $A=\left(
\begin{array}{ccc}
0&5&5\\
1&0&2\\
-2&-1&1
\end{array}
\right)$

En développant le déterminant de cette matrice suivant sa première colonne,  on obtient $\text{det}(A)=-10-2\times10=-30\neq0$. Ce sous-système est de \textsc{Cramer} et admet donc l'unique solution $(\lambda,\mu,\nu)=(0,0,0)$. Par suite, la famille $(e_1,e_2,e_3)$ est libre.}
    \item \question{$(e_1,e_2,e_3,e_4)$ où $e_1=(1,1,1,1)$, $e_2=(1,1,1,-1)$, $e_3=(1,1,-1,1)$ et $e_4=(1,-1,1,1)$.}
\reponse{\begin{align*}\ensuremath
\left|
\begin{array}{cccc}
1&1&1&1\\
1&1&1&-1\\
1&1&-1&1\\
1&-1&1&1
\end{array}
\right|
&=\left|
\begin{array}{cccc}
1&1&1&1\\
0&0&0&-2\\
0&0&-2&0\\
0&-2&0&0
\end{array}
\right|\;(\text{pour}\;2\leqslant i\leqslant4,\;L_i\leftarrow L_i-L_1)\\
 &=\left|
\begin{array}{ccc}
0&0&-2\\
0&-2&0\\
-2&0&0
\end{array}
\right|=-8\neq0.
\end{align*}

Donc la famille $(e_1,e_2,e_3,e_4)$ est une famille libre (et donc une base de $E$).}
    \item \question{$(e_1,e_2,e_3,e_4)$ où $e_1=(0,0,1,0)$, $e_2=(0,0,0,1)$, $e_3=(1,0,0,0)$ et $e_4=(0,1,0,0)$.}
\reponse{Notons $(u_1,u_2,u_3,u_4)$ la base canonique de $\Rr^4$.

La famille $(e_1,e_2,e_3,e_4)=(u_3,u_4,u_1,u_2)$ a même rang que la famille $(u_1,u_2,u_3,u_4)$ c'est-à-dire $4$. La famille $(e_1,e_2,e_3,e_4)$ est donc une base de $\Rr^4$.}
    \item \question{$(e_1,e_2,e_3,e_4)$ où $e_1=(2,-1,3,1)$, $e_2=(1,1,1,1)$, $e_3=(4,1,5,3)$ et $e_4=(1,-2,2,0)$.}
\reponse{La matrice de la famille $(e_2,e_1,e_3,e_4)$ dans la base canonique de $\Rr^4$ est 
$\left(
\begin{array}{cccc}
1&2&4&1\\
1&-1&1&-2\\
1&3&5&2\\
1&1&3&0
\end{array}
\right)$. Cette matrice a même rang que les matrices suivantes :

$\left(
\begin{array}{cccc}
1&0&0&0\\
1&-3&-3&-3\\
1&1&1&1\\
1&-1&-1&-1
\end{array}
\right)$  ($e_5=e_1-2e_2$, $e_6=e_3-4e_2$ et $e_7=e_4-e_2$)

$\left(
\begin{array}{cccc}
1&0&0&0\\
1&-3&0&0\\
1&1&10&0\\
1&-1&0&0
\end{array}
\right)$ ($e_8=e_6-e_5$ et $e_9=e_7-e_5$).

La matrice ci-dessus est de rang $2$. Il en est de même de la famille $(e_1,e_2,e_3,e_4)$ qui est en particulier liée. La nullité de la troisième colonne fournit $0=e_8=e_6-e_5=(e_3-4e_2)-(e_1-2e_2)=-e_1-2e_2+e_3$ et donc $e_3=e_1+2e_2$. La nullité de la quatrième colonne fournit $0=e_9=e_7-e_5=(e_4-e_2)-(e_1-2e_2)=e_4+e_2-e_1$ et donc $e_4=e_1-e_2$.}
\end{enumerate}
}

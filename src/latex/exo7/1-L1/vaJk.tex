\uuid{vaJk}
\exo7id{3332}
\auteur{quercia}
\organisation{exo7}
\datecreate{2010-03-09}
\isIndication{false}
\isCorrection{true}
\chapitre{Application linéaire}
\sousChapitre{Image et noyau, théorème du rang}

\contenu{
\texte{
Soient $E,F$ deux ev, $E$ de dimension finie, et $f,g \in \mathcal{L}(E,F)$.
}
\begin{enumerate}
    \item \question{Démontrer que $\mathrm{rg}(f+g) \le \mathrm{rg}(f) + \mathrm{rg}(g)$.}
    \item \question{Montrer qu'il y a égalité si et seulement si
    $\Im f \cap \Im g = \{\vec 0_F \}$ et $\mathrm{Ker} f + \mathrm{Ker} g = E$.}
\reponse{
2. $\mathrm{rg}(f+g) = \mathrm{rg}(f) +\mathrm{rg}(g)$
           $\iff \begin{cases}\Im f \cap \Im g = \{\vec 0_F \} \cr
                        \Im( f+g ) = \Im f + \Im g       \cr\end{cases}$ \par
           $\iff \begin{cases}\Im f \cap \Im g = \{\vec 0_F \} \cr
                        \forall\ \vec x,\vec y,\ \exists\ \vec z\text{ tq } f(\vec x)+g(\vec y)=(f+g)(\vec z).\end{cases}$

    

    $ \Rightarrow $ : Donc $f(\vec x - \vec z) = g(\vec z - \vec y) = \vec 0$.
            Pour $\vec y = \vec 0 : \vec x = (\vec x - \vec z) + \vec z \in \mathrm{Ker} f + \mathrm{Ker} g$.

    $\Leftarrow$ : Soient $\vec x = \vec x_f + \vec x_g$ et $\vec y = \vec y_f + \vec y_g$ :
            Alors $f(\vec x) + g(\vec y) = f(\vec x_g) + g(\vec y_f) = (f+g)(\vec x_g + \vec y_f)$.
}
\end{enumerate}
}

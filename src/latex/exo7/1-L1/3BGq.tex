\uuid{3BGq}
\exo7id{6958}
\titre{exo7 6958}
\auteur{blanc-centi}
\organisation{exo7}
\datecreate{2014-04-01}
\video{FljXmG2ie3Q}
\isIndication{true}
\isCorrection{true}
\chapitre{Polynôme, fraction rationnelle}
\sousChapitre{Pgcd}
\module{Algèbre}
\niveau{L1}
\difficulte{}

\contenu{
\texte{
\
}
\begin{enumerate}
    \item \question{Montrer que si $A$ et $B$ sont deux polynômes à coefficients 
dans $\Q$, alors le quotient et le reste de la division 
euclidienne de $A$ par $B$, ainsi que $\pgcd(A,B)$, sont 
aussi à coefficients dans $\Qq$.}
\reponse{Lorsqu'on effectue la division euclidienne $A=BQ+R$, 
les coefficients de $Q$ sont obtenus par des opérations 
élémentaires (multiplication, division, addition) à partir 
des coefficients de $A$ et $B$ : ils restent donc dans $\Qq$. 
De plus, $R=A-BQ$ est alors encore à coefficients rationnels. 

Alors $\pgcd(A,B)=\pgcd(B,R)$ et pour l'obtenir, on fait 
la division euclidienne de $B$ par $R$ (dont le quotient 
et le reste sont encore à coefficients dans $\Qq$), puis on 
recommence... Le pgcd est le dernier reste non nul, c'est 
donc encore un polynôme à coefficients rationnels.}
    \item \question{Soit $a,b,c\in\Cc^*$ distincts, et $0<p<q<r$ des entiers. 
Montrer que si $P(X)=(X-a)^p(X-b)^q(X-c)^r$ est à coefficients 
dans $\Qq$, alors $a,b,c \in \Qq$.}
\reponse{Notons $P_1=\pgcd(P,P')$: comme $P$ est à coefficients 
rationnels, $P'$ aussi et donc $P_1$ aussi. 
Or $P_1(X)=(X-a)^{p-1}(X-b)^{q-1}(X-c)^{r-1}$. En itérant 
le processus, on obtient que $P_{r-1}(X)=(X-c)$ est à 
coefficients rationnels, donc $c\in\Qq$.

On remonte alors les étapes: $P_{q-1}(X)=(X-b)(X-c)^{r-q+1}$ 
est à coefficients rationnels, et $X-b$ aussi en tant que 
quotient de $P_{q-1}$ par le polynôme à coefficients rationnels 
$(X-c)^{r-q+1}$, donc $b\in\Qq$. De même, en considérant 
$P_{p-1}$, on obtient $a\in\Qq$.}
\indication{Calculer $\pgcd(P,P')$.}
\end{enumerate}
}

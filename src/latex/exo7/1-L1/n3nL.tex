\uuid{n3nL}
\exo7id{7198}
\titre{exo7 7198}
\auteur{megy}
\organisation{exo7}
\datecreate{2019-07-23}
\isIndication{false}
\isCorrection{false}
\chapitre{Logique, ensemble, raisonnement}
\sousChapitre{Relation d'équivalence, relation d'ordre}

\contenu{
\texte{
(Produit de deux relations) 
Soient $\mathcal R$ et $\mathcal S$ deux relations sur $E$. Leur \emph{produit}, noté $\mathcal R \mathcal S$, est la relation binaire définie par:
\[ \forall x,y\in E, x \mathcal R \mathcal S y 
\iff \exists a\in E, \: (x \mathcal R a \text{ et } a \mathcal S y)
\]
}
\begin{enumerate}
    \item \question{Prouver par un exemple qu'en général, les relations $\mathcal R \mathcal S$ et $\mathcal S \mathcal R$ sont distinctes.}
    \item \question{Montrer que le produit de relations est néanmoins associatif, autrement dit si $\mathcal R$, $\mathcal S$ et $\mathcal T$ sont trois relations, on a 
\[ (\mathcal R \mathcal S) \mathcal T = \mathcal R (\mathcal S \mathcal T)\]}
\end{enumerate}
}

\uuid{kCw0}
\exo7id{3295}
\auteur{quercia}
\organisation{exo7}
\datecreate{2010-03-08}
\isIndication{false}
\isCorrection{true}
\chapitre{Polynôme, fraction rationnelle}
\sousChapitre{Fraction rationnelle}

\contenu{
\texte{

}
\begin{enumerate}
    \item \question{Effectuer la division suivant les puissances croissantes {\`a} un ordre
    queclonque de $1-X^2$ par $1 - 2X\cos\theta + X^2$.}
\reponse{$1-X^2 = (1-2X\cos\theta+X^2)(1 + 2X\cos\theta + \dots + 2X^n\cos n\theta)
                    + 2X^{n+1}\cos(n+1)\theta - 2X^{n+2}\cos n\theta$.}
    \item \question{En d{\'e}duire la valeur de $1 + 2\sum_{k=1}^n \cos k\theta$,
    ($\theta \not\equiv 0 (\mathrm{mod}\,{2\pi})$).}
\reponse{$=\frac{\cos n\theta - \cos(n+1)\theta}{1-\cos\theta}$.}
\end{enumerate}
}

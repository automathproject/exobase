\uuid{MOyv}
\exo7id{2905}
\auteur{quercia}
\organisation{exo7}
\datecreate{2010-03-08}
\isIndication{false}
\isCorrection{true}
\chapitre{Dénombrement}
\sousChapitre{Binôme de Newton et combinaison}

\contenu{
\texte{
Soient $n,p \in \N$. On note $\Gamma_n^p$ le nombre de $n$-uplets
$(x_1,\dots, x_n) \in \N^n$ tels que $x_1 + \dots + x_n = p$.
}
\begin{enumerate}
    \item \question{D{\'e}terminer $\Gamma_n^0$, $\Gamma_n^1$, $\Gamma_n^2$, $\Gamma_2^n$.}
    \item \question{D{\'e}montrer que $\Gamma_{n+1}^{p+1} = \Gamma_{n+1}^p + \Gamma_n^{p+1}$
    (on classera les $(n+1)$-uplets tels que $x_1 + \dots + x_{n+1} = p + 1$
    suivant que $x_1 = 0$ ou non).}
    \item \question{En d{\'e}duire que $\Gamma_n^p = C_{n+p-1}^p$.}
\reponse{
$\Gamma_n^0 = 1$, $\Gamma_n^1 = n$, $\Gamma_n^2 = \frac {n(n+1)}2$,
    $\Gamma_2^n = n+1$.
}
\end{enumerate}
}

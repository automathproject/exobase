\uuid{h6b4}
\exo7id{1016}
\auteur{legall}
\organisation{exo7}
\datecreate{1998-09-01}
\video{QdZN5i_YucI}
\isIndication{true}
\isCorrection{true}
\chapitre{Espace vectoriel}
\sousChapitre{Dimension}

\contenu{
\texte{
Montrer que tout sous-espace vectoriel d'un espace vectoriel de dimension finie
est de dimension finie.
}
\indication{On peut utiliser des familles libres.}
\reponse{
Soit $E$ un espace vectoriel de dimension $n$ et $F$ un sous-espace vectoriel de $E$.
Par l'absurde supposons que $F$ ne soit pas de dimension finie, 
alors il existe $v_1,\ldots,v_{n+1}$, $n+1$ vecteurs de $F$ lin\'eairement ind\'ependants dans $F$.
Mais ils sont aussi lin\'eairement ind\'ependants dans $E$.
Donc la dimension de $E$ est au moins $n+1$. Contradiction.

\bigskip

Deux remarques :
\begin{itemize}
  \item En fait on a m\^eme  montré que la dimension de $F$ est plus petite que la dimension de $E$.
  \item On a utilisé le r\'esultat suivant : si $E$ admet une famille libre \`a $k$ \'el\'ements alors la dimension de $E$ 
est plus grande que $k$ (ou est infini). Ce r\'esultat est une cons\'equence imm\'ediate du th\'eor\`eme de la base incompl\`ete.
\end{itemize}
}
}

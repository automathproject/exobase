\uuid{LC3n}
\exo7id{966}
\titre{exo7 966}
\auteur{legall}
\organisation{exo7}
\datecreate{1998-09-01}
\isIndication{false}
\isCorrection{false}
\chapitre{Application linéaire}
\sousChapitre{Morphismes particuliers}
\module{Algèbre}
\niveau{L1}
\difficulte{}

\contenu{
\texte{
Soient $  U   $ et $  V   $ deux ensembles non vides et $  f   $ une application de
$  U   $ \`a valeurs dans $  V  . $ Le {\em graphe} de $  f  $ est le sous-ensemble
de $  U\times V   $ d\' efini par $  \mathcal{G} _f=\{ (x, y) \in U\times V
 \hbox{ tels que } y=f(x) \}  .$
}
\begin{enumerate}
    \item \question{On suppose maintenant que $  U   $ et $  V   $ sont des espaces vectoriels. Rappeler la d\' efinition de la structure d'espace vectoriel
de $  U\times V  .$}
    \item \question{Montrer qu'une partie $  H   $ de
$  U\times V  $ est le graphe d'une application lin\' eaire de $  U  $ dans $
V  $ si et seulement si les trois conditions qui suivent
sont satisfaites :

 {\em i)} La projection canonique $  H \rightarrow U   $ d\' efinie par $  (x,y) \mapsto x   $ est surjective.

{\em ii)} $  H   $ est un sous-espace vectoriel de $  U\times V  .$

{\em iii)} $  H \cap \left( \{ 0_U \}  )\times V \right) =\{ 0_{U\times V}\} .  $ ($  0_U   $
et $  0_{U\times V}  $ sont les \' el\' ements neutres respectifs de $  U   $ et $  U\times V  .)$}
    \item \question{On identifie $  { \Rr}^4  $ \`a $  { \Rr}^2\times { \Rr}^2  $ par l'isomorphisme
$  (x,y,z,t)\mapsto \left( (x,y),(z,t)\right)   .$ Enoncer des conditions n\' ec\' essaires et suffisantes
pour que $  E  $ soit le graphe d'une application lin\' eaire de $  { \Rr}^2  $ dans lui-m\^eme.}
    \item \question{Montrer que $  E  $ est le graphe d'une application lin\' eaire $  \varphi   $
de $  { \Rr}^2  $ dans lui-m\^eme. D\' eterminer sa matrice dans une base que l'on d\' efinira au  pr\' ealabe.}
\end{enumerate}
}

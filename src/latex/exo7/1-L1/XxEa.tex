\uuid{XxEa}
\exo7id{2922}
\auteur{quercia}
\organisation{exo7}
\datecreate{2010-03-08}
\isIndication{false}
\isCorrection{false}
\chapitre{Dénombrement}
\sousChapitre{Cardinal}

\contenu{
\texte{
Soit $E$ un ensemble ordonn{\'e}. Un {\'e}l{\'e}ment $a \in E$ est dit {\it maximal\/}
s'il n'existe pas de $b \in E$ tq $b > a$.
}
\begin{enumerate}
    \item \question{Si $E$ est totalement ordonn{\'e}, montrer que :
      {\it maximal\/} $\iff$ {\it maximum.}}
    \item \question{$E = \{1,2,3,4,5,6\}$ ordonn{\'e} par la divisibilit{\'e}. Chercher les
      {\'e}l{\'e}ments maximaux.}
    \item \question{Si $E$ est fini, montrer qu'il existe un {\'e}l{\'e}ment maximal.}
    \item \question{Si $E$ est fini et n'a qu'un seul {\'e}l{\'e}ment maximal, montrer que cet {\'e}l{\'e}ment est
      maximum.}
\end{enumerate}
}

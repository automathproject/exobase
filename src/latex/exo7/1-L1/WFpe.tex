\uuid{WFpe}
\exo7id{5071}
\titre{exo7 5071}
\auteur{rouget}
\organisation{exo7}
\datecreate{2010-06-30}
\isIndication{false}
\isCorrection{true}
\chapitre{Nombres complexes}
\sousChapitre{Trigonométrie}

\contenu{
\texte{
Résoudre dans $\Rr$ l'équation $2^{4\cos^2x+1}+16.2^{4\sin^2x-3}=20$.
}
\reponse{
Soit $x\in\Rr$.

\begin{align*}
2^{4\cos^2x+1}+16.2^{4\sin^2x-3}=20&\Leftrightarrow2^{4\cos^2x+1}+16.2^{1-4\cos^2x}=20\Leftrightarrow2^{4\cos^2x}-10+16\times2^{-4\cos^2x}=0\\
 &\Leftrightarrow2^{4\cos^2x}-10+\frac{16}{2^{4\cos^2x}}=0\Leftrightarrow(2^{4\cos^2x})^2-10\times2^{4\cos^2x}+16=0\\
 &\Leftrightarrow2^{4\cos^2x}=2\;\mbox{ou}\;2^{4\cos^2x}=8\Leftrightarrow4\cos^2x=1\;\mbox{ou}\;4\cos^2x=3\\
 &\Leftrightarrow\cos x=\frac{1}{2}\;\mbox{ou}\;\cos x=-\frac{1}{2}\;\mbox{ou}\;\cos x=\frac{\sqrt{3}}{2}\;\mbox{ou}\;\cos
x=-\frac{\sqrt{3}}{2}\\
 &\Leftrightarrow x\in\left(\frac{\pi}{6}+\frac{\pi}{2}\Zz\right)\cup\left(\frac{\pi}{3}+\frac{\pi}{2}\Zz\right).
\end{align*}
}
}

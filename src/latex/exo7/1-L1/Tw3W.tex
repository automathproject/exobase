\uuid{Tw3W}
\exo7id{5322}
\titre{exo7 5322}
\auteur{rouget}
\organisation{exo7}
\datecreate{2010-07-04}
\isIndication{false}
\isCorrection{true}
\chapitre{Polynôme, fraction rationnelle}
\sousChapitre{Autre}
\module{Algèbre}
\niveau{L1}
\difficulte{}

\contenu{
\texte{
Soit $E$ la partie de $\Cc[X]$ formée des polynômes $P$ vérifiant $\forall a\in\Zz,\;P(a)\in\Zz$.
}
\begin{enumerate}
    \item \question{On pose $P_0=1$ et pour $n$ entier naturel non nul, $P_n=\frac{1}{n!}\prod_{k=1}^{n}(X+k)$ (on peut définir la notation $P_n=C_{X+n}{n}$). Montrer que $\forall n\in\Nn,\;P_n\in E$.}
\reponse{Déjà, $P_0$ est dans $E$.

Soit $n$ un naturel non nul. $P_n=\frac{1}{n!}(X+1)...(X+n)$ et donc, si $k$ est élément de $\{-1,...,-n\}$, $P_n(k)=0\in\Zz$.

Si $k$ est un entier positif, $P_n(k)=\frac{1}{n!}(k+1)...(k+n)=C_{n+k}^n\in\Zz$.

Enfin, si $k$ est un entier strictement plus petit que $-n$, 

$$P_n(k)=\frac{1}{n!}(k+1)...(k+n)=(-1)^n\frac{1}{n!}(-k-1)...(-k-n)=(-1)^nC_{-k-1}^n\in\Zz.$$

Ainsi, $\forall k\in\Zz,\;P_(k)\in\Zz$, ou encore $P_(\Zz)\subset\Zz$.}
    \item \question{Montrer que toute combinaison linéaire à coefficients entiers relatifs des $P_n$ est encore un élément de $E$.}
\reponse{Evident}
    \item \question{Montrer que $E$ est l'ensemble des combinaisons linéaires à coefficients entiers relatifs des $P_n$.}
\reponse{Soit $P\in\Cc[X]\setminus\{0\}$ tel que $\forall k\in\Zz,\;P(k)\in\Zz$ (si $P$ est nul, $P$ est combinaison linéaire à coefficients entiers des $P_k$).

Puisque $\forall k\in\Nn,\;\mbox{deg}(P_k)=k$, on sait que pour tout entier naturel $n$, $(P_k)_{0\leq k\leq n}$ est une base de $\Cc_n[X]$ et donc, $(P_k)_{k\in\Nn}$ est une base de $\Cc[X]$ (tout polynôme non nul ayant un degré $n$, s'écrit donc de manière unique comme combinaison linéaire des $P_k$).

Soit $n=\mbox{deg}P$.

Il existe $n+1$ nombres complexes $a_0$,..., $a_n$ tels que $P=a_0P_0+...+a_nP_n$. Il reste à montrer que les $a_i$ sont des entiers relatifs.

L'égalité $P(-1)$ est dans $\Zz$, fournit~:~$a_0$ est dans $\Zz$.

L'égalité $P(-2)$ est dans $\Zz$, fournit~:~$a_0-a_1$ est dans $\Zz$ et donc $a_1$ est dans $\Zz$.

L'égalité $P(-3)$ est dans $\Zz$, fournit~:~$a_0-2a_1+a_2$ est dans $\Zz$ et donc $a_2$ est dans $\Zz$...

L'égalité $P(-(k+1))$ est dans $\Zz$, fournit~:~$a_0-a_1+...+(-1)^ka_k$ est dans $\Zz$ et si par hypothèse de récurrence, $a_0$,..., $a_{k-1}$ sont des entiers relatifs alors $a_k$ l'est encore.

Tous les coefficients $a_k$ sont des entiers relatifs et $E$ est donc constitué des combinaisons linéaires à coefficients entiers relatifs des $P_k$.}
\end{enumerate}
}

\uuid{HBeF}
\exo7id{214}
\titre{exo7 214}
\auteur{bodin}
\organisation{exo7}
\datecreate{1998-09-01}
\isIndication{false}
\isCorrection{false}
\chapitre{Logique, ensemble, raisonnement}
\sousChapitre{Relation d'équivalence, relation d'ordre}
\module{Algèbre}
\niveau{L1}
\difficulte{}

\contenu{
\texte{
\'Etudier les propri\'et\'es des relations suivantes. Dans le cas
d'une relation d'\'equivalence, pr\'eciser les classes ; dans le cas
d'une relation d'ordre, pr\'eciser si elle est totale, si l'ensemble admet
un plus petit ou plus grand \'el\'ement.
}
\begin{enumerate}
    \item \question{Dans $\mathcal{P}(E)$ : $A\mathcal{R}_1 B \Leftrightarrow A\subset B\quad ;
\quad A\mathcal{R}_2 B \Leftrightarrow A\cap B=\emptyset$.}
    \item \question{Dans $\Zz$ : $a\mathcal{R}_3 b \Leftrightarrow a$ et $b$ ont la m\^eme parit\'e $\quad;
\quad a\mathcal{R}_4 b \Leftrightarrow \exists n\in \Nn \ \,a-b=3n \quad ;
\quad a\mathcal{R}_5 b \Leftrightarrow a-b$ est divisible par $3$.}
\end{enumerate}
}

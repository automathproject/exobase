\uuid{CyXG}
\exo7id{184}
\titre{exo7 184}
\auteur{ridde}
\organisation{exo7}
\datecreate{1999-11-01}
\isIndication{false}
\isCorrection{false}
\chapitre{Logique, ensemble, raisonnement}
\sousChapitre{Autre}
\module{Algèbre}
\niveau{L1}
\difficulte{}

\contenu{
\texte{
Pour calculer des sommes portant sur deux indices, on a int\'erêt
\`a repr\'esenter la z{o}ne du plan couverte par ces indices et
\`a sommer en lignes, colonnes ou diagonales... Calculer :
}
\begin{enumerate}
    \item \question{$\sum\limits_{1\leq i \leq j \leq n}ij$.}
    \item \question{$\sum\limits_{1\leq i < j \leq n}i (j-1)$.}
    \item \question{$\sum\limits_{1\leq i < j \leq n} (i-1)j$.}
    \item \question{$\sum\limits_{1\leq i \leq j \leq n} (n-i) (n-j)$.}
    \item \question{$\sum\limits_{1\leq p, q \leq n} (p + q)^2$ (on posera $k = p + q$).}
\end{enumerate}
}

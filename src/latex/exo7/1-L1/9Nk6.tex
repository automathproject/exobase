\uuid{9Nk6}
\exo7id{153}
\auteur{bodin}
\organisation{exo7}
\datecreate{1998-09-01}
\video{SUMAdDoMPto}
\isIndication{false}
\isCorrection{true}
\chapitre{Logique, ensemble, raisonnement}
\sousChapitre{Récurrence}

\contenu{
\texte{
\label{ex104}
Montrer :
}
\begin{enumerate}
    \item \question{$\displaystyle{\sum_{k=1}^{n}k=\frac{n(n+1)}{2} \quad \forall n \in \Nn^{*} \,.}$}
    \item \question{$\displaystyle{\sum_{k=1}^{n}k^2=\frac{n(n+1)(2n+1)}{6} \quad \forall n \in \Nn^{*} \,.}$}
\reponse{
R\'edigeons la deuxi\`eme \'egalit\'e.
Soit $\mathcal{A}_n$, $n\in \Nn^*$ l'assertion suivante:
$$(\mathcal{A}_n) \ \ \ \sum_{k=1}^n k^2= \frac{n(n+1)(2n+1)}{6}.$$
\begin{itemize}
  \item $\mathcal{A}_0$ est vraie ($1=1$).
  \item \'Etant donn\'e $n\in\Nn^*$ supposons que $\mathcal{A}_n$
soit vraie. Alors
\begin{align*}
\sum_{k=1}^{n+1} k^2
  &= \sum_{k=1}^n k^2 \ \ + (n+1)^2 \\
  &= \frac{n(n+1)(2n+1)}{6} + (n+1)^2\\
  &= \frac{n(n+1)(2n+1)+6(n+1)^2}{6} \\
  &= \frac{(n+1)(n(2n+1)+6(n+1))}{6} \\
  &= \frac{(n+1)(n+2)(2(n+1)+1)}{6} \\
\end{align*}
Ce qui prouve $\mathcal{A}_{n+1}$.
  \item Par le principe de r\'ecurrence nous venons de montrer 
que $\mathcal{A}_n$ est vraie pour tout $n\in\Nn^*$.
\end{itemize}
}
\end{enumerate}
}

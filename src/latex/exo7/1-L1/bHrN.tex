\uuid{bHrN}
\exo7id{908}
\titre{exo7 908}
\auteur{legall}
\organisation{exo7}
\datecreate{1998-09-01}
\video{SmI00EAW31k}
\isIndication{true}
\isCorrection{true}
\chapitre{Espace vectoriel}
\sousChapitre{Système de vecteurs}

\contenu{
\texte{
Soit $E$ le 
sous-espace vectoriel de ${\Rr}^3$ engendr\'e par
les vecteurs $v_1=(2, 3, -1)$ et $v_2=(1, -1, -2)$  et $F$ celui engendré par
 $w_1=(3, 7, 0)$ et $w_2=(5, 0, -7)$. Montrer que  $E$  et  $F$  sont
 \'egaux.
}
\indication{Montrer la double inclusion. 
Utiliser le fait que de mani\`ere g\'en\'erale pour  $E=\mathrm{Vect}(v_1,\ldots,v_n)$
alors : 
$$E\subset F \iff \forall i=1,\ldots,n \quad v_i\in F.$$}
\reponse{
Montrons d'abord que $E \subset F$.
On va d'abord montrer que $v_1 \in F$ et $v_2 \in F$.

Tout d'abord 
$v_1\in F \iff v_1 \in \text{Vect}\{w_1,w_2\} \iff \exists \lambda, \mu \quad v_1 = \lambda w_1+\mu w_2 $.

Il s'agit donc de trouver ces $\lambda,\mu$. Cela se fait en résolvant un système
(ici on peut même le faire de tête) on trouve la relation 
$7 (2, 3, -1) = 3(3, 7, 0)-(5, 0, -7)$
ce qui donne la relation $v_1 = \frac37 w_1 -\frac 17 w_2$ et donc $v_1\in F$.

De même $7v_2=-w_1+2w_2$ donc $v_2\in F$.

Maintenant $v_1$ et $v_2$ sont dans l'espace vectoriel $F$, donc toute combinaison linéaire
de $v_1$ et $v_2$ aussi, c'est-à-dire : pour tout $\lambda,\mu$, on a $\lambda v_1+\mu v_2 \in F$.
Ce qui implique $E \subset F$.


\bigskip

Il reste à montrer $F\subset E$. Il s'agit donc d'écrire $w_1$ (puis $w_2$) en fonction de $v_1$ et $v_2$.
On trouve $w_1= 2v_1-v_2$ et $w_2=v_1+3v_2$. Encore une fois cela entraîne $w_1 \in E$ et $w_2 \in E$ donc
$\text{Vect}\{w_1,w_2\} \subset E$ d'où $F\subset E$.

Par double inclusion on a montré $E=F$.
}
}

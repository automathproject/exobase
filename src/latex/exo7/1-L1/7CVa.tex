\uuid{7CVa}
\exo7id{5260}
\titre{exo7 5260}
\auteur{rouget}
\organisation{exo7}
\datecreate{2010-07-04}
\isIndication{false}
\isCorrection{true}
\chapitre{Matrice}
\sousChapitre{Noyau, image}
\module{Algèbre}
\niveau{L1}
\difficulte{}

\contenu{
\texte{
Soit $\begin{array}[t]{cccc}
f~:&\Rr_n[X]&\rightarrow&\Rr_{n+1}[X]\\
 &P&\mapsto&Q=e^{X^2}(Pe^{-X^2})'
\end{array}$.
}
\begin{enumerate}
    \item \question{Vérifier que $f\in(\mathcal{L}(\Rr_n[X],\Rr_{n+1}[X])$.}
\reponse{Pour $P$ élément de $\Rr_n[X]$, 

$$f(P)=e^{X^2}(Pe^{-X^2})'=e^{X^2}(P'e^{-X^2}-2XPe^{-X^2})=P'-2XP.$$

Ainsi, si $P$ est un polynôme de degré infèrieur ou égal à $n$, $f(P)=P'-2XP$ est un polynôme de degré inférieur ou égal à $n+1$, et $f$ est bien une application de $\Rr_n[X]$ dans $\Rr_{n+1}[X]$.

De plus, pour $(\lambda,\mu)\in\Rr^2$ et $(P,Q)\in\Rr_n[X]$, on a~:

$$f(\lambda P+\mu Q)=(\lambda P+\mu Q)'-2X(\lambda P+\mu Q)=\lambda(P'-2XP)+\mu(Q'-2XQ)=\lambda f(P)+\mu f(Q).$$

$f$ est élément de $\mathcal{L}(\Rr_n[X],\Rr_{n+1}[X])$.}
    \item \question{Déterminer la matrice de $f$ relativement aux bases canoniques de $\Rr_n[X]$ et $\Rr_{n+1}[X]$.}
\reponse{La matrice $A$ cherchée est élément de $\mathcal{M}_{n+1,n}(\Rr)$.

Pour $k=0$, $f(X^k)=f(1)=-2X$ et pour $1\leq k\leq n$, $f(X^k)=kX^{k-1}-2X^{k+1}$. On a donc~:

$$A=
\left(
\begin{array}{cccccc}
0&1&0&\ldots&\ldots&0\\
-2&0&2&0& &\vdots\\
0&-2&0&\ddots&\ddots&\vdots\\
\vdots&\ddots&\ddots&\ddots&\ddots&0\\
 & & &\ddots&\ddots&n\\
\vdots& & &\ddots&-2&0\\
0&\ldots& &\ldots&0&-2
\end{array}
\right).$$}
    \item \question{Déterminer $\mbox{Ker}f$ et $\mbox{rg}f$.}
\reponse{Soit $P\in\Rr_n[X]$ tel que $f(P)=0$.

Si $P$ n'est pas nul, $-2XP$ a un degré strictement plus grand que $P'$ et donc $f(P)$ n'est pas nul. Par suite, $\mbox{Ker}f=\{0\}$ ($f$ est donc injective) et d'après le théorème du rang, $\mbox{rg}f=\mbox{dim}(\Rr_n[X])-0=n+1$, ce qui montre que $\mbox{Im}f$ n'est pas $\Rr_{n+1}[X]$ ($f$ n'est pas surjective).}
\end{enumerate}
}

\uuid{T5sn}
\exo7id{5580}
\titre{exo7 5580}
\auteur{rouget}
\organisation{exo7}
\datecreate{2010-10-16}
\isIndication{false}
\isCorrection{true}
\chapitre{Espace vectoriel}
\sousChapitre{Dimension}
\module{Algèbre}
\niveau{L1}
\difficulte{}

\contenu{
\texte{
Soient $E$ et $F$ deux espaces vectoriels de dimension finie et soient $f$ et $g$ deux applications linéaires de $E$ dans $F$. Montrer que $|\text{rg}f-\text{rg}g|\leqslant\text{rg}(f+g)\leqslant\text{rg}f+\text{rg}g$.
}
\reponse{
$\text{Im}(f+g)=\{f(x)+g(x),\;x\in E\}\subset\{f(x)+g(x'),\;(x,x')\in E^2\}=\text{Im}f+\text{Im}g$. Donc 

\begin{center}
$\text{rg}(f+g)\leqslant\text{dim}(\text{Im}f+\text{Im}g)\leqslant\text{rg}f+\text{rg}g$
\end{center}

puis $\text{rg}f=\text{rg}((f+g)+(-g))\leqslant\text{rg}(f+g)+\text{rg}(-g)=\text{rg}(f+g)+\text{rg}g$ (car $\text{Im}(-g)=\{-g(x),\;x\in E\}=\{g(-x),\;x\in E\}=\{g(x'),\;x'\in E\}=\text{Im}g$) et donc $\text{rg}(f+g)\geqslant\text{rg}f-\text{rg}g$. De même , en échangeant les rôles de $f$ et $g$, $\text{rg}(f+g)\geqslant\text{rg}g-\text{rg}f$ et finalement $\text{rg}(f+g)\geqslant|\text{rg}f -\text{rg}g|$.

\begin{center}
\shadowbox{
$\forall(f,g)\in\mathcal{L}(E,F)^2,\;|\text{rg}f -\text{rg}g|\leqslant\text{rg}(f+g)\leqslant\text{rg}f+\text{rg}g$.
}
\end{center}
}
}

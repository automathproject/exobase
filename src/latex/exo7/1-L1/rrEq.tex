\uuid{rrEq}
\exo7id{5105}
\titre{exo7 5105}
\auteur{rouget}
\organisation{exo7}
\datecreate{2010-06-30}
\isIndication{false}
\isCorrection{true}
\chapitre{Logique, ensemble, raisonnement}
\sousChapitre{Logique}
\module{Algèbre}
\niveau{L1}
\difficulte{}

\contenu{
\texte{
Les phrases suivantes sont-elles équivalentes~?
}
\begin{enumerate}
    \item \question{\og~$\forall x\in\Rr,\;(f(x)=0\;\mbox{et}\;g(x)=0)$~\fg~et \og~$(\forall
x\in\Rr,\;f(x)=0)\;\mbox{et}\;(\forall x\in\Rr,\;g(x)=0)$~\fg.}
\reponse{Oui. Dans les deux cas, chaque fois que l'on se donne un réel $x_0$, $f(x_0)$ et $g(x_0)$ sont tous deux nuls.}
    \item \question{\og~$\forall x\in\Rr,\;(f(x)=0\;\mbox{ou}\;g(x)=0)$~\fg~et \og~$(\forall
x\in\Rr,\;f(x)=0)\;\mbox{ou}\;(\forall x\in\Rr,\;g(x)=0)$~\fg.}
\reponse{Non. La deuxième affirmation implique la première mais la première n'implique pas la deuxième. La première
phrase est la traduction avec des quantificateurs de l'égalité $fg=0$. La deuxième phrase est la traduction avec
quantificateurs de $(f=0\;\mbox{ou}\;g=0)$.
Voici un exemple de fonctions $f$ et $g$ toutes deux non nulles dont le produit est nul.
Soient $\begin{array}[t]{cccc}
f~:&\Rr&\rightarrow&\Rr\\
 &x&\mapsto&\left\{
\begin{array}{l}
0\;\mbox{si}\;x<0\\
x\;\mbox{si}\;x\geq0
\end{array}
\right.
\end{array}$ et $\begin{array}[t]{cccc}
g~:&\Rr&\rightarrow&\Rr\\
 &x&\mapsto&\left\{
\begin{array}{l}
0\;\mbox{si}\;x>0\\
x\;\mbox{si}\;x\leq0
\end{array}
\right.
\end{array}$. Pour chaque valeur de $x$, on a soit $f(x)=0$ (quand $x\leq0$), soit $g(x)=0$ (quand $x\geq0$). On a
donc~:~$\forall x\in\Rr,\;(f(x)=0\;\mbox{ou}\;g(x)=0)$ ou encore $\forall x\in\Rr,\;f(x)g(x)=0$ ou enfin, $fg=0$.
Cependant, $f(1)=1\neq0$ et donc $f\neq0$, et $g(-1)=-1\neq0$ et donc $g\neq0$. Ainsi, on n'a pas $(f=0\;\mbox{ou}\;g=0)$ ou
encore, on n'a pas $((\forall x\in\Rr,\;f(x)=0)\;\mbox{ou}\;(\forall x\in\Rr,\;g(x)=0))$.}
\end{enumerate}
}

\uuid{rQk9}
\exo7id{5349}
\titre{exo7 5349}
\auteur{rouget}
\organisation{exo7}
\datecreate{2010-07-04}
\isIndication{false}
\isCorrection{true}
\chapitre{Polynôme, fraction rationnelle}
\sousChapitre{Racine, décomposition en facteurs irréductibles}

\contenu{
\texte{
Déterminer $\lambda$ et $\mu$ complexes tels que les zéros de $z^4-4z^3-36z^2+\lambda z+\mu$ soient en progression arithmétique. Résoudre alors l'équation.
}
\reponse{
Posons $P=X^4-4X^3-36X^2+\lambda X+\mu$.
 
\begin{align*}\ensuremath
(\lambda,\mu)\;\mbox{solution}&\Leftrightarrow\exists(z,r)\in\Cc^2/\;\mbox{les racines de}\;P\;\mbox{soient}\;z,z+r,z+2r,z+3r\\
  &\Leftrightarrow\exists(z,r)\in\Cc^2/\;\left\{
 \begin{array}{l}
 \sigma_1=4\\
 \sigma_2=-36\\
 \sigma_3=-\lambda\\
 \sigma_4=\mu
 \end{array}
 \right.
 \\
 &\Leftrightarrow\exists(z,r)\in\Cc^2/\;\left\{
 \begin{array}{l}
 4z+6r=4\\
 z(3z+6r)+(z+r)(2z+5r)+(z+2r)(z+3r)=-36\\
 \sigma_3=-\lambda\\
 \sigma_4=\mu
 \end{array}
 \right.
 \\
 &\Leftrightarrow\exists(z,r)\in\Cc^2/\;\left\{
 \begin{array}{l}
 2z+3r=2\\
 6z^2+18rz+11r^2=-36\\
 \sigma_3=-\lambda\\
 \sigma_4=\mu
 \end{array}
 \right.
 \\
 &\Leftrightarrow\exists(z,r)\in\Cc^2/\;\left\{
 \begin{array}{l}
 z=1-\frac{3}{2}r\\
 6(1-\frac{3}{2}r)^2+18(1-\frac{3}{2}r)r+11r^2=-36\\
 \sigma_3=-\lambda\\
 \sigma_4=\mu
 \end{array}
 \right.
 \\
 &\Leftrightarrow\exists(z,r)\in\Cc^2/\;\left\{
 \begin{array}{l}
 -\frac{5}{2}r^2+42=0\\
 z=1-\frac{3}{2}r\\
 \sigma_3=-\lambda\\
 \sigma_4=\mu
 \end{array}
 \right.
\end{align*}

D'où la solution (les deux valeurs opposées de $r$ fournissent évidemment la même progression arithmétique)
$r=2\sqrt{\frac{21}{5}}$ puis $z=1-3\sqrt{\frac{21}{5}}$ puis les racines $z_1=1-3\sqrt{\frac{21}{5}}$, $z_2=1-\sqrt{\frac{21}{5}}$, $z_3=1+\sqrt{\frac{21}{5}}$ et $z_4=1+3\sqrt{\frac{21}{5}}$, obtenues pour 

$$\lambda=z_1z_2z_3z_4=(1-3\sqrt{\frac{21}{5}})(1-\sqrt{\frac{21}{5}})(1+\sqrt{\frac{21}{5}})
(1+3\sqrt{\frac{21}{5}})= (1-9\frac{21}{5})(1-\frac{21}{5})=\frac{2994}{25},$$
 
et 

\begin{align*}\ensuremath
\mu&=(1-3\sqrt{\frac{21}{5}})(1-\frac{21}{5})+(1-9\frac{21}{5})(1-\sqrt{\frac{21}{5}})+(1-9\frac{21}{5})(1+\sqrt{\frac{21}{5}})+(1-\frac{21}{5})(1+3\sqrt{\frac{21}{5}})\\
 &=2(1-\frac{21}{5})+2(1-9\frac{21}{5})=2(2-10\frac{21}{5})=-80
\end{align*}
}
}

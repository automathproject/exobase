\uuid{R0dj}
\exo7id{5570}
\auteur{rouget}
\organisation{exo7}
\datecreate{2010-10-16}
\isIndication{false}
\isCorrection{true}
\chapitre{Espace vectoriel}
\sousChapitre{Système de vecteurs}

\contenu{
\texte{
On pose $f_a(x)=e^{ax}$ pour $a$ et $x$ réels. Etudier la liberté de la famille de fonctions $(f_a)_{a\in\Rr}$.
}
\reponse{
Soient $a_1< ... <a_n$ $n$ réels deux à deux distincts et $\lambda_1$,..., $\lambda_n$ n réels tels que $\sum_{i=1}^{n}\lambda_if_{a_i}=0$ $(*)$.

\textbf{Première solution.} Après multiplication des deux
 membres de $(*)$ par $e^{-a_nx}$ puis passage à la limite quand $x$ tend vers $+\infty$, on obtient $\lambda_n=0$. En réitérant, on obtient donc $\lambda_n=\lambda_{n-1}=...=\lambda_1=0$.
 

\textbf{Deuxième solution.} On note $f$ la fonction apparaissant au premier membre de $(*)$.

\begin{align*}\ensuremath
f=0&\Rightarrow \forall k\in\llbracket0,,n-1\rrbracket,\;f^{(k)}(0)=0\\
 &\Rightarrow\forall k\in\llbracket0,n-1\rrbracket,\;\lambda_1a_1^k+...+\lambda_na_n^k=0.
\end{align*}

Le système prédédent d' inconnues $\lambda_i$, $1\leqslant n$, est un système linéaire homogène à $n$ équations et $n$ inconnues. Son déterminant est le déterminant de \text{Vandermonde} des $a_i$ et est non nul puisque les $a_i$ sont deux à deux distincts. Le système est donc de \textsc{Cramer} et admet l'unique solution $(0,...,0)$.

\textbf{Troisième solution.} ( dans le cas où on se restreint à démontrer la liberté de la famille $(x\mapsto e^{nx})_{n\in\Nn}$).

Soient $n_1< ... < n_p$ $p$ entiers naturels deux à deux distincts.
Supposons que pour tout réel $x$ on ait $\sum_{i=1}^{n}\lambda_ie^{n_ix}=0$. On en déduit que pour tout réel strictement positif $t$, on a $\sum_{i=1}^{n}\lambda_it^{n_i}=0$ et donc le polynôme  $\sum_{i=1}^{n}\lambda_iX^{n_i}$ est nul (car a une infinité de racines) ou encore les coefficients du polynôme $\sum_{i=1}^{n}\lambda_iX^{n_i}$ à savoir les $\lambda_i$ sont tous nuls.

\textbf{Quatrième solution.} (pour les redoublants)
L'application $\varphi$ qui à $f$ de classe $C^\infty$ fait correspondre sa dérivée est un endomorphisme de l'espace des fonctions de classe $C^\infty$ sur $\Rr$ à valeurs dans $\Rr$.
Pour $a$ réel donné, $\varphi(f_a)=af_a$ et la famille $(f_a)_{a\in\Rr}$ est constituée de vecteurs propres de $\varphi$ (les $f_a$ sont non nulles) associés à des valeurs propres deux à deux distinctes. On sait qu'une telle famille est libre.
}
}

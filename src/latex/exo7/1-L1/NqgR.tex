\uuid{NqgR}
\exo7id{1049}
\titre{exo7 1049}
\auteur{liousse}
\organisation{exo7}
\datecreate{2003-10-01}
\isIndication{false}
\isCorrection{false}
\chapitre{Matrice}
\sousChapitre{Propriétés élémentaires, généralités}

\contenu{
\texte{
Soit $A$ une matrice carr\'{e}e d'ordre $n$ ; on suppose que $A^{2}$ est
une combinaison lin\'{e}aire de $A$ et $I_{n}$ : $A^{2}=\alpha A+\beta I_{n}.$
}
\begin{enumerate}
    \item \question{Montrer que $A^{n}$ est \'{e}galement une combinaison lin\'{e}aire de 
$A$ et $I_{n}$ pour tout $n\in \Nn^{\ast }.$\\}
    \item \question{Montrer que si $\beta $ est non nul, alors $A$ est inversible et que $
A^{-1}$ est encore combinaison lin\'{e}aire de $A$ et $I_{n}.$\\}
    \item \question{Application 1 : soit $A=J_{n}-I_{n},$ o\`{u} $J_{n}$ est la matrice
Attila (envahie par les uns...), avec $n\geq 1$. Montrer que $
A^{2}=\left( n-2\right) A+\left( n-1\right) I_{n}$ ; en d\'{e}duire que $A$
est inversible, et d\'{e}terminer son inverse.\\}
    \item \question{Application 2 : montrer que si $n=2$, $A^{2}$ est toujours une
combinaison lin\'{e}aire de $A$ et $I_{2},$ et retrouver la formule donnant $
A^{-1}$ en utilisant 2.}
\end{enumerate}
}

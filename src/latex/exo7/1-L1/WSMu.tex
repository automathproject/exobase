\uuid{WSMu}
\exo7id{5317}
\titre{exo7 5317}
\auteur{rouget}
\organisation{exo7}
\datecreate{2010-07-04}
\isIndication{false}
\isCorrection{true}
\chapitre{Polynôme, fraction rationnelle}
\sousChapitre{Pgcd}
\module{Algèbre}
\niveau{L1}
\difficulte{}

\contenu{
\texte{
Déterminer le PGCD de $X^6-7X^4+8X^3-7X+7$ et $3X^5-7X^3+3X^2-7$.
}
\reponse{
$X^6-7X^4+8X^3-7X+7=(X^6+8X^3+7)-(7X^4+7X)=(X^3+1)(X^3+7)-7X(X^3+1)=(X^3+1)(X^3-7X+7)$ et $3X^5-7X^3+3X^2-7=3X^2(X^3+1)-7(X^3+1)=(X^3+1)(3X^2-7)$. Donc,

$$(X^6-7X^4+8X^3-7X+7)\wedge(3X^5-7X^3+3X^2-7)=(X^3+1)((X^3-7X+7)\wedge(3X^2-7)).$$

Maintenant, pour $\varepsilon\in\{-1,1\}$, $(\varepsilon\sqrt{\frac{7}{3}})^3-7(\varepsilon\sqrt{\frac{7}{3}})+7=-(\varepsilon\frac{14}{3}\sqrt{\frac{7}{3}})+7\neq0$.

Les polynômes $(X^3-7X+7)$ et $(3X^2-7)$ n'ont pas de racines communes dans $\Cc$ et sont donc premiers entre eux. Donc, $(X^6-7X^4+8X^3-7X+7)\wedge(3X^5-7X^3+3X^2-7)=X^3+1$.
}
}

\uuid{Z8Y5}
\exo7id{3345}
\titre{exo7 3345}
\auteur{quercia}
\organisation{exo7}
\datecreate{2010-03-09}
\isIndication{false}
\isCorrection{false}
\chapitre{Application linéaire}
\sousChapitre{Morphismes particuliers}
\module{Algèbre}
\niveau{L1}
\difficulte{}

\contenu{
\texte{
Soit $E$ un $ K$-espace vectoriel de dimension finie et $f\in \mathcal{L}(E)$ nilpotente
d'indice~$n$.

Soit $\phi : {\mathcal{L}(E)} \to {\mathcal{L}(E)}, g \mapsto{f\circ g - g\circ f.}$
}
\begin{enumerate}
    \item \question{Montrer que $\phi^p(g) = \sum_{k=0}^p (-1)^kC_p^k f^{p-k}\circ g \circ f^k$.
    En déduire que $\phi$ est nilpotente.}
    \item \question{Soit $a\in \mathcal{L}(E)$. Montrer qu'il existe $b\in \mathcal{L}(E)$ tel que
    $a\circ b \circ a = a$. En déduire l'indice de nilpotence de~$\phi$.}
\end{enumerate}
}

\uuid{4DLm}
\exo7id{2959}
\auteur{quercia}
\organisation{exo7}
\datecreate{2010-03-08}
\isIndication{false}
\isCorrection{true}
\chapitre{Nombres complexes}
\sousChapitre{Trigonométrie}

\contenu{
\texte{
Soit $z \in \mathbb{U}$. Peut-on trouver $a \in \R$ tel que $z = \frac {1+ia}{1-ia}$ ?
}
\reponse{
$z = e^{i\theta}  \Rightarrow  a = \tan\frac\theta2$
         pour $\theta \not \equiv \pi (\mathrm{mod}\,{2\pi})$.
}
}

\uuid{iv9i}
\exo7id{3187}
\titre{exo7 3187}
\auteur{quercia}
\organisation{exo7}
\datecreate{2010-03-08}
\isIndication{false}
\isCorrection{true}
\chapitre{Polynôme, fraction rationnelle}
\sousChapitre{Autre}
\module{Algèbre}
\niveau{L1}
\difficulte{}

\contenu{
\texte{

}
\begin{enumerate}
    \item \question{Donner un isomorphisme $f$ en\-tre $\C^{n+1}$ et $\C_n[X]$.}
    \item \question{Montrer que $\sigma:{\C^{n+1}}\to{\C^{n+1}},
    {(a_0,\dots,a_n)}\mapsto{(a_n,a_0,\dots,a_{n-1})}$ est lin{\'e}aire.}
    \item \question{Si $(P,Q)\in(\C[X])^2$, on d{\'e}finit le produit $\overline{PQ}$
    comme le reste de la division euclidienne de $PQ$ par $X^{n+1}-1$.
    Montrer que l'application induite par~$\sigma$ sur~$\C_n[X]$
    (c'est-{\`a}-dire $f\circ \sigma\circ f^{-1}$) est l'application
    qui {\`a}~$P$ associe $\overline{XP}$.}
    \item \question{Soit~$F$ un sous-espace de~$\C^{n+1}$ stable par~$\sigma$.

    Montrer qu'il existe un polyn{\^o}me $Q$ tel que
    $f(F) = \{\overline{RQ},\ R\in\C_n[X]\}$.}
\reponse{
Trivialement vrai ou trivialement faux selon le choix qu'on
    a fait en {\bf 1}.
Soit $Q\in \C[X]$ et $F_Q = \{\overline{RQ},\ R\in\C_n[X]\}$.
    On a $F_Q = \{\overline{RQ},\ R\in\C[X]\}$ de mani{\`e}re {\'e}vidente, donc
    $F_Q$ est stable par la multiplication modulaire par~$X$.
    
    Soit r{\'e}ciproquement $F$ un sev de $\C_n[X]$ stable par la multiplication
    modulaire par $X$. Si $(P_1,\dots,P_k)$ est une famille g{\'e}n{\'e}ratrice de~$F$
    alors $Q = \mathrm{pgcd}(P_1,\dots,P_k)\in F$ d'apr{\`e}s la relation de
    B{\'e}zout et la stabilit{\'e} de~$F$ donc $F_Q\subset F$ et $P_i\in F_Q$ puisque
    $Q$ divise $P_i$ d'o{\`u} $F\subset F_Q$ et $F=F_Q$.
}
\end{enumerate}
}

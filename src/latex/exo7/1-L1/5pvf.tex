\uuid{5pvf}
\exo7id{320}
\titre{exo7 320}
\auteur{cousquer}
\organisation{exo7}
\datecreate{2003-10-01}
\isIndication{false}
\isCorrection{false}
\chapitre{Arithmétique dans Z}
\sousChapitre{Pgcd, ppcm, algorithme d'Euclide}

\contenu{
\texte{
En divisant un nombre par $8$, un élève  a obtenu $4$ pour reste~; en divisant
ce même  nombre par $12$, il a obtenu $3$ pour reste. Qu'en pensez-vous~?

 Le fort en calcul de la classe, qui ne fait jamais d'erreur, a divisé le
millésime de l'année par $29$, il a trouvé $25$ pour reste~; il a divisé le
même millésime par $69$, il a trouvé $7$ pour reste. En quelle année cela
se passait-il~?
}
}

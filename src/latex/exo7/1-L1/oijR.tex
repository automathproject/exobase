\uuid{oijR}
\exo7id{5190}
\titre{exo7 5190}
\auteur{rouget}
\organisation{exo7}
\datecreate{2010-06-30}
\isIndication{false}
\isCorrection{true}
\chapitre{Application linéaire}
\sousChapitre{Image et noyau, théorème du rang}

\contenu{
\texte{
Soient $\Kk$ un sous-corps de $\Cc$ et $E$ et $F$ deux $\Kk$-espaces vectoriels de dimensions finies sur $\Kk$ et $u$ et $v$
deux applications linéaires de $E$ dans $F$. Montrer que ~:~
$|\mbox{rg}u-\mbox{rg}v|\leq\mbox{rg}(u+v)\leq\mbox{rg}u+\mbox{rg}v$.
}
\reponse{
Par définition, $\mbox{rg }(u+v)=\mbox{dim }\left(\mbox{Im }(u+v)\right)$.

\begin{center}
$\mbox{Im }(u+v)=\{u(x)+v(x),\;x\in E\}\subset\{u(x)+v(y),\;(x,y)\in E^2\}=\mbox{Im }u+\mbox{Im }v$.
\end{center}
Donc,

\begin{align*}
rg(u+v)&=\mbox{dim }\left(\mbox{Im }(u+v)\right)\\
 &\leq\mbox{dim }(\mbox{Im }u+\mbox{Im }v)=\mbox{dim }(\mbox{Im }u)+\mbox{dim }(\mbox{Im }v)-\mbox{dim }(\mbox{Im }u\cap\mbox{
Im}v)\\
 &\leq\mbox{dim }(\mbox{Im }u)+\mbox{dim }(\mbox{Im }v)=\mbox{rg }u+\mbox{rg }v.
\end{align*}
On a montré que~:

\begin{center}
\shadowbox{
$\forall(u,v)\in(\mathcal{L}(E,F))^2,\;\mbox{rg }(u+v)\leq\mbox{rg }u+\mbox{rg }v.$
}
\end{center}
Ensuite, $$\mbox{rg }u=\mbox{rg }(u+v-v)\leq\mbox{rg }(u+v)+\mbox{rg }(-v)=\mbox{rg }(u+v)+\mbox{rg }v,$$
(il est clair que $\mbox{Im }(-v)=\mbox{Im }v)$ et donc $\mbox{rg }u-\mbox{rg }v\leq\mbox{rg }(u+v)$. En échangeant les rôles de
$u$ et $v$, on a aussi $\mbox{rg }v-\mbox{rg }u=\mbox{rg }(u+v)$ et finalement

\begin{center}
\shadowbox{
$\forall(u,v)\in(\mathcal{L}(E,F))^2,\;|\mbox{rg }u-\mbox{rg }v|\leq\mbox{rg }(u+v).$
}
\end{center}
}
}

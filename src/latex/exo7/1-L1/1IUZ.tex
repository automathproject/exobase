\uuid{1IUZ}
\exo7id{2951}
\titre{exo7 2951}
\auteur{quercia}
\organisation{exo7}
\datecreate{2010-03-08}
\isIndication{false}
\isCorrection{true}
\chapitre{Nombres complexes}
\sousChapitre{Trigonométrie}

\contenu{
\texte{
A l'aide de formules du bin{\^o}me, simplifier :
}
\begin{enumerate}
    \item \question{$\sum_{k=0}^{[n/3]}\,C_n^{3k}$.}
\reponse{$\frac {2^n+2\cos(n\pi/3)}3$.}
    \item \question{$\sum_{k=0}^{[n/2]}\,C_n^{2k}(-3)^k$.}
\reponse{$2^n\cos(n\pi/3)$.}
    \item \question{$\sum_{k=0}^n\,C_n^k\cos(k\theta)$.}
\reponse{$\left(2\cos\frac\theta2\right)^n\cos\frac{n\theta}2$.}
    \item \question{$\sum_{k=0}^n\,C_n^k\sin\bigl((k+1)\theta\bigr)$.}
\reponse{$\left(2\cos\frac\theta2\right)^n\sin\frac{(n+2)\theta}2$.}
    \item \question{$\cos a + C_n^1\cos(a+b) + C_n^2\cos(a+2b) + \dots + C_n^n\cos(a+nb)$.}
\reponse{$\left(2\cos\frac b2\right)^n\cos\left(a+\frac {nb}2\right)$.}
\end{enumerate}
}

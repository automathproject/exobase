\uuid{WkWR}
\exo7id{5284}
\titre{exo7 5284}
\auteur{rouget}
\organisation{exo7}
\datecreate{2010-07-04}
\video{4EJcG9wA3cI}
\isIndication{true}
\isCorrection{true}
\chapitre{Dénombrement}
\sousChapitre{Cardinal}

\contenu{
\texte{
On part du point de coordonnées $(0,0)$ pour rejoindre le point de coordonnées $(p,q)$ ($p$ et $q$ entiers naturels donnés) en se déplaçant à chaque étape d'une unité vers la droite ou vers le haut. Combien y a-t-il de chemins possibles~?
}
\indication{Coder un chemin par un mot : $D$ pour droite, $H$ pour haut.}
\reponse{
On pose $H=$ "vers le haut"  et $D=$ "vers la droite". 
Un exemple de chemin de $(0,0)$ à $(p,q)$ est le mot $DD...DHH...H$ 
où $D$ est écrit $p$ fois et $H$ est écrit $q$ fois. 
Le nombre de chemins cherché est clairement le nombre d'anagrammes 
du mot précédent. 


Le nombre de choix de l'emplacement du $H$ est $C_{p+q}^q$. Une fois que les lettres $H$
sont placées il n'y a plus de choix pour les lettres $D$.
Il y a donc $C_{p+q}^q$ chemins possibles.

Remarque : si on place d'abord les lettres $D$ alors on a $C_{p+q}^p$ choix possibles.
Mais on trouve bien sûr le même nombre de chemins car $C_{p+q}^p = C_{p+q}^{(p+q) - p} = C_{p+q}^q$.
}
}

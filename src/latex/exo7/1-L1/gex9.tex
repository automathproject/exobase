\uuid{gex9}
\exo7id{919}
\auteur{liousse}
\organisation{exo7}
\datecreate{2003-10-01}
\video{lE3VGElRQrs}
\isIndication{true}
\isCorrection{true}
\chapitre{Espace vectoriel}
\sousChapitre{Somme directe}

\contenu{
\texte{
Soient $v_1=(0,1,-2,1),
v_2=(1,0,2,-1), v_3=(3,2,2,-1), v_4 = (0,0,1,0)$ et
$v_5=(0,0,0,1)$ des vecteurs de $\Rr^4$.  Les propositions
suivantes sont-elles vraies ou fausses ?  Justifier votre r\'eponse.
}
\begin{enumerate}
    \item \question{$\text{Vect}\{ v_1, v_2, v_3 \} =  \text{Vect}\{(1,1,0,0),(-1,1,-4,2)\}$.}
\reponse{Vrai. $\text{Vect}\{(1,1,0,0),(-1,1,-4,2)\}$ est inclus dans $\text{Vect} \{v_1,v_2,v_3\}$, car
$(1,1,0,0) = v_1+v_2$ et $(-1,1,-4,2)=v_1-v_2$. Comme ils sont de m\^eme dimension ils sont \'egaux 
(autrement dit : comme un plan est inclus dans un autre alors ils sont égaux).}
    \item \question{$(1,1,0,0) \in \text{Vect}\{ v_1, v_2 \} \cap \text{Vect}\{ v_2, v_3, v_4 \}$.}
\reponse{Vrai. On a $(1,1,0,0) = v_1+v_2$ donc  $(1,1,0,0)\in \text{Vect}\{v_1,v_2\}$, or 
$\text{Vect}\{v_1,v_2\}=\text{Vect}\{v_2,v_3\}\subset \text{Vect}\{v_2,v_3,v_4\}$. 
Donc $(1,1,0,0) \in \text{Vect}\{v_1, v_2\} \cap    \text{Vect}\{v_2, v_3, v_4\}$.}
    \item \question{$\dim(\text{Vect}\{ v_1, v_2 \} \cap \text{Vect}\{ v_2, v_3, v_4 \})=1$ (c'est-à-dire c'est une droite vectorielle).}
\reponse{Faux. Toujours la m\^eme relation nous donne que 
$\text{Vect} \{v_1, v_2\} \cap \text{Vect} \{v_2, v_3, v_4\} = \text{Vect}\{v_1, v_2\}$ donc est de dimension $2$.
C'est donc un plan vectoriel et pas une droite.}
    \item \question{$\text{Vect}\{ v_1, v_2 \} + \text{Vect}\{ v_2, v_3, v_4 \}= \Rr^4$.}
\reponse{Faux. Encore une fois la relation donne que 
$\text{Vect}\{v_1, v_2\}+\text{Vect}\{v_2, v_3, v_4\} = \text{Vect}\{v_1, v_2, v_4\}$, or $3$ vecteurs ne
   peuvent engendrer $\Rr^4$ qui est de dimension $4$.}
    \item \question{$\text{Vect}\{ v_4, v_5 \}$ est un sous-espace vectoriel 
   suppl\'ementaire de $\text{Vect}\{ v_1, v_2, v_3 \}$ dans $\Rr^4$.}
\reponse{Vrai.  Faire le calcul : l'intersection est $\{0\}$ et la somme est $\Rr^4$.}
\indication{\begin{enumerate}
\item Vrai.
\item Vrai.
\item Faux.
\item Faux.
\item Vrai.
\end{enumerate}}
\end{enumerate}
}

\uuid{gaaV}
\exo7id{5931}
\titre{exo7 5931}
\auteur{tumpach}
\organisation{exo7}
\datecreate{2010-11-11}
\isIndication{false}
\isCorrection{true}
\chapitre{Autre}
\sousChapitre{Autre}
\module{Théorie de la mesure, intégrale de Lebesgue}
\niveau{L3}
\difficulte{}

\contenu{
\texte{

}
\begin{enumerate}
    \item \question{Calculer $\int_{-\infty}^{+\infty} e^{-x^2}\,dx$.\\
\emph{Indication~:} On pourra d'abord calculer
$\int_{\mathbb{R}^{2}} e^{-(x^2 + y^2)}\, dx dy$ en passant en
coordonn\'ees polaires.}
\reponse{Soit $I = \int_{-\infty}^{+\infty} e^{-x^2}\,dx$. On a~:
$$
I^{2} = \int_{\mathbb{R}^{2}} e^{-(x^2 + y^2)}\, dx dy.
$$
L'application $\Phi~: \mathbb{R}_{+}^{*} \times ]0, 2\pi[
\rightarrow \mathbb{R}^{2} \setminus \{ (x, 0), x \geq 0\}$
d\'efinie par~:
$$
\Phi(r, \theta) = \left(r\cos \theta, r\sin \theta \right)
$$
est un $\mathcal{C}^{1}$-diff\'eomorphisme. De plus
$$
\int_{\mathbb{R}^{2}} e^{-(x^2 + y^2)}\, dx dy =
\int_{\mathbb{R}^{2}\setminus\{ (x, 0), x \geq 0\}} e^{-(x^2 + y^2)}\, dx dy
$$
car l'ensemble $\{ (x, 0), x \geq 0\}$ est n\'egligeable. On en
d\'eduit que~:
$$
I^2 = \int_{r=0}^{+\infty}\int_{\theta=0}^{2\pi} e^{-r^2} r \,dr
d\theta = 2\pi \left[\frac{-e^{-r^2}}{2} \right]_{0}^{+\infty} =
\pi.
$$
Ainsi $I = \int_{-\infty}^{+\infty} e^{-x^2}\,dx  = \sqrt{\pi}$.}
    \item \question{\emph{Calcul de l'aire de la
sph\`ere unit\'e de $\mathbb{R}^{n}$.} Soit $ \mathcal{S}_{n-1} =
\{(x_1, \dots, x_n) \in \mathbb{R}^n ,\, \sum_{i=1}^{n} x_{i}^2 =
1 \} $ la sph\`ere unit\'e de $\mathbb{R}^{n}$. On note
$\mathcal{A}_{n-1}$ son aire. Calculer
$$
\int_{\mathbb{R}^n} e^{-\sum_{i=1}^{n} x_{i}^2} \,dx_{1}\dots
dx_{n}
$$
en fonction de $\mathcal{A}_{n-1}$. En d\'eduire l'expression de
$\mathcal{A}_{n-1}$ en fonction de la fonction $\Gamma$~:
$$
\Gamma(s) := \int_{0}^{+\infty} x^{s-1} e^{-x}\,dx.
$$}
\reponse{\emph{Calcul de l'aire de la sph\`ere unit\'e de
$\mathbb{R}^{n}$.} Soit $ \mathcal{S}_{n-1} = \{(x_1, \dots, x_n)
\in \mathbb{R}^n ,\, \sum_{i=1}^{n} x_{i}^2 = 1 \} $ la sph\`ere
unit\'e de $\mathbb{R}^{n}$. On note $\mathcal{A}_{n-1}$ son aire.
D'apr\`es la question pr\'ec\'edente, on a~:
$$
\int_{\mathbb{R}^n} e^{-\sum_{i=1}^{n} x_{i}^2} \,dx_{1}\dots
dx_{n} = {\pi}^{\frac{n}{2}}.
$$
D'autre part, puisque l'aire de la sph\`ere de rayon $r$ dans
$\mathbb{R}^n$ vaut $r^{n-1}\mathcal{A}_{n-1}$, il vient~:
$$
\int_{\mathbb{R}^n} e^{-\sum_{i=1}^{n} x_{i}^2} \,dx_{1}\dots
dx_{n} = \mathcal{A}_{n-1} \int_{0}^{+\infty} e^{-r^2}
r^{n-1}\,dr.$$ En posant le changement de variable $x = r^{2}$, on
obtient~:
$$
\int_{\mathbb{R}^n} e^{-\sum_{i=1}^{n} x_{i}^2} \,dx_{1}\dots
dx_{n} = \frac{1}{2}\mathcal{A}_{n-1}\int_{0}^{+\infty} e^{-x}
x^{\frac{n}{2}-1} \,dx,
$$
d'o\`u~:
$$
\mathcal{A}_{n-1} = \frac{2{\pi}^{\frac{n}{2}}}{\Gamma\left(
\frac{n}{2}\right)}.
$$}
    \item \question{\emph{Calcul du volume de la boule unit\'e de
$\mathbb{R}^{n}$.} Soit $ \mathcal{B}_{n} = \{(x_1, \dots, x_n)
\in \mathbb{R}^n ,\, \sum_{i=1}^{n} x_{i}^2 \leq 1 \} $ la boule
ferm\'ee de rayon 1 dans $\mathbb{R}^n$. On note $\mathcal{V}_{n}$
son volume. Montrer que $\mathcal{V}_{n} =
\frac{\mathcal{A}_{n-1}}{n}$. En d\'eduire que~:
$$
\mathcal{V}_{n} =
\frac{{\pi}^{\frac{n}{2}}}{\Gamma\left(\frac{n}{2} + 1 \right)}.
$$}
\reponse{\emph{Calcul du volume de la boule unit\'e de
$\mathbb{R}^{n}$.} Soit $ \mathcal{B}_{n} = \{(x_1, \dots, x_n)
\in \mathbb{R}^n ,\, \sum_{i=1}^{n} x_{i}^2 \leq 1 \} $ la boule
ferm\'ee de rayon 1 dans $\mathbb{R}^n$. On note $\mathcal{V}_{n}$
son volume. On a~:
$$\mathcal{V}_{n} = \int_{0}^{1} r^{n-1}
\mathcal{A}_{n-1} \,dr =  \mathcal{A}_{n-1} \left[\frac{r^{n}}{n}
\right]_{0}^{1} = \frac{\mathcal{A}_{n-1}}{n}.
$$
On en d\'eduit que~:
$$
\mathcal{V}_{n} = \frac{2{\pi}^{\frac{n}{2}}}{n\Gamma\left(
\frac{n}{2}\right)}.
$$
Ce qui se r\'eduit \`a~:
$$
\mathcal{V}_{n} =
\frac{{\pi}^{\frac{n}{2}}}{\Gamma\left(\frac{n}{2} + 1 \right)}
$$
en utilisant l'identit\'e~: $\Gamma(s + 1) = s\Gamma(s)$.}
    \item \question{\emph{Application~:} Que vaut l'aire de la sph\`ere de rayon
$R$ dans $\mathbb{R}^2$? $\mathbb{R}^3$? Que vaut le volume de la
boule de rayon $R$ dans $\mathbb{R}$? $\mathbb{R}^{2}$?
$\mathbb{R}^{3}$?}
\reponse{\emph{Application~:} L'aire de la sph\`ere de rayon $R$ dans
$\mathbb{R}^2$ vaut $$\mathcal{A}_1 R = \frac{2\pi}{\Gamma(1)} R =
2\pi R, $$ qui est bien le p\'erim\`etre du cercle de rayon $R$
dans le plan.

Sachant $\Gamma(\frac 12) = \sqrt \pi$, l'aire de la sph\`ere de rayon $R$ dans $\mathbb{R}^3$ vaut
 $$
 \mathcal{A}_{2} R^2 = \frac{2 {\pi}^{\frac{3}{2}}}{\Gamma\left(\frac{3}{2}
 \right)}
 R^2
 = \frac{2\pi \sqrt{\pi}}{\frac{1}{2}\Gamma\left(\frac{1}{2}\right)} R^2 = 4\pi R^2
$$
qui est bien l'aire de la sph\`ere $S^2$.


  Le volume de la boule de rayon
$R$ dans $\mathbb{R}$ vaut
$$
\mathcal{V}_{1} R = \frac{2{\pi}^{\frac{1}{2}}}{\Gamma\left(
\frac{1}{2}\right)} R = \frac{2 \sqrt{\pi}}{\sqrt{\pi}} R = 2R,
$$
qui est bien la longueur du segment $[-R, R]$.



Le volume de la boule de rayon $R$ dans  $\mathbb{R}^{2}$ vaut
$$
\mathcal{V}_{2} R^{2} = \frac{2\pi}{2\Gamma(1)} R^2 = \pi R^2,
$$
qui est bien l'aire du disque de rayon $R$.



Le volume de la boule de rayon $R$ dans $\mathbb{R}^{3}$ vaut
$$
\mathcal{V}_{3} R^3 = \frac{\mathcal{A}_{2}}{3} R^3 = \frac{4}{3}
\pi R^3.
$$}
\end{enumerate}
}

\uuid{SzS8}
\exo7id{5966}
\auteur{tumpach}
\organisation{exo7}
\datecreate{2010-11-11}
\isIndication{false}
\isCorrection{true}
\chapitre{Espace L^p}
\sousChapitre{Espace Lp}

\contenu{
\texte{
Soit $\Omega = \mathbb{R}^{n}$ muni de la mesure de Lebesgue. Pour
tout $1\leq p <+\infty$, on note $L^p(\mathbb{R}^{n})$ l'espace
des fonctions $f~:\mathbb{R}^{n} \rightarrow \mathbb{C}$ telles
que $\|f\|_{p} := \left(\int_{\mathbb{R}^{n}}
|f|^{p}(x)\,dx\right)^{\frac{1}{p}}<+\infty$ modulo
l'\'equivalence $f\sim g \Leftrightarrow f-g = 0 ~\mu-p.p$.
L'espace des fonctions essentiellement born\'ees sera not\'e
$L^{\infty}(\mathbb{R}^n).$
}
\begin{enumerate}
    \item \question{\begin{itemize}}
\reponse{\begin{itemize}}
    \item \question{Pour quelle valeur de $\alpha$ la
fonction $x\mapsto \frac{1}{\left(1 + |x|^2 \right)^{\alpha}}$
appartient-elle \`a $L^{p}(\mathbb{R}^{n})$ ?}
\reponse{La fonction $x\mapsto
\frac{1}{\left(1 + |x|^2 \right)^{\alpha}}$ appartient \`a
$L^{p}(\mathbb{R}^{n})$ pour $2\alpha p > n$.}
    \item \question{Pour
quelle valeur de $\beta$ la fonction $x\mapsto
\frac{1}{|x|^{\beta}} e^{-\frac{|x|^2}{2}}$ appartient-elle \`a
$L^{p}(\mathbb{R}^{n})$ ?}
\reponse{La fonction $x\mapsto \frac{1}{|x|^{\beta}} e^{-\frac{|x|^2}{2}}$
appartient \`a $L^{p}(\mathbb{R}^{n})$ pour $p\beta< n$.}
    \item \question{Soit $1\leq q < p \leq
+\infty$. En utilisant $(a)$ et $(b)$, trouver une fonction $f$
qui appartienne \`a $L^{q}(\mathbb{R}^{n})$ mais pas \`a
$L^{p}(\mathbb{R}^{n})$ et une fonction $g$ qui appartienne \`a
$L^{p}(\mathbb{R}^{n})$ mais pas \`a $L^{q}(\mathbb{R}^{n})$.
  \end{itemize}}
\reponse{Soit $1\leq q < p \leq +\infty$. La fonction
$$
f(x) = (1+|x|^2)^{-\frac{n}{p+q}}
$$
v\'erifient $f \in L^{p}(\mathbb{R}^{n})$ et $f\notin
L^{q}(\mathbb{R}^{n})$.  La fonction
$$
g(x) = |x|^{-\frac{n}{2}\left(\frac{1}{p}+\frac{1}{q}\right)}
e^{-\frac{|x|^2}{2}}
$$
v\'erifient $g\in L^{q}(\mathbb{R}^{n})$ et $g\notin
L^{p}(\mathbb{R}^{n})$.

\end{itemize}}
    \item \question{\begin{itemize}}
\reponse{\begin{itemize}}
    \item \question{Soit $1\leq q< p<+\infty$.
Montrer que l'espace $L^{p}(\mathbb{R}^{n}) \cap
L^{q}(\mathbb{R}^{n})$ est un espace de Banach pour la norme
$\|\cdot \|_{p,q} = \| \cdot \|_{p} + \| \cdot \|_{q}$.}
\reponse{Soit $1\leq q< p<+\infty$ et
$f_n$ une suite de Cauchy pour la norme $\|\cdot\|_{p, q}= \|
\cdot \|_{p} + \| \cdot \|_{q}$. Comme $\|\cdot\|_{p} \leq
\|\cdot\|_{p, q}$, $f_{n}$ est une suite de Cauchy dans
$L^{p}(\mathbb{R}^{n})$, donc elle converge vers une fonction
$f\in L^{p}(\mathbb{R}^{n})$ pour la norme $\|\cdot\|_{p}$. De
m\^eme,  $\|\cdot\|_{q}\leq \|\cdot\|_{p, q}$, donc $f_{n}$
converge vers une fonction $g\in L^{q}(\mathbb{R}^{n})$ pour la
norme $\|\cdot\|_q$. De plus, il existe une sous-suite de
$f_{n_{k}}$ qui converge vers $f$ presque-partout et il existe une
sous-suite de $f_{n_{k}}$ qui converge vers $g$ presque-partout.
Ainsi $f=g ~\mu$-p.p. et $f_{n}$ converge vers $f = g$ dans
$L^{p}(\mathbb{R}^{n})\cap L^{q}(\mathbb{R}^{n})$.}
    \item \question{Soit $r$ tel que $q< r<p$. Montrer que
$$
\|f\|_{r} \leq \|f\|_{p}^{\alpha}\|f\|_{q}^{1-\alpha}
$$
o\`u $\frac{1}{r} = \frac{\alpha}{p} + \frac{1-\alpha}{q}$,
$\alpha\in[0, 1]$. On pourra \'ecrire $r = r\alpha + r(1-\alpha)$
et utiliser l'in\'egalit\'e de H\"older pour un couple de r\'eels
conjugu\'es bien choisi.}
\reponse{Soit $r$ tel que $q< r<p$. Montrons que
$$
\|f\|_{r} \leq \|f\|_{p}^{\alpha}\|f\|_{q}^{1-\alpha}
$$
o\`u $\frac{1}{r} = \frac{\alpha}{p} + \frac{1-\alpha}{q}$,
$\alpha\in[0, 1]$. Puisque $1 = \frac{\alpha r}{p} +
\frac{(1-\alpha)r}{q}$, les r\'eels $p' = \frac{p}{\alpha r}$ et
$q' = \frac{q}{(1 - \alpha)r}$ sont conjugu\'es. D'apr\`es
l'in\'egalit\'e de H\"older,
\begin{eqnarray*}
\int_{\mathbb{R}^{n}} |f|^{r}(x)\,dx& =& \int_{\mathbb{R}^{n}}
|f|^{r\alpha}(x)\cdot |f|^{(1-\alpha) r}(x)\,dx\\
& \leq & \left(\int_{\mathbb{R}^{n}} |f|^{\alpha r p'}(x)\,dx
\right)^{\frac{1}{p'}} \left( \int_{\mathbb{R}^{n}}
|f|^{(1-\alpha)r q'}(x)\,dx\right)^{\frac{1}{q'}}\\
& \leq & \left(\int_{\mathbb{R}^{n}} |f|^{p}(x)\,dx
\right)^{\frac{\alpha r}{p}} \left( \int_{\mathbb{R}^{n}}
|f|^{q}(x)\,dx\right)^{\frac{(1-\alpha)r}{q}}\\
& \leq & \|f\|_{p}^{\alpha r} \|f\|_{q}^{(1-\alpha)r},
\end{eqnarray*}
ce qui \'equivaut \`a $\|f\|_{r} \leq
\|f\|_{p}^{\alpha}\|f\|_{q}^{(1-\alpha)}$. On peut \'egalement
\'ecrire  $r = \beta q + (1-\beta)p$ avec $\beta \in ]0, 1[$ et
appliquer H\"older avec les r\'eels conjugu\'es $\frac{1}{\beta}$
et $ \frac{1}{1-\beta}$~:
\begin{eqnarray*}
\int_{\mathbb{R}^{n}} |f|^{r}(x)\,dx& =& \int_{\mathbb{R}^{n}}
|f|^{\beta q}(x)\cdot|f|^{(1-\beta)p}(x)\,dx \\
&\leq & \left(\int_{\mathbb{R}^{n}} |f|^{q}(x)\,dx \right)^{\beta}
\left( \int_{\mathbb{R}^{n}} |f|^{p}(x)\,dx\right)^{(1-\beta)},
\end{eqnarray*}
ce qui implique
$$
\|f\|_{r} \leq
\|f\|_{q}^{\frac{q\beta}{r}}\|f\|_{p}^{\frac{p(1-\beta)}{r}}
$$
qui est l'in\'egalit\'e cherch\'ee car $\alpha = \frac{p\beta}{r}$
v\'erifie bien $\frac{1}{r} = \frac{\alpha}{p} +
\frac{1-\alpha}{q}$.}
    \item \question{En d\'eduire que si $f_{n}$
converge vers $f$ dans $L^{p}(\mathbb{R}^{n}) \cap
L^{q}(\mathbb{R}^{n})$ alors $f_{n}$ converge vers $f$ dans
$L^{r}(\mathbb{R}^{n})$, i.e $L^{p}(\mathbb{R}^{n}) \cap
L^{q}(\mathbb{R}^{n})$ est un sous-espace de Banach de
$L^{r}(\mathbb{R}^{n})$.
\end{itemize}}
\reponse{Si $f_{n}$ converge vers $f$ dans
$L^{p}(\mathbb{R}^{n}) \cap L^{q}(\mathbb{R}^{n})$ alors $f_{n}$
converge vers $f$ dans $L^{p}(\mathbb{R}^{n})$ et dans
$L^{q}(\mathbb{R}^{n})$, donc dans $L^{r}(\mathbb{R}^{n})$
d'apr\`es l'in\'egalit\'e pr\'ec\'edente. En conclusion,
$L^{p}(\mathbb{R}^{n}) \cap L^{q}(\mathbb{R}^{n})$ est ferm\'e
dans $L^{r}(\mathbb{R}^{n})$ donc un sous-espace de Banach de
$L^{r}(\mathbb{R}^{n})$.
\end{itemize}}
    \item \question{Soit $f\in L^{p}([0, +\infty[)\cap
L^{q}([0, +\infty[)$ avec $1\leq q<2<p$. Montrer que la fonction
$h$ d\'efinie par $h(r) = \frac{1}{\sqrt{r}}f(r)$ appartient \`a
$L^{1}([0, +\infty[)$ et trouver des constantes $C_{p,q}$ et
$\gamma$ telles que $\|h\|_{1} \leq C_{p, q}
\|f\|_{q}^{\gamma}\|f\|_{p}^{(1-\gamma)}$.}
\reponse{Soit $f\in L^{p}([0, +\infty[)\cap L^{q}([0, +\infty[)$ et
$h$ la fonction d\'efinie par $h(r) = \frac{1}{\sqrt{r}}f(r)$. On
notera $p'$ le conjugu\'e de $p$ et $q'$ le conjugu\'e de $q$.
Montrons que $h$ appartient \`a $L^{1}([0, +\infty[)$. On a~:
\begin{eqnarray*}
\int_{0}^{+\infty} \frac{1}{\sqrt{r}} |f(r)|\,dr = \int_{0}^{R}
\frac{1}{\sqrt{r}} |f(r)|\,dr + \int_{R}^{+\infty}
\frac{1}{\sqrt{r}} |f(r)|\,dr\\
\leq \left( \int_{0}^{R}
r^{-\frac{p'}{2}}\,dr\right)^{\frac{1}{p'}} \left(\int_{0}^{R}
|f(r)|^{p} \,dr\right)^{\frac{1}{p}} + \left(\int_{R}^{+\infty}
r^{-\frac{q'}{2}}\,dr \right)^{\frac{1}{q'}} \left(
\int_{R}^{+\infty}
|f(r)|^{q}\,dr\right)^{\frac{1}{q}}\\
 \leq  \left(\frac{1}{1-\frac{p'}{2}}\right)^{\frac{1}{p'}}
R^{\left(\frac{1}{p'}-\frac{1}{2}\right)}\|f\|_{p} +
\left(\frac{1}{\frac{q'}{2} - 1}\right)^{\frac{1}{q'}}
R^{\left(\frac{1}{q'}-\frac{1}{2}\right)} \|f\|_{q}.
\end{eqnarray*}
En optimisant par rapport \`a $R$, on obtient~:
\begin{eqnarray*}
\int_{0}^{+\infty} \frac{1}{\sqrt{r}} |f(r)|\,dr &\leq & C_{p,q}
\|f\|_{p}^{1-\gamma}\|f\|_{q}^{\gamma},
\end{eqnarray*}
o\`u, en posant $\alpha =  \frac{1}{2}- \frac{1}{p}$ et $\beta =
 \frac{1}{q} - \frac{1}{2}$, on a $\gamma =
\frac{\alpha}{\alpha+\beta}$, et $C_{p,q} = \frac{\alpha +
\beta}{\alpha^{\gamma} \beta^{1-\gamma}}\left(1-\frac{p'}{2}
\right)^{-\frac{1-\gamma}{p'}}\left(\frac{q'}{2} - 1
\right)^{-\frac{\gamma}{q'}}$.}
\end{enumerate}
}

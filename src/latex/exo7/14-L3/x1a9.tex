\uuid{x1a9}
\exo7id{5979}
\titre{exo7 5979}
\auteur{tumpach}
\organisation{exo7}
\datecreate{2010-11-11}
\isIndication{false}
\isCorrection{true}
\chapitre{Autre}
\sousChapitre{Autre}

\contenu{
\texte{
\textbf{D\'efinition.}
Soit $h\in \mathbb{R}^{n}$. On d\'efinit l'op\'erateur de
translation par $h$, not\'e $\tau_{h}$, agissant sur une fonction
$f~:\mathbb{R}^n \rightarrow \mathbb{R}$ par
$
\tau_{h}f(x) := f(x-h), \quad \forall x\in\mathbb{R}^n.
$

\bigskip

\textbf{Th\'eor\`eme.}
%\label{th:t1}
Si $f\in L^{p}(\mathbb{R}^n)$ avec $1\leq p < +\infty$, alors
$
\lim_{h\rightarrow 0}\|\tau_{h}f - f\|_{p} = 0,
$
i.e. $\tau_{h}f$ tend vers $f$ dans $L^{p}(\mathbb{R}^n)$ lorsque
$h$ tend vers $0$.

\bigskip

Le but de cet exercice est de d\'emontrer ce th\'eor\`eme.
Soit $1 \leq p < +\infty$.
}
\begin{enumerate}
    \item \question{Montrer que si $f$ est continue \`a support compact dans la
boule $\mathcal{B}(0, M)$ centr\'ee en $0$ et de rayon $M$, et si
$|h|\leq 1$, alors
$$
 |f(x - h) - f(x)|^p \leq \mathbf{1}_{B(0, M+1)}
2^p \|f\|_{\infty}^p.
$$
o\`u $\mathcal{B}(0, M+1)$ est la boule centr\'ee en $0$ de rayon
$M+1$.}
\reponse{Si $f$ est continue \`a support compact dans la boule
$\mathcal{B}(0, M)$ centr\'ee en $0$ et de rayon $M$, et si
$|h|\leq 1$, alors
$$
\begin{array}{lcl}
 |f(x - h) - f(x)|^p & \leq & \left(|f(x - h)| + |f(x)|\right)^p
   \leq  \left(2\|f\|_{\infty}\mathbf{1}_{\mathcal{B}(0, M+1)}\right)^p =  \mathbf{1}_{B(0, M+1)}
2^p \|f\|_{\infty}^p.
\end{array}
$$
o\`u $\mathcal{B}(0, M+1)$ est la boule centr\'ee en $0$ de rayon
$M+1$.}
    \item \question{En d\'eduire que pour $f$ continue \`a support compact, on a
$$
\lim_{h\rightarrow 0}\|\tau_{h}f - f\|_{p} = 0.
$$}
\reponse{Pour $f$ continue, on a $\lim_{h \rightarrow 0} |f(x- h) -
f(x)| = 0.$ Puisque la fonction $g(x) = 2^p
\|f\|_{\infty}^p\,\mathbf{1}_{B(0, M+1)}(x) $ appartient \`a
$L^1(\mathbb{R}^{n})$, le th\'eor\`eme de convergence domin\'ee
permet d'intervertir limite et int\'egrale, et il vient~:
$$
\lim_{h \rightarrow 0}\|\tau_{h}f - f\|_{p}^p = \lim_{h
\rightarrow 0}\int_{\mathbb{R}^n} |f(x - h) - f(x)|^{p}\,dx =
\int_{\mathbb{R}^n} \lim_{h \rightarrow 0} |f(x - h) -
f(x)|^{p}\,dx = 0.
$$}
    \item \question{D\'emontrer le th\'eor\`eme pour une fonction quelconque
dans $L^p(\mathbb{R}^n)$, $1\leq p <+\infty$.}
\reponse{Soit $f$ une fonction quelconque dans $L^p(\mathbb{R}^n)$,
$1\leq p <+\infty$. Par densit\'e des fonctions continues \`a
support compact dans $L^{p}(\mathbb{R}^n)$, pour tout $\varepsilon
> 0$, il existe $f_{\varepsilon}$ continue \`a support compact
telle que $\|f - f_{\varepsilon}\|_{p} \leq
\frac{\varepsilon}{3}$. Ainsi~:
$$
\begin{array}{lcl}
\|\tau_{h}f - f\|_{p} & = & \|\tau_{h}(f - f_{\varepsilon}) - (f -
f_{\varepsilon}) + \tau_{h}f_{\varepsilon} - f_{\varepsilon}\|_p\\
& \leq & \|\tau_{h}(f - f_{\varepsilon})\|_p + \|f -
f_{\varepsilon}\|_p + \|\tau_{h}f_{\varepsilon} -
f_{\varepsilon}\|_p\\
& = & 2 \|f - f_{\varepsilon}\|_p + \|\tau_{h}f_{\varepsilon} -
f_{\varepsilon}\|_p\\
& = & \frac{2}{3}\varepsilon + \|\tau_{h}f_{\varepsilon} -
f_{\varepsilon}\|_p.
\end{array}
$$
Puisque $f_{\varepsilon}$ est continue \`a support compact,
d'apr\`es la question pr\'ec\'edente, il existe $\delta>0$ tel que
pour $|h|<\delta$,  $\|\tau_{h}f_{\varepsilon} -
f_{\varepsilon}\|_p < \frac{\varepsilon}{3}$. Ainsi, pour $|h| <
\delta$, on a $\|\tau_{h}f - f\|_{p}< \varepsilon$. En d'autre
termes $\lim_{h \rightarrow 0}\|\tau_{h}f - f\|_{p}  = 0$.}
    \item \question{Que se passe-t-il pour $p = \infty$ ?}
\reponse{Pour $p = \infty$, les fonctions continues \`a support
compact ne sont pas denses dans $L^{\infty}(\mathbb{R}^n)$ ce qui
fait que la d\'emonstration pr\'ec\'edente ne peut pas s'appliquer
dans ce cas. De plus, on v\'erifie que, pour $f = \mathbf{1}_{B(0, 1)}$ et
$h \neq 0$, on a
$$
\|\tau_{h}f - f\|_{\infty} = 1.
$$
Alors que pour $h = 0$, on a $\|\tau_{h}f - f\|_{\infty} = 0. $ On
peut \'egalement v\'erifier que $\lim_{h\rightarrow 0}\|\tau_{h}f
- f \|_{\infty} = 0$ si et seulement si la fonction $f$ poss\`ede
un repr\'esentant uniform\'ement continu.}
\end{enumerate}
}

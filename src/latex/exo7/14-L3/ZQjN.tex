\uuid{ZQjN}
\exo7id{5952}
\titre{exo7 5952}
\auteur{tumpach}
\organisation{exo7}
\datecreate{2010-11-11}
\isIndication{false}
\isCorrection{true}
\chapitre{Théorème de convergence dominée}
\sousChapitre{Théorème de convergence dominée}

\contenu{
\texte{
\label{derivation}
Montrer le th\'eor\`eme suivant, $\Omega$ étant un espace mesurable.
(On pourra utiliser le th\'eor\`eme des accroissements finis.) \\
\textbf{Th\'eor\`eme.}(D\'erivation sous le signe $\int$) \\
Soit $f~:\Omega \times \mathbb{R} \rightarrow \mathbb{C}$ une
fonction telle que
}
\begin{enumerate}
    \item \question{[(i)] Pour tout $s\in[s_{1}, s_{2}]$, la fonction $x\mapsto
f(x, s)$ est int\'egrable~;}
    \item \question{[(ii)] pour presque tout $x$, la fonction $s\mapsto f(x, s) $
est d\'erivable sur $(s_1, s_2)$~;}
    \item \question{[(ii)] il existe $g\in \mathcal{L}^1(\Omega, \mathbb{R}^{+})$
tel que pour tout $s \in [s_1, s_2]$ et pour presque tout $x\in\Omega$ on ait $|\frac{\partial f(x,s)}{\partial s}| \leq g(x)$ .}
\reponse{
Cf le th\'eor\`eme 24.2 dans le polycopi\'e de Marc Troyanov.
}
\end{enumerate}
}

\uuid{ARHJ}
\exo7id{5962}
\titre{exo7 5962}
\auteur{tumpach}
\organisation{exo7}
\datecreate{2010-11-11}
\isIndication{false}
\isCorrection{true}
\chapitre{Intégrales multiples, théorème de Fubini}
\sousChapitre{Intégrales multiples, théorème de Fubini}

\contenu{
\texte{
Soient $f, g\in L^1(\mu)$ o\`u $\mu$ est la mesure de Lebesgue sur
$\mathbb{R}^n$. On note $\hat{f}$ la transform\'ee de Fourier
d\'efinie par
$$
\hat{f}(y) = \int_{\mathbb{R}^n} f(x)\,e^{-2\pi i (y, x)} \,dx,
$$
o\`u $(\cdot, \cdot)$ d\'esigne le produit scalaire de
$\mathbb{R}^n.$ Montrer que
}
\begin{enumerate}
    \item \question{$\quad \int_{\mathbb{R}^n} f(x)\hat{g}(x)\,dx =
\int_{\mathbb{R}^n} \hat{f}(x) g(x)\,dx.$}
\reponse{On a $\|\hat{g}\|_{\infty} \leq \|g\|_1$, ce qui
implique que $f\, \hat{g}$ est int\'egrable. De m\^eme
$\hat{f}\,g$ est int\'egrable. De plus $F(x, y) = f(x) g(y)
e^{-2\pi i (x, y)}$ appartient \`a
$L^1(\mathbb{R}^n\times\mathbb{R}^n)$. D'apr\`es le th\'eor\`eme
de Fubini, \begin{eqnarray*} \int_{\mathbb{R}^n}
f(x)\hat{g}(x)\,dx &=& \int_{\mathbb{R}^n}\,dx\, f(x)
\int_{\mathbb{R}^n} g(y) e^{-2\pi i (x, y)} \,dy \\ &=&
\int_{\mathbb{R}^n}\,dy \,g(y) \int_{\mathbb{R}^n} f(x)e^{-2\pi i
(x, y)} \,dx = \int_{\mathbb{R}^n} \hat{f}(y) g(y)\,dx.
\end{eqnarray*}}
    \item \question{$\quad
\widehat{f*g} = \hat{f}\,\hat{g}.$}
\reponse{On a
\begin{eqnarray*}
\widehat{f*g}(x) &=& \int_{\mathbb{R}^n} f*g(y)\,e^{-2\pi i (x,
y)}\,dy = \int_{\mathbb{R}^n}\,dy \,e^{-2\pi i (x,
y)}\int_{\mathbb{R}^n}f(y-z)\,g(z) \,dz\\
&=& \int_{\mathbb{R}^n} \,dy \int_{\mathbb{R}^n}\,e^{-2\pi i
(x, y-z)}e^{-2\pi i (x,
z)}f(y-z)\,g(z)\,dz \\
& = & \int_{\mathbb{R}^n} \,e^{-2\pi i (x, u)} f(u)\,du
\int_{\mathbb{R}^n}e^{-2\pi i (x, z)}\,g(z) \,dz\\
& = & \hat{f}(x) \,\hat{g}(x).
\end{eqnarray*}}
\end{enumerate}
}

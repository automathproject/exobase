\uuid{n9qs}
\exo7id{5961}
\titre{exo7 5961}
\auteur{tumpach}
\organisation{exo7}
\datecreate{2010-11-11}
\isIndication{false}
\isCorrection{true}
\chapitre{Intégrales multiples, théorème de Fubini}
\sousChapitre{Intégrales multiples, théorème de Fubini}

\contenu{
\texte{

}
\begin{enumerate}
    \item \question{Pour tout $t>0$, on pose~:
$$
f_{t}(x) = \left(4\pi t \right)^{-\frac{n}{2}}
e^{-\frac{|x|^2}{4t}}.
$$
\begin{enumerate}}
\reponse{Pour tout $t>0$, on pose~:
$$
f_{t}(x) = \left(4\pi t \right)^{-\frac{n}{2}}
e^{-\frac{|x|^2}{4t}}.
$$
\begin{enumerate}}
    \item \question{Montrer que, pour tout $t>0$, $\int_{\mathbb{R}^n}
f_t(x)\,dx = 1$.}
\reponse{On a
\begin{eqnarray*}
\int_{\mathbb{R}^n} f_t(x)\,dx & = & \left(4\pi t
\right)^{-\frac{n}{2}} \int_{\mathbb{R}^n}
e^{-\frac{|x|^2}{4t}}\,dx\\
& = & \left(4\pi t \right)^{-\frac{n}{2}} \prod_{i= 1}^{n}
\int_{\mathbb{R}} e^{-\frac{x_i^2}{4t}}\,dx_i.
\end{eqnarray*}
Sachant que $\int_{\mathbb{R}} e^{-t^2} \,dt = \sqrt{\pi}$, on en
d\'eduit que
$$
\int_{\mathbb{R}^n} f_t(x)\,dx  = 1.
$$}
    \item \question{Montrer  que, pour tout $\delta>0$, $\lim_{t\rightarrow
0}\int_{\{|x|>\delta\}} f_t(x)\,dx = 0$.}
\reponse{Soit $\varepsilon>0$. Puisque $f_1$ est int\'egrable sur $\mathbb{R}^{n}$,
 il existe un $R>0$ tel que $$\int_{\mathcal{B}(0, R)^c} f_1(x)\,dx < \varepsilon.$$
 On remarque que $f_t(x) = t^{-\frac{n}{2}}\,f_1\left(\frac{x}{\sqrt{t}}
 \right)$. On a alors,
\begin{eqnarray*}
\int_{\mathcal{B}(0, \delta)^c} f_t(x)\,dx &=&
\int_{\mathcal{B}(0,
\delta)^c}t^{-\frac{n}{2}}\,f_1\left(\frac{x}{\sqrt{t}}
 \right) \,dx = t^{-\frac{n}{2}}\,\int_{\mathcal{B}\left(0,
\frac{\delta}{\sqrt{t}}\right)^c} f_1(z) \,t^{\frac{n}{2}}\,dz
\\&=& \int_{\mathcal{B}\left(0, \frac{\delta}{\sqrt{t}}\right)^c} f_1(z)\,dz
\leq \varepsilon,
\end{eqnarray*}
d\`es que $t < \frac{\delta^2}{R^2}$.}
\end{enumerate}
}

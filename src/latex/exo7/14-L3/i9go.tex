\uuid{i9go}
\exo7id{5969}
\titre{exo7 5969}
\auteur{tumpach}
\organisation{exo7}
\datecreate{2010-11-11}
\isIndication{false}
\isCorrection{true}
\chapitre{Autre}
\sousChapitre{Autre}

\contenu{
\texte{
\textbf{D\'efinition.}

On dit qu'un espace m\'etrique $E$ est \emph{s\'eparable} s'il
existe un sous-ensemble $\mathcal{F} \subset E$ d\'enombrable et
dense. 


\bigskip

\textbf{Th\'eor\`eme}
L'espace $L^p(\mathbb{R}^{n})$ est s\'eparable pour $1\leq p
<+\infty$.

\bigskip

Le but de cet exercice  est de d\'emontrer ce th\'eor\`eme.
}
\begin{enumerate}
    \item \question{Pour $j = 1, 2, 3, \dots$ et $m = (m_1,\ldots,m_n) \in\mathbb{Z}^{n}$, on
consid\`ere les cubes
$$
\Gamma_{j, m}~:= \{ x\in\mathbb{R}^n,~ 2^{-j}m_{i} < x_i \leq
2^{-j}(m_i + 1),~ i = 1, \dots, n\}.
$$
Montrer que pour tout $j \in \mathbb{N}^*$,
$\bigcup_{m\in\mathbb{Z}^n} \Gamma_{j, m} = \mathbb{R}^{n}$.}
    \item \question{Pour $j\in\mathbb{N}^*$, on consid\`ere l'ensemble
$\mathcal{F}_{j}$ de fonctions $\varphi$ de la forme~:
$$
\varphi(x) = \sum_{m\in\mathbb{Z}^n} c_{j, m}\, \mathbf{1}_{\Gamma_{j,
m}},
$$
o\`u les constantes $c_{j, m}\in\mathbb{Q}$ et sont nulles sauf un
nombre fini. Montrer que l'ensemble
$$
\mathcal{F} = \bigcup_{j = 1}^{\infty} \mathcal{F}_{j}
$$
est d\'enombrable.}
    \item \question{Le but de cette question est de montrer que toute fonction
continue \`a support compact peut \^etre approch\'ee \`a
$\varepsilon$ pr\`es en norme $L^{p}$ par un \'el\'ement de la
famille $\mathcal{F}$. Soit $\tilde{f}$ une fonction continue \`a
support compact et soit $\varepsilon>0$ fix\'e.

\begin{itemize}}
    \item \question{Montrer que pour tout $\varepsilon'>0$, il existe $j
\in\mathbb{N}^*$, tel que $\forall m\in\mathbb{Z}^n$,
$$
x, y \in \Gamma_{j, m} \Rightarrow |\tilde{f}(x) - \tilde{f}(y)|<
\varepsilon'.
$$}
    \item \question{Soit $\varepsilon'>0$ fix\'e et $j$ comme dans la
question pr\'ec\'edente. On consid\`ere la fonction
$\tilde{f}_{j}$ d\'efinie par~:
$$
\tilde{f}_{j}(x) = 2^{nj} \int_{\Gamma_{j, m}}\tilde{f}(y)\,dy
\quad\text{lorsque}\quad x\in\Gamma_{j, m},
$$
i.e. la valeur de $\tilde{f}_{j}$ en un point $x\in\mathbb{R}^{n}$
est la valeur moyenne de la fonction $\tilde{f}$ sur le cube
$\Gamma_{j, m}$ de cot\'e $2^{-j}$ qui contient $x$. Montrer que
$\forall m\in\mathbb{Z}^n$,
$$
x \in \Gamma_{j, m} \Rightarrow |\tilde{f}(x) - \tilde{f}_{j}(x)|<
\varepsilon',
$$
et en d\'eduire que
$$
\|\tilde{f} - \tilde{f}_{j}\|_p <
\text{Volume}(\gamma)^{\frac{1}{p}}\cdot\varepsilon'
$$
o\`u $\gamma$ est un cube de la forme $\{x\in\mathbb{R}^{n},
-2^{J} \leq x_i \leq 2^J\}$ en dehors duquel $\tilde{f}$ est
nulle.}
    \item \question{En d\'eduire qu'il existe $f_{j}\in\mathcal{F}_{j}$
telle que $\|\tilde{f} -f_{j}\|_{p} < \varepsilon.$ (On rappelle
que les \'el\'ements de $\mathcal{F}_{j}$ ne prennent que des
valeurs rationnelles.)
\end{itemize}}
    \item \question{Montrer que toute fonction $f\in L^{p}(\mathbb{R}^{n})$,
$1\leq p <+\infty$, peut \^etre approch\'ee \`a $\varepsilon$
pr\`es en norme $L^{p}$ par un \'el\'ement de la famille
$\mathcal{F}$. Conclure.}
\reponse{
Voir le lemme 2.17 p.61 dans \textit{Analysis} de E.~Lieb et M.~Loss, American Mathematical Society (2001).
}
\end{enumerate}
}

\uuid{xXAE}
\exo7id{2224}
\titre{exo7 2224}
\auteur{matos}
\organisation{exo7}
\datecreate{2008-04-23}
\isIndication{false}
\isCorrection{true}
\chapitre{Autre}
\sousChapitre{Autre}

\contenu{
\texte{
Montrer que la factorisation LU pr\'eserve la structure des matrices bande au sens suivant : 
$$a_{ij}=0 \mbox{ pour } |i-j| \geq p \Rightarrow \left\{ \begin{array}{ll} 
l_{ij}=0 & \mbox{ pour } i-j \geq p\\ 
u_{ij}=0 & \mbox{ pour } j-i \geq p 
\end{array}\right.$$
}
\reponse{
Montrons par r\'ecurrence que $A_n=U$ est une matrice bande.

$A_1=A, \quad A_{k+1}=L_kA_k= L_kL_{k-1} \cdots L_1A, \quad k=1,\cdots , n-1.$

Supposons que $A_k$ est une matrice bande i.e., $a_{ij}^k=0$ pour $|i-j|\geq p$ et montrons que $A_{k+1}$ est une matrice bande.
$$a_{ij}^{k+1}=a_{ij}^k-\frac{a_{ik}^k a_{kj}^k}{a_{kk}^k}$$
Soit $|i-j|\geq p\Leftrightarrow |(i-k)-(j-k)|\geq p$. On consid\`ere deux cas:
\begin{itemize}
\item $k+1\leq i\leq n$ et $k\leq j\leq n$. Alors $i-k\geq p$ ou $j-k\geq p$ $\Rightarrow a_{ik}^ka_{kj}^k=0\Rightarrow a_{ij}^{k+1}=a_{ij}^k=0$
\item $i\leq k$ ou $j\leq k-1$ alors $ a_{ij}^{k+1}=a_{ij}^k=0$
\end{itemize}
donc $A_{k+1}$ est une matrice bande et $U$ est une matrice bande. On a $A=LU$ et la matrice triangulaire inf\'erieure $L$ a pour \'el\'ements $l_{ij}=a_{ij}^j/a_{jj}^j,\quad j\leq i\leq n$. Toutes les matrices $ A_j$ \'etant des matrices bandes on a $a_{ij}^j=0$ pour $i-j\geq p\Rightarrow l_{ij}=0$ pour $i-j\geq p$.
}
}

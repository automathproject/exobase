\uuid{Wvdi}
\exo7id{5043}
\titre{exo7 5043}
\auteur{quercia}
\organisation{exo7}
\datecreate{2010-03-17}
\isIndication{false}
\isCorrection{true}
\chapitre{Courbes planes}
\sousChapitre{Courbes dans l'espace}
\module{Géométrie}
\niveau{L2}
\difficulte{}

\contenu{
\texte{
Soit $\mathcal{C}$ une courbe de l'espace, et $\Gamma$ la courbe décrite par le centre
de courbure, $I$, en un point $M$ de $\mathcal{C}$.
On suppose que la courbure de $\mathcal{C}$ est constante et sa torsion non nulle.
}
\begin{enumerate}
    \item \question{Montrer que la courbure de $\Gamma$ est aussi constante.}
\reponse{$\frac{d\vec I}{ds} = -\frac \tau c \vec B  \Rightarrow 
              \vec T_1 = \vec B$, $\frac {ds_1}{ds} = -\frac \tau c$,
              $\vec N_1 = -\vec N$, $c_1=c$.}
    \item \question{Chercher la torsion de $\Gamma$ en $I$ en fonction de la courbure et la
    torsion de $\mathcal{C}$ en $M$.}
\reponse{$\tau_1 = -\frac{c^2}{\tau}$.}
\end{enumerate}
}

\uuid{QQOK}
\exo7id{5831}
\titre{exo7 5831}
\auteur{rouget}
\organisation{exo7}
\datecreate{2010-10-16}
\isIndication{false}
\isCorrection{true}
\chapitre{Conique}
\sousChapitre{Quadrique}

\contenu{
\texte{
Pour quelles valeurs de $\lambda$ la surface $(\mathcal{S})$ d'équation $x(\lambda-y)+y(\lambda-z)+z(\lambda-x)-\lambda=0$ est-elle un cône du second degré ? En préciser alors le sommet et une directrice.
}
\reponse{
Une équation de $(\mathcal{S})$ est encore $xy+yz+zx-\lambda x-\lambda y-\lambda z+\lambda=0$.

La matrice de la forme quadratique $Q~:~(x,y,z)\mapsto xy+yz+zx$ dans la base $(i,j,k)$ est $A= \frac{1}{2}
\left(
\begin{array}{ccc}
0&1&1\\
1&0&1\\
1&1&0
\end{array}
\right)$ est les valeurs propres de cette matrice sont $- \frac{1}{2}$, $- \frac{1}{2}$ et $1$. Le rang de $Q$ est $3$ et sa signature est $(1,2)$. La surface $(\mathcal{S})$ est à priori soit un hyperboloïde, soit un cône du second degré. Donc $(\mathcal{S})$ est un cône du second degré si et seulement si son (unique) centre de symétrie qui est aussi l'unique point critique de la fonction $f~:~(x,y,z)\mapsto x(\lambda-y)+y(\lambda-z)+z(\lambda-x)-\lambda$ appartient à $(\mathcal{S})$.

\textbf{Point critique.}

\begin{center}
$\left\{
\begin{array}{l}
 \frac{\partial f}{\partial x}(x,y,z)=0\\
 \frac{\partial f}{\partial y}(x,y,z)=0\\
 \frac{\partial f}{\partial z}(x,y,z)=0
\end{array}
\right.
\Leftrightarrow \left\{
\begin{array}{l}
y+z=\lambda\\
z+x=\lambda\\
x+y=\lambda
\end{array}
\right.\Leftrightarrow x=y=z= \frac{\lambda}{2}$.
\end{center} 

On note alors $\Omega$ le point de coordonnées $\left( \frac{\lambda}{2}, \frac{\lambda}{2}, \frac{\lambda}{2}\right)$.

$(\mathcal{S})$ est un cône $\Leftrightarrow\Omega\in(\mathcal{S})\Leftrightarrow \frac{3\lambda^2}{4}-\lambda =0\Leftrightarrow\lambda\in\left\{0, \frac{4}{3}\right\}$.

\textbullet~Si $\lambda=0$, $(\mathcal{S})$ admet pour équation $xy+yz+zx = 0$.
Dans le repère $(O,X,Y,Z)$ où $X=/dfrac{1}{\sqrt{2}}(x-y)$, $Y = \frac{1}{\sqrt{6}}(x+y-2z)$ et $Z = \frac{1}{\sqrt{3}}(x+y+z)$, $(\mathcal{S})$ admet pour équation cartésienne $- \frac{1}{2}X^2-  \frac{1}{2}Y^2+ \frac{1}{2}Z^2 = 0$ ou encore $(\mathcal{S})$ est le cône de révolution de sommet 
$O$ et de section droite le cercle d'équations $\left\{
\begin{array}{l}
Z=1\\
X^2+Y^2+Z^2=3
\end{array}
\right.$ dans $(O,X,Y,Z)$ ou encore $\left\{
\begin{array}{l}
x+y+z=\sqrt{3}\\
x^2+y^2+z^2=3
\end{array}
\right.$ dans $(O,x,y,z)$.

Puisque $(\mathcal{S})$ est un cône de révolution de sommet $O$ et d'axe la droite d'équations $x=y=z$, il est plus interessant de fournir le demi angle au sommet $\theta$. Le point $A(1,1,1)$ est sur l'axe et le point $M(2,2-1)$ est sur le cône. Donc $\theta=\Arccos\left( \frac{\left|\overrightarrow{OA}.\overrightarrow{OM}\right|}{OA\times OM}\right)=\Arccos\left( \frac{3}{3\sqrt{3}}\right)=\Arccos\left( \frac{1}{\sqrt{3}}\right)$. 

\textbullet~Si $\lambda= \frac{4}{3}$, $(\mathcal{S})$ admet pour équation $xy+yz+zx- \frac{4}{3}(x+y+z) + \frac{4}{3}= 0$ dans $(O,i,j,k)$  ou encore 
$XY+XZ+YZ = 0$ dans $(\Omega,i,j,k)$ ce qui ramène au cas précédent.
}
}

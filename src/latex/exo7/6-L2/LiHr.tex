\uuid{LiHr}
\exo7id{4911}
\auteur{quercia}
\organisation{exo7}
\datecreate{2010-03-17}
\isIndication{false}
\isCorrection{true}
\chapitre{Conique}
\sousChapitre{Parabole}

\contenu{
\texte{
Pour $p>0$ on donne la courbe~$\Gamma$ d'équation $y^2=2px$.
Soit un carré $ABCD$ tel que $B,D\in \Gamma$ et $A,C$ appartiennent
à l'axe de symétrie de~$\Gamma$.
}
\begin{enumerate}
    \item \question{Quelle relation lie les abscisses de $A$ et~$C$~?}
\reponse{$x_C-x_A = 2p\pm\sqrt{4p^2+8pxA}$.}
    \item \question{On construit une suite $(M_n)$ de points de~$Ox$, $M_n$ d'abscisse~$x_n$,
    telle que $x_{n+1}>x_n$ et $M_nM_{n+1}$ est la diagonale
    d'un carré dont les deux autres sommets appartiennent à~$\Gamma$.
    Déterminer un équivalent de~$x_n$ quand~$n\to\infty$.}
\reponse{$x_{n+1} = x_n + \sqrt{8px_n+4p^2}+2p
                      = x_n\Bigl(1+\sqrt{\frac{8p}{x_n}}+ o\Bigl(\frac1{\sqrt{x_n}}\Bigr)\Bigr)$
    donc $\sqrt{x_{n+1}} = \sqrt{x_n} + \sqrt{2p}+ o(1)$ et $x_n\sim 2pn^2$.}
\end{enumerate}
}

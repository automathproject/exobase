\uuid{ZFIn}
\exo7id{7505}
\auteur{mourougane}
\organisation{exo7}
\datecreate{2021-08-10}
\isIndication{false}
\isCorrection{true}
\chapitre{Géométrie affine euclidienne}
\sousChapitre{Géométrie affine euclidienne du plan}

\contenu{
\texte{
On munit le plan affine euclidien $(P,<,>)$ d'un repère orthonormé
$(O,\vec{\imath},\vec{\jmath})$.
On considère les points $A$ de coordonnées $(-2,1)$ et $B$ de coordonnées $(4,4)$.
}
\begin{enumerate}
    \item \question{Déterminer les coordonnées du barycentre $G$ des points massiques $A(2)$ et $B(1)$.}
\reponse{Comme la somme des masses $2+1$ est non nulle, le barycentre est bien défini.
    L'abscisse de $G$ est $\frac{2\times (-2)+1\times 4}{2+1}=0$
    et l'ordonnée $\frac{2\times 1+1\times 4}{2+1}=2$.}
    \item \question{Calculer $2GA^ 2+GB^2$.}
\reponse{$2GA^ 2+GB^2=2((-2)^2+(1-2)^2)+(4^2+(4-2)^2)=30$.}
    \item \question{Démontrer que pour tout point $M$ du plan, on a $$2MA^2+MB^2=3MG^2 +2GA^ 2+GB^2.$$}
\reponse{Soit $M$ un point du plan $P$.
    \begin{eqnarray*}
        2MA^2+MB^2&=&2\vec{MA}^2+\vec{MB}^2=2(\vec{MG}+\vec{GA})^2+(\vec{MG}+\vec{GB})^2\\
        &=&3\vec{MG}^2+<\vec{MG},2\vec{GA}+\vec{GB}>+2\vec{GA}^ 2+\vec{GB}^2\\
        &=&3MG^2+2GA^ 2+GB^2.
    \end{eqnarray*}}
    \item \question{Déterminer l'ensemble $\mathcal L$ des points $M$ du plan tels que $2MA^2+MB^2=42$.}
\reponse{Soit $M$ un point du plan $P$.
    \begin{eqnarray*}
        M\in\mathcal{L}&\iff & 2MA^2+MB^2=42\iff 3MG^2+2GA^ 2+GB^2=42\\
        &\iff& 3MG^2+30=42\iff MG^2=4.
    \end{eqnarray*}
    L'ensemble $\mathcal L$ est donc le cercle de centre $G$ et de rayon $2$.}
\end{enumerate}
}

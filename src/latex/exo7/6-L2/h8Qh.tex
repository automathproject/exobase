\uuid{h8Qh}
\exo7id{5206}
\titre{exo7 5206}
\auteur{rouget}
\organisation{exo7}
\datecreate{2010-06-30}
\isIndication{false}
\isCorrection{true}
\chapitre{Géométrie affine dans le plan et dans l'espace}
\sousChapitre{Propriétés des triangles}
\module{Géométrie}
\niveau{L2}
\difficulte{}

\contenu{
\texte{
Montrer qu'il n'existe pas de triangle équilatéral dont les sommets appartiennent aux points d'intersection des lignes d'une feuille blanche quadrillée usuelle.
}
\reponse{
Il revient au même de démontrer que, si le plan est rapporté à un repère orthonormé, il n'existe pas de triangle équilatéral dont les sommets ont pour coordonnées des nombres entiers.

Le plan est muni d'un repère orthonormé direct. Soient $A$, $B$ et $C$ trois points deux à deux distincts, non alignés et à coordonnées entières. On sait que $\cos(\overrightarrow{AB},\overrightarrow{AC})=\frac{\overrightarrow{AB}.\overrightarrow{AC}}
{||\overrightarrow{AB}||.||\overrightarrow{AC}||}$ et $\sin(\overrightarrow{AB},\overrightarrow{AC})=\frac{\mbox{det}(\overrightarrow{AB},\overrightarrow{AC})}
{||\overrightarrow{AB}||.||\overrightarrow{AC}||}$. 

Par suite, ou bien le triangle $(ABC)$ est rectangle en $A$ (et n'est donc pas équilatéral), ou bien 

$\tan(\overrightarrow{AB},\overrightarrow{AC})=\frac{\mbox{det}(\overrightarrow{AB},\overrightarrow{AC})}
{\overrightarrow{AB}.\overrightarrow{AC}}$. Dans ce dernier cas, $\tan(\overrightarrow{AB},\overrightarrow{AC})$ est un quotient de deux nombres entiers, et est donc un rationnel. Malheureusement, pour un triangle équilatéral, la tangente de chacun de ses angles vaut $\sqrt{3}$ qui n'est pas un rationnel.

Quand le repère est orthonormé, il n'existe pas de triangle équilatéral dont les sommets sont à coordonnées entières.
}
}

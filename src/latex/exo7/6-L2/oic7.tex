\uuid{oic7}
\exo7id{7107}
\titre{exo7 7107}
\auteur{megy}
\organisation{exo7}
\datecreate{2017-01-21}
\isIndication{true}
\isCorrection{false}
\chapitre{Géométrie affine euclidienne}
\sousChapitre{Géométrie affine euclidienne du plan}
\module{Géométrie}
\niveau{L2}
\difficulte{}

\contenu{
\texte{
% source : Debart rotations aux collège
% rotations, collège
Soit $ABCD$ un carré de centre $O$, et $M$ un point sur $(AB)$. À partir de $M$, on construit le triangle isocèle $OMN$, rectangle en $O$. Montrer que les points $B$, $C$ et $N$ sont alignés.
}
\indication{Considérer la rotation de centre $O$ et d'angle $\pi/2$.}
}

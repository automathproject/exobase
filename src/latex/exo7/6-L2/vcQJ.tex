\uuid{vcQJ}
\exo7id{4950}
\titre{exo7 4950}
\auteur{quercia}
\organisation{exo7}
\datecreate{2010-03-17}
\isIndication{false}
\isCorrection{true}
\chapitre{Géométrie affine euclidienne}
\sousChapitre{Géométrie affine euclidienne du plan}

\contenu{
\texte{
Soit $ABC$ un triangle et $\mathcal{C}$ son cercle circonscrit.
Soit $M$ un point du plan de coordonnées barycentriques $(x,y,z)$ dans
le repère affine $(ABC)$.

Montrer que : $M\in\mathcal{C} \Leftrightarrow xAM^2 + yBM^2 + zCM^2 = 0
              \Leftrightarrow xyAB^2 + xzAC^2 + yzBC^2 = 0$.
}
\reponse{
$xAM^2 + yBM^2 + zCM^2 = r^2 - OM^2$ avec $\mathcal{C} = \mathcal{C}(O,r)$.\par
         $xyAB^2 + xzAC^2 + yzBC^2 = xAM^2 + yBM^2 + zCM^2$.
}
}

\uuid{cYG6}
\exo7id{6878}
\auteur{gammella}
\organisation{exo7}
\datecreate{2012-05-29}
\isIndication{false}
\isCorrection{true}
\chapitre{Analyse vectorielle}
\sousChapitre{Forme différentielle, champ de vecteurs, circulation}

\contenu{
\texte{
Calculer la circulation du champ vectoriel
$\vec{V}(x,y)=(3x,x+y)$ le long du cercle $C$ de centre $O$ et de rayon $1$, 
parcouru dans le sens direct.
}
\reponse{
Soit $\omega=3x dx + (x+y) dy$ la forme différentielle naturellement associée à $\vec{V}(x,y)$ et considérons $x=\cos t$ et $y=\sin t$ 
comme paramétrage du cercle de centre $O$ et de rayon $1$ (avec $t\in [0;2\pi]$). Il s'ensuit que la circulation $ \int_{C} \vec{V}.\vec{dl}$
n'est autre que :
$$ \int_{C} \vec{V}.\vec{dl}= \int_{C} w
=\int_0^{2\pi} (3\cos t (-\sin t) + (\cos t +\sin t) \cos t)dt.$$
Comme $\cos^2t= \frac{\cos(2t)+1}{2}$, on obtient :
$$ \int_{C} \vec{V}.\vec{dl}= \int_{0}^{2\pi} (-2\sin t\cos t + \frac{\cos (2t)+1}{2}) dt
= [\cos^2(t)+ \frac{1}{4}\sin(2t) +\frac{t}{2}]_0^{2\pi}=\pi.$$
Remarquons que si la forme $\omega$ avait été exacte, on aurait obtenu $ \int_{C} \vec{V}.\vec{dl}=0$ comme réponse, puisque l'intégrale curviligne d'une forme exacte sur une courbe fermée
est nulle.
}
}

\uuid{kopV}
\exo7id{2034}
\auteur{liousse}
\organisation{exo7}
\datecreate{2003-10-01}
\video{W0w1WraqsVc}
\isIndication{false}
\isCorrection{true}
\chapitre{Géométrie affine euclidienne}
\sousChapitre{Géométrie affine euclidienne de l'espace}

\contenu{
\texte{

}
\begin{enumerate}
    \item \question{On considère le point $A (-2,4,1) $, les vecteurs ${\buildrel\rightarrow \over {u}}
 (1,1,1), {\buildrel\rightarrow \over {v}} (2,2,-4) $, 
${\buildrel\rightarrow \over {w}} (3,-1,1)$
et le repère $(A, {\buildrel\rightarrow \over {u}} , 
{\buildrel\rightarrow \over {v}} , {\buildrel\rightarrow \over {w}}) $.
On note $x',y'$et $z'$ les coordonnées dans ce repère. Donner les formules analytiques du changement de 
repère exprimant $x,y,z$ en fonction de $x',y',z'$.}
\reponse{Notons $\mathcal{R}$ le repère initial $(0,\vec i,\vec j,\vec k)$.
Dire qu'un point $M$ du plan a pour coordonnées $(x,y,z)$ dans $\mathcal{R}$ 
signifie $\overrightarrow{OM} = x \vec i + y \vec j+ z\vec k$.

Si $\mathcal{R}'$ désigne un autre repère $(A, \vec u, \vec v,\vec w)$
alors le même point $M$ a pour coordonnées $(x',y',z')$ dans $\mathcal{R}'$ signifie
$\overrightarrow{AM} = x' \vec u + y'\vec v+z'\vec w$.

La formule de changement c'est simplement écrire les coordonnées de l'égalité
$\overrightarrow{OM} = \overrightarrow{OA}+\overrightarrow{AM}$.


$$\begin{pmatrix} x \\ y \\ z \end{pmatrix}
= \begin{pmatrix} -2 \\ 4 \\ 1 \end{pmatrix} + x'\vec u + y'\vec v+z'\vec w 
$$
Mais on connaît les coordonnées de 
$\vec u, \vec v,\vec w$ dans $\mathcal{R}$ :
$$\begin{pmatrix} x \\ y \\ z \end{pmatrix}
= \begin{pmatrix} -2 \\ 4 \\ 1 \end{pmatrix} +
 x' \begin{pmatrix} 1 \\ 1 \\ 1 \end{pmatrix}
+ y' \begin{pmatrix} 2 \\ 2 \\ -4 \end{pmatrix}
+ z' \begin{pmatrix} 3 \\ -1 \\ 1 \end{pmatrix}$$

D'où l'égalité de changement de repère :
$$(\mathcal{S}) \qquad \left\lbrace
\begin{array}{l}
 x = -2 + x' + 2y' + 3z' \\  
 y = 4 + x' + 2y' - z' \\ 
 z = 1 + x'- 4y' + z' \\
\end{array}
\right.$$}
    \item \question{On considère la droite $(D):\left\{\begin{array}{l} y-z=3 \\ x+y=2 \end{array}\right.$. 
Utiliser le changement de repère pour donner une équation de $D$ dans le repère
$(A, {\buildrel\rightarrow \over {u}} , 
{\buildrel\rightarrow \over {v}} , {\buildrel\rightarrow \over {w}}) $.}
\reponse{Dans l'équation de la droite $(D)$ 
$\left\{\begin{array}{l} y-z=3 \\ x+y=2 \end{array}\right.$ dans le repère $\mathcal{R}$
on remplace $x,y,z$ par la formule $(\mathcal{S})$  obtenue à la question précédente.

On obtient :
$$\left\{\begin{array}{l} \big(4 + x' + 2y' - z'\big)-
\big(1 + x'- 4y' + z'\big)=3 \\ 
\big(-2 + x' + 2y' + 3z'\big)+
\big(4 + x' + 2y' - z'\big) = 2 \end{array}\right..$$
Ce qui donne une équation de $(D)$ dans le repère $\mathcal{R}'$ :
$$\left\{\begin{array}{l} 
6y'-2z'=0   \\ 
2x'+4y'+2z'=0 \end{array}\right.
\quad \text{ou encore} \quad 
\left\{\begin{array}{l} 
3y'-z'=0   \\ 
x'+2y'+z'=0 \end{array}\right.$$
En particulier en faisant $(x',y',z')=(0,0,0)$ on remarque que cette droite passe par $A$.}
    \item \question{Donner les formules analytiques du changement de repère inverse.}
\reponse{Nous avions obtenu l'égalité $(\mathcal{S})$ de changement de repère de $\mathcal{R}'$ 
vers $\mathcal{R}$ qui s'écrit :
$$\left\lbrace
\begin{array}{l}
 x + 2 =  x' + 2y' + 3z' \\  
 y - 4 =  x' + 2y' - z' \\ 
 z - 1 =  x'- 4y' + z' \\
\end{array}
\right. 
\quad \implies \quad 
\left\lbrace
\begin{array}{l}
 X =  x' + 2y' + 3z' \\  
 Y =  x' + 2y' - z' \\ 
 Z =  x'- 4y' + z' \\
\end{array}
\right. $$
Où l'on a noté $X=x+2$, $Y=y-4$, $Z=z-1$. On inverse le système 
par la méthode de Gauss pour obtenir après calculs :
$$\left\lbrace
\begin{array}{l}
 x' =  \frac{1}{12}\big(X+7Y+4Z\big) \\  
 y' =  \frac{1}{12}\big(X+Y-2Z\big) \\ 
 z' =  \frac{1}{12}\big(3X-3Y\big) \\
\end{array}
\right. $$

Donc 
$$\left\lbrace
\begin{array}{l}
 x' =  \frac{1}{12}\big(x+7y+4z -30\big) \\  
 y' =  \frac{1}{12}\big(x+y-2z\big) \\ 
 z' =  \frac{1}{12}\big(3x-3y +18\big) \\
\end{array}
\right. $$

\bigskip

Avec les matrices cela se fait ainsi :
le système $(\mathcal{S})$ devient
$$\begin{pmatrix} x+2 \\ y-4 \\ z-1 \end{pmatrix}= M \begin{pmatrix} x' \\ y' \\ z' \end{pmatrix}
\quad \text{ où } \quad 
M = \begin{pmatrix}
     1 & 2 & 3 \\ 1 & 2 & -1 \\ 1 & -4 & 1 \\
    \end{pmatrix}.
$$
Ainsi
$$\begin{pmatrix} x' \\ y' \\ z' \end{pmatrix}= M^{-1} \begin{pmatrix} x+2 \\ y-4 \\ z-1 \end{pmatrix}
\quad \text{ où } \quad 
M^{-1} = \frac{1}{12}\begin{pmatrix}
     1 & 7 & 4 \\ 1 & 1 & -2 \\ 3 & -3 & 0 \\
    \end{pmatrix}.
$$}
\end{enumerate}
}

\uuid{ZDhj}
\exo7id{5835}
\titre{exo7 5835}
\auteur{rouget}
\organisation{exo7}
\datecreate{2010-10-16}
\isIndication{false}
\isCorrection{true}
\chapitre{Conique}
\sousChapitre{Quadrique}
\module{Géométrie}
\niveau{L2}
\difficulte{}

\contenu{
\texte{
Equation cartésienne du cylindre de révolution $(\mathcal{C})$ de rayon $R$ et d'axe $(\mathcal{D})$ d'équations $\left\{
\begin{array}{l}
x=z+2\\
y=z+1
\end{array}
\right.$. Déterminer $R$ pour que la droite $(Oz)$ soit tangente au cylindre.
}
\reponse{
Un repère de $(\mathcal{D})$ est $\left(A,\overrightarrow{u}\right)$ où $A(2,1,0)$ et $\overrightarrow{u}(1,1,1)$.

\begin{align*}\ensuremath
M\in(\mathcal{C})&\Leftrightarrow d(M,(\mathcal{D})) =R\Leftrightarrow\|\overrightarrow{AM}\wedge\overrightarrow{u}\|^2 = R^2\|\overrightarrow{u}\|^2\Leftrightarrow\|(x-2,y-1,z)\wedge(1,1,1)\|^2 = R^2\|(1,1,1)\|^2\\
 &\Leftrightarrow (y-z-1)^2 + (x-z-2)^2 + (x-y-1)^2 = 3R^2 \\
 &\Leftrightarrow3x^2+3y^2+3z^2-2xy-2xz-2yz-6x+6z+6-3R^2 = 0.
\end{align*}

La droite $(Oz)$ est tangente à $(\mathcal{C})$ si et seulement si $d((Oz),(\mathcal{D})) = R$.

\begin{center}
$(Oz)$ est tangente à $(\mathcal{C})\Leftrightarrow \frac{\left[\overrightarrow{OA},\overrightarrow{k},\overrightarrow{u}\right]^2}{\left\|\overrightarrow{k}\wedge\overrightarrow{u}\right\|^2}= R^2\Leftrightarrow\left|\begin{array}{ccc}
2&0&1\\
1&0&1\\
0&1&1
\end{array}
\right|^2=\|(-1,1,0\|^2\Leftrightarrow 1 = 2R^2\Leftrightarrow R = \frac{1}{\sqrt{2}}$.
\end{center}
}
}

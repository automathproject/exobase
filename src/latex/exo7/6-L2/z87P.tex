\uuid{z87P}
\exo7id{2490}
\titre{exo7 2490}
\auteur{matexo1}
\organisation{exo7}
\datecreate{2002-02-01}
\isIndication{false}
\isCorrection{false}
\chapitre{Analyse vectorielle}
\sousChapitre{Forme différentielle, champ de vecteurs, circulation}

\contenu{
\texte{
Soit $C$ une courbe ferm\'ee du plan enclosant une aire $S$, et $a$,
$b$ deux r\'eels.
}
\begin{enumerate}
    \item \question{Calculer $\oint_C ay\,dx+bx\,dy$.
En d\'eduire que 
$$S = \oint_C x\,dy = -\oint_C y\, dx = {1\over2}\oint_C x\,dy-y\,dx. $$}
    \item \question{En utilisant la formule pr\'ec\'edente, calculer l'aire comprise
entre l'axe $Ox$ et l'arche de la cyclo\"{\i}de d'\'equations 
$x = t-\sin t, y = 1-\cos t$, avec $0\leq t\leq 2\pi$.}
    \item \question{De m\^eme, trouver l'aire
int\'erieure \`a la boucle du folium de Descartes d'\'equation
$x^3+y^3=3xy$, qui est comprise dans le quadrant $x>0$, $y>0$ (on
pourra chercher une repr\'esentation param\'etrique de la boucle du
folium en posant $y = tx$).}
\end{enumerate}
}

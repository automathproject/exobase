\uuid{gVJP}
\exo7id{2483}
\auteur{matexo1}
\organisation{exo7}
\datecreate{2002-02-01}
\isIndication{false}
\isCorrection{false}
\chapitre{Analyse vectorielle}
\sousChapitre{Forme différentielle, champ de vecteurs, circulation}

\contenu{
\texte{
Montrer que si $\Sigma $ est une surface ferm\'ee de $\R^3$ et ${\bf U}$
un champ de vecteurs $C^1$ sur $\Sigma $, alors $\int_\Sigma  \operatorname{rot}{\bf
U}\cdot{\bf n} \,d\sigma  = 0$. En d\'eduire la valeur de $\int_S \operatorname{rot}{\bf
U}\cdot{\bf n} \,ds$, o\`u ${\bf U} = (-y^3, x^3+z, z^3)$ et $S$
l'h\'emisph\`ere $z>0$ de la sph\`ere unit\'e.
}
}

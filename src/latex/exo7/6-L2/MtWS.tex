\uuid{MtWS}
\exo7id{2485}
\titre{exo7 2485}
\auteur{matexo1}
\organisation{exo7}
\datecreate{2002-02-01}
\isIndication{false}
\isCorrection{false}
\chapitre{Analyse vectorielle}
\sousChapitre{Forme différentielle, champ de vecteurs, circulation}
\module{Géométrie}
\niveau{L2}
\difficulte{}

\contenu{
\texte{
Sous les conditions du th\'eor\`eme de Stokes, montrer les identit\'es
suivantes, o\`u $\bf V$ est un champ de vecteur arbitraire,  $\Sigma$ une
surface de bord $\Gamma$, et $\phi $ et $\psi $ des fonctions $C^1$\,:
\begin{eqnarray*}
\oint \phi  \,d{\bf r} 
&=& \int_\Sigma  d{{\bf\sigma }}\wedge\nabla  \varphi,  \\
\oint_{\Gamma } \,d{\bf r}\wedge{\bf U} 
&=& \int_\Sigma  \left({\bf n}\wedge\nabla \right)\wedge {\bf U} \,d\sigma, \\
\oint_{\Gamma } \varphi  \nabla \psi  \cdot d{\bf r} 
&=& \int_\Sigma  \nabla \phi \wedge\nabla \psi  \cdot{\bf n} \,d\sigma.
\end{eqnarray*}
}
}

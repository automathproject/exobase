\uuid{t1t3}
\exo7id{1964}
\titre{exo7 1964}
\auteur{liousse}
\organisation{exo7}
\datecreate{2003-10-01}
\isIndication{false}
\isCorrection{false}
\chapitre{Géométrie affine dans le plan et dans l'espace}
\sousChapitre{Géométrie affine dans le plan et dans l'espace}
\module{Géométrie}
\niveau{L2}
\difficulte{}

\contenu{
\texte{
Dans le triangle $ABC$, on consid\`ere trois points $P, Q, R$, sur 
les droites $(BC), (AC)$ et $(AB)$ respectivement, ces points n'\'etant pas 
les points $A,B$ ou $C$.
Montrer que les droites $(AP),$ $(BQ)$ et $(CR)$ sont 
concourantes ou parall\`eles si et seulement si
$$ {\overline {PB} \over \overline {PC}}.{\overline {QC}\over \overline {QA}}.
{\overline {RA} \over \overline {RB}}=-1$$
}
}

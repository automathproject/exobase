\uuid{ntZ8}
\exo7id{5048}
\titre{exo7 5048}
\auteur{quercia}
\organisation{exo7}
\datecreate{2010-03-17}
\isIndication{false}
\isCorrection{true}
\chapitre{Surfaces}
\sousChapitre{Surfaces paramétrées}
\module{Géométrie}
\niveau{L2}
\difficulte{}

\contenu{
\texte{
Soit la courbe d'équations dans $\R^3$ :
$$(\Gamma)\quad \left\{\begin{array}{cc} x^2-y^2-4x+2&=0 \cr x+z&=1.\cr\end{array}\right.$$\par
Déterminer la surface engendrée par la rotation de $(\Gamma)$ autour de $Oz$.
}
\reponse{
$(\Gamma)$ est l'intersection d'un cylindre hyperbolique et d'un plan.
         C'est une hyperbole dans ce plan.\par
         Pour $M(x,y,z) \in \Gamma$, on
         pose $r=\sqrt{x^2+y^2}$ et on élimine $x$ et $y$ entre les équations :
         $$\begin{cases}x^2+y^2=r^2\cr
                  x^2-y^2-4x+2=0\cr
                  x+z=1.\cr\end{cases}$$
         ce qui donne $2z^2=r^2$, donc $\Gamma$ est incluse dans l'hyperboloïde
         de révolution d'équation $2z^2=x^2+y^2$ et la surface cherchée itou.
	 La réciproque est évidente.
}
}

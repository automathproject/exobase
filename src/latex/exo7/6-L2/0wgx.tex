\uuid{0wgx}
\exo7id{7285}
\titre{exo7 7285}
\auteur{mourougane}
\organisation{exo7}
\datecreate{2021-08-10}
\isIndication{false}
\isCorrection{false}
\chapitre{Géométrie affine euclidienne}
\sousChapitre{Géométrie affine euclidienne du plan}

\contenu{
\texte{

}
\begin{enumerate}
    \item \question{Si \( \Omega\) est un point et si \( \lambda\) est un réel non nul, 
l'\emph{inversion} de centre \( \Omega\) et de rapport \( \lambda\) est l'application 
du plan privé de \(P\) dans lui-même qui au point \(P\) associe l'unique 
point \(P' \in ( \Omega P)\) tel que \(\vec{ \Omega P} \cdot \vec{ \Omega P'} = \lambda\). Montrer que 
les inversions sont involutives, c'est-à-dire que ce sont des bijections 
qui sont leur propre bijection réciproque.}
    \item \question{Montrer que l'ensemble des points fixes d'une inversion est un 
cercle.}
    \item \question{\label{q-inv-puissance} Montrer que si \(\mathcal{C}\) est un cercle 
tel que la puissance de \( \Omega\) par rapport à \(\mathcal{C}\) est \( \lambda\), 
et si \( \iota\)~est l'inversion de centre \( \Omega\) et de rapport \( \lambda\), alors 
\( \iota(\mathcal{C}) = \mathcal{C}\).}
    \item \question{Si \( \Delta\) est une droite qui ne passe pas par \( \Omega\), montrer que son 
image par l'inversion de centre \( \Omega\) et de rapport \( \lambda\) est un cercle 
passant par \( \Omega\) privé du point \( \Omega\). (Indication: noter \(H\) le 
projeté orthogonal de \( \Omega\) sur \( \Delta\), \( \iota\)~l'inversion; en appliquant 
une homothétie de centre \( \Omega\) bien choisie on peut supposer que 
\( \iota(H) = H\); montrer alors que \(P \in \Delta\) si et seulement si 
\(\vec{ \Omega \iota(P)} \cdot \vec{H \iota(P)} = 0\).)}
    \item \question{Montrer que l'image par une inversion d'un cercle ne passant pas 
par le centre de l'inversion est un cercle. (Indication: utiliser une 
homothétie de même centre que l'inversion pour se ramener à la 
question~\ref{q-inv-puissance}.)}
\end{enumerate}
}

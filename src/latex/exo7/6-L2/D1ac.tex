\uuid{D1ac}
\exo7id{7259}
\titre{exo7 7259}
\auteur{mourougane}
\organisation{exo7}
\datecreate{2021-08-10}
\isIndication{false}
\isCorrection{false}
\chapitre{Géométrie affine euclidienne}
\sousChapitre{Géométrie affine euclidienne du plan}

\contenu{
\texte{

}
\begin{enumerate}
    \item \question{On dit qu'un angle est inscrit dans un cercle si son sommet appartient à ce cercle. Démontrer le théorème des angles inscrits :\\

Deux angles de vecteurs inscrits dans un cercle interceptant le même arc de cercle sont de même mesure.}
    \item \question{Soient $\mathcal{C}_1$ et $\mathcal{C}_2$ deux cercles ayant deux points d'intersection $I$ et $J$. Soient $A$ et $M$ deux points distincts de $\mathcal{C}_1$ (et différents de $I$ et $J$). 
On note $B$ le point d'intersection de la droite $(AJ)$ avec $\mathcal{C}_2$ et $N$ le point d'intersection de la droite $(MJ)$ avec $\mathcal{C}_2$.\\

En considérant la somme des mesures des angles des triangles $AIB$ et $MIN$, montrer que $\text{Mes}\widehat{AIB}=\text{Mes}\widehat{MIN}$.}
\end{enumerate}
}

\uuid{EayP}
\exo7id{7160}
\titre{exo7 7160}
\auteur{megy}
\organisation{exo7}
\datecreate{2017-05-13}
\isIndication{false}
\isCorrection{true}
\chapitre{Géométrie affine euclidienne}
\sousChapitre{Géométrie affine euclidienne du plan}

\contenu{
\texte{
Soit $\mathcal T=ABC$ un triangle isocèle non équilatéral. Déterminer son groupe d'isométries ainsi qu'un isomorphisme entre ce groupe et un groupe classique.
}
\reponse{
Comme une isométrie est affine, elle conserve les barycentres. Soit $P$ un sommet du triangle. Comme ce n'est pas un barycentre d'autres points du triangle, son image par une isométrie fixant le triangle non plus, c'est-à-dire que son image est un sommet. On en déduit que les sommets sont envoyés sur les sommets.

Dans la suite on suppose que $ABC$ est isocèle en $A$. Soit $f$ une isométrie de $ABC$. Comme une isométrie conserve les angles non orientés et que le triangle est isocèle en $A$ et non équilatéral, on en déduit que $B$ et $C$ sont envoyés soit sur $B$ soit sur $C$. Le point $A$ est donc fixe.

Comme une application affine est déterminée par l'image de trois points non alignés, on conclut que le triangle $ABC$ n'admet que deux isométries : l'identité, et la réflexion $\sigma$ suivante la médiatrice de $[BC]$.

Le groupe ${\rm Isom}(\mathcal T)$ est donc de cardinal deux, donc isomorphe à $\Z/2\Z$.

(Un isomorphisme est donné par $\phi : \Z/2\Z \to {\rm Isom}(\mathcal T)$ l'application qui envoie $[0]=2\Z$ sur l'identité et $[1]=1+2\Z$ sur $\sigma$. C'est une bijection, et on vérifie que c'est un isomorphisme de groupes.)
}
}

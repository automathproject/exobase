\uuid{Tn0u}
\exo7id{2027}
\titre{exo7 2027}
\auteur{cousquer}
\organisation{exo7}
\datecreate{2003-10-01}
\isIndication{false}
\isCorrection{false}
\chapitre{Géométrie affine dans le plan et dans l'espace}
\sousChapitre{Géométrie affine dans le plan et dans l'espace}

\contenu{
\texte{
Soit deux plans 
$\left\{
\begin{array}{lrcl}
    \pi: & ux+vy+wz+h & = & 0  \\
    \pi': & u'x+v'y+w'z+h' & = & 0
\end{array}
\right.$.
}
\begin{enumerate}
    \item \question{Montrer que si $\pi$ et $\pi'$ sont sécants, tout plan passant
par leur droite d'intersection~$D$ a une équation du type
$$\lambda(ux+vy+wz+h)+\mu(u'x+v'y+w'z+h')=0$$
et réciproquement, tout plan ayant une équation de ce type, 
(pour un couple $(\lambda, \mu$) donné)
passe par~$D$.}
    \item \question{Si $\pi$ et $\pi'$ sont parallèles, que représente
l'ensemble des plans d'équation~:
$$\lambda(ux+vy+wz+h)+\mu(u'x+v'y+w'z+h')=0$$}
\end{enumerate}
}

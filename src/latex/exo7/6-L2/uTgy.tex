\uuid{uTgy}
\exo7id{5045}
\titre{exo7 5045}
\auteur{quercia}
\organisation{exo7}
\datecreate{2010-03-17}
\isIndication{false}
\isCorrection{true}
\chapitre{Courbes planes}
\sousChapitre{Courbes dans l'espace}

\contenu{
\texte{
Soit $s \mapsto M_s$ une courbe de l'espace de classe $\mathcal{C}^3$ paramétrée par une
abscisse curviligne.
Pour tout $s$ on choisit une normale à la courbe en $M_s$ : $\Delta_s$.
A quelle condition les droites $\Delta_s$ admettent-elles une enveloppe~?
}
\reponse{
Pt caractéristique : $P = M + a(s)\vec N + b(s)\vec B$ :
         CNS ${}\Leftrightarrow \begin{cases} a = \frac1c \cr
                             \frac{ab'-a'b}{a^2+b^2} = \Arctan(b/a)'= \tau.\cr \end{cases}$
Rmq : le point caractéristique se projette sur $I$.
}
}

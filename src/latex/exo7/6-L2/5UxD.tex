\uuid{5UxD}
\exo7id{7438}
\auteur{mourougane}
\organisation{exo7}
\datecreate{2021-08-10}
\isIndication{false}
\isCorrection{false}
\chapitre{Géométrie affine dans le plan et dans l'espace}
\sousChapitre{Géométrie affine dans le plan et dans l'espace}

\contenu{
\texte{
Dans l'espace affine $\Rr^3$ muni d'un repère affine
$A_0,A_1,A_2,A_3$, donner l'expression analytique
}
\begin{enumerate}
    \item \question{\begin{enumerate}}
    \item \question{de l'homothétie $h$ de centre 
$\displaystyle C\left(\begin{array}{c}
1\\2\\3\end{array}\right)$ et de rapport $4$.}
    \item \question{de la symétrie $s$ d'axe $(x+y+z=1)$ parallèlement à
 $\vec{A_0A_1}$.}
    \item \question{de l'affinité $a$ de base $(x+y+z=1)$ de rapport $3$
 parallèlement à  $\vec{A_0A_1}$.}
    \item \question{de la transvection $t$ de base $(x+y+z=1)$ qui envoie $A_0$ sur 
$\displaystyle \left(\begin{array}{c}
1\\1\\ -2\end{array}\right)$}
\end{enumerate}
}

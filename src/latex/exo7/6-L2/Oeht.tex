\uuid{Oeht}
\exo7id{2008}
\auteur{ridde}
\organisation{exo7}
\datecreate{1999-11-01}
\isIndication{false}
\isCorrection{false}
\chapitre{Géométrie affine dans le plan et dans l'espace}
\sousChapitre{Géométrie affine dans le plan et dans l'espace}

\contenu{
\texte{
Soit $ABCD$ un carr\'e direct et $M$ un point de la droite $ (DC)$. La perpendiculaire
\`a $ (AM)$ passant par $A$ coupe $ (BC)$ en $N$. On note I le milieu de $[MN]$.
D\'eterminer le lieu des points $I$ lorsque $M$ d\'ecrit la droite $ (DC)$.
}
}

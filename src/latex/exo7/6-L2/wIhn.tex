\uuid{wIhn}
\exo7id{7489}
\auteur{mourougane}
\organisation{exo7}
\datecreate{2021-08-10}
\isIndication{false}
\isCorrection{false}
\chapitre{Géométrie affine euclidienne}
\sousChapitre{Géométrie affine euclidienne du plan}

\contenu{
\texte{
Dans le plan affine $P$ réel, muni d'un repère affine $(A_0,A_1,A_2)$, 
on considère les points donnés en coordonnées barycentriques par $A(2,-1,5)$ et $B(1,1,2)$ et $C(2,3,0)$.
Déterminer les coordonnées barycentriques normalisées, du barycentre $G$ des points massiques $(A,1)$, $(B,2)$, $(C,-1)$.
}
}

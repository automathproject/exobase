\uuid{KFsN}
\exo7id{7138}
\titre{exo7 7138}
\auteur{megy}
\organisation{exo7}
\datecreate{2017-02-08}
\isIndication{false}
\isCorrection{false}
\chapitre{Géométrie affine euclidienne}
\sousChapitre{Géométrie affine euclidienne du plan}
\module{Géométrie}
\niveau{L2}
\difficulte{}

\contenu{
\texte{
On considère un carré $ABCD$ et on place quatre points $E$, $F$, $G$, et $H$ sur les côtés de ce carré (en-dehors des sommets). Puis, on efface le carré. L'objectif est de reconstruire le carré en utilisant le théorème de l'angle inscrit.

Si $A$ est le sommet entre $E$ et $F$, montrer que la diagonale du carré partant de $A$ passe par l'intersection du cercle de diamètre $[EF]$ avec la médiatrice de $[EF]$. % angle inscrit
En déduire une construction des diagonales du carré, puis du carré.
}
}

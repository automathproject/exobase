\uuid{tS9J}
\exo7id{4953}
\auteur{quercia}
\organisation{exo7}
\datecreate{2010-03-17}
\isIndication{false}
\isCorrection{false}
\chapitre{Géométrie affine euclidienne}
\sousChapitre{Géométrie affine euclidienne du plan}

\contenu{
\texte{
Soient $D,D'$ deux droites distinctes sécantes en $O$.

On note ${\cal H} = \{ M$ tq $d(M,D) = d(M,D') \}$.
}
\begin{enumerate}
    \item \question{Montrer que $\cal H$ est la réunion de deux droites perpendiculaires.
    (appelées bissectrices de $(D,D')$)}
    \item \question{Soit $s$ une symétrie orthogonale telle que $s(D) = D'$.
    Montrer que l'axe de $s$ est l'une des droites de $\cal H$}
    \item \question{Soit $\cal C$ un cercle du plan tangent à $D$.
    Montrer que $\cal C$ est tangent à $D$ et à $D'$
    si et seulement si son centre appartient à $\cal H$.}
\end{enumerate}
}

\uuid{P7Uk}
\exo7id{2045}
\auteur{liousse}
\organisation{exo7}
\datecreate{2003-10-01}
\isIndication{false}
\isCorrection{false}
\chapitre{Géométrie affine euclidienne}
\sousChapitre{Géométrie affine euclidienne de l'espace}

\contenu{
\texte{
Soient $A$, $B$ et $C$ trois points distincts et non align\'es de l'espace affine 
tridimensionnel $\mathcal E$.
On note $P$ le plan qui contient $A$, $B$ et $C$.
Soit $O$ un point de $\mathcal E$ n'appartenant pas \`a $P$.
}
\begin{enumerate}
    \item \question{\begin{enumerate}}
    \item \question{Expliquer rapidement pourquoi $\mathcal R = (O,\overrightarrow{OA},\overrightarrow{OB},
\overrightarrow{OC})$ est un rep\`ere  cart\'esien de $\mathcal E$.}
    \item \question{Dans ce rep\`ere $\mathcal R$, \'ecrire les coordonn\'ees des points $O$, $A$, $B$ et $C$, et d\'eterminer une
\'equation cart\'esienne du plan $P$.}
\end{enumerate}
}

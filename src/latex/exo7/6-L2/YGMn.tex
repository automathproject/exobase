\uuid{YGMn}
\exo7id{7085}
\titre{exo7 7085}
\auteur{megy}
\organisation{exo7}
\datecreate{2017-01-21}
\isIndication{true}
\isCorrection{true}
\chapitre{Géométrie affine euclidienne}
\sousChapitre{Géométrie affine euclidienne du plan}
\module{Géométrie}
\niveau{L2}
\difficulte{}

\contenu{
\texte{
% homothétie, Thalès
Soit $ABC$ un triangle. Construire un carré dont un sommet appartient à $[AB]$, un à $[AC]$ et deux sommets adjacents appartiennent à $[BC]$.
}
\indication{En faisant  une figure avec le carré déjà construit, on voit alors deux segments parallèles, ce qui invite à utiliser  une homothétie.}
\reponse{
Analyse. Traçons comme suggéré une figure avec le carré déjà construit : on trace un carré puis on trace un triangle adéquat autour. On constate qu'un des côtés du carré, notons-le $[IJ]$, est parallèle à $[BC]$. Il y a une homothétie $h$ de centre $A$ qui envoie $[IJ]$ sur $[BC]$. Alors, l'image du carré $IJKL$ par $h$ est un carré dont un des côtés est $[BC]$. Notons $BCDE$ ce carré et traçons-le. On constate que $h(K)=D$ et $h(L)=E$, c'est-à-dire $K=h^{-1}(D)$ et $L = h^{-1}(E)$. Il ne reste plus qu'à faire la synthèse.
}
}

\uuid{RrAz}
\exo7id{5516}
\titre{exo7 5516}
\auteur{rouget}
\organisation{exo7}
\datecreate{2010-07-15}
\isIndication{false}
\isCorrection{true}
\chapitre{Géométrie affine dans le plan et dans l'espace}
\sousChapitre{Géométrie affine dans le plan et dans l'espace}

\contenu{
\texte{
Montrer que les plans $(P_1)$~:~$z-2y=5$, $(P_2)$~:~$2x-3z=0$ et $(P_3)~:~3y-x=0$ admettent une parallèle commune. Ils définissent ainsi un prisme. Déterminer l'aire d'une section perpendiculaire.
}
\reponse{
$\overrightarrow{u}\in\overrightarrow{P_1}\cap\overrightarrow{P_2}\cap\overrightarrow{P_3}\Leftrightarrow\left\{
\begin{array}{l}
z-2y=0\\
2x-3z=0\\
3y-x=0
\end{array}
\right.\Leftrightarrow\left\{
\begin{array}{l}
x=3y\\
z=2y
\end{array}
\right.$.
Ainsi, les plans $(P_1)$, $(P_2)$ et $(P_3)$ sont tous trois parallèles à la droite affine $(D)$ d'équations $\left\{
\begin{array}{l}
x=3y\\
z=2y
\end{array}
\right.$. Ces plans définissent donc un prisme.
Déterminons alors l'aire d'une section droite. Le plan $(P)$ d'équation $3x+y+2z=0$ est perpendiculaire à la droite $(D)$. Son intersection avec les plans $(P_1)$, $(P_2)$ et $(P_3)$ définit donc une section droite du prisme.
\textbullet~Soit $M(x,y,z)$ un point de l'espace.

\begin{center}
$M\in(P_1)\cap(P_2)\cap(P)\Leftrightarrow
\left\{
\begin{array}{l}
z-2y=5\\
2x-3z=0\\
3x+y+2z=0
\end{array}
\right.\Leftrightarrow\left\{
\begin{array}{l}
y=\frac{z-5}{2}\\
x=\frac{3}{2}z\\
\frac{9}{2}z+\frac{z-5}{2}+2z=0
\end{array}
\right.\Leftrightarrow\left\{
\begin{array}{l}
z=\frac{5}{14}\\
y=-\frac{65}{28}\rule[-4mm]{0mm}{11mm}\\
x=\frac{15}{28}
\end{array}
\right.$
\end{center}
Notons $A\left(\frac{15}{28},-\frac{65}{28},\frac{5}{14}\right)$.
\textbullet~Soit $M(x,y,z)$ un point de l'espace.

\begin{center}
$M\in(P_1)\cap(P_3)\cap(P)\Leftrightarrow
\left\{
\begin{array}{l}
z-2y=5\\
3y-x=0\\
3x+y+2z=0
\end{array}
\right.\Leftrightarrow\left\{
\begin{array}{l}
z=2y+5\\
x=3y\\
9y+y+2(2y+5)=0
\end{array}
\right.\Leftrightarrow\left\{
\begin{array}{l}
y=-\frac{5}{7}\\
x=-\frac{15}{7}\rule[-4mm]{0mm}{11mm}\\
z=\frac{25}{7}
\end{array}
\right.$
\end{center}
Notons $B\left(-\frac{15}{7},-\frac{5}{7},\frac{25}{7}\right)$.

\textbullet~Soit $M(x,y,z)$ un point de l'espace.

\begin{center}
$M\in(P_2)\cap(P_3)\cap(P)\Leftrightarrow
\left\{
\begin{array}{l}
2x-3z=0\\
3y-x=0\\
3x+y+2z=0
\end{array}
\right.\Leftrightarrow x=y=z=0$
\end{center}
Une section droite est $OAB$ où $A\left(\frac{15}{28},-\frac{65}{28},\frac{5}{14}\right)$ et $B\left(-\frac{15}{7},-\frac{5}{7},\frac{25}{7}\right)$. De plus

\begin{align*}\ensuremath
\text{aire de}(OAB)&=\frac{1}{2}\left\|\overrightarrow{OA}\wedge\overrightarrow{OB}\right\|=\frac{1}{2}\times\frac{5}{28}\times\frac{5}{7}\left\|\left(
\begin{array}{c}
3\\
-13\\
2
\end{array}\right)
\wedge
\left(
\begin{array}{c}
-3\\
-1\\
5
\end{array}\right)\right\|=\frac{1}{2}\times\frac{5}{28}\times\frac{5}{7}\sqrt{63^2+21^2+42^2}\\
 &=\frac{1}{2}\times\frac{5}{28}\times\frac{5}{7}\times21\sqrt{3^2+1^2+2^2}=\frac{75}{4\sqrt{14}}
\end{align*}
\begin{center}
\shadowbox{
L'aire d'une section droite est $\frac{75}{4\sqrt{14}}$.
}
\end{center}
}
}

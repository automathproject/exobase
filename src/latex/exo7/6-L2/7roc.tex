\uuid{7roc}
\exo7id{5838}
\titre{exo7 5838}
\auteur{rouget}
\organisation{exo7}
\datecreate{2010-10-16}
\isIndication{false}
\isCorrection{true}
\chapitre{Conique}
\sousChapitre{Quadrique}

\contenu{
\texte{

}
\begin{enumerate}
    \item \question{Equation du cylindre de révolution $(\mathcal{C})$ d'axe la droite d'équations $x=y+1=3z-6$ et de rayon $3$.}
\reponse{Un repère de $(\mathcal{D})$ est $\left(A,\overrightarrow{u}\right)$ où $A(0,-1,2)$ et $\overrightarrow{u}(3,3,1)$.

\begin{align*}\ensuremath
M(x,y,z)\in(\mathcal{C})&\Leftrightarrow d(M,(\mathcal{D}))=3\Leftrightarrow \left\|\overrightarrow{AM}\wedge\overrightarrow{u}\right\|^2=9\|\overrightarrow{u}\|^2\\
 &\Leftrightarrow\|(x,y+1,z-2)\wedge(3,3,1)\|^2=9\times19\Leftrightarrow(y-3z+7)^2+(x-3z+6)^2+9(x-y-1)^2=171.
\end{align*}}
    \item \question{Equation du cône de révolution $(\mathcal{C})$ d'axe la droite d'équations $x=y+1=3z-6$, de sommet $S(0,-1,2)$ et de demi-angle au sommet $ \frac{\pi}{3}$.}
\reponse{Un repère de $(\mathcal{D})$ est $\left(A,\overrightarrow{u}\right)$ où $A(0,-1,2)$ et $\overrightarrow{u}(3,3,1)$. De plus, $S=A$.

\begin{align*}\ensuremath
M(x,y,z)\in(\mathcal{C})&\Leftrightarrow M=A\;\text{ou}\;M\neq A\;\text{et}\; \frac{\left|\overrightarrow{AM}.\overrightarrow{u}\right|}{AM\times\|\overrightarrow{u}\|}=\cos\left( \frac{\pi}{3}\right)\Leftrightarrow \left(\overrightarrow{AM}.\overrightarrow{u}\right)^2= \frac{1}{4}AM^2\|\overrightarrow{u}\|^2\\
 &\Leftrightarrow4(3x+3(y+1)+(z-2))^2=19(x^2+(y+1)^2+(z-2)^2)\\
 &\Leftrightarrow4(3x+3y+z+1)^2-19(x^2+(y+1)^2+(z-2)^2)=0\\
 &\Leftrightarrow17x^2+17y^2-15z^2+72xy+24xz+24yz+24x-14y+84z-91=0.
\end{align*}}
\end{enumerate}
}

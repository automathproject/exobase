\uuid{feRh}
\exo7id{7079}
\titre{exo7 7079}
\auteur{megy}
\organisation{exo7}
\datecreate{2017-01-21}
\isIndication{false}
\isCorrection{false}
\chapitre{Géométrie affine euclidienne}
\sousChapitre{Géométrie affine euclidienne du plan}

\contenu{
\texte{

}
\begin{enumerate}
    \item \question{À quelle condition sur quatre points $P_1, ... P_4$ existe-t-il une homothétie $h$ telle que $h(P_1)=P_2$ et $h(P_3)=P_4$ ?}
    \item \question{Soit $\phi$ une homothétie. On donne deux points $A$ et $B$, ainsi que leurs images $\phi(A)$ et $\phi(B)$. Le centre de l'homothétie n'est pas donné. Le construire, y compris si les quatre points donnés sont alignés.
% dans le deuxième cas, tracer un triangle ABC, et tracer son image par l'homthétie en tracant les parallèles. On trouve un point C', et CC' intersecte la droite (AB) en O.}
    \item \question{Soit $\rho$ une rotation du plan. On donne deux points $A$ et $B$, ainsi que leurs images $\rho(A)$ et $\rho(B)$. Construire le centre de la rotation, en distinguant les cas.}
\end{enumerate}
}

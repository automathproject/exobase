\uuid{BLUF}
\exo7id{7260}
\titre{exo7 7260}
\auteur{mourougane}
\organisation{exo7}
\datecreate{2021-08-10}
\isIndication{false}
\isCorrection{false}
\chapitre{Géométrie affine euclidienne}
\sousChapitre{Géométrie affine euclidienne du plan}
\module{Géométrie}
\niveau{L2}
\difficulte{}

\contenu{
\texte{
Soit $E$ un plan euclidien orienté, muni d'un repère $(O,\vec{\imath},\vec{\jmath})$ orthonormé direct.
}
\begin{enumerate}
    \item \question{Soit $n$ un entier naturel supérieur à $3$.
Exprimer à l'aide des fonctions trigonométriques $\cos$ et $\sin$,
 le périmètre $p_n$ d'un polygone régulier à $n$ côtés inscrit dans le cercle trigonométrique
(c'est à dire le cercle de centre $0$ et de rayon $1$.)}
    \item \question{On rappelle que pour tout $\theta\in ]0,\pi/2[$, $$\theta\cos\theta\leq \sin\theta\leq \theta.$$
Montrer que la suite $(p_n)_{n\in\mathbb{N}}$ admet une limite et déterminer cette limite.}
\end{enumerate}
}

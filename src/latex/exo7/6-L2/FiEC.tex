\uuid{FiEC}
\exo7id{4882}
\titre{exo7 4882}
\auteur{quercia}
\organisation{exo7}
\datecreate{2010-03-17}
\isIndication{false}
\isCorrection{false}
\chapitre{Géométrie affine dans le plan et dans l'espace}
\sousChapitre{Applications affines}
\module{Géométrie}
\niveau{L2}
\difficulte{}

\contenu{
\texte{
Dans l'espace, on considère un point $O$ et un plan $\cal P$ ne passant pas
par $O$. On définit l'application $f$: $M \longmapsto M'$ où $M'$ est le
point intersection de $\cal P$ et $(OM)$.
(Projection stéréographique sur $\cal P$ de pôle $O$)
}
\begin{enumerate}
    \item \question{Est-ce que $f$ est affine ?}
    \item \question{Etudier l'image par $f$ d'une droite, d'un plan, d'une partie convexe.}
\end{enumerate}
}

\uuid{ya0C}
\exo7id{7444}
\titre{exo7 7444}
\auteur{mourougane}
\organisation{exo7}
\datecreate{2021-08-10}
\isIndication{false}
\isCorrection{false}
\chapitre{Géométrie affine dans le plan et dans l'espace}
\sousChapitre{Géométrie affine dans le plan et dans l'espace}
\module{Géométrie}
\niveau{L2}
\difficulte{}

\contenu{
\texte{
On  considère $\Rr^{4}$  muni de  son  produit scalaire
usuel. Soit $F$ le sous espace engendré par $v_{1}=(1,1,0,0)$,
$v_{2}=(0,1,-1,1)$. Déterminer une base orthonormée de $F$ et la
compléter pour obtenir une base orthonormée de $\Rr^{4}$.
}
}

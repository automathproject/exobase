\uuid{i8C5}
\exo7id{7479}
\titre{exo7 7479}
\auteur{mourougane}
\organisation{exo7}
\datecreate{2021-08-10}
\isIndication{false}
\isCorrection{false}
\chapitre{Géométrie affine euclidienne}
\sousChapitre{Géométrie affine euclidienne de l'espace}
\module{Géométrie}
\niveau{L2}
\difficulte{}

\contenu{
\texte{
Soient $E$ un espace affine euclidien de dimension 3 et $\mathcal{R}$ 
un repère cartésien orthonormé
de $E$. Soit
$n$ un entier $\geq 3$. On considère l'ensemble $X$ des points de $E$ 
dont les coordonnées $(x,y,z)$ dans
$\mathcal{R}$ satisfont aux deux conditions suivantes :
}
\begin{enumerate}
    \item \question{Les nombres $x$, $y$ et $z$ sont dans $\Zz$ ;}
    \item \question{Le nombre $x+y+z$ est divisible par $n$.}
\end{enumerate}
}

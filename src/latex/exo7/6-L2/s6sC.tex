\uuid{s6sC}
\exo7id{7130}
\titre{exo7 7130}
\auteur{megy}
\organisation{exo7}
\datecreate{2017-02-08}
\isIndication{false}
\isCorrection{true}
\chapitre{Géométrie affine euclidienne}
\sousChapitre{Géométrie affine euclidienne du plan}
\module{Géométrie}
\niveau{L2}
\difficulte{}

\contenu{
\texte{
% nom :  centre similitude
% source : ? voir ausi Audin, p.91-93
On donne deux segments $[AB]$ et $[CD]$ non parallèles et de longueur différente. On admet qu'il existe une similitude directe $\phi$ envoyant $A$ sur $C$ et $B$ sur $D$. Le but de l'exercice est de construire le centre $O$ de cette similitude.
}
\begin{enumerate}
    \item \question{Montrer que l'angle de la similitude est $\widehat{(\overrightarrow{AB},\overrightarrow{CD}}$.}
\reponse{Soit $\theta$ l'angle de la similitude. La partie linéaire $\overrightarrow{\phi}$ de $\phi$ est une similitude vectorielle d'angle $\theta$. On a donc $\theta =  \widehat{(\overrightarrow{AB},\overrightarrow \phi\overrightarrow{AB}}= \widehat{(\overrightarrow{AB},\overrightarrow{CD}}$.}
    \item \question{On note $Q = (AB) \cap (CD)$. Montrer que $AQCO$ est inscriptible.}
\reponse{On a $\theta = \widehat{(\overrightarrow{AB},\overrightarrow{CD}} = \widehat{(\overrightarrow{AQ},\overrightarrow{CQ}}$ et d'autre part $\theta:\widehat{(\overrightarrow{OA},\overrightarrow{O \phi(A)}}=\widehat{(\overrightarrow{OA},\overrightarrow{OC}}$. Donc $AQCO$ est inscriptible.}
    \item \question{Terminer le raisonnement et construire $O$.}
\reponse{De même $BQDO$ est inscriptible. Donc $O$ appartient à l'intersection des cercles circonscrits à $ACQ$ et $BDQ$.}
    \item \question{Que faire si les segments sont parallèles ? De même longueur ? Réfléchir à d'autres méthodes pour construire le centre, sans utiliser de cercles.}
\reponse{Si les segments sont parallèles, la similitude est une homothétie. S'ils sont de plus de même longueur, c'est une translation ou une symétrie centrale.}
\end{enumerate}
}

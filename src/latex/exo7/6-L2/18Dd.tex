\uuid{18Dd}
\exo7id{7455}
\titre{exo7 7455}
\auteur{mourougane}
\organisation{exo7}
\datecreate{2021-08-10}
\isIndication{false}
\isCorrection{false}
\chapitre{Géométrie affine dans le plan et dans l'espace}
\sousChapitre{Géométrie affine dans le plan et dans l'espace}

\contenu{
\texte{
Le but de cet exercice est de 
démontrer que le symétrique de l'orthocentre $H$ d'un triangle non plat $ABC$ par rapport à un des cotés (par exemple $(AC)$) est sur le cercle circonscrit.

Soit $ABC$ un triangle non plat. La hauteur issue de $A$ coupe $(BC)$ en $A'$ et la hauteur issue de $C$ coupe $(AB)$ en $C'$. Démontrer que les points $A',B,C'$ et $H$ sont cocycliques. Démontrer que les angles de droites $((BC'),(BA'))$ et $((HC'),(HA'))$ sont égaux.
Conclure.
}
}

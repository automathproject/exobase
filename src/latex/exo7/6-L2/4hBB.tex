\uuid{4hBB}
\exo7id{4978}
\auteur{quercia}
\organisation{exo7}
\datecreate{2010-03-17}
\isIndication{false}
\isCorrection{true}
\chapitre{Géométrie affine euclidienne}
\sousChapitre{Géométrie affine euclidienne de l'espace}

\contenu{
\texte{
Soit $ABCD$ un tétraèdre régulier, et $d_{AB},d_{AC},d_{AD}$ les $\frac12$-tours
autour des droites $(AB),(AC),(AD)$.
Simplifier $f = d_{AB} \circ d_{AC} \circ d_{AD}$.
}
\reponse{
$d_{AD}(D) = D,\quad d_{AC}(D) = D'$ tq $\vec{AD'} = \vec{DC}$. \par
La droite $(AD')$ est parallèle à $(DC)$ qui est perpendiculaire à $(AB)$.

Donc $f(D) = d_{AB}(D') = D''$ tq $\vec{AD''} = \vec{CD}$.

$d_{AD}(D'') = C,\quad d_{AC}(C) = C,\quad f(D'') = d_{AB}(C) = C'$
tq $\vec{AC'} = \vec{CB}$.

$d_{AD}(C') = C''$ tq $\vec{AC''} = \vec{BC},\quad d_{AC}(C'') = B,\quad
 f(C') = d_{AB}(B) = B$.

$d_{AD}(B) = B'$ tq $\vec{AB'} = \vec{BD},\quad d_{AC}(B') = B''$
tq $\vec{AB''} = \vec{DB},\quad f(B) = d_{AB}(B'') = D$.

Soit $E$ le symétrique de $C$ par rapport à $(BD)$ : $f$ est le
$\frac14$-tour autour de $(AE)$ envoyant $D$ sur $B$.
}
}

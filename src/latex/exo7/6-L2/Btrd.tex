\uuid{Btrd}
\exo7id{6875}
\titre{exo7 6875}
\auteur{gammella}
\organisation{exo7}
\datecreate{2012-05-29}
\isIndication{false}
\isCorrection{true}
\chapitre{Analyse vectorielle}
\sousChapitre{Forme différentielle, champ de vecteurs, circulation}
\module{Géométrie}
\niveau{L2}
\difficulte{}

\contenu{
\texte{
On considère la forme différentielle
$\omega=(x^2+y^2+2x) dx+2y dy$.
}
\begin{enumerate}
    \item \question{Montrer que $\omega$ n'est pas exacte.}
\reponse{Posons $P(x,y)=x^2+y^2+2x$ et $Q(x,y)=2y$. On voit facilement
que $ \frac{\partial P}{\partial y}\not=\frac{\partial Q}{\partial x}$. La forme
$\omega$ n'est donc pas exacte.}
    \item \question{Trouver une fonction $\psi(x)$ telle que $\psi(x) \omega=df$. Préciser
alors $f$. (On dit que $\psi$ est un facteur intégrant.)}
\reponse{Comme $\omega$ est définie sur $\Rr^2$, il suffit que $\psi\omega$ soit exacte pour que $f$ existe.
Maintenant, $\psi\omega$ est exacte si et seulement si
$$ \frac{\partial (\psi(x)(x^2+y^2+2x))}{\partial y } =  \frac{\partial(\psi(x) 2y)} {\partial x}.$$
Ceci équivaut à $2y \psi(x)=2y\psi'(x)$. Ainsi, $\psi(x)=\psi'(x)$ pour tout $x$. Donc
$\psi(x)=k e^x$ avec $k$ constante. On peut choisir $k=1$.
Ainsi  $$\psi\omega=e^x(x^2+y^2+2x) dx+e^x(2y)dy.$$ On cherche ensuite $f$ telle que :
$$  \left\{ \begin{array}{lll}
\frac{\partial f} {\partial x}& = &e^x(x^2+y^2+2x) \\
\frac{\partial f}{\partial y} &= &e^x(2y) \\
\end{array} \right .$$
En intégrant la deuxième équation par rapport à $y$, on trouve 
$$f(x,y)=e^xy^2+c(x).$$
En dérivant cette expression par rapport à $x$ et en égalisant avec la première équation du système, on obtient
$$ e^xy^2+c'(x)=e^x(x^2+y^2+2x)$$ c'est-à-dire
$$c'(x)=e^x(x^2+2x).$$
Il en résulte que $c(x)=x^2 e^x+c$ et donc que
$$f(x,y)=e^x (x^2+y^2)+c$$ avec $c$ dans $\Rr$.}
\end{enumerate}
}

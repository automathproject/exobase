\uuid{N5Fg}
\exo7id{5549}
\auteur{rouget}
\organisation{exo7}
\datecreate{2010-07-15}
\isIndication{false}
\isCorrection{true}
\chapitre{Conique}
\sousChapitre{Hyperbole}

\contenu{
\texte{
Que vaut l'excentricité de l'hyperbole équilatère (une hyperbole est équilatère si et seulement
si ses asymptotes sont perpendiculaires)~?
}
\reponse{
Soit $\mathcal{H}$ une hyperbole. Il existe un repère orthonormé dans lequel $\mathcal{H}$ admet
une équation catésienne de la forme $\frac{x^2}{a^2}-\frac{y^2}{b^2}=1$, ($a>0$, $b>0$).
Dans ce repère, les asymptotes ont pour équations $y=\frac{b}{a}x$ et $y=-\frac{b}{a}x$. Elles sont perpendiculaires si
et seulement si $\frac{b}{a}\left(-\frac{b}{a}\right)^2=-1$ ou encore si et seulement si $a=b$. L'excentricité de $\mathcal{H}$ est alors

$$e=\frac{c}{a}=\frac{\sqrt{a^2+b^2}}{a}=\sqrt{1+\frac{b^2}{a^2}}=\sqrt{2}.$$

\begin{center}
\shadowbox{
L'excentricité de l'hyperbole équilatère vaut $\sqrt{2}$.
}
\end{center}
}
}

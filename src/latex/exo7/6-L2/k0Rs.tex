\uuid{k0Rs}
\exo7id{2016}
\titre{exo7 2016}
\auteur{liousse}
\organisation{exo7}
\datecreate{2003-10-01}
\isIndication{false}
\isCorrection{false}
\chapitre{Géométrie affine dans le plan et dans l'espace}
\sousChapitre{Géométrie affine dans le plan et dans l'espace}
\module{Géométrie}
\niveau{L2}
\difficulte{}

\contenu{
\texte{
On consid\`ere les cinq points suivants:
 $A(1,2,-1)$, $B(3,2,0)$, $C(2,1,-1)$, $D(1,0,4)$ et $E(-1,1,1)$.
}
\begin{enumerate}
    \item \question{Ces quatre points sont-ils coplanaires ?}
    \item \question{D\'eterminer la nature du triangle $ABC$. $A$, $B$ et $C$ sont-ils align\'es, si non
 donner une  \'equation cat\'esienne du plan $P$ qui les contient.}
    \item \question{D\'eterminer les coordonn\'ees
du barycentre $G$ des points $A$, $B$, $C$ et $D$.}
    \item \question{Montrer que $O$, $D$ et $G$ sont align\'es et que la droite $OD$ est perpendiculaire \`a $P$.}
\end{enumerate}
}

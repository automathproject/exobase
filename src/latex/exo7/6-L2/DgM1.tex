\uuid{DgM1}
\exo7id{7078}
\auteur{megy}
\organisation{exo7}
\datecreate{2017-01-21}
\isIndication{false}
\isCorrection{false}
\chapitre{Géométrie affine euclidienne}
\sousChapitre{Géométrie affine euclidienne du plan}

\contenu{
\texte{

}
\begin{enumerate}
    \item \question{On donne deux points $A$ et $B$. Soit $\tau$ la translation de vecteur $\overrightarrow{AB}$ et $M$ un point du plan. Construire $\tau(M)$.}
    \item \question{Soit $\mathcal D$ une droite du plan, et $\sigma$ la réflexion orthogonale d'axe $\mathcal D$. Soit $M$ un point du plan. Construire $\sigma(M)$. Réciproquement, soit $\sigma$ une réflexion orthogonale, et supposons donnés un point $A$ et son image $\sigma(A)$. Construire $\mathcal D$.}
    \item \question{On donne un point $O$, et $\phi$ l'homothétie de centre $O$ et de rapport $5/8$. Soit $M$ un point du plan. Construire $\phi(M)$. Remarque : ceci marche pour tout rapport rationnel.}
    \item \question{On donne un triangle auxiliaire $ABC$ et on considère l'angle $\alpha=(\overrightarrow{AB},\overrightarrow{AC})$. Soit $\rho$ la rotation de centre $O$ et d'angle $\alpha$. Si $M$ est un point du plan, construire $\rho(M)$.}
\end{enumerate}
}

\uuid{5aN0}
\exo7id{7501}
\auteur{mourougane}
\organisation{exo7}
\datecreate{2021-08-10}
\isIndication{false}
\isCorrection{false}
\chapitre{Géométrie affine euclidienne}
\sousChapitre{Géométrie affine euclidienne du plan}

\contenu{
\texte{

}
\begin{enumerate}
    \item \question{Soit $P$ un plan affine et $(A,B,C)$ un repère affine.
Donner la condition d'alignement de trois points
 en coordonnées barycentriques.}
    \item \question{Démontrer que si $f$ est une application linéaire orthogonale 
d'un espace vectoriel euclidien $\vec{E}$ de dimension finie, alors
$Ker (f-Id_{\vec{E}})$ et $Im (f-Id_{\vec{E}})$ sont supplémentaires orthogonaux.}
    \item \question{Donner l'exemple d'une application affine qui n'est pas une isométrie.}
    \item \question{Les isométries de déterminant $-1$
dans un espace affine euclidien de dimension trois
ont-elles toutes un point fixe ?}
    \item \question{Peut-on écrire une rotation dans le plan euclidien comme composée de cinq
réflexions ?}
    \item \question{Dans l'espace affine euclidien muni d'un repère orthonormé, 
on considère le cube de sommets $A(-1,-1,-1)$, $B(-1,-1,1)$, 
$C(-1,1,1)$, $D(-1,1,-1)$, $E(1,-1,-1)$, $F(1,-1,1)$, $G(1,1,1)$, $H(1,1,-1)$.
 Existe-t-il une isométrie de l'espace qui envoie $A$ sur $B$ et $G$ sur $F$ ?
Si oui, la déterminer.}
\end{enumerate}
}

\uuid{NaYF}
\exo7id{7420}
\titre{exo7 7420}
\auteur{mourougane}
\organisation{exo7}
\datecreate{2021-08-10}
\isIndication{false}
\isCorrection{false}
\chapitre{Géométrie affine dans le plan et dans l'espace}
\sousChapitre{Géométrie affine dans le plan et dans l'espace}

\contenu{
\texte{
Soit l'espace affine $\Rr^3$ muni d'un repère affine
$A_0,A_1,A_2,A_3$.
}
\begin{enumerate}
    \item \question{Trouver un système d'équations pour le sous-espace affine
$\mathcal{A}$ passant par le point
 $\displaystyle A\left(\begin{array}{c}
1\\2\\3\end{array}\right)$ et parallèle au plan d'équation 
$(2x-y-z=5)$.}
    \item \question{Trouver un système d'équations pour le sous-espace affine
$\mathcal{B}$ passant par
le point
 $\displaystyle A\left(\begin{array}{c}
1\\2\\3\end{array}\right)$ de direction $\Rr \vec{u}\oplus \Rr\vec{v}$
 o{ù} $\vec{u}$
est le vecteur de coordonées $\left(\begin{array}{c}
-1\\0\\3\end{array}\right)$ et $\vec{v}$ le vecteur de coordonées
$\left(\begin{array}{c} 
2\\1\\3\end{array}\right)$
dans la base
$(\vec{A_0A_1},\vec{A_0A_2}\vec{A_0A_3})$.}
    \item \question{Trouver un système d'équations pour le sous-espace affine
$\mathcal{C}$ engendré par les points
 $\displaystyle A\left(\begin{array}{c}
1\\2\\3\end{array}\right)$, $\displaystyle B\left(\begin{array}{c}
4\\0\\-1\end{array}\right)$ et $\displaystyle C\left(\begin{array}{c}
1\\1\\0\end{array}\right)$.}
\end{enumerate}
}

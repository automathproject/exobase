\uuid{utG7}
\exo7id{7787}
\titre{exo7 7787}
\auteur{mourougane}
\organisation{exo7}
\datecreate{2021-08-11}
\isIndication{false}
\isCorrection{false}
\chapitre{Sous-groupe distingué}
\sousChapitre{Sous-groupe distingué}

\contenu{
\texte{
Le but de l'exercice est de déterminer les sous-groupes distingués de
 $SL(E)$.
 On supposera que le corps $k$ est de
caractéristique nulle ou que sa caractéristique est différente de $2$
et première avec la dimension $n$ de $E$. On supposera aussi $n\geq
 3$.
}
\begin{enumerate}
    \item \question{Donner l'exemple d'un sous-groupe non distingué de $SL(E)$.}
    \item \question{Soit $\phi~:G\to H$ un morphisme surjectif de groupes.
 Montrer que l'image par $\phi$ d'un
 sous-groupe distingué de $G$ est un sous-groupe distingué de $H$.}
    \item \question{Soit $H$ un sous-groupe de $SL(E)$. Déterminer les possibilités
 pour son image dans $PSL(E)$ pour la projection canonique.}
    \item \question{Montrer que tout sous-groupe du centre de $SL(E)$ est distingué dans $SL(E)$.}
    \item \question{On supposera désormais que $H$ n'est pas un sous-groupe du
 centre de $SL(E)$.
Soit $\tau$ une transvection. Montrer que $\tau^n$ est une
 transvection de $H$.}
    \item \question{Montrer que $H$ contient toutes les transvections de $E$.}
    \item \question{Conclure.}
\end{enumerate}
}

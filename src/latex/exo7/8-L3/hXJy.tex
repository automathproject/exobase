\uuid{hXJy}
\exo7id{2131}
\titre{exo7 2131}
\auteur{debes}
\organisation{exo7}
\datecreate{2008-02-12}
\isIndication{false}
\isCorrection{true}
\chapitre{Ordre d'un élément}
\sousChapitre{Ordre d'un élément}
\module{Théorie des groupes}
\niveau{L3}
\difficulte{}

\contenu{
\texte{
D\' eterminer tous les sous-groupes du groupe sym\'etrique $S_3$.
}
\reponse{
D'apr\`es le th\'eor\`eme de Lagrange, les sous-groupes de $S_3$ sont d'ordre $1$, $2$, $3$ ou $6$. Les sous-groupes d'ordre $1$ et $6$ sont les sous-groupes triviaux $\{1\}$ et $S_3$ respectivement.
Comme $2$ et $3$ sont premiers, les sous-groupes d'ordre $2$ et $3$ sont cycliques. Un sous-groupe d'ordre $2$ est tout sous-groupe engendr\'e par une transposition: il y en a $3$. Il existe un seul sous-groupe d'ordre $3$, celui engendr\'e par le $3$-cycle $(1 \hskip 2pt 2 \hskip 2pt 3)$.
}
}

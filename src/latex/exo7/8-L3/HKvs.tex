\uuid{HKvs}
\exo7id{2197}
\titre{exo7 2197}
\auteur{debes}
\organisation{exo7}
\datecreate{2008-02-12}
\isIndication{true}
\isCorrection{false}
\chapitre{Théorème de Sylow}
\sousChapitre{Théorème de Sylow}
\module{Théorie des groupes}
\niveau{L3}
\difficulte{}

\contenu{
\texte{
Soient $G$ un groupe et $H$ un sous-groupe distingu\'e de
$G$. On se donne un nombre premier $p$ et l'on suppose que $H$ admet
un unique $p$-Sylow $S$. Montrer que $S$ est distingu\'e dans $G$.
}
\indication{Pour tout $g\in G$, $gSg^{-1}$ est un $p$-Sylow de $gHg^{-1}=H$
et donc $gSg^{-1}=S$.}
}

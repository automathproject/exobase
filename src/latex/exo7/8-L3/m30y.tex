\uuid{m30y}
\exo7id{2124}
\auteur{debes}
\organisation{exo7}
\datecreate{2008-02-12}
\isIndication{false}
\isCorrection{true}
\chapitre{Ordre d'un élément}
\sousChapitre{Ordre d'un élément}

\contenu{
\texte{
\label{ex:deb24}
Soit $G$ un groupe ab\'elien et $a$ et $b$ deux \'el\'ements d'ordres finis.
Montrer que $ab$ est d'ordre fini et que l'ordre de $ab$ divise le ppcm des ordres de $a$
et $b$. Montrer que si les ordres de $a$ et $b$ sont premiers entre eux, l'ordre de $ab$
est \'egal au ppcm des ordres de $a$ et de $b$.
}
\reponse{
Soient $a,b\in G$ d'ordre respectifs $m$ et $n$. Posons $\mu = \mathrm{ppcm}(m,n)$. 
On a $(ab)^\mu=a^\mu \cdot b^\mu = e\cdot e = e$ ($a^\mu = b^\mu = e$ r\'esultant du fait
que $m$ et $n$ divisent $\mu$). L'ordre de $ab$ divise donc $\mu$.\smallskip

Supposons que $\mathrm{pgcd}(m,n)=1$. Soit $k\in \Z$ tel que $(ab)^k=1$, soit $a^k=b^{-k}$. On en
d\'eduit que $a^{nk} = e$ et $b^{mk}=e$. D'o\`u $m|nk$ et $n|mk$. L'hypoth\`ese 
$\mathrm{pgcd}(m,n)=1$ donne alors $m|k$ et $n|k$ et donc $\mathrm{ppcm}(m,n)|k$. Cela combin\'e \`a la
premi\`ere partie montre que $ab$ est d'ordre $\mathrm{ppcm}(m,n)=mn$.
}
}

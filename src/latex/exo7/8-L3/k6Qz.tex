\uuid{k6Qz}
\exo7id{2132}
\titre{exo7 2132}
\auteur{debes}
\organisation{exo7}
\datecreate{2008-02-12}
\isIndication{true}
\isCorrection{true}
\chapitre{Ordre d'un élément}
\sousChapitre{Ordre d'un élément}

\contenu{
\texte{
\label{ex:deb32}
Montrer que dans un groupe d'ordre $35$, il existe un \'el\'ement d'ordre $5$ et un \'el\'ement d'ordre $7$.
}
\indication{Commencer par analyser l'ordre possible des \'el\'ements de $G$.}
\reponse{
Les \'el\'ements diff\'erents de $1$ sont d'ordre $5$, $7$ ou $35$. S'il existe un \'el\'ement $g$ d'ordre $35$ (\textit{i.e.}, si le groupe est cyclique d'ordre $35$), alors $g^5$ est d'ordre $7$ et $g^7$ est d'ordre $5$. Supposons que le groupe n'est pas cyclique et qu'il n'existe pas d'\'el\'ement d'ordre $7$. Tout 
\'el\'ement  diff\'erent de $1$ serait alors d'ordre $5$ et le groupe serait r\'eunion de sous-groupes d'ordre $5$. Mais de tels sous-groupes sont soit \'egaux soit d'intersection $\{1\}$ (car $5$ est premier).
On aurait alors $35 = 4n+1$ avec $n$ le nombre de sous-groupes distincts d'ordre $5$, ce qui donne la contradiction cherch\'ee. Le raisonnement est le m\^eme s'il n'existe pas d'\'el\'ement d'ordre $5$.
}
}

\uuid{IXk6}
\exo7id{2139}
\titre{exo7 2139}
\auteur{debes}
\organisation{exo7}
\datecreate{2008-02-12}
\isIndication{false}
\isCorrection{true}
\chapitre{Sous-groupe distingué}
\sousChapitre{Sous-groupe distingué}
\module{Théorie des groupes}
\niveau{L3}
\difficulte{}

\contenu{
\texte{
Montrer qu'un sous-groupe d'indice $2$ dans un groupe
$G$ est distingu\'e dans $G$.
}
\reponse{
Le sous-groupe $H$ est \`a la fois la classe \`a gauche et la classe \`a
droite modulo $H$ de l'\'el\'ement neutre. Si $[G:H]=2$, son compl\'ementaire $H^c$
dans $G$ est donc l'autre classe, \`a droite et \`a gauche. Classes \`a droite et
classes \`a gauche   coincident donc, soit $gH=Hg$ et donc $gHg^{-1}=H g g^{-1}=H$
pour tout $g\in G$.
}
}

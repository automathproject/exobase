\uuid{RQxw}
\exo7id{2138}
\titre{exo7 2138}
\auteur{debes}
\organisation{exo7}
\datecreate{2008-02-12}
\isIndication{false}
\isCorrection{true}
\chapitre{Sous-groupe distingué}
\sousChapitre{Sous-groupe distingué}

\contenu{
\texte{
Montrer que le groupe des automorphismes du groupe $\Z /2\Z \times \Z /2 \Z$ est
isomorphe au groupe sym\'etrique $S_3$.
}
\reponse{
Tout automorphisme $\varphi$ du groupe $G=\Z/2\Z \times \Z/2\Z$ permute les trois \'el\'ements
d'ordre $2$, c'est-\`a-dire l'ensemble $G^\ast$ des trois \'el\'ements non triviaux. La
correspondance qui \`a $\varphi \in \hbox{\rm Aut}(\Z/2\Z \times \Z/2\Z)$ associe sa
restriction \`a $G^\ast$ induit un morphisme $\chi: \hbox{\rm Aut}(\Z/2\Z \times
\Z/2\Z) \rightarrow S_3$. Tout morphisme $\varphi \in \hbox{\rm Aut}(\Z/2\Z \times \Z/2\Z)$
\'etant d\'etermin\'e par sa restriction \`a $G^\ast$, ce morphisme $\chi$ est injectif.
De plus, tout automorphisme lin\'eaire (pour la structure de $\Z/2\Z$-espace vectoriel de
$\Z/2\Z \times \Z/2\Z$) est un automorphisme de groupes. Il y a $6$ tels automorphismes
(autant qu'il y a de bases). L'image de $\chi$ contient donc au moins $6$ \'el\'ements.
Comme c'est un sous-groupe de $S_3$, c'est $S_3$ lui-m\^eme et $\chi$ est un isomorphisme.
}
}

\uuid{ffYf}
\exo7id{2107}
\auteur{debes}
\organisation{exo7}
\datecreate{2008-02-12}
\isIndication{true}
\isCorrection{true}
\chapitre{Ordre d'un élément}
\sousChapitre{Ordre d'un élément}

\contenu{
\texte{
On consid\`ere sur $\R$ la loi de composition d\'efinie par $x\star y=
x+y-xy$. Cette loi est-elle associative, commutative? Admet-elle un \'el\'ement neutre?
 Un r\'eel $x$ admet-il un inverse pour cette loi? Donner une formule pour la puissance
$n$-i\`eme d'un \'el\'ement $x$ pour cette loi.
}
\indication{Les premi\`eres questions ne pr\'esentent aucune difficult\'e.

Pour la derni\`ere, le plus difficile (et le plus int\'eressant) est de deviner
la formule. Pour cela, calculer la puissance $n$-i\`eme pour $n=1,2,3,4,5\ldots$.
(La formule est donn\'ee dans la page ``Corrections'').}
\reponse{
Pour la derni\`ere question, v\'erifier par r\'ecurrence que $\displaystyle x^{\star\hskip 2pt n}=
\sum_{k=1}^n (-1)^{k-1} C_n^k x^k$.
}
}

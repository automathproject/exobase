\uuid{to0I}
\exo7id{7779}
\auteur{mourougane}
\organisation{exo7}
\datecreate{2021-08-11}
\isIndication{false}
\isCorrection{false}
\chapitre{Sous-groupe distingué}
\sousChapitre{Sous-groupe distingué}

\contenu{
\texte{
Le but de l'exercice est de déterminer les groupes dérivés successifs de $\mathcal{S}_4$.
On notera $V_4$ le sous-groupe des permutations de profil $(\cdot,\cdot)(\cdot,\cdot)$ (avec l'identité).
}
\begin{enumerate}
    \item \question{Montrer que $D(\mathcal{S}_4) \subset \mathcal{A}_4$.}
    \item \question{Calculer les commutateurs $(1,2)(1,3)(1,2)^{-1}(1,3)^{-1}$ et $(1,2,3)(1,2,4)(1,2,3)^{-1}(1,2,4)^{-1}$.}
    \item \question{Montrer que $D(\mathcal{S} _4)=A_4$.}
    \item \question{Montrer que $V_4\subset D(\mathcal{A}_4)$.}
    \item \question{Vérifier que $V_4$ est distingué dans $\mathcal{A}_4$ et que le quotient $\mathcal{A}_4/V_4$ est un groupe abélien. En déduire que $D(\mathcal{A}_4)\subset V_4$.}
    \item \question{En déduire $D^2(\mathcal{S}_4)$.}
    \item \question{Calculer les autres groupes dérivés de $\mathcal{S}_4$.}
\end{enumerate}
}

\uuid{6gHm}
\exo7id{2103}
\auteur{debes}
\organisation{exo7}
\datecreate{2008-02-12}
\isIndication{false}
\isCorrection{true}
\chapitre{Ordre d'un élément}
\sousChapitre{Ordre d'un élément}

\contenu{
\texte{
Soit $X$ un ensemble non vide et ordonn\'e. Montrer qu'il existe une partie
$Y$ totalement ordonn\'ee de $X$ qui v\'erifie la propri\'et\'e
$$\forall x \notin Y \quad \exists y\in X \quad x\  \hbox {\rm et} \ y \ \hbox
{non comparables}$$

L'ensemble $Y$ est-il unique?
}
\reponse{
Pour tout $x\in X$, posons $C(x)=\{ y\in X \hskip 2pt |\hskip 2pt  x\ \hbox{\rm et}\ y \
\hbox{\rm sont comparables}\}\ $ et consid\'erons $Y=\bigcap_{x\in X}C(x)$. La partie $Y$ est
totalement ordonn\'ee puisque d\`es que $y,y^\prime \in Y$, alors $y^\prime \in C(y)$ et donc
$y$ et
$y^\prime$ sont comparables. De plus, pour tout $x\notin Y$, il existe $y\in X$ tel que 
$x\notin C(y)$, c'est-\`a-dire, $y$ et $x$ non comparables.\smallskip

Il n'y a pas unicit\'e de l'ensemble $Y$ en g\'en\'eral. En effet, dans un ensemble
ordonn\'e o\`u il existe un \'el\'ement $y$ qui n'est comparable qu'\`a lui-m\^eme,
on peut prendre $Y=C(y)=\{y\}$. Il est facile de construire des ensembles ordonn\'es
poss\'edant plusieurs tels \'el\'ements $y$ (penser \`a la relation d'\'egalit\'e, dont     
le graphe est la diagonale).
}
}

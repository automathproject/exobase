\uuid{VSFv}
\exo7id{2144}
\titre{exo7 2144}
\auteur{debes}
\organisation{exo7}
\datecreate{2008-02-12}
\isIndication{true}
\isCorrection{false}
\chapitre{Sous-groupe distingué}
\sousChapitre{Sous-groupe distingué}
\module{Théorie des groupes}
\niveau{L3}
\difficulte{}

\contenu{
\texte{
(a) Montrer que si $m$ et $n$ sont des entiers premiers entre eux et qu'un
\'el\'ement $z$ d'un groupe $G$ v\'erifie $z^m=z^n=e$ o\`u $e$ d\'esigne l'\'el\'ement
neutre de $G$, alors
$z=e$.
\smallskip

(b) Montrer que si $m$ et $n$ sont deux entiers premiers entre eux, l'application
$$\phi: \mu_m \times \mu_n \rightarrow \mu_{mn} $$ qui au couple $(s,t)$ fait correspondre le
produit
$st$ est un isomorphisme de groupes
}
\indication{(a) B\'ezout. (b) $\phi$ est injectif et ensembles de d\'epart et d'arriv\'ee ont m\^eme
cardinal.}
}

\uuid{9DAf}
\exo7id{2171}
\titre{exo7 2171}
\auteur{debes}
\organisation{exo7}
\datecreate{2008-02-12}
\isIndication{false}
\isCorrection{true}
\chapitre{Action de groupe}
\sousChapitre{Action de groupe}
\module{Théorie des groupes}
\niveau{L3}
\difficulte{}

\contenu{
\texte{
Soit $H$ un sous-groupe
distingu\'e de $S_n$ contenant une
transposition. Montrer que $H=S_n$.
}
\reponse{
Un sous-groupe distingu\'e de $S_n$ qui contient une transposition
contient toute sa classe de conjugaison, c'est-\`a-dire, toutes les transpositions (cf
les indications de l'exercice \ref{ex:deb68}, ``Rappel'') et donc le groupe qu'elles engendrent, 
c'est-\`a-dire $S_n$.
}
}

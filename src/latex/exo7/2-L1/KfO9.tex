\uuid{KfO9}
\exo7id{4012}
\auteur{quercia}
\organisation{exo7}
\datecreate{2010-03-11}
\isIndication{false}
\isCorrection{false}
\chapitre{Développement limité}
\sousChapitre{Formule de Taylor}

\contenu{
\texte{
\label{diffinie}Différences finies
Soit $f:\R \to \R$ de classe $\mathcal{C}^\infty$ et $h > 0$.
On pose : $$\Delta_hf(x) = \frac {f(x+h/2) - f(x-h/2)}h
  \qquad \text{et}\quad
\Delta_h^p = \underbrace{\Delta_h \circ \Delta_h \circ \dots \circ \Delta_h}%
_{p \text{ fois}}.$$

Par exemple, $\Delta_h^2f(x) = \frac {f(x+h) - 2f(x) + f(x-h)}{h^2}$.
}
\begin{enumerate}
    \item \question{\begin{enumerate}}
    \item \question{Montrer que : $\forall\ x \in \R,\ \exists\ \theta \in {]-1,1[} \text{ tq } \Delta_hf(x) = f'\left(x + \frac {\theta h}2\right)$.}
    \item \question{Montrer que : $\forall\ x \in \R,\ \exists\ \theta'\in {]-1,1[} \text{ tq } \Delta_hf(x) = f'(x) + \frac {h^2}{24} f^{(3)}\left( x + \frac{\theta' h}2\right)$.}
\end{enumerate}
}

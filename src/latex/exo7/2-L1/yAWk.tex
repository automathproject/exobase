\uuid{yAWk}
\exo7id{2097}
\titre{exo7 2097}
\auteur{bodin}
\organisation{exo7}
\datecreate{2008-02-04}
\video{JSvF3eC5EuA}
\isIndication{true}
\isCorrection{true}
\chapitre{Calcul d'intégrales}
\sousChapitre{Fraction rationnelle}

\contenu{
\texte{
Soit $\displaystyle I_{n} = \int_0^1 \frac{x^n}{1 + x}d x$.
}
\begin{enumerate}
    \item \question{En majorant la fonction int\'egr\'ee, montrer que
$\lim_{n\to +\infty} I_{n}=0$.}
\reponse{Pour $x>0$ on a $\frac{x^n}{1+x} \leqslant x^n$,
donc 
$$I_n  \leqslant \int_0^1 x^n dx = \left[ \frac{1}{n+1} x^{n+1} \right]_0^1=\frac{1}{n+1}.$$
Donc $I_n \to 0$ lorsque $n\to +\infty$.}
    \item \question{Calculer $I_n + I_{n + 1}$.}
\reponse{$I_n+I_{n+1}=\int_0^1 x^n \frac {1+x}{1+x} dx =  \int_0^1 x^n dx=\frac{1}{n+1}$.}
    \item \question{D\'eterminer $\displaystyle \lim_{n \rightarrow  + \infty} \left(\sum_{k = 1}^n \frac{
 (-1)^{k + 1}}k\right)$.}
\reponse{Soit $S_n = 1-\frac 12 + \frac13-\frac 14 +\cdots \pm \frac 1n = \sum_{k = 1}^n \frac{
 (-1)^{k + 1}}k$.
Par la question pr\'ec\'edente nous avons 
$S_n = (I_0+I_1)-(I_1+I_2)+(I_2+I_3)- \cdots \pm (I_{n-1}+I_n)$.
Mais d'autre part cette somme \'etant t\'elescopique cela conduit à $S_n = I_0 \pm I_n$.
 Alors la limite de $S_n$ et donc de $\sum_{k = 1}^n \frac{
 (-1)^{k + 1}}k$ (quand $n\to +\infty$) est $I_0$ car $I_n \to 0$.
Un petit calcul montre que $I_0 = \int_0^1 \frac {dx}{1+x} = \ln 2$.
Donc la somme altern\'ee des inverses des entiers converge vers $\ln 2$.}
\indication{\begin{enumerate}
  \item Majorer par $x^n$.
  \item
  \item On pourra calculer $(I_0+I_1)-(I_1+I_2)+(I_2+I_3)- \cdots$
  \end{enumerate}}
\end{enumerate}
}

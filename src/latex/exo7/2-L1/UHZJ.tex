\uuid{UHZJ}
\exo7id{5146}
\titre{exo7 5146}
\auteur{rouget}
\organisation{exo7}
\datecreate{2010-06-30}
\isIndication{false}
\isCorrection{true}
\chapitre{Propriétés de R}
\sousChapitre{Autre}
\module{Analyse}
\niveau{L1}
\difficulte{}

\contenu{
\texte{
Soient $x$ et $y$ deux réels tels que $0<x\leq y$. On pose $m=\frac{x+y}{2}$ (moyenne arithmétique), $g=\sqrt{xy}$
(moyenne géométrique) et $\frac{1}{h}=\frac{1}{2}(\frac{1}{x}+\frac{1}{y})$ (moyenne harmonique). Montrer que $x\leq
h\leq g\leq m\leq y$.
}
\reponse{
On a déjà $x=\frac{x+x}{2}\leq\frac{x+y}{2}=m\leq\frac{y+y}{2}=y$ et donc \shadowbox{$x\leq m\leq y$}.

(on peut aussi écrire~:~$m-x=\frac{x+y}{2}-x=\frac{y-x}{2}\geq0$).
On a ensuite $x=\sqrt{x.x}\leq\sqrt{xy}=g\leq\sqrt{y.y}=y$ et donc \shadowbox{$x\leq g\leq y$}.
$m-g=\frac{x+y}{2}-\sqrt{xy}=\frac{1}{2}((\sqrt{x})^2-2\sqrt{xy}+(\sqrt{y})^2)=\frac{1}{2}(\sqrt{y}-\sqrt{x})^2
\geq0$ et donc, \shadowbox{$x\leq g\leq m\leq y$}.
D'après 1), la moyenne arithmétique de $\frac{1}{x}$ et $\frac{1}{y}$ est comprise entre $\frac{1}{x}$ et
$\frac{1}{y}$, ce qui fournit $\frac{1}{y}\leq\frac{1}{h}\leq\frac{1}{x}$, ou encore \shadowbox{$x\leq h\leq y$}.
D'après 3), la moyenne géométrique des deux réels $\frac{1}{x}$ et $\frac{1}{y}$ est inférieure ou égale à leur moyenne arithmétique. Ceci
fournit $\sqrt{\frac{1}{x}.\frac{1}{y}}\leq\frac{1}{2}(\frac{1}{x}+\frac{1}{y})$ ou encore
$\frac{1}{g}\leq\frac{1}{h}$ et finalement 
\begin{center}
\shadowbox{$x\leq h\leq g\leq m\leq y$ où $\frac{1}{h}=\frac{1}{2}\left(\frac{1}{x}+\frac{1}{y}\right)$, $g=\sqrt{xy}$ et $m=\frac{x+y}{2}$.}
\end{center}
}
}

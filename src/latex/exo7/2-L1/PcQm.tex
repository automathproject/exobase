\uuid{PcQm}
\exo7id{4037}
\titre{exo7 4037}
\auteur{quercia}
\organisation{exo7}
\datecreate{2010-03-11}
\video{jBtvOPwMUdQ}
\isIndication{false}
\isCorrection{true}
\chapitre{Développement limité}
\sousChapitre{Développements limités implicites}
\module{Analyse}
\niveau{L1}
\difficulte{}

\contenu{
\texte{

}
\begin{enumerate}
    \item \question{Montrer que l'équation $\tan x = x$ possède une unique solution
   $x_n$ dans
   $\left]n\pi-\frac \pi2, n\pi+\frac \pi2\right[$ $(n\in \N)$.}
\reponse{Notons $I_n$ l'intervalle  $\left]n\pi-\frac \pi2, n\pi+\frac \pi2\right[$.
Alors sur chaque $I_n$ la fonction définie par $f(x) = \tan x-x$ est un fonction continue et dérivable.
De plus $f'(x) = 1+\tan^2x -1 = \tan ^2 x$. La dérivée est strictement positive sauf en un point où elle est nulle et ainsi 
la fonction $f$ est strictement croissante sur $I_n$. La limite à gauche est $-\infty$ et
la limite à droite est $+\infty$. Par le théorème des valeurs intermédiaires il existe un unique $x_n \in I_n$
tel que $f(x_n)=0$ c'est-à-dire $\tan x_n=x_n$.}
    \item \question{Quelle relation lie $x_n$ et $\arctan(x_n)$ ? \label{relation}}
\reponse{$x \mapsto \arctan x$ est la bijection réciproque de la restriction de la tangente  
$\tan_| : ]-\frac\pi2,+\frac\pi2[ \to ]-\infty,+\infty[$. Sur ces intervalles on a bien
$\tan x = y \iff x = \arctan y$. Mais si $y \notin ]-\frac\pi2,+\frac\pi2[$ il faut d'abord se ramener dans l'intervalle
 $]-\frac\pi2,+\frac\pi2[$.

Ainsi $x_n \in I_n$ donc $x_n-n\pi \in ]-\frac\pi2,+\frac\pi2[$.
Maintenant $x_n = \tan (x_n)=\tan(x_n-n\pi)$. 

Donc $\arctan x_n = \arctan \big(\tan(x_n-n\pi) \big) = x_n - n\pi$.
Ainsi $$x_n = \arctan x_n + n\pi.$$


L'erreur classique est de penser que $\arctan (\tan x) = x$. Ce qui n'est vrai que pour 
$x \in ]-\frac\pi2,+\frac\pi2[$ !}
    \item \question{Donner un DL de $x_n$ en fonction de $n$ à l'ordre $0$ pour $n\to\infty$.}
\reponse{Comme $x_n \in I_n$ alors $x_n \to +\infty$ lorsque $n\to +\infty$.

On sait par ailleurs que pour $x>0$ on a $\arctan x + \arctan \frac 1x = \frac\pi 2$.
Ainsi $\arctan x_n = \frac \pi 2 - \arctan \frac{1}{x_n}$

Lorsque $n$ tend vers $+\infty$ alors $\frac{1}{x_n}\to 0$ donc 
$\arctan \frac{1}{x_n} \to 0$.

Ainsi 
$$x_n = n\pi + \arctan x_n = n\pi +\frac \pi 2 - \arctan \frac{1}{x_n} = n\pi +\frac \pi 2 + o(1).$$}
    \item \question{En reportant dans la relation trouvée en \ref{relation},
     obtenir un DL de $x_n$ à l'ordre 2.}
\reponse{On va utiliser le dl obtenu précédemment pour obtenir un dl à un ordre plus grand :

\begin{align*}
x_n 
 & = n\pi + \arctan x_n \\
 & = n\pi +\frac \pi 2 - \arctan \frac{1}{x_n} \\
 & = n\pi +\frac \pi 2 - \arctan \frac{1}{n\pi +\frac \pi 2 + o(1)} \\
 & = n\pi +\frac \pi 2 - \frac{1}{n\pi +\frac \pi 2 + o(1)} +o(\frac1{n^2}) \qquad \text{ car } \arctan u = u + o(u^2) \text{ en } u=0\\
 & = n\pi +\frac \pi 2 - \frac{1}{n\pi} \frac{1}{1+\frac{1}{2n} + o(\frac1n)}+o(\frac1{n^2}) \\
 & = n\pi +\frac \pi 2 - \frac{1}{n\pi}\big(1-\frac{1}{2n} + o(\frac1n) \big) +o(\frac1{n^2})\\
 & = n\pi +\frac \pi 2 -\frac{1}{n\pi} + \frac{1}{2\pi n^2} + o(\frac1{n^2}) \\
\end{align*}

Ainsi en $+\infty$ on a le développement :
$$x_n  = n\pi +\frac \pi 2 -\frac{1}{n\pi} + \frac{1}{2\pi n^2} + o(\frac1{n^2}).$$}
\end{enumerate}
}

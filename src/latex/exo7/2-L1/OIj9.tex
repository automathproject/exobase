\uuid{OIj9}
\exo7id{4472}
\titre{exo7 4472}
\auteur{quercia}
\organisation{exo7}
\datecreate{2010-03-14}
\isIndication{false}
\isCorrection{true}
\chapitre{Série numérique}
\sousChapitre{Autre}
\module{Analyse}
\niveau{L1}
\difficulte{}

\contenu{
\texte{
Soit $\sum_{n\ge1} x_n$ une série absolument convergente telle que
pour tout entier $k \ge 1$ on a $\sum_{n=1}^\infty x_{kn} = 0$.

Montrer que : $\forall\ n\in\N^*,\ x_n = 0$.
}
\reponse{
Démonstration pour $x_1$ :
$\sum x_n = 0$, $\sum x_{2n} = 0  \Rightarrow  \sum_{n\text{ impair}} x_n = 0$.
On retire les multiples impairs de 3 ($\sum x_{3n} - \sum x_{6n} = 0$)
$ \Rightarrow  \sum_{n\not\equiv 0[2] ; n\not\equiv 0[3]} x_n = 0$.
On retire les multiples restants de $5,7,\dots$
On obtient ainsi une suite $(s_p)_{p \text{ premier}}$ nulle qui converge vers
$x_1$, donc $x_1 = 0$.\par
Peut-on se passer de la convergence absolue ?
}
}

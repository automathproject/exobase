\uuid{zs1x}
\exo7id{810}
\titre{exo7 810}
\auteur{cousquer}
\organisation{exo7}
\datecreate{2003-10-01}
\isIndication{false}
\isCorrection{true}
\chapitre{Calcul d'intégrales}
\sousChapitre{Longueur, aire, volume}
\module{Analyse}
\niveau{L1}
\difficulte{}

\contenu{
\texte{
Soit $C$ un cercle fixe de rayon $R$. Un cercle $C'$ de même rayon
roule sans glisser sur $C$ en restant dans un plan (variable) perpendiculaire
à celui de~$C$. Un point~$M$ lié au cercle~$C'$ décrit une courbe~$\Gamma$. 
Montrer que suivant un repère convenablement choisi, $\Gamma$
admet la représentation paramétrique~:
$\left\lbrace
\begin{array}{rcl}
    x & = & R(\cos t+\sin^2t)  \\
    y & = & R\sin t(1-\cos t)  \\
    z & = & R(1-\cos t)
\end{array}\right.$.
En déduire la longueur $L$ de $\Gamma$.
Représenter les projections de $\Gamma$ sur chacun des trois plans
de coordonnées.
}
\reponse{
$L=4R\bigl(\sqrt2+\ln(1+\sqrt2)\bigr)$.
}
}

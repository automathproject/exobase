\uuid{yZAY}
\exo7id{3148}
\auteur{quercia}
\organisation{exo7}
\datecreate{2010-03-08}
\isIndication{false}
\isCorrection{true}
\chapitre{Propriétés de R}
\sousChapitre{Les rationnels}

\contenu{
\texte{
Soient $\frac ab < \frac cd$ deux rationnels avec $a,c \in \Z$, et $b,d \in \N^*$.
}
\begin{enumerate}
    \item \question{Prouver que tout rationnel s'{\'e}crit : $x = \frac{ma+nc}{mb+nd}$ avec $m,n\in \Z$,
    et $mb+nd \ne 0$.}
\reponse{Pour $x = \frac pq$, on peut prendre :
             $m = qc-pd$, et $n = pb-qa$.}
    \item \question{{\'E}tudier l'unicit{\'e} d'une telle {\'e}criture.}
\reponse{$(m,n)$ est unique {\`a} un facteur pr{\`e}s.}
    \item \question{Montrer que $\frac{ma+nc}{mb+nd}$ est compris entre $\frac ab$ et $\frac cd$
    si et seulement si $m$ et $n$ ont m{\^e}me signe.}
\reponse{$\frac{ma+nc}{mb+nd} - \frac ab = \frac{n(bc-ad)}{b(mb+nd)}$,
          et $\frac cd - \frac{ma+nc}{mb+nd} = \frac{m(bc-ad)}{d(mb+nd)}$.}
\end{enumerate}
}

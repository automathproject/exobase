\uuid{Ky60}
\exo7id{4488}
\titre{exo7 4488}
\auteur{quercia}
\organisation{exo7}
\datecreate{2010-03-14}
\isIndication{false}
\isCorrection{true}
\chapitre{Série numérique}
\sousChapitre{Familles sommables}

\contenu{
\texte{

}
\begin{enumerate}
    \item \question{Soit $f:\R \to \R$ croissante. Montrer que l'ensemble des points
de discontinuité de~$f$ est dénombrable (pour $[a,b]\subset \R$,
considérer la famille $(f(x^+)-f(x^-))_{x\in{[a,b]}}$).}
    \item \question{Donner un exemple de fonction $f:\R \to \R$ croissante ayant
une infinité dénombrable de discontinuités.}
    \item \question{$(**)$ Trouver une fonction $f:\R \to \R$ {\it strictement\/} croissante dont
l'ensemble des points de discontinuité est égal à~$\Q$.}
\reponse{
Soit $(r_n)$ une énumération de~$Q$.
On pose $f(x) = \sum_{r_n<x} \frac 1{(n+1)^2}$.
$f$ est strictement croissante car pour $x< y$ il existe $n\in\N$
tel que $x< r_n< y$ donc $f(y)-f(x)\ge \frac1{(n+1)^2}$.
Si $x\in\Q$, $x=r_k$ alors $f(x^+)-f(x^-)\ge \frac 1{(k+1)^2}$
d'où $f$ est discontinue en~$x$.
Si $x\in\R\setminus\Q$ et $n\in\N$ alors il existe un voisinage de~$x$
ne contenant aucun $r_i$, $i\le n$ d'où

$f(x^+)-f(x^-)\le \sum_{i> n}\frac1{(i+1)^2}$
et $f$ est continue en~$x$.
}
\end{enumerate}
}

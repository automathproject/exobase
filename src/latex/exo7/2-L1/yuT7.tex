\uuid{yuT7}
\exo7id{5387}
\titre{exo7 5387}
\auteur{rouget}
\organisation{exo7}
\datecreate{2010-07-06}
\isIndication{false}
\isCorrection{true}
\chapitre{Continuité, limite et étude de fonctions réelles}
\sousChapitre{Continuité : théorie}

\contenu{
\texte{
Montrer que la fonction caractéristique de $\Qq$ est discontinue en chacun de ses points.
}
\reponse{
Soit $\chi$ la fonction caractéristique de $\Qq$.
Soit $x_0$ un réel. On note que

$$x_0\in\Qq\Leftrightarrow\forall n\in\Nn^*,\;x_0+\frac{1}{n}\in\Qq Q\Leftrightarrow\forall n\in\Nn^*,\;x_0+\frac{\pi}{n}\notin\Qq.$$

Donc, $\lim_{n\rightarrow +\infty}\chi(x_0+\frac{1}{n})$ existe, $\lim_{n\rightarrow +\infty}\chi(x_0+\frac{\pi}{n})$ existe et$\lim_{n\rightarrow +\infty}\chi(x_0+\frac{1}{n})\neq\lim_{n\rightarrow +\infty}\chi(x_0+\frac{\pi}{n})$ (bien que $\lim_{n\rightarrow +\infty}x_0+\frac{1}{n}=\lim_{n\rightarrow +\infty}x_0+\frac{\pi}{n}=x_0$. Ainsi, pour tout réel $x_0\in\Rr$, la fonction caractéristique de $\Qq$ n'a pas de limite en $x_0$ et est donc discontinue en $x_0$.
}
}

\uuid{OIEk}
\exo7id{670}
\titre{exo7 670}
\auteur{vignal}
\organisation{exo7}
\datecreate{2001-09-01}
\video{AqwFsD85iiU}
\isIndication{true}
\isCorrection{true}
\chapitre{Continuité, limite et étude de fonctions réelles}
\sousChapitre{Continuité : pratique}
\module{Analyse}
\niveau{L1}
\difficulte{}

\contenu{
\texte{
Soit $f : \R\setminus\{1/3\}\rightarrow \R$ 
telle que $f(x)= \frac{2x+3}{3x-1}$.

Pour tout $\epsilon>0$ d\'eterminer $\delta$ tel que,
($x\not=1/3$ et $|x|\leq\delta)\Rightarrow |f(x)+3|\leq \epsilon$.

Que peut-on en conclure ?
}
\indication{Le ``$\epsilon$'' vous est donné, il ne faut pas y toucher.
Par contre c'est à vous de trouver le ``$\delta$''.}
\reponse{
Commençons par la fin, trouver un tel $\delta$ montrera que 
$$\forall \epsilon > 0 \quad \exists \delta > 0 \quad |x-x_0| < \delta \Rightarrow |f(x)-(-3)| < \epsilon$$
autrement dit la limite de $f$ en $x_0=0$ est $-3$.
Comme $f(0)=-3$ alors cela montre aussi que $f$ est continue en $x_0=0$.

\bigskip

On nous donne un $\epsilon>0$, à nous de trouver ce fameux $\delta$.
Tout d'abord 
$$\left| f(x)+3 \right| = \left| \frac{2x+3}{3x-1} + 3 \right| = \frac{11|x|}{|3x-1|}.$$
Donc notre condition devient :
$$ \left| f(x)+3 \right| < \epsilon \quad
 \Leftrightarrow \quad \frac{11|x|}{|3x-1|} < \epsilon 
\quad  \Leftrightarrow \quad |x| < \epsilon\frac{|3x-1|}{11}.$$

Comme nous voulons éviter les problèmes en $x = \frac 13$ pour lequel la fonction $f$ n'est pas définie, nous
allons nous placer ``loin'' de $\frac 13$.
Considérons seulement les $x \in \Rr$ tel que $|x| < \frac 16$.
Nous avons :
$$|x| < \frac 16 \Rightarrow -\frac 16 < x < + \frac 16 \quad  \Rightarrow \quad  -\frac 32 < 3x-1 < -\frac 12 \quad \Rightarrow \quad \frac 12 < |3x-1|.$$
Et maintenant explicitons $\delta$ :
prenons $\delta < \epsilon \cdot \frac{1}{2}\cdot \frac{1}{11}$.
Alors pour $|x| < \delta$ nous avons 
$$|x| < \delta = \epsilon \cdot  \frac{1}{2} \cdot \frac{1}{11} < \epsilon \cdot|3x-1|\cdot \frac{1}{11}$$
ce qui implique par les équivalences précédentes que 
$\left| f(x)+3 \right| < \epsilon$.

Il y a juste une petite correction à apporter à notre $\delta$ : au cours de nos calculs
nous avons supposé que $|x| < \frac 16$, mais rien ne garantie que $\delta \le \frac 16$
(car $\delta$ dépend de $\epsilon$ qui pourrait bien être très grand, même
si habituellement ce sont les $\epsilon$ petits qui nous intéressent).
Au final le $\delta$ qui convient est donc :
$$\delta = \min (\frac 16, \frac{\epsilon}{22}).$$

\bigskip

Remarque finale :
bien sûr on savait dès le début que $f$ est continue en $x_0=0$. En effet
$f$ est le quotient de deux fonctions continues, le dénominateur ne s'annulant
pas en $x_0$. Donc nous savons dès le départ qu'un tel $\delta$ existe,
mais ici nous avons fait plus, nous avons trouvé une formule explicite pour ce $\delta$.
}
}

\uuid{u767}
\exo7id{484}
\titre{exo7 484}
\auteur{cousquer}
\organisation{exo7}
\datecreate{2003-10-01}
\isIndication{false}
\isCorrection{false}
\chapitre{Propriétés de R}
\sousChapitre{Maximum, minimum, borne supérieure}

\contenu{
\texte{
Soit $(u_n)_{n\in \mathbb{N}}$ une suite bornée.
On pose $A_p = \sup_{n>p} u_n$ et $B_p= \inf_{n>p} u_n$.
Montrer que $(A_p)_{p\in\mathbb{N}}$ est une suite décroissante bornée
et que $(B_p)_{p\in \mathbb{N}}$ est une suite croissante bornée.
Soit $L = \lim_{p \to\infty} A_p$ et $l= \lim_{p \to\infty} B_p$.
}
\begin{enumerate}
    \item \question{Dans le cas particulier où $u_n= {n+2\over n+1} \cos{n\pi\over3}$,
calculer $L$ et $l$.}
    \item \question{Montrer que :
 $$\displaylines{
\forall\epsilon>0,\;\exists p\in\mathbb{N},\;\forall n \geq p,\;
u_n > l- \epsilon\cr
\forall\epsilon>0,\;\forall p\in\mathbb{N},\;\exists n \geq p,\;
u_n < l+ \epsilon \cr}$$}
    \item \question{Interpréter ces propriétés. Énoncer des propriétés analogues pour $L$.
Démontrez-les.}
    \item \question{Que peut-on dire de $(u_n)$ si $L=l$ ?}
\end{enumerate}
}

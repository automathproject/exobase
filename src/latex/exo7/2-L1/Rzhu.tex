\uuid{Rzhu}
\exo7id{3952}
\titre{exo7 3952}
\auteur{quercia}
\organisation{exo7}
\datecreate{2010-03-11}
\isIndication{false}
\isCorrection{false}
\chapitre{Dérivabilité des fonctions réelles}
\sousChapitre{Autre}
\module{Analyse}
\niveau{L1}
\difficulte{}

\contenu{
\texte{
Soit $f:\R \to \R$ de classe $\mathcal{C}^n$, $a_1, \dots, a_n$ $n$ points distincts dans $\R$,
et $P$ le polynôme de Lagrange prenant les mêmes valeurs que $f$ aux points $a_i$.
On pose $Q(x) = \frac 1{n!} \prod_{i=1}^n (x-a_i)$.


Montrer que : $\forall\ b \in \R,\ \exists\ c \in \R \text{ tq } f(b) = P(b) + Q(b)f^{(n)}(c)$

(considérer $g(t) = f(t) - P(t) - \lambda Q(t)$ où $\lambda$ est
choisi de sorte que $g(b) = 0$).
}
}

\uuid{UqSA}
\exo7id{6978}
\titre{exo7 6978}
\auteur{blanc-centi}
\organisation{exo7}
\datecreate{2014-05-06}
\video{UkRPGUQfPP8}
\isIndication{false}
\isCorrection{true}
\chapitre{Fonctions circulaires et hyperboliques inverses}
\sousChapitre{Fonctions hyperboliques et hyperboliques inverses}

\contenu{
\texte{
Simplifier les expressions suivantes:
}
\begin{enumerate}
    \item \question{$\ch(\Argsh x),\quad \tanh(\Argsh x),\quad \sh(2\Argsh x)$.}
    \item \question{$ \sh(\Argch x),\quad \tanh(\Argch x),\quad \ch(3\Argch x)$.}
\reponse{
\begin{enumerate}
On sait que $\ch^2 u=1+\sh^2u$. Comme de plus la fonction $\ch$ est à valeurs positives, 
$\ch u=\sqrt{1+\sh^2u}$ et donc $\ch(\Argsh x)=\sqrt{1+\sh^2(\Argsh x)} = \sqrt{1+x^2}$.
Alors
$$\tanh(\Argsh x)=\frac{\sh(\Argsh x)}{\ch(\Argsh x)}=\frac{x}{\sqrt{1+x^2}}.$$
Et $\sh(2\Argsh x)=2\ch(\Argsh x)\sh(\Argsh x)=2x\sqrt{1+x^2}$.
}
\end{enumerate}
}

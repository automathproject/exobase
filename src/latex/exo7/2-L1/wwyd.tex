\uuid{wwyd}
\exo7id{5221}
\titre{exo7 5221}
\auteur{rouget}
\organisation{exo7}
\datecreate{2010-06-30}
\isIndication{false}
\isCorrection{true}
\chapitre{Suite}
\sousChapitre{Convergence}
\module{Analyse}
\niveau{L1}
\difficulte{}

\contenu{
\texte{
Soit $(u_n)_{n\in\Nn}$ une suite réelle. Montrer que si la suite $(u_n)_{n\in\Nn}$ converge au sens de \textsc{Césaro} et est monotone, alors la suite $(u_n)_{n\in\Nn}$ converge.
}
\reponse{
Supposons sans perte de généralité $u$ croissante (quite à remplacer $u$ par $-u$).
Dans ce cas, ou bien $u$ converge, ou bien $u$ tend vers $+\infty$.
Supposons que $u$ tende vers $+\infty$, et montrons qu'il en est de même pour la suite $v$.
Soit $A\in\Rr$. Il existe un rang $n_0$ tel que pour n naturel supérieur ou égal à $n_0$, $u_n\geq2A$.
Pour $n\geq n_0+1$, on a alors,

\begin{align*}
v_n&=\frac{1}{n+1}\left(\sum_{k=0}^{n_0}u_k+\sum_{k=n_0+1}^{n}u_k\right)\geq \frac{1}{n+1}\sum_{k=0}^{n_0}u_k+\frac{(n-n_0)2A}{n+1}
\end{align*}
Maintenant, quand $n$ tend vers $+\infty$, $\frac{1}{n+1}\sum_{k=0}^{n_0}u_k+\frac{(n-n_0)2A}{n+1}$ tend vers $2A$ et donc, il existe un rang $n_1$ à partir duquel $v_n\geq\frac{1}{n+1}\sum_{k=0}^{n_0}u_k+\frac{(n-n_0)2A}{n+1}>A$.
On a montré que~:~$\forall n\in\Nn,\;\exists n_1\in\Nn/\;(\forall n\in\Nn),\;(n\geq n_1\Rightarrow v_n>A)$. Par suite, $\lim_{n\rightarrow +\infty}v_n=+\infty$. Par contraposition, si $v$ ne tend pas vers $+\infty$, la suite $u$ ne tend pas vers $+\infty$ et donc converge, d'après la remarque initiale.
}
}

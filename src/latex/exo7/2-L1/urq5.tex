\uuid{urq5}
\exo7id{1268}
\auteur{legall}
\organisation{exo7}
\datecreate{1998-09-01}
\video{kxEXonvTl5g}
\isIndication{true}
\isCorrection{true}
\chapitre{Développement limité}
\sousChapitre{Formule de Taylor}

\contenu{
\texte{
Soit $a$ un nombre réel et $f : ] a , +\infty [ \rightarrow \Rr$ une application de classe $C^2$. 
On suppose $f$ et $f''$ bornées ; on pose $\displaystyle  M_0=\sup _{x>a}\vert f(x)\vert$ et 
$\displaystyle  M_2=\sup_{x> a}\vert f''(x)\vert$.
}
\begin{enumerate}
    \item \question{En appliquant une formule de Taylor reliant $f(x)$ et $f(x+h)$, montrer que, pour tout 
$x>a$ et tout $h>0$, on a~: $\displaystyle \vert f'(x)\vert \leq \frac{h}{2}M_2+\frac{2}{h}M_0$.}
\reponse{La formule de Taylor-Lagrange à l'ordre $2$ entre $x$ et $x+h$ (avec $h>0$) donne :
$$f(x+h) = f(x) + f'(x) h  + f''(c_{x,h}) \frac{h^2}{2!} $$
où $c_{x,h} \in ]x,x+h[$.

Cela donne :
$$f'(x) h = f(x+h) - f(x)  - f''(c_{x,h}) \frac{h^2}{2!}.$$

On peut maintenant majorer $f'(x)$ :
\begin{align*}
h|f'(x)| 
   & \le \left|f(x+h) \right| + \left| f(x) \right| + \frac{h^2}{2}\left| f''(c_{x,h})\right|  \\
   & \le 2M_0 + \frac {h^2}{2} M_2 \\
\end{align*}
Donc
$$|f'(x)|\le \frac{2}{h}{M_0} + \frac{h}{2}M_2.$$}
    \item \question{En déduire que $f'$ est bornée sur $]a,+\infty[$.}
\reponse{Soit $\phi : ]0,+\infty[ \rightarrow \Rr$ la fonction définie par $\phi(h) = \frac {h}{2}M_2+\frac{2}{h}M_0$.
C'est une fonction continue et dérivable. La limite en $0$ et $+\infty$ est $+\infty$.
La dérivée $\phi'(h)=\frac12 M_2-\frac{2M_0}{h^2}$ s'annule en $h_0 = 2\sqrt{\frac{M_0}{M_2}}$ et en ce point 
$\phi$ atteint son minimum
$\phi(h_0) = 2\sqrt{M_0M_2}$.

Fixons $x>a$. Comme pour tout $h>0$ on a $|f'(x)| \le \frac {h}{2}M_2+\frac{2}{h}M_0 =\phi(h)$ alors en particulier pour 
$h=h_0$ on obtient $|f'(x)| \le \phi(h_0)= 2\sqrt{M_0M_2}$. Et donc $f'$ est bornée.}
    \item \question{\'Etablir le résultat suivant : soit $g : ]0,+\infty[ \rightarrow \Rr$ une application
de classe $C^2$ à dérivée seconde bornée et telle que $\displaystyle \lim _{x\rightarrow +\infty }g(x)=0$. 
Alors $\displaystyle\lim _{x\rightarrow +\infty }g'(x)=0$.}
\reponse{Fixons $\epsilon >0$. $g''$ est bornée, notons $M_2 = \sup_{x> 0}\vert g''(x)\vert$. Comme $g(x)\to 0$ alors il 
existe $a>0$ tel que sur l'intervalle $]a,+\infty[$, $g$ soit aussi petit que l'on veut. Plus précisément nous choisissons $a$ de sorte que 
$$M_0 = \sup_{x>a}\vert g(x)\vert \le \frac{\epsilon^2}{4M_2}.$$


La première question appliquée à $g$ sur l'intervalle $]a,+\infty[$ implique 
que pour tout $h>0$ : 
$$|g'(x)| \le  \frac{2}{h}M_0 + \frac {h}{2} M_2 $$

En particulier pour $h = \frac{\epsilon}{M_2}$ et en utilisant $M_0  \le \frac{\epsilon^2}{4M_2}$
on obtient :
$$|g'(x)| \le  \frac{2}{\frac{\epsilon}{M_2}}\frac{\epsilon^2}{4M_2} + \frac{\frac{\epsilon}{M_2}}{2} M_2 \le \epsilon.$$

Ainsi pour chaque $\epsilon$ on a trouvé $a>0$ tel que si $x>a$ alors $|g'(x)|\le \epsilon$.
C'est exactement dire que $\lim_{x\to+\infty} g'(x)=0$.}
\indication{\begin{enumerate}
  \item La formule à appliquer est celle de Taylor-Lagrange à l'ordre $2$.
  \item \'Etudier la fonction $\phi(h) = \frac{h}{2}M_2+\frac{2}{h}M_0$ et trouver $\inf_{h>0} \phi(h)$.
  \item Il faut choisir un $a>0$ tel que $g(x)$ soit assez petit sur $]a,+\infty[$ ; puis appliquer
les questions précédentes à $g$ sur cet intervalle.
\end{enumerate}}
\end{enumerate}
}

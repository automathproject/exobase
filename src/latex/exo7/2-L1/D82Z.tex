\uuid{D82Z}
\exo7id{786}
\titre{exo7 786}
\auteur{gourio}
\organisation{exo7}
\datecreate{2001-09-01}
\isIndication{false}
\isCorrection{false}
\chapitre{Calcul d'intégrales}
\sousChapitre{Théorie}
\module{Analyse}
\niveau{L1}
\difficulte{}

\contenu{
\texte{

}
\begin{enumerate}
    \item \question{Soit $(a,b)\in (\Nn^{*})^{2}$, $n\in \Nn^{*},$ montrer que le
polyn\^{o}me $P_{n}=\frac{X^{n}(bX-a)^{n}}{n!}$ et ses d\'{e}riv\'{e}es
successives prennent, en $0$ et $\frac{a}{b}$, des valeurs enti\`{e}res.}
    \item \question{Montrer que :
$$I_{n}=\int_{0}^{\pi }P_{n}(t)\sin (t)dt\rightarrow 0\text{ quand }
n\rightarrow \infty . $$}
    \item \question{Montrer par l'absurde que $\pi \in {\Rr}\setminus\Qq.$}
\end{enumerate}
}

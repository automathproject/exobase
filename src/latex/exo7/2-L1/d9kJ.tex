\uuid{d9kJ}
\exo7id{1267}
\auteur{legall}
\organisation{exo7}
\datecreate{1998-09-01}
\video{MY2kvANRkJo}
\isIndication{true}
\isCorrection{true}
\chapitre{Développement limité}
\sousChapitre{Formule de Taylor}

\contenu{
\texte{
Soit $f$ l'application de $\Rr$ dans $\Rr$ définie par
$f(x)=\displaystyle{\frac{x^3}{1+x^6}}.$ Calculer $f^{(n)}(0)$ pour tout $n \in \Nn.$
}
\indication{Calculer d'abord le dl puis utiliser une formule de Taylor.}
\reponse{
Habituellement on trouve le développement limité d'une fonction 
à partir des dérivées successives. Ici on va faire l'inverse.

Calcul du dl (à un certain ordre) :
\begin{align*}
f(x) & = \frac{x^3}{1+x^6} = x^3 \frac{1}{1+x^6}\\
     & =  x^3 \left( 1-x^6 + x^{12} - \cdots \pm x^{6\ell} \cdots \right) \\
     &= x^3 - x^9 + x^{15} -\cdots \pm x^{3+6\ell} \cdots \\
     &= \sum_{\ell \ge 0} (-1)^\ell x^{3+6\ell} \\
\end{align*}

Il s'agit d'identifier ce développement avec la formule de Taylor :
$$f(x)=f(0)+f'(0)x+\frac{f''(0)}{2!}x^2 + \cdots + \frac{f^{(n)}(0)}{n!} x^n + \cdots$$


Par unicité des DL, en identifiant les coefficients devant $x^n$ on trouve :
$$\begin{cases}
\frac{f^{(n)}(0)}{n!} = (-1)^\ell & \text{ si } n = 3+6\ell \\
\frac{f^{(n)}(0)}{n!} = 0         & \text{ sinon.} \\    
\end{cases}$$

Si $n=3+6\ell$ (avec $\ell\in \Nn$) alors on peut écrire $\ell = \frac{n-3}{6}$
et donc on peut conclure :
$$\begin{cases}
f^{(n)}(0) = (-1)^{\frac{n-3}{6}} \cdot n! & \text{ si } n \equiv 3 \pmod{6} \\
f^{(n)}(0) = 0         & \text{ sinon.} \\    
\end{cases}$$
}
}

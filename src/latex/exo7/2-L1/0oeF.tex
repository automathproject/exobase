\uuid{0oeF}
\exo7id{4003}
\titre{exo7 4003}
\auteur{quercia}
\organisation{exo7}
\datecreate{2010-03-11}
\isIndication{false}
\isCorrection{true}
\chapitre{Dérivabilité des fonctions réelles}
\sousChapitre{Fonctions convexes}
\module{Analyse}
\niveau{L1}
\difficulte{}

\contenu{
\texte{
Soit $f$ continue et croissante sur $\R^+$. On pose
$F(x)= \int_0^x f$, et l'on suppose que $F(x)=x^2+ o(x)$. Montrer que
$f(x)=2x+ o(\sqrt x)$.
}
\reponse{
Soit $F(x) = x^2 + xG(x)$.
On a pour $h>0$~: $f(x) \le \frac{F(x+xh)-F(x)}{xh} = 2x + xh + \frac{G(x+xh)-G(x)}h + G(x+xh)$.
Soit $\varepsilon>0$ et $A$ tel que $y\ge A \Rightarrow |G(y)|\le\varepsilon^2$. On prend
$h=\varepsilon/\sqrt{x}$ et on obtient
$f(x)- 2x -\varepsilon\sqrt x \le \varepsilon\sqrt x + \varepsilon^2$
d'où $f(x)\le 2x+ o(\sqrt x)$. L'inégalité inverse se montre de même.
}
}

\uuid{2dru}
\exo7id{4043}
\titre{exo7 4043}
\auteur{quercia}
\organisation{exo7}
\datecreate{2010-03-11}
\isIndication{false}
\isCorrection{true}
\chapitre{Développement limité}
\sousChapitre{Développements limités implicites}

\contenu{
\texte{
Pour tout $n$ entier naturel non nul, on donne $f_n(x) = nx^{n+1} - (n+1)x^n - \frac12$.
}
\begin{enumerate}
    \item \question{Montrer que $f_n$ admet une unique racine positive notée $x_n$.}
    \item \question{Montrer que la suite $(x_n)$ converge vers une limite~$\ell$ et trouver
    un équivalent de $x_n-\ell$.}
\reponse{
\begin{align*} nx_n^n(x_n - 1) = x_n^n+\frac 12 & \Rightarrow 
    f_{n+1}(x_n) = \frac{n+1}n\Bigl(x_n^{n+1} + \frac{x_n}2\Bigr) - x_n^{n+1} - \frac12
                 = \frac{x_n^{n+1}}n + \frac{(n+1)x_n - n}{\strut 2n} > 0\\ & \Rightarrow  x_{n+1} < x_n.\\ \end{align*}
    Donc la suite $(x_n)$ est décroissante et minorée par~$1$, elle converge vers~$\ell \ge 1$.
    
    $0 \le x_n - 1 = \frac 1n+ \frac1{2nx_n^n} \to 0$ (lorsque $n\to\infty$) donc $\ell = 1$.
    
    Soit $y_n = n(x_n-1) = 1 + \frac1{2x_n^n}$.
    On a $f(y_n) = \frac{\ln(2(y_n-1))}{y_n} = -\frac{\ln x_n}{x_n-1} = -g(x_n)$
    et $f,g$ sont strictement coissantes sur $[1,+\infty[$ donc les suites $(x_n)$
    et $(y_n)$ varient en sens contraire. On en déduit que la suite $(y_n)$
    décroît donc admet une limite $\lambda \ge 1$,
    soit $x_n = 1 + \frac\lambda n + o\Bigl(\frac1n\Bigr)$.
    Alors $x_n^n \to e^\lambda$ (lorsque $n\to\infty$) d'où $\lambda = 1 + \frac1{2e^\lambda}$.
}
\end{enumerate}
}

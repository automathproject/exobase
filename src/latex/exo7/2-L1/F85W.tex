\uuid{F85W}
\exo7id{758}
\auteur{gourio}
\organisation{exo7}
\datecreate{2001-09-01}
\isIndication{true}
\isCorrection{true}
\chapitre{Fonctions circulaires et hyperboliques inverses}
\sousChapitre{Fonctions hyperboliques et hyperboliques inverses}

\contenu{
\texte{

}
\begin{enumerate}
    \item \question{Montrer qu'il n'existe pas de fonction $f:[1;+\infty [\rightarrow \Rr$
v\'{e}rifiant:
$$ \forall x\in \Rr, \quad f(\ch x)=e^{x}. $$}
\reponse{Si $f$ existe alors pour $x=1$ on a $f(\ch 1)= e$
et pour $x=-1$ on a $f(\ch -1)=f(\ch 1)=1/e$. Une fonction ne peut prendre deux valeurs diff\'erentes au m\^eme point (ici $t=\ch1$).}
    \item \question{D\'{e}terminer toutes les fonctions $f:\Rr^{+*}\rightarrow \Rr$ telles que:
$$ \forall x\in \Rr, \quad f(e^{x})=\ch x.$$
Pr\'{e}ciser le nombre de solutions.}
\reponse{Notons $X = e^x$, l'\'equation devient 
$$f(X)=\frac{e^x+e^{-x}}{2} = \frac12(X+\frac1X).$$
Comme la fonction exponentielle est une bijection de $\Rr$ sur $]0,+\infty[$, alors l'unique fa\c{c}on de d\'efinir $f$ sur $]0,+\infty[$ est par la formule $f(t)=\frac12(t+\frac1t)$.}
    \item \question{D\'{e}terminer toutes les fonctions $f:\Rr^{+}\rightarrow \Rr$ telles que:
$$\forall x\in \Rr, \quad f(e^{x})=\ch x.$$
Pr\'{e}ciser le nombre de solutions ; y a t-il des solutions continues sur $\Rr^{+}$ ?}
\reponse{Comme $e^x$ est toujours non nul, alors $f$ peut prendre n'importe quelle valeur en $0$. $f(0)=c \in \Rr$ et
$f(t)=\frac12(t+\frac1t)$ pour $t>0$. Il y a une infinit\'e de solutions.
Mais aucune de ces solutions n'est continue car la limite de $f(t)$
quand $t>0$ et $t\rightarrow 0$ est $+\infty$.}
\indication{\begin{enumerate}
  \item Regarder ce qui se passe en deux valeurs oppos\'ees $x$ et $-x$.
  \item Poser $X = e^x$.
 \end{enumerate}}
\end{enumerate}
}

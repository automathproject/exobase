\uuid{hzO4}
\exo7id{5710}
\titre{exo7 5710}
\auteur{rouget}
\organisation{exo7}
\datecreate{2010-10-16}
\isIndication{false}
\isCorrection{true}
\chapitre{Série numérique}
\sousChapitre{Séries alternées}

\contenu{
\texte{
Calculer $\sum_{n=0}^{+\infty}\frac{(-1)^n}{3n+1}$.
}
\reponse{
La suite $\left((-1)^n\frac{1}{3n+1}\right)_{n\in\Nn}$ est alternée en signe et sa valeur absolue tend vers $0$ en décroissant. Donc la série de terme général $(-1)^n\frac{1}{3n+1}$, $n\geqslant 1$, converge en vertu du critère spécial aux séries alternées.

Soit $n\in\Nn$.

\begin{center}
$\sum_{k=0}^{n}\frac{(-1)^k}{3k+1}=\sum_{k=0}^{n}(-1)^k\int_{0}^{1}t^{3k}\;dt=\int_{0}^{1}\frac{1-(-t^3)^{n+1}}{1-(-t^3)}\;dt=\int_{0}^{1}\frac{1}{1+t^3}\;dt+(-1)^n\int_{0}^{1}\frac{t^{3n+3}}{1+t^3}\;dt$.
\end{center}

Mais $\left|(-1)^n\int_{0}^{1}\frac{t^{3n+3}}{1+t^3}\;dt\right|=\int_{0}^{1}\frac{t^{3n+3}}{1+t^3}\;dt\leqslant\int_{0}^{1}t^{3n+3}\;dt=\frac{1}{3n+4}$. On en déduit que $(-1)^n\int_{0}^{1}\frac{t^{3n+3}}{1+t^3}\;dt$ tend vers $0$ quand $n$ tend vers $+\infty$ et donc que

\begin{center}
$\sum_{n=0}^{+\infty}\frac{(-1)^n}{3n+1}=\int_{0}^{1}\frac{1}{1+t^3}\;dt$.
\end{center}

Calculons cette dernière intégrale.

\begin{align*}\ensuremath
\frac{1}{X^3+1}&=\frac{1}{(X+1)(X+j)(X+j^2)}=\frac{1}{3}\left(\frac{1}{X+1}+\frac{j}{X+j}+\frac{j^2}{X+j^2}\right)=\frac{1}{3}\left(\frac{1}{X+1}+\frac{-X+2}{X^2-X+1}\right)\\
 &\frac{1}{3}\left(\frac{1}{X+1}-\frac{1}{2}\frac{2X-1}{X^2-X+1}+
 \frac{3}{2}\frac{1}{\left(X-\frac{1}{2}\right)^2+\left(\frac{\sqrt{3}}{2}\right)^2}
 \right).
\end{align*}

 
Donc, 

\begin{center}
$\sum_{n=0}^{+\infty}\frac{(-1)^n}{3n+1}=\frac{1}{3}\left[\ln(t+1)-\frac{1}{2}\ln(t^2-t+1)+\sqrt{3}\Arctan\left(\frac{2t-1}{\sqrt{3}}\right)\right]_0^1=\frac{1}{3}\left(\ln2+\sqrt{3}\left(\frac{\pi}{6}-\left(-\frac{\pi}{6}\right)\right)\right)=\frac{3\ln2+\pi\sqrt{3}}{9}$.
\end{center}

\begin{center}
\shadowbox{
$\sum_{n=0}^{+\infty}\frac{(-1)^n}{3n+1}=\frac{3\ln2+\pi\sqrt{3}}{9}$.
}
\end{center}
}
}

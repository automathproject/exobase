\uuid{axTV}
\exo7id{4045}
\titre{exo7 4045}
\auteur{quercia}
\organisation{exo7}
\datecreate{2010-03-11}
\video{OchDDAXSqog}
\isIndication{true}
\isCorrection{true}
\chapitre{Développement limité}
\sousChapitre{Equivalents}

\contenu{
\texte{
Trouver $a,b\in\R$ tels que 
$$\cos x - \frac{1+ax^2}{1+bx^2}$$
soit un $o(x^n)$ en $0$ avec $n$ maximal.
}
\indication{Identifier les dl de $\cos x$ et $\frac{1+ax^2}{1+bx^2}$ en $x=0$.}
\reponse{
Le dl de $\cos x$ en $0$ à l'ordre $6$ est :
$$\cos x = 1 - \frac{1}{2!} x^2 + \frac{1}{4!}x^4  - \frac{1}{6!} x^6 + o(x^6).$$

Calculons celui de $\frac{1+ax^2}{1+bx^2}$ :

\begin{align*}
\frac{1+ax^2}{1+bx^2} 
  & = (1+ax^2) \times \frac{1}{1+bx^2} \\
  & = (1+ax^2)\times \big(1-bx^2+b^2x^4-b^3x^6+o(x^6) \big) \quad \text{ car } \frac{1}{1+u} = 1-u+u^2 - u^3+o(u^3) \\
  &= \ \ \cdots  \qquad \text{ on développe } \\
  &= 1 + (a-b) x^2 - b(a-b) x^4 + b^2(a-b) x^6 + o(x^6) \\
\end{align*}

Notons $\Delta(x) = \cos x - \frac{1+ax^2}{1+bx^2}$ alors 
$$\Delta(x) = \big(-\frac12-(a-b)\big)x^2 + \big(\frac{1}{24} + b(a-b)\big) x^4 
+ \big(-\frac{1}{720}-b^2(a-b)\big) x^6 + o(x^6).$$

Pour que cette différence soit la plus petite possible (lorsque $x$ est proche de $0$)
il faut annuler le plus possible de coefficients de bas degré.
On souhaite donc avoir 
$$-\frac12-(a-b) = 0 \qquad \text{et} \qquad \frac{1}{24} + b(a-b)=0.$$
En substituant l'égalité de gauche dans celle de droite on trouve :
$$a=-\frac{5}{12}  \qquad \text{et} \qquad b=\frac{1}{12}.$$

On obtient alors 
$$\Delta(x) =  \big(-\frac{1}{720}-b^2(a-b)\big) x^6 + o(x^6) = \frac{1}{480} x^6 + o(x^6).$$

\bigskip

Avec notre choix de $a,b$ nous avons obtenu une très bonne approximation de $\cos x$.
Par exemple lorsque l'on évalue  $\frac{1+ax^2}{1+bx^2}$ (avec $a=-\frac{5}{12}$ et $b=\frac{1}{12}$)
en $x=0.1$ on trouve :
$$0.9950041631\ldots$$
Alors que 
$$\cos(0.1)=0.9950041652\ldots$$
En l'on trouve ici $\Delta(0.1) \simeq 2\times 10^{-9}$.
}
}

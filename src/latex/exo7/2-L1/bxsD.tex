\uuid{bxsD}
\exo7id{5705}
\titre{exo7 5705}
\auteur{rouget}
\organisation{exo7}
\datecreate{2010-10-16}
\isIndication{false}
\isCorrection{true}
\chapitre{Série numérique}
\sousChapitre{Série à  termes positifs}
\module{Analyse}
\niveau{L1}
\difficulte{}

\contenu{
\texte{
Déterminer un équivalent simple de $\frac{n!}{(a+1)(a+2)\ldots(a+n)}$ quand $n$ tend vers l'infini ($a$ réel positif donné).
}
\reponse{
(On applique la règle de \textsc{Raabe}-\textsc{Duhamel} qui n'est pas un résultat de cours.) Pour $n\in\Nn$, posons $u_n=\frac{n!}{(a+1)(a+2)\ldots(a+n)}$.

\begin{center}
$\frac{u_{n+1}}{u_n}=\frac{n+1}{a+n+1}=\left(1+\frac{1}{n}\right)\left(1+\frac{a+1}{n}\right)^{-1}\underset{n\rightarrow+\infty}{=}\left(1+\frac{1}{n}\right)\left(1-\frac{a+1}{n}+O\left(\frac{1}{n^2}\right)\right)\underset{n\rightarrow+\infty}{=}1-\frac{a}{n}+O\left(\frac{1}{n^2}\right)$,
\end{center}

et \og on sait \fg~qu'il existe un réel strictement positif $K$ tel que $u_n\underset{n\rightarrow+\infty}{\sim}\frac{K}{n^a}$.
}
}

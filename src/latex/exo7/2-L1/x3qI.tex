\uuid{x3qI}
\exo7id{5924}
\titre{exo7 5924}
\auteur{tumpach}
\organisation{exo7}
\datecreate{2010-11-11}
\isIndication{false}
\isCorrection{true}
\chapitre{Calcul d'intégrales}
\sousChapitre{Théorie}

\contenu{
\texte{

}
\begin{enumerate}
    \item \question{Montrer que si $f~:[a,b]\rightarrow\mathbb{R}$ est
Riemann-int\'egrable, alors
\begin{equation*}\int_{a}^{b}f(x)\,dx =
\int_{a}^{b}f(a+b-x)\,dx. \end{equation*}}
\reponse{\begin{multline*}
\overset{b}{\underset{a}{\int }}f(a+b-x)dx=-\overset{b}{\underset{a}{\int }}%
f(a+b-x)(a+b-x)^{\prime }dx\\
=-\overset{\varphi (b)}{\underset{\varphi (a)}{%
\int }}f(t)dt=-\overset{a}{\underset{b}{\int }}f(t)dt=\overset{b}{\underset{a%
}{\int }}f(t)dt
\end{multline*}

o\`{u} $\varphi :[a,b]\rightarrow \lbrack a,b]$, $\varphi
(x)=a+b-x$ est une fonction de classe $C^{1}$.}
    \item \question{Calculer (en
utilisant 1.) les int\'egrales suivantes~:
\begin{equation*}
a)\quad\int_{0}^{\pi}\frac{x\sin
x}{1+{\cos}^{2}x}\,dx\quad\quad\quad\quad
b)\quad\int_{0}^{\frac{\pi}{4}}\log\left(1+\tan x\right)\,dx.
\end{equation*}}
\reponse{$a)$
\begin{multline*}
I:=\overset{\pi }{\underset{0}{\int }}\frac{x\sin x}{1+\cos ^{2}x}dx=\overset%
{\pi }{\underset{0}{\int }}\frac{(\pi -x)\sin (\pi -x)}{1+\cos ^{2}(\pi -x)}%
dx=\overset{\pi }{\underset{0}{\int }}\frac{(\pi -x)\sin x}{1+\cos ^{2}x}%
dx\\
=\pi \overset{\pi }{\underset{0}{\int }}\frac{\sin x}{1+\cos
^{2}x}dx-I
\end{multline*}

\begin{multline*}
I =\frac{\pi }{2}\overset{\pi }{\underset{0}{\int }}\frac{\sin
x}{1+\cos ^{2}x}dx=-\frac{\pi }{2}\overset{\pi }{\underset{0}{\int
}}\frac{(\cos
x)^{\prime }}{1+\cos ^{2}x}dx=-\frac{\pi }{2}\overset{\varphi (\pi )}{%
\underset{\varphi (0)}{\int }}\frac{1}{1+t^{2}}dt\\
=-\frac{\pi }{2}\overset{-1}%
{\underset{1}{\int }}\frac{1}{1+t^{2}}dt=\frac{\pi }{2}\overset{1}{\underset{-1}{\int }}\frac{1}{1+t^{2}}dt=\frac{%
\pi ^{2}}{4}.
\end{multline*}

o\`{u} $\varphi :[0,\pi ]\rightarrow \lbrack -1,1]$, $\varphi
(x)=\cos x$ est une fonction de classe $C^{1}$.

$b)$
\begin{multline*}
J :=\overset{\pi /4}{\underset{0}{\int }}\log (1+\tan x)dx=\overset{\pi /4}{%
\underset{0}{\int }}\log \left(1+\tan (\frac{\pi }{4}-x)\right)dx\\
=\overset{\pi /4}{%
\underset{0}{\int }}\log \left(1+\frac{1-\tan x}{1+\tan x}\right)dx
=\overset{\pi /4}{\underset%
{0}{\int }}\log \left(\frac{2}{1+\tan x}\right)dx =\frac{\pi }{4}\log
2-J
\end{multline*}

d'o\`{u} la valeur de l'int\'{e}grale est $J=\frac{\pi }{8}\log
2.$}
\end{enumerate}
}

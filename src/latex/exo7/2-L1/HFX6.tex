\uuid{HFX6}
\exo7id{1199}
\titre{exo7 1199}
\auteur{legall}
\organisation{exo7}
\datecreate{1998-09-01}
\isIndication{false}
\isCorrection{false}
\chapitre{Suite}
\sousChapitre{Convergence}
\module{Analyse}
\niveau{L1}
\difficulte{}

\contenu{
\texte{
Soit $  (u_n)_{n\in { \Nn}}  $ une suite de nombres r\'eels et $
\displaystyle{v_n=\frac{u_1+u_2+\cdots+u_n}{ n}}  $ o\`u $  n\in { \Nn}^*  .$
}
\begin{enumerate}
    \item \question{Montrer que si $  (u_n)_{n\geq 1}  $ converge vers $  \ell   ,$ alors
$  (v_n)_{n\geq 1}  $ converge vers $  \ell   .$ La r\'eciproque est elle vraie~?}
    \item \question{Calculer $  \displaystyle{ \lim_{n \rightarrow +\infty } \sum _{k=1}^n \frac{k+1 }{ 2nk+k}}  .$}
    \item \question{Soit $  (a_n)_{n\geq 0}  $ une suite telle que
$  \displaystyle{ \lim_{n\rightarrow +\infty}(a_{n+1}-a_n)=\ell }  .$ Prouver que $
\displaystyle{\lim_{n\rightarrow +\infty}\frac{a_n}{ n}=\ell}  .$}
    \item \question{Soit $  (u_n)_{n\geq 1}  $ une suite strictement positive telle
que $  \displaystyle{\lim_{n\rightarrow
+\infty}\frac{u_{n+1}}{ u_n}=\ell}  .$ D\'emontrer que
$  \displaystyle{\lim_{n\rightarrow +\infty}(u_n)^{1/n} =\ell }  .$}
\end{enumerate}
}

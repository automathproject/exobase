\uuid{SaGe}
\exo7id{5920}
\titre{exo7 5920}
\auteur{tumpach}
\organisation{exo7}
\datecreate{2010-11-11}
\isIndication{false}
\isCorrection{true}
\chapitre{Calcul d'intégrales}
\sousChapitre{Théorie}
\module{Analyse}
\niveau{L1}
\difficulte{}

\contenu{
\texte{

}
\begin{enumerate}
    \item \question{Montrer que la fonction $f~:[0,1]\rightarrow \mathbb{R}$
d\'efinie par~:
$$
 f(x) = \left\{\begin{array}{l} 1 \quad\text{si}\quad  x\in\mathbb{Q}\\
0 \quad\text{si} \quad x\in\mathbb{R}\setminus\mathbb{Q}
\end{array}\right.
$$
n'est pas Riemann-int\'egrable sur $[0,1]$.}
\reponse{Consid\'erons la fonction $f~:[0,1]\rightarrow \mathbb{R}$
d\'efinie par~:
$$
 f(x) = \left\{\begin{array}{l} 1 \quad\text{si}\quad  x\in\mathbb{Q}\\
0 \quad\text{si} \quad x\in\mathbb{R}\setminus\mathbb{Q}
\end{array}\right..
$$
Pour toute subdivision $\sigma$ de $[a,b]$, on a~:
\begin{equation*}
\overline{S}_{f}^{\sigma} = 1 \quad \text{et}\quad
\underline{S}_{f}^{\sigma} = 0.
\end{equation*}
On en d\'eduit que $1 = \sup_{\sigma}\overline{S}_{f}^{\sigma}
\neq \inf_{\sigma}\underline{S}_{f}^{\sigma} = 0$, ce qui implique
que $f$ n'est pas Riemann-int\'egrable sur $[0, 1]$.}
    \item \question{Montrer que la
fonction $g~:[0,1]\rightarrow \mathbb{R}$ d\'efinie par~:
$$
 g(x) = \left\{\begin{array}{l} \frac{1}{q} \quad\text{si}\quad
 \, x=\frac{p}{q} \quad\text{avec}\,\, p \,\,\text{et} \,\,q \,\,\text{premiers entre eux}\\
0 \quad\text{si} \quad x\in\mathbb{R}\setminus\mathbb{Q} \text{ ou } x=0
\end{array}\right.
$$
est Riemann-int\'egrable sur $[0,1]$.}
\reponse{Consid\'erons la fonction $g~:[0,1]\rightarrow \mathbb{R}$
d\'efinie par~:
$$
 g(x) = \left\{\begin{array}{l} \frac{1}{q} \quad\text{si}\quad
 \, x=\frac{p}{q} \quad\text{avec}\,\, p \,\,\text{et} \,\,q \,\,\text{premiers entre eux}\\
0 \quad\text{si} \quad x\in\mathbb{R}\setminus\mathbb{Q} \text{ ou } x=0
\end{array}\right..
$$
Pour toute subdivision $\sigma$ de $[a,b]$, on a~:
\begin{equation*}
\underline{S}_{g}^{\sigma} = 0.
\end{equation*}
Pour tout $\varepsilon>0$ donn\'e, la fonction $g$ prend des
valeurs sup\'erieures \`a $\frac{\varepsilon}{b-a}$ en un nombre fini de
points seulement (les points $\frac{k}{q}$, avec
$\frac{1}{q}>\frac{\varepsilon}{b-a}$ ce qui équivaut à $q < \frac{b-a}{\varepsilon}$). Notons $x_{i}$, $i= 1,\dots, p$ ces
points ordonn\'es par ordre (strictement) croissant. 

Sur $[0,1]\setminus \{x_1,\ldots,x_p\}$ la fonction $g$ prend des valeurs
$\le \varepsilon$ et $\ge 0$. Ainsi avec la subdivision
$\sigma = \{x_1,\ldots,x_p\}$ nous obtenons :
\begin{equation*}
 0 \leq \overline{S}_{g}^{\sigma} \leq \frac{\varepsilon}{b-a} (b-a) = \varepsilon
\end{equation*}

% On a $p \leq \frac{1}{\varepsilon^2}$.
%  Notons \'egalement $\delta$ le minimum
% de la distance entre deux de ces points. Consid\'erons la
% subdivision $\sigma=\{a_{0}=a<\dots<a_{2p}=b\}$ d\'efinie par
% $a_{2k-1}=x_{i}-\min(\frac{\delta}{3}, \varepsilon^3)$ pour $k=1,
% \dots, p$, $a_{2k}=x_{i}+\min(\frac{\delta}{3},\varepsilon^3)$
% pour $k = 1, \dots, p-1$, et $a_{2p} = b$. Alors~:
% \begin{equation*}
% \overline{S}_{g}^{\sigma} \leq \varepsilon +
% p\min\left(\frac{\delta}{3},\varepsilon^3\right) \leq \varepsilon
% + \frac{1}{\varepsilon^{2}}\varepsilon^3 \leq 2\varepsilon.
% \end{equation*}

Comme On en conclut que $g$ est Riemann-int\'egrable sur $[0,1]$.}
\end{enumerate}
}

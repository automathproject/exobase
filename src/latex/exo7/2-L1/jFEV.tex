\uuid{jFEV}
\exo7id{5435}
\titre{exo7 5435}
\auteur{rouget}
\organisation{exo7}
\datecreate{2010-07-06}
\isIndication{false}
\isCorrection{true}
\chapitre{Suite}
\sousChapitre{Suite définie par une relation de récurrence}

\contenu{
\texte{
Soit $u_0\in]0,\frac{\pi}{2}]$. Pour $n\in\Nn$, on pose $u_{n+1}=\sin(u_n)$.
}
\begin{enumerate}
    \item \question{Montrer brièvement que la suite $u$ est strictement positive et converge vers $0$.}
    \item \question{\begin{enumerate}}
    \item \question{Déterminer un réel $\alpha$ tel que la suite $u_{n+1}^\alpha-u_n^\alpha$ ait une limite finie non nulle.}
    \item \question{En utilisant le lemme de \textsc{Cesaro}, déterminer un équivalent simple de $u_n$.}
\reponse{
Pour $x\in\left[0,\frac{\pi}{2}\right]$, posons $f(x)=\sin x$. On a $f\left(\left]0,\frac{\pi}{2}\right]\right)=]0,1]\subset\left]0,\frac{\pi}{2}\right]$. Donc, puisque $u_0\in\left]0,\frac{\pi}{2}\right]$, on en déduit que $\forall n\in\Nn,\;u_n\in\left]0,\frac{\pi}{2}\right]$.\rule[-5mm]{0mm}{10mm}
Il est connu que $\forall x\in\left]0,\frac{\pi}{2}\right]$, $\sin x<x$ et de plus, pour $x\in\left[0,\frac{\pi}{2}\right]$, $\sin x=x\Leftrightarrow x=0$.
La suite $u$ est à valeurs dans $\left]0,\frac{\pi}{2}\right]$ et donc $\forall n\in\Nn,\;u_{n+1}=\sin(u_n)<u_n$. La suite $u$ est donc strictement décroissante et, étant minorée par $0$, converge vers un réel $\ell$ de $\left[0,\frac{\pi}{2}\right]$ qui vérifie ($f$ étant continue sur le segment $\left[0,\frac{\pi}{2}\right]$) $f(\ell)=\ell$ ou encore $\ell=0$.
En résumé,

\begin{center}
\shadowbox{
la suite $u$ est strictement positive, strictement décroissante et converge vers $0$.
}
\end{center}
Soit $\alpha$ un réel quelconque. Puisque la suite $u$ tend vers 0 , on a

\begin{align*}\ensuremath
u_{n+1}^{\alpha}-u_n^{\alpha}=(\sin u_n)^{\alpha}-u_n^{\alpha}&\underset{n\rightarrow+\infty}{=}\left(u_n-\frac{u_n^3}{6}+o(u_n^3)\right)^{\alpha}-u_n^{\alpha}\\
 &=u_n^{\alpha}\left(\left(1-\frac{u_n^2}{6}+o(u_n^2)\right)^{\alpha}-1\right)=u_n^{\alpha}\left(-\alpha\frac{u_n^2}{6}+o(u_n^2)\right)\\
 &=-\alpha\frac{u_n^{\alpha+2}}{6}+o(u_n^{\alpha+2})
\end{align*}
Pour $\alpha=-2$ on a donc 

$$\frac{1}{u_{n+1}^2}-\frac{1}{u_n^2}=\frac{1}{3}+o(1).$$
D'après le lemme de \textsc{Cesaro}, $\frac{1}{n}\sum_{k=0}^{n-1}\left(\frac{1}{u_{k+1}^2}-\frac{1}{u_k^2}\right)=\frac{1}{3}+o(1)$ ou encore $\frac{1}{n}\left(\frac{1}{u_n^2}-\frac{1}{u_0^2}\right)=\frac{1}{3}+o(1)$ ou enfin, 

\begin{center}
$\frac{1}{u_n^2}\underset{n\rightarrow+\infty}{=}\frac{n}{3}+\frac{1}{u_0^2}+o(n)\underset{n\rightarrow+\infty}{=}\frac{n}{3}+o(n)\underset{n\rightarrow+\infty}{\sim}\frac{n}{3}$.
\end{center}
Par suite, puisque la suite $u$ est strictement positive, 

\begin{center}
\shadowbox{
$u_n\underset{n\rightarrow+\infty}{\sim}\sqrt{\frac{3}{n}}.$
}
\end{center}
}
\end{enumerate}
}

\uuid{HEeS}
\exo7id{4682}
\titre{exo7 4682}
\auteur{quercia}
\organisation{exo7}
\datecreate{2010-03-16}
\isIndication{false}
\isCorrection{false}
\chapitre{Suite}
\sousChapitre{Convergence}

\contenu{
\texte{
Soit $(u_n)$ une suite r{\'e}elle. On pose $v_n = \frac {u_1 + \dots + u_n}n$.
}
\begin{enumerate}
    \item \question{Montrer que si $u_n \xrightarrow[n\to\infty]{} 0$, alors $v_n \xrightarrow[n\to\infty]{} 0$.}
    \item \question{Montrer que si $u_n \xrightarrow[n\to\infty]{} \ell$, alors $v_n \xrightarrow[n\to\infty]{} \ell$.
     ($\ell \in \overline{\R}$)}
    \item \question{Donner un exemple o{\`u} $(v_n)$ converge mais $(u_n)$ diverge.}
\end{enumerate}
}

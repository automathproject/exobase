\uuid{rdzt}
\exo7id{535}
\titre{exo7 535}
\auteur{cousquer}
\organisation{exo7}
\datecreate{2003-10-01}
\isIndication{false}
\isCorrection{false}
\chapitre{Suite}
\sousChapitre{Convergence}

\contenu{
\texte{
Un ivrogne part à un instant donné d'un point donné. À chaque seconde,
il fait un pas dans une direction inconnue (et qui peut changer de façon
arbitraire à chaque pas). Comme il se fatigue, ses pas sont de plus en plus
courts. Peut-on prévoir qu'au bout d'un certain temps il restera à moins
d'un mètre d'une certaine position si on admet que la longueur de son
$n$-ième pas est~:
}
\begin{enumerate}
    \item \question{$1/n$ mètre~?}
    \item \question{$1/n^2$ mètre~?}
\end{enumerate}
}

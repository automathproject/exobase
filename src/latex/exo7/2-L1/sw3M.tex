\uuid{sw3M}
\exo7id{4290}
\titre{exo7 4290}
\auteur{quercia}
\organisation{exo7}
\datecreate{2010-03-12}
\isIndication{false}
\isCorrection{true}
\chapitre{Calcul d'intégrales}
\sousChapitre{Intégrale impropre}
\module{Analyse}
\niveau{L1}
\difficulte{}

\contenu{
\texte{
Soit $f : {\R^+} \to \R$. On pose, sous réserve de convergence,
$g(t) = \sum_{n=0}^\infty f(nt)$ pour $t>0$.
}
\begin{enumerate}
    \item \question{Si $f$ est monotone et intégrable, montrer que $g(t)$ existe
pour tout $t>0$ et que l'on a $\smash{tg(t)\to  \int_{u=0}^{+\infty} f(u)\,d u}$ lorsque $t\to0^+$.}
\reponse{En supposant $f$ positive décroissante, $ \int_0^{+\infty}f \le tg(t) \le tf(0) +  \int_0^{+\infty}f$.}
    \item \question{Même question en supposant $f$ de classe $\mathcal{C}^1$ et $f$,$f'$ intégrables.}
\reponse{$ \int_{u=Pt}^{Qt}f(u)\,d u - \sum_{n=P}^{Q-1}tf(nt) =  \int_{u=Pt}^{Qt}(f(u)-f(t[u/t]))\,d u =  \int_{u=Pt}^{Qt}t(1-\{u/t\})f'(u)\,d u\to 0$ lorsque $P,Q\to\infty$.

Donc la série de terme général $tf(nt)$ est de Cauchy~; elle converge.

On a alors ${ \int_0^{+\infty}\!\!f} - tg(t) =  \int_{u=0}^{+\infty}t(1-\{u/t\})f'(u)\,d u\to 0$ lorsque $t\to0^+$.}
    \item \question{On suppose maintenant $f$ de classe $\mathcal{C}^2$ et $f$, $f'$, $f''$ intégrables.

Montrer que $g(t) = \frac1t \int_{u=0}^{+\infty} f(u)\,d u + \frac{f(0)}2 + O_{t\to0^+}(t)$.}
\reponse{${ \int_0^{+\infty}\!\!2f} - 2tg(t) =  \int_{u=0}^{+\infty}tf'(u)\,d u +  \int_{u=0}^{+\infty}t(1-2\{u/t\})f'(u)\,d u = tf(0) -  \int_{u=0}^{+\infty}t^2\{u/t\}(1-\{u/t\})f''(u)\,d u$.}
\end{enumerate}
}

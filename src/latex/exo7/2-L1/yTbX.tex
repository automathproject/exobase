\uuid{yTbX}
\exo7id{6980}
\titre{exo7 6980}
\auteur{blanc-centi}
\organisation{exo7}
\datecreate{2014-05-06}
\video{QkvgaXwOwyU}
\isIndication{true}
\isCorrection{true}
\chapitre{Fonctions circulaires et hyperboliques inverses}
\sousChapitre{Fonctions hyperboliques et hyperboliques inverses}
\module{Analyse}
\niveau{L1}
\difficulte{}

\contenu{
\texte{
Montrer que l'équation $\Argsh x+\Argch x=1$ admet une unique solution, puis la déterminer.
}
\indication{Faire le tableau de variations de $f:x\mapsto\Argsh x+\Argch x$.}
\reponse{
Soit $f(x)=\Argsh x+\Argch x$. La fonction $f$ est bien définie, continue, et strictement croissante, 
sur $[1,+\infty[$ (comme somme de deux fonctions continues strictement croissantes). 

\begin{center}
\begin{tikzpicture}[scale=2]
      \draw[->,>=latex, gray] (-0.5,0)--(2.5,0) node[below,black] {$x$};
      \draw[->,>=latex, gray] (0,-0.5)--(0,3) node[right,black] {$y$};  
 
      \draw[ultra thick, color=red,domain=1.0:2,samples=100,smooth] plot (\x,{ln(\x+sqrt(\x^2+1)) + ln(\x+sqrt(\x^2-1))}) node[above left] {$f(x)$}; 

   \draw (0,1.1)--(1.03,1.1)--(1.03,0);

   
   \draw (0,0.9)--(0.98,0.9)--(0.98,0);
   

     \fill (0,0) circle (1pt);
     \fill (0,0.9) circle (1pt); 
     \fill (0.96,0) circle (1pt);
     \fill (1.04,0) circle (1pt);
     
     \fill (0,1.1) circle (1pt);     
     \node at (0,1)[above left] {$1$};
     \node at (0,0.9)[below left] {$f(1)$};
     \node at (1,0)[below left] {$1$};
     \node at (1.03,0)[below right] {$a$};
     \node at (0,0)[below right] {$0$};
\end{tikzpicture}  
\end{center}

De plus, $f(x)\xrightarrow[x\to +\infty]{}+\infty$, donc $f$ atteint exactement une fois 
toute valeur de l'intervalle $[f(1),+\infty[$. Comme (par la formule logarithmique)
$f(1)=\Argsh 1=\ln(1+\sqrt{2})<\ln(e)=1$, 
on a $1\in[f(1),+\infty[$. Par le théorème des valeurs intermédiaires
l'équation $f(x)=1$ admet une unique solution, que l'on notera $a$. 

\medskip

Déterminons la solution :
\begin{eqnarray*}
\sh 1&=&\sh(\Argsh a+\Argch a)\\ 
 &=&\sh(\Argsh a)\ch(\Argch a)+\sh(\Argch a)\ch(\Argsh a)\\
 &=&a^2+\sqrt{a^2-1}\sqrt{a^2+1}=a^2+\sqrt{a^4-1}
\end{eqnarray*}
donc $\sqrt{a^4-1}=\sh 1-a^2$. En élevant au carré et en simplifiant, on obtient
$a^2=\frac{1+\sh^21}{2\sh 1}=\frac{\ch^2 1}{2\sh 1}$. Comme on cherche $a$ positif 
(et que $\ch 1>0$), on en déduit $a=\frac{\ch 1}{\sqrt{2\sh 1}}$. 
Cette valeur est la seule solution possible de l'équation $f(x)=1$, il faudrait normalement 
vérifier qu'elle convient bien, puisqu'on a seulement raisonné par implication (et pas par équivalence).
Or on sait déjà que l'équation admet une unique solution: c'est donc nécessairement 
$$a=\frac{\ch 1}{\sqrt{2\sh 1}}= \tfrac{1}{2}\frac{e+\frac1e}{\sqrt{e-\frac1e}} = 1,0065\ldots.$$
}
}

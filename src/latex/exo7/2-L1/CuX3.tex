\uuid{CuX3}
\exo7id{740}
\titre{exo7 740}
\auteur{monthub}
\organisation{exo7}
\datecreate{2001-11-01}
\video{EVKKvovB3ps}
\isIndication{false}
\isCorrection{true}
\chapitre{Dérivabilité des fonctions réelles}
\sousChapitre{Autre}
\module{Analyse}
\niveau{L1}
\difficulte{}

\contenu{
\texte{
On consid\`ere la fonction $f : \R \to \R$ d\'efinie par
\begin{equation*}
f(t) =
\begin{cases}
  e^{1/t} & \mathrm{\ si\ } t<0\\
0  & \mathrm{\ si\ } t \geq 0
\end{cases}
\end{equation*}
}
\begin{enumerate}
    \item \question{D\'emontrer  que $f$ est d\'erivable sur $\R$, en particulier en $t=0$.}
\reponse{$f$ est d\'erivable sur $\R_-^*$ en tant que compos\'ee de fonctions
  d\'erivables, et sur $\R_+^*$ car elle est nulle sur cet intervalle ; \'etudions donc la
  d\'erivabilit\'e en $0$.

On a
$$\frac{f(t)-f(0)}{t}=\begin{cases}
  e^{1/t}/t & \mathrm{\ si\ } t<0\\
0  & \mathrm{\ si\ } t > 0
\end{cases}$$
or $e^{1/t}/t$ tend vers $0$ quand $t$ tend vers $0$ par valeurs
n\'egatives. Donc $f$ est d\'erivable \`a gauche et \`a droite en 0 et ces
d\'eriv\'ees sont identiques, donc $f$ est d\'erivable et $f'(0)=0$.}
    \item \question{Etudier l'existence de $f''(0)$.}
\reponse{On a
$$f'(t)=\begin{cases}
  -e^{1/t}/t^2 & \mathrm{\ si\ } t<0\\
0  & \mathrm{\ si\ } t \geq 0
\end{cases}$$
donc le taux d'accroissement de $f'$ au voisinage de 0 est
$$\frac{f'(t)-f'(0)}{t}=\begin{cases}
  -e^{1/t}/t^3& \mathrm{\ si\ } t<0\\
0  & \mathrm{\ si\ } t > 0
\end{cases}$$
et il tend vers $0$ quand $t$ tend vers $0$ par valeurs sup\'erieures
comme inf\'erieures. Donc $f$ admet une d\'eriv\'ee seconde en $0$, et
$f''(0)=0$.}
    \item \question{On veut montrer que pour $t<0$, la d\'eriv\'ee $n$-i\`eme de $f$ s'\'ecrit
$$f^{(n)}(t)=\frac{P_n(t)}{t^{2n}}e^{1/t}$$
o\`u $P_n$ est un polyn\^ome.
\begin{enumerate}}
\reponse{\begin{enumerate}}
    \item \question{Trouver $P_1$ et $P_2$.}
\reponse{On a d\'ej\`a trouv\'e que $f'(t)=-e^{1/t}/t^2$, donc  $f'(t)=P_1(t)/t^2
    e^{1/t}$ si on pose $P_1(t)=-1$.

Par ailleurs, $f''(t)=e^{1/t}/t^4+e^{1/t}
(2/t^3)=\frac{1+2t}{t^4}e^{1/t}$ donc la formule est vraie pour
$n=2$ en posant $P_2(t)=1+2t$.}
    \item \question{Trouver une relation de r\'ecurrence entre $P_{n+1}, P_n$ et $P'_n$ pour
  $n\in \N^*$.}
\reponse{Supposons que la formule est vraie au rang $n$.
Alors $f^{(n)}(t)=\frac{P_n(t)}{t^{2n}}e^{1/t}$ d'o\`u
\begin{equation*}
\begin{split}
  f^{(n+1)}(t)&=\frac{P'_n(t)t^{2n}-P_n(t)(2n)t^{2n-1}}{t^{4n}}e^{1/t}+\frac{P_n(t)}{t^{2n}}e^{1/t}(-1/t^2)\\
&= \frac{P'_n(t)t^{2}-(2n t+1) P_n(t)}{t^{2(n+1)}}e^{1/t}
\end{split}
\end{equation*}
donc la formule est vraie au rang $n+1$ avec
$$P_{n+1}(t)=P'_n(t)t^{2}-(2n t+1) P_n(t).$$}
\end{enumerate}
}

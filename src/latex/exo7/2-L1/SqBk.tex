\uuid{SqBk}
\exo7id{832}
\auteur{cousquer}
\organisation{exo7}
\datecreate{2003-10-01}
\isIndication{false}
\isCorrection{true}
\chapitre{Calcul d'intégrales}
\sousChapitre{Primitives diverses}

\contenu{
\texte{
Calculer les primitives suivantes.
}
\begin{enumerate}
    \item \question{$\displaystyle\int e^{\sin^2x} \sin2x\,dx$.}
\reponse{Changement de variable $u= \sin^2x$ (ou d'abord $u=\sin x$)~;
$e^{\sin^2x}+C$.}
    \item \question{$\displaystyle\int\cos^5t\,dt$ ; $\displaystyle\int\cosh^3t\,dt$; 
$\displaystyle\int\cos^4t\,dt$ ; $\displaystyle\int\sinh^4t\,dt$.}
\reponse{Deux méthodes~: changement de variable $u=\sin t$ (ou $u=\sinh t$),
ou linéarisation.\newline
$ {1\over15}(15\sin t -10\sin^3 t+3\sin^5 t)+C$ ou
$ {1\over80}\sin5t+{5\over48}\sin3t+{5\over8}\sin t +C$~;\newline
$ \sinh t + {1\over3}\sinh^3 t +C$ ou ${1\over12}\sinh3t +{3\over4}\sinh t+C$~;\newline
${1\over32}(\sin4t+8\sin2t+12t) +C$~;
${1 \over32}(\sinh4t-8\sinh2t+12t)+C$.}
    \item \question{$\displaystyle\int x^3 e^x\,dx$.}
\reponse{Intégrations par parties~: $(x^3-3x^2+6x-6)e^x+C$.}
    \item \question{$\displaystyle\int\ln x \,dx$ ; $\displaystyle\int x\ln x\,dx$; 
$\displaystyle\int\arcsin x \,dx$.}
\reponse{Intégration par parties~: $x\ln x-x+C$~;
${x^2\over2}\ln x -{x^2\over4}+C$~; $x\arcsin x+\sqrt{1-x^2}+C$.}
    \item \question{$\displaystyle\int\cosh t \sin t \, dt$.}
\reponse{Intégrations par parties~: ${1\over2}(\sinh t\sin t -\cosh t\cos t)+C$.}
    \item \question{$\displaystyle\int{dx \over\sin x}$.}
\reponse{Changement de variable $t=\tan{x\over2}$~;
$\ln\bigl\vert\tan{x\over2}\bigr\vert+C$ sur chaque intervalle\dots}
    \item \question{$\displaystyle\int\sqrt{a^2-x^2}\,dx$.}
\reponse{Changement de variable $x=a\sin u$~;
${a^2\over2}\arcsin{x\over a} + {x\over2}\sqrt{a^2-x^2} + C$.}
    \item \question{$\displaystyle\int{e^{2x}\over\sqrt{e^x+1}}\,dx$.}
\reponse{Changement de variable $u=e^x$~; ${2\over3}\sqrt{e^x+1}(e^x-2)+C$.}
    \item \question{$\displaystyle\int e^{ax}\cos bx \,dx$~; 
$\displaystyle\int e^{ax}\sin bx \,dx$.}
\reponse{Intégrations par parties~:
${1\over a^2+b^2}e^{ax}(a\cos bx + b\sin bx)+C$~;\newline
${1\over a^2+b^2}e^{ax}(-b\cos bx + a\sin bx)+C$.}
    \item \question{$\displaystyle\int\sqrt{x \over(1-x)^3}\,dx$\quad pour\quad$0<x<1$.}
\reponse{Changement de variable $t=\sqrt{x\over1-x}$~;
$2\sqrt{x\over1-x}-2\arctan\sqrt{x\over1-x}+C$.}
    \item \question{$\displaystyle\int{x^2 \over\sqrt{1-x^2}}\,dx$.}
\reponse{Changement de variable $t=\arcsin x$~;
${1\over2}(\arcsin x - x\sqrt{1-x^2}) + C$.}
    \item \question{$\displaystyle\int{dx \over\cos x+2\sin x +3}$.}
\reponse{Changements de variable $u=\tan{x\over2}$, $t=1+u$~;
$\arctan(\tan{x\over2}+1)+C$ sur chaque intervalle\dots{} Mais, au fait, ne
cherchait-on pas une primitive {\it sur} $\bf R$~?}
    \item \question{$\displaystyle\int{\sqrt x\,dx \over\sqrt{a^3-x^3}}$ \quad avec 
\quad $0<x<a$.}
\reponse{Changement de variable $x^3=u^2$~;
${2\over3}\arcsin\sqrt{x^3\over a^3}+C$.}
    \item \question{$\displaystyle\int{\cosh x\over\cosh x+\sinh x}\,dx$.}
\reponse{Multiplier et diviser par $\cosh x-\sinh x$, ou passer en $e^x$~;
${x\over2} +{\sinh2x\over4}-{\cosh2x\over4}+C$ ou
${x\over2}-{e^{-2x}\over4}+C$.}
\end{enumerate}
}

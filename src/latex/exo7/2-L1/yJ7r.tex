\uuid{yJ7r}
\exo7id{5230}
\titre{exo7 5230}
\auteur{rouget}
\organisation{exo7}
\datecreate{2010-06-30}
\isIndication{false}
\isCorrection{true}
\chapitre{Suite}
\sousChapitre{Suite définie par une relation de récurrence}
\module{Analyse}
\niveau{L1}
\difficulte{}

\contenu{
\texte{
Soient $(u_n)$, $(v_n)$ et $(w_n)$ les suites définies par la donnée de $u_0$,
$v_0$ et $w_0$ et les relations de récurrence 
$$u_{n+1}=\frac{v_n+w_n}{2},\,v_{n+1}=\frac{u_n+w_n}{2}\;\mbox{et}\;
w_{n+1}=\frac{u_n+v_n}{2}.$$

Etudier les suites $u$, $v$ et $w$ puis déterminer $u_n$, $v_n$ et $w_n$ en fonction de $n$
en recherchant des combinaisons linéaires intéressantes de $u$, $v$ et $w$. En
déduire $\lim_{n\rightarrow +\infty}u_n$, $\lim_{n\rightarrow +\infty}v_n$et $\lim_{n\rightarrow +\infty}w_n$.
}
\reponse{
Pour tout entier naturel $n$, on a $u_{n+1}-v_{n+1}=-\frac{1}{2}(u_n-v_n)$  et donc, $u_n-v_n=\left(-\frac{1}{2}\right)^n(u_0-v_0)$.
De même, en échangeant les rôles de $u$, $v$ et $w$, $v_n-w_n=\left(-\frac{1}{2}\right)^n(v_0-w_0)$ et $w_n-u_n=\left(-\frac{1}{2}\right)^n(w_0-v_0)$ (attention, cette dernière égalité n'est autre que la somme des deux premières et il manque encore une équation).
On a aussi, $u_{n+1}+v_{n+1}+w_{n+1}=u_n+v_n+w_n$ et donc, pour tout naturel $n$, $u_n+v_n+w_n=u_0+v_0+w_0$.
Ainsi, $u_n$, $v_n$ et $w_n$ sont solutions du système

$$\left\{
\begin{array}{l}
v_n-u_n=\left(-\frac{1}{2}\right)^n(v_0-u_0)\\
\rule{0mm}{7mm}w_n-u_n=\left(-\frac{1}{2}\right)^n(w_0-u_0)\\
\rule{0mm}{4mm}u_n+v_n+w_n=u_0+v_0+w_0
\end{array}
\right..$$
Par suite, pour tout entier naturel $n$, on a

$$\left\{
\begin{array}{l}
u_n=\frac{1}{3}\left((u_0+v_0+w_0)+\left(-\frac{1}{2}\right)^n(2u_0-v_0-w_0)\right)\\
v_n=\frac{1}{3}\left((u_0+v_0+w_0)+\left(-\frac{1}{2}\right)^n(-u_0+2v_0-w_0)\right)\\
w_n=\frac{1}{3}\left((u_0+v_0+w_0)+\left(-\frac{1}{2}\right)^n(-u_0-v_0+2w_0)\right)
\end{array}
\right..$$
Les suites $u$, $v$ et $w$ convergent vers $\frac{u_0+v_0+w_0}{3}$.
}
}

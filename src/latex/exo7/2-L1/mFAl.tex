\uuid{mFAl}
\exo7id{2092}
\auteur{bodin}
\organisation{exo7}
\datecreate{2008-02-04}
\isIndication{true}
\isCorrection{true}
\chapitre{Calcul d'intégrales}
\sousChapitre{Théorie}

\contenu{
\texte{
\label{ex:compint}
Soient $u$ et $v$ deux fonctions d\'erivables sur $\R$ et $f$ une fonction
continue sur $\R$.
}
\begin{enumerate}
    \item \question{On pose $\displaystyle F(x)=\displaystyle \int_{u(x)}^{v(x)}f(t)d t$. Montrer que $F$ est
d\'erivable sur $\R$ et calculer sa d\'eriv\'ee.}
\reponse{Commen\c{c}ons plus simplement avec la fonction 
$$H(x) =  \int_{a}^{v(x)}f(t)dt.$$
En fait $H$ est la composition de la fonction $x\mapsto v(x)$ avec la fonction
$G : x \mapsto \int_{a}^{x}f(t)dt$ : 
$$H = G \circ v.$$
La fonction $v$ est d\'erivable et la fonction $G$ aussi (c'est une primitive) donc
la compos\'ee $H=G\circ v$ est d\'erivable, de plus $H'(x) = v'(x)\cdot G'(v(x))$.
En pratique comme $G'(x) = f(x)$ cela donne $H'(x) = v'(x) f(v(x))$.

\emph{Remarque :} Il n'est pas n\'ecessaire de conna\^\i tre cette formule mais il est important de savoir refaire ce petit raisonnement.

On montrerait de m\^eme que la fonction $x\to\int_{u(x)}^{a}f(t)dt$ est d\'erivable de d\'eriv\'ee $-u'(x)f(u(x))$.
Revenons \`a notre fonction $F(x) = \int_{u(x)}^{v(x)}f(t)d t =  \int_{u(x)}^{a}f(t)dt+ \int_{a}^{v(x)}f(t)dt$, 
c'est la somme de deux fonctions d\'erivables donc est d\'erivable de d\'eriv\'ee :
$$F'(x) = v(x)f(v(x))-u'(x)f(u(x)).$$}
    \item \question{Calculer la d\'eriv\'ee de $\displaystyle G(x)=\int_x^{2x}\frac{d t}{1+t^2+t^4}$.}
\reponse{On applique ceci \`a $u(x) = x$ et $v(x) = 2x$ nous obtenons :
$$G'(x) = \frac{2}{1+(2x)^2+(2x)^4}-\frac{1}{1+x^2+x^4}.$$}
\indication{Se ramener \`a une composition de fonctions ou revenir \`a la d\'efinition de la d\'eriv\'ee avec le taux d'accroissement.}
\end{enumerate}
}

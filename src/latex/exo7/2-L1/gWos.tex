\uuid{gWos}
\exo7id{3976}
\titre{exo7 3976}
\auteur{quercia}
\organisation{exo7}
\datecreate{2010-03-11}
\isIndication{false}
\isCorrection{true}
\chapitre{Dérivabilité des fonctions réelles}
\sousChapitre{Autre}
\module{Analyse}
\niveau{L1}
\difficulte{}

\contenu{
\texte{
Soit $f$ une application dérivable de $\R$ dans $\R$ telle que 
$\forall\ x\in\R,\quad f(x)f'(x)\ge 0$.
Montrer que $f^{-1}(\R^*)$ est un intervalle.
}
\reponse{
Si $f$ change de signe, soit par exemple $f(a) > 0$,
$f(b) < 0$, $a<b$ et $c = \sup\{x\text{ tq }f_{|[a,x]}\text{ est croissante}\}$.
Alors $f$ est croissante sur $[a,c]$ et $f(c) = 0$, contradiction.
}
}

\uuid{Ws1v}
\exo7id{747}
\titre{exo7 747}
\auteur{bodin}
\organisation{exo7}
\datecreate{1998-09-01}
\video{elLpf1K7wu4}
\isIndication{true}
\isCorrection{true}
\chapitre{Fonctions circulaires et hyperboliques inverses}
\sousChapitre{Fonctions circulaires inverses}

\contenu{
\texte{
\'Ecrire sous forme d'expression algébrique
}
\begin{enumerate}
    \item \question{$ \sin(\Arccos x),\quad \cos(\Arcsin x),\quad \cos(2 \Arcsin x)$.}
\reponse{$\sin^2 y = 1-\cos^2 y$, donc
$\sin y = \pm\sqrt{1-\cos^2 y}$.
Avec $y=\Arccos x$, il vient $\sin(\Arccos x)= \pm\sqrt{1-x^2}$. Or $\Arccos x \in [0,\pi]$, 
donc $\sin(\Arccos x)$ est positif et finalement $\sin(\Arccos x)=  +\sqrt{1-x^2}$.
De la m\^eme mani\`ere on trouve $\cos(\Arcsin x) = \pm\sqrt{1-x^2}$. 
Or $\Arcsin x \in [-\frac{\pi}{2},\frac{\pi}{2}]$, donc $\cos(\Arcsin x)$ est positif 
et finalement $\cos(\Arcsin x)=+\sqrt{1-x^2}$.

Ces deux égalités sont à connaître ou à savoir retrouver très rapidement :
$$\sin(\Arccos x)  = \sqrt{1-x^2} = \cos(\Arcsin x).$$


Enfin, puisque $\cos(2y) = \cos^2 y - \sin^2 y$, on obtient avec $y=\Arcsin x$,
$$\cos(2\Arcsin x)=(\sqrt{1-x^2})^2-x^2=1-2x^2.$$}
    \item \question{$ \sin(\Arctan x),\quad \cos(\Arctan x),\quad \sin(3 \Arctan x)$.}
\reponse{Commen\c{c}ons par calculer $\sin(\Arctan x)$, $\cos(\Arctan x)$.
On utilise l'identité $1+\tan^2y=\frac 1{\cos^2 y}$ avec $y = \Arctan x$, ce qui donne 
$\cos^2 y = \frac{1}{1+x^2}$ et
 $\sin^2 y = 1-\cos^2 y=\frac{x^2}{1+x^2}$.
 Il reste à déterminer les signes de $\cos(\Arctan x)= \pm\frac{1}{\sqrt{1+x^2}}$ et $\sin(\Arctan x) = \pm\frac{x}{\sqrt{1+x^2}}$
Or $y=\Arctan x$ donc $y\in]-\frac{\pi}{2},\frac{\pi}{2}[$ et $y$ a le même signe que $x$ : 
ainsi $\cos y>0$, et $\sin y$ a le même signe que $y$ et donc que $x$. Finalement, on a 
$\cos(\Arctan x)= \frac{1}{\sqrt{1+x^2}}$ et $\sin(\Arctan x) = \frac{x}{\sqrt{1+x^2}}$.

\medskip


Il ne reste plus qu'à linéariser $\sin(3y)$ :
\begin{eqnarray*}
\sin(3y) &=& \sin(2y+y)=\cos(2y)\sin(y)+\cos(y)\sin(2y)\\
 &=&(2\cos^2 y-1)\sin y +2\sin y\cos^2y\\
 &=&4 \sin y \cos^2 y - \sin y
\end{eqnarray*}



Maintenant
\begin{eqnarray*}
\sin(3\Arctan x) &=& \sin(3y)= 4 \sin y \cos^2 y - \sin y\\
 &=& 4{\frac {x}{\left (1+{x}^{2}\right )^{3/2}}}-{\frac {x}{\sqrt {1+{x}^{2}}}}={\frac {x(3-x^2)}{\left (1+{x}^{2}\right )^{3/2}}}
\end{eqnarray*}

\bigskip

\emph{Remarque :} la méthode générale pour obtenir la formule de linéarisation de $\sin(3y)$ est 
d'utiliser les nombres complexes
et la formule de Moivre. On développe 
$$\cos(3y) + i \sin(3y) = (\cos y + i \sin y)^3 = \cos^3 y + 3i\cos^2y \sin y + \cdots$$ 
puis on identifie les parties imaginaires pour avoir $\sin(3y)$, 
ou les parties réelles pour avoir $\cos(3y)$.}
\indication{Il faut utiliser les identités trigonométriques classiques.}
\end{enumerate}
}

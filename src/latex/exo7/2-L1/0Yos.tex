\uuid{0Yos}
\exo7id{5086}
\auteur{rouget}
\organisation{exo7}
\datecreate{2010-06-30}
\isIndication{false}
\isCorrection{true}
\chapitre{Fonctions circulaires et hyperboliques inverses}
\sousChapitre{Fonctions hyperboliques et hyperboliques inverses}

\contenu{
\texte{
\label{exo:suprou3}
Etablir pour ch, sh et th les formules d'addition, de duplication et de linéarisation.
}
\reponse{
\ 
\begin{center}
\shadowbox{
$\begin{array}{ccc}
\ch(a+b)=\ch a\ch b+\sh a\sh b&\mbox{et}&\ch(a-b)=\ch a\ch b-\sh a\sh b,\\
\sh(a+b)=\sh a\ch b+\ch a\sh b&\mbox{et}&\sh(a-b)=\sh a\ch b-\sh b\ch a\\
\tanh(a+b)=\frac{\tanh a+\tanh
b}{1+\tanh a\tanh b}&\mbox{et}&\tanh(a-b)=\frac{\tanh a-\tanh b}{1-\tanh a\tanh b}.
\end{array}$
}
\end{center}
Deux
démonstrations~:

$$\ch a\ch b+\sh a\sh b=\frac{1}{4}((e^a+e^{-a})(e^b+e^{-b})+(e^a-e^{-a})(e^b-e^{-b})) =
\frac{1}{2}(e^{a+b}+e^{-a-b})=\ch(a+b).$$

$$\tanh(a+b)=\frac{\sh(a+b)}{\ch(a+b)}=\frac{\sh a\ch b+\sh b\ch
a}{\ch a\ch b+\sh a\sh b}=\frac{\tanh a+\tanh b}{1+\tanh a\tanh b}$$
après division
du numérateur et du dénominateur par le nombre non nul $\ch a\ch b$.
En appliquant à $a=b=x$, on
obtient~:~

\begin{center}
\shadowbox{
$\forall x\in\Rr,\;\ch(2x)=\ch^2x+sh^2x=2ch^2x-1=2sh^2x+1,\;\sh(2x)=2\sh x\ch x\;\mbox{et}\;
\tanh(2x)=\frac{2\tanh
x}{1+\tanh^2x}.$
}
\end{center}
En additionnant entre elles les formules d'addition, on obtient les formules de linéarisation~:

$$\ch a\ch
b=\frac{1}{2}(\ch(a+b)+\ch(a-b)),\;\sh a\sh
b=\frac{1}{2}(\ch(a+b)-\ch(a-b))\;\mbox{et}\;\sh a\ch b=\frac{1}{2}(\sh(a+b)+\sh(a-b)),$$
et en particulier

$$\ch^2x=\frac{\ch(2x)+1}{2}\;\mbox{et}\;\sh^2x=\frac{\ch(2x)-1}{2}.$$
}
}

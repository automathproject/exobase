\uuid{Jliz}
\exo7id{4700}
\titre{exo7 4700}
\auteur{quercia}
\organisation{exo7}
\datecreate{2010-03-16}
\isIndication{false}
\isCorrection{true}
\chapitre{Suite}
\sousChapitre{Convergence}
\module{Analyse}
\niveau{L1}
\difficulte{}

\contenu{
\texte{
Soit $u_n$ une suite r{\'e}elle born{\'e}e. On suppose que
$u_n^2+u_n-u_{n+1}\xrightarrow[n\to\infty]{} 0$. Montrer que $u_n\rightarrow 0$.
}
\reponse{
Soit $E$ l'ensemble des valeurs d'adh{\'e}rence de~$(u_n)$. Si $u_{n_k}\xrightarrow[k\to\infty]{}\lambda$
alors $u_{n_k+1}\xrightarrow[k\to\infty]{}\lambda+\lambda^2$ donc $E$ est stable
par l'application $f$ : $x \mapsto x+x^2$. En fait $E$ est invariant par cette application
car la suite $(u_{n_k-1})$ admet une valeur d'adh{\'e}rence $\mu \in E$ et on a
$\mu^2+\mu = \lambda$. En particulier l'intervalle $[\inf(E),\sup(E)]$ est invariant par~$f$ ce qui implique $\sup(E) = \inf(E) = 0$.
}
}

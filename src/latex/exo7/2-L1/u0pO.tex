\uuid{u0pO}
\exo7id{809}
\auteur{cousquer}
\organisation{exo7}
\datecreate{2003-10-01}
\isIndication{false}
\isCorrection{true}
\chapitre{Calcul d'intégrales}
\sousChapitre{Longueur, aire, volume}

\contenu{
\texte{
On appelle \emph{épicycloïde} la courbe décrite par un point d'un
cercle de rayon~$r$, lié à ce cercle, quand celui-ci roule sans glisser
sur un cercle de rayon~$R$ en restant tangent extérieurement à ce dernier, 
et dans son plan. On pose $n=R/r$.
Montrer que dans un repère que l'on précisera, l'épicycloïde
admet la représentation paramétrique~:
$$\left\lbrace
\begin{array}{rcl}
    x & = & r\bigl((n+1)\cos t-\cos(n+1)t\bigr)  \\
    y & = & r\bigl((n+1)\sin t-\sin(n+1)t\bigr)  
\end{array}\right.$$
Représenter la courbe pour $n=1,2,3$.
En supposant $n$ entier, calculer la longueur $L$ de la courbe et l'aire $A$
limitée par celle-ci.
Dans le cas $n=1$ (\emph{cardioïde}), calculer de plus l'aire $S$ de la
surface de révolution obtenue en faisant tourner la courbe autour de son
axe de symétrie, ainsi que le volume $V$ limitée par cette surface.
}
\reponse{
$\displaystyle L=8(n+1)r=8\frac{n+1}{n}R,\quad
A=\pi(n+1)(n+2)r^2=\pi\frac{(n+1)(n+2)}{n^2}R^2,\quad
S=\frac{128\pi R^2}{5},\quad
V=\frac{64\pi R^3}{3}$.
}
}

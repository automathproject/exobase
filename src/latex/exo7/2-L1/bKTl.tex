\uuid{bKTl}
\exo7id{3885}
\titre{exo7 3885}
\auteur{quercia}
\organisation{exo7}
\datecreate{2010-03-11}
\isIndication{false}
\isCorrection{true}
\chapitre{Continuité, limite et étude de fonctions réelles}
\sousChapitre{Etude de fonctions}

\contenu{
\texte{
Soit $f : \R \to \R$ vérifiant~:
$\forall\ x,y,z\in\R,\ |x-y|<|x-z|  \Rightarrow  |f(x)-f(y)|<|f(x)-f(z)|$.
Montrer successivement que $f$ est injective, monotone, continue, et enfin
affine.
}
\reponse{
Injectivité évidente.

Monotonie~: pour $a<b<c$ on a $|a-b| < |a-c|$ et $|c-b| < |c-a|$ d'où les
mêmes inégalités pour $f(a),f(b),f(c)$ ce qui prouve que $f(b)$ est strictement
compris entre $f(a)$ et $f(c)$.

Continuité~: soit $a\in\R$, $\delta>0$, $x=a-\delta$, $y=a+\delta$ et $z=a-4\delta$.
On a $2\delta=|x-y|<|x-z|=3\delta$ donc $|f(x)-f(y)|<|f(x)-f(z)|$ et
en faisant tendre $\delta$ vers~$0^+$~: $|f(a^-)-f(a^+)| \le |f(a^-)-f(a^-)| = 0$.

Affine~: soient $x\in\R$, $h>0$, $z=x+h$ et $x-h<y<x$. On a $|f(x)-f(y)|<|f(x)-f(x+h)|$
d'où en faisant tendre $y$ vers $(x-h)^+$~: $|f(x)-f(x-h)|\le|f(x)-f(x+h)|$.
On obtient l'inégalité inverse en permutant $y$ et $z$, donc $f(x-h)$ et $f(x+h)$
sont équidistants de $f(x)$ et, par injectivité de~$f$~:
$f(x) = \frac{f(x-h)+f(x+h)}2$ ce qui permet de conclure avec la continuité de~$f$.
}
}

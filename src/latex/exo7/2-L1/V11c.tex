\uuid{V11c}
\exo7id{3908}
\auteur{quercia}
\organisation{exo7}
\datecreate{2010-03-11}
\isIndication{false}
\isCorrection{true}
\chapitre{Continuité, limite et étude de fonctions réelles}
\sousChapitre{Etude de fonctions}

\contenu{
\texte{
Soient $\alpha,\beta,\gamma \in \R$ tels que $\alpha+\beta+\gamma = \pi$.
}
\begin{enumerate}
    \item \question{Démontrer que : $1 - \cos\alpha + \cos\beta + \cos\gamma =
                 4\sin\frac\alpha2 \cos\frac\beta2 \cos\frac\gamma2$.}
\reponse{$1 - \cos\alpha = 2\sin\frac\alpha2 \cos\frac {\beta+\gamma}2$,

$\cos\beta + \cos\gamma = 2\sin\frac\alpha2 \cos\frac {\beta-\gamma}2$.}
    \item \question{Simplifier $\tan\frac\alpha2 \tan\frac\beta2  +
             \tan\frac\beta2  \tan\frac\gamma2 +
             \tan\frac\gamma2 \tan\frac\alpha2 $.}
\reponse{$=1$.}
\end{enumerate}
}

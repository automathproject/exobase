\uuid{Re8d}
\exo7id{5214}
\titre{exo7 5214}
\auteur{rouget}
\organisation{exo7}
\datecreate{2010-06-30}
\isIndication{false}
\isCorrection{true}
\chapitre{Propriétés de R}
\sousChapitre{Les rationnels}

\contenu{
\texte{
Soit $u_n$ le chiffre des unités de $C_n^k$, $k$ entier naturel fixé non nul et $n$ entier naturel supèrieur ou égal à $k$. Montrer que le nombre $0,u_ku_{k+1}u_{k+2}...$ est rationnel.
}
\reponse{
Soient $k$ un entier naturel non nul et $n$ un entier naturel supérieur ou égal à $k$.

\begin{align*}
\binom{n+10\times k!}{k}&=\frac{(n+10\times k!)(n+10\times k!-1)...(n+10\times k!-k+1)}{k!}\\
 &=\frac{n(n-1)...(n-k+1)+10\times k!\times K}{k!}\quad(\mbox{pour un certain entier}\;K)\\
 &=\frac{n(n-1)...(n-k+1)}{k!}+10K=\binom{n}{k}+10K.
\end{align*}
La différence $\dbinom{n+10\times k!}{k}-\dbinom{n}{k}$ est donc divisible par $10$. Par suite, $\dbinom{n+10\times k!}{k}$ et $\dbinom{n}{k}$ ont même chiffre des unités en base 10. Ainsi, $\forall n\geq k,\;u_{n+10\times k!}=u_n$ et donc la suite $u$ est donc $10k!$-périodique. On sait alors que 

\begin{center}
\shadowbox{
$0,u_ku_{k+1}u_{k+2}...$ est rationnel.
}
\end{center}
}
}

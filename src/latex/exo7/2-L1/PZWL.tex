\uuid{PZWL}
\exo7id{2722}
\titre{exo7 2722}
\auteur{matexo1}
\organisation{exo7}
\datecreate{2002-02-01}
\isIndication{false}
\isCorrection{true}
\chapitre{Série numérique}
\sousChapitre{Série à  termes positifs}
\module{Analyse}
\niveau{L1}
\difficulte{}

\contenu{
\texte{
Montrer qu'il existe deux r\'eels $\alpha$, $\beta$, tels que pour tout $n\in\N^*$, 
$$ \int_0^\pi  (\alpha t+\beta t^2) \cos(nt) \,dt = \frac 1{n^2}.$$
En d\'eduire la valeur de
$$S = \sum_{n=1}^{+ \infty} \frac 1{n^2}.$$
}
\reponse{
Soit $\alpha, \beta \in \mathbb{R}$ et 
$I_{(\alpha,\beta)} = \int_{0}^{\pi} (\alpha t + \beta t^2) \cos(nt) \, dt$.
Une intégration par partie nous donne 
\[ 
I_{(\alpha,\beta)} = \underbrace{  \left[ (\alpha t + \beta t^2)\frac{\sin(nt)}{n} \right]_{0}^{\pi} }_{ = 0 }
                       - \frac{1}{n} \int_{0}^{\pi} (\alpha + 2\beta t)\sin(nt)\, dt 
\]
En faisant une intégration par partie sur la deuxième intégrale, on a: 
\begin{eqnarray*}
I_{(\alpha,\beta)} &=& 
\left[ \frac{\alpha + 2\beta t}{n^2}\cos(nt) \right]_{0}^{\pi}  + \int_{0}^{\pi} \frac{2\beta \cos(nt)}{n^2}\, dt 
\\
&=& \left( \frac{\alpha + 2\beta \pi}{n^2}\cos(n\pi) - \frac{\alpha}{n^2} \right) + \underbrace{ \frac{2\beta}{n^2} \left[ \frac{\sin(nt)}{n} \right]_{0}^{\pi} }_{ = 0}  \; \cdotp
\end{eqnarray*}
On obtient $I_{(\alpha,\beta)} = \frac{\alpha + 2 \beta \pi}{n^2}\cos(n \pi) - \frac{\alpha}{n^2}$.
\begin{eqnarray*} 
I_{(\alpha,\beta)} = \frac{1}{n^2} & \Longleftrightarrow &
(\alpha + 2 \beta \pi) \underbrace{ \cos(n\pi) }_{ = (-1)^n} - \alpha = 1 
\\
& \Longleftrightarrow & (\alpha + 2 \beta \pi)(-1)^n - (1 +\alpha) = 0 
{}\end{eqnarray*}
Donc pour tout $n\in \mathbb{N}$
\[
\left\{
\begin{array}{rcl}
(\alpha + 2 \beta \pi)(-1)^n &=& 0 \\
(1 +\alpha) &=& 0
\end{array}
\right.
\]
Ainsi en prenant $\alpha = -1$ et $\beta = \frac{1}{2\pi}$, on obtient: 
\[ 
I_{ (-1,\frac{1}{2\pi}) } = \int_{0}^{\pi} \left(-t + \frac{1}{2\pi} t^2 \right) \cos(nt)\, dt = \frac{1}{n^2} 
\; \cdotp
\]
D'où,
\[ 
\sum_{k = 1}^{n} \frac{1}{k^2} = \int_{0}^{\pi} \left(-t + \frac{1}{2\pi} t^2 \right) \sum_{k = 1}^{n}\cos(kt)\, dt  \qquad \qquad (1) 
\]
Or 
\begin{eqnarray*}
\sum_{k = 1}^{n} \cos(kt) &=& 
\Re \left(\sum_{k = 0}^{n}e^{ikt}\right) - 1 = \cfrac{ \sin(\frac{n+1}{2}t)} { \sin(\frac{t}{2}) }\text{Re}(e^{i \frac{nt}{2} })  - 1 
\\
&=& \cos \left( \frac{nt}{2} \right) \times \cfrac{ \sin( \frac{n}{2}t )\cos( \frac{t}{2} ) + \cos( \frac{n}{2}t )\sin( \frac{t}{2} ) } { \sin(\frac{t}{2}) } - 1 
\\
&=& \cos \left( \frac{nt}{2} \right) \sin \left( \frac{nt}{2} \right) \cot \left( \frac{t}{2} \right) + \underbrace{ \cos^2 \left( \frac{nt}{2} \right) - 1 }_{ = -\sin^2(\frac{nt}{2}) }  
\end{eqnarray*}
Donc $\sum_{k = 1}^{n} \cos(kt) 
= \frac{1}{2}\sin(nt)\cot \left( \frac{t}{2} \right) - \sin^2 \left( \frac{nt}{2} \right)$.

En appliquant ce résultat à (1) on obtient:
\[  
\sum_{k = 1}^{n} \frac{1}{k^2} 
=  \int_{0}^{\pi} \frac{1}{2} \left(-t + \frac{t^2}{2\pi} \right) \sin(nt)\cot \left( \frac{t}{2} \right) \, dt 
- \int_{0}^{\pi} \left(-t + \frac{t^2}{2\pi} \right) \sin^2 \left(\frac{nt}{2} \right) \, dt  
\] 
En posant $ \phi (t) = (-t + \frac{1}{2\pi} t^2)\cot(\frac{t}{2}) $, on a:
\[ \sum_{k = 1}^{+\infty} \frac{1}{k^2} 
= \lim_{n \to +\infty} \int_{0}^{\pi} \phi (t)\sin(nt)\, dt 
- \lim_{n \to +\infty}\int_{0}^{\pi} \left(-t + \frac{1}{2\pi} t^2 \right) \sin^2 \left(\frac{nt}{2} \right) \, dt 
\]
Comme
\[ 
(-t + \frac{1}{2\pi} t^2)\cot \frac{t}{2}
= (-t + \frac{1}{2\pi} t^2) \cfrac{\cos \frac{t}{2}}{\sin \frac{t}{2}}  
\thicksim_{t \to 0} - t \cfrac{\cos \frac{t}{2}} {\frac{t}{2}} = - 2 \cos \frac{t}{2}
\thicksim - 2
\] 
l'application $\phi$ se prolonge par continuité en $0$.
Utilisons le résultat classique suivant: si $h$ est une fonction continue sur $[0, \pi]$, alors
\[ 
\lim_{n \to +\infty} \int_{0}^{\pi} h(t)\sin(nt)\, dt = 0 \quad \text{ et } \quad
\lim_{n \to +\infty} \int_{0}^{\pi} h(t)\cos(nt)\, dt = 0 \; .
\]
appliqué à $\phi$: $\lim_{n \to +\infty} \int_{0}^{\pi} \phi (t)\sin(nt)\, dt = 0$.
De plus, \[ \int_{0}^{\pi} \left(-t + \frac{t^2}{2\pi} \right) \sin^2 \left(\frac{nt}{2} \right) \, dt 
          = \int_{0}^{\pi} \left(-t + \frac{t^2}{2\pi} \right) \left(\frac{1 - \cos(nt)}{2} \right)\, dt 
\]
par conséquent 
\[
\lim_{n \to +\infty} \int_{0}^{\pi} \left(-t + \frac{t^2}{2\pi} \right) \sin^2 \left( \frac{nt}{2} \right) \, dt 
= \int_{0}^{\pi} \frac{1}{2} \left( -t + \frac{t^2}{2\pi} \right) \, dt \; .
\]
Finalement
\[ 
\sum_{k = 1}^{+\infty} \frac{1}{k^2} 
= - \int_{0}^{\pi} \frac{1}{2} \left(-t + \frac{t^2}{2\pi} \right) \, dt 
= -\frac{1}{2} \left( \left[ -\frac{t^2}{2} \right]_{0}^{\pi} + \frac{1}{2\pi} 
\left[ \frac{1}{3}t^3 \right]_{0}^{\pi} \right) 
= \frac{\pi^2}{4} - \frac{\pi^2}{12} = \frac{\pi^2}{6}
\]

\medskip

(\emph{Corrigé de Eugène Ndiaye})
}
}

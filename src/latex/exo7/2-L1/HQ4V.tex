\uuid{HQ4V}
\exo7id{4693}
\titre{exo7 4693}
\auteur{quercia}
\organisation{exo7}
\datecreate{2010-03-16}
\isIndication{false}
\isCorrection{true}
\chapitre{Suite}
\sousChapitre{Convergence}
\module{Analyse}
\niveau{L1}
\difficulte{}

\contenu{
\texte{
Soit $(x_n)$ une suite de r{\'e}els strictement positifs convergeant vers 0.
}
\begin{enumerate}
    \item \question{Montrer qu'il existe une infinit{\'e} d'indices $n$ tels que
    $x_n = \max(x_n,x_{n+1},x_{n+2},\dots)$.}
\reponse{Sinon, on construit une sous-suite strictement croissante.}
    \item \question{Montrer qu'il existe une infinit{\'e} d'indices $n$ tels que
    $x_n = \min(x_0,x_1,\dots,x_n)$.}
\reponse{La suite $(\min(x_0,\dots,x_n))$ converge vers $0$, et prend
             une infinit{\'e} de valeurs diff{\'e}rentes.}
\end{enumerate}
}

\uuid{DfyA}
\exo7id{4490}
\titre{exo7 4490}
\auteur{quercia}
\organisation{exo7}
\datecreate{2010-03-14}
\isIndication{false}
\isCorrection{true}
\chapitre{Série numérique}
\sousChapitre{Familles sommables}
\module{Analyse}
\niveau{L1}
\difficulte{}

\contenu{
\texte{
\'Etudier la finitude des sommes suivantes~:
}
\begin{enumerate}
    \item \question{$\sum_{(i,j)\in(\N^*)^2}\frac1{(i+j)^\alpha}$.}
\reponse{Regroupement à $i+j$ constant $ \Rightarrow $ CV ssi $\alpha > 2$.}
    \item \question{$\sum_{(i,j)\in(\N^*)^2}\frac1{\strut i^\alpha +j^\alpha}$.}
\reponse{Pour $\alpha\ge 1$ on a par convexité~:
             $2^{1-\alpha}(i+j)^\alpha \le i^\alpha+j^\alpha \le (i+j)^\alpha$
             donc il y a convergence ssi $\alpha > 2$.}
    \item \question{$\sum_{x\in\Q\cap[1,+\infty[}\frac1{x^2}$.}
\reponse{Il y a une infinité de termes supérieurs à~$1/4$.}
    \item \question{$\sum_{(p,q)\in\N^2}\frac1{a^p+b^q}$, ${a>1,b>1}$.}
\reponse{$\frac1{a^p+b^q} \le \frac1{2\sqrt a^p\sqrt b^q}  \Rightarrow $ sommable.}
\end{enumerate}
}

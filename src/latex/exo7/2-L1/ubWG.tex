\uuid{ubWG}
\exo7id{465}
\titre{exo7 465}
\auteur{monthub}
\organisation{exo7}
\datecreate{2001-11-01}
\video{icn0oS9sQ-0}
\isIndication{true}
\isCorrection{true}
\chapitre{Propriétés de R}
\sousChapitre{Maximum, minimum, borne supérieure}
\module{Analyse}
\niveau{L1}
\difficulte{}

\contenu{
\texte{
D{\'e}terminer la borne sup{\'e}rieure et inf{\'e}rieure
(si elles existent) de : $A=\{u_n \mid n\in\N\}$ en posant
$u_n=2^n$ si $n$ est pair et  $u_n=2^{-n}$ sinon.
}
\indication{$\inf A =0$, $A$ n'a pas de borne supérieure.}
\reponse{
$(u_{2k})_k$ tend vers $+\infty$ et donc 
$A$ ne possède pas de majorant, ainsi
$A$ n'a pas de borne supérieure (cependant certains écrivent alors $\sup A = +\infty$). 
D'autre part toutes les valeurs de $(u_n)$ sont
positives et $(u_{2k+1})_k$ tend vers $0$, donc $\inf A =0$.
}
}

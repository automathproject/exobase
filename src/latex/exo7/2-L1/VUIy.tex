\uuid{VUIy}
\exo7id{5090}
\auteur{rouget}
\organisation{exo7}
\datecreate{2010-06-30}
\isIndication{false}
\isCorrection{true}
\chapitre{Fonctions circulaires et hyperboliques inverses}
\sousChapitre{Fonctions circulaires inverses}

\contenu{
\texte{
Calculer $u_n=\Arctan\frac{2}{1^2}+\Arctan\frac{2}{2^2}+...+\Arctan\frac{2}{n^2}$ pour $n$ entier naturel non nul
donné puis déterminer $\lim_{n\rightarrow +\infty}u_n$. (Utiliser l'exercice \ref{exo:suprou2} 4))
}
\reponse{
(On va retrouver le résultat de l'exercice \ref{exo:suprou2} dans un cas particulier)
 
 
 Soient $a$ et $b$ deux réels positifs. Alors, $\Arctan a\in\left[0,\frac{\pi}{2}\right[$, $\Arctan b\in\left[0,\frac{\pi}{2}\right[$ et donc, $\Arctan a-\Arctan b\in\left]-\frac{\pi}{2},\frac{\pi}{2}\right[$. De plus, 

$$\tan(\Arctan a-\Arctan b)=\frac{\tan(\Arctan a)-\tan(\Arctan b)}{1+\tan(\Arctan a)\tan(\Arctan b)}=\frac{a-b}{1+ab},$$
et donc, puisque $\Arctan a-\Arctan b\in]-\frac{\pi}{2},\frac{\pi}{2}[$,

\begin{center}
\shadowbox{
$\forall a\geq0,\;\forall b\geq0,\;\Arctan a-\Arctan b
=\Arctan\left(\frac{a-b}{1+ab}\right).$
}
\end{center}
Soit alors $k$ un entier naturel non nul.
$\Arctan\frac{2}{k^2}=\Arctan\frac{(k+1)-(k-1)}{1+(k-1)(k+1)}=\Arctan(k+1)-\Arctan(k-1)$ (puisque $k-1$ et $k+1$ sont
positifs). Par suite, si $n$ est un entier naturel non nul donné,

\begin{align*}
u_n&=\sum_{k=1}^{n}\Arctan\frac{2}{k^2}=\sum_{k=1}^{n}(\Arctan(k+1)-\Arctan(k-1))=\sum_{k=2}^{n+1}\Arctan
k-\sum_{k=0}^{n-1}\Arctan k\\
 &=\Arctan(n+1)+\Arctan n-\frac{\pi}{4}.
\end{align*}
La limite de $u_n$ vaut donc
$\frac{\pi}{2}+\frac{\pi}{2}-\frac{\pi}{4}=\frac{3\pi}{4}$.

\begin{center}
\shadowbox{
$\lim_{n\rightarrow +\infty}u_n=\frac{3\pi}{4}$.
}
\end{center}
}
}

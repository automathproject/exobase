\uuid{nh5O}
\exo7id{1280}
\titre{exo7 1280}
\auteur{ridde}
\organisation{exo7}
\datecreate{1999-11-01}
\isIndication{false}
\isCorrection{true}
\chapitre{Calcul d'intégrales}
\sousChapitre{Intégrale impropre}

\contenu{
\texte{
Donner la nature des int\'egrales suivantes :
$$\displaystyle{\int _{0}^{\infty }\!{\frac {{e^{-x}}}{\sqrt {x}}}{dx}}.$$
$$ \displaystyle{\int _{1}^{\infty }\!{x}^{x}{dx} }.$$

$$\displaystyle{\int _{0}^{\infty }\!{\frac {\sqrt {x}\sin(\frac 1x)}{\ln (1+x)}}{dx}} .$$

Nature et calcul des int\'egrales suivantes :
$$\displaystyle{\int _{1}^{2}\!{\frac {1}{\sqrt {{x}^{2}-1}}}{dx}}.$$
$$\displaystyle{\int _{0}^{\infty }\!{\frac {{x}^{5}}{{x}^{12}+1}}{dx}}.$$
$$\displaystyle{\int _{0}^{\infty }\!{e^{-\sqrt {x}}}{dx}} .$$
$$\displaystyle{\int _{1}^{\infty }\frac 1{\text{sh} (bile)}d (bile) }.$$
}
\reponse{
$\displaystyle{ \int _{0}^{\infty }\!
{\frac {{e^{-x}}}{\sqrt {x}}}{dx} }$ est convergente (en fait elle
vaut $\sqrt {\pi }$).
$\displaystyle{\int _{1}^{\infty }\!{x^x}{dx} }$ est divergente.
$\displaystyle{\int _{0}^{\infty }\!{\frac {\sqrt {x}\sin({x}^{-1})}{\ln (1+x)}}{dx} }$ est divergente.
$\displaystyle{\int _{1}^{2}\!{\frac {1}{\sqrt {{x}^{2}-1}}}{dx}=
\ln (2+\sqrt {3}) }$.
$\displaystyle{\int _{0}^{\infty }\!{\frac {{x}^{5}}{{x}^{12}+1}}{dx}=
1/12\,\pi  }$.
$\displaystyle{\int _{0}^{\infty }\!{e^{-\sqrt {x}}}{dx}=2 }$.
$\displaystyle{\int _{1}^{\infty }\! \frac 1 {\sinh(x)}{dx}
= - \ln \tanh(1/2) }$.
}
}

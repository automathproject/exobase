\uuid{D326}
\exo7id{5153}
\titre{exo7 5153}
\auteur{rouget}
\organisation{exo7}
\datecreate{2010-06-30}
\isIndication{false}
\isCorrection{true}
\chapitre{Propriétés de R}
\sousChapitre{Autre}

\contenu{
\texte{
Tout entier naturel non nul $n$ s'écrit de manière unique sous la forme

$$n=a_0+10a_1+...+10^pa_p,$$

où $p$ est un entier naturel et les $a_i$ sont des entiers éléments de $\{0,...,9\}$, $a_p$ étant non nul. Déterminer
$p$ en fonction de $n$.
}
\reponse{
$p$ est déterminé par l'encadrement~:~$10^p\leq n<10^{p+1}$ qui s'écrit encore $p\leq\frac{\ln
n}{\ln 10}<p+1$. Par suite,
\begin{center}
\shadowbox{
$p=E(\mbox{log}_{10}(n)).$
}
\end{center}

\begin{center}
\shadowbox{
Le nombre de chiffres d'un entier $n$ en base $10$ est donc $E(\mbox{log}_{10}(n))+1$.
}
\end{center}
}
}

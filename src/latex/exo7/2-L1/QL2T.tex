\uuid{QL2T}
\exo7id{4449}
\titre{exo7 4449}
\auteur{quercia}
\organisation{exo7}
\datecreate{2010-03-14}
\isIndication{false}
\isCorrection{true}
\chapitre{Série numérique}
\sousChapitre{Autre}

\contenu{
\texte{
Pour $x > 1$ on note $\zeta(x) = \sum_{k=1}^\infty \frac 1{k^x}$.
En comparant $\zeta(x)$ à une intégrale,
trouver $\lim_{x\to1^+} (x-1)\zeta(x)$.
}
\reponse{
La fonction $t \mapsto \frac{1}{t^x}$ étant décroissante sur l'intervalle $[n, n+1]$,
$\displaystyle 0 < \frac{1}{t^x} \leq \frac{1}{n^x}$. Donc,
\begin{eqnarray*}
\int_{n}^{n+1} \frac{dt}{t^x}  & \leq & \int_{n}^{n+1} \frac{dt}{n^x} = \frac{1}{n^x} \quad \text{par positivité de l'intégrale}
\end{eqnarray*}
puis
\[
\int_{1}^{N+1} \frac{dt}{t^x} = \sum_{n=1}^{N} \int_{n}^{n+1} \frac{dt}{t^x}
\le \sum_{n=1}^{N} \frac{1}{n^x} \quad \text{ par la relation de Chasles.}
\]
De même sur $[n-1,n]$, $\displaystyle \frac{1}{t^x} \geq \frac{1}{n^x} > 0$ et
\[
\int_{n-1}^{n} \frac{dt}{t^x} \ge \int_{n-1}^{n} \frac{dt}{n^x} = \frac{1}{n^x}
\]
\[
\int_{1}^{N} \frac{dt}{t^x} = \sum_{n=2}^{N} \int_{n-1}^{n} \frac{dt}{t^x}
\ge \sum_{n=2}^{N} \frac{1}{n^x} \quad \text{par la relation de Chasles.}
\]
Finalement
\[
1 + \int_{1}^{N} \frac{dt}{t^x} \ge \sum_{n=1}^{N} \frac{1}{n^x} \, \cdotp
\]
\begin{equation}
\text{Donc, on a:} \qquad \int_{1}^{N+1} \frac{dt}{t^x} \leq \sum_{n=1}^{N} \frac{1}{n^x} \leq 1 + \int_{1}^{N} \frac{dt}{t^x} \, \cdotp \label{I}
\end{equation}
Calculons dans un premier temps, $ \int_{1}^{N} \frac{dt}{t^x}$:
\[ 
\int_{1}^{N} \frac{dt}{t^x} = \left[\frac{t^{1-x}}{1-x}\right]_{1}^{N} = \frac {N^{1-x}-1}{1-x} 
\xrightarrow[N \to +\infty]{} \frac{1}{x-1} 
\]
avec la même limite pour $\int_{1}^{N+1} \frac{dt}{t^x}$.
Ainsi on a montré que $\sum_{n \geq 0} \frac{1}{n^x}$, série de Riemann avec $x > 1$, est convergente.
On déduit alors de \eqref{I}, en faisant tendre $N$ vers $+\infty$:
\[ \begin{array}{lrcccl}
& \frac{1}{x - 1} & \leq & \sum_{n = 1}^{+\infty} \dfrac{1}{n^x} & \leq & 1 + \frac{1}{x - 1}
\\
\text{donc} \quad & \frac{x- 1}{x - 1} & \leq & (x - 1) \zeta (x) & \leq & (x - 1)\left(1 + \frac{1}{x - 1}\right) \\
\text{puis} \qquad & 1 & \leq & (x - 1) \zeta (x) & \leq & x \, .
\end{array} \]
D'où, par le théorème des gendarmes:
\[ 
\lim_{x \rightarrow 1^{+}} (x - 1) \zeta (x) = 1 \, . 
\]

\medskip

(\emph{Corrigé d'Antoine Poulain})
}
}

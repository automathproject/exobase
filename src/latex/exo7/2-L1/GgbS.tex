\uuid{GgbS}
\exo7id{5252}
\titre{exo7 5252}
\auteur{rouget}
\organisation{exo7}
\datecreate{2010-07-04}
\isIndication{false}
\isCorrection{true}
\chapitre{Suite}
\sousChapitre{Suites équivalentes, suites négligeables}
\module{Analyse}
\niveau{L1}
\difficulte{}

\contenu{
\texte{
Déterminer un équivalent le plus simple possible de chacune des suites suivantes quand $n$ tend vers $+\infty$.

$$
\begin{array}{lllll}
1)\;\Arccos\frac{n-1}{n}&2)\;\Arccos\frac{1}{n}&3)\;\ch(\sqrt{n})&4)\;\left(1+\frac{1}{n}\right)^n&5)\;\frac{\Argch n}{\sqrt{n^4+n^2-1}}\\
6)\;(1+\sqrt{n})^{-\sqrt{n}}&7)\;\ln(\cos\frac{1}{n})(\ln\sin\frac{1}{n})&8)\;(\frac{\pi}{2})^{3/5}-(\Arctan n)^{3/5}
&9)\;\sqrt{1+\frac{(-1)^n}{\sqrt{n}}}-1
\end{array}
$$
}
\reponse{
Tout d'abord, pour $n\geq1$, $\frac{n-1}{n}$ existe et est élément de $[-1,1]$. Donc, $\Arccos\frac{n-1}{n}$ existe pour tout entier naturel non nul $n$.

Quand $n$ tend vers $+\infty$, $\frac{n-1}{n}$ tend vers $1$ et donc $\Arccos\frac{n-1}{n}$ tend vers $0$. Mais alors,

$$\Arccos\frac{n-1}{n}\sim\sin(\Arccos\frac{n-1}{n})=\sqrt{1-(\frac{n-1}{n})^2}=\frac{\sqrt{2n-1}}{n}\sim\frac{\sqrt{2}}{\sqrt{n}}.$$
$\Arccos\frac{1}{n}$ tend vers $1$ et donc $\Arccos\frac{1}{n}\sim1$.
$\ch(\sqrt{n})=\frac{1}{2}(e^{\sqrt{n}}+e^{-\sqrt{n}})\sim\frac{1}{2}e^{\sqrt{n}}$.
$n\ln(1+\frac{1}{n})\sim n.\frac{1}{n}=1$ et donc, $\left(1+\frac{1}{n}\right)^n=e^{n\ln(1+1/n)}$ tend vers $e$. Par suite, $\left(1+\frac{1}{n}\right)^n\sim e$.
$\Argch n$ existe pour $n\geq1$ et comme, pour $n\geq1$, $n^4+n^2-1\geq n^4>0$, $\frac{\Argch n}{\sqrt{n^4+n^2-1}}$ existe pour $n\geq1$.

$$\Argch n=\ln(n+\sqrt{n^2-1})\sim\ln(n+n)=\ln(2n)=\ln n+\ln2\sim\ln n.$$

Donc, $\frac{\Argch n}{\sqrt{n^4+n^2-1}}\sim\frac{\ln n}{\sqrt{n^4}}=\frac{\ln n}{n^2}$.
$-\sqrt{n}\ln(\sqrt{n}+1)=-\sqrt{n}\ln(\sqrt{n})-\sqrt{n}\ln(1+\frac{1}{\sqrt{n}})=-\sqrt{n}\ln(\sqrt{n})-\sqrt{n}(\frac{1}{\sqrt{n}}+o(\frac{1}{\sqrt{n}}))=-\sqrt{n}\ln(\sqrt{n})-1+o(1)$, et donc

$$(1+\sqrt{n})^{-\sqrt{n}}=e^{-\sqrt{n}\ln(\sqrt{n})-1+o(1)}\sim e^{-\sqrt{n}\ln(\sqrt{n})-1}=\frac{1}{e}\frac{1}{\sqrt{n}^{\sqrt{n}}}.$$
$$\ln(\cos\frac{1}{n})(\ln\sin\frac{1}{n})\sim(\cos\frac{1}{n}-1)\ln(\frac{1}{n})\sim(-\frac{1}{2n^2})(-\ln n)=\frac{\ln n}{2n^2}.$$
$(\Arctan n)^{3/5}=(\frac{\pi}{2}-\Arctan\frac{1}{n})^{3/5}=(\frac{\pi}{2})^{3/5}(1-\frac{2}{\pi}(\frac{1}{n}+o(\frac{1}{n})))^{3/5}=(\frac{\pi}{2})^{3/5}(1-\frac{6}{5n\pi}+o(\frac{1}{n}))$, et donc

$$(\frac{\pi}{2})^{3/5}-(\Arctan n)^{3/5}=(\frac{\pi}{2})^{3/5}(1-1+\frac{6}{5n\pi}+o(\frac{1}{n}))
\sim(\frac{\pi}{2})^{3/5}\frac{6}{5n\pi}$$
Tout d'abord, pour $n\geq1$, $\left|\frac{(-1)^n}{\sqrt{n}}\right|=\frac{1}{\sqrt{n}}\leq 1$, et donc $1+\frac{(-1)^n}{\sqrt{n}}\geq0$, puis $\sqrt{1+\frac{(-1)^n}{\sqrt{n}}}-1$ existe. Ensuite, quand $n$ tend vers $+\infty$,

$$\sqrt{1+\frac{(-1)^n}{\sqrt{n}}}-1\sim\frac{(-1)^n}{2\sqrt{n}}.$$
}
}

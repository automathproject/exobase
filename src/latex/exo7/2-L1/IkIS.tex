\uuid{IkIS}
\exo7id{2096}
\auteur{bodin}
\organisation{exo7}
\datecreate{2008-02-04}
\video{Rp5pIHte82w}
\isIndication{true}
\isCorrection{true}
\chapitre{Calcul d'intégrales}
\sousChapitre{Polynôme en sin, cos ou en sh, ch}

\contenu{
\texte{
Soit $\displaystyle I_n=\int_0^{\frac \pi 2}(\sin x)^n \, d x$ \ \ pour $n\in \N$.
}
\begin{enumerate}
    \item \question{Montrer que $I_{n+2}=\frac{n+1}{n+2}I_n$. Expliciter $I_n$. En déduire $\int_{-1}^1\left( 1-x^2\right) ^n d x$.}
    \item \question{Montrer que $\left( I_n\right) _n$ est positive décroissante. Montrer que $I_n\sim I_{n+1}$}
    \item \question{Simplifier $I_n \cdot I_{n+1}$. Montrer que $I_n\sim \sqrt{%
\frac \pi {2n}}$. En déduire 
$\frac{1 \cdot 3 \cdots \left( 2n+1\right) }{2 \cdot 4 \cdots \left( 2n\right) }\sim 2\sqrt{\frac n \pi }$.}
\reponse{
\begin{enumerate}
$$ I_{n+2}  = \int_0^{\frac \pi 2} \sin^{n+1} x \cdot \sin x \,  dx.$$
En posant $u(x) = \sin^{n+1} x$ et $v'(x) = \sin x$ et en intégrant par parties nous obtenons
\begin{align*}
I_{n+2} &= \bigg[ -\cos x \sin^{n+1}x \bigg]_0^{\frac \pi 2} \ \  + \ \ (n+1)\int_0^{\frac \pi 2} \cos^2x \sin^nx \, dx \\
 &= 0 \ \  + \ \ (n+1)\int_0^{\frac \pi 2} (1-\sin^2x)\sin^nx \, dx \\
 &= (n+1)I_n-(n+1)I_{n+2}.  \\
\end{align*}

Donc $(n+2)I_{n+2}=(n+1)I_n$.
Conclusion 
$$I_{n+2} = \frac{n+1}{n+2} I_n.$$
Nous avons donc une formule de récurrence pour $I_n$ qui s'exprime en fonction de $I_{n-2}$
qui a son tour s'exprime en fonction de $I_{n-4}$, etc. On se ramène ainsi à l'intégrale de $I_0$ (si $n$ est pair) 
ou bien de $I_1$ (si $n$ est impair). Un petit calcul donne $I_0=\frac \pi 2$ et $I_1=1$.
Par récurrence nous avons donc pour $n$ pair :
$$I_n = \frac{1\cdot3 \cdots (n-1) }{2 \cdot 4 \cdots n} \frac \pi 2,$$
et pour $n$ impair :
$$I_n = \frac{2 \cdot 4 \cdots (n-1)}{1 \cdot 3 \cdots n}.$$
Pour calculer $\int_{-1}^1\left( 1-x^2\right) ^n d x$ nous allons nous ramener à une intégrale de Wallis.
Avec le changement de variable $x=\cos u$, on montre assez facilement que :
\begin{align*}
 \int_{-1}^1\left( 1-x^2\right) ^n d x 
   &=  2\int_0^1\left( 1-x^2\right) ^n d x\\
   &=  2\int_{\frac \pi 2}^{0}  \left( 1-\cos ^2 u\right) ^n (- \sin u \, du)  \quad \text{ avec } x=\cos u \\
   &=  2\int_0^{\frac \pi 2}  \sin^{2n+1} u \, du  \\
   &= 2I_{2n+1}\\
\end{align*}
}
\indication{\begin{enumerate}
  \item Faire une intégration par parties afin d'exprimer $I_{n+2}$ en fonction de $I_n$. 
Pour le calcul explicite on distinguera le cas des $n$ pairs et impairs.
  \item Rappel : $u_n\sim v_n$ est équivalent à $\frac{u_n}{v_n} \to 1$. 
Utiliser la décroissance de $I_n$ pour encadrer $\frac{I_{n+1}}{I_n}$.
  \end{enumerate}}
\end{enumerate}
}

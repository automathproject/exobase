\uuid{6mrB}
\exo7id{3147}
\titre{exo7 3147}
\auteur{quercia}
\organisation{exo7}
\datecreate{2010-03-08}
\isIndication{false}
\isCorrection{true}
\chapitre{Propriétés de R}
\sousChapitre{Les rationnels}
\module{Analyse}
\niveau{L1}
\difficulte{}

\contenu{
\texte{
Soit $x\in\Q$, $0<x<1$. On d{\'e}finit une suite $(x_n)$ de rationnels par
r{\'e}currence :
\begin{itemize}
 \item $x_0 = x$,
 \item  Si $x_n$ existe et est non nul, soit $k_n \in \N^*$ le plus
              petit entier tel que $\frac1{k_n}\le x_n$. On pose
              $x_{n+1} = x_n - \frac1{k_n}$,
 \item Si $x_n = 0$, on s'arr{\^e}te. Dans ce cas,
              $x = \frac 1{k_0} + \frac 1{k_1} + \dots + \frac1{k_{n-1}}$.
\end{itemize}
}
\begin{enumerate}
    \item \question{Montrer que la suite est toujours finie.}
\reponse{Si $x_n = \frac pq \ne 0$ : $\frac 1{k_n}\le \frac pq < \frac 1{k_n-1},
     \Rightarrow  x_{n+1} = \frac {k_np-q}{k_nq}$ et $0\le k_np-q < p$.
    Donc la suite des num{\'e}rateurs est strictement d{\'e}croissante.}
    \item \question{Montrer que si $k_{i+1}$ existe, alors $k_{i+1} > k_i(k_i-1)$.}
\reponse{Car $x_{n+1} = \frac pq-\frac1{k_n} < \frac1{k_n-1}-\frac1{k_n}$.}
    \item \question{R{\'e}ciproquement, soit une d{\'e}composition :
    $x = \frac1{n_0} + \dots + \frac1{n_p}$ avec $n_i \in \N^*$ et
    $n_{i+1} > n_i(n_i-1)$. Montrer que pour tout $i$, on a $n_i = k_i$.}
\reponse{$n_p > n_{p-1}(n_{p-1}-1)  \Rightarrow  \frac 1{n_{p-1}} + \frac 1{n_p} < \frac 1{n_{p-1}-1}$.\par
    $n_{p-1}-1 \ge n_{p-2}(n_{p-2}-1)  \Rightarrow  \frac 1{n_{p-2}} + \frac 1{n_{p-1}-1} \le \frac 1{n_{p-2}-1}$, etc.\par
    Finalement, $x < \frac 1{n_0-1}  \Rightarrow  n_0 = k_0$.\par
    (cf. INA opt. 1977)}
\end{enumerate}
}

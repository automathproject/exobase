\uuid{1Mug}
\exo7id{5390}
\titre{exo7 5390}
\auteur{rouget}
\organisation{exo7}
\datecreate{2010-07-06}
\isIndication{false}
\isCorrection{true}
\chapitre{Continuité, limite et étude de fonctions réelles}
\sousChapitre{Continuité : théorie}

\contenu{
\texte{
Trouver $f$  bijective de $[0,1]$ sur lui-même et discontinue en chacun de ses points.
}
\reponse{
Soit $f(x)=\left\{
\begin{array}{l}
x\;\mbox{si}\;x\in(\Qq\cap[0,1])\setminus\{0,\frac{1}{2}\}\\
1-x\;\mbox{si}\;x\in(\Rr\setminus\Qq)\cap[0,1]\\
0\;\mbox{si}\;x=\frac{1}{2}\;\mbox{et}\;\frac{1}{2}\;\mbox{si}\;x=0
\end{array}
\right.$. $f$ est bien une application définie sur $[0,1]$ à valeurs dans $[0,1]$. De plus, si $x\in(\Qq\cap[0,1])\setminus\{0,\frac{1}{2}\}$, alors $f(f(x))=f(x)=x$.

Si $x\in(\Rr\setminus\Qq)\cap[0,1]$, alors $1-x\in(\Rr\setminus\Qq)\cap[0,1]$ et donc $f(f(x))=f(1-x)=1-(1-x)=x$.

Enfin, $f(f(0))=f(\frac{1}{2})=0$ et $f(f(\frac{1}{2}))=f(0)=\frac{1}{2}$.

Finalement, $f\circ f=Id_{[0,1]}$ et $f$, étant une involution de $[0,1]$, est une permutation de $[0,1]$.

Soit $a$ un réel de $[0,1]$. On note que $\lim_{x\rightarrow a,\;x\in(\Rr\setminus\Qq)}f(x)=1-a$ et $\lim_{x\rightarrow a,\;x\in\Qq}f(x)=a$. Donc, si $f$ a une limite en $a$, nécessairement $1-a=a$ et donc $a=\frac{1}{2}$. Mais, si $a=\frac{1}{2}$, $\lim_{x\rightarrow a,\;x\in\Qq,\;x\neq a}f(x)=a=\frac{1}{2}\neq0=f(\frac{1}{2})$ et donc $f$ est discontinue en tout point de $[0,1]$.
}
}

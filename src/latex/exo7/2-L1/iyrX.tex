\uuid{iyrX}
\exo7id{2099}
\titre{exo7 2099}
\auteur{bodin}
\organisation{exo7}
\datecreate{2008-02-04}
\video{1rhApdE7JPY}
\isIndication{true}
\isCorrection{true}
\chapitre{Calcul d'intégrales}
\sousChapitre{Longueur, aire, volume}
\module{Analyse}
\niveau{L1}
\difficulte{}

\contenu{
\texte{
Calculer l'aire de la région délimitée par les courbes
d'équation $\displaystyle y=\frac{x^2}2$ et $\displaystyle y=\frac 1{1+x^2}$.
}
\indication{Un dessin ne fait pas de mal !
Il faut ensuite résoudre l'équation $\frac{x^2}2=\frac 1{x^2+1}$
puis calculer deux intégrales.}
\reponse{
La courbe d'équation $y=x^2/2$ est une parabole, la courbe 
d'équation $y=\frac 1{1+x^2}$ est une courbe en cloche. Dessinez les deux graphes.
Ces deux courbes délimitent une région dont nous allons calculer l'aire.
Tout d'abord ces deux courbes s'intersectent
aux points d'abscisses $x=+1$ et $x=-1$ : cela se devine sur le graphique puis
se vérifie en résolvant l'équation $\frac{x^2}2=\frac 1{x^2+1}$.

Nous allons calculer deux aires :
\begin{itemize}
  \item L'aire $\mathcal{A}_1$ de la région sous la parabole, au-dessus de l'axe des abscisses et 
entre les droites d'équation $(x=-1)$ et $(x=+1)$.
Alors 
$$\mathcal{A}_1 = \int_{-1}^{+1} \frac{x^2}2 \, dx = \left[ \frac{x^3}{6} \right]_{-1}^{+1} = \frac 13.$$

  \item L'aire $\mathcal{A}_2$ de la région sous la cloche, au-dessus de l'axe des abscisses et 
entre les droites d'équation $(x=-1)$ et $(x=+1)$.
Alors 
$$\mathcal{A}_2 = \int_{-1}^{+1} \frac 1{x^2+1} \, dx = \left[ \arctan x \right]_{-1}^{+1} = \frac{\pi}{2}.$$

  \item L'aire $\mathcal{A}$ sous la cloche et au-dessus de la parabole vaut maintenant
$$\mathcal{A}= \mathcal{A}_2 - \mathcal{A}_1 = \frac{\pi}{2} - \frac 13.$$
\end{itemize}
}
}

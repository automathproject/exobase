\uuid{hF9B}
\exo7id{5720}
\auteur{rouget}
\organisation{exo7}
\datecreate{2010-10-16}
\isIndication{false}
\isCorrection{true}
\chapitre{Calcul d'intégrales}
\sousChapitre{Intégrale impropre}

\contenu{
\texte{
\textbf{1) (** I)} Trouver un équivalent simple quand $x$ tend vers $+\infty$ de $e^{x^2}\int_{x}^{+\infty}e^{-t^2}\;dt$.

\textbf{2) (***)} Montrer que $\int_{a}^{+\infty}\frac{\cos x}{x}\;dx\underset{a\rightarrow0}{\sim}-\ln a$.

\textbf{3) (*)} Montrer que $\int_{0}^{1}\frac{1}{x^3+a^2}\;dx\underset{a\rightarrow+\infty}{\sim}\frac{1}{a^2}$.
}
\reponse{
La fonction $t\mapsto e^{-t^2}$ est continue, positive et intégrable sur $[0,+\infty[$. De plus, quand $t$ tend $+\infty$,  

\begin{center}
$e^{-t^2}\sim\left(1+\frac{1}{t^2}\right)=\frac{d}{dt}\left(-\frac{1}{2t}e^{-t^2}\right)$.
\end{center}

D'après un théorème de sommation des relations de comparaison, quand $x$ tend vers $+\infty$,   

\begin{center}
$\int_{x}^{+\infty}e^{-t^2}\;dt\sim\int_{x}^{+\infty}\left(-\frac{1}{2t}e^{-t^2}\right)'\;dt =\frac{1}{2x}e^{-x^2}$,
\end{center}

et donc 

\begin{center}
\shadowbox{
$e^{x^2}\int_{x}^{+\infty}e^{-t^2}\;dt\underset{x\rightarrow+\infty}{\sim}\frac{1}{2x}$.
}
\end{center}
Pour $a> 0$ fixé, $\int_{a}^{+\infty}\frac{\cos x}{x}\;dx$ converge (se montre en intégrant par parties (voir exercice \ref{ex:rou3})) puis

\begin{align*}\ensuremath
\int_{a}^{+\infty}\frac{\cos x}{x}\;dx&=-\int_{1}^{a}\frac{\cos x}{x}\;dx+\int_{1}^{+\infty}\frac{\cos x}{x}\;dx\underset{a\rightarrow0}{=}-\int_{1}^{a}\frac{\cos x}{x}\;dx + O(1)\\
 &\underset{a\rightarrow0}{=}-\int_{1}^{a}\frac{1}{x}\;dx+\int_{1}^{a}\frac{1-\cos x}{x}\;dx+O(1)\underset{a\rightarrow0}{=}-\ln a+\int_{1}^{a}\frac{1-\cos x}{x}\;dx+O(1) .
\end{align*}

Maintenant, $\frac{1-\cos x}{x}\underset{x\rightarrow0}{\sim}\frac{x}{2}$ et en particulier, $\frac{1-\cos x}{x}$ tend vers $0$ quand $x$ tend vers $0$. Par suite, la fonction $x\mapsto\frac{1-\cos x}{x}$ est continue sur $]0,1]$ et  se prolonge par continuité en $0$. Cette fonction est donc intégrable sur $]0,1]$ et en particulier, $\int_{1}^{a}\frac{1-\cos x}{x}\;dx$ a une limite réelle quand $a$ tend vers $0$. On en déduit que $\int_{a}^{+\infty}\frac{\cos x}{x}\;dx\underset{a\rightarrow0}{=}-\ln a + O(1)$ et finalement

\begin{center}
\shadowbox{
$\int_{a}^{+\infty}\frac{\cos x}{x}\;dx\underset{a\rightarrow0}{\sim}-\ln a$.
}
\end{center}
Soit $a>0$.

\begin{center}
$\left|\int_{0}^{1}\frac{1}{x^3+a^2}\;dx -\frac{1}{a^2}\right| =\left|\int_{0}^{1}\left(\frac{1}{x^3+a^2}-\frac{1}{a^2}\right)\;dx\right| =\int_{0}^{1}\frac{x^3}{(x^3+a^2)a^2}\;dx\leqslant\frac{1}{a^4}$
\end{center}

Donc, $\int_{0}^{1}\frac{1}{x^3+a^2}\;dx\underset{a\rightarrow+\infty}{=}\frac{1}{a^2}+o\left(\frac{1}{a^2}\right)$ ou encore

\begin{center}
\shadowbox{
$\int_{0}^{1}\frac{1}{x^3+a^2}\;dx\underset{a\rightarrow+\infty}{\sim}\frac{1}{a^2}$.
}
\end{center}
}
}

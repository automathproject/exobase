%%%%%%%%%%%%%%%%%% PREAMBULE %%%%%%%%%%%%%%%%%%

\documentclass[12pt,a4paper]{article}

\usepackage{amsfonts,amsmath,amssymb,amsthm}
\usepackage[francais]{babel}
\usepackage[utf8]{inputenc}
\usepackage[T1]{fontenc}

%----- Ensemles : entiers, reels, complexes -----
\newcommand{\Nn}{\mathbb{N}} \newcommand{\N}{\mathbb{N}}
\newcommand{\Zz}{\mathbb{Z}} \newcommand{\Z}{\mathbb{Z}}
\newcommand{\Qq}{\mathbb{Q}} \newcommand{\Q}{\mathbb{Q}}
\newcommand{\Rr}{\mathbb{R}} \newcommand{\R}{\mathbb{R}}
\newcommand{\Cc}{\mathbb{C}} \newcommand{\C}{\mathbb{C}}

%----- Modifications de symboles -----
\renewcommand {\epsilon}{\varepsilon}
\renewcommand {\Re}{\mathop{\mathrm{Re}}\nolimits}
\renewcommand {\Im}{\mathop{\mathrm{Im}}\nolimits}

%----- Fonctions usuelles -----
\newcommand{\ch}{\mathop{\mathrm{ch}}\nolimits}
\newcommand{\sh}{\mathop{\mathrm{sh}}\nolimits}
\renewcommand{\tanh}{\mathop{\mathrm{th}}\nolimits}
\newcommand{\Arcsin}{\mathop{\mathrm{Arcsin}}\nolimits}
\newcommand{\Arccos}{\mathop{\mathrm{Arccos}}\nolimits}
\newcommand{\Arctan}{\mathop{\mathrm{Arctan}}\nolimits}
\newcommand{\Argsh}{\mathop{\mathrm{Argsh}}\nolimits}
\newcommand{\Argch}{\mathop{\mathrm{Argch}}\nolimits}
\newcommand{\Argth}{\mathop{\mathrm{Argth}}\nolimits}
\newcommand{\pgcd}{\mathop{\mathrm{pgcd}}\nolimits} 

%----- Commandes special dessin a ajouter localement ------
\usepackage{geometry}
\usepackage{pstricks}
\usepackage{pst-plot}
\usepackage{pst-node}
\usepackage{graphics,epsfig}

\pagestyle{empty}

% Que faire avec ce fichier monimage.tex ?
%   1/ latex monimage.tex
%   2/ dvips monimage.dvi
%   3/ ps2eps monimage.ps
%   4/ ps2pdf -dEPSCrop monimage.eps
%   5/ Dans votre fichier d'exos \includegraphics{monimage}

\begin{document}

\begin{center}
\begin{pspicture}(-7,-7.2)(7,1)
\psline{->}(-7,0)(7,0)
\psline{->}(0,-7)(0,1)
\parametricplot[linecolor=blue,plotpoints=1000]{-180}{250}
{
2.718 t 180 div 3.141 mul exp t cos mul 0.1 mul
2.718 t 180 div 3.141 mul exp t sin mul 0.1 mul
}
\parametricplot[linecolor=red,plotpoints=1000]{-180}{250}
{
2.718 t 180 div 3.141 mul exp t 90 add cos mul 0.1 mul
2.718 t 180 div 3.141 mul exp t 90 add sin mul 0.1 mul
}
\psplot[linecolor=red,linewidth=0.2mm]{-1.3}{-0.2}{3.6 x 1.187 add mul neg 0.685 add}
\psplot[linecolor=blue,linewidth=0.2mm]{-3}{-0.2}{1 3.6 div x 1.187 add mul 0.685 add}
\pscircle[linecolor=red,linestyle=dashed](-0.685,-1.187){1.93}
\psdots[linecolor=blue](-1.187,0.685)
\psdots[linecolor=red](-0.685,-1.187)
\uput[ul](-1.187,0.685){\textcolor{blue}{$M(\theta)$}}
\uput[l](-0.685,-1.187){\textcolor{red}{$\Omega(\theta)$}}

\end{pspicture}
\end{center}

\end{document}

%%%%%%%%%%%%%%%%%% PREAMBULE %%%%%%%%%%%%%%%%%%

\documentclass[12pt,a4paper]{article}

\usepackage{amsfonts,amsmath,amssymb,amsthm}
\usepackage[francais]{babel}
\usepackage[utf8]{inputenc}
\usepackage[T1]{fontenc}

%----- Ensemles : entiers, reels, complexes -----
\newcommand{\Nn}{\mathbb{N}} \newcommand{\N}{\mathbb{N}}
\newcommand{\Zz}{\mathbb{Z}} \newcommand{\Z}{\mathbb{Z}}
\newcommand{\Qq}{\mathbb{Q}} \newcommand{\Q}{\mathbb{Q}}
\newcommand{\Rr}{\mathbb{R}} \newcommand{\R}{\mathbb{R}}
\newcommand{\Cc}{\mathbb{C}} \newcommand{\C}{\mathbb{C}}

%----- Modifications de symboles -----
\renewcommand {\epsilon}{\varepsilon}
\renewcommand {\Re}{\mathop{\mathrm{Re}}\nolimits}
\renewcommand {\Im}{\mathop{\mathrm{Im}}\nolimits}

%----- Fonctions usuelles -----
\newcommand{\ch}{\mathop{\mathrm{ch}}\nolimits}
\newcommand{\sh}{\mathop{\mathrm{sh}}\nolimits}
\renewcommand{\tanh}{\mathop{\mathrm{th}}\nolimits}
\newcommand{\Arcsin}{\mathop{\mathrm{Arcsin}}\nolimits}
\newcommand{\Arccos}{\mathop{\mathrm{Arccos}}\nolimits}
\newcommand{\Arctan}{\mathop{\mathrm{Arctan}}\nolimits}
\newcommand{\Argsh}{\mathop{\mathrm{Argsh}}\nolimits}
\newcommand{\Argch}{\mathop{\mathrm{Argch}}\nolimits}
\newcommand{\Argth}{\mathop{\mathrm{Argth}}\nolimits}
\newcommand{\pgcd}{\mathop{\mathrm{pgcd}}\nolimits} 

%----- Commandes special dessin a ajouter localement ------
\usepackage{geometry}
\usepackage{pstricks}
\usepackage{pst-plot}
\usepackage{pst-node}
\usepackage{graphics,epsfig}

\pagestyle{empty}

% Que faire avec ce fichier monimage.tex ?
%   1/ latex monimage.tex
%   2/ dvips monimage.dvi
%   3/ ps2eps monimage.ps
%   4/ ps2pdf -dEPSCrop monimage.eps
%   5/ Dans votre fichier d'exos \includegraphics{monimage}

\begin{document}

\begin{pspicture}(-6.1,-4.6)(0.1,4.1)
\psset{xunit=1cm,yunit=1cm}
\psaxes{->}(0,0)(-1.5,-4.5)(4.5,3.5)
\psplot[plotpoints=10000]{-0.95}{4.5}{x 1 add ln}
\psline(-2.5,-2.5)(2.9,2.9)
\psline[linestyle=dashed](-1,-4)(-1,3)
\psline[linestyle=dashed](-0.3,0)(-0.3,-0.356)
\psline[linestyle=dashed](-0.3,-0.356)(-0.356,-0.356)
\psline[linestyle=dashed](-0.356,-0.356)(-0.356,-0.441)
\psline[linestyle=dashed](-0.356,-0.441)(-0.441,-0.441)
\psline[linestyle=dashed](-0.441,-0.441)(-0.441,-0.581)
\psline[linestyle=dashed](-0.441,-0.581)(-0.581,-0.581)
\psline[linestyle=dashed](-0.581,-0.581)(-0.581,-0.871)
\psline[linestyle=dashed](-0.581,-0.871)(-0.871,-0.871)
\psline[linestyle=dashed](-0.871,-0.871)(-0.871,-2.054)
\psline[linestyle=dashed](-0.871,-2.054)(-2.054,-2.054)

\psline[linestyle=dashed](4.1,0)(4.1,1.629)
\psline[linestyle=dashed](4.1,1.629)(1.629,1.629)
\psline[linestyle=dashed](1.629,1.629)(1.629,0.966)
\psline[linestyle=dashed](1.629,0.966)(0.966,0.966)
\psline[linestyle=dashed](0.966,0.966)(0.966,0.676)
\psline[linestyle=dashed](0.966,0.676)(0.676,0.676)

\uput[u](-0.3,0){$u_0$}
\uput[ur](4.1,0){$u_0'$}
\uput[u](2.9,2.9){$y=x$}
\uput[u](4.5,1.704){$y=\ln(1+x)$}
\end{pspicture}

\end{document}

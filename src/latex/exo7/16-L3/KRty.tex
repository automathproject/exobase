\uuid{KRty}
\exo7id{7217}
\titre{exo7 7217}
\auteur{megy}
\organisation{exo7}
\datecreate{2021-02-22}
\isIndication{false}
\isCorrection{false}
\chapitre{Fonction holomorphe}
\sousChapitre{Fonction holomorphe}
\module{Analyse complexe}
\niveau{L3}
\difficulte{}

\contenu{
\texte{
Soit $U\subset \C$  un ouvert connexe. Soit \(f:U\to \C\) une fonction holomorphe. On note \(f=u+iv\) l'écriture sous forme algébrique des valeurs de $f$. Montrer que les assertions suivantes sont équivalentes.
}
\begin{enumerate}
    \item \question{\(f\) est constante.}
    \item \question{Il existe une fonction $\phi \in \mathcal C^1(\R,\R)$ telle que $v=\phi(u)$.}
    \item \question{Il existe une fonction $\psi \in \mathcal C^1(\R^2,\R)$ dont la différentielle ne s'annule qu'en un nombre fini de points, et telle que $\psi(u,v)$ est constante.}
\end{enumerate}
}

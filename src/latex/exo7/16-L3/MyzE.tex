\uuid{MyzE}
\exo7id{2786}
\titre{exo7 2786}
\auteur{burnol}
\organisation{exo7}
\datecreate{2009-12-15}
\isIndication{false}
\isCorrection{true}
\chapitre{Fonction holomorphe}
\sousChapitre{Fonction holomorphe}
\module{Analyse complexe}
\niveau{L3}
\difficulte{}

\contenu{
\texte{
\label{ex:burnol1.1.4}
Montrer la formule pour la dérivée d'une composition $g\circ f$.
}
\reponse{
On utilise de nouveau la d\'efinition de la d\'eriv\'ee, d'abord pour $f$ en
$z$ puis pour $g$ au point $f(z)$:
$$f(z+h)=f(z)+ f'(z)h +h \epsilon(h).$$
Notons $w_h=f'(z)h +h\epsilon(h)$. Alors (et comme dans les exercices
précédents on utilise \og epsilon\fg{} pour n'importe quelle fonction tendant
vers zéro lorsque sa variable tend vers zéro):
$$g(f(z+h))=g(f(z)+w_h) =g(f(z))+g'(f(z))w_h +w_h \epsilon(w_h).$$
Ainsi:
$$\frac{1}{h} \big(g(f(z+h))-g(f(z))\big) = \big(g'(f(z)) +
\epsilon(w_h)\big)\frac{w_h}h.$$
Lorsque $h\to0$, on a $w_h\to0$, donc $\epsilon(w_h)\to0$ et par ailleurs
$\frac{w_h}h\to f'(z)$. Au final
$$
(g\circ f)'(z) = \lim_{h\to 0} \big(g'(f(z)) +
\epsilon(w_h)\big)\frac{w_h}h = g'(f(z)) f'(z).
$$
}
}

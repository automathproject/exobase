\uuid{qGXC}
\exo7id{2828}
\titre{exo7 2828}
\auteur{burnol}
\organisation{exo7}
\datecreate{2009-12-15}
\isIndication{false}
\isCorrection{false}
\chapitre{Théorème des résidus}
\sousChapitre{Théorème des résidus}
\module{Analyse complexe}
\niveau{L3}
\difficulte{}

\contenu{
\texte{

}
\begin{enumerate}
    \item \question{Soit $f$ analytique sur un disque $|z - z_0|\leq R$ et
telle qu'il existe un certain $z_1$ avec $|z_1
- z_0|<R$ tel que $|f(z)|>|f(z_1)|$ pour $|z-z_0| = R$.
Montrer que $f$ s'annule au moins une fois dans le
disque ouvert $D(z_0,R)$. 
\emph{Indication :} considérer sinon ce
que dit le principe du maximum pour la fonction
$\frac1f$.}
    \item \question{\emph{Théorème de Hurwitz.} Soit $f_n$ des
fonctions holomorphes sur un voisinage commun $U$ de
$\overline{D(0,1)}$ qui convergent uniformément sur
$U$. Soit $F$ la fonction limite. On suppose que $F$ n'a
aucun zéro sur le cercle $|z|=1$, et qu'elle a au moins un
zéro dans le disque ouvert $D(0,1)$. Montrer  en appliquant
la question précédente à $f_n$ que pour $n\gg1$ la fonction
$f_n$ a au moins un zéro dans $D(0,1)$.\footnote{On verra
plus tard en cours ou en exercice que pour $n\gg1$ chaque
$f_n$ a, comptés avec leurs multiplicités, exactement le même
nombre de zéros que $F$ dans $D(0,1)$.} 
Ce résultat est
souvent appliqué sous sa forme réciproque: \emph{si des
fonctions holomorphes $f_n$ sans zéro convergent
uniformément sur un ouvert connexe vers $F$ alors soit $F$
est identiquement nulle soit $F$ n'a aucun zéro.} Justifier
cette dernière reformulation.}
\end{enumerate}
}

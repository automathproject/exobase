\uuid{0IRn}
\exo7id{6718}
\titre{exo7 6718}
\auteur{queffelec}
\organisation{exo7}
\datecreate{2011-10-16}
\isIndication{false}
\isCorrection{false}
\chapitre{Formule de Cauchy}
\sousChapitre{Formule de Cauchy}

\contenu{
\texte{
Soit $f$ une fonction entière telle que $\vert f(z)\vert =1$ si
$\vert z\vert =1$. Soit $D$ le disque unité fermé.
}
\begin{enumerate}
    \item \question{Supposons que $f$ n'a pas de zéros dans $D$. Montrer qu'il existe
$k\in\C$ tel que $f(z)=k$ pour tout $z$ de $\C$.}
    \item \question{Soit $a_1,\dots,a_n$ (pourquoi un nombre fini ?) les zéros de $f$
dans $D$ , chacun étant compté avec son ordre de multiplicité. En
étudiant la fonction
$$g(z)=f(z)\prod_{j=1}^n{1-\overline{a_j}z\over z-a_j}$$
montrer qu'il existe $k\in\C$, $\vert k\vert =1$, et $n\in {\Nn}$ tels
que $f(z)=kz^n$.}
\end{enumerate}
}

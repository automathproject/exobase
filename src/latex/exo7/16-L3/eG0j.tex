\uuid{eG0j}
\exo7id{2811}
\titre{exo7 2811}
\auteur{burnol}
\organisation{exo7}
\datecreate{2009-12-15}
\isIndication{false}
\isCorrection{true}
\chapitre{Formule de Cauchy}
\sousChapitre{Formule de Cauchy}
\module{Analyse complexe}
\niveau{L3}
\difficulte{}

\contenu{
\texte{
Avec les mêmes notations on veut évaluer $\int_\gamma
  \overline{z^{n}} \,dz$, $n\in\Zz$. Justifier les étapes
  suivantes:
\begin{align*}
 \int_\gamma
  \overline{z^{n}} \,dz &= \overline{\int_\gamma z^n \,\overline{dz}}\\
\int_\gamma z^n \,\overline{dz}&= \int_{[B,C]}  z^n \,dz - \int_{[C,D]}
  z^n  \,dz + \int_{[D,A]} z^n \,dz - \int_{[A,B]}z^n \,dz\;,
\end{align*}
et compléter le calcul, pour tout $n\in\Zz$.
}
\reponse{
Pour toute fonction $f=u+iv$ \`a valeurs dans $\C$,  on a :
$$\int _\gamma \overline{f(z)} dz =\int_0^1 (u-iv)(\gamma'_1+i\gamma'_2)dt =\overline{\int_0^1 (u+iv)(\gamma'_1-i\gamma'_2)dt}=
\overline{\int_\gamma f(z) \overline{dz}}.$$
Voir la correction de l'exercice \ref{ex:burnol2.2.1}.
}
}

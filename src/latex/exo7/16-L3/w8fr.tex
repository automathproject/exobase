\uuid{w8fr}
\exo7id{7559}
\auteur{mourougane}
\organisation{exo7}
\datecreate{2021-08-10}
\isIndication{false}
\isCorrection{false}
\chapitre{Théorème des résidus}
\sousChapitre{Théorème des résidus}

\contenu{
\texte{
Soit $D$ un ouvert de $\Cc$ et $f : D\to\Cc$ une application holomorphe.
Soit $T$ un triangle (plein) inclus dans le domaine $D$. On notera $p$ son périmètre.
Le but de l'exercice est de montrer que 
$$\int_{\partial T}f(z)dz=0.$$
}
\begin{enumerate}
    \item \question{En considérant les quatre triangles obtenus en traçant les segments entre deux milieux de côtés de $T$,
montrer que pour l'un de ces triangles noté $T_1$,
$$|\int_{\partial T}f(z)dz|\leq 4|\int_{\partial T_1}f(z)dz|.$$}
    \item \question{En itérant cette construction, montrer que pour tout $n$, il existe un triangle $T_n\subset T_{n-1}$ de périmètre $\frac{p}{2^n}$
tel que $$|\int_{\partial T}f(z)dz|\leq 4^n|\int_{\partial T_1}f(z)dz|.$$}
    \item \question{Montrer que l'intersection des $T_n$ est un point $c$ de $D$.}
    \item \question{Montrer qu'un existe une fonction continue $h$ sur $D$ nulle en $c$ telle que pour tout $z\in D$,
$$f(z)=f(c)+(z-c)f'(c)+(z-c)h(z).$$}
    \item \question{Calculer $\int_{\partial T_n}f(c)dz$ et $\int_{\partial T_n}(z-c)f'(c)dz$.}
    \item \question{Montrer que pour tout $\epsilon >0$ il existe $N$ tel que pour $n\geq N$,$$|\int_{\partial T_n}(z-c)h(z)dz|\leq (\frac{p}{2^n})^2\varepsilon.$$}
    \item \question{Conclure.}
    \item \question{Montrer en choisissant un découpage de $T$ avec un petit triangle autour de $a$ que 
 si $f : D\to\Cc$ est une application continue et holomorphe sur $D-\{a\}$, alors pour tout triangle de sommet $a$ dans $D$,
 $\int_{\partial T}f(z)dz=0$.}
\end{enumerate}
}

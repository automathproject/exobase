\uuid{awUV}
\exo7id{7231}
\titre{exo7 7231}
\auteur{megy}
\organisation{exo7}
\datecreate{2021-03-06}
\isIndication{false}
\isCorrection{false}
\chapitre{Formule de Cauchy}
\sousChapitre{Formule de Cauchy}
\module{Analyse complexe}
\niveau{L3}
\difficulte{}

\contenu{
\texte{
(Variations sur l'aire.)
}
\begin{enumerate}
    \item \question{Pour tout \(r>0\), montrer que 
\[\int_{\partial B(0,r)}\bar z dz=2i\mathrm{Aire}\big(B(0,r)\big).\]}
    \item \question{Soient \(z_1,z_2,z_3\in \C\) des points non-alignés. On note \(\Delta\) le triangle de sommets \(z_1,z_2,z_3\). Montrer que 
\[\int_{\partial \Delta}\bar z dz=2i \mathrm{Aire}(\Delta).\]}
    \item \question{De manière générale, soit \(K\) un compact à bord régulier. Montrer que 
\[\int_{\partial K}\bar z dz=2i\mathrm{Aire}(K).\]}
\end{enumerate}
}

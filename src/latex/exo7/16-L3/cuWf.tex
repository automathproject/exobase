\uuid{cuWf}
\exo7id{6722}
\auteur{queffelec}
\organisation{exo7}
\datecreate{2011-10-16}
\isIndication{false}
\isCorrection{false}
\chapitre{Singularité}
\sousChapitre{Singularité}

\contenu{
\texte{
Soit $\alpha\in\Rr$. Montrer que le développement en série de Laurent en $0$ de
la fonction
$f(z)=\hbox{exp}{\alpha\over2}(z+{1\over z})$ est de la forme
$a_0+\sum_{n\geq1}a_n(z^n+{1\over z^n})$, où 
$$a_0={1\over 2\pi}\int_0^{2\pi}\hbox{exp}(\alpha\cos t)\ dt,\quad a_n={1\over
2\pi}\int_0^{2\pi}\hbox{exp}(\alpha\cos t)\cos(nt)\ dt \ \hbox{si}\ n\geq1.$$

(Calculer $f$ pour $|z|=1$ et conclure avec le prolongement analytique.)
}
}

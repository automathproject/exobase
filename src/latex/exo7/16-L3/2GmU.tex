\uuid{2GmU}
\exo7id{6716}
\auteur{queffelec}
\organisation{exo7}
\datecreate{2011-10-16}
\isIndication{false}
\isCorrection{false}
\chapitre{Formule de Cauchy}
\sousChapitre{Formule de Cauchy}

\contenu{
\texte{
Soit $D$ le disque unité ouvert et $f$ une fonction holomorphe de
$D$ dans $D$. On suppose que $f$ admet au moins deux points fixes,
c'est-à-dire qu'il existe $a$ et $b$ dans $D$ , $a\ne b $, tels que $f(a)=a$
et $f(b)=b$. Montrer que $f$ est l'identité de $D$. On pourra utiliser
l'application  $\Phi _a$ définie dans l'exercice \ref{gijsexophi} pour se
ramener au cas où l'un des points fixes est $0$.
}
}

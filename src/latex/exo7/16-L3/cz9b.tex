\uuid{cz9b}
\exo7id{2807}
\auteur{burnol}
\organisation{exo7}
\datecreate{2009-12-15}
\isIndication{false}
\isCorrection{true}
\chapitre{Formule de Cauchy}
\sousChapitre{Formule de Cauchy}

\contenu{
\texte{
\label{exo:CRpolaires}
  On veut exprimer les équations de Cauchy-Riemann avec les coordonnées
  polaires $r$ et $\theta$. Les équations de Cauchy-Riemann
 peuvent s'écrire sous la forme:
\[  \left(\frac\partial{\partial x} +
 i\frac\partial{\partial y}\right) F = 0\]
donc il s'agit d'exprimer $\frac\partial{\partial x}$ et
$\frac\partial{\partial y}$ en fonction de
$\frac\partial{\partial r}$ et de $\frac\partial{\partial
\theta}$. Lorsque l'on travaille sur un ouvert (ne
 contenant pas l'origine) sur lequel une détermination
 continue de l'argument $\theta$ est possible (par exemple
 sur $\Omega = \Cc\setminus]-\infty,0]$). Montrer:
\begin{align*}
 \frac\partial{\partial r} &=
 \cos(\theta)\frac\partial{\partial x} +
 \sin(\theta)\frac\partial{\partial y}\\
 \frac\partial{\partial\theta} &= -r\sin(\theta)\frac\partial{\partial x} +
 r\cos(\theta)\frac\partial{\partial y}
\end{align*}
En déduire $\frac\partial{\partial x} =
 \cos(\theta)\frac\partial{\partial r} - \sin(\theta)\frac1r
 \frac\partial{\partial\theta}$ et $\frac\partial{\partial y} =
 \sin(\theta)\frac\partial{\partial r} + \cos(\theta)\frac1r
 \frac\partial{\partial\theta}$. Montrer alors:
\[ \frac\partial{\partial x} +
 i\frac\partial{\partial y} = e^{i\theta}\left(\frac\partial{\partial r} +
 i\frac1 r\frac\partial{\partial \theta}\right) =
 e^{i\theta}\frac1 r\left(r\frac\partial{\partial r} + 
 i\frac\partial{\partial \theta}\right)\]
En déduire qu'en coordonnées polaires les équations de
 Cauchy-Riemann peuvent s'écrire (en particulier) sous la forme:
\[ \frac{\partial F}{\partial\theta} = i\, r\frac{\partial
 F}{\partial r}\]
}
\reponse{
Cet exercice et les suivants concernent des changements de variables. Rappelons que, si
$\Phi:U\to V$ est un diff\'eomorphisme entre ouverts $U,V$ de $\R ^n$ et si on note $y=\Phi (x)$,
$x=(x_1,...,x_n)\in U$ et $y=(y_1,...,y_n)\in V$, alors
$$\frac{\partial }{\partial x_i}=\sum_{j=1}^n \frac{\partial \Phi _j}{\partial x_i}\frac{\partial }{\partial y_j}$$
pour tout $i,j\in \{1,...,n\}$.
\bigskip
On a $x=r\, \cos \theta$ et $y=r\, \sin \theta$. Donc
$$\frac{\partial }{\partial r}= \frac{\partial (r\, \cos \theta)}{\partial r}\frac{\partial }{\partial x}+\frac{\partial (r\, \sin \theta)}{\partial r}\frac{\partial }{\partial y}=
\cos  \theta \frac{\partial }{\partial x} + \sin \theta \frac{\partial }{\partial y}$$
et
$$\frac{\partial }{\partial \theta}= \frac{\partial (r\, \cos \theta)}{\partial \theta}\frac{\partial }{\partial x}+
\frac{\partial (r\, \sin \theta)}{\partial \theta}\frac{\partial }{\partial y}=
-r\sin  \theta \frac{\partial }{\partial x} + r \cos \theta \frac{\partial }{\partial y}.$$
On peut r\'e\'ecrire ceci en:
$$\binom{\frac{\partial }{\partial r}}{\frac{\partial }{\partial \theta}}=M \binom{\frac{\partial }{\partial x}}{\frac{\partial }{\partial y}}
\quad \text{avec} \quad M= (Jac(f))^t = \left(
                                   \begin{array}{cc}
                                     \frac{\partial (r\, \cos \theta)}{\partial r} & \frac{\partial (r\, \sin \theta)}{\partial r} \\
                                     \frac{\partial (r\, \cos \theta)}{\partial \theta} & \frac{\partial (r\, \sin \theta)}{\partial \theta} \\
                                   \end{array}
                                 \right) .$$
Pour retrouver les $\frac{\partial }{\partial x}$, $\frac{\partial }{\partial y}$ en fonction de $\frac{\partial }{\partial r}$, $\frac{\partial }{\partial \theta}$
il suffit d'inverser cette matrice $M$. Le reste est clair.
}
}

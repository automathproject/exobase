\uuid{XE8U}
\exo7id{6847}
\auteur{gijs}
\organisation{exo7}
\datecreate{2011-10-16}
\isIndication{false}
\isCorrection{false}
\chapitre{Autre}
\sousChapitre{Autre}

\contenu{
\texte{
Soit $n$, $p$ deux entiers tels que $n\ge p+2 \ge 2$,
soit $f(z) = \dfrac{z^p}{1+z^n}$,  soit
$R>0$ et soit $D =  \{z\in
\Cc \mid 0<\mathrm{Arg}(z)<2\pi/n\ ,\  0<|z|<R\,\}$.
}
\begin{enumerate}
    \item \question{Démontrer que l'intégrale
$ \int_0^\infty
\dfrac{x^p}{1+x^n}\,dx$ converge.}
    \item \question{Dessiner $D$, déterminer les singularités
isolés de $f$
%, calculer le résidu en chacun des point
%singuliers isolés 
et calculer $  \int_{\partial D}
f(z) \, dz$.}
    \item \question{Déduire de 2. la valeur de 
$ \int_0^\infty 
\dfrac{x^p}{1+x^n}\,dx$. 
N'oubliez pas de justifier les passages à la limite que
vous effectuez.}
\end{enumerate}
}

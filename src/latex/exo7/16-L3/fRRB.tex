\uuid{fRRB}
\exo7id{2803}
\titre{exo7 2803}
\auteur{burnol}
\organisation{exo7}
\datecreate{2009-12-15}
\isIndication{false}
\isCorrection{true}
\chapitre{Fonction holomorphe}
\sousChapitre{Fonction holomorphe}
\module{Analyse complexe}
\niveau{L3}
\difficulte{}

\contenu{
\texte{
Montrer $\sin(a+ib) =
   \sin(a)\ch(b) + i\cos(a)\sh(b)$. Puis en prenant
   dorénavant $a$ et $b$ réels, prouver:
\[ a,b\in\Rr\implies\quad |\sin(a+ib)|^2 = \sin^2(a) + \sh^2(b)\]
Déterminer alors les nombres complexes $z = a+ib$ tels que
   $\sin(z) = 0$. Donner une autre preuve.
}
\reponse{
$$\begin{aligned}
\sin(a)ch(b)+i\, \cos(a)\sh(b)&=\frac{1}{4i} \big\{ (e^{ia}-e^{-ia})(e^{b}+e^{-b})-(e^{ia}+e^{-ia})(e^{b}-e^{-b})\big\}\\
&=\frac{1}{2i}\big(e^{ia-b}-e^{-ia+b}\big)=\sin(a+ib).
\end{aligned}$$
Si $a,b\in \R$, alors :
$$\begin{aligned}
|\sin(a+ib)|^2&= (\sin(a)\ch(b))^2+(\cos(a)\sh(b))^2\\
&= \sin^2(a)(1+\sh^2(b))+(1-\sin^2(a))\sh^2(b)\\
&= \sin^2(a) + \sh^2 (b).
\end{aligned}$$
Cette somme de carrés de nombres réels ne peut être nulle que si $\sin(a)=0$
et $\sh(b)=0$, c'est-à-dire $a\in\pi\Zz$ et $b=0$. Donc $\sin(z)=0\iff z\in\pi\Zz$.
}
}

\uuid{a5G5}
\exo7id{6860}
\auteur{gijs}
\organisation{exo7}
\datecreate{2011-10-16}
\isIndication{false}
\isCorrection{false}
\chapitre{Autre}
\sousChapitre{Autre}

\contenu{
\texte{
On se propose de calculer, à l'aide du théorème
des résidus, la valeur de l'intégrale
 $$I=\int_{-\infty}^{+\infty}\exp{(-\pi t^2)}\ dt.$$
}
\begin{enumerate}
    \item \question{Soit $a\in {\Rr}^*$ ; en intégrant la fonction $g(z)=\exp{(-\pi z^2)}$
sur le rectangle de sommets $-R$, $R$, $R+ia$, $-R+ia$ ($R>0$), montrer
que
$$I=\int_{-\infty}^{+\infty}\exp{\left(-\pi(t+ia)^2\right)}\ dt.$$

\medskip

On pose : $f(z)=\exp{(i\pi z^2)}\tan{(\pi z)}$.}
    \item \question{Quels sont les pôles de $f$ ? Préciser leur ordre.

\medskip

Soit $R>0$ ; on considère le parallélogramme $\Gamma _R$ de sommets
$A=R+1+iR$, $B=R+iR$, $C=-R-iR$ et $D=-R+1-iR$, orienté dans le sens
direct.}
    \item \question{Montrer que
$$\int_{\Gamma _R}f(z)\ dz=2e^{-i\pi/4}.$$}
    \item \question{\begin{enumerate}}
    \item \question{Montrer que 
$$\forall t\in [0,1],\ \vert \tan{(\pi (R+t+iR))\vert \le \coth{(\pi R)}}.$$}
    \item \question{En déduire que l'intégrale de $f$ sur le segment orienté $[A,B]$
tend vers 0 quand $R$ tend vers $+\infty$.}
    \item \question{Montrer de même que l'intégrale de $f$ sur $[C,D]$ tend vers 0
quand $R$ tend vers $+\infty$.}
\end{enumerate}
}

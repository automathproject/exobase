\uuid{8G9I}
\exo7id{6627}
\auteur{queffelec}
\organisation{exo7}
\datecreate{2011-10-16}
\isIndication{false}
\isCorrection{false}
\chapitre{Fonction holomorphe}
\sousChapitre{Fonction holomorphe}

\contenu{
\texte{
Soit $\sum a_nz^n$ une série entière de rayon $1$. On pose
$s_n=a_0+a_1+\cdots+a_n$, $t_n={1\over n+1}(s_0+s_1+\cdots+s_n)$, $u(z)=\sum
s_nz^n$ et $v(z)=\sum t_nz^n$.
}
\begin{enumerate}
    \item \question{Montrer que les rayons de convergence de $u$ et de $v$ sont égaux à $1$.}
    \item \question{Etablir pour tout $|z|<1$, $u(z)={1\over 1-z}\sum a_nz^n$. Retrouver ainsi le
théorème d'Abel : soit $f(z)=\sum a_nz^n$ une série entière de rayon $1$, telle
que
$\sum a_n$ converge vers $A$. Alors $f(z)$ tend vers $A$ quand $z\to 1$ non
tangentiellement.}
\end{enumerate}
}

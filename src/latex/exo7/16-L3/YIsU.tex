\uuid{YIsU}
\exo7id{2831}
\titre{exo7 2831}
\auteur{burnol}
\organisation{exo7}
\datecreate{2009-12-15}
\isIndication{false}
\isCorrection{true}
\chapitre{Théorème des résidus}
\sousChapitre{Théorème des résidus}

\contenu{
\texte{
Soit $f$ une fonction entière vérifiant $\lim_{|z|\to\infty}
|f(z)| = +\infty$. Donner plusieurs démonstrations que $f$
est un polynôme:
\begin{itemize}
\item en montrant, par un théorème du cours, que $w=0$ est
  une singularité polaire de $g(w) = f(\frac1w)$, et en en
  déduisant qu'il existe un polynôme $P$ tel que $f(z) -
  P(z)$ tende vers $0$ pour $|z|\to\infty$, puis Liouville,
\item ou en montrant que $f$ n'a qu'un nombre fini de zéros
  $z_j$, $1\leq j\leq n$, et en appliquant à $(z-z_1)\dots
  (z- z_n)/f(z)$ le résultat de l'exercice précédent, plus
  quelques réflexions de conclusion pour achever la preuve.
\end{itemize}
  Montrer que la fonction entière $z+e^z$ tend vers l'infini
le long de tout rayon partant de l'origine. D'après
ce qui précède $z+e^z$ est donc un
polynôme. Commentaires?
}
\reponse{
Soit $z=Re^{i\theta }$. Alors $|e^z| =e^{R\Re (e^{i\theta })} =e^{R\cos(\theta
  )}$. Pour $f(z)= z+e^z$, on a,  pour $\theta $ tel que $\cos(\theta )\leq
0$, $|f(z)|\geq R - e^{R\cos(\theta)}\geq R-1$. Si par contre $\cos(\theta )>
0$ alors $|f(z)|\geq e^{R\cos\theta} - R$. Dans les deux cas
$$|f(Re^{i\theta})| \to\infty \quad \text{pour}\;\; R \to \infty .$$
Nos calculs n'impliquent pas  $\lim_{|z|\to \infty } |f(z)|
=\infty$! Comme en fait $e^z$ n'est PAS un polynôme, on peut même affirmer que $\lim_{|z|\to \infty } |f(z)|
=\infty$ est FAUX.
}
}

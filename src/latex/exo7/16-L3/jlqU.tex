\uuid{jlqU}
\exo7id{7611}
\auteur{mourougane}
\organisation{exo7}
\datecreate{2021-08-10}
\isIndication{false}
\isCorrection{true}
\chapitre{Autre}
\sousChapitre{Autre}

\contenu{
\texte{

}
\begin{enumerate}
    \item \question{Soit $\sum_{n\geq 1}a_nz^n$ une série entière convergente (avec un rayon de convergence strictement positif).
Déterminer le rayon de convergence de la série $\sum_{n\geq 1}\frac{a_n}{n!}z^n$.}
\reponse{On note $R$ le rayon de convergence de la série entière $\sum_{n\geq 1}a_nz^n$ et $r\in]0,R[$. 
On sait alors que $(a_n r^n)_{n\in\Nn}$ est une suite bornée en module.
Soit $z\in\Cc$
$$\left|\frac{a_n}{n!}z^n\right|\leq \left|a_n r^n\right| \frac{\left|\frac{z}{r}\right|^n}{n!}$$
est le terme général d'une suite bornée en module car $\frac{\left|\frac{z}{r}\right|^n}{n!}$ 
est borné.
Par conséquent, $\sum_{n\geq 1}\frac{a_n}{n!}z^n$ a un rayon de convergence infini.}
    \item \question{Soit deux séries entières centrées en $0$ de rayon de convergence $R>0$ et de somme respective $f$ et $g$.
On suppose que pour tout $x\in]-R,R[, f(x)=g(x).$ Montrer que pour tout $z\in\Delta_r, f(z)=g(z)$.}
\reponse{L'application $f-g$ est somme sur $\Delta_r$ connexe d'une série entière centrée en $0$ et s'annule sur l'ensemble non discret $]-R,R[$.
Par le principe des zéros isolés, $f=g$ sur $\Delta_r$.}
    \item \question{En déduire que pour tout $a\in \Rr$, et pour tout $z\in\Cc$, $\sin(a+z)=\sin(a)\cos (z)+\cos(a)\sin (z).$}
\reponse{Soit $a\in\Rr$. Les deux applications $z\mapsto \sin(a+z)$ et $z\mapsto \sin(a)\cos (z)+\cos(a)\sin (z)$ sont développables en séries entières
sur $\Cc$ et coïncident sur $\Rr$ par la formule d'addition des sinus sur $\Rr$.
Par la question précédente, elles coïncident sur $\Cc$.}
\end{enumerate}
}

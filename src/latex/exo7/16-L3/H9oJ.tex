\uuid{H9oJ}
\exo7id{7614}
\auteur{mourougane}
\organisation{exo7}
\datecreate{2021-08-10}
\isIndication{false}
\isCorrection{true}
\chapitre{Autre}
\sousChapitre{Autre}

\contenu{
\texte{

}
\begin{enumerate}
    \item \question{Déterminer et représenter l'ensemble $D:=\{z\in\Cc, z^3\in \Cc^-\}$.}
    \item \question{Soit $f : D\to \Cc$, $z\mapsto \log (z^3)$. 
En quels points de $\Cc-D$ peut-on prolonger $f$ par continuité ?}
    \item \question{Comparer les fonctions $f$ et $3\log$ sur l'intersection $D\cap \Cc^-$ de leur domaine de définition.}
\reponse{
Soit $z=re^{i\theta}\in\Cc$ avec $r\geq 0$ et $\theta\in ]-\pi,\pi]$.
\begin{eqnarray*}
 z^3\not\in\Cc^-&\iff& z^3\in\Rr^-\iff r^3e^{i3\theta}\in\Rr^-\iff 3\theta\simeq \pi \mod 2\pi\\
 &\iff& \theta \simeq \frac{\pi}{3} \mod \frac{2\pi}{3}\iff \theta=\frac{-\pi}{3}\quad\text{ou}\quad \theta=\frac{\pi}{3}\quad\text{ou}\quad \theta=\pi.
\end{eqnarray*}

\begin{tikzpicture}
%\draw[help lines, step=0.5, very thin] (-6,-4) grid (6,4);
\draw [thick] (-5,0) -- (0,0) ;\draw [densely dotted] (5,0) -- (0,0) ;
\draw (0.5,0) arc (0:60:0.5) ; \draw (0.5,0.5) node[right]{$\pi/3$} ;
\draw [thick](0,0) -- (2.5,4.33) ;
\draw (1,0) arc (0:-60:1) ;\draw (1,-1) node[right]{$-\pi/3$} ;
\draw [thick](0,0) -- (2.5,-4.33) ;
\end{tikzpicture}


Le domaine $D$ est composé de $\Cc$ moins les demi-droites noires. 
% \begin{tikzpicture}
%  \draw [double distance=3pt, thick] (0.5,0) arc (0:60:0.5);
%  \draw (-1.5,0) arc (0:30:0.5);
%  \draw [color=gray](0,0) circle (2cm);
%  \draw[thick] (0,0)-- (1,1.73);
%  \draw [thick] (0,0) -- (2,0);
%  \draw [densely dotted] (-2,0)-- (1,1.73);
%  \draw [densely dotted](-2,0)-- (2,0);
%  \draw (1.26,-1.29) arc (60:90:0.5);
%  \draw [densely dotted] (1,-1.73) -- (2,0);
%  \draw [densely dotted] (1,-1.73) -- (1,1.73);
% \end{tikzpicture}
Soit $z=re^{i\theta}\in\Cc$ avec $r\geq 0$ et $\theta\in ]-\pi,\pi]$.
\begin{itemize}
Si $\theta\in]-\frac{\pi}{3},\frac{\pi}{3}[$, $ 3\theta\in]-\pi,\pi[$
 et donc, $f(z)=\log(z^3)=\log_\Rr(r^3)+i3\theta=3\log_\Rr(r)+3i\theta=3\log(z)$.
Si $\theta\in]\frac{\pi}{3},\pi[$, $ 3\theta\in]\pi,3\pi[$ et $ 3\theta-2\pi\in]-\pi,\pi[$
 et donc, $f(z)=\log(z^3)=\log_\Rr(r^3)+i(3\theta-2\pi)=3\log_\Rr(r)+3i\theta-2i\pi=3\log(z)-2i\pi$.
Si $\theta\in]-\pi,-\frac{\pi}{3},[$, $ 3\theta\in]-3\pi,-\pi[$ et $ 3\theta+2\pi\in]-\pi,\pi[$
 et donc, $f(z)=\log(z^3)=\log_\Rr(r^3)+i(3\theta+2\pi)=3\log_\Rr(r)+3i\theta+2i\pi=3\log(z)+2i\pi$.
\end{itemize}
La fonction $f$ n'est donc pas continu aux points de la demi-droite d'angle $\frac{\pi}{3}$ 
ni aux points de la demi-droite d'angle $-\frac{\pi}{3}$ où la fonction $_log$ est continue.
On sait que le saut de la branche principale $\log$ du logarithme entre $(-\pi^+$ et $\pi^-$
est de $2i\pi$. Le saut de $f$ est donc de $6i\pi$. Elle n'est donc pas continue
aux points de la demi-droite d'angle $\pi$.
La réponse a été donnée dans la question précédente.
}
\end{enumerate}
}

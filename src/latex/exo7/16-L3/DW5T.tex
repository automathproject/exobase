\uuid{DW5T}
\exo7id{6764}
\titre{exo7 6764}
\auteur{queffelec}
\organisation{exo7}
\datecreate{2011-10-16}
\isIndication{false}
\isCorrection{false}
\chapitre{Autre}
\sousChapitre{Autre}

\contenu{
\texte{
On introduit la fonction ``Zeta'' :
$$\zeta (s)=\sum_{n\ge 1}{1\over n^s}$$

\subparagraph{I  Produit d'Euler}

Montrer que $\zeta $ est holomorphe
dans l'ouvert $\Omega =\{ z\in\C\vert \ \Re z>1\}$.

Soient $p_1=2,p_2=3,\dots,p_n,\dots$ la suite des nombres premiers.
Montrer que dans $\Omega $, on a $$\zeta (s)=\prod_{n\ge 1}{1\over
1-p_n^{-s}}$$ 
(produit d'Euler).

\subparagraph{II Relation de $\zeta $ avec la répartition des
nombres premiers}
}
\begin{enumerate}
    \item \question{Montrer que $${\zeta '(s)\over \zeta (s)}=-\sum_{n\ge1}
\lambda (n)n^{-s}$$ où $\lambda (n)=\ln p$ si $n$ est une puissance
d'un nombre $p$ premier et $\lambda (n)=0$ si $n$ a au moins deux
diviseurs premiers distincts.}
    \item \question{On a le théorème suivant :\\
Théorème des nombres premiers (Hadamard-De la Vallée Poussin 1896) :
Lorsque $x$ tend vers $+\infty$, la somme des $\lambda (n)$ pour $n\le
x$ est équivalente à $x$.

Démontrer que cette assertion est équivalente à dire que le nombre de
nombres premiers plus petits que $x$ est équivalent à $x/\ln x$.}
\end{enumerate}
}

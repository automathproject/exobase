\uuid{sSuf}
\exo7id{6701}
\titre{exo7 6701}
\auteur{queffelec}
\organisation{exo7}
\datecreate{2011-10-16}
\isIndication{false}
\isCorrection{false}
\chapitre{Formule de Cauchy}
\sousChapitre{Formule de Cauchy}
\module{Analyse complexe}
\niveau{L3}
\difficulte{}

\contenu{
\texte{
Soit $f$ une fonction entière non constante ne s'annulant pas. 
Mon\-trer que 
$$\forall \varepsilon >0,\ \forall r>0,\ \exists z\in\C, \vert z\vert >r
\mbox{ et }\vert f(z)\vert <\varepsilon$$
Application : tout polynôme non constant admet un zéro dans $\C$ 
(théorème de d'Alembert).
}
}

\uuid{S1Tr}
\exo7id{7556}
\auteur{mourougane}
\organisation{exo7}
\datecreate{2021-08-10}
\isIndication{false}
\isCorrection{false}
\chapitre{Théorème des résidus}
\sousChapitre{Théorème des résidus}

\contenu{
\texte{

}
\begin{enumerate}
    \item \question{On considère le bord $\mathcal{C}$ du triangle de sommet : $ z=0 $, $ z=1$, et $ z=i$, orienté dans le sens direct.
Calculer les intégrales 
$ \int_{\mathcal{C}} xdz \quad \text{ et } \quad \int_{\mathcal{C}} ze^{z}dz.$}
    \item \question{Soit le cercle unité $\mathcal{C}$ parcouru dans le sens direct. Pour tout entier relatif $n$, calculer l'intégrale 
$ \int_{\mathcal{C}} z^{n}dz.$
Expliquer le cas particulier où $n=-1$.}
    \item \question{Montrer $\left|\int_\gamma \frac{dz}{z^2+1}\right| \le \frac{\pi}{3}, \mbox{ lorsque }\gamma(t)=2e^{it}, \; 0\le t\le \frac{\pi}{2}.$}
    \item \question{Montrer $\left|\int_\gamma \frac{e^z}{z} dz\right| \le \pi e, \mbox{ lorsque }\gamma(t)=e^{it}, \; 0\le t\le \pi.$}
\end{enumerate}
}

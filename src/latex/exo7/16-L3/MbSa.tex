\uuid{MbSa}
\exo7id{7560}
\auteur{mourougane}
\organisation{exo7}
\datecreate{2021-08-10}
\isIndication{false}
\isCorrection{false}
\chapitre{Théorème des résidus}
\sousChapitre{Théorème des résidus}

\contenu{
\texte{
Soit $a$ un nombre réel tel que $|a|\leq 1$. 
On considère la fonction $f :\Cc\to\Cc$, $z\mapsto e^{-z^2}$ et pour tout réel strictement positif $r$,
les chemins suivants dans le plan complexe
$$
\setlength{\unitlength}{.4in}
\begin{picture}(7,3.5)(0,-.25)
\linethickness{1pt}
\put(0,0){\line(1,0){4}}
\put(4,0){\line(0,1){3}}
\put(0,0){\line(4,3){4}}
%\put(2,0){\makebox(0,0){$\to$}}
\put(2,-.25){\makebox(0,0){$\gamma_1\to$}}
\put(4.5,1.5){\makebox(0,0){$\uparrow\gamma_2$}}
\put(2,2){\makebox(0,0){$\gamma_3\nearrow$}}
\put(-.25,-.25){\makebox(0,0){$0$}}
\put(4.25,-.25){\makebox(0,0){$r$}}
\put(4.25,3.25){\makebox(0,0){$r(1+ai)$}}
\end{picture}
$$
}
\begin{enumerate}
    \item \question{Calculer $\int_{\gamma_1}f(z)dz$.}
    \item \question{Montrer que $|\int_{\gamma_2}f(z)dz|\leq \frac{1}{r}.$}
    \item \question{En déduire $$\int_0^{+\infty}e^{-(1+ai)^2t^2}dt=\frac{\sqrt{\pi}}{2}\frac{1}{1+ia}.$$}
    \item \question{En déduire les valeurs des intégrales $\int_0^{+\infty}\cos(t^2)dt$ et $\int_0^{+\infty}\sin(t^2)dt$.}
\end{enumerate}
}

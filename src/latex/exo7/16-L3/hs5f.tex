\uuid{hs5f}
\exo7id{2816}
\auteur{burnol}
\organisation{exo7}
\datecreate{2009-12-15}
\isIndication{false}
\isCorrection{false}
\chapitre{Formule de Cauchy}
\sousChapitre{Formule de Cauchy}

\contenu{
\texte{
On pose $J_m = \int_0^{\pi/2} \cos^{2m+1} t\,dt$, pour
  $m\in\Nn$. En intégrant par parties $J_{m+1}$ obtenir la relation de
  récurrence $J_{m+1} = \frac{2m+2}{2m+3}J_m$ et
  prouver:\footnote{par convention lorsque qu'un produit porte sur un
  ensemble vide il vaut $1$. Donc la formule est bien
  compatible avec $J_0 = 1$.}
\[ J_m = \frac{2.4.\cdots.(2m)}{3.5.\cdots.(2m+1)}\]
}
}

\uuid{pK6D}
\exo7id{6677}
\titre{exo7 6677}
\auteur{queffelec}
\organisation{exo7}
\datecreate{2011-10-16}
\isIndication{false}
\isCorrection{false}
\chapitre{Formule de Cauchy}
\sousChapitre{Formule de Cauchy}
\module{Analyse complexe}
\niveau{L3}
\difficulte{}

\contenu{
\texte{
Soit $I(z)=\int_0^{+\infty}{\ln t\over t^2+z^2}dt$.
}
\begin{enumerate}
    \item \question{Pour quelles valeurs de $z$ $I(z)$ est-elle définie ?}
    \item \question{Montrer que pour $\Re z>0$, on a
$$I(z)={1\over z}\left( {\pi\over 2}\mathrm{Log} z+
\int_0^{+\infty}{\ln t\over 1+t^2}dt\right) ,$$
où $\mathrm{Log}$ est la détermination principale du logarithme. On pourra
considérer le chemin fermé $\Gamma_{\varepsilon, R}=[\varepsilon,
R]+\gamma_R+[Re^{i\varphi},\varepsilon
e^{i\varphi}]-\gamma_{\varepsilon}$, où $\varphi=\mathrm{Arg} z$ et
$\gamma_r\colon t\mapsto re^{it}$, $t\in [0,\varphi]$.}
    \item \question{Qu'obtient-on pour $\Re z<0$ ?}
\end{enumerate}
}

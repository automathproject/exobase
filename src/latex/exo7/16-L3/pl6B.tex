\uuid{pl6B}
\exo7id{7612}
\auteur{mourougane}
\organisation{exo7}
\datecreate{2021-08-10}
\isIndication{false}
\isCorrection{true}
\chapitre{Autre}
\sousChapitre{Autre}

\contenu{
\texte{

}
\begin{enumerate}
    \item \question{Soit $c\in \Cc$. Montrer que si $z\in\Cc$, alors $\sin z=c\iff (e^{iz})^2-2ic e^{iz}-1=0$.}
\reponse{Soit $z\in\Cc$.

\begin{eqnarray*}
 \sin z=c&\iff& \frac{e^{iz}-e^{-iz}}{2i}=c\iff e^{iz}-2ic-e^{-iz}=0\\
 &\iff& (e^{iz})^2-2ic e^{iz}-1=0 \textrm{ car } e^{iz}\not=0.
\end{eqnarray*}}
    \item \question{Soit $c\in [-1,1]$. Montrer que toutes les solutions dans $\Cc$ de $\sin z=c$ sont réelles.}
\reponse{Soit $c\in [-1,1]$. Soit $z\in\Cc$ une solution de $\sin z = c$. Alors $e^{iz}$ est solution de $X^2-2icX-1=0$.
Comme $c\in [-1,1]$, les solutions de cette équation sont $ic+\sqrt{1-c^2}$ et $ic-\sqrt{1-c^2}$,
toutes les deux de module $1$, 
$$|e^{iz}|=e^{-y}=1$$
et donc $y=0$, et $z$ est donc réel.}
    \item \question{Soit $a$ et $b$ deux nombres complexes. Calculer $e^{i(a+b)}$ et $e^{-i(a+b)}$ en fonction de $\sin a$, $\sin b$, $\cos a$ et $\cos b$.}
\reponse{\begin{eqnarray*}
 e^{i(a+b)}&=&e^{ia}e^{ib}=(\cos a+i\sin a)(\cos b+i\sin b)\\
 &=&(\cos a\cos b -\sin a\sin b)+i(\sin a\cos b+\cos a\sin b).
\end{eqnarray*}
\begin{eqnarray*}
 e^{-i(a+b)}&=&e^{-ia}e^{-ib}=(\cos a-i\sin a)(\cos b-i\sin b)\\
 &=&(\cos a\cos b -\sin a\sin b)-i(\sin a\cos b+\cos a\sin b).
\end{eqnarray*}}
    \item \question{Soit $a$ et $b$ deux nombres complexes. Démontrer la formule pour $\sin (a+b)$ en fonction de $\sin a$, $\sin b$, $\cos a$ et $\cos b$.}
\reponse{\begin{eqnarray*}
 \sin (a+b)&=&\frac{e^{i(a+b)}-e^{-i(a+b)}}{2i}=\sin a\cos b+\cos a\sin b
\end{eqnarray*}
d'après la question précédente.}
    \item \question{Résoudre dans $\Cc$, l'équation $\cos z+\sin z=2$.}
\reponse{Soit $z\in\Cc$.
\begin{eqnarray*}
 \cos z+\sin z=2&\iff& \frac{1}{\sqrt{2}}\cos z+\frac{1}{\sqrt{2}}\sin z=\sqrt{2}\\
 &\iff& \sin (\frac{\pi}{4}+z)=\sqrt{2}\\
 &\iff& (e^{iz}) \textrm{ est solution de }X^2-2i\sqrt{2}X-1=0\\
 &\iff& e^{iz}=i\sqrt{2}+i \textrm { ou } e^{iz}=i\sqrt{2}-i\\
 &\iff& e^{iz}=(\sqrt{2}+1)e^{\frac{\pi}{2}} \textrm { ou } e^{iz}=(\sqrt{2}-1)e^{\frac{\pi}{2}}\\
 &\iff&\exists k\in\Zz, iz=\log_\Rr(\sqrt{2}+1)+i(\frac{\pi}{2}+2k\pi) \\&& \textrm { ou } iz=\log_\Rr(\sqrt{2}-1)+i(\frac{\pi}{2}+2k\pi)\\
 &\iff&\exists k\in\Zz, z=(\frac{pi}{2}+2k\pi)-i\log_\Rr(\sqrt{2}+1) \\&& \textrm { ou } z=(\frac{\pi}{2}+2k\pi)-i\log_\Rr(\sqrt{2}-1)
\end{eqnarray*}
Les soultions de l'équation $\cos z+\sin z=2$ dans $\Cc$ sont les nombres complexes de la forme
$(\frac{pi}{2}+2k\pi)-i\log_\Rr(\sqrt{2}+1)$ ou $z=(\frac{pi}{2}+2k\pi)-i\log_\Rr(\sqrt{2}-1)$
avec $k$ entier.}
\end{enumerate}
}

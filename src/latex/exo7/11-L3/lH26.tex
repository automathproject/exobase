\uuid{lH26}
\exo7id{7672}
\titre{exo7 7672}
\auteur{mourougane}
\organisation{exo7}
\datecreate{2021-08-11}
\isIndication{false}
\isCorrection{false}
\chapitre{Sous-variété}
\sousChapitre{Sous-variété}

\contenu{
\texte{
Soit $S_1$ d'équation $x^6+y^6+z^6=1$ et $S_ 2$ d'équation $x^2+y^2+z^2=1$.
}
\begin{enumerate}
    \item \question{Montrer que $S_1$ et $S_2$ sont deux sous-surfaces différentiables de $\Rr^3$.}
    \item \question{On considère l'application $f$ de $S_1$ vers $S_2$, qui à $p$ de coordonnées $(x,y,z)$
associe le point de coordonnées $(x^3, y^3, z^3)$. Montrer que $f$ est bijective de $S_1$ sur $S_2$.}
    \item \question{Montrer que $f$ est différentiable.}
    \item \question{Déterminer la différentielle de $f$. Est-elle inversible ?}
    \item \question{La bijection réciproque $f^{-1}$ est-elle différentiable ?}
    \item \question{Reprenez l'exercice pour l'application $g$ de $S_1$ vers le plan d'équation $z=0$,
qui à $p$ de coordonnées $(x,y,z)$
associe le point de coordonnées $(x,x,0)$.}
\end{enumerate}
}

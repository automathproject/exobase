\uuid{9BfR}
\exo7id{7698}
\auteur{mourougane}
\organisation{exo7}
\datecreate{2021-08-11}
\isIndication{false}
\isCorrection{true}
\chapitre{Sous-variété}
\sousChapitre{Sous-variété}

\contenu{
\texte{
La courbure moyenne de la surface paramétrée suivante est-elle partout nulle ? 
$$\begin{array}{ccc}
 F: \Rr \times ]-\pi,\pi[&\to&\Rr ^3\\ 
\left(\begin{array}{c}u \\ v\end{array}\right)
&\mapsto& 
\left(\begin{array}{c}\cosh(u)\cos(v)\\ \cosh(u)\sin(v)\\ u\end{array}\right).
\end{array}$$
}
\reponse{
L'application $F$ est injective et différentiable car ses composantes le sont.
 On calcule
$$X=_u=\frac{\partial F}{\partial u}=\begin{pmatrix}\sinh(u)\cos(v)\\ \sinh(u)\sin(v)\\ 1\end{pmatrix}
\quad \text{ et } \quad X_v=\frac{\partial F}{\partial v}=\begin{pmatrix}-\cosh(u)\sin(v)\\ \cosh(u)\cos(v)\\0\end{pmatrix}$$
pour déterminer 
le rang de $dF$, qui est donc $2$. Par conséquent, l'image $S$ de $F$ est une surface régulière.
On peut aussi déterminer
 $$G=\cosh^2 u\begin{pmatrix} 1&0\\0&1\end{pmatrix}.$$
On s'en sert aussi pour trouver un champs de vecteurs normaux unitaires différentiable
              $N=\frac{1}{\cosh u}\begin{pmatrix}-\cos v\\-\sin v\\\sinh u \end{pmatrix}$.
On peut ensuite, soit calculer 
$$W(X_u)=-dN\cdot X_u=-\frac{\partial N}{\partial u}=-(\cosh^{-2}u) X_u$$
$$\text{ et } \quad W(X_v)=-dN\cdot X_v=-\frac{\partial N}{\partial u}=(\cosh^{-2}u)
X_v$$
et en déduire que les valeurs propres de l'endomorphisme de Weingarten $W$ sont $-\cosh^{-2}u$ et $\cosh^{-2}u$.
Donc, $W$ est de demi-trace nulle, et la courbure moyenne de $Im F$ est partout nulle.

On peut aussi calculer
$\frac{\partial^2 F}{\partial u^2}$,
$\frac{\partial^2 F}{\partial u\partial v}$ et 
$\frac{\partial^2 F}{\partial u^2}$ pour obtenir $H=\begin{pmatrix}-1&0\\0&1\end{pmatrix}$.
On calcule $G^{-1}=\cosh^{-2} u\begin{pmatrix} 1&0\\0&1\end{pmatrix}$
Donc $W=G^{-1}H=\begin{pmatrix}-\cosh^{-2} u&0\\0&\cosh^{-2} u\end{pmatrix}$.
Comme la demi-trace de $W$ est nulle, la surface $S$ est partout de courbure moyenne nulle.
}
}

\uuid{Adm1}
\exo7id{7676}
\titre{exo7 7676}
\auteur{mourougane}
\organisation{exo7}
\datecreate{2021-08-11}
\isIndication{false}
\isCorrection{false}
\chapitre{Sous-variété}
\sousChapitre{Sous-variété}

\contenu{
\texte{

}
\begin{enumerate}
    \item \question{On considère le plan d'équation $z=0$ avec le paramétrage 
$\left\{\begin{array}{ccc}
 F: \Rr^2&\to&\Rr^3\\ (x,y)&\mapsto& (x,y,0).
\end{array}\right.$\\
Déterminer la matrice de la première forme fondamentale dans la base $\mathcal{B}_F$ correspondante.}
    \item \question{On considère le plan d'équation $z=0$ avec le paramétrage local 
\begin{eqnarray*} G: ]0,+\infty[\times ]-\pi,\pi[&\to&\Rr^3\\ (r,\theta)&\mapsto& (r\cos\theta,r\sin\theta,0).\end{eqnarray*}
Déterminer la matrice de la première forme fondamentale dans la base $\mathcal{B}_G$ correspondante.}
    \item \question{Déterminer la matrice de changement de base au point de coordonnées $(1,0,0)$ de la base $\mathcal{B}_F$
dans la base $\mathcal{B}_G$. Relier les deux matrices obtenues pour la première forme fondamentale.}
\end{enumerate}
}

\uuid{MgA9}
\exo7id{7701}
\titre{exo7 7701}
\auteur{mourougane}
\organisation{exo7}
\datecreate{2021-08-11}
\isIndication{false}
\isCorrection{true}
\chapitre{Sous-variété}
\sousChapitre{Sous-variété}
\module{Géométrie différentielle}
\niveau{L3}
\difficulte{}

\contenu{
\texte{
On considère la courbe paramétrée
$$\begin{array}{cccc}
        c~:~&]0,1[&\to&\Rr ^3\\
&t&\mapsto&\left(\begin{array}{c}3\cos t\\3\sin t\\4t\end{array}\right)
        \end{array}
$$
}
\begin{enumerate}
    \item \question{La courbe $c$ est-elle régulière ?}
\reponse{Comme les composantes de $c$ sont de classe $\mathcal{C}^\infty$, elle est de classe $\mathcal{C}^\infty$.
  La dérivée de la dernière coordonnée ne s'annule jamais.
 La courbe $c$ est donc régulière.}
    \item \question{Paramétrer l'image de $c$ par la longueur d'arc.}
\reponse{Le vecteur vitesse
  $$\dot{c}(t)=\begin{pmatrix}-3\sin t\\3\cos t\\4\end{pmatrix}$$
 est de norme $5$. Comme la courbe $c$ est régulière, on peut trouver un paramétrage par longueur d'arc.
 Le paramétrage
 $$\begin{array}{cccc}
         e~:~&]0,5[&\to&\Rr ^3\\
 &s&\mapsto&\left(\begin{array}{c}3\cos (s/5)\\3\sin (s/5)\\4s/5\end{array}\right)
         \end{array}
 $$
 de la forme $e(s)=c(\phi(s))$ est un reparamétrage par longueur d'arc, car $de/ds$ est partout de norme $1$.}
    \item \question{Déterminer en tout point de $c$ le repère de Frenet.}
\reponse{On utilise le paramétrage par longueur d'arc.
  Le premier vecteur est le vecteur vitesse $\dot{e}(s)=v(s)=\begin{pmatrix}-3/5\sin (s/5)\\3/5\cos (s/5)\\4/5\end{pmatrix}$.
 Le deuxième est le vecteur normal. Le vecteur accélération est $\ddot{e}(s)=\begin{pmatrix}-3/25\cos (s/5)\\-3/25\sin (s/5)\\0\end{pmatrix}$.
 On trouve que la courbure est $3/25$ et le vecteur normal 
 $n(s)=\begin{pmatrix}-\cos (s/5)\\-\sin (s/5)\\0\end{pmatrix}$.
 Le troisième vecteur est le vecteur binormal
 $b(s)=v(s)\wedge n(s)=\begin{pmatrix}-3/5\sin (s/5)\\3/5\cos (s/5)\\4/5\end{pmatrix}
 \wedge\begin{pmatrix}-\cos (s/5)\\-\sin (s/5)\\0\end{pmatrix}
 =\begin{pmatrix} 4/5\sin (s/5)\\-4/5 \cos (s/5)\\ 3/5\end{pmatrix}$.}
\end{enumerate}
}

\uuid{gzVI}
\exo7id{7402}
\titre{exo7 7402}
\auteur{mourougane}
\organisation{exo7}
\datecreate{2021-08-10}
\isIndication{false}
\isCorrection{true}
\chapitre{Groupe, anneau, corps}
\sousChapitre{Autre}

\contenu{
\texte{

}
\begin{enumerate}
    \item \question{Les groupes $\Z/9\Z$ et $\Z/3\Z\times \Z/3\Z$ sont-ils isomorphes ?}
\reponse{Non, car $\Z/3\Z\times \Z/3\Z$ ne contient aucun élément d'ordre 9.}
    \item \question{Les groupes $\mathbb{F}_7$ et $\Z/2\Z\times \Z/3\Z$ sont-ils isomorphes ? Justifier.}
\reponse{Non, car ils n'ont pas le même ordre.}
    \item \question{Les groupes $(\mathbb{F}_7)^\star$ et $\Z/2\Z\times \Z/3\Z$ sont-ils isomorphes ? Justifier.}
\reponse{Les groupes $(\mathbb{F}_7)^\star$ et $\Z/2\Z\times \Z/3\Z$ sont-ils isomorphes ? Par le théorème sur le groupe des inversibles d'un corps fini et par le théorème chinois, ils sont tous les deux cycliques d'ordre $6$, donc isomorphes.}
\end{enumerate}
}

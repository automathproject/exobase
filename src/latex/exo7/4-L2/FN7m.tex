\uuid{FN7m}
\exo7id{5357}
\titre{exo7 5357}
\auteur{rouget}
\organisation{exo7}
\datecreate{2010-07-04}
\isIndication{false}
\isCorrection{true}
\chapitre{Groupe, anneau, corps}
\sousChapitre{Groupe de permutation}
\module{Algèbre}
\niveau{L2}
\difficulte{}

\contenu{
\texte{
Soit $(G,\times)$ un groupe. Montrer que $(G,\times)$ est isomorphe à un sous-groupe de $(S(G),\circ)$ et que, en particulier, tout groupe fini d'ordre $n$ est isomorphe à un sous-groupe de $S_n$ (théorème de \textsc{Cayley}). (Indication~:~montrer que pour chaque $x$ de $G$, l'application $y\mapsto xy$ est une permutation de $G$.)
}
\reponse{
Soit $(G,\times)$ un groupe. Pour $x$ élément de $G$, on considère 
$\begin{array}[t]{cccc}
f_x~:&G&\rightarrow&G\\
 &y&\mapsto&xy
\end{array}$. $f_x$ est une application de $G$ vers $G$ et de plus, clairement $f_x\circ f_{x^{-1}}=f_{x^{-1}}\circ f_x=Id_G$. Donc, pour tout élément $x$ de $G$, $f_x$ est une permutation de $G$.

Soit alors $\begin{array}[t]{cccc}
\varphi~:&(G,\times)&\rightarrow&(S_G,\circ)\\
 &x&\mapsto&f_x
\end{array}$. D'après ce qui précède, $\varphi$ est une application. De plus, $\varphi$ est de plus un morphisme de groupes. En effet, pour $(x,x',y)\in G^3$, on a~:

$$\varphi((xx'))(y)=f_{xx'}(y)=xx'y=f_x(f_x'(y)) = f_x\circ f_{x'}(y)=(\varphi(x)\circ\varphi(x'))(y),$$

et donc $\forall(x,x')\in G^2,\;\varphi(xx')=\varphi(x)o\varphi(x')$.

Enfin, $\varphi$ est injectif car, pour $x$ élément de $G$~:

$$\varphi(x)= Id\Rightarrow\forall y\in G,\;xy=y\Rightarrow xe=e\Rightarrow x=e.$$

Donc, $\mbox{Ker}\varphi=\{e\}$, et $\varphi$ est injectif.

$\varphi$ est ainsi un isomorphisme de groupes de $(G,\times)$ sur $(f(G),\circ)$ qui est un sous groupe de $(S_G,\circ)$. $(G,\times)$ est bien isomorphe à un sous groupe de $(S_G,\circ)$.
}
}

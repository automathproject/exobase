\uuid{8MaR}
\exo7id{5663}
\titre{exo7 5663}
\auteur{rouget}
\organisation{exo7}
\datecreate{2010-10-16}
\isIndication{false}
\isCorrection{true}
\chapitre{Réduction d'endomorphisme, polynôme annulateur}
\sousChapitre{Diagonalisation}

\contenu{
\texte{
Soient $A$ un élément de $\mathcal{M}_n(\Cc$) et $M$ l'élément de $\mathcal{M}_{2n}(\Cc)$ défini par blocs par $M=\left(
\begin{array}{cc}
A&4A\\
A&A
\end{array}
\right)$.

Calculer $\text{det}M$. Déterminer les éléments propres de $M$ puis montrer que $M$ est diagonalisable si et seulement si $A$ est diagonalisable.
}
\reponse{
$\text{det}M=\text{det}\left(
\begin{array}{cc}
A&4A\\
A&A
\end{array}
\right)=\text{det}\left(
\begin{array}{cc}
-3A&4A\\
0&A
\end{array}
\right)$  ($\forall k\in\llbracket1,n\rrbracket$, $L_k\leftarrow L_k-L_{n+k}$) et donc $\text{det}M=\text{det}(A)\text{det}(-3A) = (-3)^n(\text{det}A)^2$.

\begin{center}
\shadowbox{
$\text{det}M=(-3)^n(\text{det}A)^2$.
}
\end{center}

L'idée de l'étude de $M$ qui suit vient de l'étude de la matrice de format $2$, $B=\left(
\begin{array}{cc}
1&4\\
1&1
\end{array}
\right)$.

Une diagonalisation rapide amène à $B=\left(
\begin{array}{cc}
-2&2\\
1&1
\end{array}
\right)\times\left(
\begin{array}{cc}
-1&0\\
0&3
\end{array}
\right)\times\frac{1}{4}\left(
\begin{array}{cc}
-1&2\\
1&2
\end{array}
\right)$. Soit alors $P$ la matrice de format $2n$ définie par blocs par $P=\left(
\begin{array}{cc}
-2I_n&2I_{n}\\
I_n&I_n
\end{array}
\right)$.
Un calcul par blocs montre que $P$ est inversible et que $P^{-1}=\frac{1}{4}\left(
\begin{array}{cc}
-I_n&2I_{n}\\
I_n&2I_n
\end{array}
\right)$ puis que

\begin{center} 
$P^{-1}MP=\frac{1}{4}\left(
\begin{array}{cc}
-I_n&2I_{n}\\
I_n&2I_n
\end{array}
\right)\left(
\begin{array}{cc}
A&4A\\
A&A
\end{array}
\right)\left(
\begin{array}{cc}
-2I_n&2I_{n}\\
I_n&I_n
\end{array}
\right)=\frac{1}{4}\left(
\begin{array}{cc}
A&-2A\\
3A&6A
\end{array}
\right)\left(
\begin{array}{cc}
-2I_n&2I_{n}\\
I_n&I_n
\end{array}
\right)
=\left(
\begin{array}{cc}
-A&0\\
0&3A
\end{array}
\right)$.
\end{center}

On pose $N=\left(
\begin{array}{cc}
-A&0\\
0&3A
\end{array}
\right)$. Puisque les matrices $M$ et $N$ sont semblables, $M$ et $N$ ont même polynôme caractéristique et de plus $M$ est diagonalisable si et seulement si $N$ l'est.

Cherchons les vecteurs propres $Z$ de $N$ sous la forme $Z=\left(
\begin{array}{c}
X\\
Y
\end{array}
\right)$ où $X$ et $Y$ sont des vecteurs colonnes de format $n$. Sois $\lambda\in\Cc$.

\begin{center}
$NZ=\lambda Z\Leftrightarrow\left(
\begin{array}{cc}
-A&0\\
0&3A
\end{array}
\right)\left(
\begin{array}{c}
X\\
Y
\end{array}
\right)=\lambda\left(
\begin{array}{c}
X\\
Y
\end{array}
\right)\Leftrightarrow -AX =\lambda X\;\text{et}\;3AY =\lambda Y$.
\end{center}

Par suite

\begin{align*}\ensuremath
Z\;\text{est vecteur propre de}\;N\;\text{associé à}\;\lambda\Leftrightarrow(X\neq0\;\text{ou}\;Y\neq0)\;\text{et}\; 
(X\in\text{Ker}(A+\lambda I)\;\text{et}\;Y\in\text{Ker}\left(A-\frac{\lambda}{3}I\right)).
\end{align*}

Une discussion suivant $\lambda$ s'en suit :

\textbf{1er cas.} Si $-\lambda$ et $\frac{\lambda}{3}$ ne sont pas valeurs propres de $A$ alors $\lambda$ n'est pas valeur propre de $M$.

\textbf{2ème cas.} Si $-\lambda$ est dans $\text{Sp}A$ et $\frac{\lambda}{3}$ n'y est pas, alors $\lambda$ est valeur propre de $M$.
Le sous-espace propre associé est l'ensemble des $P\left(
\begin{array}{c}
X\\
0
\end{array}
\right)=\left(
\begin{array}{c}
-2X\\
X
\end{array}
\right)$ où $X$ décrit $\text{Ker}(A+\lambda I)$. La dimension de $E_\lambda$ est alors $\text{dim}(\text{Ker}(A+\lambda I)$.

\textbf{3ème cas.} Si $-\lambda$ n'est pas dans $\text{Sp}A$ et $\frac{\lambda}{3}$ y est, alors $\lambda$ est valeur propre de $M$.
Le sous-espace propre associé est l'ensemble des $P\left(
\begin{array}{c}
0\\
Y
\end{array}
\right)=\left(
\begin{array}{c}
2Y\\
Y
\end{array}
\right)$ où $Y$ décrit $\text{Ker}\left(A-\frac{\lambda}{3}I\right)$. La dimension de $E_\lambda$ est alors $\text{dim}\left(\text{Ker}\left(A-\frac{\lambda}{3}I\right)\right)$.

\textbf{4ème cas.} Si $-\lambda$ est dans $\text{Sp}A$ et $\frac{\lambda}{3}$  aussi, alors $\lambda$ est valeur propre de $M$.
Le sous-espace propre associé est l'ensemble des $P\left(
\begin{array}{c}
X\\
Y
\end{array}
\right)=\left(
\begin{array}{c}
-2X+2Y\\
X+Y
\end{array}
\right)$ où $X$ décrit $\text{Ker}(A+\lambda I)$  et $Y$ décrit $\text{Ker}\left(A-\frac{\lambda}{3}I\right)$. La dimension de $E_\lambda$ est alors $\text{dim}(\text{Ker}(A+\lambda I)) +\text{dim}\left(\text{Ker}\left(A-\frac{\lambda}{3}I\right)\right)$.

Dans tous les cas, $\text{dim}(E_\lambda(M))=\text{dim}(E_{-\lambda}(A))+\text{dim}(E_{\lambda/3}(A))$ (et en particulier $\text{dim}(\text{Ker}M)=2\text{dim}(\text{Ker}A)$).
Comme les applications $\lambda\mapsto -\lambda$ et $\lambda\mapsto\frac{\lambda}{3}$ sont des bijections de $\Cc$ sur lui-même,

\begin{align*}\ensuremath
A\;\text{est diagonalisable}&\Leftrightarrow\sum_{\lambda\in\Cc}^{}\text{dim}(E_{\lambda}(A)) = n\\
 &\Leftrightarrow\sum_{\lambda\in\Cc}^{}\text{dim}(E_{\lambda}(A))+\sum_{\lambda\in\Cc}^{}\text{dim}(E_{\lambda}(A))= 2n\\
 &\Leftrightarrow\sum_{\lambda\in\Cc}^{}\text{dim}(E_{-\lambda}(A))+\sum_{\lambda\in\Cc}^{}\text{dim}(E_{\lambda/3}(A))= 2n\\
 &\Leftrightarrow M\;\text{est diagonalisable}.
\end{align*}
}
}

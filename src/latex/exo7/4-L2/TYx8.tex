\uuid{TYx8}
\exo7id{5793}
\titre{exo7 5793}
\auteur{rouget}
\organisation{exo7}
\datecreate{2010-10-16}
\isIndication{false}
\isCorrection{true}
\chapitre{Espace euclidien, espace normé}
\sousChapitre{Espace vectoriel euclidien de dimension 3}
\module{Algèbre}
\niveau{L2}
\difficulte{}

\contenu{
\texte{
Soit $(e_1,e_2,e_3)$ une base orthonormée directe d'un espace euclidien orienté $E$ de dimension $3$. Matrice de la rotation d'angle $\frac{\pi}{3}$ autour de $e_1+e_2$.
}
\reponse{
Posons $k =\frac{1}{\sqrt{2}}(e_1+e_2)$. Soit $u=xe_1+ye_2+ze_3\in E$. On sait que

\begin{align*}\ensuremath
r(u)&=(\cos\theta)u+(1-\cos\theta)\left(u|k\right)k+(\sin\theta)k\wedge u=\frac{1}{2}u +\frac{1}{4}(u|(e_1+e_2))(e_1+e_2)+\frac{\sqrt{6}}{4}(e_1+e_2)\wedge u\\
 &\text{\og}=\text{\fg}\;\frac{1}{2}\left(
 \begin{array}{c}
 x\\
 y\\
 z
 \end{array}
 \right)+\frac{1}{4}(x+y)\left(
 \begin{array}{c}
 1\\
 1\\
 0
 \end{array}
 \right)+\frac{\sqrt{6}}{4}\left(
 \begin{array}{c}
 1\\
 1\\
 0
 \end{array}
 \right)\wedge\left(
 \begin{array}{c}
 x\\
 y\\
 z
 \end{array}
 \right)\\
  &=\frac{1}{2}\left(
 \begin{array}{c}
 x\\
 y\\
 z
 \end{array}
 \right)+\frac{1}{4}(x+y)\left(
 \begin{array}{c}
 1\\
 1\\
 0
 \end{array}
 \right)+\frac{\sqrt{6}}{4}\left(
 \begin{array}{c}
 z\\
 -z\\
 -x+y
 \end{array}
 \right)\\
 &=\left(
 \begin{array}{c}
 \frac{3}{4}x+\frac{1}{4}y+\frac{\sqrt{6}}{4}z\\
\rule[-4mm]{0mm}{11mm}\frac{1}{4}x+\frac{3}{4}y-\frac{\sqrt{6}}{4}z\\
-\frac{\sqrt{6}}{4}x+\frac{\sqrt{6}}{4}y+\frac{1}{2}z
 \end{array}
 \right)=\left(
 \begin{array}{cccc}
 \frac{3}{4}&\frac{1}{4}&\frac{\sqrt{6}}{4}\\
\rule[-4mm]{0mm}{11mm}\frac{1}{4}&\frac{3}{4}&-\frac{\sqrt{6}}{4}\\
-\frac{\sqrt{6}}{4}&\frac{\sqrt{6}}{4}&\frac{1}{2}
 \end{array}
 \right)\left(
 \begin{array}{c}
x\\
y\\
z
 \end{array}
 \right)
\end{align*}

et la matrice cherchée est

\begin{center}
\shadowbox{
$M=\left(
 \begin{array}{cccc}
 \frac{3}{4}&\frac{1}{4}&\frac{\sqrt{6}}{4}\\
\rule[-4mm]{0mm}{11mm}\frac{1}{4}&\frac{3}{4}&-\frac{\sqrt{6}}{4}\\
-\frac{\sqrt{6}}{4}&\frac{\sqrt{6}}{4}&\frac{1}{2}
 \end{array}
 \right)
 $.
 }
 \end{center}
}
}

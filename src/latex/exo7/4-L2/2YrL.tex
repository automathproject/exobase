\uuid{2YrL}
\exo7id{3565}
\titre{exo7 3565}
\auteur{quercia}
\organisation{exo7}
\datecreate{2010-03-10}
\isIndication{false}
\isCorrection{true}
\chapitre{Réduction d'endomorphisme, polynôme annulateur}
\sousChapitre{Polynôme annulateur}

\contenu{
\texte{
Le polyn{\^o}me $X^{4}+X^{3}+2X^{2}+X+1$ peut-il {\^e}tre le polyn{\^o}me minimal d'une matrice de $M_{5}(\R)$ ?
}
\reponse{
Le polyn{\^o}me s'{\'e}crit $(X^{2}+1)(X^{2}+X+1)$. Il n'a donc pas de racine r{\'e}elle. Or tout {\'e}l{\'e}ment de $M_{5}(\R)$
poss{\`e}de au moins une valeur propre et cette valeur propre devrait {\^e}tre {\'e}galement racine du polyn{\^o}me minimal. Par cons{\'e}quent
 $X^{4}+X^{3}+2X^{2}+X+1$ ne peut pas {\^e}tre le polyn{\^o}me minimal d'une matrice de $M_{5}(\R)$.
}
}

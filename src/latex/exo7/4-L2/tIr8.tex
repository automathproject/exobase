\uuid{tIr8}
\exo7id{3793}
\titre{exo7 3793}
\auteur{quercia}
\organisation{exo7}
\datecreate{2010-03-11}
\isIndication{false}
\isCorrection{false}
\chapitre{Espace euclidien, espace normé}
\sousChapitre{Endomorphismes auto-adjoints}
\module{Algèbre}
\niveau{L2}
\difficulte{}

\contenu{
\texte{
Soit $f\in\mathcal{L}(E)$.
}
\begin{enumerate}
    \item \question{En considérant l'endomorphisme $f^*\circ f$, montrer que si~$f$
    est inversible alors $f$ se décompose de manière unique sous
    la forme $f = u\circ h$ avec $u$ unitaire et $h$ hermitien positif.}
    \item \question{Si $f$ est non inversible, montrer qu'une telle décomposition
    existe mais n'est pas unique (on rappelle que $U(E)$ est compact).}
    \item \question{Montrer que l'application $f  \mapsto (u,h)$ est continue sur $GL(E)$.}
\end{enumerate}
}

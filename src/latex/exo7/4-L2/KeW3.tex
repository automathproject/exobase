\uuid{KeW3}
\exo7id{5792}
\titre{exo7 5792}
\auteur{rouget}
\organisation{exo7}
\datecreate{2010-10-16}
\isIndication{false}
\isCorrection{true}
\chapitre{Espace euclidien, espace normé}
\sousChapitre{Espace orthogonal}
\module{Algèbre}
\niveau{L2}
\difficulte{}

\contenu{
\texte{
On munit $E =\mathcal{M}_3(\Rr)$ muni du produit scalaire usuel.
}
\begin{enumerate}
    \item \question{Déterminer l'orthogonal de $\mathcal{A}_3(\Rr)$.}
\reponse{Déterminons l'orthogonal de $\mathcal{A}_3(\Rr)$ dans $\mathcal{M}_3(\Rr)$. Soit $(A,B)\in\mathcal{S}_(\Rr)\times\mathcal{A}_3(\Rr)$.

\begin{center}
$\left(A|B\right)=\text{Tr}({^t}AB) =\text{Tr}(AB) =\text{Tr}(BA) = -\text{Tr}({^t}BA) = -\left(B|A\right)$.
\end{center}

et donc $\left(A|B\right)= 0$. Donc $\mathcal{S}_3(\Rr)\subset\left(\mathcal{A}_3(\Rr)\right)^\bot$ et comme de plus, $\text{dim}\left(\mathcal{S}_3(\Rr)\right)=\text{dim}\left((\left(\mathcal{A}_3(\Rr)\right)^\bot\right)$, on a montré que

\begin{center}
\shadowbox{
$\left(\mathcal{A}_3(\Rr)\right)^\bot=\mathcal{S}_3(\Rr)$.
}
\end{center}}
    \item \question{Calculer la distance de la matrice $M =\left(
\begin{array}{ccc}
0&1&0\\
0&0&1\\
0&0&0
\end{array}
\right)$  au sous-espace vectoriel des matrices antisymétriques.}
\reponse{Ainsi, la projection orthogonale de $M$ sur $\mathcal{A}_3(\Rr)$ est exactement la partie antisymétrique $p_a(M)$ de $M$ et la distance cherchée est la norme de $M-p_a(M)=p_s(M)$ avec 

\begin{center}
$p_s(M)=\frac{1}{2}\left(\left(
\begin{array}{ccc}
0&1&0\\
0&0&1\\
0&0&0
\end{array}
\right)+
\left(
\begin{array}{ccc}
0&0&0\\
1&0&0\\
0&1&0
\end{array}
\right)\right)=\frac{1}{2}\left(
\begin{array}{ccc}
0&1&0\\
1&0&1\\
0&1&0
\end{array}
\right)$.
\end{center}

Par suite,

\begin{center}
$d\left(M,\mathcal{A}_3(R)\right)=\|p_s(M)\|=\frac{1}{2}\sqrt{1^2+1^2+1^2+1^2}=1$.
\end{center}}
\end{enumerate}
}

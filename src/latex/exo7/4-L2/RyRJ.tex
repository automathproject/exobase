\uuid{RyRJ}
\exo7id{1638}
\titre{exo7 1638}
\auteur{barraud}
\organisation{exo7}
\datecreate{2003-09-01}
\isIndication{false}
\isCorrection{false}
\chapitre{Réduction d'endomorphisme, polynôme annulateur}
\sousChapitre{Diagonalisation}
\module{Algèbre}
\niveau{L2}
\difficulte{}

\contenu{
\texte{
A $n$ nombres complexes $(a_{1},...,a_{n})\in\C^{n}$ avec $a_{2}\neq 0$,
on associe la matrice $A_{n}=
\begin{pmatrix}
  a_{1} &a_{2} &\cdots          &a_{n} \\
  a_{2} &      &                &      \\        
  \vdots&      &\text{\huge{0}} &      \\    
  a_{n} &      &                &     
\end{pmatrix}
$.
}
\begin{enumerate}
    \item \question{Quel est le rang de $A_{n}$. Qu'en déduit-on pour le polynôme
  caractéristique $\chi_{n}$ de $A_{n}$~?}
    \item \question{Calculer $\chi_{2}$, $\chi_{3}$.}
    \item \question{On pose $b_{n}=a_{2}^{2}+\cdots+a_{n}^{2}$. Par récurrence, montrer
  que $\chi_{n}=(-X)^{n-2}(X^{2}-a_{1}X-b_{n})$.}
    \item \question{Si $b_{n}=0$, $A_{n}$ est-elle diagonalisable~?}
    \item \question{Si $b_{n}\neq0$, à quelle condition $A_{n}$ est-elle
  diagonalisable~?}
\end{enumerate}
}

\uuid{OwR1}
\exo7id{3631}
\titre{exo7 3631}
\auteur{quercia}
\organisation{exo7}
\datecreate{2010-03-10}
\isIndication{false}
\isCorrection{true}
\chapitre{Endomorphisme particulier}
\sousChapitre{Autre}

\contenu{
\texte{
Soit $E = \R_3[X]$ et $a,b,c \in \R$ distincts.
On considère les formes linéaires sur E :
$$f_a\ :\ P \mapsto P(a),\quad
  f_b\ :\ P \mapsto P(b),\quad
  f_c\ :\ P \mapsto P(c),\quad
  \varphi\ :\ P  \mapsto  \int_{t=a}^b P(t)\,d t.$$
\'Etudier la liberté de $(f_a,f_b,f_c,\varphi)$.
}
\reponse{
$M = \begin{pmatrix} 1   &1   &1   &b-a         \cr
                        a   &b   &c   &(b^2-a^2)/2 \cr
                        a^2 &b^2 &c^2 &(b^3-a^3)/3 \cr
                        a^3 &b^3 &c^3 &(b^4-a^4)/4 \cr \end{pmatrix}$ et
         $\det(M) = (b-a)^4(c-a)(c-b)\frac{2c-a-b}{12}$,
         donc la famille est libre si et seulement si $c \ne \frac {a+b}2$.
}
}

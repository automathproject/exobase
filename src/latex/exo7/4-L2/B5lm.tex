\uuid{B5lm}
\exo7id{2564}
\auteur{delaunay}
\organisation{exo7}
\datecreate{2009-05-19}
\isIndication{false}
\isCorrection{true}
\chapitre{Endomorphisme particulier}
\sousChapitre{Endomorphisme du plan}

\contenu{
\texte{
Soit $E$ un espace vectoriel sur un corps $K$ ($K=\R$ ou $\C$), on appelle {\it projecteur}
un endomorphisme $p$ de $E$ v\'erifiant $p\circ p=p$. Soit $p$ un projecteur.
}
\begin{enumerate}
    \item \question{Montrer que $\mathrm{\mathrm{Id}}_E-p$ est un projecteur, calculer $p\circ(\mathrm{Id}_E-p)$ et $(\mathrm{Id}_E-p)\circ p$.}
\reponse{{\it Montrons que $\mathrm{Id}_E-p$ est un projecteur et calculons $p\circ(\mathrm{Id}_E-p)$ et $(\mathrm{Id}_E-p)\circ p$.}

On a $(\mathrm{Id}_E-p)\circ(\mathrm{Id}_E-p)=\mathrm{Id}_E-p-p+p^2=\mathrm{Id}_E-p$, car $p^2=p$, ce qui prouve que $\mathrm{Id}_E-p$ est un projecteur.

Par ailleurs, on a
$$p\circ(\mathrm{Id}_E-p)=p-p^2=p-p=0=(\mathrm{Id}_E-p)\circ p$$
donc pour tout $\vec x\in E$, on a $p(\vec x-p(\vec x))=\vec 0$.}
    \item \question{Montrer que pour tout $\vec x\in \Im p$, on a $p(\vec x)=\vec x$.}
\reponse{{\it Montrons que pour tout $\vec x\in \Im p$, on a $p(\vec x)=\vec x$.}

Soit $\vec x\in \Im p$, il existe $\vec y\in E$ tel que $\vec x=p(\vec y)$, on a donc 
$p(\vec x)=p^2(\vec y)=p(\vec y)=\vec x$.}
    \item \question{En d\'eduire que $\Im p$ et $\ker p$ sont suppl\'ementaires.}
\reponse{{\it On en d\'eduit que $\Im p$ et $\ker p$ sont suppl\'ementaires.}

Soit $\vec x\in E$, on peut \'ecrire $\vec x=p(\vec x)+\vec x -p(\vec x)$, consid\'erons $\vec x -p(\vec x)$, on a
$p(\vec x-p(\vec x))=0$

ce qui prouve que $\vec x -p(\vec x)\in \ker p$. Ainsi tout \'el\'ement de $E$ s'\'ecrit comme somme d'un 
\'el\'ement de $\Im p$, $p(\vec x)$, et d'un \'el\'ement de $\ker p$, $\vec x -p(\vec x)$, il nous reste \`a 
d\'emontrer que la somme est directe.

Soit $\vec x\in \Im p\cap\ker p$, on a, d'une part $p(\vec x)=\vec x$ d'apr\`es la question $2)$ car $\vec x\in \Im p$
et, d'autre part $p(\vec x)=\vec 0$ car $\vec x\in \ker p$, d'o\`u $\vec x=\vec 0$. On a donc 
$$E=\Im p\oplus\ker p.$$
(Sachant que $\dim E=\dim\ker p+\dim\Im p$, on pouvait se contenter de d\'emontrer que $\Im p\cap \ker p={\vec 0}$, ici nous avons
explicitement la d\'ecomposition.)}
    \item \question{Montrer que le rang de $p$ est \'egal \`a la trace de $p$. (On rappelle que la trace de la matrice
 d'un endomorphisme ne d\'epend pas de la base dans laquelle on exprime cette matrice.)}
\reponse{{\it Montrons que le rang de $p$ est \'egal \`a la trace de $p$.}

Notons $n$ la dimension de $E$ et consid\'erons une base de $E$ de la forme
$$(\vec{e_1},\cdots,\vec{e_k},\vec {e_{k+1}},\cdots,\vec{e_n})$$
o\`u $(\vec{e_1},\cdots,\vec{e_k})$ est une base de $\Im p$ et $(\vec {e_{k+1}},\cdots,\vec{e_n})$ une base de $\ker p$.
dans une telle base, la matrice de $p$ s'\'ecrit
$$M=\begin{pmatrix}I_k&0 \\ 0&0\end{pmatrix}$$
o\`u $I_k$ d\'esigne la matrice identit\'e $k\times k$, et les $0$ des blocs de z\'eros. Le rang de $p$ est \'egal \`a la 
dimension de $\Im p$ c'est-\`a-dire ici \`a $k$ et on a bien $k=\mathrm{Tr} M=\mathrm{Tr} p.$}
\end{enumerate}
}

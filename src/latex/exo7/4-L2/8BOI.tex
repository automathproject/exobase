\uuid{8BOI}
\exo7id{5794}
\titre{exo7 5794}
\auteur{rouget}
\organisation{exo7}
\datecreate{2010-10-16}
\isIndication{false}
\isCorrection{true}
\chapitre{Espace euclidien, espace normé}
\sousChapitre{Problèmes matriciels}
\module{Algèbre}
\niveau{L2}
\difficulte{}

\contenu{
\texte{
Soit $A$ une matrice carrée réelle symétrique positive de format $n$.
Montrer que $1 +\sqrt[n]{\text{det}(A)}\leqslant\sqrt[n]{\text{det}(I_n+A)}$.
}
\reponse{
La matrice $A$ est symétrique réelle positive. Donc ses valeurs propres $\lambda_1$,..., $\lambda_n$ sont des réels positifs. De plus,

\begin{center}
$\text{det}A =\lambda_1...\lambda_n$ et $\text{det}(I_n+A) =\chi_A(-1) =(1+\lambda_1)...(1+\lambda_n)$.
\end{center}

L'inégalité à démontrer équivaut donc à :

\begin{center}
$\forall(\lambda_1,...,\lambda_n)\in(\Rr^+)^n$, $1+\sqrt[n]{\prod_{k=1}^{n}\lambda_k}\leqslant\sqrt[n]{\prod_{k=1}^{n}(1+\lambda_k)}$.
\end{center}

Soit donc $(\lambda_1,...,\lambda_n)\in(\Rr^+)^n$. Si l'un des $\lambda_k$ est nul, l'inégalité est immédiate.

Supposons dorénavant tous les $\lambda_k$ strictement positifs. L'inégalité à démontrer s'écrit

\begin{center}
$\ln\left(1+\text{exp}\left(\frac{1}{n}\left(\ln(\lambda_1)+...+\ln(\lambda_n)\right)\right)\right)\leqslant\frac{1}{n}\left(\ln(1+\text{exp}(\ln(\lambda_1)))+ ... +\ln(1+\text{exp}(\ln(\lambda_n)))\right)$\quad$(*)$
\end{center}

ou encore $f\left(\frac{1}{n}(x_1+...+x_n)\right)\leqslant\frac{1}{n}(f(x_1)+...+f(x_n))$ où $\forall x\in\Rr$, $f(x) =\ln(1+e^x)$ et $\forall k\in\llbracket1,n\rrbracket$, $x_k =\ln(\lambda_k)$.

L'inégalité à démontrer est une inégalité de convexité. La fonction $f$ est deux fois dérivable sur $\Rr$ et pour tout réel $x$,

\begin{center}
$f'(x)=\frac{e^x}{e^x+1}=1-\frac{1}{e^x+1}$ puis $f''(x) =\frac{e^x}{(e^x+1)^2}\geqslant0$.
\end{center}

La fonction $f$ est donc convexe sur $\Rr$ ce qui démontre l'inégalité $(*)$.

\begin{center}
\shadowbox{
$\forall A\in\mathcal{S}_n^+(\Rr)$, $1 +\sqrt[n]{\text{det}(A)}\leqslant\sqrt[n]{\text{det}(I_n+A)}$.
}
\end{center}
}
}

\uuid{08cg}
\exo7id{5294}
\titre{exo7 5294}
\auteur{rouget}
\organisation{exo7}
\datecreate{2010-07-04}
\isIndication{false}
\isCorrection{true}
\chapitre{Arithmétique}
\sousChapitre{Arithmétique de Z}
\module{Algèbre}
\niveau{L2}
\difficulte{}

\contenu{
\texte{
Pour $n\in\Nn^*$, on pose $(1+\sqrt{2})^n=a_n+b_n\sqrt{2}$ où $(a_n,b_n)\in(\Nn^*)^2$. Montrer que $a_n\wedge b_n=1$.
}
\reponse{
Soit $n\in\Nn^*$. En développant $(1+\sqrt{2})^n$ par la formule du binôme de \textsc{Newton} et en séparant les termes où $\sqrt{2}$ apparaît à un exposant pair des termes où $\sqrt{2}$ apparaît à un exposant impair, on écrit $(1+\sqrt{2})^n$ sous la forme $a_n+b_n\sqrt{2}$ où $a_n$ et $b_n$ sont des entiers naturels non nuls.

Mais alors $(1-\sqrt{2})^n=a_n-b_n\sqrt{2}$ et donc 

$$(-1)^n=(1+\sqrt{2})^n(1-\sqrt{2})^n=(a_n+b_n\sqrt{2})(a_n-b_n\sqrt{2})=a_n^2-2b_n^2$$
ou finalement,
 
$$((-1)^na_n)a_n+(2(-1)^{n+1}b_n)b_n=1$$
 
où $(-1)^na_n=u$ et $2(-1)^{n+1}b_n=v$ sont des entiers relatifs. Le théorème de \textsc{Bezout} permet d'affirmer que $a_n$ et $b_n$ sont premiers entre eux.
}
}

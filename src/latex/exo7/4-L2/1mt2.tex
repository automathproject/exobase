\uuid{1mt2}
\exo7id{2608}
\titre{exo7 2608}
\auteur{delaunay}
\organisation{exo7}
\datecreate{2009-05-19}
\isIndication{false}
\isCorrection{true}
\chapitre{Réduction d'endomorphisme, polynôme annulateur}
\sousChapitre{Diagonalisation}

\contenu{
\texte{
Soit $A$ la matrice 
$$A=\begin{pmatrix}1&0&0 \\  1&-1&0 \\  -1&2&-1\end{pmatrix}$$ 
et $f$ l'endomorphisme de $\R^3$ associ\'e.
}
\begin{enumerate}
    \item \question{Factoriser le polyn\^ome caract\'eristique de $A$.}
\reponse{{\it  Déterminons et factorisons le polyn\^ome caract\'eristique de $A$.} 

On note $P_A(X)$ le polynôme caractéristique, on a
$$P_A(X)=\begin{vmatrix}1-X&0&0 \\  1&-1-X&0 \\  -1&2&-1-X\end{vmatrix}=(-1-X)\begin{vmatrix}1-X&0 \\  1&-1-X\end{vmatrix}=(-1-X)^2(1-X).$$
La matrice $A$ admet deux valeurs propres, $1$, valeur propre simple, et $-1$, valeur propre double.}
    \item \question{D\'eterminer les sous-espaces propres et caract\'eristiques de $A$.}
\reponse{{\it D\'eterminons les sous-espaces propres et les sous-espaces caract\'eristiques de $A$.}

La valeur propre $1$ est simple, le sous-espace propre associé est égal au sous-espace caractéristique, c'est l'ensemble
$$E_1=N_1=\{\vec u=(x,y,z)\in\R^3,\ A\vec u =\vec u\}.$$
On a 
$$\vec u\in E_1\iff \left\{\begin{align*}x&=x \\  x-y&=y \\  -x+2y-z&=z\end{align*}\right.\iff\left\{\begin{align*}x&=2y \\  z&=0\end{align*}\right.$$ 
L'espace $E_1$ est une droite vectorielle dont un vecteur directeur $\vec e_1$ est donné, par exemple, par $\vec e_1=(2,1,0)$.

Le sous-espace propre associé à la valeur propre $-1$ est défini par
$$E_{-1}=\{\vec u=(x,y,z)\in\R^3,\ A\vec u =-\vec u\}.$$
On a 
$$\vec u\in E_{-1}\iff \left\{\begin{align*}x&=-x \\  x-y&=-y \\  -x+2y-z&=-z\end{align*}\right.
\iff\left\{ \begin{align*}x&=0 \\  y&=0\end{align*}\right.$$ 
L'espace $E_{-1}$ est une droite vectorielle dont un vecteur directeur $\vec e_2$ est donné, par exemple, par $\vec e_2=(0,0,1)$. La dimension de $E_{-1}$ n'est pas égale à la multiplicité de la racine, la matrice n'est pas diagonalisable. Déterminons le sous-espace caractéristique associé à la valeur propre $-1$. Pour cela calculons la matrice $(A+I)^2$.
$$(A+I)^2=\begin{pmatrix}2&0&0 \\  1&0&0 \\  -1&2&0\end{pmatrix}^2=\begin{pmatrix}4&0&0 \\  2&0&0 \\  0&0&0\end{pmatrix}$$
$$N_{-1}=\ker(A+I)^2=\{\vec u=(x,y,z)\in\R^3,\ x=0\}$$
Le sous-espace caractéristique $N_{-1}$ est le plan vectoriel engendré par les vecteurs 

$e_2=(0,0,1)$ et $e_3=(0,1,0)$.}
    \item \question{D\'emontrer qu'il existe une base de $\R^3$ dans laquelle la matrice de $f$ est 
$$B=\begin{pmatrix}1 & 0 & 0 \\  0&-1&2 \\ 0&0&-1\end{pmatrix}$$
et trouver une matrice $P$ inversible telle que $ AP=PB$ (ou $A=PBP^{-1}$).}
\reponse{{\it D\'emontrons qu'il existe une base de $\R^3$ dans laquelle la matrice de $f$ est 
$$B=\begin{pmatrix}1 & 0 & 0 \\  0&-1&2 \\ 0&0&-1\end{pmatrix}$$
et trouvons une matrice $P$ inversible telle que $ AP=PB$ (ou $A=PBP^{-1}$).} 

On considère les vecteurs $\vec e_1=(2,1,0)$ et $\vec e_2=(0,0,1)$ et on cherche un vecteur $\vec e\in N_{-1}$ tel que 
$f(\vec e)=2\vec e_2-\vec e$. Notons $\vec e=(0,y,z)$,
$$f(\vec e)=2\vec e_2-\vec e\iff y=1,$$
le vecteur $\vec e=\vec e_3=(0,1,0)$ convient, on pouvait le voir directement sur la deuxième colonne de la matrice $A$.
Ainsi, dans la base $(\vec e_1,\vec e_2,\vec e_3)$ avec $\vec e_1=(2,1,0)$, $\vec e_2=(0,0,1)$ et $\vec e_3=(0,1,0)$, la matrice de $f$ s'écrit
$$B=\begin{pmatrix}1 & 0 & 0 \\  0&-1&2 \\ 0&0&-1\end{pmatrix}.$$
La matrice $P$ cherchée est la matrice de passage qui exprime la base $(\vec e_1,\vec e_2,\vec e_3)$ dans la base canonique $(\vec i,\vec j,\vec k)$. On a
$$P=\begin{pmatrix}2 & 0 & 0 \\  1&0&1 \\ 0&1&0\end{pmatrix}$$
et $AP=PB$ ou $A=PBP^{-1}$. On peut calculer $P^{-1}$, c'est la matrice qui exprime les vecteurs $\vec i,\vec j$ et $\vec k$ dans la base 
$(\vec e_1,\vec e_2,\vec e_3)$,
$$P^{-1}=\begin{pmatrix}1/2 & 0 & 0 \\  0&0&1 \\ -1/2&1&0\end{pmatrix}.$$}
    \item \question{Ecrire la d\'ecomposition de Dunford de $B$ (justifier).}
\reponse{{\it Ecrivons la d\'ecomposition de Dunford de $B$.}  

On a
$$B=\begin{pmatrix}1 & 0 & 0 \\  0&-1&2 \\ 0&0&-1\end{pmatrix}=\underbrace{\begin{pmatrix}1 & 0 & 0 \\  0&-1&0 \\ 0&0&-1\end{pmatrix}}_D+\underbrace{\begin{pmatrix}0 & 0 & 0 \\  0&0&2 \\ 0&0&0\end{pmatrix}}_N.$$
Il est clair que la matrice $D$ est diagonalisable puisque diagonale, on vérifie facilement que $N^2=0$, c'est-à-dire que la matrice $N$ est nilpotente et que les deux matrices commutent, $DN=ND$. Ainsi la décomposition $B=D+N$ est bien la décomposition de Dunford de la matrice $B$.}
    \item \question{Pour $t\in\R$, calculer $\exp tB$.}
\reponse{{\it Pour $t\in\R$, calculons $\exp tB$}.

 On utilise la décomposition de Dunford de la matrice $tB$, $tB=tD+tN$, on a donc 
 $$\exp(tB)=\exp(tD+tN)=\exp(tD).\exp(tN)$$
car les matrices commutent, par ailleurs, comme $D$ est diagonale, on a
$$\exp(tD)=\begin{pmatrix}e^t & 0 & 0 \\  0&e^{-t}&0 \\ 0&0&e^{-t}\end{pmatrix},$$ 
et comme $N^2=0$, on a
$$\exp(tN)=I+tN=\begin{pmatrix}1 & 0 & 0 \\  0&1&2t \\ 0&0&1\end{pmatrix}.$$
D'où
$$\exp(tB)=\begin{pmatrix}e^t & 0 & 0 \\  0&e^{-t}&0 \\ 0&0&e^{-t}\end{pmatrix}\begin{pmatrix}1 & 0 & 0 \\  0&1&2t \\ 0&0&1\end{pmatrix}=\begin{pmatrix}e^t & 0 & 0 \\  0&e^{-t}&2te^{-t} \\ 0&0&e^{-t}\end{pmatrix}.$$}
    \item \question{Donner les solutions des syst\`emes diff\'erentiels $y'=By$ et $x'=Ax$, où $x$ et $y$ désignent des fonctions réelles à valeurs dans $\R^3$.}
\reponse{{\it Donnons les solutions des syst\`emes diff\'erentiels $y'=By$ et $x'=Ax$, où $x$ et $y$ désignent des fonctions réelles à valeurs dans $\R^3$.} 

Les solutions du système différentiel $y'(t)=B.y(t)$ sont les fonctions $y(t)=\exp(tB).V$ où $V$ est un vecteur quelconque de $\R^3$. Donc
$$y(t)=\begin{pmatrix}e^t & 0 & 0 \\  0&e^{-t}&2te^{-t} \\ 0&0&e^{-t}\end{pmatrix}\begin{pmatrix}a \\  b \\  c\end{pmatrix}=\begin{pmatrix}ae^t  \\  be^{-t}+2cte^{-t} \\  ce^{-t}\end{pmatrix},$$
$(a,b,c)\in\R^3$.

Pour trouver les solutions du système différentiel $x'(t)=A.x(t)$, on utilise le fait suivant
$$P.y\quad{\hbox{est solution de}}\quad x'=A.x\iff y\quad{\hbox{est solution de}}\quad y'=(P^{-1}AP).y=B.y.$$
Ainsi, les solutions du système $x'=A.x$ s'écrivent
$$x(t)=P.\exp(tB).V$$
où $V$ est un vecteur quelconque de $\R^3$. On remarque qu'il n'est pas utile de calculer la martrice $P^{-1}$. C'est-à-dire
$$x(t)=\begin{pmatrix}2 & 0 & 0 \\  1&0&1 \\ 0&1&0\end{pmatrix}\begin{pmatrix}ae^t  \\  be^{-t}+2cte^{-t} \\  ce^{-t}\end{pmatrix}=\begin{pmatrix}2ae^t  \\  ae^{t}+ce^{-t} \\  be^{-t}+2cte^{-t}\end{pmatrix},$$
$(a,b,c)\in\R^3$.}
\end{enumerate}
}

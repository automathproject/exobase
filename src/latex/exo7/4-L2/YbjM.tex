\uuid{YbjM}
\exo7id{1341}
\titre{exo7 1341}
\auteur{ortiz}
\organisation{exo7}
\datecreate{1999-04-01}
\isIndication{false}
\isCorrection{true}
\chapitre{Groupe, anneau, corps}
\sousChapitre{Ordre d'un élément}

\contenu{
\texte{

}
\begin{enumerate}
    \item \question{Soient  $G$ un groupe et $x, y\in G$ des \'el\'ements
qui commutent et d'ordres respectifs $m$ et $n$
premiers entre eux. Montrer que $xy$ est d'ordre
$mn.$ Montrer que l'hypoth\`ese $m$ et $n$
\textit{premiers entre eux} est indispensable.}
\reponse{Déjà $(xy)^{mn} = x^{mn}y^{mn}= (x^m)^n(y^n)^m=e.e=e$. Soit $p$
tel que $(xy)^p = e$, alors $e= (xy)^{mp} = x^{mp}y^{mp}=y^{mp}$,
et donc $mp$ est divisible par l'ordre de $y$ , c'est-à-dire $n$.
Comme $m$ et $n$ sont premiers entre eux alors d'après le théorème
de Gauss $n$ divise $p$. Un raisonnement semblable à partir de
$(xy)^{np}=e$ conduit à : $m$ divise $p$. Finalement $m|p$ et
$n|p$ donc $mn|p$ car $m$ et $n$ sont premiers entre eux.

Voici un contre exemple dans le cas o\`u $m$ et $n$ ne sont pas
premiers entre eux : dans le groupe $\Zz/12\Zz$ : $\bar{2}$ est
d'ordre $6$, $\bar{4}$ est d'ordre $3$, mais $\bar{2}+\bar{4} =
\bar{6}$ est d'ordre $2 \not= 3\times 6$.}
    \item \question{Montrer que $A:=\left(\begin{smallmatrix} 0&-1\\1&0
\end{smallmatrix}\right)$ et $B:=\left(\begin{smallmatrix}
0&1\\-1&-1\end{smallmatrix}\right)$ sont des
\'el\'ements de $\text{GL}(2,\Rr)$ d'ordres finis
et que $AB$ n'est pas d'ordre fini.}
\reponse{$A$ est d'ordre $4$, $B$ est d'ordre $3$, $(AB)^n =
\begin{pmatrix} 1&n\\0&1\\ \end{pmatrix}$ n'est jamais la matrice identité pout $n\geq 1$.}
\end{enumerate}
}

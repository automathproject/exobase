\uuid{X8pg}
\exo7id{1409}
\auteur{barraud}
\organisation{exo7}
\datecreate{2003-09-01}
\isIndication{false}
\isCorrection{false}
\chapitre{Groupe, anneau, corps}
\sousChapitre{Groupe de permutation}

\contenu{
\texte{
On note $\mathcal{S}_n$ le groupe sym\'{e}trique des permutations sur $n$ \'{e}l\'{e}ments.

Soit $\rho$ un \emph{morphisme de groupes} de $(\mathcal{S}_n,\circ)$ dans $(\{-1,1\},\cdot)$,
c'est \`{a} dire une application de $\mathcal{S}_n$ dans $\{-1,1\}$ satisfaisant
$$
  \forall(\sigma,\tau)\in\mathcal{S}_n\;\; \rho(\sigma\tau)=\rho(\sigma)\rho(\tau)
$$
}
\begin{enumerate}
    \item \question{Calculer $\rho(\mathrm{id})$. Pour tout cycle $\gamma$ de longueur $p$, calculer $\gamma^{p}$. En
d\'{e}duire que lorsque $p$ est impair, $\rho(\gamma)=1$.}
    \item \question{On suppose que pour toute transposition $\tau$, $\rho(\tau)=1$. Montrer que
$\forall\sigma\in\mathcal{S}_n,\;\rho(\sigma)=1$}
    \item \question{On suppose maintenant qu'il existe une transposition $\tau_{0}=(a,b)$ pour laquelle
$\rho(\tau_{0})=-1$.


\begin{enumerate}}
    \item \question{[(a)]
Pour un \'{e}l\'{e}ment $c\in\{1,\ldots,n\}\setminus\{a,b\}$, calculer $(a,b)(a,c)$. En d\'{e}duire
que $\rho(a,c)=-1$.}
    \item \question{[(b)]
Pour deux \'{e}l\'{e}ments distincts $c$ et $d$ de $\{1,\ldots,n\}$, calculer $(a,c)(a,d)(a,c)$.
En d\'{e}duire que $\rho(c,d)=-1$.}
    \item \question{[(c)] En déduire que pour toute transposition $\tau$, $\rho(\tau)=-1$ puis 
montrer que pour toute permutation $\sigma\in\mathcal{S}_n$, $\rho(\sigma)$ est la signature de
$\sigma$.}
\end{enumerate}
}

\uuid{lMKY}
\exo7id{5772}
\titre{exo7 5772}
\auteur{rouget}
\organisation{exo7}
\datecreate{2010-10-16}
\isIndication{false}
\isCorrection{true}
\chapitre{Espace euclidien, espace normé}
\sousChapitre{Orthonormalisation}
\module{Algèbre}
\niveau{L2}
\difficulte{}

\contenu{
\texte{
Soit $E =\Rr[X]$. On munit $E$ du produit scalaire $P|Q =\int_{-1}^{1}P(t)Q(t)\;dt$.
}
\begin{enumerate}
    \item \question{Pour $n\in\Nn$, on pose $L_n=\left((X^2-1)^n\right)^{(n)}$.
   \begin{enumerate}}
\reponse{\begin{enumerate}}
    \item \question{Montrer que la famille $(L_n)_{n\in\Nn}$ est une base orthogonale de l'espace préhilbertien $(E,\;|\;)$.}
\reponse{Soient $n\in\Nn^*$ et $P\in E$. Une intégration par parties fournit

\begin{center}
$\left(L_n|P\right)=\int_{-1}^{1}L_n(x)P(x)\;dx =\int_{-1}^{1}(\ell_n)^{(n)}(x) P(x)\;dx =\left[(\ell_n)^{(n-1)}(x) P(x)\right]_{-1}^1-\int_{-1}^{1}(\ell_n)^{(n-1)}(x) P'(x)\;dx$.
\end{center}

Maintenant, $-1$ et $1$ sont racines d'ordre $n$ du polynôme $\ell_n$ et donc, pour tout $k\in\llbracket0,n\rrbracket$, $-1$ et $1$ sont racines d'ordre $n-k$ de $\ell_n^{(k)}$ et en particulier racines de $(\ell_n)^{(k}$ pour $k\in\llbracket0,n-1\rrbracket$. Donc

\begin{center}
$\left(L_n|P\right)=-\int_{-1}^{1}(\ell_n)^{(n-1)}(x) P'(x)\;dx$.
\end{center}

Plus généralement, si pour un entier $k\in\llbracket0,n-1\rrbracket$, $\left(L_n|P\right)= (-1)^{k}\int_{-1}^{1}(\ell_n)^{(n-k)}(x) P^{(k)}(x)\;dx$ alors

\begin{align*}\ensuremath
\left(L_n|P\right)&= (-1)^{k}\left(\left[(\ell_n^{n-k-1}(x)P^{(k)}(x)\right]_{-1}^{1}-\int_{-1}^{1}(\ell_n)^{(n-k-1)}(x)P^{(k+1)}(x)\;dx\right)\\
 &=(-1)^{k+1}\int_{-1}^{1}(\ell_n)^{(n-k-1)}(x) P^{(k+1)}(x)\;dx.
\end{align*}

On a montré par récurrence que pour tout entier $k\in\llbracket0,n\rrbracket$, 
$\left(L_n|P\right)=(-1)^{k}\int_{-1}^{1}(\ell_n)^{(n-k)}(x) P^{(k)}(x)\;dx$. En particulier 

\begin{center}
$\left(L_n|P\right)=(-1)^{n}\int_{-1}^{1}\ell_n(x)P^{(n)}(x)\;dx=\int_{-1}^{1}(1-x^2)^nP^{(n)}(x)\;dx$/quad$(*)$.
\end{center}

Cette dernière égalité reste vraie pour $n=0$ et on a montré que

\begin{center}
\shadowbox{
$\forall n\in\Nn$, $\forall P\in\Rr[X]$, $\left(L_n|P\right)=\int_{-1}^{1}\left(1-x^2\right)^nP^{(n)}(x)\;dx$.
}
\end{center}

Soient alors $n$ et $p$ deux entiers naturels tels que $0\leqslant p<n$. Puisque $\text{deg}(L_p)=p<n$, on a $\left(L_n|L_p\right)=0$. On a montré que

\begin{center}
\shadowbox{
La famille $(L_k)_{0\leqslant k\leqslant n}$ est une base orthogonale de l'espace $(\Rr[X],\;|\;)$.
}
\end{center}}
    \item \question{Déterminer $\|L_n\|$ pour $n\in\Nn$.}
\reponse{On applique maintenant la formule $(*)$ dans le cas particulier $P =L_n$. On obtient 

\begin{align*}\ensuremath
\|L_n\|^2&=\int_{-1}^{1}\left(1-x^2\right)^nL_n^{(n)}(x)\;dx=2\times(2n)!\int_{0}^{1}(1-x^2)^n\;dx=2\times(2n)!\int_{\pi/2}^{0}(1-\cos^2t)^n(-\sin t)\;dt\\
 &=2\times(2n)!\int_{0}^{\pi/2}\sin^{2n+1}t\;dt=2\times(2n)!W_{2n+1}\;(\text{intégrales de \textsc{Wallis}}).
\end{align*}

On \og sait \fg que  $\forall n\in\Nn$, $W_{2n+1}=\frac{2n}{2n+1}W_{2n-1}=\frac{(2n)\times(2n-2)\times\ldots\times2}{(2n+1)\times(2n-1)\times\ldots\times3}W_1 =\frac{2^{2n}n!^2}{(2n+1)!}$. On obtient alors 

\begin{center}
$\|L_n\|^2=\frac{2^{2n}n!^2}{(2n+1)!}\times2\times(2n)!=\frac{2^{2n+1}n!^2}{2n+1}$,
\end{center}

\begin{center}
\shadowbox{
$\forall n\in\Nn$, $\|L_n\|=\sqrt{\frac{2}{2n+1}}\frac{2^nn!}{2n+1}$.
}
\end{center}

On en déduit que la famille $\left(\sqrt{\frac{2n+1}{2}}\frac{1}{2^nn!}L_n\right)_{n\in\Nn}$ est une base orthonormale de $(\Rr[X],\;|\;)$. Pour $n\in\Nn$, on pose $P_n=\sqrt{\frac{2n+1}{2}}\frac{1}{2^nn!}\left((X^2-1)^n\right)^{(n)}$.}
\end{enumerate}
}

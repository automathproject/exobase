\uuid{DSQH}
\exo7id{3802}
\titre{exo7 3802}
\auteur{quercia}
\organisation{exo7}
\datecreate{2010-03-11}
\isIndication{false}
\isCorrection{true}
\chapitre{Espace euclidien, espace normé}
\sousChapitre{Endomorphismes auto-adjoints}

\contenu{
\texte{

}
\begin{enumerate}
    \item \question{Que peut-on dire de l'adjoint d'un projecteur orthogonal d'un espace euclidien~? Réciproque~?}
\reponse{$p$ est un projecteur orthogonal $\Leftrightarrow$ $p$ est un projecteur et $p=p^*$ $\Leftrightarrow$ $p^*$ est un projecteur orthogonal.}
    \item \question{Soit $p$ un projecteur d'un espace euclidien tel que $p\circ p^*=p^*\circ p$. Montrer que $p$ est 
    un projecteur orthogonal.}
\reponse{$p$ et $p^*$ commutent donc $\mathrm{Ker} p$ et $\Im p$ sont stables par
    $p$ et par $p^*$, d'où $p^*_{|\mathrm{Ker} p} = (p_{|\mathrm{Ker} p})^* = 0_{\mathrm{Ker} p}$
    et $p^*_{|\Im p} = (p_{|\Im p})^* = \mathrm{id}_{\Im p}$. Ainsi $p=p^*$ ce
    qui implique $\mathrm{Ker} p \bot \Im p$.}
\end{enumerate}
}

\uuid{PfiW}
\exo7id{3154}
\titre{exo7 3154}
\auteur{quercia}
\organisation{exo7}
\datecreate{2010-03-08}
\isIndication{false}
\isCorrection{false}
\chapitre{Arithmétique}
\sousChapitre{Anneau Z/nZ, théorème chinois}
\module{Algèbre}
\niveau{L2}
\difficulte{}

\contenu{
\texte{
Soient $n,p \in \N^*$ tels que $n \wedge p = 1$.
Pour $x\in\Z$ on note $\overline{x}^{\,n}$, $\overline{x}^{\,p}$ et $\overline{x}^{\,np}$
les classes d'{\'e}quivalence de $x$ modulo $n$, $p$ et $np$.
}
\begin{enumerate}
    \item \question{Montrer que l'application $\phi : 
      {\Z/\bigl(np\Z\bigr)} \to 
      {\bigl(\Z/n\Z\bigr) \times \bigl(\Z/p\Z\bigr)},
      {\overline x^{\,np}} \mapsto {(\overline x^{\,n}, \overline x^{\,p})}$
      est un morphisme d'anneaux.}
    \item \question{En d{\'e}duire que $\varphi(np) = \varphi(n)\varphi(p)$ ($\varphi$ = fonction d'Euler).}
    \item \question{V{\'e}rifier que l'hypoth{\`e}se $n \wedge p = 1$ est n{\'e}c{\'e}ssaire.}
\end{enumerate}
}

\uuid{qeM8}
\exo7id{5376}
\auteur{rouget}
\organisation{exo7}
\datecreate{2010-07-06}
\isIndication{false}
\isCorrection{true}
\chapitre{Déterminant, système linéaire}
\sousChapitre{Calcul de déterminants}

\contenu{
\texte{
Donner une base du sous-espace vectoriel de $\Rr^5$ défini par~:

$$\left\{
\begin{array}{l}
x_1+2x_2-x_3+3x_4+x_5=0\\
x_2+x_3-2x_4+2x_5=0\\
2x_1+x_2-5x_3-4x_5=0
\end{array}
\right..$$
}
\reponse{
$\left|
\begin{array}{ccc}
1&2&3\\
0&1&-2\\
2&1&0
\end{array}
\right|=2+2(-7)=-12\neq0$ et le sytème est de \textsc{Cramer} en $x_1$, $x_2$ et $x_4$. On note aussi que le système est homogène de rang $3$ et donc que l'ensemble des solutions $F$ est un sous-espace vectoriel de $\Rr^5$ de dimension $5-3=2$.

\begin{align*}\ensuremath
\left\{
\begin{array}{l}
x_1+2x_2-x_3+3x_4+x_5=0\\
x_2+x_3-2x_4+2x_5=0\\
2x_1+x_2-5x_3-4x_5=0
\end{array}
\right.&\Leftrightarrow\left\{
\begin{array}{l}
x_1+2x_2+3x_4=x_3-x_5\\
x_2-2x_4=-x_3-2x_5\\
2x_1+x_2=5x_3+4x_5
\end{array}
\right.\Leftrightarrow\left\{
\begin{array}{l}
x_2=-2x_1+5x_3+4x_5\\
x_4=\frac{1}{2}((-2x_1+5x_3+4x_5)+x_3+2x_5)\\
x_1+2x_2+3x_4=x_3-x_5
\end{array}
\right.
\\
 &\Leftrightarrow\left\{
\begin{array}{l}
x_2=-2x_1+5x_3+4x_5\\
x_4=-x_1+3x_3+3x_5\\
x_1+2(-2x_1+5x_3+4x_5)+3(-x_1+3x_3+3x_5)=x_3-x_5
\end{array}
\right.
\\
 &\Leftrightarrow
 \left\{
\begin{array}{l}
x_1=3x_3+3x_5
x_2=-x_3-2x_5\\
x_4=0
\end{array}
\right.
\end{align*}

L'ensemble des solutions est $F=\{(3x_3+3x_5,-x_3-2x_5,x_3,0,x_5),\;(x_3,x_5)\in\Rr^2\}=\mbox{Vect}(e_1,e_2)$ où $e_1=(3,-1,1,0,0)$ et $e_2=(3,-2,0,0,1)$ et, puisque $\mbox{dim}F=2$, une base de $F$ est $(e_1,e_2)$.
}
}

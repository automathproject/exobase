\uuid{hBMJ}
\exo7id{1635}
\titre{exo7 1635}
\auteur{barraud}
\organisation{exo7}
\datecreate{2003-09-01}
\isIndication{false}
\isCorrection{true}
\chapitre{Réduction d'endomorphisme, polynôme annulateur}
\sousChapitre{Diagonalisation}

\contenu{
\texte{
La matrice suivante est-elle diagonalisable, triangularisable~? Effectuer
explicitement la réduction.
  $$
  A=
  \begin{pmatrix}
     3& 2& 4\\ 
    -1& 3&-1\\
    -2&-1&-3
  \end{pmatrix}
  $$
}
\reponse{
$\chi_{A}=(-1-X)(2-X)^{2}$. Donc $A$ est diagonalisable ssi $\dim
  \ker(A-2I)=2$. Or $\mathrm{rg}(A-2I)=2$, donc $\dim \ker(A-2I)=1$ donc $A$
  n'est pas diagonalisable. Cependant, $\chi_{A}$ est scindé sur $\R$
  donc $A$ est triangularisable sur $\R$.

  $\Big(\begin{smallmatrix}x\\y\\z\end{smallmatrix}\Big)\in\ker(A+I)%
  \Leftrightarrow\Big\{\begin{smallmatrix}%
    4x+2y+4z=0\\-x+4y-z=0\\-2x-y-2z=0%
  \end{smallmatrix}
  \Leftrightarrow\big\{\begin{smallmatrix}%
    x+z=0\\y=0%
  \end{smallmatrix}$ donc $\ker(A+I)=\R\Big(\begin{smallmatrix}%
    1\\0\\-1%
  \end{smallmatrix}\Big)$

De m\^eme,
 
  $\Big(\begin{smallmatrix}x\\y\\z\end{smallmatrix}\Big)\in\ker(A-2I)%
  \Leftrightarrow\Big\{\begin{smallmatrix}%
    x+2y+4z=0\\-x+y-z=0\\-2x-y-5z=0%
  \end{smallmatrix}
  \Leftrightarrow\big\{\begin{smallmatrix}%
    x=2y\\z=-y%
  \end{smallmatrix}$ donc $\ker(A+I)=\R\Big(\begin{smallmatrix}%
    2\\1\\-1%
  \end{smallmatrix}\Big)$

On sait que $\ker((A-2I)^{2})$ est de dimension 2, et que $\Big(\begin{smallmatrix}%
    2\\1\\-1%
  \end{smallmatrix}\Big)\in\ker(A-2I)\subset\ker((A-2I)^{2})$.
 On cherche donc un deuxième vecteur dans $\ker((A-2I)^{2})$, linéairement
 indépendant de $\Big(\begin{smallmatrix}%
    2\\1\\-1%
  \end{smallmatrix}\Big)$.

$(A-2I)^{2}=\Big(\begin{smallmatrix}%
    -9&0&-18\\0&0&0\\9&0&18%
  \end{smallmatrix}\Big)$
donc $\Big(\begin{smallmatrix}%
    0\\1\\0%
  \end{smallmatrix}\Big)$
convient. De plus~: $A\Big(\begin{smallmatrix}%
    0\\1\\0%
  \end{smallmatrix}\Big)=\Big(\begin{smallmatrix}%
    2\\3\\-1%
  \end{smallmatrix}\Big)=\Big(\begin{smallmatrix}%
    2\\1\\-1%
  \end{smallmatrix}\Big)+2\Big(\begin{smallmatrix}%
    0\\1\\0%
  \end{smallmatrix}\Big)$.

Donc en posant $P=\Big(\begin{smallmatrix}%
    1&2&0\\0&1&1\\-1&-1&0%
  \end{smallmatrix}\Big)$, on obtient
$P^{-1}AP=\Big(\begin{smallmatrix}%
    -1&0&0\\0&2&1\\0&0&2%
  \end{smallmatrix}\Big)$.
}
}

\uuid{ZGJT}
\exo7id{2569}
\auteur{delaunay}
\organisation{exo7}
\datecreate{2009-05-19}
\isIndication{false}
\isCorrection{true}
\chapitre{Réduction d'endomorphisme, polynôme annulateur}
\sousChapitre{Polynôme annulateur}

\contenu{
\texte{
Soit $A$ la matrice suivante
$$A=\begin{pmatrix}0&1&1 \\  1&0&1 \\  1&1&0\end{pmatrix}$$
Calculer $A^2$ et v\'erifier que $A^2=A+2I_3$. En d\'eduire que $A$ est inversible et donner son inverse en fonction de $A$.
}
\reponse{
Soit $A$ la matrice suivante
$$A=\begin{pmatrix}0&1&1 \\  1&0&1 \\  1&1&0\end{pmatrix}$$
{\it Calculons $A^2$ et v\'erifions que $A^2=A+2I_3$}. 
On a 
$$A^2=\begin{pmatrix}0&1&1 \\  1&0&1 \\  1&1&0\end{pmatrix}\begin{pmatrix}0&1&1 \\  1&0&1 \\  1&1&0\end{pmatrix}=\begin{pmatrix}2&1&1 \\  1&2&1 \\  1&1&2\end{pmatrix}=A+2I_3.$$
On a donc $A^2-A=2I_3$, c'est-\`a-dire $A(A-I_3)=2I_3$, ou encore $A.{\frac{1}{2}}(A-I_3)=I_3$. Ce qui prouve que $A$ est in versible et que son inverse 
est $A^{-1}={\frac{1}{2}}(A-I_3)$
}
}

\uuid{V7Fu}
\exo7id{5775}
\auteur{rouget}
\organisation{exo7}
\datecreate{2010-10-16}
\isIndication{false}
\isCorrection{true}
\chapitre{Espace euclidien, espace normé}
\sousChapitre{Produit scalaire, norme}

\contenu{
\texte{
On note $E$ l'ensemble des suites réelles de carrés sommables c'est-dire les suites réelles $(u_n)_{n\in\Nn}$ telles que

\begin{center}
$\sum_{n=0}^{+\infty}u_n^2<+\infty$.
\end{center}
}
\begin{enumerate}
    \item \question{Montrer que $E$ est un $\Rr$-espace vectoriel.}
\reponse{Montrons que $E$ est un sous-espace de $(\Rr^\Nn,+,.)$. La suite nulle est élément de $E$. Soient $(u,v)\in E^2$ et $(\lambda,\mu)\in\Rr^2$.

\begin{center}
$0\leqslant(\lambda u+\mu v)^2=\lambda^2u^2+2\lambda\mu uv+\mu^2v^2\leqslant\lambda^2u^2+\lambda\mu(u^2+v^2)+\mu^2v^2=(\lambda^2+\lambda\mu)u^2+(\lambda\mu+\mu^2)v^2$.
\end{center}

Par hypothèse, la série de terme général $(\lambda^2+\lambda\mu)u_n^2+(\lambda\mu+\mu^2)v_n^2$ converge et on en déduit que la suite $\lambda u+\mu v$ est de carré sommable. On a montré que

\begin{center}
\shadowbox{
$E$ est un sous-espace vectoriel de l'espace vectoriel $(\Rr^\Nn,+,.)$.
}
\end{center}}
    \item \question{Pour $(u,v)\in E^2$, on pose $\varphi(u,v)=\sum_{n=0}^{+\infty}u_nv_n$. Montrer que $\varphi$ est un produit scalaire sur $E$.}
\reponse{\textbullet~Soient $u$ et $v$ deux éléments de $E$. Pour tout entier naturel $n$,

\begin{center}
$|u_nv_n\leqslant\frac{1}{2}(u_n^2+v_n^2)$.
\end{center}

Ainsi, la série de terme général $u_nv_n$ est absolument convergente et donc convergente. Ceci montre que $\varphi(u,v)$ existe dans $\Rr$.

\textbullet~La symétrie, la bilinéarité et la positivité de $\varphi$ sont claires. De plus, pour $u\in E$,

\begin{center}
$\varphi(u,u)=0\Rightarrow\sum_{n=0}^{+\infty}u_n^2=0\Rightarrow\forall n\in\Nn,\;u_n^2=0\Rightarrow u=0$.
\end{center}

En résumé, l'application $\varphi$ est une forme bilinéaire, symétrique, définie, positive et donc

\begin{center}
\shadowbox{
l'application $\varphi$ est un produit scalaire sur $E$.
}
\end{center}}
\end{enumerate}
}

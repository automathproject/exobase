\uuid{PWRa}
\exo7id{5359}
\titre{exo7 5359}
\auteur{rouget}
\organisation{exo7}
\datecreate{2010-07-04}
\isIndication{false}
\isCorrection{true}
\chapitre{Groupe, anneau, corps}
\sousChapitre{Groupe de permutation}
\module{Algèbre}
\niveau{L2}
\difficulte{}

\contenu{
\texte{
$\sigma$ étant une permutation de $\{1,...,n\}$ donnée, on définit la matrice notée $P_\sigma$, carrée d'ordre $n$ dont le terme ligne $i$ colonne $j$ est $\delta_{i,\sigma(j)}$ (où $\delta_{i,j}$ est le symbôle de \textsc{Kronecker}. On note $G$ l'ensemble des $P_\sigma$ où $\sigma$ décrit $S_n$.
}
\begin{enumerate}
    \item \question{\begin{enumerate}}
\reponse{\begin{enumerate}}
    \item \question{$\sigma$ et $\sigma'$ étant deux éléments de $S_n$, calculer $P_\sigma\times P_{\sigma'}$.}
\reponse{Soient $\sigma$ et $\sigma'$ deux éléments de $S_n$. Soit $(i,j)\in\{1,...,n\}^2$. Le coefficient ligne $i$, colonne $j$ de $P_\sigma P_{\sigma'}$ vaut

$$\sum_{k=1}^{n}\delta_{i,\sigma(k)}\delta_{k,\sigma'(j)}=\delta_{i,\sigma(\sigma'(j))},$$

et est donc aussi le coefficient ligne $i$, colonne $j$ de la matrice $P_{\sigma\circ\sigma'}$. Par suite,

$$\forall(\sigma,\sigma')\in(S_n)^2,\;P_\sigma\times P_{\sigma'}=P_{\sigma\circ\sigma'}.$$}
    \item \question{En déduire que $(G,\times)$ est un sous-groupe de $(GL_n(\Rr),\times)$, isomorphe à $(S_n,\circ)$ (les matrices $P_\sigma$ sont appelées \og~matrices de permutation~\fg).}
\reponse{Soit $\sigma\in S_n$. D'après a), $P_\sigma P_{\sigma^{-1}}=P_{\sigma\circ\sigma^{-1}}=P_{Id}=I_n=P_{\sigma^{-1}}P_\sigma$. On en déduit que toute matrice $P_\sigma$ est inversible, d'inverse $P_{\sigma^{-1}}$. Par suite, $G\subset GL_n(\Rr)$ (et clairement, $G\neq\emptyset$).

Soit alors $(\sigma,\sigma')\in(S_n)^2$.

$$P_\sigma P_{\sigma'}^{-1}=P_\sigma P_{{\sigma'}^{-1}}=P_{\sigma\circ{\sigma'}^{-1}}\in G.$$

On a montré que $G$ est un sous-groupe de $(GL_n(\Rr),\times)$.

Soit $\begin{array}[t]{cccc}
\varphi~:&S_n&\rightarrow&G\\
 &\sigma&\mapsto&P_\sigma
\end{array}$. D'après a), $\varphi$ est un morphisme de groupes. $\varphi$ est clairement surjectif. Il reste à vérifier que $\varphi$ est injectif.

Soit $\sigma\in S_n$.

\begin{align*}\ensuremath
\sigma\in\mbox{Ker}\varphi\Rightarrow P_\sigma=I_n&\Rightarrow\forall(i,j)\in\{1,...,n\}^2,\;\delta_{i,\sigma(j)}=\delta_{i,j}\\
 &\Rightarrow \forall i\in\{1,...,n\},\;\delta_{i,\sigma(i)}=1\Rightarrow\forall i\in\{1,...,n\},\;\sigma(i)=i\\
 &\Rightarrow\sigma=Id.
\end{align*}

Puisque le noyau du morphisme $\varphi$ est réduit à $\{Id\}$, $\varphi$ est injectif.

Ainsi, $\varphi$ est un isomorphisme du groupe $(S_n,\circ)$ sur le groupe $(G,\times)$ et on a montré que $(G,\times)$ est un sous-groupe de $(GL_n(\Rr),\times)$, isomorphe à $(S_n,\circ)$.}
\end{enumerate}
}

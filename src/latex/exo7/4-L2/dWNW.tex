\uuid{dWNW}
\exo7id{1628}
\titre{exo7 1628}
\auteur{barraud}
\organisation{exo7}
\datecreate{2003-09-01}
\isIndication{false}
\isCorrection{false}
\chapitre{Réduction d'endomorphisme, polynôme annulateur}
\sousChapitre{Diagonalisation}
\module{Algèbre}
\niveau{L2}
\difficulte{}

\contenu{
\texte{
Soient $u$ et $v$ deux endomorphismes diagonalisables de $E$, qui commutent (c'est \`{a} dire
tels que $u\circ v=v\circ u$). On note $\lambda_{1},\ldots,\lambda_{p}$ (resp.
$\mu_{1},\ldots,\mu_{q}$) les valeurs propres de $u$ (resp. de $v$), et
$F_{1},\ldots,F_{p}$ les espaces propres associ\'{e}s (resp. $G_{1},\ldots,G_{q}$).
}
\begin{enumerate}
    \item \question{Montrer que chaque $G_{j}$ (resp. $F_{i}$) est stable par $u$ (resp. $v$) (c'est \`{a} dire
que $u(G_{j})\subset G_{j}$).}
    \item \question{On pose $H_{ij}=F_{i}\cap G_{j}$. Soit $i\in\{1,\ldots,p\}$. Montrer que $F_{i}$ est la
somme directe des espaces $(H_{ij})_{1\leq j\leq q}$.}
    \item \question{En d\'{e}duire l'\'{e}nonc\'{e} suivant : {\sl Lorsque deux endomorphismes diagonalisables $u$ et $v$
commutent, il existe une base form\'{e}e de vecteurs propres communs \`{a} $u$ et \`{a} $v$ (en
d'autres termes, $u$ et $v$ sont diagonalisables simultan\'{e}ment dans la m\^{e}me base).}}
\end{enumerate}
}

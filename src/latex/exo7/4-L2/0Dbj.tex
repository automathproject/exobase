\uuid{0Dbj}
\exo7id{3078}
\titre{exo7 3078}
\auteur{quercia}
\organisation{exo7}
\datecreate{2010-03-08}
\isIndication{false}
\isCorrection{true}
\chapitre{Groupe, anneau, corps}
\sousChapitre{Groupe de permutation}
\module{Algèbre}
\niveau{L2}
\difficulte{}

\contenu{
\texte{
Soit $\sigma \in {\cal S}_n$. On note $c$ le nombre de cycles {\`a} supports disjoints
constituant $\sigma$, et $f$ le nombre de points fixes.

Calculer $\epsilon(\sigma)$ en fonction de $n$, $c$, et $f$.
}
\reponse{
$\epsilon(\sigma) = (-1)^{n+c+f}$.
}
}

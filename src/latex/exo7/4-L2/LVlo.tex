\uuid{LVlo}
\exo7id{3502}
\titre{exo7 3502}
\auteur{quercia}
\organisation{exo7}
\datecreate{2010-03-10}
\isIndication{false}
\isCorrection{true}
\chapitre{Réduction d'endomorphisme, polynôme annulateur}
\sousChapitre{Valeur propre, vecteur propre}

\contenu{
\texte{
Déterminer les valeurs propres de la matrice
$A = \begin{pmatrix} 1  &-1     &       &        &(0)     \cr
               -1 &2      &-1     &        &        \cr
                  &\ddots &\ddots &\ddots  &        \cr
                  &       &-1     &2       &-1      \cr
               (0)&       &       &-1      &1       \cr\end{pmatrix} \in \mathcal{M}_n(\R)$.
}
\reponse{
Soit $P_n(x)$ le polynôme caractéristique de~$x$ et $Q_n(x)$ celui de la
matrice obtenue à partir de~$A$ en rempla\c cant le premier~$1$ par $2$.
On a les relations de récurrence~:
$$P_n(x) = (1-x)Q_{n-1}(x) - Q_{n-2}(x),
  \qquad
  Q_n(x) = (2-x)Q_{n-1}(x) - Q_{n-2}(x).$$
D'où pour $x\notin\{0,4\}$~:
$$P_n(x) = \frac{(1-\alpha)(1-\alpha^{2n})}{\alpha^n(1+\alpha)},  \qquad\text{avec }
x = 2 - \alpha - \frac1\alpha.$$
Les valeurs propres de~$A$ autres que $0$ et $4$ sont les réels $x_k=2(1-\cos(k\pi/n))$
avec $0< k< n$ et $0$ est aussi valeur propre (somme des colonnes nulle) donc
il n'y en a pas d'autres.
}
}

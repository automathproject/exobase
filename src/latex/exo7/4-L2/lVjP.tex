\uuid{lVjP}
\exo7id{5500}
\auteur{rouget}
\organisation{exo7}
\datecreate{2010-07-10}
\isIndication{false}
\isCorrection{true}
\chapitre{Espace euclidien, espace normé}
\sousChapitre{Orthonormalisation}

\contenu{
\texte{
Sur $E=\Rr_n[X]$, on pose $P|Q=\int_{-1}^{1}P(t)Q(t)\;dt$.
}
\begin{enumerate}
    \item \question{Montrer que $(E,|)$ est un espace euclidien.}
    \item \question{Pour $p$ entier naturel compris entre $0$ et $n$, on pose $L_p=((X^2-1)^p)^{(p)}$.
Montrer que $\left(\frac{L_p}{||L_p||}\right)_{0\leq p\leq n}$ est l'orthonormalisée de \textsc{Schmidt} de la base canonique de $E$.

Déterminer $||Lp||$.}
\reponse{
La symétrie, la bilinéarité et la positivité sont claires. Soit alors $P\in\Rr_n[X]$. 

\begin{align*}\ensuremath
P|P=0&\Rightarrow\int_{0}^{1}P^2(t)\;dt=0\\
 &\Rightarrow\forall t\in[0,1],\;P^2(t)=0\;(\mbox{fonction continue, positive, d'intégrale nulle})\\
 &\Rightarrow P=0\;(\mbox{polynôme ayant une infinité de racines}).
\end{align*}
Ainsi, l'application $(P,Q)\mapsto\int_{0}^{1}P(t)Q(t)\;dt$ est un produit scalaire sur $\Rr_n[X]$.
Pour vérifier que la famille $\left(\frac{L_p}{||L_p||}\right)_{0\leq p\leq n}$ est l'orthonormalisée de \textsc{Schmidt} de la base canonique de $E$, nous allons vérifier que
  \begin{enumerate}
$\forall p\in\llbracket0,n\rrbracket,\;\mbox{Vect}(L_0,L_1,...,L_p)=\mbox{Vect}(1,X,...,X^p)$,\rule[-4mm]{0mm}{0mm}
la famille $\left(\frac{L_p}{||L_p||}\right)_{0\leq p\leq n}$ est orthonormale,
$\forall p\in\llbracket0,n\rrbracket,\;L_p|X^p>0$.\rule{0mm}{5mm}
}
\end{enumerate}
}

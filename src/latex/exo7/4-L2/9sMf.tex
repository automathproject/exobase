\uuid{9sMf}
\exo7id{3077}
\titre{exo7 3077}
\auteur{quercia}
\organisation{exo7}
\datecreate{2010-03-08}
\isIndication{false}
\isCorrection{true}
\chapitre{Groupe, anneau, corps}
\sousChapitre{Groupe de permutation}
\module{Algèbre}
\niveau{L2}
\difficulte{}

\contenu{
\texte{
Soit $n \in \N,n \ge 4$.
}
\begin{enumerate}
    \item \question{Soit $i,j \in \{3,\dots,n \}\ , i \ne j $.\par
    D{\'e}composer en cycles {\`a} supports disjoints la permutation :
    $\sigma=(1\ i\ 2)\circ(1\ 2\ j)\circ(1\ i\ 2)$.}
    \item \question{On note $\cal H$ le sous-groupe de ${\cal A}_n$ engendr{\'e} par les 3-cycles
    $(1\ 2\ k)$,\quad $3 \le k \le n$.
  \begin{enumerate}}
    \item \question{Montrer que : $\forall\ i,j \ge 3$, avec $i \ne j$, $\cal H$ contient $(1\ 2)\circ(i\ j)$ et $(i\ j)\circ(1\ 2)$.}
    \item \question{Montrer que : $\forall\ j  \ge 3$, $\cal H$ contient $(1\ 2)\circ(1\ j)$ et $(1\ 2)\circ(2\ j)$.}
    \item \question{Montrer que : $\forall\ i \ne j$, $\forall\ k \ne l$,\quad $(i\ j)\circ(k\ l) \in \cal H$.}
    \item \question{Montrer que ${\cal H}={\cal A}_n$.}
\reponse{
$(1\ 2)\circ(i\ j)$.
}
\end{enumerate}
}

\uuid{Wysw}
\exo7id{7335}
\auteur{mourougane}
\organisation{exo7}
\datecreate{2021-08-10}
\isIndication{false}
\isCorrection{false}
\chapitre{Groupe, anneau, corps}
\sousChapitre{Anneau}

\contenu{
\texte{

}
\begin{enumerate}
    \item \question{Quelle identité obtient-on quand on écrit que la norme de $(a+ib)(c+id)$ dans $\Z[i]$ est le produit de la norme de $(a+ib)$ et de celle de $(c+id)$ ?}
    \item \question{\'Ecrire $2425=5^2\cdot 97$ et $754=2\cdot 13 \cdot 29$ comme sommes de deux carrés.}
    \item \question{Tous les entiers naturels sont-ils sommes de trois carrés ?
%1. Montrer que tout entier positif n de la forme 8k + 7 ne peut pas s'écrire comme somme
%de 3 carrés.
%Indication : on pourra raisonner modulo 8...
%2. Montrer que si n est un entier positif tel que 4n est somme de 3 carrés, alors n l'est aussi.
%Indication : après avoir écrit 4n comme somme de 3 carrés, on pourra étudier la parité
%de chacun des termes de la somme et raisonner modulo 4.
%3. A l'aide des deux questions précédentes, deviner une condition qui emp^eche un entier
%positif de s'écrire comme somme de 3 carrés.}
\end{enumerate}
}

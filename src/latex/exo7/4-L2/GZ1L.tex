\uuid{GZ1L}
\exo7id{2727}
\titre{exo7 2727}
\auteur{tumpach}
\organisation{exo7}
\datecreate{2009-10-25}
\isIndication{false}
\isCorrection{false}
\chapitre{Groupe, anneau, corps}
\sousChapitre{Groupe, sous-groupe}
\module{Algèbre}
\niveau{L2}
\difficulte{}

\contenu{
\texte{
On munit l'ensemble $G = \{a, b, c, d\}$ d'une loi de composition interne dont la table de Pythagore est
$$
\begin{tabular}{r|rrrr}
$\star$ & \textbf{a} & \textbf{b}&\textbf{c}&\textbf{d}\\
\hline
\textbf{a} & c & a & c & a\\
\textbf{b}& a & d& c& b\\
\textbf{c}& c & c& c & c\\
\textbf{d} & a & b & c & d 
\end{tabular}
$$
(La premi\`ere ligne se lit $a \star a = a$, $a \star b = a$, $a \star c = c$,\ldots.)
}
\begin{enumerate}
    \item \question{Cette loi poss\`ede-t-elle un \'el\'ement neutre~?}
    \item \question{Cette loi est-elle commutative~?}
    \item \question{Cette loi est-elle associative~?}
    \item \question{Est-ce une loi de groupe~?}
\end{enumerate}
}

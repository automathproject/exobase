\uuid{PNI6}
\exo7id{2597}
\titre{exo7 2597}
\auteur{delaunay}
\organisation{exo7}
\datecreate{2009-05-19}
\isIndication{false}
\isCorrection{true}
\chapitre{Réduction d'endomorphisme, polynôme annulateur}
\sousChapitre{Applications}

\contenu{
\texte{
Soit $A$ la matrice 
$$A=\begin{pmatrix}1&-1&0 \\  1&0&-1 \\  -1&0&2\end{pmatrix}$$ 
et $f$ l'endomorphisme de $\R^3$ associ\'e.
}
\begin{enumerate}
    \item \question{Factoriser le polyn\^ome caract\'eristique de $A$.}
\reponse{{\it Factorisons le polyn\^ome caract\'eristique de $A$}.

Le polyn\^ome caract\'eristique de $A$ est le polyn\^ome 
$$\begin{align*}P_A(X)=\begin{vmatrix}2-X&0&-1 \\ -1&1-X&1 \\  0&-1&-X\end{vmatrix}
&=(2-X)\begin{vmatrix}1-X&1 \\ -1&-X\end{vmatrix}-\begin{vmatrix}-1&1-X \\  0&-1\end{vmatrix} \\ 
&=(2-X)(X^2-X+1)-1 \\ 
&=-X^3+3X^2-3X+1=(1-X)^3\end{align*}$$}
    \item \question{D\'eterminer les sous-espaces propres et caract\'eristiques de $A$.}
\reponse{{\it D\'eterminons les sous-espaces propres et caract\'eristiques de $A$}.

La matrice $A$ admet une unique valeur propre $\lambda=1$, comme $A\neq I$, elle n'est pas diagonalisable. Son sous-espace caract\'eristique est \'egal \`a $\ker(A-I_3)^3=\R^3$. En effet, d'apr\`es le th\'eor\`eme de Hamilton-Cayley, on a $P_A(A)=0$, c'est-\`a-dire $(A-I_3)^3=0$. Son sous-espace propre est \'egal \`a $\ker(A-I_3)$.
$$\ker(A-I_3)=\{\vec u\in\R^3/\ A\vec u=\vec u\}=\{(x,y,z)\in\R^3/\ x=z=-y\}.$$
C'est la droite vectorielle engendr\'ee par le vecteur $\vec u_1=(1,-1,1)$.}
    \item \question{D\'emontrer qu'il existe une base de $\R^3$ dans laquelle la matrice de $f$ est 
$$B=\begin{pmatrix}1 & 1 & 0 \\  0&1&1 \\ 0&0&1\end{pmatrix}$$
et trouver une matrice $P$ inversible telle que $A=PBP^{-1}$.}
\reponse{{\it D\'emontrons qu'il existe une base de $\R^3$ dans laquelle la matrice de $f$ est 
$$B=\begin{pmatrix}1 & 1 & 0 \\  0&1&1 \\ 0&0&1\end{pmatrix}$$
et trouver une matrice $P$ inversible telle que $A=PBP^{-1}$}.

Le vecteur $\vec u_1$ v\'erifie $A\vec u_1=u_1$, on cherche un vecteur $\vec u_2=(x,y,z)$ tel que 
$A\vec u_2=\vec u_1+\vec u_2$.
$$\begin{align*}A\vec u_2=\vec u_1+\vec u_2&\iff\begin{pmatrix}2&0&-1 \\  -1&1&1 \\  0&-1&0\end{pmatrix}
\begin{pmatrix}x \\  y \\  z\end{pmatrix}=\begin{pmatrix}1+x \\  -1+y \\  1+z\end{pmatrix} \\ 
&\iff\begin{pmatrix}2x-z \\  -x+y+z \\  -y\end{pmatrix}=\begin{pmatrix}1+x \\  -1+y \\  1+z\end{pmatrix}\end{align*}$$
On obtient donc $z-x=z+y=-1$, le vecteur $\vec u_2=(1,-1,0)$ convient. On cherche alors un vecteur $\vec u_3=(x,y,z)$ tel que $A\vec u_3=\vec u_2+\vec u_3$.
$$\begin{align*}A\vec u_3=\vec u_2+\vec u_3\iff
\begin{pmatrix}2x-z \\  -x+y+z \\  -y\end{pmatrix}=\begin{pmatrix}1+x \\ -1+y \\  z\end{pmatrix}\end{align*}$$
on obtient alors $x+y=x-z=1$. Le vecteur $\vec u_3=(1,0,0)$ convient.
Ainsi, dans la base $(\vec u_1, \vec u_2, \vec u_3)$ la matrice de $f$ est \'egale \`a $B$. La matrice $P$ cherch\'ee est 
$$P=\begin{pmatrix}1& 1 & 1 \\ -1&-1&0 \\ 1&0&0\end{pmatrix},$$
on a bien $A=PBP^{-1}$ et $B=P^{-1}AP$.}
    \item \question{Ecrire la d\'ecomposition de Dunford de $B$ (justifier).}
\reponse{{\it Ecrivons la d\'ecomposition de Dunford de $B$}.

On a 
$$B=\begin{pmatrix}1 & 1 & 0 \\  0&1&1 \\ 0&0&1\end{pmatrix}=\begin{pmatrix}1 & 0 & 0 \\  0&1&0 \\ 0&0&1\end{pmatrix}+
\begin{pmatrix}0& 1 & 0 \\  0&0&1 \\ 0&0&0\end{pmatrix}=I_3+N$$
Si $N=\begin{pmatrix}0& 1 & 0 \\  0&0&1 \\ 0&0&0\end{pmatrix}$ alors $N^2=\begin{pmatrix}0& 0 & 1 \\  0&0&0 \\ 0&0&0\end{pmatrix}$ et $N^3=0$. La matrice $I_3$ est diagonale, la matrice $N$ est nilpotente, les matrices $I_3$ et $N$ commutent, c'est donc bien la d\'ecomposition de Dunford de $B$.}
    \item \question{Pour $t\in\R$, calculer $\exp tB$.}
\reponse{{\it Pour $t\in\R$, calculons} $\exp tB$. 

On a, pour $t\in\R$, $tB=tI_3+tN$, ainsi $\exp tB=\exp(tI_3+tN)=(\exp tI_3)(\exp tN)$ car les matrices $tI_3$ et $tN$ commutent. Par ailleurs, $\exp tI_3=e^tI_3$ et $\exp tN=I+tN+{\frac{t^2}{2}}N^2$. On a donc
$$\exp tB=e^t\begin{pmatrix}1&t&t^2/2 \\ 0&1&t \\ 0&0&1\end{pmatrix}.$$}
    \item \question{Donner les solutions des syst\`emes diff\'erentiels $Y'=BY$ et $X'=AX$.}
\reponse{{\it Donnons les solutions des syst\`emes diff\'erentiels $Y'=BY$ et $X'=AX$.}

Int\'egrons le syst\`eme $Y'=BY$, sa solution g\'en\'erale s'\'ecrit
$$Y(t)=(\exp tB) Y_0,$$
o\`u $Y_0$ est un vecteur de $\R^3$. 

Int\'egrons alors le syst\`eme $X'=AX$. Remarquons que si $PY$ est solution de $X'=AX$, on a 
$$(PY)'=A(PY)\iff PY'=APY\iff Y'=P^{-1}APY\iff Y'=BY$$
ainsi $Y$ est solution de $Y'=BY$, la solution g\'en\'erale du syst\`eme $X'=AX$ s\'ecrit donc
$$\begin{align*}X(t)=P(\exp tB)Y_0&=e^t\begin{pmatrix}1& 1 &1 \\  -1&-1&0 \\ 1&0&0\end{pmatrix}\begin{pmatrix}1&t&t^2/2 \\ 0&1&t \\ 0&0&1\end{pmatrix}Y_0 \\  &=e^t
\begin{pmatrix}1& t+1& t^2/2+t+1 \\  -1&-t-1&-t^2/2-t \\  1&t&t^2/2\end{pmatrix}\begin{pmatrix}a \\  b \\  c\end{pmatrix}\end{align*}$$
o\`u $Y_0=(a,b,c)$ est un vecteur de $\R^3$. Ou encore
$$\left\{\begin{align*}
x(t)&=e^t(a+b(t+1)+c(t^2/2+t+1)) \\  y(t)&=e^t(-a-b(t+1)-c(t^2/2+t)) \\  z(t)&=e^t(a+bt+ct^2/2)
\end{align*}\right.$$ ou encore

$$\begin{pmatrix}x(t) \\  y(t) \\  z(t)\end{pmatrix}=a\begin{pmatrix}e^t \\  -e^t \\  e^t\end{pmatrix}+
b\begin{pmatrix}e^t(t+1) \\  -e^t(t+1) \\  te^t\end{pmatrix}+c\begin{pmatrix}e^t(t^2/2+t+1) \\  -e^t(t^2/2+t) \\  e^t t^2/2\end{pmatrix}$$
avec $(a,b,c)\in\R^3$.}
\end{enumerate}
}

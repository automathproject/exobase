\uuid{i9sy}
\exo7id{5801}
\auteur{rouget}
\organisation{exo7}
\datecreate{2010-10-16}
\isIndication{false}
\isCorrection{true}
\chapitre{Espace euclidien, espace normé}
\sousChapitre{Espace vectoriel euclidien de dimension 3}

\contenu{
\texte{
Valeurs et vecteurs propres de l'endomorphisme de $\Rr^3$ euclidien orienté défini par

\begin{center}
$\forall x\in\Rr^3$, $f(x) = a\wedge(a\wedge x)$ où $a$ est un vecteur donné.
\end{center}
}
\reponse{
Si $a = 0$, $f = 0$ et il n'y a plus rien à dire.

Si $a\neq0$, puisque $f(a) = 0$, $0$ est valeur propre de $f$ et $\text{Vect}(a)\subset E_0(f)$.
D'autre part, si $x$ est orthogonal à $a$, d'après la formule du double produit vectoriel

\begin{center}
$f(x) = (a.x)a -\|a\|^2x = -\|a\|^2x$.
\end{center}

Donc le réel non nul $-\|a\|^2$ est valeur propre de $f$ et $a^\bot\subset E_{-\|a\|^2}$. Maintenant,  $\text{dim}\text{Vect}(a) +\text{dim}a^\bot= 3$ et donc $\text{Sp}(f)=(0,-\|a\|^2,-\|a\|^2)$ puis $E_0(f) = \text{Vect}(a)$ et  $E_{-\|a\|^2}= a^\bot$. On en déduit aussi que $f$ est diagonalisable. On peut noter que, puisque $f$ est diagonalisable et que les sous-espaces propres sont orthogonaux, $f$ est un endomorphisme symétrique.
}
}

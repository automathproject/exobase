\uuid{v8Jd}
\exo7id{3759}
\titre{exo7 3759}
\auteur{quercia}
\organisation{exo7}
\datecreate{2010-03-11}
\isIndication{false}
\isCorrection{true}
\chapitre{Espace euclidien, espace normé}
\sousChapitre{Projection, symétrie}

\contenu{
\texte{
Soit $f : {\R^n} \to {\R^n}, {(x_1,\dots,x_n)} \mapsto {(x_1-x_n,x_2-x_1,\dots,x_n-x_{n-1}).}$
Avec la structure euclidienne canonique de~$\R^n$, calculer la norme de~$f$.
}
\reponse{
$f = \mathrm{id}-r$ où $r(x_1,\dots,x_n) = (x_n,x_1,\dots,x_{n-1})$.
Donc $f^*\circ f = 2\mathrm{id}-r-r^{-1}$ a pour valeurs propres les nombres
$2-2\cos(2k\pi/n)$, $k\in{[[0,2n-1]]}$ et
$\|f\| = \begin{cases}2&\text{ si } n \text{ est pair}\cr2\cos(\pi/2n)&\text{ si } n \text{ est impair.}\cr \end{cases}$.
}
}

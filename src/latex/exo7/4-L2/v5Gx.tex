\uuid{v5Gx}
\exo7id{5637}
\titre{exo7 5637}
\auteur{rouget}
\organisation{exo7}
\datecreate{2010-10-16}
\isIndication{false}
\isCorrection{true}
\chapitre{Déterminant, système linéaire}
\sousChapitre{Calcul de déterminants}
\module{Algèbre}
\niveau{L2}
\difficulte{}

\contenu{
\texte{
Soient $x_0$,..., $x_{n-1}$ $n$ nombres complexes. Calculer $\text{Van}(x_0,...,x_{n-1})= \text{det}(x_{j-1}^{i-1})_{1\leqslant i,j\leqslant n}$.
}
\reponse{
Soit $n$ un entier naturel non nul. On note $L_0$, $L_1$,\ldots, $L_n$ les lignes  du déterminant $\text{Van}(x_0,\ldots,x_n)$

A la ligne numéro $n$ du déterminant $\text{Van}(x_0,\ldots,x_n)$,  on ajoute une combinaison linéaire des lignes précédentes du type $L_{n}\leftarrow L_{n}+\sum_{i=0}^{n-1}\lambda_iL_i$. La valeur du déterminant n'a pas changé mais sa dernière ligne s'écrit maintenant $(P(x_0),...,P(x_n))$ où $P$ est un polynôme unitaire de degré $n$. On choisit alors pour
$P$ (le choix des $\lambda_i$ équivaut au choix de $P$) le polynôme $P=\prod_{i=0}^{n-1}(X-x_i)$ (qui est bien unitaire de degré $n$).
La dernière ligne s'écrit alors $(0,...,0,P(x_{n+1}))$ et en développant ce déterminant suivant cette dernière ligne, on obtient la relation de récurrence :

\begin{center}
$\forall n\in\Nn^*,\;\text{Van}(x_0,\ldots,x_n)=P(x_n)\text{Van}(x_0,\ldots,x_{n-1})=\prod_{i=0}^{n-1}(x_n-x_i)\text{Van}(x_0,\ldots,x_{n-1})$.
\end{center}

En tenant compte de $\text{Van}(x_0)=1$, on obtient donc par récurrence

\begin{center}
\shadowbox{
$\forall n\in\Nn^*,\;\forall(x_i)_{0\leqslant i\leqslant n}\in\Kk^n,\;\text{Van}(x_i)_{0\leqslant i\leqslant n-1}=\prod_{0\leqslant i<j\leqslant n-1}^{}(x_j-x_i)$.
}
\end{center}
En particulier, $\text{Van}(x_i)_{0\leqslant i\leqslant n-1}\neq 0$ si et seulement si les $x_i$ sont deux à deux distincts.
}
}

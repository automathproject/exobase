\uuid{4Jn5}
\exo7id{3472}
\titre{exo7 3472}
\auteur{quercia}
\organisation{exo7}
\datecreate{2010-03-10}
\isIndication{false}
\isCorrection{true}
\chapitre{Déterminant, système linéaire}
\sousChapitre{Système linéaire, rang}

\contenu{
\texte{
Soit $A \in \mathcal{M}_n(\R)$, et $B = {}^t\!AA$.
}
\begin{enumerate}
    \item \question{Montrer que : $\forall\ Y \in \mathcal{M}_{n,1}(\R)$, ${}^tYY = 0 \Leftrightarrow Y = 0$.}
    \item \question{Montrer que : $\forall\ X \in \mathcal{M}_{n,1}(\R)$, $BX = 0 \Leftrightarrow AX = 0$.}
    \item \question{En déduire que $\mathrm{rg}(A) = \mathrm{rg}(B)$.}
    \item \question{Trouver une matrice $A \in \mathcal{M}_2(\C)$ telle que $\mathrm{rg}(A) \ne \mathrm{rg}({}^t\!AA)$.}
\reponse{
4. $A = \begin{pmatrix}1 & i \cr i &-1 \cr \end{pmatrix}$.
}
\end{enumerate}
}

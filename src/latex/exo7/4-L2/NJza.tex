\uuid{NJza}
\exo7id{3797}
\titre{exo7 3797}
\auteur{quercia}
\organisation{exo7}
\datecreate{2010-03-11}
\isIndication{false}
\isCorrection{true}
\chapitre{Espace euclidien, espace normé}
\sousChapitre{Endomorphismes auto-adjoints}

\contenu{
\texte{
Soit $H$ un espace de Hilbert et $(u_n)$ une suite d'endomorphismes
de~$H$ autoadjoints positifs continus telle que la suite
$(u_0+\dots+u_n)$ est bornée dans~${\cal L}_c(H)$.
Montrer que pour tout~$x\in H$ la série $\sum_{n=0}^\infty u_n(x)$ est
convergente.
}
\reponse{
Soit $K = \sup\{\|u_0+\dots+u_n\|\}$ et~$x\in H$.
On note $v_{p,q} = \sum_{n=p}^q u_n$ pour $p\le q$.
La série $\sum (u_n(x)\mid x)$ est convergente (termes positifs, sommes
partielles majorées) donc elle vérifie le critère de Cauchy~:
$(v_{p,q}(x)\mid x) \to 0$ lorsque $p,q\to\infty$.

Comme $v_{p,q}$ est positif, il vérifie l'inégalité de Cauchy-Schwarz~:
$$|(v_{p,q}(x)\mid y)|^2 \le (v_{p,q}(x)\mid x)(v_{p,q}(y)\mid y) \le 2K\|y\|^2(v_{p,q}(x)\mid x).$$
En particulier pour $y=v_{p,q}(x)$ on obtient~: $\|v_{p,q}(x)\|^2 \le 2K(v_{p,q}(x)\mid x)$
donc la série $\sum u_n(x)$ est de Cauchy.


\emph{Remarque :} exemple où $\sum u_n$ ne converge pas dans~${\cal L}_c(H)$~:
$H = \ell^2(\N)$ et $u_n =$ projection orthogonale sur $<\!e_n\!>$ où $e_n(p) = \delta_{n,p}$.
$\sum u_n$ converge simplement et non uniformément vers l'identité.
}
}

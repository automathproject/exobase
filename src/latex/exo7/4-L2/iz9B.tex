\uuid{iz9B}
\exo7id{3441}
\titre{exo7 3441}
\auteur{quercia}
\organisation{exo7}
\datecreate{2010-03-10}
\isIndication{false}
\isCorrection{true}
\chapitre{Déterminant, système linéaire}
\sousChapitre{Autre}

\contenu{
\texte{
On note $SL_n( K) = \{ M \in \mathcal{M}_n(K) \text{ tq } \det M = 1 \}$.
}
\begin{enumerate}
    \item \question{\begin{enumerate}}
    \item \question{Démontrer que $SL_n( K)$ est un groupe pour le produit matriciel.}
    \item \question{Démontrer que $SL_n( K)$ est engendré par les matrices :
    $I + \lambda E_{ij}$, $( j \ne i )$
    où $(E_{ij})$ est la base canonique de $\mathcal{M}_n(K)$, et $\lambda \in  K$
    (transformer une matrice $M \in SL_n( K)$ en $I$ par opérations
    élémentaires).}
\reponse{
2. (b) $(I + E_{ij})^k = I + kE_{ij}$.
              Calculer le pgcd d'une ligne par opérations élémentaires à l'aide
              de Bézout. Ce pgcd vaut 1 sinon $M \notin SL_n(\Z)$.
}
\end{enumerate}
}

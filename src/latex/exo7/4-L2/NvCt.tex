\uuid{NvCt}
\exo7id{1116}
\titre{exo7 1116}
\auteur{legall}
\organisation{exo7}
\datecreate{1998-09-01}
\isIndication{false}
\isCorrection{false}
\chapitre{Déterminant, système linéaire}
\sousChapitre{Calcul de déterminants}
\module{Algèbre}
\niveau{L2}
\difficulte{}

\contenu{
\texte{

}
\begin{enumerate}
    \item \question{Soient $ A \in M_p({\Rr}) $ et $ B \in
M_q({\Rr}).$ Calculer
(en fonction de $ \hbox{det}(A) $ et $ \hbox{det}(B) $) le
d\'eterminant de la matrice $ M =\begin{pmatrix}A & 0\cr 0 & B \cr \end{pmatrix} \in
M_{p+q}({\Rr}).$ (On pourra pour cela
d\'ecomposer $ M $ comme produit de deux matrices de d\'eterminant
\'evident et utiliser la multiplicativit\'e du d\'eterminant.)}
    \item \question{Soient $ A \in M_p({\Rr}) ,$ $ B \in M_q({\Rr}) $ et $ C
\in M_{p,q}({\Rr}) .$
Calculer le d\'eterminant de la matrice $ M =\begin{pmatrix}A & C\cr 0 & B \cr
\end{pmatrix} \in M_{p+q}({\Rr}) .$
(On pourra g\'en\'eraliser la m\'ethode de 1.)}
\end{enumerate}
}

\uuid{nIQ2}
\exo7id{5483}
\titre{exo7 5483}
\auteur{rouget}
\organisation{exo7}
\datecreate{2010-07-10}
\isIndication{false}
\isCorrection{true}
\chapitre{Espace euclidien, espace normé}
\sousChapitre{Produit scalaire, norme}

\contenu{
\texte{
Soit $E$ un $\Rr$ espace vectoriel de dimension finie. Soit $||\;||$ une norme sur $E$ vérifiant l'identité du parallèlogramme, c'est-à-dire~:~$\forall(x,y)\in E^2,\;||x+y||^2+||x-y||^2=2(||x||^2+||y||^2)$. On se propose de démontrer que $||\;||$ est associée à un produit scalaire.
On définit sur $E^2$ une application $f$ par~:~$\forall(x,y)\in E^2,\;f(x,y)=\frac{1}{4}(||x+y||^2-||x-y||^2)$.
}
\begin{enumerate}
    \item \question{Montrer que pour tout $(x,y,z)$ de $E^3$, on a~:~$f(x+z,y)+f(x-z,y)=2f(x,y)$.}
\reponse{Soit $(x,y,z)\in\Rr^3$.

\begin{align*}\ensuremath f(x+z,y)+f(x-z,y)&=\frac{1}{4}(||x+z+y||^2+||x-z+y||^2-||x+z-y||^2-||x-z-y||^2)\\
 &=\frac{1}{4}\left(2(||x+y||^2+||z||^2)-2(||x-y||^2+||z||^2)\right)=2f(x,y).
\end{align*}}
    \item \question{Montrer que pour tout $(x,y)$ de $E^2$, on a~:~$f(2x,y)=2f(x,y)$.}
\reponse{$2f(x,y)= f(x+x,y)+f(x-x,y)=f(2x,y)+f(0,y)$ mais $f(0,y)=(||y||^2-||-y||^2)=0$ (définition d'une norme).}
    \item \question{Montrer que pour tout $(x,y)$ de $E^2$ et tout rationnel $r$, on a~:~$f(rx,y)=rf(x,y)$.

On admettra que pour tout réel $\lambda$ et tout $(x,y)$ de $E^2$ on a~:~$f(\lambda x,y)=\lambda f(x,y)$ ( ce résultat provient de la continuité de $f$).}
\reponse{\textbullet~Montrons par récurrence que $\forall n\in\Nn,\;f(nx,y)=nf(x,y)$.
C'est clair pour $n=0$ et $n=1$.
Soit $n\geq0$. Si l'égalité est vraie pour $n$ et $n+1$ alors d'après 1), 

$$f((n+2)x,y)+f(nx,y)=f((n+1)x+x,y)+f((n+1)x-x,y)=2f((n+1)x,y),$$
et donc, par hypothèse de récurrence,

$$f((n+2)x,y)=2f((n+1)x,y)-f(nx,y)=2(n+1)f(x,y)-nf(x,y)=(n+2)f(x,y).$$
Le résultat est démontré par récurrence.
\textbullet~Soit $n\in\Nn^*$, $f(x,y)=f\left(n\times\frac{1}{n}.x,y\right)=nf\left(\frac{1}{n}x,y\right)$ et donc $f\left(\frac{1}{n}x,y\right)=\frac{1}{n}f(x,y)$.
\textbullet~Soit alors $r=\frac{p}{q}$, $p\in\Nn$, $q\in\Nn^*$, $f(rx,y)=\frac{1}{q}f(px,y)=p\frac{1}{q}f(x,y)=rf(x,y)$ et donc, pour tout rationnel positif $r$, $f(rx,y)=rf(x,y)$.
Enfin, si $r\leq0$, $f(rx,y)+f(-rx,y)=2f(0,y)=0$ (d'après 1)) et donc= $f(-rx,y)=-f(-rx,y)=rf(x,y)$.

\begin{center}
$\forall(x,y)\in E^2,\;\forall r\in\Qq,\;f(rx,y)=rf(x,y)$.
\end{center}}
    \item \question{Montrer que pour tout $(u,v,w)$ de $E^3$, $f(u,w)+f(v,w)=f(u+v,w)$.}
\reponse{On pose $x=\frac{1}{2}(u+v)$ et $y=\frac{1}{2}(u-v)$.

$$f(u,w)+f(v,w)=f(x+y,w)+f(x-y,w)=2f(x,w)=2f\left(\frac{1}{2}(u+v),w\right)=f(u+v,w).$$}
    \item \question{Montrer que $f$ est bilinéaire.}
\reponse{f est symétrique (définition d'une norme) et linéaire par rapport à sa première variable (d'après 3) et 4)).
Donc f est bilinéaire.}
    \item \question{Montrer que $||\;||$ est une norme euclidienne.}
\reponse{f est une forme bilinéaire symétrique.
Pour $x\in E$, $f(x,x)=\frac{1}{4}(||x+x||^2+||x-x||^2)=\frac{1}{4}||2x||^2=||x||^2$ (définition d'une norme) ce qui montre tout à la fois que $f$ est définie positive et donc un produit scalaire, et que $||\;||$ est la norme associée. $||\;||$ est donc une norme euclidienne.}
\end{enumerate}
}

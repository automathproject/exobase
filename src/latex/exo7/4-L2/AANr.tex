\uuid{AANr}
\exo7id{3496}
\auteur{quercia}
\organisation{exo7}
\datecreate{2010-03-10}
\isIndication{false}
\isCorrection{true}
\chapitre{Réduction d'endomorphisme, polynôme annulateur}
\sousChapitre{Diagonalisation}

\contenu{
\texte{
Diagonaliser les matrices suivantes :
}
\begin{enumerate}
    \item \question{$A=\begin{pmatrix} 1 & 5 \cr 2 & 4 \cr\end{pmatrix}$}
\reponse{$     P=\begin{pmatrix} 1 &-5 \cr 1 & 2 \cr\end{pmatrix}$,\,
$    D= \begin{pmatrix} 6 & 0 \cr 0&-1 \cr\end{pmatrix}$}
    \item \question{$A=\begin{pmatrix} 2 & 5 \cr  4 & 3 \cr\end{pmatrix}$}
\reponse{$    P= \begin{pmatrix} 5 & 1 \cr -4 & 1 \cr\end{pmatrix}$,\,
$     D=\begin{pmatrix}-2 & 0 \cr 0& 7 \cr\end{pmatrix}$}
    \item \question{$A=\begin{pmatrix} 5 & 3 \cr -8 &-6 \cr\end{pmatrix}$}
\reponse{$ P=\begin{pmatrix} 3 & 1 \cr -8 &-1 \cr\end{pmatrix}$,\,
      $ \begin{pmatrix}-3 & 0 \cr 0& 2 \cr\end{pmatrix}$}
    \item \question{$A=\begin{pmatrix} 4 & 4 \cr 1 & 4 \cr\end{pmatrix}$}
\reponse{$P=\begin{pmatrix} 2 &-2 \cr 1 & 1 \cr\end{pmatrix}$,\,
    $ D=\begin{pmatrix} 6 & 0 \cr 0 & 2 \cr\end{pmatrix}$}
\end{enumerate}
}

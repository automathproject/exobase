\uuid{0IuB}
\exo7id{3553}
\auteur{quercia}
\organisation{exo7}
\datecreate{2010-03-10}
\isIndication{false}
\isCorrection{true}
\chapitre{Réduction d'endomorphisme, polynôme annulateur}
\sousChapitre{Polynôme annulateur}

\contenu{
\texte{
Soit $A \in \mathcal{M}_n(\C)$ telle que $A^n = I$ et $(I,A,\dots,A^{n-1})$ est libre. Montrer qu'alors on a $\mathrm{tr}(A) = 0$.
}
\reponse{
$A$ est diagonalisable et a $n$ valeurs propres distinctes,
	     sinon il existerait un polynôme annulateur de degré inférieur ou
	     égal à $n-1$. Ces racines sont les $n$ racines $n$-èmes de 1 et
	     leur somme est nulle.
}
}

\uuid{WoOZ}
\exo7id{1550}
\titre{exo7 1550}
\auteur{barraud}
\organisation{exo7}
\datecreate{2003-09-01}
\isIndication{false}
\isCorrection{false}
\chapitre{Endomorphisme particulier}
\sousChapitre{Endomorphisme orthogonal}
\module{Algèbre}
\niveau{L2}
\difficulte{}

\contenu{
\texte{
Dans l'espace vectoriel $\R^{4}$ muni de son produit scalaire
canonique, on considère l'endomorphisme $f$ dont la matrice dans la
base canonique est :
$$
 A =\frac{1}{7} 
 \begin{pmatrix}
   -1 & -4 &  4 & -4 \\
   -4 &  5 &  2 & -2 \\
    4 &  2 &  5 &  2 \\
   -4 & -2 &  2 &  5 
 \end{pmatrix}
\rlap{\qquad {(\it attention au $\frac{1}{7}$...)}}
$$
}
\begin{enumerate}
    \item \question{Sans calculs, dire pourquoi $f$ est diagonalisable dans une base
orthonormée.}
    \item \question{Montrer que $f$ est orthogonal. En déduire les seules valeurs propres
possibles pour $f$.}
    \item \question{Sans calculer le polynôme caractéristique de $f$, déterminer à l'aide
de la trace l'ordre de multiplicité des valeurs propres de $f$. En
déduire le polynôme caractéristique de $f$.}
    \item \question{Déterminer l'espace propre $E_{1}$ associé à la valeur propre
1. En donner une base, puis lui appliquer le procédé de Schmidt pour 
obtenir une base orthonormée de $E_{1}.$}
    \item \question{Montrer que l'espace propre $E_{-1}$ associé à la valeur propre -1
satisfait $E_{-1}=(E_{1})^{\bot}$. En utilisant l'équation
caractérisant $E_{1}$, en déduire un vecteur générateur de $E_{-1}$.}
    \item \question{Donner une base orthonormée dans laquelle la matrice de $f$ est
diagonale. Donner une interprétation géométrique de $f$.}
\end{enumerate}
}

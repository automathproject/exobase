\uuid{vSVy}
\exo7id{3790}
\auteur{quercia}
\organisation{exo7}
\datecreate{2010-03-11}
\isIndication{false}
\isCorrection{true}
\chapitre{Espace euclidien, espace normé}
\sousChapitre{Endomorphismes auto-adjoints}

\contenu{
\texte{
Soient $h \in \mathcal{L}(E)$ autoadjoint, $\vec x_0 \in E$ unitaire, $p$ la projection
orthogonale sur $\text{vect}(\vec x_0)$, et $f = h+p$.

On note $\lambda_1 \le \dots \le \lambda_n$ les valeurs propres de $h$ et
$\mu_1 \le \dots \le \mu_n$ celles de $f$.

Montrer que $\lambda_1 \le \mu_1 \le \dots \le \lambda_n \le \mu_n$.
}
\reponse{
Soit
$(\vec h_i)$ une base diagonale pour $h$,
$H_i = \text{vect}\{\vec h_1, \dots, \vec h_i\}$
et $(\vec f_i)$, $F_i$ idem pour $f$.

Pour $\vec x \in F_k \cap H_{k-1}^\perp$,
$\lambda_k\|\vec x\,\|^2 + (\vec x\mid\vec x_0)^2 \le
 (h(\vec x)\mid\vec x) + (\vec x\mid\vec x_0)^2 =
 (f(\vec x)\mid\vec x) \le \mu_k\|\vec x\,\|^2$.

Pour $\vec x \in H_{k+1} \cap F_{k-1}^\perp \cap \vec x_0^\perp$,
$\mu_k\|\vec x\,\|^2 \le (f(\vec x)\mid\vec x) = (h(\vec x)\mid\vec x)
 \le \lambda_{k+1}\|\vec x\,\|^2$.
}
}

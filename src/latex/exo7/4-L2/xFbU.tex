\uuid{xFbU}
\exo7id{3820}
\auteur{quercia}
\organisation{exo7}
\datecreate{2010-03-11}
\isIndication{false}
\isCorrection{true}
\chapitre{Espace euclidien, espace normé}
\sousChapitre{Problèmes matriciels}

\contenu{
\texte{
Soient $A$ et $B$ deux matrices hermitiennes et  $C=A+B$. On note
$a_1\ge a_2\ge \dots\ge a_n$ les valeurs propres de la première,
$b_1\ge b_2\ge \dots\ge b_n$ celles de la deuxième, $c_1\ge c_2\ge \dots\ge c_n$ 
celles de la troisième.
Montrez que pour tout $i$ on a $c_i\ge a_i+b_n$.
{\it Indication~:} se ramener au cas $b_n=0$.
}
\reponse{
On remplace $A$ par $A + b_nI$ et $B$ par $B-b_nI$ ce
qui ne modifie pas $C$. Maintenant les valeurs propres de~$B$ sont
positves donc pour tou~$x\in\C^n$ on a $(Ax\mid x) \le (Cx\mid x)$.
Soit $(x_1,\dots,x_n)$ une base orthonormale propre pour~$A$
et $(y_1,\dots,y_n)$ une base orthonormale propre pour~$C$.
Si $z\in\mathrm{vect}(x_1,\dots x_i)$ alors $(Az\mid z) \ge a_i\|z\|^2$
et si $z\in\mathrm{vect}(y_i,\dots y_n)$ alors $(Az\mid z) \le (Cz\mid z) \le c_i\|z\|^2$.
Or $\mathrm{vect}(x_1,\dots x_i)$ et $\mathrm{vect}(y_i,\dots y_n)$ ont une intersection
non triviale (la somme des dimensions est égale à $n+1$) donc il existe
$z\ne 0$ tel que $a_i\|z\|^2 \le c_i\|z\|^2$ d'où $a_i\le c_i$.
}
}

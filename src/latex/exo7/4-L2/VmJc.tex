\uuid{VmJc}
\exo7id{5487}
\titre{exo7 5487}
\auteur{rouget}
\organisation{exo7}
\datecreate{2010-07-10}
\isIndication{false}
\isCorrection{true}
\chapitre{Espace euclidien, espace normé}
\sousChapitre{Autre}
\module{Algèbre}
\niveau{L2}
\difficulte{}

\contenu{
\texte{
Soit $a$ un vecteur non nul de l'espce euclidien $\Rr^3$. On définit $f$ de $\Rr^3$ dans lui même par~:~$\forall x\in\Rr^3,\;f(x)=a\wedge(a\wedge x)$. Montrer que $f$ est linéaire puis déterminer les vecteurs non nuls colinéaires à leur image par $f$.
}
\reponse{
Je vous laisse vérifier la linéarité.
Si $x$ est colinéaire à $a$, $f(x)=0$ et les vecteurs de $\mbox{Vect}(a)\setminus\{0\}$ sont des vecteurs non nuls colinéaires à leur image.
Si $x$ n'est pas colinéaire à $a$, $a\wedge x$ est un vecteur non nul orthogonal à $a$ et il en est de même de $f(x)=a\wedge(a\wedge x)$. Donc, si $x$ est colinéaire à $f(x)$, $x$ est nécessairement orthogonal à $a$.
Réciproquement, si $x$ est un vecteur non nul orthogonal à $a$, $f(x)=(a.x)a-\|a\|^2x=-||a||^2x$ et $x$ est colinéaire à $f(x)$. Les vecteurs non nuls colinéaires à leur image sont les vecteurs non nuls de $\mbox{Vect}(a)$ et de $a^\bot$.
}
}

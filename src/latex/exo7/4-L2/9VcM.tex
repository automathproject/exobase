\uuid{9VcM}
\exo7id{5355}
\titre{exo7 5355}
\auteur{rouget}
\organisation{exo7}
\datecreate{2010-07-04}
\isIndication{false}
\isCorrection{true}
\chapitre{Groupe, anneau, corps}
\sousChapitre{Groupe de permutation}

\contenu{
\texte{
Démontrer que $A_n$ est engendré par les cycles de longueur $3$ (pour $n\geq3$).
}
\reponse{
Les éléments de $A_n$ sont les produits pairs de transpositions. Il suffit donc de vérifier qu'un produit de deux transpositions est un produit de cycles de longueur $3$.

Soient $i$, $j$ et $k$ trois éléments deux à deux distincts de $\{1,...,n\}$. $\tau_{i,k}\circ\tau_{i,j}$ est le $3$-cycle~:~$i\rightarrow j$ $j\rightarrow k$ $k\rightarrow i$, ce qui montre qu'un $3$-cycle est pair et que le produit de deux transpositions dont les supports ont en commun un singleton est un $3$-cycle.

Le cas $\tau_{i,j}\circ\tau_{i,j}=Id=(2 3 1)(3 1 2)$ est immédiat. Il reste à étudier le produit de deux transpositions à supports disjoints.

Soient $i$, $j$, $k$ et $l$ quatre éléments de deux à deux distincts de $\{1,...,n\}$.

$$\tau_{i,j}\circ\tau_{k,l}=(j i k l)(i j l k)=(j i l k)=(j k i l)(l j i k).$$ 

Donc, $\tau_{i,j}\circ\tau_{k,l}$ est un bien un produit de $3$-cycles ce qui achève la démonstration.
}
}

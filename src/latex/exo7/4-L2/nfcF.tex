\uuid{nfcF}
\exo7id{3749}
\titre{exo7 3749}
\auteur{quercia}
\organisation{exo7}
\datecreate{2010-03-11}
\isIndication{false}
\isCorrection{false}
\chapitre{Espace euclidien, espace normé}
\sousChapitre{Projection, symétrie}
\module{Algèbre}
\niveau{L2}
\difficulte{}

\contenu{
\texte{
Soit $E$ un espace euclidien de dimension $n$.
}
\begin{enumerate}
    \item \question{Soit $F$ un sous-espace vectoriel de $E$ et $f : F \to E$ une application orthogonale.
    Démontrer qu'on peut prolonger $f$ en une application orthogonale
    de $E$ dans $E$.}
    \item \question{Soient $\vec u_1,\dots,\vec u_n$, $\vec v_1,\dots,\vec v_n$ des vecteurs
    de $E$ tels que :
    $\forall\ i,j,\ (\vec u_i \mid \vec u_j) = (\vec v_i \mid \vec v_j)$.
  \begin{enumerate}}
    \item \question{Si $\sum \lambda_i \vec u_i = \vec0$, démontrer que
    $\sum \lambda_i \vec v_i = \vec0$.}
    \item \question{En déduire qu'il existe $f \in {\cal O}(E)$ telle que :
    $\forall\ i,\ f(\vec u_i) = \vec v_i$.}
\end{enumerate}
}

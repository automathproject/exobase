\uuid{9jo6}
\exo7id{1642}
\titre{exo7 1642}
\auteur{barraud}
\organisation{exo7}
\datecreate{2003-09-01}
\isIndication{false}
\isCorrection{true}
\chapitre{Réduction d'endomorphisme, polynôme annulateur}
\sousChapitre{Diagonalisation}
\module{Algèbre}
\niveau{L2}
\difficulte{}

\contenu{
\texte{
Soit $A$ la matrice $A=
  \begin{pmatrix}
    1&-1&-1\\
   -1& 1&-1\\
   -1&-1& 1
  \end{pmatrix}
  $.
}
\begin{enumerate}
    \item \question{Calculer $^{t}\!A$. La matrice $A$ est-elle diagonalisable~?}
\reponse{${}^t{A}=A$ donc $A$ est diagonalisable dans une base
    orthonormée.}
    \item \question{Diagonaliser $A$.}
\reponse{Par exemple~: $P=
    \begin{pmatrix}
      1& 1& 1\\
      1&-1& 0\\
      1& 0&-1
    \end{pmatrix}
$, $P^{-1}AP=
\begin{pmatrix}
  -1& 0& 0\\
   0& 2& 0\\
   0& 0& 2
 \end{pmatrix}
$.}
    \item \question{Diagonaliser $A$ dans une base orthonormée (pour le produit scalaire
    usuel de $\R^{3}$).}
\reponse{$Q=
   \begin{pmatrix}
     1/\sqrt{3} & 1/\sqrt{2} & 1/\sqrt{6} \\
     1/\sqrt{3} &-1/\sqrt{2} & 1/\sqrt{6} \\
     1/\sqrt{3} &          0 &-2/\sqrt{6} 
   \end{pmatrix}
$, $Q^{-1}AQ=\begin{pmatrix}
  -1& 0& 0\\
   0& 2& 0\\
   0& 0& 2
 \end{pmatrix}
$ et ${}^t{Q}=Q^{-1}$}
\end{enumerate}
}

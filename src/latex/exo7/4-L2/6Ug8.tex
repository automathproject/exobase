\uuid{6Ug8}
\exo7id{3011}
\auteur{quercia}
\organisation{exo7}
\datecreate{2010-03-08}
\isIndication{false}
\isCorrection{false}
\chapitre{Groupe, anneau, corps}
\sousChapitre{Anneau}

\contenu{
\texte{
Soit $A$ un anneau commutatif, et $a \in A$. On dit que $a$ est nilpotent s'il
existe $n \in \N$ tel que $a^n = 0$.
}
\begin{enumerate}
    \item \question{Exemple : D{\'e}terminer les {\'e}l{\'e}ments nilpotents de $\Z/36\Z$.}
    \item \question{Montrer que l'ensemble des {\'e}l{\'e}ments nilpotents est un id{\'e}al de $A$.}
    \item \question{Soit $a$ nilpotent. Montrer que $1 - a$ est inversible
    (remarquer que $1 = 1^n - a^n$).}
    \item \question{Soient $a$ nilpotent et $b$ inversible. Montrer que $a + b$ est inversible.}
\end{enumerate}
}

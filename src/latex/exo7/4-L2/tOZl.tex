\uuid{tOZl}
\exo7id{7363}
\auteur{mourougane}
\organisation{exo7}
\datecreate{2021-08-10}
\isIndication{false}
\isCorrection{true}
\chapitre{Groupe, anneau, corps}
\sousChapitre{Groupe, sous-groupe}

\contenu{
\texte{

}
\begin{enumerate}
    \item \question{Quels sont les ordres possibles des éléments d'un groupe d'ordre $6$ ?}
\reponse{Les ordres possibles des éléments d'un groupe d'ordre $6$ sont, par le théorème de Lagrange les diviseurs de $6$ c'est à dire $1,2,3$ et $6$.}
    \item \question{Quels sont les éléments inversibles de $(\Z/7\Z,\times)$ ?}
\reponse{Les éléments inversibles de $(\Z/7\Z,\times)$ sont les classes des entiers premiers à $7$ c'est à dire toutes les classes non nulles.}
    \item \question{Ecrire la table de multiplication de $((\Z/7\Z)^\star,\times)$.}
\reponse{la table de multiplication de $((\Z/7\Z)^\star,\times)$ est


\begin{center}
\begin{tabular}{|c|c|c|c|c|c|c|} 
\hline
$\times$ &1&2&3&4&5&6\\
\hline
1&1&2&3&4&5&6\\
2&2&4&6&1&3&5\\
3&3&6&2&5&1&4\\
4&4&1&5&2&6&3\\
5&5&3&1&6&4&2\\
6&6&5&4&3&2&1\\
\hline
\end{tabular}
\end{center}}
    \item \question{Déterminer s'il en existe un générateur de $((\Z/7\Z)^\star,\times)$.}
\reponse{On calcule l'ordre multiplicatif de $2$. Comme $2^3=1$ l'ordre de $2$ est $3$. Comme $3^2=2$ et $3^3=6$, $3$ est d'ordre 6. C'est donc un générateur
 de $((\Z/7\Z)^\star,\times)$.}
    \item \question{Combien $((\Z/7\Z)^\star,\times)$ a-t-il de générateurs ?}
\reponse{On déduit de la question précédente que $((\Z/7\Z)^\star,\times)$ est un groupe cyclique d'ordre $6$. Il a donc $\phi(6)=\phi(2)\phi(3)=2$ générateurs.}
    \item \question{Le groupe $\Z/2\Z\times \Z/4\Z$ est-il cyclique ?}
\reponse{Dans le groupe $\Z/2\Z\times \Z/4\Z$ tous les éléments $g$ vérifient $4g=0$. Il n'y a donc aucun élément d'ordre $8$. Ce groupe n'est donc pas cyclique.}
    \item \question{Le groupe des inversibles de $\Z/20\Z$ est-il cyclique ?}
\reponse{Par le théorème chinois, puisque $4$ et $5$ sont premiers entre eux, 
$\Z/20\Z$ est isomorphe à $\Z/4\Z\times \Z/5\Z$.
Le groupe  $(\Z/20\Z)^\star$ est donc isomorphe au produit 
$(\Z/4\Z)^\star\times (\Z/5\Z)^\star$. Comme $(\Z/4\Z)^\star$ est un groupe d'ordre $2$, il est isomorphe à $\Z/2\Z$. Comme $5$ est premier, $(\Z/5\Z)^\star$ est un groupe cyclique d'ordre $4$ donc isomorphe à $\Z/4\Z$.
Par conséquent, le groupe  $(\Z/20\Z)^\star$ n'est pas cyclique.}
\end{enumerate}
}

\uuid{Ji7x}
\exo7id{2453}
\titre{exo7 2453}
\auteur{matexo1}
\organisation{exo7}
\datecreate{2002-02-01}
\video{NwFtb_SYHA8}
\isIndication{true}
\isCorrection{true}
\chapitre{Déterminant, système linéaire}
\sousChapitre{Calcul de déterminants}

\contenu{
\texte{
Montrer que
$$\left|
\begin{array}{ccccc}
1 & t_1 & t_1^2 & \ldots & t_1^{n-1} \\
1 & t_2 & t_2^2 & \ldots & t_2^{n-1} \\\
\ldots&\ldots&\ldots& \ldots & \ldots \\
1 & t_n & t_n^2 & \ldots & t_n^{n-1}
\end{array}\right|
 = \prod_{1 \le i < j \le n} (t_j - t_i) $$
}
\indication{Faire les opérations suivantes sur les colonnes
$C_n \leftarrow C_n-t_n C_{n-1}$,
puis $C_{n-1} \leftarrow C_{n-1}-t_n C_{n-2}$,...,
$C_2 \leftarrow  C_2-t_nC_1$.
Développer par rapport a la bonne ligne et reconnaître
que l'on obtient le déterminant recherché mais au rang $n-1$.}
\reponse{
Notons $V_n$ le déterminant à calculer
et $C_1,\ldots,C_n$ les colonnes de la matrice correspondante.

Nous allons faire les opérations suivantes sur les colonnes
en partant de la dernière colonne.
$C_n$ est remplacée par $C_n-t_n C_{n-1}$,
puis $C_{n-1}$ est remplacée par $C_{n-1}-t_n C_{n-2}$,...
jusqu'à $C_2$ qui est remplacée par $C_2-t_nC_1$.
On obtient donc

$$V_n=\begin{vmatrix}
1 & t_1 & t_1^2 & \ldots & t_1^{n-1} \\
1 & t_2 & t_2^2 & \ldots & t_2^{n-1} \\\
\ldots&\ldots&\ldots& \ldots & \ldots \\
1 & t_n & t_n^2 & \ldots & t_n^{n-1}
\end{vmatrix} 
= 
\begin{vmatrix}
1 & t_1-t_n & t_1^2-t_1t_n & \ldots & t_1^{n-1}-t_1^{n-2}t_n \\
1 & t_2-t_n & t_2^2-t_2t_n & \ldots & t_2^{n-1}-t_2^{n-2}t_n \\\
\ldots&\ldots&\ldots& \ldots & \ldots \\
1 & 0 & 0 & \ldots & 0
\end{vmatrix}
$$

On développe par rapport à la dernière ligne et on écrit $t_i^k-t_i^{k-1}t_n=t_i^{k-1}(t_i-t_n)$ pour obtenir :
$$V_n = (-1)^{n-1}\begin{vmatrix}
 t_1-t_n & t_1(t_1-t_n) & \ldots & t_1^{n-2}(t_1-t_n) \\
 t_2-t_n & t_2(t_2-t_n) & \ldots & t_2^{n-2}(t_2-t_n) \\\
\ldots&\ldots& \ldots & \ldots \\
 t_{n-1}-t_n & \ldots & \ldots & \ldots
\end{vmatrix}$$

Nous utilisons maintenant la linéarité du déterminant par rapport à chacune des lignes :
on factorise la première ligne par $t_1-t_n$ ; la second par $t_2-t_n$,...
On obtient 
$$V_n = (-1)^{n-1}(t_1-t_n)(t_2-t_n)\cdots(t_{n-1}-t_n)
\begin{vmatrix}
1 & t_1 & t_1^2 & \ldots & t_1^{n-2} \\
1 & t_2 & t_2^2 & \ldots & t_2^{n-2} \\\
\ldots&\ldots&\ldots& \ldots & \ldots \\
1 & t_{n-1} & t_{n-1}^2 & \ldots & t_{n-1}^{n-2}
\end{vmatrix} 
$$
Donc $$V_n = V_{n-1}\prod_{j=1}^{n-1}(t_n-t_j).$$

Si maintenant on suppose la formule connue pour $V_{n-1}$
c'est-à-dire $V_{n-1}(t_1,\ldots,t_{n-1})
= \prod_{1 \le i < j \le n-1} (t_j - t_i)$

Alors on obtient par récurrence que
$$V_n(t_1,\ldots,t_{n-1},t_n)= V_{n-1}(t_1,\ldots,t_{n-1})\prod_{j=1}^{n-1}(t_n-t_j) =  \prod_{1 \le i < j \le n} (t_j - t_i).$$
}
}

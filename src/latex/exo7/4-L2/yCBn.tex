\uuid{yCBn}
\exo7id{1314}
\titre{exo7 1314}
\auteur{ortiz}
\organisation{exo7}
\datecreate{1999-04-01}
\isIndication{false}
\isCorrection{true}
\chapitre{Groupe, anneau, corps}
\sousChapitre{Groupe, sous-groupe}
\module{Algèbre}
\niveau{L2}
\difficulte{}

\contenu{
\texte{
Pour la multiplication usuelles des matrices
carr\'ees, les ensembles suivants sont-ils des groupes : $${
\text{GL}(2,\Rr)\cap\mathcal{M}_2(\Zz),\quad
\left\{M\in\mathcal{M}_2(\Zz):\det M=1\right\} \text{ ?} }$$
}
\reponse{
Le premier ensemble n'est pas un groupe car, par exemple, la
matrice $\begin{pmatrix}2&0\\0&2\\ \end{pmatrix}$ ne peut avoir
pour inverse que $\begin{pmatrix} \frac{1}{2}&0\\0&\frac{1}{2}\\
\end{pmatrix}$ qui n'appartient pas à l'ensemble.

Notons $G = \{ M \in \mathcal{M}_{2}(\Zz) : \det M = 1 \}$ et
montrons que $G$ est un sous-groupe de $Gl(2,\Rr)$.
\begin{itemize}
  \item la matrice identité appartient à $G$.
  \item si $A,B \in G$ alors $AB \in \mathcal{M}_{2}(\Zz)$ et
$\det AB = \det A \times \det B = 1\times 1 =1$, et donc $AB \in
G$.
  \item Si $A = \begin{pmatrix} a&b\\c&d\\ \end{pmatrix}$ ($a,b,c,d \in \Zz$) alors
$\frac{1}{\det A}\begin{pmatrix} d&-b\\-c&a\\ \end{pmatrix}
=\begin{pmatrix} d&-b\\-c&a\\ \end{pmatrix}$ appartient à $G$ et
est l'inverse de  $A$.
\end{itemize}
}
}

\uuid{h05r}
\exo7id{3436}
\titre{exo7 3436}
\auteur{quercia}
\organisation{exo7}
\datecreate{2010-03-10}
\isIndication{false}
\isCorrection{true}
\chapitre{Déterminant, système linéaire}
\sousChapitre{Autre}

\contenu{
\texte{
\def \dddd#1#2{\text{det}\bigl(\vec #1,\vec #2\bigr)}\relax
Soient $\vec a, \vec b, \vec c, \vec d$ quatre vecteurs d'un ev $E$ de
dimension 2. On note $\det$ le déterminant dans une base fixée de $E$.
}
\begin{enumerate}
    \item \question{Démontrer que : $\dddd ab \dddd cd + \dddd ac \dddd db + \dddd ad \dddd bc = 0$.
    \par
    (Commencer par le cas où $(\vec a,\vec b)$ est libre)}
    \item \question{On donne six scalaires : $d_{ab}, d_{ac}, d_{ad}, d_{cd}, d_{db}, d_{bc}$
    tels que $d_{ab} d_{cd} + d_{ac} d_{db} + d_{ad} d_{bc} = 0$.
    \par
    Montrer qu'il existe des vecteurs $\vec a, \vec b, \vec c, \vec d$ tels que :
    $\forall\ x,y,\ d_{xy} = \d xy$.}
\reponse{
%\def \dddd#1#2{\text{det}\bigl(\vec #1,\vec #2\bigr)}\relax
    Si $d_{ab} \ne 0$, prendre 
    $$\left\{
    \begin{array}{lllll}
    \vec c &=& -\frac {d_{bc}}{d_{ab}}\vec a &+& \frac {d_{ac}}{d_{ab}}\vec b \cr
    \vec d &=& \phantom{-}\frac {d_{db}}{d_{ab}}\vec a
                           &+& \frac {d_{ad}}{d_{ab}}\vec b.\cr
\end{array}\right. $$
}
\end{enumerate}
}

\uuid{WT9e}
\exo7id{3568}
\auteur{quercia}
\organisation{exo7}
\datecreate{2010-03-10}
\isIndication{false}
\isCorrection{true}
\chapitre{Endomorphisme particulier}
\sousChapitre{Autre}

\contenu{
\texte{
Soit~$E$ un espace vectoriel de dimension finie et $u\in\mathcal{L}(E)$. On considère l'application~$\Phi_u$ qui à $v\in\mathcal{L}(E)$
associe~$v\circ u$.
}
\begin{enumerate}
    \item \question{Montrer que $\Phi_u\in\mathcal{L}({\mathcal{L}(E)})$.}
    \item \question{Montrer l'équivalence~: ($u$ est diagonalisable) $\Leftrightarrow$ ($\Phi_u$ est diagonalisable)\dots
  \begin{enumerate}}
    \item \question{en considérant les polynômes annulateurs de~$u$ et de~$\Phi_u$.}
    \item \question{en considérant les spectres et sous-espaces propres de~$u$ et de~$\Phi_u$.}
\reponse{
\begin{enumerate}
Pour $p\in K[X]$ on a $P(\Phi_u) = v \mapsto v\circ P(u)$ donc $u$ et $\Phi_u$ ont
    mêmes polynômes annulateurs.
$(\lambda\in\mathrm{Spec}(\Phi_u)) \Leftrightarrow (\exists\ v\ne 0\text{ tq }v\circ(u-\lambda\mathrm{id}_E) = 0)
    \Leftrightarrow (u-\lambda\mathrm{id}_E\text{ n'est pas surjectif}) \Leftrightarrow (\lambda\in\mathrm{Spec}(u))$.
    Ainsi $\Phi_u$ et $u$ ont même spectre.
    Si $\lambda\in\mathrm{Spec}(u)$ et $v\in\mathcal{L}(E)$ on a~:
    $(\Phi_u(v) = \lambda v) \Leftrightarrow (\Im(u-\lambda\mathrm{id}_E) \subset\mathrm{Ker} v)$ donc
    $\mathrm{Ker}(\Phi_u-\lambda\mathrm{id}_{\mathcal{L}(E)})$ est isomorphe à $\mathcal{L}({H,E})$ où $H$ est un supplémentaire
    de~$\Im(u-\lambda\mathrm{id}_E)$.
    On en déduit~:
    $\dim(\mathrm{Ker}(\Phi_u-\lambda\mathrm{id}_{\mathcal{L}(E)})) = \dim(E)\dim(\mathrm{Ker}(u-\lambda\mathrm{id}_E)$.
}
\end{enumerate}
}

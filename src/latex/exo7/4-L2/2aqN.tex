\uuid{2aqN}
\exo7id{1552}
\titre{exo7 1552}
\auteur{barraud}
\organisation{exo7}
\datecreate{2003-09-01}
\isIndication{false}
\isCorrection{false}
\chapitre{Endomorphisme particulier}
\sousChapitre{Endomorphisme orthogonal}

\contenu{
\texte{
Dans un espace euclidien $(E,<\cdot ,\cdot>)$, on considère un vecteur
  $v$ non nul, un scalaire $\lambda$ et l'endomorphisme~:
$$
u:
\begin{array}{ccl}
E &  \rightarrow &E  \\
x &\mapsto  &x+\lambda <x,v> v
\end{array}
$$
}
\begin{enumerate}
    \item \question{Pour $x\in E$, calculer $\Vert u(x)\Vert^{2}$.}
    \item \question{Donner une condition nécessaire et suffisante sur $\lambda$ et $v$
  pour que $u$ soit une transformation orthogonale.}
    \item \question{Lorsque $u$ est orthogonale, dire a priori quelles sont les valeurs
  propres possibles de $u$, puis dire si elles sont effectivement valeur
  propre en étudiant les espaces propres associés.}
    \item \question{Lorsque $u$ est orthogonale, donner une interprétation géométrique de
  $u$.}
\end{enumerate}
}

\uuid{pQzQ}
\exo7id{3753}
\titre{exo7 3753}
\auteur{quercia}
\organisation{exo7}
\datecreate{2010-03-11}
\isIndication{false}
\isCorrection{true}
\chapitre{Espace euclidien, espace normé}
\sousChapitre{Projection, symétrie}
\module{Algèbre}
\niveau{L2}
\difficulte{}

\contenu{
\texte{

}
\begin{enumerate}
    \item \question{Trouver les matrices $A \in \mathcal{M}_n(\R)$ telles que :
    $A{}^t\!A + A + {}^t\!A = 0$.}
\reponse{$A = P-I$, $P \in {\cal O}(n)$.}
    \item \question{Montrer que pour une telle matrice, $|\det A| \le 2^n$.}
\reponse{Hadamard.}
\end{enumerate}
}

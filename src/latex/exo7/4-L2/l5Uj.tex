\uuid{l5Uj}
\exo7id{7396}
\auteur{mourougane}
\organisation{exo7}
\datecreate{2021-08-10}
\isIndication{false}
\isCorrection{true}
\chapitre{Groupe, anneau, corps}
\sousChapitre{Autre}

\contenu{
\texte{

}
\begin{enumerate}
    \item \question{Enoncer le théorème de Lagrange.}
\reponse{Si $G$ est un groupe fini et $H$ un sous-groupe de $G$, alors $\text{card}(G)=\text{card}(G/H)\text{card}(H)$. En particulier, l'ordre d'un élément d'un groupe fini divise l'ordre du groupe.}
    \item \question{Soit $G$ un groupe et $a$ un élément d'ordre $k$ dans $G$. Soit $p$ un entier naturel. Quel est l'ordre de $a^p$ ?}
\reponse{L'ordre de $a^p$ est $\frac{k}{k\wedge p}$.}
    \item \question{Le groupe $\mathcal{S}_3$ des permutations de $\{1,2,3\}$ est-il cyclique ? (justifier)}
\reponse{On note que l'ordre de $\mathcal{S}_3$ est $6$. Un groupe fini d'ordre $6$ est cyclique si est seulement s'il possède un élément d'ordre $6$. Il y a trois éléments d'ordre 2, deux éléments d'ordre 3, et un élément d'ordre 1 dans $\mathcal{S}_3$. Alors le groupe n'est pas cyclique.
Noter aussi qu'il n'est pas commutatif.}
    \item \question{Enoncer le théorème de la division euclidienne dans $k[X]$.}
\reponse{Soit $D\in k[X]$ un polynôme non nul. Pour tout polynôme $A$ de $k[X]$ il existe un unique couple $(Q,R)\in (k[X])^2$
tel que $A=DQ+R$ et $\text{deg} (R)< \text{deg} (D)$.}
    \item \question{Enoncer le théorème de la division euclidienne dans $\Z[i]$.}
\reponse{Soit $A$ et $B$ deux entiers de Gauss avec $B\neq 0$. Alors il existe deux entiers de Gauss $Q$ et $R$ tels que 
$A=BQ+R$ et $N(R)<N(B)$.}
\end{enumerate}
}

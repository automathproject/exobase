\uuid{icQG}
\exo7id{3575}
\titre{exo7 3575}
\auteur{quercia}
\organisation{exo7}
\datecreate{2010-03-10}
\isIndication{false}
\isCorrection{true}
\chapitre{Endomorphisme particulier}
\sousChapitre{Autre}

\contenu{
\texte{
Soit~$E$ un espace vectoriel de dimension $n$ finie, $p$ un projecteur de rang~$r$
et $\varphi : {\mathcal{L}(E)} \to {\mathcal{L}(E)}, u \mapsto {\frac12(p\circ u + u\circ p).}$
}
\begin{enumerate}
    \item \question{Est-ce que~$\varphi$ est diagonalisable~?}
\reponse{Oui, les applications $u \mapsto p\circ u$ et $u \mapsto u\circ p$ le sont
    (ce sont des projecteurs) et elles commutent.}
    \item \question{Déterminer les valeurs propres de~$\varphi$ et les dimensions des sous-espaces propres.}
\reponse{Soit $\cal B$ une base de~$E$ obtenue par concaténation d'une
    base de~$\mathrm{Ker} p$ et d'une base de~$\Im p$.

    Si $\mathrm{mat}_{\cal B}(u) = \left(\begin{smallmatrix}A&B\cr C&D\cr\end{smallmatrix}\right)$
    alors $\mathrm{mat}_{\cal B}(\varphi(u)) = \left(\begin{smallmatrix}A&B/2\cr C/2&D\cr\end{smallmatrix}\right)$,
    d'où $\mathrm{Spec}(\varphi) \subset \{0,\frac12,1\}$ et $d_0 = (n-r)^2$,
    $d_1=r^2$ et $d_{1/2} = 2r(n-r)$.}
\end{enumerate}
}

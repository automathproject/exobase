\uuid{6Vph}
\exo7id{3592}
\auteur{quercia}
\organisation{exo7}
\datecreate{2010-03-10}
\isIndication{false}
\isCorrection{true}
\chapitre{Réduction d'endomorphisme, polynôme annulateur}
\sousChapitre{Applications}

\contenu{
\texte{
Soit $A = \begin{pmatrix} 9 &0 &0 \cr 1 &4 &0 \cr 1 &1 &1 \cr \end{pmatrix}$.
Trouver les matrices $M \in \mathcal{M}_{3}(\R)$ telles que $M^2 = A$.
}
\reponse{
$M = \pm\begin{pmatrix} 3 &0 &0 \cr 1/5 &\pm2 &0 \cr 7/30 &\pm1/3 &\pm1 \cr \end{pmatrix}$
      ou $M = \pm\begin{pmatrix} 3 &0 &0 \cr 1   &\mp2 &0 \cr 1/2  &\mp1   &\pm1 \cr \end{pmatrix}$.
}
}

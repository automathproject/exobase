\uuid{bZBe}
\exo7id{1553}
\titre{exo7 1553}
\auteur{barraud}
\organisation{exo7}
\datecreate{2003-09-01}
\isIndication{false}
\isCorrection{false}
\chapitre{Endomorphisme particulier}
\sousChapitre{Endomorphisme orthogonal}

\contenu{
\texte{
On considère un espace euclidien $(E,<>)$. On dit qu'un endomorphisme
$u$ de $E$ est une similitude de $E$ si et seulement si il existe un
réel $\lambda>0$ tel que 
$$
 u^{*}u=\lambda\mathrm{id}
$$ 
Montrer que les trois assertions suivantes sont équivalentes :
}
\begin{enumerate}
    \item \question{[\it(i)] 
$u$ est une similitude}
    \item \question{[\it(ii)]
$u$ est colinéaire à une transformation orthogonale, c'est à dire
$$
 \exists\alpha\in\R\setminus\{0\},\ \exists v\in O(E)\ /\ u=\alpha v
$$}
    \item \question{[\it(iii)]
$u$ conserve l'orthogonalité, c'est à dire :
$$
 \forall(x,y)\in E^{2}, <x,y>=0\Rightarrow <u(x),u(y)>=0
$$
{\it
  Pour (i)$\Leftrightarrow$(ii), on pourra commencer par montrer que
  (ii)$\Rightarrow$(i).

  Pour (i)$\Rightarrow$(iii), on commencera par
  montrer que $x$ et $u^{*}u(x)$ sont toujours colinéaires, c'est à
  dire que 
  $$\forall x\in E\exists\lambda_{x}/u^{*}u(x)=\lambda_{x}x$$
  puis on montrera que $\lambda_{x}$ est indépendant de $x$.}}
\end{enumerate}
}

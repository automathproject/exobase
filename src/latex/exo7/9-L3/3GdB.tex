\uuid{3GdB}
\exo7id{6435}
\titre{exo7 6435}
\auteur{potyag}
\organisation{exo7}
\datecreate{2011-10-16}
\isIndication{false}
\isCorrection{false}
\chapitre{Géométrie et trigonométrie hyperbolique}
\sousChapitre{Géométrie et trigonométrie hyperbolique}

\contenu{
\texte{

}
\begin{enumerate}
    \item \question{Démontrer que chaque homographie non-triviale  $g\in M^+(2)\setminus\{{\rm id}\}$ possède soit un point fixe soit deux.
   Un élément $g\in M^+(2)$ est dit {\it parabolique} si son ensemble
   des points fixes : ${\rm fix}(g)=\{x\in\overline{\C}\ \vert\ f(x)=x\}$
   est un singleton. Montrer qu'un élément est parabolique {ssi}
   il est conjugué dans $M^+(2)$ à la translation $z\mapsto z+1$.}
    \item \question{Un élément $f\in M^+(2)$ est dit {\it elliptique}
  s'il est conjugué dans $M^+(2)$ à une rotation $z\mapsto k\cdot z$
  où $\vert k\vert = 1$ , $k\in\C\setminus \{1\}$.

  Un élément $g\in M^+(2)$ est dit {\it loxodromique} s'il est conjugué
  dans $M^+(2)$ à $z\mapsto k\cdot z$ où $\vert k\vert\not=1$.
  De plus, un élément loxodromique est dit {\it hyperbolique} si
  $k\in\R^*_+\setminus\{1\}$ ; un élément loxodromique est dit
  srtictement loxodromique si $k=\lambda\cdot e^{i\theta},\ \lambda > 0\ ,\
  \theta\not=2\pi m$.
  Pour un élément $g\in M^+(2)$ on utilise la même lettre pour une de deux
  matrices dans $SL_2\C$ qui le représentent (en fait c'est $g$ ou $-g$).
  Montrer que pour tout élément $g\in M^+(2)$ l'une des
  possibilités suivantes peut avoir lieu :


\begin{enumerate}}
    \item \question{$g$ est parabolique ssi ${\rm tr}^2(g)=4$, où $tr^2$
  est la trace carré de la matrice $g$.}
    \item \question{$g$ est elliptique ssi ${\rm
  tr}^2(g)\in [0,4[.$}
    \item \question{$g$ est hyperbolique ssi ${\rm tr}^2(g)\in ]4, \infty[.$}
    \item \question{$g$ est strictement loxodromique {\bf ssi} ${\rm
tr}^2(g)\not\in[0,\infty[$.}
\end{enumerate}
}

\uuid{U92k}
\exo7id{6391}
\titre{exo7 6391}
\auteur{potyag}
\organisation{exo7}
\datecreate{2011-10-16}
\isIndication{false}
\isCorrection{false}
\chapitre{Isométrie euclidienne}
\sousChapitre{Isométrie euclidienne}

\contenu{
\texte{
Une application du type $G(l,a)= \tau _l. t_a$ s'appelle {\it réflexion
glissée} si le vecteur $a$ est parallèle à la droite $l$ $\subset \Rr^2$.
}
\begin{enumerate}
    \item \question{Si $G = \tau _l. t_a$ est une réflexion glissée alors
montrer que $\tau_l . t_a = t_a  .\tau _l$ et  $G^2 = t_{2a}$.}
    \item \question{Montrer que $G = \tau _l t_a$ est une réflexion si $ l$ et
$a$ sont perpendiculaires et est une réflexion glissée si $l $
et $a$ ne sont pas perpendiculaires.}
    \item \question{En regardant l'ensemble des points fixes $fix(f) := \{ x \in \Rr^2 \vert
f(x) =x\}$ d'une isométrie $f \in Iso(\Rr^2)$ montrer que :
  \begin{enumerate}}
    \item \question{si $fix (f) \not= \emptyset$ alors $f=R(a,\alpha )$ ou $f = \tau_l$}
    \item \question{si $fix (f) = \emptyset$ alors $f=t_a$ ou $f=G(l,a)$ ({\it indication : utiliser
  la question 2. et l'exercice \ref{exo:avant}, question 2}).}
\end{enumerate}
}

\uuid{GIxF}
\exo7id{7752}
\auteur{mourougane}
\organisation{exo7}
\datecreate{2021-08-11}
\isIndication{false}
\isCorrection{false}
\chapitre{Géométrie projective}
\sousChapitre{Géométrie projective}

\contenu{
\texte{
Soit $\vec{V}$ un espace vectoriel et $P:=P(\vec{V})$.
}
\begin{enumerate}
    \item \question{Soit $F$ et $G$ deux sous-espaces projectifs disjoints de $P$.
Montrer qu'il existe un unique sous-espace projectif $<F,G>$ de $P$ de
dimension $\dim F+\dim G+1$ contenant $F$ et $G$.}
    \item \question{Soit $F$ et $G$ deux sous-espaces projectifs disjoints de $P$
 tels que $\dim F+\dim G= \dim P-1 $. Quel est le domaine de
 définition de l'application (appelée perspective) ?
$$\begin{array}{ccccc}
P(\vec{V})&\to& G &\subset& P(\vec{V})\\
M&\mapsto & G\cap <M,F>&=&P (\vec{G}\cap (\vec{M}\oplus\vec{F}))
\end{array}$$
Montrer que c'est une application projective.}
\end{enumerate}
}

\uuid{jzWA}
\exo7id{6270}
\titre{exo7 6270}
\auteur{queffelec}
\organisation{exo7}
\datecreate{2011-10-16}
\isIndication{false}
\isCorrection{false}
\chapitre{Difféomorphisme, théorème d'inversion locale et des fonctions implicites}
\sousChapitre{Difféomorphisme, théorème d'inversion locale et des fonctions implicites}
\module{Calcul différentiel}
\niveau{L3}
\difficulte{}

\contenu{
\texte{
Soit $(u,v)\in{\Rr}^2\to F(u,v)\in{\Rr}$ une application de classe $C^1$, telle que $F(0,0)=0$ et
${\partial F\over\partial v}(0,0)\neq 0$. On considère $\varphi:{\Rr}^3\to {\Rr}^2$
telle que $\varphi(x,y,z)=(xy, x^2-y^2-z)$ et l'application $f=F\circ\varphi$.
Montrer que l'équation $f(x,y,z)=0$ définit au voisinage de $(0,0)$ une
applica\-tion $z=\psi(x,y)$ vérifiant 
$$x{\partial \psi\over\partial x}-y{\partial \psi\over\partial y}=2(x^2+y^2).$$
}
}

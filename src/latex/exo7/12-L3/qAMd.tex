\uuid{qAMd}
\exo7id{2517}
\auteur{mayer}
\organisation{exo7}
\datecreate{2009-04-01}
\isIndication{false}
\isCorrection{false}
\chapitre{Différentiabilité, calcul de différentielles}
\sousChapitre{Différentiabilité, calcul de différentielles}

\contenu{
\texte{
Soit ${\cal E}$ un espace
vectoriel r\'eel muni d'un produit scalaire $(x,y) \mapsto \langle
x,y \rangle$ et de la norme associ\'ee $\|x\| = \langle x,x
\rangle ^\frac{1}{2}$.
 Soit $u$ un endomorphisme continu de ${\cal E}$ que l'on suppose sym\'etrique, i.e.
$$\langle u(x),y\rangle =\langle x,u(y)\rangle  \quad  \text{pour tout} \;\; x,y\in {\cal E} \; .$$
}
\begin{enumerate}
    \item \question{Montrer que l'application $x\in {\cal E} \mapsto \langle
u(x),x\rangle $ est diff\'erentiable sur ${\cal E}$ et calculer sa
diff\'erentielle. L'application $x\mapsto \|x\|^2$ est donc
diff\'erentiable.}
    \item \question{On d\'efinit une application $\varphi :
{\cal E} \setminus \{0\} \to \Rr$ en posant $\varphi (x) =
\frac{\langle u(x),x\rangle }{\langle x,x\rangle }$. \'Etablir
qu'il s'agit d'une application diff\'erentiable. Calculer ensuite
$D\varphi$. Montrer que, pour un \'el\'ement non nul $a\in {\cal
E}$, on a $D\varphi (a) =0$ si et seulement si $a$ est vecteur
propre de $u$.}
\end{enumerate}
}

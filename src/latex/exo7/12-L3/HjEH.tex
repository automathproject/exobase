\uuid{HjEH}
\exo7id{6883}
\auteur{romon}
\organisation{exo7}
\datecreate{2012-08-30}
\isIndication{false}
\isCorrection{true}
\chapitre{Extremum, extremum lié}
\sousChapitre{Extremum, extremum lié}

\contenu{
\texte{

}
\begin{enumerate}
    \item \question{Soit $E$ un espace vectoriel de dimension finie, $F$ un fermé non 
  vide de $E$ et $x \in E$ un point, appartenant ou non à $F$.
  Montrer qu'il existe un point $\bar{x} \in F$ tel que
  \[ 
  \| x - \bar{x} \| = \inf_{z \in F} \| x - z \| . 
  \]
  (Question subsidiaire: ce point est-il unique?)}
\reponse{Remarque: si $x\in F$, alors clairement $\bar{x}=x$ convient. \\
  Comme $F \neq \varnothing$, il existe un point $y_0 \in F$ et posons
  $r = \| x - y_0 \| \ge 0$ par hypothèse. On peut découper $F$
  en deux morceaux : $F_0 = F \cap \bar{B} (x, r)$ et $F_1 = F -
  F_0$. $F_0$ est non vide car il contient $y_0$. Quant aux points de $F_1$,
  ils sont tous à une distance de $x$ strictement supérieure à $r$ donc
  l'infimum de la distance sur $F$ est l'infimum sur $F_0$ :
  \[ \inf_{z \in F} \| x - z \| = \inf_{z \in F_0} \| x - z
     \| \le r \]
  Or $F_0$ est fermé et borné, donc compact (car la dimension est finie),
  et la fonction $z \mapsto \| x - z \|$ étant continue, elle est
  bornée et atteint son minimum en un point $y$. 
  \\
  Question subsidiaire: $\bar{x}$ n'est pas unique (sauf dans certains cas, par exemple si $F$ est convexe);
  contre-exemple: $F$ est un cercle et $x$ est son centre.}
    \item \question{On se place désormais dans l'espace vectoriel $\mathcal{M}_2 (
  \mathbb{R})$ des matrices $2 \times 2$ à coefficients réels,
  muni de la norme
  \[ \left\| \left(\begin{array}{cc}
       a & c\\
       b & d
     \end{array} \right) \right\| = \sqrt{a^2 + b^2 + c^2 + d^2} \]
  et on considère l'ensemble $\text{SL}_2 (\mathbb{R})$
  des matrices $\left(\begin{array}{cc}
    a & c\\
    b & d
  \end{array} \right)$ de déterminant égal à $1$.
  \begin{enumerate}}
\reponse{\begin{enumerate}}
    \item \question{Montrer que $\text{SL}_2(\mathbb{R})$ est fermé dans $\mathcal{M}_2 (\mathbb{R})$.}
\reponse{On voit que $\mathcal{M}_2 (\mathbb{R}) \simeq
    \mathbb{R}^4$ et que la fonction déterminant est $g : (a, b, c, d
) \mapsto ad - bc$. Cette fonction est continue car polynomiale en
    les coordonnées donc $\text{SL}_2 (\mathbb{R}) = g^{-1}(\{1\})$ est un fermé.}
    \item \question{Montrer que $\text{SL}_2(\mathbb{R})$ n'est \emph{pas} bornée.}
\reponse{La suite de matrices 
    $\displaystyle M_n = \begin{pmatrix} 1 & n \\ 0 & 1 \end{pmatrix}$ 
    est dans $\text{SL}_2(\mathbb{R})$ sans être bornée ($\| M_n \| = \sqrt{2+n^2}$).}
    \item \question{Soit $f : \mathcal{M}_2 (\mathbb{R}) \rightarrow
    \mathbb{R}$ la fonction qui à une matrice $M$ associe $f (M
) = \| M \|$. On cherche la ou les matrices $M$ réalisant l'infimum de la fonction $f$
    sur $\text{SL}_2(\mathbb{R})$, autrement dit la ou les matrices les plus proches de la matrice nulle.
    \begin{enumerate}}
\reponse{\begin{enumerate}}
    \item \question{Montrer que $f_{|\text{SL}_2(\mathbb{R})}$ est minorée et atteint son infimum.}
\reponse{C'est une conséquence directe du (1) en prenant pour $x$ la matrice nulle.}
    \item \question{Calculer le gradient de $f$. Est-il toujours défini?}
\reponse{Écrivons $f (M) = \sqrt{a^2 + b^2 + c^2 + d^2}$,
      alors $\nabla f (a, b, c, d) = \frac{1}{f (M)}
      (a, b, c, d)$, défini à condition que $M$ ne soit pas
      la matrice nulle (qui n'appartient pas $\text{SL}_2 (\mathbb{R}
   )$ donc il n'y~a pas de problème).}
    \item \question{Trouver le ou les extrema de $f_{|\text{SL}_2(\mathbb{R})}$. Montrer qu'il s'agit du
      minimum et en déduire
      \[
      \inf_{M \in \text{SL}_2 (\mathbb{R})}  \| M \|.
      \]}
\reponse{En un extremum on doit avoir $\nabla f = \lambda \nabla g$ (extrema liés) donc
      \[ \frac{1}{\| M \|}  \left(\begin{array}{c}
           a\\
           b\\
           c\\
           d
         \end{array} \right) = \lambda \left(\begin{array}{c}
           d\\
           - c\\
           - b\\
           a
         \end{array} \right) 
      \]
      ce qui n'est possible que si $\lambda = \epsilon / \| M \|$
      où $\epsilon \in \{ - 1, + 1 \}$, et que $d = \epsilon a$
      et $c = - \epsilon b$ donc $M = \left(\begin{array}{cc}
        a & - \epsilon b\\
        b & \epsilon a
      \end{array} \right)$. Si l'on rajoute la contrainte $\det M = 1$, il
      vient que $\epsilon = + 1$ et $a^2 + b^2 = 1$. Donc les seuls extrema
      possibles sont de la forme $M_{\theta} = \left(\begin{array}{cc}
        \cos \theta & - \sin \theta\\
        \sin \theta & \cos \theta
      \end{array} \right)$ avec $\theta \in \mathbb{R}$, et pour toutes ces
      matrices $\| M_{\theta} \| = \sqrt{2}$.
      
      Or d'après le (i) on sait que le minimum est atteint, et
      de plus il doit satisfaire la relation de Lagrange car les gradients
      sont bien définis. Il y a donc un infinité de minima (les rotations
      $M_{\theta}$) et ils prennent la même valeur
      \[ 
      \inf_{M \in \text{SL}_2 (\mathbb{R})}  \| M
      \| = \min_{M \in \text{SL}_2 (\mathbb{R})} 
      \| M \| = \sqrt{2} . 
      \]}
\end{enumerate}
}

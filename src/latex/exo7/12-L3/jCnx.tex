\uuid{jCnx}
\exo7id{2541}
\auteur{tahani}
\organisation{exo7}
\datecreate{2009-04-01}
\isIndication{false}
\isCorrection{true}
\chapitre{Difféomorphisme, théorème d'inversion locale et des fonctions implicites}
\sousChapitre{Difféomorphisme, théorème d'inversion locale et des fonctions implicites}

\contenu{
\texte{
Soit $f:\mathbb{R}^3 \rightarrow \mathbb{R}^2$ d\'efinie par
$f(x,y,z)=(x^2-y^2+z^2-1, xyz-1)$. Soit $(x_0,y_0,z_0) \in
\mathbb{R}^3$ tel que $f(x_0,y_0,z_0)=(0,0)$. Montrez qu'il existe
un intervalle $I$ contenant $x_0$ et une application $\varphi: I
\rightarrow \mathbb{R}^2$ tels que $\varphi(x_0)=(y_0,z_0)$ et
$f(x,\varphi(x))=0$ pour tout $x\in I$.
}
\reponse{
Soit $(x_0,y_0,z_0) \in \mathbb{R}^3$ tel que
$f(x_0,y_0,z_0)=(0,0)$ (par exemple $(1,1,1)$). $f$ est $C^1$ car
coordonn\'ees polynomiales.

$$\text{Mat}\, D_2f(x_0,y_0,z_0)=
\left(
\begin{array}{cc}
\frac{\partial f_1}{\partial y}(x_0,y_0,z_0) &\frac{\partial
f_1}{\partial
z}(x_0,y_0,z_0)\\
\frac{\partial f_2}{\partial y}(x_0,y_0,z_0) & \frac{\partial
f_2}{\partial z}(x_0,y_0,z_0)
\end{array} \right )=\left(
\begin{array}{cc}
-2y_0 & 2z_0 \\
x_0z_0 & x_0y_0
\end{array} \right )$$

$\det(\text{Mat}\, D_2f(x_0,y_0,z_0))=-2x_0(y_0^2+z_0^2) \neq 0$ car
$x_0y_0z_0=1$ donc $x_0 \neq 0, y_0 \neq 0, z_0 \neq 0$. D'apr\`es
le th\'eor\`eme des fonctions implicites, il existe $I$ intervalle
contenant $x_0$ et $\varphi:I \rightarrow \mathbb{R}^2$ tel que
$f(x,\varphi(x))=0$ pour tout $x \in I$ et $\varphi(x_0)=(y_0,z_0)$.
}
}

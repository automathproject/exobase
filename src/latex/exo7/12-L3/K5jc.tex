\uuid{K5jc}
\exo7id{1860}
\titre{exo7 1860}
\auteur{maillot}
\organisation{exo7}
\datecreate{2001-09-01}
\isIndication{false}
\isCorrection{false}
\chapitre{Difféomorphisme, théorème d'inversion locale et des fonctions implicites}
\sousChapitre{Difféomorphisme, théorème d'inversion locale et des fonctions implicites}
\module{Calcul différentiel}
\niveau{L3}
\difficulte{}

\contenu{
\texte{
Montrer que la relation $$f(x,y,z)=x^3+y^3+z^3 -2z(x+y)-2x+y-2z-1=0$$
d\'efinit au voisinage de $(0,0,-1)$ une fonction implicite $z=
\phi(x,y)$. Donner un d\'eve\-lop\-pe\-ment limit\'e de $\phi$ \`a
l'ordre 2 en $(0,0)$.
}
}

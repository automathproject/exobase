\uuid{RPBn}
\exo7id{2580}
\titre{exo7 2580}
\auteur{delaunay}
\organisation{exo7}
\datecreate{2009-05-19}
\isIndication{false}
\isCorrection{true}
\chapitre{Réduction d'endomorphisme, polynôme annulateur}
\sousChapitre{Valeur propre, vecteur propre}
\module{Algèbre}
\niveau{L2}
\difficulte{}

\contenu{
\texte{
Soit $M$ la matrice de $\R^4$ suivante
$$M=\begin{pmatrix}0&1&0&0 \\  2&0&-1&0 \\  0&7&0&6 \\  0&0&3&0\end{pmatrix}$$
}
\begin{enumerate}
    \item \question{D\'eterminer les valeurs propres de $M$ et ses sous-espaces propres.}
\reponse{D\'eterminons les valeurs propres de $M$ et ses sous-espaces propres.

Les valeurs propres de $M$ sont les r\'eels $\lambda$ tels que $\det (M-\lambda I)=0$.
$$\det(M-\lambda I)=\begin{vmatrix}-\lambda&1&0&0 \\  2&-\lambda&-1&0 \\  0&7&-\lambda&6 \\  0&0&3&-\lambda\end{vmatrix}=\lambda^4-13\lambda^2+36=(\lambda^2-4)(\lambda^2-9)=(\lambda-2)(\lambda+2)(\lambda-3)(\lambda+3).$$
Les valeurs propres de $M$ sont donc $2,-2,3$ et $-3$. Notons $E_2$, $E_{-2}$, $E_3$ et $E_{-3}$ les sous-espaces propres associ\'es.

\begin{align*}
E_2&=\{X\in\R^4,MX=2X\} \\ &=\left\{(x,y,z,t)\in \R^4,y=2x,2x-z=2y,7y+6t=2z,3z=2t\right\} 
\end{align*}
$${\hbox{or}}\ \left\{\begin{align*}y&=2x \\  2x-z&=2y \\  7y+6t&=2z \\  3z&=2t\end{align*}\right.\iff
\left\{\begin{align*}y&=2x \\  2x-z&=4x \\  14x+9z&=2z \\  3z&=2t\end{align*}\right.\iff
\left\{\begin{align*}y&=2x \\  z&=-2x \\  t&=-3x\end{align*}\right.$$
ainsi, $E_2$ est la droite vectorielle engendr\'ee par le vecteur $u_1=(1,2,-2,-3)$.

\begin{align*}
E_{-2}&=\{X\in\R^4,MX=-2X\} \\ &=\left\{(x,y,z,t)\in \R^4,y=-2x,2x-z=-2y,7y+6t=-2z,3z=-2t\right\} 
\end{align*}
$${\hbox{or}}\ \left\{\begin{align*}y&=-2x \\  2x-z&=-2y \\  7y+6t&=-2z \\  3z&=-2t\end{align*}\right.\iff
\left\{\begin{align*}y&=-2x \\  2x-z&=4x \\  -14x-9z&=2z \\  3z&=-2t\end{align*}\right.\iff
\left\{\begin{align*}y&=-2x \\  z&=-2x \\  t&=3x\end{align*}\right.$$
ainsi, $E_{-2}$ est la droite vectorielle engendr\'ee par le vecteur $u_2=(1,-2,-2,3)$.

\begin{align*}
E_3&=\{X\in\R^4,MX=3X\} \\  &=\left\{(x,y,z,t)\in \R^4,y=3x,2x-z=3y,7y+6t=3z,3z=3t\right\}
\end{align*}
$${\hbox{or}}\ \left\{\begin{align*}y&=3x \\  2x-z&=3y \\  7y+6t&=3z \\  3z&=3t\end{align*}\right.\iff
\left\{\begin{align*}y&=3x \\  2x-z&=9x \\  21x+6t&=3z \\  z&=t\end{align*}\right.\iff
\left\{\begin{align*}y&=3x \\  z&=-7x \\  t&=-7x\end{align*}\right.$$
ainsi, $E_3$ est la droite vectorielle engendr\'ee par le vecteur $u_3=(1,3,-7,-7)$.

\begin{align*}
E_{-3}&=\{X\in\R^4,MX=-3X\} \\ &=\left\{(x,y,z,t)\in \R^4,y=-3x,2x-z=-3y,7y+6t=-3z,3z=-3t\right\}
\end{align*}
$${\hbox{or}}\ \left\{\begin{align*}y&=-3x \\  2x-z&=-3y \\  7y+6t&=-3z \\  3z&=-3t\end{align*}\right.\iff
\left\{\begin{align*}y&=-3x \\  2x-z&=9x \\  -21x-6z&=-3z \\  z&=-t\end{align*}\right.\iff
\left\{\begin{align*}y&=-3x \\  z&=-7x \\  t&=7x\end{align*}\right.$$
ainsi, $E_{-3}$ est la droite vectorielle engendr\'ee par le vecteur $u_4=(1,-3,-7,7)$.}
    \item \question{Montrer que $M$ est diagonalisable.}
\reponse{Montrons que $M$ est diagonalisable. 

La matrice $M$ admet quatre valeurs propres distinctes, ce qui prouve que les quatres vecteurs propres correspondants sont lin\'eairement ind\'ependants. En effet, les vecteurs $u_1,u_2,u_3$ et $u_4$ d\'etermin\'es en $1)$ forment une base de $\R^4$. L'endomorphisme dont la matrice est $M$ dans la base canonique de $\R^4$ est repr\'esent\'e par une matrice diagonale dans la base
$(u_1,u_2,u_3,u_4)$ puisque $Mu_1=2u_1, Mu_2=-2u_2$, 
$Mu_3=3u_3$ et $Mu_4=-3u_4$.}
    \item \question{D\'eterminer une base de vecteurs propres et $P$ la matrice de passage.}
\reponse{D\'eterminons une base de vecteurs propres et $P$ la matrice de passage.

Une base de vecteurs propres a \'et\'e d\'etermin\'ee dans les questions pr\'ec\'edentes. C'est la base $(u_1,u_2,u_3,u_4)$ et la matrice de passage est la matrice 
$$P=\begin{pmatrix}1&1&1&1 \\  2&-2&3&-3 \\  -2&-2&-7&-7 \\  -3&3&-7&7\end{pmatrix}$$}
    \item \question{On a $D=P^{-1}MP$, pour $k\in\N$ exprimer $M^k$ en fonction de $D^k$, puis calculer $M^k$.}
\reponse{On a $D=P^{-1}MP$, pour $k\in\N$ exprimons $M^k$ en fonction de $D^k$, puis calculons $M^k$.

On a $$D=\begin{pmatrix}2&0&0&0 \\  0&-2&0&0 \\  0&0&3&0 \\  0&0&0&-3\end{pmatrix}\ {\hbox{donc}}\ 
D^k=\begin{pmatrix}2^k&0&0&0 \\  0&(-2)^k&0&0 \\  0&0&3^k&0 \\  0&0&0&(-3)^k\end{pmatrix}.$$

Mais, $M=PDP^{-1}$, d'o\`u, pour $k\in\N$, $M^k=(PDP^{-1})^k=PD^kP^{-1}$. 

Pour calculer $M^k$, il faut donc d\'eterminer la matrice $P^{-1}$ qui exprime les coordonn\'ees des vecteurs de la base canonique de $\R^4$ dans la base $(u_1,u_2,u_3,u_4)$.

On r\'esout le syst\`eme, et on a :
$$\begin{align*}u_1&=i+2j-2k-3l \\  u_2&=i-2j-2k+3l \\  u_3&=i+3j-7k-7t \\  u_4&=i-3j-7k+7l\end{align*}\iff
\left\{\begin{align*}i&={\frac{1}{10}}(7u_1+7u_2-2u_3-2u_4) \\  
j&={\frac{1}{10}}(7u_1-7u_2-3u_3+3u_4) \\  
k&={\frac{1}{10}}(u_1+u_2-u_3-u_4) \\  
l&={\frac{1}{10}}(3u_1-3u_2-2u_3+2u_4)\end{align*}\right.$$
d'o\`u
$$P^{-1}={\frac{1}{10}}\begin{pmatrix}7&7&1&3 \\  7&-7&1&-3 \\  -2&-3&-1&-2 \\  -2&3&-1&2\end{pmatrix}$$
et 
$$M^k=PD^kP^{-1}={\frac{1}{10}}\begin{pmatrix}1&1&1&1 \\  2&-2&3&-3 \\  -2&-2&-7&-7 \\  -3&3&7&7\end{pmatrix}
\begin{pmatrix}2^k&0&0&0 \\  0&(-2)^k&0&0 \\  0&0&3^k&0 \\  0&0&0&(-3)^k\end{pmatrix}
\begin{pmatrix}7&7&1&3 \\  7&-7&1&-3 \\  -2&-3&-1&-2 \\  -2&3&-1&2\end{pmatrix}.$$}
\end{enumerate}
}

\uuid{DeCg}
\exo7id{1636}
\titre{exo7 1636}
\auteur{barraud}
\organisation{exo7}
\datecreate{2003-09-01}
\isIndication{false}
\isCorrection{true}
\chapitre{Réduction d'endomorphisme, polynôme annulateur}
\sousChapitre{Diagonalisation}
\module{Algèbre}
\niveau{L2}
\difficulte{}

\contenu{
\texte{
Soit $J=\left(
  \begin{smallmatrix}
    \frac{1}{2}&\frac{1}{2}  \\[.5ex]
    \frac{1}{2}&\frac{1}{2}
  \end{smallmatrix}\right)
 $ et $A=
  \begin{pmatrix}
    0 &\vline& J  \\ \hline
    J &\vline&0
  \end{pmatrix}$. 
  Calculer $A^{2}$, puis $A^{3}$. A l'aide d'un polynôme annulateur de
  $A$, montrer que $A$ est diagonalisable.

  Sans chercher à calculer le polynôme caractéristique de $A$, donner un
  ensemble fini contenant toutes les valeurs propres de $A$, puis donner
  les valeurs valeurs propres elles mêmes ainsi que leurs multiplicités. En
  déduire le polynôme caractéristique de $A$.
}
\reponse{
On a $A^{3}=A$, donc $P=X^{3}-X=(X-1)(X+1)X$ est un polyn\^ome
  annulateur de $A$. Il s'agit d'un poyn\^ome scindé à racine simples
  donc $A$ est diagonalisable. Les valeurs propres de $A$ sont des
  racines de $P$ donc $\mathrm{Sp}(A)\subset\{0,1,-1\}$. On a $\mathrm{rg} A=2$
  donc $0$ est valeur propre de multiplicité 2. La résolution de système
  $(A+I)X=0$ montre que $\ker (A+I)=\R\Big(\begin{smallmatrix}%
    1\\1\\-1\\-1%
  \end{smallmatrix}\Big)$,
  donc $-1$ est valeur propre de multiplicité 1 donc $1$ est
  nécessairement valeur propre de multiplicité 1~: on en déduit que
  $\chi_{A}=X^{2}(X-1)(X+1)$.
}
}

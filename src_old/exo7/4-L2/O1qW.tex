\uuid{O1qW}
\exo7id{5685}
\titre{exo7 5685}
\auteur{rouget}
\organisation{exo7}
\datecreate{2010-10-16}
\isIndication{false}
\isCorrection{true}
\chapitre{Réduction d'endomorphisme, polynôme annulateur}
\sousChapitre{Applications}
\module{Algèbre}
\niveau{L2}
\difficulte{}

\contenu{
\texte{
Commutant de $\left(
\begin{array}{ccc}
1&0&-1\\
1&2&1\\
2&2&3
\end{array}
\right)$.
}
\reponse{
Soit $A=\left(
\begin{array}{ccc}
1&0&-1\\
1&2&1\\
2&2&3
\end{array}
\right)$. $\chi_A=\left|
\begin{array}{ccc}
1-X&0&-1\\
1&2-X&1\\
2&2&3-X
\end{array}
\right|=(1-X)(X^2-5X+4)-(-2+2X)=(1-X)(X^2-5X+4+2)=-(X-1)(X-2)(X-3)$.

$A$ est à valeurs propres réelles et simples. $A$ est diagonalisable dans $\Rr$ et les sous-espaces propres sont des droites.

Si $M$ est une matrice qui commute avec $A$, $M$ laisse stable ces droites et donc si $P$ est une matrice inversible telle que $P^{-1}AP$ soit diagonale alors la matrice $P^{-1}MP$ est diagonale. Réciproquement une telle matrice commute avec $A$.

\begin{center}
$C(A)=\{P\text{diag}(a,b,c)P^{-1},\;(a,b,c)\in\Cc^3\}$.
\end{center}

On trouve $C(A)=\left\{\left(\begin{array}{ccc}
2b-c&-a+2b-c&\frac{a-c}{2}\\
-b+c&a-b+c&(-a+c)/2\\
2c-2b&-2b+c&c
\end{array}
\right),\;(a,b,c)\in\Cc^3\right\}$. On peut vérifier que $C(A)=\text{Vect}(I,A,A^2)$.
}
}

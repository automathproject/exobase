\uuid{yGRY}
\exo7id{5795}
\titre{exo7 5795}
\auteur{rouget}
\organisation{exo7}
\datecreate{2010-10-16}
\isIndication{false}
\isCorrection{true}
\chapitre{Espace euclidien, espace normé}
\sousChapitre{Problèmes matriciels}
\module{Algèbre}
\niveau{L2}
\difficulte{}

\contenu{
\texte{
Déterminer $\text{card}(O_n(\Rr)\cap\mathcal{M}_n(\Zz))$.
}
\reponse{
Soit $A$ une matrice orthogonale à coefficients entiers. Puisque les colonnes ou les lignes de $A$ sont unitaires, on trouve par ligne ou par colonne un et un seul coefficient de valeur absolue égale à $1$, les autres coefficients étant nuls. $A$ est donc obtenue en multipliant chaque coefficient d'une matrice de permutation par $1$ ou $-1$. Réciproquement, une telle matrice est orthogonale à coefficients entiers.

Il y a $n!$ matrices de permutation et pour chaque matrice de permutation $2^n$ façons d'attribuer un signe $+$ ou $-$ à chaque coefficient égal à $1$. Donc

\begin{center}
\shadowbox{
$\text{card}(O_n(\Rr)\cap\mathcal{M}_n(\Zz)) = 2^nn!$.
}
\end{center}
}
}

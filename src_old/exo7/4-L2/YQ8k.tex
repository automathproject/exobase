\uuid{YQ8k}
\exo7id{5650}
\titre{exo7 5650}
\auteur{rouget}
\organisation{exo7}
\datecreate{2010-10-16}
\isIndication{false}
\isCorrection{true}
\chapitre{Déterminant, système linéaire}
\sousChapitre{Calcul de déterminants}
\module{Algèbre}
\niveau{L2}
\difficulte{}

\contenu{
\texte{
Calculer les déterminants suivants :
}
\begin{enumerate}
    \item \question{$\text{det}A$ où $A\in M_{2n}(\Kk)$ est telle que $a_{i,i}= a$ et $a_{i,2n+1-i}=b$ et $a_{i,j}= 0$ sinon.}
\reponse{Sans modifier la valeur de $\text{det}A$, on effectue les transformations :$\forall j\in\llbracket1,n\rrbracket$, $C_j\leftarrow C_j+C_{2n+1-j}$.

On obtient alors par linéarité du déterminant par rapport à chacune des $n$ premières colonnes

\begin{center}
$\text{det}A =(a+b)^p\left|
\begin{array}{cccccccc}
1&0&\ldots&0&0&\ldots&0&b\\
0&\ddots&\ddots&\vdots&\vdots& & &0\\
\vdots&\ddots&\ddots&0&0& & &\vdots\\
0&\ldots&0&1&b&0&\ldots&0\\
0&\ldots&0&1&a&0&\ldots&0\\
\vdots& & &0&0&\ddots&\ddots&\vdots\\
0& & &\vdots&\vdots&\ddots&\ddots&0\\
1&0&\ldots&0&0&\ldots&0&a 
\end{array}
\right|$.
\end{center}

On effectue ensuite les transformations : $\forall i\in\llbracket n+1, 2n\rrbracket$, $L_i\leftarrow L_i - L_{2n+1-i}$ et par linéarité du déterminant par rapport aux $n$ dernières lignes, on obtient

\begin{center}
$\text{det}A = (a+b)^n(a-b)^n= (a^2-b^2)^n$.
\end{center}}
    \item \question{$\left|\begin{array}{cccccc}
1&0&\ldots&\ldots&0&1\\
0&0& & &0&0\\
\vdots& & & & &\vdots\\
\vdots& & & & &\vdots\\
0&0& & &0&0\\
1&0&\ldots&\ldots&0&1
\end{array}
\right|$}
\reponse{Ce déterminant a deux colonnes égales et est donc nul.}
    \item \question{$\left|\begin{array}{ccccc}
1&\ldots& &\ldots&1\\
\vdots&0&1&\ldots&1\\
 &1&\ddots&\ddots&\vdots\\
\vdots&\vdots&\ddots&\ddots&1\\
1&1&\ldots&1&0
\end{array}
\right|$ et $\left|\begin{array}{ccccc}
0&1&\ldots&\ldots&1\\
1&\ddots&\ddots& &\vdots\\
\vdots&\ddots& &\ddots&\vdots\\
\vdots& &\ddots&\ddots&1\\
1&\ldots&\ldots&1&0
\end{array}
\right|$ $(n\geqslant2$)}
\reponse{On retranche la première colonne à toutes les autres et on obtient un déterminant triangulaire : $D_n=(-1)^{n-1}$.

Pour le deuxième déterminant, on ajoute les $n-1$ dernières colonnes à la première puis on met $n-1$ en facteur de la première colonne et on retombe sur le déterminant précédent. On obtient :
$D_n=(-1)^{n-1}(n-1)$.}
    \item \question{(I) $\left|\begin{array}{cccc}
a&b&\ldots&b\\
b&\ddots&\ddots&\vdots\\
\vdots&\ddots&\ddots&b\\
b&\ldots&b&a
\end{array}
\right|$ $(n\geqslant2$).}
\reponse{On ajoute les $n-1$ dernières colonnes à la première puis on met $a+(n-1)b$ en facteur de la première colonne. On obtient

\begin{center}
$D_n=(a+(n-1)b)\left|
\begin{array}{ccccc}
1&b&\ldots&\ldots&b\\
\vdots&a&\ddots& &\vdots\\
 &b&\ddots&\ddots&\vdots\\
\vdots&\vdots&\ddots&\ddots&b\\
1&b&\ldots&b&a
\end{array}
\right|$.
\end{center}

On retranche ensuite la première ligne à toutes les autres et on obtient

\begin{center}
$D_n=(a+(n-1)b)\left|
\begin{array}{ccccc}
1&b&\ldots&\ldots&b\\
0&a-b&0&\ldots&0\\
\vdots&0&\ddots&\ddots&\vdots\\
\vdots&\vdots&\ddots&\ddots&0\\
0&0&\ldots&0&a-b
\end{array}
\right|=(a+(n-1)b)(a-b)^{n-1}$.
\end{center}

\begin{center}
\shadowbox{
$\left|\begin{array}{cccc}
a&b&\ldots&b\\
b&\ddots&\ddots&\vdots\\
\vdots&\ddots&\ddots&b\\
b&\ldots&b&a
\end{array}
\right|=(a+(n-1)b)(a-b)^{n-1}$.
}
\end{center}}
\end{enumerate}
}

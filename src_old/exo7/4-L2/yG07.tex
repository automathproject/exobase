\uuid{yG07}
\exo7id{7032}
\titre{exo7 7032}
\auteur{megy}
\organisation{exo7}
\datecreate{2016-10-28}
\isIndication{false}
\isCorrection{false}
\chapitre{Déterminant, système linéaire}
\sousChapitre{Système linéaire, rang}
\module{Algèbre}
\niveau{L2}
\difficulte{}

\contenu{
\texte{
% vu dans un cours de Term ES

Soient $a, b, c$ trois réels et $f : \R\to \R, \: x\mapsto ae^x+bx+c$. On suppose que le graphe de $f$ contient le point de coordonnées $(0,1)$, et que sa tangente en ce point contient également le point de coordonnées $(2,3)$. On suppose enfin que le graphe admet une tangente horizontale au point d'abscisse $\ln(3)$. Déterminer $a, b, c$.
}
}

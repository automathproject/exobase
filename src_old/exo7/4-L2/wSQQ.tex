\uuid{wSQQ}
\exo7id{1365}
\titre{exo7 1365}
\auteur{ortiz}
\organisation{exo7}
\datecreate{1999-04-01}
\isIndication{false}
\isCorrection{false}
\chapitre{Groupe, anneau, corps}
\sousChapitre{Anneau}
\module{Algèbre}
\niveau{L2}
\difficulte{}

\contenu{
\texte{
Soit $A$ un anneau commutatif. On dit que $a\in A$
est nilpotent s'il existe $n\in \Nn^{*}$ tel que
$a^n=0$. On pose $\mathcal{N}\left( A\right)
=\left\{ a\in A:a\text{ est nilpotent}\right\} .$
}
\begin{enumerate}
    \item \question{Dans cette question, $A=\Zz{/}72\Zz$. Montrer
que $\overline{6}\in \mathcal{N}\left( A\right) $
puis que $\mathcal{N}
\left( A\right) =\left\{ \lambda \overline{6}:\lambda \in \Zz\right\} .$}
    \item \question{Que peut-on dire de $\mathcal{N}\left( A\right) $ si $A$ est
int\`egre?}
    \item \question{Montrer que $\mathcal{N}\left( A\right) $ est un id\'eal de $A$ %(on
%pourra utiliser la formule du bin\^ome de Newton).}
\end{enumerate}
}

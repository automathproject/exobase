\uuid{zfEo}
\exo7id{1600}
\titre{exo7 1600}
\auteur{roussel}
\organisation{exo7}
\datecreate{2001-09-01}
\isIndication{false}
\isCorrection{false}
\chapitre{Réduction d'endomorphisme, polynôme annulateur}
\sousChapitre{Valeur propre, vecteur propre}
\module{Algèbre}
\niveau{L2}
\difficulte{}

\contenu{
\texte{

}
\begin{enumerate}
    \item \question{Soient $f$ et $g$ deux endomorphisme s d'un espace vectoriel $E$ de dimension
  $n$ sur \\
  $K=\mathbb{R}$ ou $\mathbb{C}$, ayant chacun $n$ valeurs propres distinctes dans
  $K$. Montrer que
  $$ f \circ g = g \circ f \quad \Longleftrightarrow \quad f \mbox{~et~} g \mbox{~ont les m\^emes vecteurs propres}.$$}
    \item \question{Supposons maintenant que $K=\mathbb{C}$ et que $ f \circ g = g \circ f $.
  Si $u$ est un endomorphisme ~on dit qu'un espace vectoriel $F$ est $u$-stable si
  $u(F) \subset F$. Montrer que tout sous-espace propre de $f$ est $g$-stable. 

  \emph{Remarque} : On peut montrer par r\'ecurrence sur $n$ qu'il existe un
  vecteur propre commun \`a $f$et $g$. On admettra ce r\'esultat.}
    \item \question{Consid\'erons $f$ et $g$ deux endomorphismes de $\mathbb{R}^3$ dont les
  matrices dans la base canonique sont respectivement
  $$ M = \left (
  \begin{array}{cccc}
    5  & -4  & -4   \\
    1  & 0  & -2   \\
    1  & -1  & 1   \\
  \end{array}
  \right )
  ~~~~~\text{et}~~~~~N = \left (
  \begin{array}{ccc}
    -2  & 2  & 2  \\
    1  & -1  & -2  \\
    -2  & 2  & 3  \\
  \end{array}
  \right ) $$
  
  \begin{itemize}}
    \item \question{V\'erifier que $f \circ g = g \circ f$ et d\'eterminer les sous-espaces
    propres de $M$ et $N$.}
    \item \question{D\'eterminer une base de $\mathbb{R}^3$ dans laquelle les matrices de $f$
    et $g$ sont diagonales.
  \end{itemize}}
\end{enumerate}
}

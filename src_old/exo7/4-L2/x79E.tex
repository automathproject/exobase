\uuid{x79E}
\exo7id{1505}
\titre{exo7 1505}
\auteur{barraud}
\organisation{exo7}
\datecreate{2003-09-01}
\isIndication{false}
\isCorrection{false}
\chapitre{Espace euclidien, espace normé}
\sousChapitre{Orthonormalisation}
\module{Algèbre}
\niveau{L2}
\difficulte{}

\contenu{
\texte{
On considère la forme bilinéaire $b$ de $\R^{4}$ définie par :
  $$
  b(x,y)=x_{1}y_{1}+2x_{2}y_{2}+4x_{3}y_{3}+18x_{4}y_{4}
  + x_{1}y_{3}+ x_{3}y_{1}
  +2x_{2}y_{4}+2x_{4}y_{2}
  +6x_{3}y_{4}+6x_{4}y_{3}
  $$
  où $x_{1},x_{2},x_{3}$ et $y_{1},y_{2},y_{3}$ sont les coordonnées
  de $x$ et $y$ dans la base canonique.
}
\begin{enumerate}
    \item \question{Montrer qu'il s'agit d'un produit scalaire.}
    \item \question{Ecrire la matrice de $b$ dans la base canonique.}
    \item \question{Trouver une base orthonormée pour $b$.}
\end{enumerate}
}

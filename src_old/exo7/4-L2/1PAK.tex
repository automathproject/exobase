\uuid{1PAK}
\exo7id{5312}
\titre{exo7 5312}
\auteur{rouget}
\organisation{exo7}
\datecreate{2010-07-04}
\isIndication{false}
\isCorrection{true}
\chapitre{Arithmétique}
\sousChapitre{Arithmétique de Z}
\module{Algèbre}
\niveau{L2}
\difficulte{}

\contenu{
\texte{
Soit $p$ un entier supérieur ou égal à $2$. Montrer que~:~$(p-1)!\equiv-1\;(p)\Rightarrow$ $p$ est premier (en fait les deux phrases sont équivalentes mais en Sup, on sait trop peu de choses en arithmétique pour pouvoir fournir une démonstration raisonnablement courte de la réciproque).
}
\reponse{
Soit $p$ un entier naturel supérieur ou égal à $2$. 

Supposons que $(p-1)!\equiv-1\;(p)$. Il existe donc un entier relatif $a$ tel que $(p-1)!=-1+ap$ $(*)$.

Soit $k\in\{1,...,p-1\}$. L'égalité $(*)$ s'écrit encore $k(-\prod_{j\neq k}^{}j)+ap=1$. Le théorème de \textsc{Bezout} permet alors d'affirmer que $k$ et $p$ sont premiers entre eux. Ainsi, $p$ est premier avec tous les entiers naturels éléments de $\{1,...,p-1\}$ et donc, $p$ est un nombre premier.
}
}

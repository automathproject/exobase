\uuid{CYvV}
\exo7id{3632}
\titre{exo7 3632}
\auteur{quercia}
\organisation{exo7}
\datecreate{2010-03-10}
\isIndication{false}
\isCorrection{true}
\chapitre{Endomorphisme particulier}
\sousChapitre{Autre}
\module{Algèbre}
\niveau{L2}
\difficulte{}

\contenu{
\texte{
Soit $E =  K_n[X]$. On note $P_0 = 1$, $P_i = X(X-1)\cdots(X-i+1)$ pour $i\ge1$,
et $f_i : P  \mapsto P(i)$.
}
\begin{enumerate}
    \item \question{Montrer que $(P_0,\dots,P_n)$ est une base de $E$ et ${\cal B} = (f_0,\dots,f_n)$
    est une base de $E^*$.}
    \item \question{Décomposer la forme linéaire $P_n^*$ dans la base $\cal B$.
    (On pourra utiliser les polynômes :
    $Q_i = \prod_{1\le j\le n; j\ne i}(X-j)$)}
    \item \question{Décomposer de même les autres formes linéaires $P_k^*$.}
\reponse{
2. terme dominant $ \Rightarrow  P_n^*(Q_i) = 1$,
             donc $P_n^* = \sum_{i=0}^n \frac{f_i}{Q_i(i)}
                         = \sum_{i=0}^n \frac{(-1)^{n-i}f_i}{i!\,(n-i)!}$.

3. $P_k^* = \sum_{i=0}^k \frac{(-1)^{k-i}f_i}{i!\,(k-i)!}$.
}
\end{enumerate}
}

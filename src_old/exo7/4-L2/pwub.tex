\uuid{pwub}
\exo7id{3589}
\titre{exo7 3589}
\auteur{quercia}
\organisation{exo7}
\datecreate{2010-03-10}
\isIndication{false}
\isCorrection{true}
\chapitre{Réduction d'endomorphisme, polynôme annulateur}
\sousChapitre{Applications}
\module{Algèbre}
\niveau{L2}
\difficulte{}

\contenu{
\texte{
Soit $(M_n)$ une suite de points dans le plan, de coordonnées $(x_n,y_n)$
définies par la relation de récurrence~:
$$\left\{
\begin{array}{lllll}
x_{n+1} &= &-x_n + 2y_n\cr y_{n+1} &= &-3x_n + 4y_n.\cr
\end{array}\right.$$
}
\begin{enumerate}
    \item \question{Montrer que, quelque soit $M_0$, les points $M_n$ sont alignés.}
\reponse{Diagonaliser ${}^tM  \Rightarrow  y_n -\frac 32x_n = \text{cste}$.}
    \item \question{Étudier la suite $(M_n)$ quand $n$ tend vers l'infini.}
\reponse{$y_n-x_n = 2^n(y_0-x_0)$ donc si $y_0\ne x_0$ alors $M_n \to\infty$
sinon la suite est constante.}
    \item \question{Quelle est la limite de $y_n/x_n$ (utiliser une méthode géométrique)~?}
\reponse{$\frac32$ si $y_0\ne x_0$.}
\end{enumerate}
}

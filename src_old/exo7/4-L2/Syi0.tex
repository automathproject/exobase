\uuid{Syi0}
\exo7id{7348}
\titre{exo7 7348}
\auteur{mourougane}
\organisation{exo7}
\datecreate{2021-08-10}
\isIndication{false}
\isCorrection{false}
\chapitre{Groupe, anneau, corps}
\sousChapitre{Algèbre, corps}
\module{Algèbre}
\niveau{L2}
\difficulte{}

\contenu{
\texte{
Alice et Bernard décident d'utiliser l'algorithme d'El Gamal.
Il utilise le corps $\mathbb{F}_{19}$ avec l'élément $G=15$.
}
\begin{enumerate}
    \item \question{Déterminer l'ordre de $15$ dans $\mathbb{F}_{19}^\times$.}
    \item \question{Bernard choisit sa clé privée $c=4$. Déterminer sa clé publique $C=G^c$.}
    \item \question{Alice choisit une clé temporaire privée $d=5$. Quelle est sa clé publique $D$ ?
    Elle souhaite envoyer le message $m=17$. Elle le chiffre en utilisant la clé publique $C$ de Bernard par $(M_1,M_2)=(D,mC^d)$. Calculer ce message chiffré.}
    \item \question{Comment Bernard retrouve-t-il le message $m$ ?}
    \item \question{Dans un second envoi, Bernard reçoit $(8,3)$.
    Quel est le message $m$ envoyé cette fois par Alice ?
    Quelle clé privée a-t-elle utilisé cette fois ?}
\end{enumerate}
}

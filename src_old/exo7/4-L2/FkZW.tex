\uuid{FkZW}
\exo7id{7403}
\titre{exo7 7403}
\auteur{mourougane}
\organisation{exo7}
\datecreate{2021-08-10}
\isIndication{false}
\isCorrection{true}
\chapitre{Groupe, anneau, corps}
\sousChapitre{Autre}
\module{Algèbre}
\niveau{L2}
\difficulte{}

\contenu{
\texte{

}
\begin{enumerate}
    \item \question{On rappelle que le seul polynôme irréductible de degré $2$ sur $\mathbb{F}_2$ est $X^2+X+1$.
	Montrer que le polynôme $X^4+X^3+1$ est irréductible dans $\mathbb{F}_2[X]$.}
\reponse{Soit $P=X^2+X+1$. Le polynôme $P$ est sans racine, donc s'il
est réductible, il se factorise en deux polynômes de degré 2. Or
$(X^2+X+1)^2=X^4+X^2+1$, qui est différent de $P$. Donc $P$ est
irréductible.}
    \item \question{On note $A:=\mathbb{F}_2[X]/(X^4+X^3+1)$ l'anneau quotient de	$\mathbb{F}_2[X]$ par l'idéal engendré par $P$.	L'anneau $A$ est-il un corps ? Combien a-t-il d'éléments ?}
\reponse{L'idéal $(X^4+X^3+1)$ est premier car $X^4+X^3+1$ est
irréductible, donc il est maximal car $\mathbb{F}_2[X]$ est principal, comme
tout anneau de polynômes sur un corps. Or le quotient par un idéal
maximal est un corps. Donc $A$ est un corps. De plus, il est
isomorphe au polynôme de $\mathbb{F}_2[X]$ de degré au plus 3, donc $A$
possède 16 éléments.}
    \item \question{On note $\alpha$ la classe du polynôme $X$ dans $A$. Déterminer $\alpha^4$ et $\alpha^{15}$ comme polynômes de degré au
	plus $3$ en $\alpha$.}
\reponse{Puisque $A$ est un corps, $A^*$ contient 15 éléments, donc
l'ordre de tout élément de $A^*$ est un diviseur de 15. En
particulier, $\alpha^{15}=1$.}
    \item \question{Déterminer toutes les puissances de $\alpha$, $\alpha$ jusqu'à $\alpha^{15}$, comme polynômes de degré au plus~$3$ en $\alpha$.}
\reponse{$$\begin{array}{l}
\alpha^1=\alpha\\
\alpha^2=\alpha^2\\
\alpha^3=\alpha^3\\
\alpha^4=\alpha^3+1\\
\alpha^5=\alpha^4+\alpha=\alpha^3+\alpha+1\\
\end{array}$$
$$\begin{array}{l}
\alpha^6=\alpha^4+\alpha^2+\alpha=\alpha^3+\alpha^2+\alpha+1\\
\alpha^7=\alpha^4+\alpha^3+\alpha^2+\alpha=\alpha^2+\alpha+1\\
\alpha^8=\alpha^3+\alpha^2+\alpha\\
\alpha^9=\alpha^4+\alpha^3+\alpha^2=\alpha^2+1\\
\alpha^{10}=\alpha^3+\alpha\\
\end{array}$$
$$\begin{array}{l}
\alpha^{11}=\alpha^4+\alpha^2=\alpha^3+\alpha^2+1\\
\alpha^{12}=\alpha^4+\alpha^3+\alpha=\alpha+1\\
\alpha^{13}=\alpha^2+\alpha\\
\alpha^{14}=\alpha^3+\alpha^2\\
\alpha^{15}=1
\end{array}$$}
    \item \question{Déterminer $\alpha^7+\alpha^8+\alpha^9$ comme polynômes de degré au plus $3$ en $\alpha$.}
\reponse{$\alpha^7+\alpha^8+\alpha^9=\alpha^7(1+\alpha+\alpha^2)=\alpha^{7+7}=\alpha^{14}=\alpha^3+\alpha^2$.}
    \item \question{Ecrire l'inverse de $1+\alpha+\alpha^3$ comme	puissance de $\alpha$.}
\reponse{$(1+\alpha+\alpha^3)^{-1}=(\alpha^5)^{-1}=\alpha^{10}$.}
\end{enumerate}
}

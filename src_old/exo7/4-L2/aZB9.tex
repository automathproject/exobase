\uuid{aZB9}
\exo7id{3642}
\titre{exo7 3642}
\auteur{quercia}
\organisation{exo7}
\datecreate{2010-03-10}
\isIndication{false}
\isCorrection{true}
\chapitre{Endomorphisme particulier}
\sousChapitre{Autre}
\module{Algèbre}
\niveau{L2}
\difficulte{}

\contenu{
\texte{
Soit $E$ l'ensemble des suites $u=(u_n)$ à termes réels telles que pour tout
$n$ : $u_{n+2} = u_{n+1} + u_n$.
}
\begin{enumerate}
    \item \question{Montrer que $E$ est un $\R$-ev de dimension finie.}
    \item \question{Soient $ {f_0} : E \to \R, u \mapsto {u_0}$ et
           ${f_1} : E \to \R, u \mapsto {u_1.}$
    Trouver la base duale de $(f_0,f_1)$.}
\reponse{
2. $\left(\frac{\varphi^{n-1} - {\bar\varphi}^{n-1}}{\varphi-\bar\varphi},
               \frac{\varphi^n - {\bar\varphi}^n}{\varphi-\bar\varphi}\right)$
             avec $\varphi=\frac{1+\sqrt5}2$, $\bar\varphi=\frac{1-\sqrt5}2$.
}
\end{enumerate}
}

\uuid{zhLh}
\exo7id{3682}
\titre{exo7 3682}
\auteur{quercia}
\organisation{exo7}
\datecreate{2010-03-11}
\isIndication{false}
\isCorrection{false}
\chapitre{Espace euclidien, espace normé}
\sousChapitre{Produit scalaire, norme}
\module{Algèbre}
\niveau{L2}
\difficulte{}

\contenu{
\texte{
Soit $E$ = $\R[X]$. On pose $(P \mid Q) =  \int_{t=0}^1 P(t)Q(t)\,d t$
}
\begin{enumerate}
    \item \question{Démontrer que $(\ \mid\ )$ est un produit scalaire sur $E$.}
    \item \question{Démontrer qu'il existe une unique famille $(P_0, P_1, \dots, P_n,\dots )$ de polynômes
    vérifiant :
   $$\begin{cases}\deg P_i = i\cr
            \text{le coefficient dominant de $P_i$ est strictement positif}\cr
            \text{la famille $(P_i)$ est orthonormée.}\cr \end{cases}$$}
\end{enumerate}
}

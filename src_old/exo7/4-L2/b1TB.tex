\uuid{b1TB}
\exo7id{5635}
\titre{exo7 5635}
\auteur{rouget}
\organisation{exo7}
\datecreate{2010-10-16}
\isIndication{false}
\isCorrection{true}
\chapitre{Déterminant, système linéaire}
\sousChapitre{Forme multilinéaire}
\module{Algèbre}
\niveau{L2}
\difficulte{}

\contenu{
\texte{
Soient $A=(a_{i,j})_{1\leqslant i,j\leqslant n}$ une matrice carrée et $B= (b_{i,j})_{1\leqslant i,j\leqslant n}$ où $b_{i,j}=(-1)^{i+j}a_{i,j}$. Calculer $\text{det}(B)$ en fonction de $\text{det}(A)$.
}
\reponse{
\textbf{1ère solution.} 

\begin{align*}\ensuremath
\text{det}B&=\sum_{\sigma\in S_n}^{}\varepsilon(\sigma)b_{\sigma(1),1}...b_{\sigma(n),n}=\sum_{\sigma\in S_n}^{}\varepsilon(\sigma)(-1)^{1+2+...+n+\sigma(1)+...+\sigma(n)}a_{\sigma(1),1}...a_{\sigma(n),n}\\
 &=\sum_{\sigma\in S_n}^{}\varepsilon(\sigma)(-1)^{2(1+2+...+n)}a_{\sigma(1),1}...a_{\sigma(n),n}=\sum_{\sigma\in S_n}^{}\varepsilon(\sigma)a_{\sigma(1),1}...a_{\sigma(n),n}\\
 &=\text{det}A.
\end{align*}

\textbf{2ème solution.} On multiplie les lignes numéros $2$, $4$,... de $B$ par $-1$ puis les colonnes numéros $2$, $4$,... de la matrice obtenue par $-1$. On obtient la matrice $A$ qui se déduit donc de la matrice $B$ par multiplication des lignes ou des colonnes par un nombre pair de $-1$ (puisqu'il y a autant de lignes portant un numéro pair que de colonnes portant un numéro pair). Par suite, $\text{det}(B)=\text{det}(A)$.
}
}

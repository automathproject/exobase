\uuid{MIzv}
\exo7id{1712}
\titre{exo7 1712}
\auteur{barraud}
\organisation{exo7}
\datecreate{2003-09-01}
\isIndication{false}
\isCorrection{true}
\chapitre{Réduction d'endomorphisme, polynôme annulateur}
\sousChapitre{Applications}
\module{Algèbre}
\niveau{L2}
\difficulte{}

\contenu{
\texte{
On considère la matrice $A=
  \begin{pmatrix}
    a&-b&-c&-d\\
    b& a& d&-c\\
    c&-d& a& b\\
    d& c&-b& a  
  \end{pmatrix}
$, avec $(b,c,d)\neq(0,0,0)$.
}
\begin{enumerate}
    \item \question{Calculer $A{}^t{A}$.
  Que vaut $\det A$ au signe près~?}
\reponse{$A{}^t{A}=(a^{2}+b^{2}+c^{2}+d^{2})\mathrm{id}$. Ainsi $\det
    A*\det{}^t A=(\det A)^{2}=(a^{2}+b^{2}+c^{2}+d^{2})^{4}$ et donc
    $\det A=\pm(a^{2}+b^{2}+c^{2}+d^{2})^{2}$.}
    \item \question{En étudiant le signe du terme en $a^{4}$ dans le déterminant de $A$,
  montrer que $\det A=(a^{2}+b^{2}+c^{2}+d^{2})^{2}$. Sans calcul
  supplémentaire, en déduire que le polynôme caractéristique de $A$ est
  $\chi_{A}=((a-X)^{2}+b^{2}+c^{2}+d^{2})^{2}$.}
\reponse{Dans l'expression $\det A =\sum_{\sigma\in
     S_{4}}\epsilon(\sigma)\alpha_{1\sigma(1)}...\alpha_{4\sigma(4)}$ où
   les coefficients de $A$ sont notés $\alpha_{ij}$, le seul terme en
   $a^{4}$ est obtenu pour $\sigma=\mathrm{id}$, soit $\epsilon(\sigma)=+1$. On
   en déduit que $\det A=(a^{2}+b^{2}+c^{2}+d^{2})^{2}$. Pour obtenir le
   polynôme caractéristique de $A$, on remplace $a$ par $a-X$ dans $A$,
   et on calcule le déterminant. On a donc
   $\chi_{A}=((a-X)^{2}+b^{2}+c^{2}+d^{2})^{2}$}
    \item \question{$A$ est-elle diagonalisable sur $\R$~? (justifier)}
\reponse{$\forall\lambda\in\R, \chi_{A}(\lambda)>0$ car $(b,c,d)\neq(0,0,0)$.
   Donc $A$ n'a pas de valeur propre réelle, donc $A$ n'est ni
   diagonalisable ni triangularisable sur $\R$.}
    \item \question{On se place maintenant dans le cas où $a=1$, $b=c=d=-1$. Vérifier que
  $(i\sqrt{3},1,1,1)$ et $(-1,i\sqrt{3},-1,1)$ sont des vecteurs propres
  de $A$, puis diagonaliser $A$ sur $\C$.}
\reponse{$A(i\sqrt{3},1,1,1)=(1-i\sqrt{3}(i\sqrt{3},1,1,1))$ et
   $A(-1,i\sqrt{3},-1,1)=(1-i\sqrt{3}(-1,i\sqrt{3},-1,1))$. Pour la
   seconde valeur propre, qui est le conjugué de $1-i\sqrt{3}$, on
   utilise les vecteurs conjugués. Ainsi, en posant $ P=
 \begin{pmatrix}
   i\sqrt{3} & -1         &-i\sqrt{3} & -1 \\ 
   1         &  i\sqrt{3} & 1         & -i\sqrt{3}\\
   1         & -1         & 1         & -1        \\
   1         &  1         & 1         &  1
 \end{pmatrix}
   $ on a $P^{-1}AP=
   \begin{pmatrix}
     2\bar\omega &0          &0      &0\\
     0           &2\bar\omega&0      &0\\  
     0           &0          &2\omega&0\\
     0           &0          &0      &2\omega
\end{pmatrix}=D
$.}
    \item \question{Application~: résoudre le système récurent suivant (il n'est pas
  nécessaire de calculer l'inverse de la matrice de passage de la
  question précédente). On notera $\omega=1/2+i\sqrt{3}/2=e^{i\pi/3}$.
$$
\left\{
  \begin{array}{ccr}
    u_{n+1} &=& u_{n}+v_{n}+w_{n}+h_{n}\\
    v_{n+1} &=&-u_{n}+v_{n}-w_{n}+h_{n}\\
    w_{n+1} &=&-u_{n}+v_{n}+w_{n}-h_{n}\\
    h_{n+1} &=&-u_{n}-v_{n}+w_{n}+h_{n}
  \end{array}
\right.
\qquad
\left\{
  \begin{array}{ccr}
    u_{0} &=&1\\
    v_{0} &=&0\\
    w_{0} &=&0\\
    h_{0} &=&0
  \end{array}
\right.
$$}
\reponse{Soit $X_{n}=(u_{n},v_{n},w_{n},h_{n})$. On a $\forall n \in\N,
  X_{n+1}=AX_{n}$, d'où $\forall n\in\N, X_{n}=A^{n}X_{0}$. Or
  $A^{n}X_{0}=PD^{n}P^{-1}X_{0}$. On en déduit que $X_{n}=
  \begin{pmatrix}
   2^{n}\bar\omega^{n}i\sqrt{3}
 & -2^{n}\bar\omega^{n}         
 &-2^{n}\omega^{n}i\sqrt{3} 
 & -2^{n}\omega^{n} \\ 
   2^{n}\bar\omega^{n}         
 &  2^{n}\bar\omega^{n}i\sqrt{3} 
 & 2^{n}\omega^{n}        
 & -2^{n}\omega^{n}i\sqrt{3}\\
   2^{n}\bar\omega^{n}         
 & -2^{n}\bar\omega^{n}         
 & 2^{n}\omega^{n}        
 & -2^{n}\omega^{n}        \\
   2^{n}\bar\omega^{n}         
 &  2^{n}\bar\omega^{n}         
 & 2^{n}\omega^{n}        
 &  2^{n}\omega^{n}
  \end{pmatrix}
$. Posons $Y_{0}=P^{-1}X_{0}$. En résolvant le système $PX_{0}=Y_{0}$,
on obtient $Y_{0}=(1/2i\sqrt{3},0,-1/2i\sqrt{3},0)$, et finalement~:
$$
X_{n}=1/2i\sqrt{3}
\begin{pmatrix}
  2^{n}(\bar\omega^{n}+\omega^{n})i\sqrt{3}\\
  2^{n}(\bar\omega^{n}-\omega^{n})\\
  2^{n}(\bar\omega^{n}-\omega^{n})\\
  2^{n}(\bar\omega^{n}-\omega^{n})
\end{pmatrix}=2^{n}
\begin{pmatrix}
  \cos \frac{n\pi}{3}\\
  -\sin \frac{n\pi}{3}\\
  -\sin \frac{n\pi}{3}\\
  -\sin \frac{n\pi}{3}
\end{pmatrix}
$$}
\end{enumerate}
}

\uuid{Dv64}
\exo7id{3629}
\titre{exo7 3629}
\auteur{quercia}
\organisation{exo7}
\datecreate{2010-03-10}
\isIndication{false}
\isCorrection{true}
\chapitre{Endomorphisme particulier}
\sousChapitre{Autre}
\module{Algèbre}
\niveau{L2}
\difficulte{}

\contenu{
\texte{
Soit $E =  \R_{2n-1}[X]$, et $x_1,\dots,x_n \in \R$ distincts.
On note :
$${\phi_i} : E\to \R, P \mapsto {P(x_i)} ;
 \qquad
 {\psi_i} : E \to \R, P \mapsto {P'(x_i)}$$
}
\begin{enumerate}
    \item \question{Montrer que $(\phi_1,\dots,\phi_n,\psi_1,\dots,\psi_n)$ est une base de
    $E^*$.}
    \item \question{Chercher la base duale.
    On notera $P_i = \prod_{j\ne i}\frac{X-x_j}{x_i-x_j}$ et
    $d_i = P_i'(x_i)$.}
\reponse{
2. $\phi_i^* = (1-2d_i(X-x_i))P_i^2$,
             $\psi_i^* = (X-x_i)P_i^2$.
}
\end{enumerate}
}

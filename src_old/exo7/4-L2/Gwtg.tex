\uuid{Gwtg}
\exo7id{3584}
\titre{exo7 3584}
\auteur{quercia}
\organisation{exo7}
\datecreate{2010-03-10}
\isIndication{false}
\isCorrection{true}
\chapitre{Réduction d'endomorphisme, polynôme annulateur}
\sousChapitre{Autre}
\module{Algèbre}
\niveau{L2}
\difficulte{}

\contenu{
\texte{
Soit~$A\in\mathcal{M}_n(\C)$ nilpotente. Montrer que~$A$ et $2A$ sont semblables.
}
\reponse{
Soit $f$ un endomorphisme d'un ev $E$ ayant $A$ pour matrice.
On doit trouver $g\in GL(E)$ tel que $f\circ g = 2g\circ f$. Construction
de~$g$ par récurrence sur $n=\dim E$.

$n\le1$~: on a $f=0$ donc $g=\mathrm{id}_E$ convient.

$0,\dots,n-1 \Rightarrow n$~: $f$ est non surjectif donc l'hypothèse de récurrence
s'applique à~$f_{|\Im(f)}$. Soit $g_1\in GL(\Im(f))$ tel que
$f(g_1(x)) = 2g_1(f(x))$ pour tout~$x\in\Im(f)$. Soit
$E = H \oplus I \oplus K \oplus L$
avec $H=\Im(f)\cap\mathrm{Ker}(f)$, $H \oplus I = \Im(f)$ et $H \oplus K = \mathrm{Ker}(f)$.
La restriction de~$f$ à $I\oplus L$ induit un isomorphisme sur~$\Im(f)$,
on note $\varphi$ l'isomorphisme réciproque. Soit~$g\in\mathcal{L}(E)$ définie par~:

$$g(h+i+k+\ell) = g_1(h+i) + k + 2\varphi(g_1(f(\ell))).$$

On vérifie facilement que $f\circ g = 2g\circ f$ et il reste à prouver
que $g$ est injective. Si $x=h+i+k+\ell\in\mathrm{Ker} g$ alors
$g(f(x)) = g_1(f(i+\ell)) = 0$ donc $i+\ell\in\mathrm{Ker} f = H\oplus K$ soit
$i=\ell=0$. Il reste $g_1(h)+k=0$ ce qui implique $h=k=0$ car $g_1(h)\in \Im f
= H\oplus I$.

Remarque~: la démonstration passe à tout corps de caractéristique différente de~$2$.
}
}

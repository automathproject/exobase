\uuid{b0lx}
\exo7id{1343}
\titre{exo7 1343}
\auteur{legall}
\organisation{exo7}
\datecreate{1998-09-01}
\isIndication{false}
\isCorrection{true}
\chapitre{Groupe, anneau, corps}
\sousChapitre{Morphisme, isomorphisme}
\module{Algèbre}
\niveau{L2}
\difficulte{}

\contenu{
\texte{
D\' ecrire tous les homomorphismes de groupes de
${\Zz}$ dans  ${\Zz}$. D\' eterminer ceux qui sont injectifs et
ceux qui sont surjectifs.
}
\reponse{
Soit $f:(\Zz,+) \longrightarrow (\Zz,+)$ un morphisme de groupe.
Comme tout morphisme $f$ v\'erifie $f(0)=0$. Notons $a = f(1)$.
Alors
$$f(2) = f(1+1) = f(1)+f(1) = a+a = 2.a.$$
De m\^eme, pour $n \ge 0$ :
$$f(n) = f(1+\cdots+1)= f(1)+\cdots+f(1) = n.f(1) = n.a.$$
Enfin comme
$$0= f(0) = f(1+(-1)) = f(1) + f(-1) = a + f(-1),$$
alors $f(-1) = -a$ et pour tout $n \in \Zz$ :
$$f(n) = n.a.$$
Donc tous les morphisme sont de la forme $n \mapsto n.a$, avec
$a\in\Zz$.

\bigskip

Un morphisme $n \mapsto n.a$ est injectif si et seulement si $a
\not= 0$, et surjectif si et seulement si $n = \pm 1$.
}
}

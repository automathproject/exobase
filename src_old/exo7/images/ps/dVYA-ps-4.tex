%%%%%%%%%%%%%%%%%% PREAMBULE %%%%%%%%%%%%%%%%%%

\documentclass[12pt,a4paper]{article}

\usepackage{amsfonts,amsmath,amssymb,amsthm}
\usepackage[francais]{babel}
\usepackage[utf8]{inputenc}
\usepackage[T1]{fontenc}

%----- Ensemles : entiers, reels, complexes -----
\newcommand{\Nn}{\mathbb{N}} \newcommand{\N}{\mathbb{N}}
\newcommand{\Zz}{\mathbb{Z}} \newcommand{\Z}{\mathbb{Z}}
\newcommand{\Qq}{\mathbb{Q}} \newcommand{\Q}{\mathbb{Q}}
\newcommand{\Rr}{\mathbb{R}} \newcommand{\R}{\mathbb{R}}
\newcommand{\Cc}{\mathbb{C}} \newcommand{\C}{\mathbb{C}}

%----- Modifications de symboles -----
\renewcommand {\epsilon}{\varepsilon}
\renewcommand {\Re}{\mathop{\mathrm{Re}}\nolimits}
\renewcommand {\Im}{\mathop{\mathrm{Im}}\nolimits}

%----- Fonctions usuelles -----
\newcommand{\ch}{\mathop{\mathrm{ch}}\nolimits}
\newcommand{\sh}{\mathop{\mathrm{sh}}\nolimits}
\renewcommand{\tanh}{\mathop{\mathrm{th}}\nolimits}
\newcommand{\Arcsin}{\mathop{\mathrm{Arcsin}}\nolimits}
\newcommand{\Arccos}{\mathop{\mathrm{Arccos}}\nolimits}
\newcommand{\Arctan}{\mathop{\mathrm{Arctan}}\nolimits}
\newcommand{\Argsh}{\mathop{\mathrm{Argsh}}\nolimits}
\newcommand{\Argch}{\mathop{\mathrm{Argch}}\nolimits}
\newcommand{\Argth}{\mathop{\mathrm{Argth}}\nolimits}
\newcommand{\pgcd}{\mathop{\mathrm{pgcd}}\nolimits} 

%----- Commandes special dessin a ajouter localement ------
\usepackage{geometry}
\usepackage{pstricks}
\usepackage{pst-plot}
\usepackage{pst-node}
\usepackage{graphics,epsfig}

\pagestyle{empty}

% Que faire avec ce fichier monimage.tex ?
%   1/ latex monimage.tex
%   2/ dvips monimage.dvi
%   3/ ps2eps monimage.ps
%   4/ ps2pdf -dEPSCrop monimage.eps
%   5/ Dans votre fichier d'exos \includegraphics{monimage}

\begin{document}

\begin{pspicture}(-8.1,-2.5)(0.1,4)
\psset{xunit=0.6cm,yunit=0.6cm}
\psaxes{->}(0,0)(-7.2,-4.3)(7.2,6.5)
\psline[linestyle=dashed](4.64,4)(4.64,-3)
\psline[linestyle=dashed](-4.64,4)(-4.64,-3)
\psplot[plotpoints=10000]{4.7}{8}{x 120 60 x mul add 12 x dup mul mul add x x mul x mul add 120 60 x mul sub 12 x dup mul mul add x x mul x mul sub div abs ln sub}
\psplot[plotpoints=10000]{-4.5}{4.5}{x 120 60 x mul add 12 x dup mul mul add x x mul x mul add 120 60 x mul sub 12 x dup mul mul add x x mul x mul sub div abs ln sub}
\psplot[plotpoints=10000]{-8}{-4.7}{x 120 60 x mul add 12 x dup mul mul add x x mul x mul add 120 60 x mul sub 12 x dup mul mul add x x mul x mul sub div abs ln sub}
\psplot[plotpoints=10000]{-6}{6}{x}
\uput[ur](6,6){$y=x$}
\uput[r](8,5.2){$y=f_3(x)$}
\end{pspicture}

\end{document}

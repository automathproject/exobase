\uuid{ibKF}
\exo7id{4408}
\titre{exo7 4408}
\auteur{quercia}
\organisation{exo7}
\datecreate{2010-03-12}
\isIndication{false}
\isCorrection{true}
\chapitre{Série numérique}
\sousChapitre{Fonction exponentielle complexe}
\module{Analyse}
\niveau{L1}
\difficulte{}

\contenu{
\texte{
Montrer qu'il existe une infinité de complexes $z$ tels que $e^z = z$
(on calculera $x$ en fonction de $y$, et on étudiera l'équation obtenue).
}
\reponse{
$e^{x+iy} = x+iy \Leftrightarrow \begin{cases}x = y/\tan y \cr e^{-y/\tan y}
         = \sin y/y.\cr\end{cases}$
         Au voisinage de $2k\pi^+$, $e^{-y/\tan y} < \sin y/y$ (point plat)
         et au voisinage de $(2k+1)\pi^-$, $e^{-y/\tan y} > \sin y/y$
         (limite infinie).
}
}

\uuid{vq3q}
\exo7id{5706}
\titre{exo7 5706}
\auteur{rouget}
\organisation{exo7}
\datecreate{2010-10-16}
\isIndication{false}
\isCorrection{true}
\chapitre{Série numérique}
\sousChapitre{Autre}
\module{Analyse}
\niveau{L1}
\difficulte{}

\contenu{
\texte{
Nature de la série de terme général $u_n=\sum_{k=1}^{n}\frac{1}{(n+k)^p}$, $p\in]0,+\infty[$.
}
\reponse{
Pour tout entier naturel non nul $n$, $0<\frac{1}{2^pn^{p-1}}=\sum_{k=1}^{n}\frac{1}{(2n)^p}\leqslant\sum_{k=1}^{n}\frac{1}{(n+k)^p}\leqslant\sum_{k=1}^{n}\frac{1}{n^p}=\frac{1}{n^{p-1}}$ et la série de terme général $u_n$ converge si et seulement si $p > 2$.
}
}

\uuid{PlGa}
\exo7id{4698}
\titre{exo7 4698}
\auteur{quercia}
\organisation{exo7}
\datecreate{2010-03-16}
\isIndication{false}
\isCorrection{true}
\chapitre{Suite}
\sousChapitre{Convergence}
\module{Analyse}
\niveau{L1}
\difficulte{}

\contenu{
\texte{
Soit $(a_n)$ une suite de r{\'e}els sup{\'e}rieurs ou {\'e}gaux {\`a} $1$ telle que pour tous $n,m$, 
$a_{n+m}\le a_n\, a_m$. On pose $b_n=\frac{\ln a_n}{n}\cdotp$
Montrer que $(b_n)$ converge vers $\inf \{b_n\,|\, n\in\N^*\}$.
}
\reponse{
Soit $\ell =\inf \{b_n\,|\, n\in\N^*\}$, $\varepsilon>0$
et $p\in\N^*$ tel que $b_p\le \ell+\varepsilon$. Pour $n\in\N^*$ on
effectue la division euclidienne de $n$ par~$p$~: $n=pq+r$ d'o{\`u}
$a_n\le a_p^qa_r$ et $b_n\le b_p+\frac{\ln a_r}{\strut n}\le \ell+2\varepsilon$
pour $n$ assez grand.
}
}

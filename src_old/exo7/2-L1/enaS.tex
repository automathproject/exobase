\uuid{enaS}
\exo7id{5417}
\titre{exo7 5417}
\auteur{rouget}
\organisation{exo7}
\datecreate{2010-07-06}
\isIndication{false}
\isCorrection{true}
\chapitre{Dérivabilité des fonctions réelles}
\sousChapitre{Autre}
\module{Analyse}
\niveau{L1}
\difficulte{}

\contenu{
\texte{
Soit $f$ une fonction dérivable sur un intervalle ouvert $I$  à valeurs dans $\Rr$. Soient $a$ et $b$ deux points distincts de $I$ vérifiant $f'(a)<f'(b)$ et soit enfin un réel $m$ tel que $f'(a)<m<f'(b)$.
}
\begin{enumerate}
    \item \question{Montrer qu'il existe $h>0$ tel que $\frac{f(a+h)-f(a)}{h}<m<\frac{f(b+h)-f(b)}{h}$.}
\reponse{Soit $m$ un élément de $]f'(a),f'(b)[$. Puisque $\lim_{h\rightarrow 0}\frac{f(a+h)-f(a)}{h}=f'(a)$ et que 
$\lim_{h\rightarrow 0}\frac{f(b+h)-f(b)}{h}=f'(b)$, on a (en prenant par exemple $\varepsilon=\mbox{Min}\{m-f'(a),f'(b)-m\}>0$) 

$$\begin{array}{l}
\exists h_1>0/\;\forall h\in]0,h_1[,\;(a+h\in I\Rightarrow\frac{f(a+h)-f(a)}{h}<m\;\mbox{et}\\
\exists h_2>0/\;\forall h\in]0,h_2[\;(b+h\in I\Rightarrow\frac{f(b+h)-f(b)}{h}> m.
\end{array}$$

L'ensemble $E=\{h\in]0,\mbox{Min}\{h_1,h_2\}[/\;a+h\;\mbox{et}\;b+h\;\mbox{sont dans}\;I\}$ n'est pas vide (car $I$ est ouvert) et pour tous les $h$ de $E$, on a~:$\frac{f(a+h)-f(a)}{h}<m<\frac{f(b+h)-f(b)}{h}$.

$h>0$ est ainsi dorénavant fixé.}
    \item \question{Montrer qu'il existe $y$ dans $[a,b]$ tel que $m=\frac{f(y+h)-f(y)}{h}$ puis qu'il exsite $x$ tel que $f'(x)=m$.}
\reponse{La fonction $f$ est continue sur $I$ et donc, la fonction $g~:~x\mapsto\frac{f(x+h)-f(x)}{h}$ est continue sur $[a,b]$. D'après le théorème des valeurs intermédiaires, comme $g(a)<m<g(b)$, $\exists y\in[a,b]/\;g(y)=m$ ou encore $\exists y\in[a,b]/\;\frac{f(y+h)-f(y)}{h}=m$.

Maintenant, d'après le théorème des accroissements finis, $\exists x\in]y,y+h[\subset I/\;m=\frac{f(y+h)-f(y)}{h}=f'(x)$.

Donc une fonction dérivée n'est pas nécessairement continue mais vérifie tout de même le théorème des valeurs intermédiaires (Théorème de \textsc{Darboux}).}
\end{enumerate}
}

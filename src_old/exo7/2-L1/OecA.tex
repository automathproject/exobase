\uuid{OecA}
\exo7id{4261}
\titre{exo7 4261}
\auteur{quercia}
\organisation{exo7}
\datecreate{2010-03-12}
\isIndication{false}
\isCorrection{true}
\chapitre{Calcul d'intégrales}
\sousChapitre{Autre}
\module{Analyse}
\niveau{L1}
\difficulte{}

\contenu{
\texte{
Soit $E = \mathcal{C}([a,b])$, et $F=\{f\in \mathcal{C}^2([a,b]),\text{ tel que } f(a)=f'(a)=f(b)=f'(b)=0\}$.
}
\begin{enumerate}
    \item \question{Soit $f\in E$. Montrer qu'il existe $g\in F$ vérifiant $g''=f$ si et
    seulement si $ \int_{x=a}^b f(x)\,d x =  \int_{x=a}^b xf(x)\,d x = 0$.}
\reponse{Il existe toujours une unique fonction $g$ de classe $\mathcal{C}^2$
    telle que $g'' = f$, $g(a)=g'(a)=0$~: $g(x) =  \int_{t=a}^x(x-t)f(t)\,d t$
    (Taylor-Intégral).}
    \item \question{Soit $f\in E$ telle que $ \int_{x=a}^b f(x)g''(x)\,d x = 0$ pour toute
    fonction $g\in F$. Montrer que $f$ est affine.}
\reponse{Soient $\lambda,\mu\in \R$ tels que $f_1\ :\ x \mapsto f(x) - \lambda-\mu x$
    vérifie $ \int_{x=a}^b f_1(x)\,d x =  \int_{x=a}^b xf_1(x)\,d x = 0$.
    On trouve $$\left\{\begin{array}{ll}(b-a)\lambda +(b^2-a^2)/2\mu &= - \int_{x=a}^bf(x)\,d x\cr
                      (b^2-a^2)/2\lambda + (b^3-a^3)/3\mu &= - \int_{x=a}^bxf(x)\,d x\cr\end{array}\right.$$
    et ce système a pour déterminant $(b-a)^4/12\ne 0$ donc $\lambda,\mu$
    existent et sont uniques. Soit $g_1\in F$ telle que $g_1''=f_1$~:
    $ \int_{x=a}^bg_1''(x)g''(x)\,d x = 0$ pour tout $g\in F$, en particulier
    pour $g=g_1$ donc $g_1''=f_1=0$ et $f(x) = \lambda + \mu x$.}
\end{enumerate}
}

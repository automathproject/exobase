\uuid{VGpH}
\exo7id{5717}
\titre{exo7 5717}
\auteur{rouget}
\organisation{exo7}
\datecreate{2010-10-16}
\isIndication{false}
\isCorrection{true}
\chapitre{Calcul d'intégrales}
\sousChapitre{Intégrale impropre}
\module{Analyse}
\niveau{L1}
\difficulte{}

\contenu{
\texte{
Deux calculs de $I =\int_{0}^{\pi/2}\ln(\sin x)\;dx$.

\textbf{1) (** I)} En utilisant $J=\int_{0}^{\pi/2}\ln(\cos x)\;dx$, calculer $I$ (et $J$).

\textbf{2) (*** I)} Calculer $P_n=\prod_{k=1}^{n-1}\sin\frac{k\pi}{2n}$ (commencer par $P_n^2$) et en déduire $I$.
}
\reponse{
Soient $I=\int_{0}^{\pi/2}\ln(\sin x)\;dx$ et $J =\int_{0}^{\pi/2}\ln(\cos x)\;dx$. Le changement de variables $x =\frac{\pi}{2}-t$ fournit $J$ existe et $J=I$. Par suite,

\begin{align*}\ensuremath
2I&= I+J =\int_{0}^{\pi/2}\ln(\sin x\cos x)\;dx=-\frac{\pi\ln2}{2}+\int_{0}^{\pi/2}\ln(\sin(2x))\;dx =-\frac{\pi\ln2}{2}+\frac{1}{2}\int_{0}^{\pi}\ln(\sin u)\;du\\
 &= -\frac{\pi\ln2}{2}+\frac{1}{2}\left(I+\int_{\pi/2}^{\pi}\ln(\sin u)\;du\right)
= -\frac{\pi\ln2}{2}+\frac{1}{2}\left(I+\int_{\pi/2}^{0}\ln(\sin (\pi-t))\;(-dt)\right)=-\frac{\pi\ln2}{2}+ I.
\end{align*}

Par suite, $I =-\frac{\pi\ln2}{2}$.

\begin{center}
\shadowbox{
$\int_{0}^{\pi/2}\ln(\sin x)\;dx=\int_{0}^{\pi/2}\ln(\cos x)\;dx=-\frac{\pi\ln2}{2}$.
}
\end{center}
Pour $n\geqslant2$, posons $P_n =\prod_{k=1}^{n-1}\sin\left(\frac{k\pi}{2n}\right)$. Pour $1\leqslant k\leqslant n-1$, on a $0<\frac{k\pi}{2n}<\frac{\pi}{2}$ et donc $P_n > 0$. D'autre part, 
$\sin\left(\frac{(2n-k)\pi}{2n}\right)=\sin\left(\frac{k\pi}{2n}\right)$ et $\sin\frac{n\pi}{2n}=1$. On en déduit que

\begin{center}
$P_n^2=\prod_{k=1}^{2n-1}\sin\left(\frac{k\pi}{2n}\right)$,
\end{center}

puis

\begin{align*}\ensuremath
P_n^2&=\prod_{k=1}^{2n-1}\frac{e^{ik\pi/(2n)}-e^{-ik\pi/(2n)}}{2i}=\frac{1}{(2i)^{2n-1}}\prod_{k=1}^{2n-1}\left(-e^{-ik\pi/(2n)}\right)\prod_{k=1}^{2n-1}\left(1-e^{2ik\pi/(2n)}\right)\\
 &=\frac{1}{(2i)^{2n-1}}(-1)^{2n-1}(e^{-i\pi/2})^{2n-1}\prod_{k=1}^{2n-1}\left(1-e^{2ik\pi/(2n)}\right)=   \frac{1}{2^{2n-1}}\prod_{k=1}^{2n-1}\left(1-e^{2ik\pi/(2n)}\right)
\end{align*}

Maintenant, le polynôme $Q$ unitaire de degré $2n-1$ dont les racines sont les $2n-1$ racines $2n$-èmes de l'unité distinctes de $1$ est

\begin{center}
$\frac{X^{2n}-1}{X-1}= 1+X+X^2+...+X^{2n-1}$
\end{center}

et  donc $\prod_{k=1}^{2n-1}\left(1-e^{2ik\pi/(2n)}\right)=Q(1)=2n$. Finalement,

\begin{center}
$\prod_{k=1}^{n-1}\sin\left(\frac{k\pi}{2n}\right)=P_n=\sqrt{\frac{2n}{2^{2n-1}}}=\frac{\sqrt{n}}{2^{n-1}}$.
\end{center}

Pour $0\leqslant k\leqslant n$, posons alors $x_k=\frac{k\pi}{2n}$ de sorte que $0 = x_0 < x_1< ...< x_n =\frac{\pi}{2}$ est une subdivision de $\left[0,\frac{\pi}{2}\right]$
à pas constant égal à $\frac{\pi}{2n}$.

Puisque la fonction $x\mapsto\ln(\sin x)$ est continue et croissante sur $\left]0,\frac{\pi}{2}\right]$, pour $1\leqslant k\leqslant n-1$, on a  $\frac{\pi}{2n}\ln(\sin(x_k))\leqslant\int_{x_k}^{x_{k+1}}\ln(\sin x)\;dx$ puis en sommant  ces inégalités, on obtient 

\begin{center}
$\frac{\pi}{2n}\ln(P_n)\leqslant\int_{\pi/(2n)}^{\pi/2}\ln(\sin x)\;dx$
\end{center}

De même, pour $0\leqslant k\leqslant n-1$,  $\int_{x_k}^{x_{k+1}}\ln(\sin x)\;dx\leqslant\frac{\pi}{2n}\ln(\sin(x_{k+1}))$ et en sommant  

\begin{center}
$\int_{0}^{\pi/2}\ln(\sin x)\;dx\leqslant\frac{\pi}{2n}\ln(P_n)$.
\end{center}

Finalement,  $\forall n\geqslant2$, $\frac{\pi}{2n}\ln(P_n)+\int_{0}^{\pi/(2n)}\ln(\sin x)\;dx\leqslant I\leqslant\frac{\pi}{2n}\ln(P_n)$. Mais $\ln(P_n) =\ln n -(n-1)\ln2$ et donc  $\frac{\pi}{2n}\ln(P_n)$ tend vers $-\frac{\pi\ln2}{2}$ quand $n$ tend vers $+\infty$ et comme d'autre part, $\int_{0}^{\pi/(2n)}\ln(\sin x)\;dx$ tend vers $0$ quand $n$ tend vers $+\infty$ (puisque la fonction $x~:~\mapsto\ln(\sin x)$ est intégrable sur $\left]0,\frac{\pi}{2}\right]$), on a redémontré que $I = -\frac{\pi\ln2}{2}$.
}
}

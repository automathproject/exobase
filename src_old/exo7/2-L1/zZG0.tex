\uuid{zZG0}
\exo7id{616}
\titre{exo7 616}
\auteur{vignal}
\organisation{exo7}
\datecreate{2001-09-01}
\video{OYJj7QRtecs}
\isIndication{true}
\isCorrection{true}
\chapitre{Continuité, limite et étude de fonctions réelles}
\sousChapitre{Limite de fonctions}
\module{Analyse}
\niveau{L1}
\difficulte{}

\contenu{
\texte{
Calculer lorsqu'elles existent les limites suivantes
$$\begin{array}{lll}
  a)\ \lim_{x\rightarrow 0}\frac{x^2+2\,|x|}{x} &\quad  b)\
\lim_{x\rightarrow -\infty}\frac{x^2+2\,|x|}{x}&
\quad c) \ \lim_{x\rightarrow 2}\frac{x^2-4}{x^2-3\,x+2}\\
\\
d) \ \lim_{x\rightarrow\pi}\frac{\sin^2x}{1+\cos x}&
\quad  e)\ \lim_{x\rightarrow 0}\frac{\sqrt{1+x}-\sqrt{1+x^2}}{x}&
\quad  f)\ \lim_{x\rightarrow +\infty}\sqrt{x+5}-\sqrt{x-3}\\
\\
  g)\ \lim_{x\rightarrow 0}\frac{\sqrt[3]{1+x^2}-1}{x^2}&
\quad  h)\ \lim_{x\rightarrow 1}\frac{x-1}{x^n-1}
\end{array}$$
}
\indication{Réponses :
\begin{enumerate}
  \item La limite \`a droite vaut $+2$, la limite \`a gauche $-2$ donc il n'y a pas de limite.
  \item $-\infty$
  \item $4$
  \item $2$
  \item $\frac 12$
  \item $0$
  \item $\frac 13$ en utilisant par exemple que $a^3-1 = (a-1)(1+a+a^2)$ pour $a = \sqrt[3]{1+x^2}$.
  \item $\frac 1n$
\end{enumerate}}
\reponse{
$\frac{x^2+2|x|}{x} = x + 2 \frac{|x|}{x}$.
Si $x > 0$ cette expression vaut $x+2$ donc la limite à droite en $x=0$ est $+2$.
Si $x<0$ l'expression vaut $-2$ donc la limite à gauche en $x=0$ est $-2$.
Les limites \`a droite et à gauche sont différentes donc il n'y a pas de limite en $x=0$.
$\frac{x^2+2|x|}{x} = x + 2 \frac{|x|}{x} = x -2$ pour $x<0$.
Donc la limite quand $x \to -\infty$ est $-\infty$.
$\frac{x^2-4}{x^2-3\,x+2}=\frac{(x-2)(x+2)}{(x-2)(x-1)} = \frac{x+2}{x-1}$,
lorsque $x\to 2$ cette expression tend vers $4$.
$\frac{\sin^2 x}{1+\cos x} = \frac{1-\cos^2 x}{1+\cos x} = \frac{(1-\cos x)(1+\cos x)}{1+\cos x} = 1-\cos x$.
Lorsque $x\to \pi$ la limite est donc $2$.
$\frac{\sqrt{1+x}-\sqrt{1+x^2}}{x} = \frac{\sqrt{1+x}-\sqrt{1+x^2}}{x}\times\frac{\sqrt{1+x}+\sqrt{1+x^2}}{\sqrt{1+x}+\sqrt{1+x^2}}
= \frac{1+x - (1+x^2)}{x(\sqrt{1+x}+\sqrt{1+x^2})} = \frac{x - x^2}{x(\sqrt{1+x}+\sqrt{1+x^2})} = \frac{1-x}{\sqrt{1+x}+\sqrt{1+x^2}}$.
Lorsque $x\to 0$ la limite vaut $\frac 12$.
$\sqrt{x+5}-\sqrt{x-3} = \left( \sqrt{x+5}-\sqrt{x-3}\right) \times \frac{\sqrt{x+5}+\sqrt{x-3}}{\sqrt{x+5}+\sqrt{x-3}}
= \frac{x+5-(x-3)}{\sqrt{x+5}+\sqrt{x-3}} = \frac{8}{\sqrt{x+5}+\sqrt{x-3}}$. Lorsque $x \to +\infty$, la limite vaut $0$.
Nous avons l'égalité $a^3-1 = (a-1)(1+a+a^2)$. Pour $a = \sqrt[3]{1+x^2}$
cela donne :
$$\frac{a-1}{x^2} = \frac{a^3-1}{x^2(1+a+a^2)} = \frac{1+x^2-1}{x^2(1+a+a^2)} = \frac{1}{1+a+a^2}.$$
Lors que $x\to 0$, alors $a \to 1$ et la limite cherchée est $\frac 13$.

Autre méthode : si l'on sait que la limite d'un taux d'accroissement correspond à la dérivée nous avons une méthode moins
astucieuse. Rappel (ou anticipation sur un prochain chapitre) : pour une fonction $f$ dérivable en $a$ alors
$$\lim_{x\to a} \frac{f(x)-f(a)}{x-a} = f'(a).$$
Pour la fonction $f(x) = \sqrt[3]{1+x} =(1+x)^{\frac 13}$ ayant $f'(x) = \frac 13 (1+x)^{-\frac 23}$ cela donne en $a=0$ :
$$\lim_{x\to 0} \frac{\sqrt[3]{1+x^2} -1}{x^2} = \lim_{x\to 0} \frac{\sqrt[3]{1+x} -1}{x} = \lim_{x\to 0} \frac{f(x)-f(0)}{x-0} = f'(0) = \frac 13.$$
$\frac{x^n-1}{x-1} = 1+x+x^2+\cdots + x^n$. Donc si $x\to 1$ la limite de  $\frac{x^n-1}{x-1}$ est  $n$.
Donc la limite de $\frac{x-1}{x^n-1}$ en $1$ est $\frac 1n$.

La méthode avec le taux d'accroissement fonctionne aussi très bien ici. Soit $f(x) = x^n$, $f'(x)=nx^{n-1}$ et $a=1$.
Alors $\frac{x^n-1}{x-1} = \frac{f(x)-f(1)}{x-1}$ tend vers $f'(1)=n$.
}
}

\uuid{LYZO}
\exo7id{5277}
\titre{exo7 5277}
\auteur{rouget}
\organisation{exo7}
\datecreate{2010-07-04}
\isIndication{false}
\isCorrection{true}
\chapitre{Suite}
\sousChapitre{Suite définie par une relation de récurrence}
\module{Analyse}
\niveau{L1}
\difficulte{}

\contenu{
\texte{
On pose $u_0=1$, $v_0=0$, puis, pour $n\in\Nn$, $u_{n+1}=2u_n+v_n$ et $v_{n+1}=u_n+2v_n$.
}
\begin{enumerate}
    \item \question{Soit $A=\left(
\begin{array}{cc}
2&1\\
1&2
\end{array}
\right)$. Pour $n\in\Nn$, calculer $A^n$. En déduire $u_n$ et $v_n$ en fonction de $n$.}
\reponse{Posons $J=\left(
\begin{array}{cc}
1&1\\
1&1
\end{array}
\right)
$ de sorte que $A=I+J$. On a $J^2=2j$ et donc, plus généralement~:~$\forall k\geq1,\;J^k=2^{k-1}J$. Mais alors, puisque $I$ et $J$ commutent, la formule du binôme de \textsc{Newton} fournit pour $n$ entier naturel non nul donné~:

\begin{align*}\ensuremath
A^n&=(I+J)^n=I+\sum_{k=1}^{n}C_n^kJ^k=I+(\sum_{k=1}^{n}C_n^k2^{k-1})J=I+\frac{1}{2}(\sum_{k=0}^{n}C_n^k2^k-1)J\\
 &=I+\frac{1}{2}((1+2)^n-1)J=I+\frac{1}{2}(3^n-1)J=\frac{1}{2}\left(
\begin{array}{cc}
3^n+1&3^n-1\\
3^n-1&3^n+1
\end{array}
\right)
\end{align*}

ce qui reste vrai pour $n=0$. Donc,

$$\forall n\in\Nn,\;A^n=\frac{1}{2}\left(
\begin{array}{cc}
3^n+1&3^n-1\\
3^n-1&3^n+1
\end{array}
\right).$$

Poour $n$ entier naturel donné, posons $X_n=\left(
\begin{array}{c}
u_n\\
v_n
\end{array}
\right)$. Pour tout entier naturel $n$, on a alors $X_{n+1}=A.X_n$ et donc,

\begin{align*}\ensuremath
X_n=A^n.X_0=\frac{1}{2}\left(
\begin{array}{cc}
3^n+1&3^n-1\\
3^n-1&3^n+1
\end{array}
\right)\left(
\begin{array}{c}
1\\
0
\end{array}
\right)=\left(
\begin{array}{c}
\frac{3^n+1}{2}\\
\frac{3^n-1}{2}
\end{array}
\right).
\end{align*}

Donc,

$$\forall n\in\Nn,\;u_n=\frac{3^n+1}{2}\;\mbox{et}\;v_n=\frac{3^n-1}{2}.$$}
    \item \question{En utilisant deux combinaisons linéaires intéressantes des suites $u$ et $v$, calculer directement $u_n$ et $v_n$ en fonction de $n$.}
\reponse{Soit $n\in\Nn$. $u_{n+1}+v_{n+1}=3(u_n+v_n)$. Donc, la suite $u+v$ est une suite géométrique de raison $3$ et de premier terme $u_0+v_0=1$. On en déduit que 

$$\forall n\in\Nn,\;u_n+v_n=3^n\;(I).$$

De même, pour tout entier naturel $n$ $u_{n+1}-v_{n+1}=u_n-v_n$. Donc, la suite $u+v$ est une suite constante. Puisque $u_0-v_0=1$, on en déduit que

$$\forall n\in\Nn,\;u_n-v_n=1\;(II).$$

En additionnant et en retranchant $(I)$ et $(II)$, on obtient

$$\forall n\in\Nn,\;u_n=\frac{3^n+1}{2}\;\mbox{et}\;v_n=\frac{3^n-1}{2}.$$}
\end{enumerate}
}

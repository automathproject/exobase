\uuid{dKch}
\exo7id{1201}
\titre{exo7 1201}
\auteur{legall}
\organisation{exo7}
\datecreate{2003-10-01}
\isIndication{false}
\isCorrection{false}
\chapitre{Suite}
\sousChapitre{Convergence}
\module{Analyse}
\niveau{L1}
\difficulte{}

\contenu{
\texte{
Une suite $(u_n)_{n\in \Nn }$ est dite de Cauchy 
lorsque, pour tout $\epsilon >0$ il existe $N\in \Nn $ tel que, si
$m,n \geq N$ alors $\vert u_n-u_m \vert < \epsilon .$
}
\begin{enumerate}
    \item \question{Montrer que toute suite convergente est de Cauchy. Montrer que 
toute suite de Cauchy est born\'ee.}
    \item \question{Soit $u_n = 1 + \dfrac12 + \ldots + \dfrac1n$.
  Montrer que, pour tout $p\in \Nn ,$ $u_{2^p }\geq \dfrac {p+2}2 .$ 
En d\'eduire que $(u_n)_{n\in \Nn}$ tend vers l'infini.}
    \item \question{Une suite $(u_n)_{n\in \Nn }$ satisfait au crit\`ere $\cal{C 
'}$ lorsque, pour tout $\epsilon >0$ il existe $N\in \Nn $ tel 
que, si
$n \geq N$ alors $\vert u_n-u_{n+1} \vert < \epsilon .$ Une suite 
satisfaisant au crit\`ere $\cal{C '}$ est-elle de Cauchy~?}
    \item \question{Montrer que les trois assertions qui suivent sont \'equivalentes~:
\begin{enumerate}}
    \item \question{Toute partie major\'ee de $\Rr $ admet une borne sup\'erieure 
et toute partie minor\'ee de $\Rr $ admet une borne inf\'erieure.}
    \item \question{Toute suite de Cauchy est convergente.}
    \item \question{Deux suites adjacentes sont convergentes.}
\end{enumerate}
}

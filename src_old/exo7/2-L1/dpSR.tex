\uuid{dpSR}
\exo7id{5923}
\titre{exo7 5923}
\auteur{tumpach}
\organisation{exo7}
\datecreate{2010-11-11}
\isIndication{false}
\isCorrection{true}
\chapitre{Calcul d'intégrales}
\sousChapitre{Théorie}
\module{Analyse}
\niveau{L1}
\difficulte{}

\contenu{
\texte{
Montrer que, si $f~:[a,b] \rightarrow \mathbb{R}$ est une fonction
int\'egrable au sens de Riemann, on a~:
\begin{equation*}
\frac{1}{b-a}\int_{a}^{b}f(t)\,dt=\lim_{n\rightarrow+\infty}\frac{1}{n}\sum_{k=1}^{n}f\left(a+k\frac{b-a}{n}\right).
\end{equation*}
En d\'eduire les limites suivantes~:
\begin{equation*}
a)\quad\lim_{n\rightarrow+\infty}
\frac{1}{n}\sum_{k=1}^{n}\tan\frac{k}{n} \quad\quad\quad
b)\quad\lim_{n\rightarrow+\infty}\sum_{k=1}^{n}\frac{n}{n^2+k^2}
\quad\quad\quad
c)\quad\lim_{n\rightarrow+\infty}\sum_{k=1}^{n}\log\left(\frac{n}{n+k}\right)^{\frac{1}{n}}
\end{equation*}
}
\reponse{
Soit $f~:[a,b] \rightarrow \mathbb{R}$ une fonction int\'egrable
au sens de Riemann. 

Notons $x_k = a+k\frac{b-a}{n}$, $k=1,\ldots,n$ les points où

Soit $a_0=a$, $a_{n+1}=b$ et $a_k = a+\frac{2k+1}{2n}$ pour $k=1,\ldots,n$.

Consid\'erons la subdivision 
$\sigma=\{a_{0}=a< \cdots<a_k<\cdots < a_{n}=b\}$ de $[a,b]$. 
Cette subdivision est presque régulière, seul le premier intervalle et le dernier
ont des longueurs différentes. Pour $k=1,\ldots,n-1$, $x_k$
est le milieu de $]a_k,a_{k+1}[$.

Notons 
$m_{k} = \inf\{ f(x), x\in ]a_{k-1},a_k[\}$ et
$M_k = \sup\{ f(x), x\in ]a_{k-1},a_k[\}$. 

Donc pour $k=1,\ldots,n-1$ on a $m_k \le f(x_k) \le M_k$.
Mais il faut aussi tenir compte de $f(x_n)=f(b)$ et des premiers et derniers intervalles.
D'où pour la minoration:
\begin{equation*}
\underline{S}_{f}^{\sigma} = (m_0+m_n)\frac{b-a}{2n} + \frac{b-a}{n} \sum_{k=1}^{n} m_k
\le (m_0+m_n)\frac{b-a}{2n} + \frac{b-a}{n} \sum_{k=1}^{n-1} f(x_k).
\end{equation*}
Cela donne 
\begin{equation*}
 \underline{S}_{f}^{\sigma} - (m_0+m_n+2f(b)) \frac{b-a}{2n} \le  \frac{b-a}{n}\sum_{k=1}^{n} f(x_k).
\end{equation*}

Quand $n$ tend vers $+\infty$ on trouve que $\underline{S}_{f}^{\sigma} \to \int_a^b f$
et $(m_0+m_n+2f(b)) \frac{b-a}{2n}\to 0$ cela donne l'inégalité :
$$\int_a^b f \le \lim_{n\rightarrow+\infty} \frac{b-a}{n}\sum_{k=1}^{n} f(x_k).$$

La somme $\overline{S}_{f}^{\sigma}$ conduit de manière similaire à l'inégalité inverse,
d'o\`u :
\begin{equation*}
\int_{a}^{b} f(x)\,dx =
\lim_{n\rightarrow+\infty}\left(\frac{b-a}{n}\right)
\sum_{k=1}^{n}f\left(a+k\frac{b-a}{n}\right).
\end{equation*}
On a~:
\begin{eqnarray*}
a)\quad\lim_{n\rightarrow+\infty}
\frac{1}{n}\sum_{k=1}^{n}\tan\frac{k}{n} = -\log(\cos 1)
\quad\quad\quad &
b)\quad\lim_{n\rightarrow+\infty}\sum_{k=1}^{n}\frac{n}{n^2+k^2} =
\frac{\pi}{4} \quad\quad\quad\\
c)\quad\lim_{n\rightarrow+\infty}\sum_{k=1}^{n}\log\left(\frac{n}{n+k}\right)^{\frac{1}{n}}
= -2\ln 2 + 1. &
\end{eqnarray*}
}
}

\uuid{RHWS}
\exo7id{638}
\titre{exo7 638}
\auteur{gourio}
\organisation{exo7}
\datecreate{2001-09-01}
\video{UAO0DkHX2EQ}
\isIndication{true}
\isCorrection{true}
\chapitre{Continuité, limite et étude de fonctions réelles}
\sousChapitre{Limite de fonctions}
\module{Analyse}
\niveau{L1}
\difficulte{}

\contenu{
\texte{
Trouver pour $(a,b)\in (\Rr^{+*})^{2}$ :
$$\lim\limits_{x\rightarrow 0^{+}}\left(\frac{a^{x}+b^{x}}{2}\right)^{\frac{1}{x}}.$$
}
\indication{R\'{e}ponse: $\sqrt{ab}$.}
\reponse{
Soit 
$$
 f(x) = \left(\frac{a^{x}+b^{x}}{2}\right)^{\frac{1}{x}}
     = \exp\left( \frac 1x \ln \left(\frac{a^{x}+b^{x}}{2}\right) \right) 
$$

$a^x \to 1$, $b^x \to 1$ donc $\frac{a^{x}+b^{x}}{2} \to 1$ lorsque $x \to 0$
et nous sommes face à une forme indéterminée.
Nous savons que $\lim_{t \to 0} \frac{\ln(1+t)}{t} = 1$.
Autrement dit il existe un fonction $\mu$ telle que $\ln(1+t) = t \cdot \mu(t)$ avec
$\mu(t) \to 1$ lorsque $t\to 0$.

Appliquons cela à $g(x) = \ln \left(\frac{a^{x}+b^{x}}{2}\right)$.
Alors 
$$g(x) = \ln \left(1+ \left(\frac{a^{x}+b^{x}}{2}-1\right)\right) =  \left(\frac{a^{x}+b^{x}}{2}-1\right) \cdot \mu(x)$$
o\` u $\mu(x) \to 1$ lorsque $x\to 0$. (Nous écrivons pour simplifier $\mu(x)$ au lieu
de $\mu(\frac{a^{x}+b^{x}}{2}-1)$.)



\bigskip

Nous savons aussi que $\lim_{t \to 0} \frac{e^t - 1}{t} = 1$.
Autrement dit il existe un fonction $\nu$ telle que $e^t - 1 = t \cdot \nu(t)$ avec
$\nu(t) \to 1$ lorsque $t\to 0$.

Appliquons ceci :

\begin{align*}
 \frac{a^{x}+b^{x}}{2}-1 
     &= \frac 12 (e^{x \ln a} + e^{x\ln b})-1 \\
     &= \frac12 (e^{x \ln a}-1 + e^{x\ln b}-1) \\
     &= \frac12( x \ln a \cdot \nu(x \ln a) + x \ln b \cdot \nu(x \ln b)) \\
     &= \frac12 x\left(\ln a \cdot \nu(x \ln a) +  \ln b \cdot \nu(x \ln b) \right) \\
\end{align*}

\bigskip

Reste à rassembler tous les éléments du puzzle :
\begin{align*}
 f(x) &= \left(\frac{a^{x}+b^{x}}{2}\right)^{\frac{1}{x}} \\
      &= \exp\left( \frac 1x \ln \left(\frac{a^{x}+b^{x}}{2}\right) \right) \\
      &= \exp\left( \frac 1x g(x) \right) \\
      &= \exp\left( \frac 1x  \left(\frac{a^{x}+b^{x}}{2}-1\right) \cdot \mu(x) \right) \\
      &= \exp\left( \frac 1x  \cdot \frac12  \cdot x\left(\ln a \cdot \nu(x \ln a) +  \ln b \cdot \nu(x \ln b) \right) \cdot \mu(x) \right) \\
      &= \exp\left( \frac12 \left(\ln a \cdot \nu(x \ln a) +  \ln b \cdot \nu(x \ln b) \right) \cdot \mu(x) \right) \\
\end{align*}

Or $\mu(x) \to 1$, $\nu(x \ln a) \to 1$, $\nu(x \ln b) \to 1$ lorsque $x \to 0$.
Donc 
$$\lim_{x\to 0} f(x) = \exp\left( \frac12 \left(\ln a + \ln b \right) \right) = \exp\left( \frac12 \ln (ab) \right) = \sqrt{ab}.$$
}
}

\uuid{1DFC}
\exo7id{699}
\titre{exo7 699}
\auteur{bodin}
\organisation{exo7}
\datecreate{1998-09-01}
\video{XPnRztcUKK4}
\isIndication{true}
\isCorrection{true}
\chapitre{Dérivabilité des fonctions réelles}
\sousChapitre{Calculs}
\module{Analyse}
\niveau{L1}
\difficulte{}

\contenu{
\texte{
D\'eterminer $a,b \in \Rr$ de mani\`ere \`a ce que la fonction $f$ d\'efinie sur $\Rr_+$ par :
$$ f(x)=\sqrt{x} \quad \text{ si } 0\leqslant x \leqslant 1 \quad \text{\ \  et\ \  } \quad f(x) = ax^2+bx+1 \quad \text{ si } x>1$$
soit d\'erivable sur $\Rr_+^*$.
}
\indication{Vous avez deux conditions : il faut que la fonction soit continue (car on veut qu'elle soit d\'erivable donc elle doit \^etre continue) et ensuite la condition de d\'erivabilit\'e proprement dite.}
\reponse{
La fonction $f$ est continue et dérivable sur $]0,1[$ et sur
$]1,+\infty[$. Le seul problème est en $x=1$.

Il faut d'abord que la fonction soit continue en $x=1$.
La limite \`a gauche est
$\lim_{x\rightarrow 1^-} \sqrt x = +1$
et \`a droite 
$\lim_{x\rightarrow 1^+} ax^2+bx+1 = a+b+1$.
Donc  $a+b+1=1$. Autrement dit $b = -a$.

Il faut maintenant que les d\'eriv\'ees \`a droite et \`a gauche soient
\'egales.
Comme la fonction $f$ restreinte à $]0,1]$ est définie par $x \mapsto \sqrt{x}$ alors elle est dérivable
à gauche et la dérivée à gauche s'obtient en évaluant la fonction dérivée $x \mapsto \frac{1}{2\sqrt x}$
en $x=1$. Donc $f'_g(1)=\frac 12$.

Pour la dérivée à droite il s'agit de calculer la limite du taux d'accroissement
$\frac{f(x)-f(1)}{x-1}$, lorsque $x \to 1$ avec $x>1$.
Or 
$$
\frac{f(x)-f(1)}{x-1} = \frac{ax^2+bx+1 - 1}{x-1} = \frac{ax^2-ax}{x-1} = \frac{ax(x-1)}{x-1} = ax.
$$
Donc $f$ est dérivable à droite et $f'_d(1) = a$.
Afin que $f$ soit dérivable, il faut et il suffit que les dérivées à droite et à gauche
existent et soient égales, donc ici la condition est $a=\frac 12$.


Le seul couple $(a,b)$ que rend $f$ dérivable sur $]0,+\infty[$ 
est $(a=\frac 12, b= -\frac 12)$.
}
}

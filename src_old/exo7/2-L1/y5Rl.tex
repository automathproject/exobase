\uuid{y5Rl}
\exo7id{4709}
\titre{exo7 4709}
\auteur{quercia}
\organisation{exo7}
\datecreate{2010-03-16}
\isIndication{false}
\isCorrection{false}
\chapitre{Suite}
\sousChapitre{Suite définie par une relation de récurrence}
\module{Analyse}
\niveau{L1}
\difficulte{}

\contenu{
\texte{
Soient $a,b \in \R^*$.
On d{\'e}finit la suite $(u_n)$ par :
$\begin{cases} u_0 \in \R^* \cr u_{n+1} = a + \frac b{u_n}.\end{cases}$

On suppose $u_0$ choisi de sorte que pour tout $n \in \N$, $u_n \ne 0$.
}
\begin{enumerate}
    \item \question{Quelles sont les limites possibles pour $(u_n)$ ?}
    \item \question{On suppose que l'{\'e}quation $x^2 = ax + b$ poss{\`e}de deux racines r{\'e}elles
    $\alpha,\beta$ avec $|\alpha| > |\beta|$.

    {\'E}tudier la suite $(v_n) = \left(\frac{u_n-\alpha}{u_n-\beta}\right)$
    et en d{\'e}duire $\lim u_n$.}
\end{enumerate}
}

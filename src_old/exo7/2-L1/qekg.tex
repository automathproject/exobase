\uuid{qekg}
\exo7id{5215}
\titre{exo7 5215}
\auteur{rouget}
\organisation{exo7}
\datecreate{2010-06-30}
\isIndication{false}
\isCorrection{true}
\chapitre{Propriétés de R}
\sousChapitre{Autre}
\module{Analyse}
\niveau{L1}
\difficulte{}

\contenu{
\texte{
Montrer que pour tout entier naturel non nul $n$, $\sum_{k=0}^{2n-1}\frac{(-1)^k}{k+1}=\sum_{k=n+1}^{2n}\frac{1}{k}$.
}
\reponse{
Pour $n=1$, $\sum_{k=0}^{2n-1}\frac{(-1)^k}{k+1}=1-\frac{1}{2}=\frac{1}{2}$ et $\sum_{k=n+1}^{2n}\frac{1}{k}=\frac{1}{2}$. L'identité proposée est donc vraie pour $n=1$.

Soit $n\geq1$. Supposons que $\sum_{k=0}^{2n-1}\frac{(-1)^k}{k+1}=\sum_{k=n+1}^{2n}\frac{1}{k}$.

On a alors
  
\begin{align*}
\sum_{k=0}^{2(n+1)-1}\frac{(-1)^k}{k+1}&=\sum_{k=0}^{2n-1}\frac{(-1)^k}{k+1}+\frac{1}{2n+1}-\frac{1}{2n+2}
=\sum_{k=n+1}^{2n}\frac{1}{k}+\frac{1}{2n+1}-\frac{1}{2(n+1)}\\
 &=\frac{1}{n+1}+\sum_{k=n+2}^{2n+1}\frac{1}{k}+\frac{1}{2(n+1)}
=\sum_{k=n+2}^{2n+1}\frac{1}{k}+\frac{1}{2n+2}=\sum_{k=n+2}^{2(n+1)}\frac{1}{k}
\end{align*}
On a montré par récurrence que $\forall n\geq1,\;\sum_{k=0}^{2n-1}\frac{(-1)^k}{k+1}=\sum_{k=n+1}^{2n}\frac{1}{k}$ (identité de \textsc{Catalan}).
}
}

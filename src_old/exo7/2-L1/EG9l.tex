\uuid{EG9l}
\exo7id{3884}
\titre{exo7 3884}
\auteur{quercia}
\organisation{exo7}
\datecreate{2010-03-11}
\isIndication{false}
\isCorrection{true}
\chapitre{Continuité, limite et étude de fonctions réelles}
\sousChapitre{Etude de fonctions}
\module{Analyse}
\niveau{L1}
\difficulte{}

\contenu{
\texte{
Soit $f : \R \to \R$ continue telle que~:
$\forall\ x\in\R,\ \exists\ \delta>0,\text{ tq }\forall\ y\in[x,x+\delta],\ f(y)\ge f(x)$.
Montrer que $f$ est croissante.
}
\reponse{
Supposons qu'il existe $a,b\in\R$ avec $a<b$ et $f(b)<f(a)$.
On note $E=\{x\in{[a,b]}\text{ tq }f(x) < f(a)\}$ et $c=\inf(E)$.
On a $c\in E$ et $c>a$ par hypothèse et donc $f(c)=\lim_{x\to c^-}f(x)\ge f(a)$, absurde.
}
}

\uuid{3v3b}
\exo7id{818}
\titre{exo7 818}
\auteur{ridde}
\organisation{exo7}
\datecreate{1999-11-01}
\isIndication{false}
\isCorrection{true}
\chapitre{Calcul d'intégrales}
\sousChapitre{Intégration par parties}
\module{Analyse}
\niveau{L1}
\difficulte{}

\contenu{
\texte{
Soit $I_{n} = \int_0^{\frac{\pi}2} \sin ^n tdt$.
}
\begin{enumerate}
    \item \question{\'Etablir une relation de r\'ecurrence entre $I_{n}$ et $I_{n + 2}$.}
\reponse{Par IPP, $I_{n+2} = \frac{n+1}{n+2} I_n$.}
    \item \question{En d\'eduire $I_{2p}$ et $I_{2p + 1}$.}
\reponse{$I_0 = \pi/2$ et $I_1 = 1$ et
\[ 
I_{2p} = \frac{ (2p-1) \times (2p-3) \times \cdots \times 1}{2p \times (2p-2) \times \cdots \times 2} I_0
= \frac{(2p)!}{2^{2p} (p!)^2} \frac{\pi}{2} ,\quad
I_{2p+1} = \frac{2p \times (2p-2) \times \cdots \times 2}{ (2p+1) \times (2p-1) \times \cdots \times 1} I_1
= \frac{2^{2p} (p!)^2}{(2p+1)!}
\]}
    \item \question{Montrer que $ (I_{n})_{n \in \Nn}$ est d\'ecroissante et strictement positive.}
\reponse{En regardant l'intégrand.}
    \item \question{En d\'eduire que $I_{n} \sim I_{n + 1}$.}
\reponse{D'après la question précédente, $0 < I_{n+2} \leq I_{n+1} \leq I_n$ donc
\[
\frac{n+1}{n+2} = \frac{I_{n+2}}{I_n} \leq \frac{I_{n+1}}{I_n} \leq 1
\]
par conséquent $\frac{I_{n+1}}{I_n} \xrightarrow[n\to \infty]{} 1$.}
    \item \question{Calculer $nI_{n}I_{n + 1}$.}
\reponse{\[
(2p-1) I_{2p-1} I_{2p} = \frac{2p-1}{2 p} \frac{\pi}{2}
, \quad
2p I_{2p} I_{2p+1} = \frac{2p}{2p+1} \frac{\pi}{2}
\]
soit $n I_n I_{n+1} = \frac{n}{n+1} \frac{\pi}{2}$, ce qui peut aussi se démontrer par récurrence.}
    \item \question{Donner alors un \'equivalent simple de $I_{n}$.}
\reponse{Comme $\frac{\pi}{2(n+1)} I_n I_{n+1} \sim I_n^2$ on en déduit que $I_n \sim \sqrt{\frac{\pi}{2n}}$.}
\end{enumerate}
}

\uuid{z84h}
\exo7id{4311}
\titre{exo7 4311}
\auteur{quercia}
\organisation{exo7}
\datecreate{2010-03-12}
\isIndication{false}
\isCorrection{true}
\chapitre{Calcul d'intégrales}
\sousChapitre{Intégrale impropre}
\module{Analyse}
\niveau{L1}
\difficulte{}

\contenu{
\texte{
Soit $f$ de classe $\mathcal{C}^2$ sur $\R^+$ à valeurs dans $\R$ telle que
$f^2$ et $f''^2$ sont intégrables sur $\R^+$. }
\question{ Montrer que $ff''$ et $f'^2$ sont intégrables sur $\R^+$, que $f$ est
uniformément continue et qu'elle tend vers zéro en $+\infty$.
}
\reponse{
$2|ff''| \le f^2 + f''^2$ donc $ff''$ est intégrable.
On en déduit que $f'^2$ admet une limite finie en $+\infty$, et cette limite
est nulle sans quoi $f^2$ ne serait pas intégrable
(si $f'(x)\to\ell$ lorsque $x\to+\infty$ alors $f(x)/x\to\ell$). Ainsi $f'$ est bornée sur~$\R^+$,
$f$ est lipschitzienne et donc uniformément continue.
De plus,
$$ \int_{t=0}^X f'^2(t)\,d t = f(X)f'(X) - f(0)f'(0) -  \int_{t=0}^X f(t)f''(t)\,d t$$
donc $f(X)f'(X)$ admet en $+\infty$ une limite finie ou $+\infty$, et le
cas $f(X)f'(X) = \frac12(f^2)'(X) \to +\infty$ lorsque $X\to+\infty$ contredit l'intégrabilité
de $f^2$ donc ce cas est impossible, ce qui prouve que $f'^2$ est intégrable
sur $\R^+$.
Enfin, $ff'$ est intégrable (produit de deux fonctions de carrés intégrables)
donc $f^2$ admet une limite finie en $+\infty$ et cette
limite vaut zéro par intégrabilité de~$f^2$.
}
}

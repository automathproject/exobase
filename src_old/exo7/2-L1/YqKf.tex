\uuid{YqKf}
\exo7id{5219}
\titre{exo7 5219}
\auteur{rouget}
\organisation{exo7}
\datecreate{2010-06-30}
\isIndication{false}
\isCorrection{true}
\chapitre{Propriétés de R}
\sousChapitre{Autre}
\module{Analyse}
\niveau{L1}
\difficulte{}

\contenu{
\texte{
Montrer que $\{r^3,\;r\in\Qq\}$ est dense dans $\Rr$.
}
\reponse{
Soient $x$ un réel et $\varepsilon$ un réel strictement positif. On a $\sqrt[3]{x}<\sqrt[3]{x+\varepsilon}$. Puisque $\Qq$ est dense dans $\Rr$, il existe un rationnel $r$ tel que $\sqrt[3]{x}<r<\sqrt[3]{x+\varepsilon}$ et donc tel que $x<r^3<x+\varepsilon$, par stricte croissance de la fonction $t\mapsto t^3$ sur $\Rr$. On a montré que

\begin{center}
\shadowbox{
$\{r^3,\;r\in\Qq\}$ est dense dans $\Rr$.
}
\end{center}
}
}

\uuid{7qJO}
\exo7id{4283}
\titre{exo7 4283}
\auteur{quercia}
\organisation{exo7}
\datecreate{2010-03-12}
\isIndication{false}
\isCorrection{false}
\chapitre{Calcul d'intégrales}
\sousChapitre{Intégrale impropre}
\module{Analyse}
\niveau{L1}
\difficulte{}

\contenu{
\texte{
Soit $f$ une application continue de $[1,+\infty[$ dans $\R$.
Montrer que si l'intégrale $ \int_{t=1}^{+\infty} f(t)\,d t$ converge, il en est de même
de l'intégrale $ \int_{t=1}^{+\infty} \frac{f(t)}t\,d t$.
On pourra introduire la fonction $F(x) =  \int_{t=1}^x f(t)\,d t$.
}
}

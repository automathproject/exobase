\uuid{VzD8}
\exo7id{4287}
\titre{exo7 4287}
\auteur{quercia}
\organisation{exo7}
\datecreate{2010-03-12}
\isIndication{false}
\isCorrection{false}
\chapitre{Calcul d'intégrales}
\sousChapitre{Intégrale impropre}
\module{Analyse}
\niveau{L1}
\difficulte{}

\contenu{
\texte{
Soit $f : {[a,b[} \to {\R^+}$ continue croissante.
On pose $S_n = \frac{b-a}n \sum_{k=0}^{n-1} f\Bigl(a+k\frac{b-a}n \Bigr)$.
}
\begin{enumerate}
    \item \question{Si $ \int_{t=a}^b f(t)\,d t$ converge, montrer que $S_n \to  \int_{t=a}^b f(t)d t$ lorsque $n\to\infty$.}
    \item \question{Si $ \int_{t=a}^b f(t)\,d t$ diverge, montrer que  $S_n \to +\infty$ lorsque $n\to\infty$.}
\end{enumerate}
}

\uuid{bJE9}
\exo7id{1198}
\titre{exo7 1198}
\auteur{legall}
\organisation{exo7}
\datecreate{1998-09-01}
\isIndication{false}
\isCorrection{false}
\chapitre{Suite}
\sousChapitre{Convergence}
\module{Analyse}
\niveau{L1}
\difficulte{}

\contenu{
\texte{

}
\begin{enumerate}
    \item \question{Soit $  (u_n)_{n\in { \Nn}}  $ une suite de nombres r\'eels
telle que les suites extraites $  (u_{2n})_{n\in { \Nn}}  $ et $  (u_{2n+1})_{n\in { \Nn}}  $
convergent vers une m\^eme limite $  \ell   .$ Montrer que $  (u_n)_{n\in { \Nn}}  $ converge
\'egale\-ment vers $  \ell   .$}
    \item \question{En d\'eduire que la suite $  (u_n)_{n\in { \Nn}}  $ de terme g\'en\'eral $  \displaystyle{
u_n =\sum _{k=0}^n \frac{ (-1)^k }{ (2k)!}}  $ converge.}
\end{enumerate}
}

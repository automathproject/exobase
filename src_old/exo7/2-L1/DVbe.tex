\uuid{DVbe}
\exo7id{571}
\titre{exo7 571}
\auteur{bodin}
\organisation{exo7}
\datecreate{1998-09-01}
\video{4ehFn1cnohc}
\isIndication{true}
\isCorrection{true}
\chapitre{Suite}
\sousChapitre{Suites équivalentes, suites négligeables}
\module{Analyse}
\niveau{L1}
\difficulte{}

\contenu{
\texte{
Soient $a$ et $b$ deux r\'eels, $a<b$. On consid\`ere la
fonction $f:\lbrack a,b\rbrack\longrightarrow \lbrack a,b\rbrack$ suppos\'ee continue
et une suite r\'ecurrente $(u_n)_n$ d\'efinie par :
$$u_0\in\lbrack a,b\rbrack\ \ \text{et pour tout }\ n\in\N,\ \ u_{n+1}=f(u_n).$$
}
\begin{enumerate}
    \item \question{On suppose ici que $f$ est croissante. Montrer que $(u_n)_n$
est monotone et en d\'eduire sa convergence vers une solution de l'\'equation
$f(x)=x$.}
\reponse{Si $u_0 \leqslant u_1$ alors comme $f$ est croissante $f(u_0)\leqslant f(u_1)$ donc $u_1 \leqslant u_2$, ensuite $f(u_1)\leqslant f(u_2)$ soit $u_2 \leqslant u_3$,... Par r\'ecurrence on montre que $(u_n)$ est d\'ecroissante. Comme elle est minor\'ee par $a$ alors elle converge. Si $u_0 \leqslant u_1$ alors la suite $(u_n)$ est croissante et major\'ee par $b$ donc converge.

Notons $\ell$ la limite de $(u_n)_n$. Comme $f$ est continue alors
$(f(u_n))$ tend vers $f(\ell)$. De plus la limite de $(u_{n+1})_n$
est aussi $\ell$. En passant \`a la limite dans l'expression
$u_{n+1}=f(u_n)$ nous obtenons l'\'egalit\'e $\ell = f(\ell)$.}
    \item \question{\emph{Application.} Calculer la limite de la suite définie par :
$$u_0=4\ \ \text{et pour tout }\ n\in\N,\ \ u_{n+1}=\frac{4u_n+5}{u_n+3}.$$}
\reponse{La fonction $f$ définie par $f(x) = \frac{4x+5}{x+3}$ est continue et d\'erivable sur l'intervalle $[0,4]$ et $f([0,4])\subset [0,4]$.
La fonction $f$ est croissante (calculez sa d\'eriv\'ee). Comme $u_0 =
4$ et $u_1= 3$ alors $(u_n)$ est d\'ecroissante. Calculons la valeur
de sa limite $\ell$. $\ell$ est solution de l'\'equation $f(x)=x$
soit $4x+5=x(x+3)$. Comme $u_n \geqslant 0$ pour tout $n$ alors $\ell
\geqslant 0$. La seule solution positive de l'équation du second degré $4x+5=x(x+3)$ est $\ell =
\frac{1+\sqrt{21}}{2}=2,7912\ldots$}
    \item \question{On suppose maintenant que $f$ est d\'ecroissante. Montrer que les suites
$(u_{2n})_n$ et $(u_{2n+1})_n$ sont monotones et convergentes.}
\reponse{Si $f$ est d\'ecroissante alors $f\circ f$ est croissante (car $x\leqslant y \Rightarrow
f(x)\geqslant f(y) \Rightarrow f\circ f(x)\leqslant  f\circ f(y)$). Nous
appliquons la premi\`ere question avec la fonction $f\circ f$. La
suite $(u_0, u_2 = f\circ f(u_0),u_4 = f\circ f(u_2),\ldots)$ est
monotone et convergente. De m\^eme pour la suite $(u_1, u_3 =
f\circ f(u_1),u_5 = f\circ f(u_3),\ldots)$.}
    \item \question{\emph{Application.} Soit
$$u_0=\frac{1}{2}\ \ \text{et pour tout }\  n\in\N,\ \ u_{n+1}=(1-u_n)^2.$$
\noindent Calculer les limites des suites $(u_{2n})_n$ et $(u_{2n+1})_n$.}
\reponse{La fonction $f$ définie par $f(x) = (1-x)^2$ est continue et d\'erivable de $[0,1]$ dans $[0,1]$.
Elle est d\'ecroissante sur cet intervalle. Nous avons $u_0 =
\frac12$, $u_1=\frac14$, $u_2=\frac{9}{16}$, $u_3 =
0,19\ldots$,... Donc la suite $(u_{2n})$ est croissante, nous
savons qu'elle converge et notons $\ell$ sa limite. La suite
$(u_{2n+1})$ et d\'ecroissante,  notons $\ell'$ sa limite. Les
limites $\ell$ et $\ell'$ sont des solutions de l'\'equation
$f\circ f(x)=x$. Cette \'equation s'\'ecrit $(1-f(x))^2=x$, ou encore
$(1-(1-x)^2)^2=x$ soit $x^2(2-x)^2=x$. Il y a deux solutions
\'evidentes $0$ et $1$. Nous factorisons le polyn\^ome
$x^2(2-x)^2-x$ en $x(x-1)(x-\lambda)(x-\mu)$ avec $\lambda$ et
$\mu$ les solutions de l'\'equation $x^2-3x+1$ : $\lambda =
\frac{3-\sqrt{5}}{2} = 0,3819\ldots$ et $\mu =
\frac{3+\sqrt{5}}{2} > 1$. Les solutions de l'\'equation $f\circ
f(x)=x$ sont donc $\{ 0,1,\lambda, \mu\}$. Comme $(u_{2n})$ est
croissante et que $u_0 = \frac12$ alors  $(u_{2n})$ converge vers
$\ell=1$ qui est le seul point fixe de $[0,1]$ sup\'erieur \`a
$\frac12$. Comme $(u_{2n+1})$ est d\'ecroissante et que $u_1 =
\frac14$ alors  $(u_{2n+1})$ converge vers $\ell'=0$ qui est le
seul point fixe de $[0,1]$ inf\'erieur \`a $\frac14$.}
\indication{Pour la premi\`ere question et la monotonie il faut raisonner par r\'ecurrence.
Pour la troisi\`eme question, remarquer que si $f$ est d\'ecroissante alors $f\circ f$ est croissante
et appliquer la premi\`ere question.}
\end{enumerate}
}

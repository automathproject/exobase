\uuid{lPYM}
\exo7id{689}
\titre{exo7 689}
\auteur{bodin}
\organisation{exo7}
\datecreate{1998-09-01}
\isIndication{false}
\isCorrection{false}
\chapitre{Continuité, limite et étude de fonctions réelles}
\sousChapitre{Etude de fonctions}
\module{Analyse}
\niveau{L1}
\difficulte{}

\contenu{
\texte{
En \'etudiant les variations de la fonction $f$
d\'efinie sur $\rbrack 0,+\infty\lbrack$ par $\displaystyle{f(x)=x^{\frac{1}{x}}}$, trouver
le plus grand \'el\'ement de l'ensemble $f(\N^*)$.
\par En d\'eduire que quels soient $m$ et $n$ appartenant \`a $\N^*$,
l'un des nombres $\sqrt[n]{m}$, $\sqrt[m]{n}$ est inf\'erieur ou \'egal
\`a $\sqrt[3]{3}$.
}
}

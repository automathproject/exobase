\uuid{ulsO}
\exo7id{2085}
\titre{exo7 2085}
\auteur{bodin}
\organisation{exo7}
\datecreate{2008-02-04}
\video{OjBbmNvQoSY}
\isIndication{true}
\isCorrection{true}
\chapitre{Calcul d'intégrales}
\sousChapitre{Théorie}
\module{Analyse}
\niveau{L1}
\difficulte{}

\contenu{
\texte{
Soit $f:[a,b]\rightarrow \R$ une fonction continue sur $[a,b]$ ($a<b$).
}
\begin{enumerate}
    \item \question{On suppose que $f(x) \ge 0$ pour tout $x\in [a,b]$, et que $f(x_0)>0$ en un point $x_0\in [a,b]$. 
Montrer que $\int_a^b f(x) d x>0$. En déduire que : <<si $f$ est une fonction continue
positive sur $[a,b]$ telle que $\int_a^b f(x) d x=0$ alors $f$ est
identiquement nulle>>.}
\reponse{\'Ecrivons la continuité de $f$ en $x_0$ avec $\epsilon = \frac {f(x_0)}{2} > 0$ :
il existe $\delta >0$ tel que pour tout $x\in [x_0-\delta, x_0+\delta]$ on ait $|f(x)-f(x_0)| \leqslant \epsilon$.
Avec notre choix de $\epsilon$ cela donne pour $x\in [x_0-\delta, x_0+\delta]$ que $f(x) \geqslant \frac {f(x_0)}{2}$.
Pour évaluer $\int_a^b f(x) \, dx$ nous la coupons en trois morceaux par linéarité de l'intégrale :
$$\int_a^b f(x) \, dx = \int_a^{x_0-\delta} f(x) dx +  \int_{x_0-\delta}^{x_0+\delta} f(x) dx +  \int_{x_0+\delta}^b f(x) dx.$$
Comme $f$ est positive alors par positivité de l'intégrale $\int_a^{x_0-\delta} f(x) dx \geqslant 0$ et
$\int_{x_0+\delta}^b f(x) dx \geqslant 0$. Pour le terme du milieu on a $f(x) \geqslant \frac {f(x_0)}{2}$ donc
$\int_{x_0-\delta}^{x_0+\delta} f(x) dx \geqslant \int_{x_0-\delta}^{x_0+\delta} \frac {f(x_0)}{2} dx =
2\delta\frac {f(x_0)}{2}$ (pour la dernière équation on calcule juste l'intégrale d'une fonction constante !).
Le bilan de tout cela est que $\int_a^b f(x) \, dx \geqslant 2\delta\frac {f(x_0)}{2} >0$.

Donc pour une fonction continue et positive $f$, si elle est strictement positive en un point alors  $\int_a^b f(x) \, dx >0$.
Par contraposition pour une fonction continue et positive si $\int_a^b f(x) \, dx =0$ alors
$f$ est identiquement nulle.}
    \item \question{On suppose que $\int_a^b f(x) d x=0$. Montrer qu'il existe $c\in [a,b]$ tel que $f(c)=0$.}
\reponse{Soit $f$ est tout le temps positive, soit elle tout le temps négative, soit elle change 
(au moins un fois) de signe. Dans le premier cas $f$ est identiquement nulle par la première question, 
dans le second cas c'est pareil (en appliquant la première question à $-f$). Pour le troisième cas 
le théorème des valeurs intermédiaires affirme qu'il existe $c$ tel que $f(c)=0$.}
    \item \question{Application: on suppose
que $f$ est une fonction continue sur $[0,1]$ telle que $\int_0^1 f(x) dx=\frac 12$. 
Montrer qu'il existe $d\in [0,1]$ tel que $f(d)=d$.}
\reponse{Posons $g(x) = f(x)-x$. Alors $\int_0^1 g(x) dx = \int_0^1 \big( f(x) - x \big)  dx= \int_0^1 f(x) dx - \frac 12 = 0$.
Donc par la question précédente, $g$ étant continue, il existe $d \in [0,1]$ tel que $g(d)=0$, ce qui est équivalent à $f(d)=d$.}
\indication{\begin{enumerate}
  \item Revenir à la définition de la continuité en $x_0$ en prenant $\epsilon = \frac {f(x_0)}{2}$ par exemple.
  \item Soit $f$ est tout le temps de même signe (et alors utiliser la première question), soit ce n'est pas le cas (et alors utiliser un théorème classique...).
  \item On remarquera que $\int_0^1 f(x) \, dx - \frac 12 = \int_0^1 (f(x) - x) dx$.
  \end{enumerate}}
\end{enumerate}
}

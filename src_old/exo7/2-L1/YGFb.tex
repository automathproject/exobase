\uuid{YGFb}
\exo7id{4029}
\titre{exo7 4029}
\auteur{quercia}
\organisation{exo7}
\datecreate{2010-03-11}
\isIndication{false}
\isCorrection{true}
\chapitre{Développement limité}
\sousChapitre{Applications}
\module{Analyse}
\niveau{L1}
\difficulte{}

\contenu{
\texte{
Rechercher si les courbes suivantes admettent une asymptote en $+\infty$ et
déterminer la position s'il y a lieu :
}
\begin{enumerate}
    \item \question{$y = \sqrt{x(x+1)}$.}
\reponse{$y = x + \frac 12 - \frac 1{8x}$.}
    \item \question{$y = \sqrt{\frac{x^3}{x-1}}$.}
\reponse{$y = x + \frac12 + \frac3{8x}$.}
    \item \question{$y = (x^2-1)\ln\left(\frac{x+1}{x-1}\right)$.}
\reponse{$y = 2x - \frac4{3x}$.}
    \item \question{$y = (x+1)\arctan(1+2/x)$.}
\reponse{$y = \frac{\pi x}4 + \frac\pi4+1 - \frac1{3x^2}$.}
    \item \question{$y = x.\arctan x.e^{1/x}$.}
\reponse{$y = \frac {\pi x}2 + \frac \pi2-1 + \frac {\pi/4-1}x$.}
    \item \question{$y = e^{2/x}\sqrt{1+x^2}\arctan x$.}
\reponse{$y = \frac{\pi x}2 + \pi-1 + \frac{5\pi/4-2}x$.}
    \item \question{$y = \sqrt{x^2-x}\exp\left(\frac 1{x+1}\right)$.}
\reponse{$y = x + \frac 12 - \frac 9{8x}$.}
\end{enumerate}
}

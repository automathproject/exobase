\uuid{468u}
\exo7id{3939}
\titre{exo7 3939}
\auteur{quercia}
\organisation{exo7}
\datecreate{2010-03-11}
\isIndication{false}
\isCorrection{true}
\chapitre{Dérivabilité des fonctions réelles}
\sousChapitre{Autre}
\module{Analyse}
\niveau{L1}
\difficulte{}

\contenu{
\texte{
Soit $f : {\R^+} \to \R$ continue telle que
$f(x) \int_{t=0}^x f^2(t)\,d t \to \ell\in\R^*$ lorsque $x\to+\infty$.
Montrer qu'il existe $\alpha,\beta\in\R^*$ tels que 
$f(x)\sim \frac\alpha {x^\beta}$ en $+\infty$.
}
\reponse{
Poser $g(x)=  \int_{t=0}^x f^2(t)\,d t$.
On obtient $(g^3)'(x)\to 3\ell^2$ lorsque $x\to+\infty$,
ce qui implique (classiquement) que $g^3(x)\sim 3\ell^2x$,
puis $f(x)\sim \sqrt[3]{\frac\ell{3x}}$.
}
}

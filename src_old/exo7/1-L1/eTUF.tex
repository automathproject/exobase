\uuid{eTUF}
\exo7id{38}
\titre{exo7 38}
\auteur{liousse}
\organisation{exo7}
\datecreate{2003-10-01}
\isIndication{false}
\isCorrection{false}
\chapitre{Nombres complexes}
\sousChapitre{Racine carrée, équation du second degré}
\module{Algèbre}
\niveau{L1}
\difficulte{}

\contenu{
\texte{

}
\begin{enumerate}
    \item \question{Pour $\alpha\in\Rr$, r\'esoudre 
dans $\Cc$ l'\'equation $z^2-2\cos(\alpha)z+1=0.$ 
En d\'eduire la forme trigonom\'etrique des solutions de l'\'equation :
$$z^{2n}-2\cos(\alpha)z^n+1=0,\hbox{ o\`u $n$ est un entier naturel non nul.}$$
 $$P_\alpha(z)=z^{2n}-2\cos(\alpha)z^n+1.$$
\begin{enumerate}}
    \item \question{Justifier la factorisation suivante de $P_\alpha$ :
$$P_\alpha(z)=\left(z^2-2\cos\left(\frac{\alpha}{n}\right)+1\right)
\left(z^2-2\cos\left(\frac{\alpha}{n}+\frac{2\pi}{n}\right)+1\right)
\dots\left(z^2-2\cos
\left(\frac{\alpha}{n}+\frac{2(n-1)\pi}{n}\right)+1\right).$$}
    \item \question{Prouver, \`a l'aide des nombres complexes par exemple, la formule suivante :
$$1-\cos\theta=2\sin^2\left(\frac{\theta}{2}\right),\quad\theta\in\Rr.$$}
    \item \question{Calculer $P_\alpha(1)$. En d\'eduire 
$$\sin^2\left(\frac{\alpha}{2n}\right)\sin^2
\left(\frac{\alpha}{2n}+\frac{\pi}{n}\right)\dots\sin^2
\left(\frac{\alpha}{2n}+\frac{(n-1)\pi}{n}\right)=
\frac{\sin^2\left(\frac{\alpha}{2}\right)}{4^{n-1}}.$$}
\end{enumerate}
}

\uuid{aElB}
\exo7id{5583}
\titre{exo7 5583}
\auteur{rouget}
\organisation{exo7}
\datecreate{2010-10-16}
\isIndication{false}
\isCorrection{true}
\chapitre{Application linéaire}
\sousChapitre{Image et noyau, théorème du rang}
\module{Algèbre}
\niveau{L1}
\difficulte{}

\contenu{
\texte{
Soient $E$ un espace vectoriel non nul de dimension finie et $f$ un endomorphisme de $E$.

Montrer que :
}
\begin{enumerate}
    \item \question{$(f\;\text{non injective})\Leftrightarrow(f=0\;\text{ou}\;f\;\text{diviseur de zéro à gauche})$.}
\reponse{$\Leftarrow$/ Si $f=0$, $f$ n'est pas injective (car $E\neq\{0\}$).

Si $f\neq 0$ et s'il existe un endomorphisme non nul $g$ de $E$ tel que $f\circ g= 0$ alors il existe un vecteur $x$ de E tel que $g(x)\neq 0$ et $f(g(x)) = 0$. Par suite $\text{Ker}f\neq\{0\}$ et $f$ n'est pas injective.

$\Rightarrow$/ Supposons $f$ non injective et non nulle. Soient $F=\text{Ker}f$ et $G$ un supplémentaire quelconque de $F$ dans $E$. Soit $p$ la projection sur $F$ parallèlement à $G$.

Puisque $F=\text{Ker}f$, on a $f\circ p=0$ et puisque $f$ n'est pas nul, $F$ est distinct de $E$ et donc $G$ n'est pas nul ($E$ étant de dimension finie) ou encore $p$ n'est pas nul. $f$ est donc diviseur de zéro à gauche.}
    \item \question{$(f\;\text{non surjective})\Leftrightarrow(f=0\;\text{ou}\;f\;\text{diviseur de zéro à droite})$.}
\reponse{$\Leftarrow$/ Si $f=0$, $f$ n'est pas surjective.

Si $f$ n'est pas nul et s'il existe un endomorphisme non nul $g$ de $E$ tel que $g\circ f=0$ alors $f$ ne peut être surjective car sinon $g(E)=g(f(E))=\{0\}$ contredisant $g\neq0$.

$\Rightarrow$/ Supposons $f$ non surjective et non nulle.

Soient $G=\text{Im}f$ et $F$ un supplémentaire quelconque de $G$ dans $E$ puis $p$ la projection sur $F$ parallèlement à $G$. $F$ et $G$ sont non nuls et distincts de $E$ et donc $p$ n'est pas nulle et vérifie $p\circ f= 0$. $f$ est donc diviseur de zéro à droite.}
\end{enumerate}
}

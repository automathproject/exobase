\uuid{8nqO}
\exo7id{5283}
\titre{exo7 5283}
\auteur{rouget}
\organisation{exo7}
\datecreate{2010-07-04}
\isIndication{false}
\isCorrection{true}
\chapitre{Dénombrement}
\sousChapitre{Autre}
\module{Algèbre}
\niveau{L1}
\difficulte{}

\contenu{
\texte{
Montrer que le premier de l'an tombe plus souvent un dimanche qu'un samedi.
}
\reponse{
Notre calendrier est $400$ ans périodique (et presque $4.7=28$ ans périodique).
En effet,
\begin{enumerate}
la répartition des années bissextiles est $400$ ans périodique ($1600$ et $2000$ sont bissextiles mais $1700$, $1800$ et $1900$ ne le sont pas (entre autre pour regagner $3$ jours tous les $400$ ans et coller le plus possible au rythme du soleil))
il y a un nombre entier de semaines dans une période de $400$ ans. En effet, sur $400$ ans, le quart des années, soit $100$ ans, moins $3$ années sont bissextiles et donc sur toute période de $400$ ans il y a $97$ années bissextiles et $303$ années non bissextiles.

Une année non bissextile de $365$ jours est constituée de $52.7+1$ jours ou encore d'un nombre entier de semaines plus un jour et une année bissextile est constituée d'un nombre entier de semaine plus deux jours.

Une période de $400$ ans est donc constituée d'un nombre entier de semaines plus~:~$97.2+303.1=194+303=497=7.71$ jours qui fournit encore un nombre entier de semaines.
}
}

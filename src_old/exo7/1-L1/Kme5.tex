\uuid{Kme5}
\exo7id{364}
\titre{exo7 364}
\auteur{bodin}
\organisation{exo7}
\datecreate{1998-09-01}
\isIndication{false}
\isCorrection{true}
\chapitre{Polynôme, fraction rationnelle}
\sousChapitre{Division euclidienne}
\module{Algèbre}
\niveau{L1}
\difficulte{}

\contenu{
\texte{
Effectuer les divisions euclidiennes de \\
$3X^5+4X^2+1 \ \text{ par }\  X^2+2X+3$,\\
$3X^5+2X^4-X^2+1 \  \text{ par }\  X^3+X+2$,\\
$X^4-X^3+X-2 \ \text{ par }\  X^2-2X+4$.
}
\reponse{
$A =  3X^5+4X^2+1$, $B = X^2+2X+3$, le quotient de $A$ par $B$ est
$3{X}^{3}-6{X}^{2}+3X+16$ et le reste $-47-41X$.
$A = 3X^5+2X^4-X^2+1$, $B = X^3+X+2$ le quotient de $A$ par $B$ est
$ 3{X}^{2}+2X-3 $ et le reste est $ 7-9{X}^{2}-X$.
$ A = X^4-X^3-X-2$, $B = X^2-2X+4$, le quotient de $A$ par $B$ est
$ {X}^{2}+X-2$ de reste $ 6-9X$.
}
}

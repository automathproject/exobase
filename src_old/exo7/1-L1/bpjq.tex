\uuid{bpjq}
\exo7id{37}
\titre{exo7 37}
\auteur{cousquer}
\organisation{exo7}
\datecreate{2003-10-01}
\isIndication{false}
\isCorrection{false}
\chapitre{Nombres complexes}
\sousChapitre{Racine carrée, équation du second degré}
\module{Algèbre}
\niveau{L1}
\difficulte{}

\contenu{
\texte{
On considère dans $\mathbb{C}$ l'équation $(E)$ suivante:
$$ z^2-\left(1+a\right)\left(1+i\right)z+\left(1+a^2\right)i=0,$$
où $a$ est un paramètre réel.
}
\begin{enumerate}
    \item \question{Calculer en fonction de $a\in\mathbb{R}$ les solutions $z_1$ et $z_2$ de $(E)$
(indication: on pourra déterminer les racines carées complexes de
$-2i(1-a)^2$).}
    \item \question{On désigne par $Z_1$ (resp. $Z_2$) les points du plan complexe
d'affixe $z_1$ (resp. $z_2$) et par $M$ le milieu de
$\left[Z_1,Z_2\right]$. Tracer la courbe du plan complexe décrite par
$M$ lorsque $a$ varie dans $\mathbb{R}$.}
\end{enumerate}
}

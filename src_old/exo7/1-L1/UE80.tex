\uuid{UE80}
\exo7id{326}
\titre{exo7 326}
\auteur{cousquer}
\organisation{exo7}
\datecreate{2003-10-01}
\isIndication{false}
\isCorrection{false}
\chapitre{Arithmétique dans Z}
\sousChapitre{Pgcd, ppcm, algorithme d'Euclide}
\module{Algèbre}
\niveau{L1}
\difficulte{}

\contenu{
\texte{

}
\begin{enumerate}
    \item \question{Dans $\mathbb{Z}/n\mathbb{Z}$, écrire l'ensemble des multiples de $\bar x$, classe
de $x$, pour $x$ variant de $0$ à $n-1$ dans chacun des cas suivants~:
$\mathbb{Z}/5\mathbb{Z}$, $\mathbb{Z}/6\mathbb{Z}$, $\mathbb{Z}/8\mathbb{Z}$.}
    \item \question{Dans $\mathbb{Z}/n\mathbb{Z}$, montrer l'équivalence des trois propositions~:
\begin{itemize}}
    \item \question{[i)] $\bar x$ est inversible~;}
    \item \question{[ii)] $x$ et $n$ sont premiers entre eux~;}
    \item \question{[iii)] $\bar x$ engendre  $\mathbb{Z}/n\mathbb{Z}$, c'est à dire que
l'ensemble des multiples de $\bar x$ est $\mathbb{Z}/n\mathbb{Z}$.
\end{itemize}}
    \item \question{La classe de $18$ est-elle inversible dans $\mathbb{Z}/49\mathbb{Z}$~? Si oui, quel
est son inverse~? (On pourra utiliser le théorème de Bézout).}
\end{enumerate}
}

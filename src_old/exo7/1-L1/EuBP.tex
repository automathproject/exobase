\uuid{EuBP}
\exo7id{901}
\titre{exo7 901}
\auteur{liousse}
\organisation{exo7}
\datecreate{2003-10-01}
\video{A-ieYnIs2KA}
\isIndication{true}
\isCorrection{true}
\chapitre{Espace vectoriel}
\sousChapitre{Système de vecteurs}
\module{Algèbre}
\niveau{L1}
\difficulte{}

\contenu{
\texte{
Dans $\Rr^4$ on consid\`ere
l'ensemble $E$ des vecteurs $(x_1,x_2,x_3,x_4)$ v\'erifiant
$x_1+x_2+x_3+x_4=0$. L'ensemble $E$ est-il un sous-espace vectoriel de
$\Rr^4$ ? Si oui, en donner une base.
}
\indication{$E$ est un sous-espace vectoriel de $\Rr^4$. Une base comporte trois vecteurs.}
\reponse{
On v\'erifie les propri\'et\'es qui font de $E$ un sous-espace
  vectoriel de $\Rr^4$ : 
  \begin{enumerate}
l'origine $(0,0,0,0)$ est dans $E$,
si $v=(x_1,x_2,x_3,x_4) \in E$ et $v'=(x_1',x_2',x_3',x_4')\in E$ alors $v+v'=(x_1+x_1',x_2+x_2',x_3+x_3',x_4+x_4')$
a des coordonnées qui vérifient l'équation et donc $v+v' \in E$.
si $v=(x_1,x_2,x_3,x_4) \in E$ et $\lambda \in \Rr$ alors les coordonnées de 
$\lambda\cdot v = (\lambda x_1,\lambda x_2,\lambda x_3,\lambda x_4)$ vérifient 
l'équation et donc $\lambda \cdot v \in E$.
}
}

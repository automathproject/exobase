\uuid{Mb8g}
\exo7id{3336}
\titre{exo7 3336}
\auteur{quercia}
\organisation{exo7}
\datecreate{2010-03-09}
\isIndication{false}
\isCorrection{true}
\chapitre{Application linéaire}
\sousChapitre{Image et noyau, théorème du rang}
\module{Algèbre}
\niveau{L1}
\difficulte{}

\contenu{
\texte{
Soit $E$ un $ K$-ev de dimension finie et $H,K$ deux sev fixés de $E$.
}
\begin{enumerate}
    \item \question{A quelle condition existe-t-il un endomorphisme $f \in \mathcal{L}(E)$ tel que
     $\Im f = H \text{ et } \mathrm{Ker} f = K$ ?}
\reponse{$\dim H + \dim K = \dim E$.}
    \item \question{On note ${\cal E} = \{ f \in \mathcal{L}(E) \text{ tq } \Im f = H \text{ et } \mathrm{Ker} f = K \}$.
     Montrer que $\cal E$ est un groupe pour $\circ$ si et seulement si
     $H \oplus K = E$.}
\reponse{Si $H \oplus K \ne E$ alors $\cal E$ n'est pas stable pour
              $\circ$.}
\end{enumerate}
}

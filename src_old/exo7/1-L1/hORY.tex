\uuid{hORY}
\exo7id{1015}
\titre{exo7 1015}
\auteur{cousquer}
\organisation{exo7}
\datecreate{2003-10-01}
\video{dSfb2GXKgeU}
\isIndication{true}
\isCorrection{true}
\chapitre{Espace vectoriel}
\sousChapitre{Dimension}
\module{Algèbre}
\niveau{L1}
\difficulte{}

\contenu{
\texte{
Soit $E$ est un espace vectoriel de dimension finie et $F$ et $G$ deux
sous-espaces vectoriels de $E$. Montrer que : 
$$\dim(F+G) = \dim F+\dim G - \dim(F\cap G).$$
}
\indication{Partir d'une base $(e_1,\ldots, e_k)$ de $F\cap G$ et la compl\'eter par des vecteurs $(f_1,\ldots,f_\ell)$ en une base de $F$.
Repartir de $(e_1,\ldots, e_k)$ pour la compléter par  des vecteurs $(g_1,\ldots,g_m)$ en une base de $G$.
Montrer que $(e_1,\ldots, e_k,f_1,\ldots,f_\ell,g_1,\ldots,g_m)$ est une base de $F+G$.}
\reponse{
$F\cap G$ est un sous-espace vectoriel de $E$ donc est de dimension finie.
Soit $(e_1,\ldots, e_k)$ une base de $F\cap G$ avec $k=\dim F\cap G$.

$(e_1,\ldots, e_k)$ est une famille libre dans $F$ donc on peut la compl\'eter en une base de $F$ 
par le th\'eor\`eme de la base incompl\`ete.
Soient donc $(f_1,\ldots,f_\ell)$ des vecteurs de $F$ tels que 
$(e_1,\ldots, e_k,f_1,\ldots,f_\ell)$ soit une base de $F$.
Nous savons que $k+\ell = \dim F$.
Remarquons que les vecteurs $f_i$ sont dans $F\setminus G$ (car ils sont dans $F$ mais pas dans $F\cap G$).

Nous repartons de la famille $(e_1,\ldots, e_k)$ mais cette fois nous la compl\'etons en une base de $G$ : soit donc $(g_1,\ldots,g_m)$ des vecteurs de $G$ tels que $(e_1,\ldots, e_k,g_1,\ldots,g_m)$ soit une base de $G$.
Nous savons que $k+m = \dim G$.
Remarquons que cette fois les vecteurs $g_i$ sont dans $G\setminus F$.
Montrons que $\mathcal{B}=(e_1,\ldots, e_k,f_1,\ldots,f_\ell,g_1,\ldots,g_m)$ est une base de $F+G$.

C'est une famille g\'en\'eratrice car $F=\mathrm{Vect}(e_1,\ldots, e_k,f_1,\ldots,f_\ell) \subset \mathrm{Vect}(\mathcal{B})$ et $G=\mathrm{Vect}(e_1,\ldots, e_k,g_1,\ldots,g_m) \subset \mathrm{Vect}(\mathcal{B})$. Donc $F+G \subset  \mathrm{Vect}(\mathcal{B})$.

C'est une famille libre : en effet soit une combinaison lin\'eaire nulle
$$a_1e_1+\cdots+ a_ke_k\quad + \quad b_1f_1+\cdots +b_\ell f_\ell \quad + \quad c_1 g_1+\cdots +c_m g_m=0.$$
Notons $e=a_1e_1+\ldots +a_ke_k$, $f=b_1f_1+\cdots +b_\ell f_\ell$, $g=c_1 g_1+\cdots +c_m g_m$.
Donc la combinaison lin\'eaire devient :
$$e+f+g=0.$$
Donc $g=-e-f$, or $e$ et $f$ sont dans $F$ donc $g$ appartient \`a $F$.
Or les vecteurs $g_i$ ne sont pas dans $F$. Donc 
$g=c_1 g_1+\cdots+ c_m g_m$ est n\'ecessairement le vecteur nul.
Nous obtenons $c_1 g_1+\cdots+ c_m g_m=0$ c'est donc une combinaison lin\'eaire nulle pour la famille libre 
$(g_1,\ldots,g_m)$. Donc tous les coefficients $c_1,\ldots,c_m$ sont nuls.

Le reste de l'\'equation devient $a_1e_1+\cdots +a_ke_k+b_1f_1+\cdots +b_\ell f_\ell=0$,
or $(e_1,\ldots, e_k,f_1,\ldots,f_\ell)$ est une base de $F$ donc tous les coefficients 
$a_1,\ldots,a_k,b_1,\ldots,b_\ell$ sont nuls.

Bilan : tous les coefficients sont nuls donc la famille est libre. Comme elle \'etait g\'en\'eratrice, c'est une base.
Puisque $\mathcal{B}$ est une base de $F+G$ alors la dimension de $F+G$ est le nombre de vecteurs de la base $\mathcal{B}$:
$$\dim (F+G) = k + \ell + m.$$
Or $k=\dim F\cap G$, $\ell = \dim F - k$, $m=\dim G - k$, donc 
$$\dim (F+G) = \dim F + \dim G - \dim (F\cap G).$$
}
}

\uuid{bFmu}
\exo7id{370}
\titre{exo7 370}
\auteur{cousquer}
\organisation{exo7}
\datecreate{2003-10-01}
\video{ndl3qcjN2vw} 
\isIndication{false}
\isCorrection{true}
\chapitre{Polynôme, fraction rationnelle}
\sousChapitre{Division euclidienne}
\module{Algèbre}
\niveau{L1}
\difficulte{}

\contenu{
\texte{
Chercher tous les polynômes $P$ tels que $P+1$ soit divisible par $(X-1)^4$ et $P-1$ par $(X+1)^4$. 

\medskip
 

\emph{Indications.} 
Commencer par trouver une solution particulière $P_0$ avec l'une des méthode suivantes :
}
\begin{enumerate}
    \item \question{à partir de la relation de Bézout entre  $(X-1)^4$ et $(X+1)^4$;}
\reponse{On remarque que si $P$ est solution, alors $P+1=(X-1)^4A$ et par ailleurs $P-1=(X+1)^4B$, ce qui donne $1=\frac{A}{2}(X-1)^4+\frac{-B}{2}(X+1)^4$. Cherchons des polynômes $A$ et $B$ qui conviennent: pour cela, on écrit la relation de Bézout entre $(X-1)^4$ et $(X+1)^4$ qui sont premiers entre eux, et on obtient 
$$\frac{A}{2}=\frac{5}{32}X^3+\frac{5}{8}X^2+\frac{29}{32}X+\frac{1}{2}$$
$$\frac{-B}{2}=-\frac{5}{32}X^3+\frac{5}{8}X^2-\frac{29}{32}X+\frac{1}{2}$$
On a alors par construction
$$(X-1)^4A-1=2\bigg(1+(X+1)^4\frac{-B}{2}\bigg)=1+(X+1)^4B$$
et $P_0=(X-1)^4A-1$ convient. En remplaçant, on obtient après calculs :
$$P_0 = \frac{5}{16}X^7-\frac{21}{16}X^5+\frac{35}{16}X^3-\frac{35}{16}X$$}
    \item \question{en considérant le polynôme dérivé $P_0'$ et en cherchant un polynôme de degré minimal.}
\reponse{Si $(X-1)^4$ divise $P+1$, alors $1$ est racine de multiplicité au moins $4$
de $P+1$, et donc racine de multiplicité au moins $3$ de $P'$ : 
alors $(X-1)^3$ divise $P'$. De même $(X+1)^3$ divise $P'$. 
Comme  $(X-1)^3$ et  $(X+1)^3$ sont premiers entre eux, nécessairement $(X-1)^3(X+1)^3$ divise $P'$. 
Cherchons un polynôme de degré minimal : on remarque que les primitives de 
$$\lambda(X-1)^3(X+1)^3=\lambda(X^2-1)^3=\lambda(X^6-3X^4+3X^2-1)$$
sont de la forme $P(X)=\lambda(\frac{1}{7}X^7-\frac{3}{5}X^5+X^3-X+a)$. 
Si $P$ convient, nécessairement $1$ est racine de $P+1$ et $-1$ est racine de $P-1$, ce qui donne
$\lambda(\frac{-16}{35}+a)=-1$ et $\lambda(\frac{16}{35}+a)=1$. 
D'où $\lambda a=0$ et comme on cherche $P$ non nul, il faut $a=0$ et $\lambda=\frac{35}{16}$. 
On vérifie que 
$$P_0(X)=\frac{35}{16}(\frac{1}{7}X^7-\frac{3}{5}X^5+X^3-X)=\frac{5}{16}X^7-\frac{21}{16}X^5
+\frac{35}{16}X^3-\frac{35}{16}X$$
 est bien solution du problème: le polynôme $A=P_0+1$ admet $1$ comme racine, i.e. $A(1)=0$, 
 et sa dérivée admet $1$ comme racine triple donc $A'(1)=A''(1)=A'''(1)=0$, 
 ainsi $1$ est racine de multiplicité au moins $4$ de $A$ et donc $(X-1)^4$ divise $A=P+1$. 
 De même, $(X+1)^4$ divise $P-1$.}
\end{enumerate}
}

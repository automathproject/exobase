\uuid{cYRu}
\exo7id{2923}
\titre{exo7 2923}
\auteur{quercia}
\organisation{exo7}
\datecreate{2010-03-08}
\isIndication{false}
\isCorrection{true}
\chapitre{Dénombrement}
\sousChapitre{Cardinal}
\module{Algèbre}
\niveau{L1}
\difficulte{}

\contenu{
\texte{
Soient $x_1,\dots,x_n$ $n$ r{\'e}els. Pour calculer la somme $x_1+\dots+x_n$,
on place des parenth{\`e}ses de fa{\c c}on {\`a} n'avoir que des additions de deux nombres
{\`a} effectuer. Soit $t_n$ le nombre de mani{\`e}res de placer les parenth{\`e}ses
(on pose $t_1 = 1$).
}
\begin{enumerate}
    \item \question{D{\'e}terminer $t_2,t_3,t_4$.}
\reponse{1,2,5.}
    \item \question{Trouver une relation de r{\'e}currence entre $t_n$ et $t_1,\dots,t_{n-1}$.}
\reponse{$t_n = \sum_{k=1}^{n-1}\,t_kt_{n-k}$.}
\end{enumerate}
}

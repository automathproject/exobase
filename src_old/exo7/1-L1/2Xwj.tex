\uuid{2Xwj}
\exo7id{5080}
\titre{exo7 5080}
\auteur{rouget}
\organisation{exo7}
\datecreate{2010-06-30}
\isIndication{false}
\isCorrection{true}
\chapitre{Nombres complexes}
\sousChapitre{Trigonométrie}
\module{Algèbre}
\niveau{L1}
\difficulte{}

\contenu{
\texte{
Calculer les sommes suivantes~:
}
\begin{enumerate}
    \item \question{$\sum_{k=0}^{n}\cos(kx)$ et $\sum_{k=0}^{n}\sin(kx)$, ($x\in\Rr$ et $n\in\Nn$ donnés).}
    \item \question{$\sum_{k=0}^{n}\cos^2(kx)$ et $\sum_{k=0}^{n}\sin^2(kx)$, ($x\in\Rr$ et $n\in\Nn$ donnés).}
    \item \question{$\sum_{k=0}^{n}\dbinom{n}{k}\cos(kx)$ et $\sum_{k=0}^{n}\dbinom{n}{k}\sin(kx)$, ($x\in\Rr$ et $n\in\Nn$
donnés).}
\reponse{
Soient $n\in\Nn$ et $x\in\Rr$. Posons $S_n=\sum_{k=0}^{n}\cos(kx)$ et $S_n'=\sum_{k=0}^{n}\sin(kx)$.
\begin{itemize}
[\textbf{1ère solution.}]

$$S_n+iS_n'=\sum_{k=0}^{n}(\cos(kx)+i\sin(kx))=\sum_{k=0}^{n}e^{ikx}=\sum_{k=0}^{n}(e^{ix})^k.$$
Maintenant, $e^{ix}=1\Leftrightarrow x\in2\pi\Zz$. Donc,

\begin{itemize}
[\textbf{1er cas.}] Si $x\in2\pi\Zz$, on a immédiatement $S_n=n+1$ et $S_n'=0$.
[\textbf{2ème cas.}] Si x$\notin2\pi\Zz$,

\begin{align*}
S_n+iS_n'&=\frac{1-e^{i(n+1)x}}{1-e^{ix}}=\frac{e^{i(n+1)x/2}}{e^{ix/2}}\frac{e^{-i(n+1)x/2}-e^{i(n+1)x/2}}
{e^{-i(n+1)x/2}+e^{i(n+1)x/2}}=e^{inx/2}\frac{-2i\sin\frac{(n+1)x}{2}}{-2i\sin\frac{x}{2}}\\
 &=e^{inx/2}\frac{\sin\frac{(n+1)x}{2}}{\sin\frac{x}{2}}
\end{align*}
Par identification des parties réelles et imaginaires, on obtient

\begin{center}
\shadowbox{
$
\sum_{k=0}^{n}\cos(kx)=
\left\{
\begin{array}{l}
\frac{\cos\frac{nx}{2}\sin\frac{(n+1)x}{2}}{\sin\frac{x}{2}}\;\mbox{si}\;x\notin2\pi\Zz\\
n+1\;\mbox{si}\;x\in2\pi\Zz
\end{array}
\right.
\;\mbox{et}\;\sum_{k=0}^{n}\sin(kx)=
\left\{
\begin{array}{l}
\frac{\sin\frac{nx}{2}\sin\frac{(n+1)x}{2}}{\sin\frac{x}{2}}\;\mbox{si}\;x\notin2\pi\Zz\\
0\;\mbox{si}\;x\in2\pi\Zz
\end{array}
\right.
$
}
\end{center}

\end{itemize}
[\textbf{2ème solution.}]
\begin{align*}
2\sin\frac{x}{2}\sum_{k=0}^{n}\cos(kx)&=\sum_{k=0}^{n}2\sin\frac{x}{2}\cos(kx)=\sum_{k=0}^{n}(\sin(k+\frac{1}{2})x-\sin(
k-\frac{1}{2})x)\\
 &=\left(\sin\frac{x}{2}-\sin\frac{-x}{2}\right)+\left(\sin\frac{3x}{2}-\sin\frac{x}{2}\right)+\ldots+\left(\sin\frac{(2n-1)x}{2}-\sin\frac{(2n-3)x}{2}\right)\\
  &+\left(\sin\frac{(2n+1)x}{2}-\sin\frac{(2n-1)x}{2}\right)\\
 &=\sin\frac{(2n+1)x}{2}+\sin\frac{x}{2}=2\sin\frac{(n+1)x}{2}\cos\frac{nx}{2}
\end{align*}
et donc, si $x\notin2\pi\Zz$,...

\end{itemize}
Soient $n\in\Nn$ et $x\in\Rr$. Posons $S_n=\sum_{k=0}^{n}\cos^2(kx)$ et $S_n'=\sum_{k=0}^{n}\sin^2(kx)$. On
a~:

$$S_n+S_n'=\sum_{k=0}^{n}(\cos^2(kx)+\sin^2(kx))=\sum_{k=0}^{n}1=n+1,$$

et

$$S_n-S_n'=\sum_{k=0}^{n}(\cos^2(kx)-\sin^2(kx))=\sum_{k=0}^{n}\cos(2kx).$$

D'après 1), si $x\in\pi\Zz$, on trouve immédiatement,

$$\sum_{k=0}^{n}\cos^2(kx)=n+1\;\mbox{et}\;\sum_{k=0}^{n}\sin^2(kx)=0,$$

et si $x\notin\pi\Zz$,

$$S_n+S_n'=n+1\;\mbox{et}\;S_n-S_n'=\frac{\cos(nx)\sin(n+1)x}{\sin x},$$

de sorte que

$$S_n=\frac{1}{2}\left(n+1+\frac{\cos(nx)\sin(n+1)x}{\sin
x}\right)\;\mbox{et}\;S_n'=\frac{1}{2}\left(n+1-\frac{\cos(nx)\sin(n+1)x}{\sin x}\right).$$
Soient $n\in\Nn$ et $x\in\Rr$.

\begin{align*}
\left(\sum_{k=0}^{n}C_n^k\cos(kx)\right)+i\left(\sum_{k=0}^{n}C_n^k\sin(kx)\right)&=\sum_{k=0}^{n}C_n^ke^{ikx}
=\sum_{k=0}^{n}C_n^k(e^{ix})^k1^{n-k}\\
 &=(1+e^{ix})^n=(e^{ix/2}+e^{-ix/2})^ne^{inx/2}=2^n\cos^n\left(\frac{x}{2}\right)\left(\cos\frac{nx}{2}+i\sin\frac{nx}{2}\right).
\end{align*}
Par identification des parties réelles et imaginaires, on obtient alors

\begin{center}
\shadowbox{
$\sum_{k=0}^{n}C_n^k\cos(kx)=2^n\cos^n\left(\frac{x}{2}\right)\cos\left(\frac{nx}{2}\right)\;\mbox{et}\;\sum_{k=0}^{n}C_n^k\sin(kx)=
2^n\cos^n\left(\frac{x}{2}\right)\sin\left(\frac{nx}{2}\right).$
}
\end{center}
}
\end{enumerate}
}

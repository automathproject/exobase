\uuid{kIJF}
\exo7id{30}
\titre{exo7 30}
\auteur{bodin}
\organisation{exo7}
\datecreate{1998-09-01}
\isIndication{false}
\isCorrection{true}
\chapitre{Nombres complexes}
\sousChapitre{Racine carrée, équation du second degré}
\module{Algèbre}
\niveau{L1}
\difficulte{}

\contenu{
\texte{
Montrer que les solutions de $az^2+bz+c=0$ avec $a$,
$b$, $c$ r\'eels, sont r\'eelles ou conjugu\'ees.
}
\reponse{
Soit $P(z) = az^2+bz+c$, et $\Delta = b^2-4ac$, si $\Delta \geq 0$
alors les racines sont r\'eelles, seul le cas o\`u $\Delta < 0$
nous int\'eresse. Premi\`ere m\'ethode : il suffit de regarder les
deux solutions et de v\'erifier qu'elles sont conjugu\'ees...

Seconde m\'ethode : si $z$ est une racine de $P$ \emph{i.e.} $P(z)
= 0$, alors
$$ P(\overline{z}) = a{\overline{z}}^2+b\overline{z}+c =
\overline{az^2+b^z+c} = \overline{P(z)} = 0.$$ Donc $\overline{z}$
est aussi une racine de $P$. Or $z$ n'est pas un nombre r\'eel
(car $\Delta < 0$ ) donc $\overline{z} \not= z$. Sachant que le
polyn\^ome $P$ de degr\'e $2$ a exactement $2$ racines, ce sont
$z$ et $\overline{z}$ et elles sont conjugu\'ees.
}
}

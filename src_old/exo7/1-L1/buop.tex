\uuid{buop}
\exo7id{3252}
\titre{exo7 3252}
\auteur{quercia}
\organisation{exo7}
\datecreate{2010-03-08}
\isIndication{false}
\isCorrection{true}
\chapitre{Polynôme, fraction rationnelle}
\sousChapitre{Racine, décomposition en facteurs irréductibles}
\module{Algèbre}
\niveau{L1}
\difficulte{}

\contenu{
\texte{
Soit $P\in\Z[X]$, $P=X^n+a_{n-1}X^{n-1}+\dots+a_0X^0$ et $p$ un
nombre premier tel que~:
$$a_0\equiv 0 (\mathrm{mod}\, p),\quad\dots,\quad a_{n-1}\equiv 0 (\mathrm{mod}\, p),\quad
a_0\not\equiv 0 (\mathrm{mod}\, {p^2}).$$
Montrer que $P$ est irr{\'e}ductible dans $\Z[X]$.
}
\reponse{
Soit $P = QR$ avec
$Q=X^{n_1}+b_{n_1-1}X^{n_1-1}+\dots+b_0X^0$ et
$R=X^{n_2}+c_{n_2-1}X^{n_2-1}+\dots+c_0X^0$.

Par hypoth{\`e}se sur $a_0 = b_0c_0$, $p$ divise un et un seul des entiers
$b_0$, $c_0$.
Supposons que $p$ divise $b_0,b_1,\dots,b_{k-1}$~:
alors $a_k \equiv b_kc_0 (\mathrm{mod}\, p)$ donc $p$ divise $b_k$.
On aboutit {\`a} \og$p$ divise le coefficient dominant
de $Q$\fg, ce qui est absurde.
}
}

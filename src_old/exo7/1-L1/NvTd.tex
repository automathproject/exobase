\uuid{NvTd}
\exo7id{7199}
\titre{exo7 7199}
\auteur{megy}
\organisation{exo7}
\datecreate{2019-07-23}
\isIndication{false}
\isCorrection{false}
\chapitre{Logique, ensemble, raisonnement}
\sousChapitre{Relation d'équivalence, relation d'ordre}
\module{Algèbre}
\niveau{L1}
\difficulte{}

\contenu{
\texte{
(Clôture transitive. Cet exercice utilise la notion de produit de relations)
Soit $\mathcal R$ une relation sur $E$. Pour $n\in \N$ et $\mathcal R$ est une relation sur $E$, on définit alors par récurrence la relation $\mathcal R^n$ (en définissant $\mathcal R^0$ comme l'égalité, puis $\mathcal R^{n+1} = \mathcal R \mathcal R^n$).

Montrer que toutes les relations suivantes sont égales:
}
\begin{enumerate}
    \item \question{$\bigvee_{n\geq 0} \mathcal R^n$;}
    \item \question{la relation dont le graphe est $\bigcup_{n\geq 0} \Gamma_{\mathcal R^{n}}$;}
    \item \question{la relation dont le graphe est l'intersection de tous les graphes de relations transitives qui contiennent $\Gamma_{\mathcal R}$.}
    \item \question{la plus fine relation  parmi toutes les relations transitives moins fines que $\mathcal R$.}
\end{enumerate}
}

\uuid{81hN}
\exo7id{65}
\titre{exo7 65}
\auteur{cousquer}
\organisation{exo7}
\datecreate{2003-10-01}
\isIndication{false}
\isCorrection{true}
\chapitre{Nombres complexes}
\sousChapitre{Géométrie}
\module{Algèbre}
\niveau{L1}
\difficulte{}

\contenu{
\texte{
\label{exo:compl}
 D\'eterminer par le calcul et g\'eom\'etriquement les nombres complexes $z$ tels
que $\Bigl\vert\frac{z-3}{z-5}\Bigr\vert=1$. G\'en\'eraliser pour
$\Bigl\vert\frac{z-a}{z-b}\Bigr\vert=1$.
}
\reponse{
En exprimant qu'un nombre complexe de module 1 peut s'\'ecrire $e^{i\theta
}$, on trouve $z={a-be^{i\theta }\over 1-e^{i\theta }}$. On peut encore \'ecrire
$z=A+B\cot{\theta \over 2}$, o\`u $A$ et $B$ sont ind\'ependants de $\theta $, ce qui
montre que le point d'affixe $z$ d\'ecrit une droite. G\'eom\'etriquement, cette droite
est bien entendu la m\'ediatrice du segment qui joint les points d'affixes $a$
et $b$.
}
}

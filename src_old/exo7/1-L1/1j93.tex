\uuid{1j93}
\exo7id{6971}
\titre{exo7 6971}
\auteur{blanc-centi}
\organisation{exo7}
\datecreate{2014-04-08}
\video{FPyfOQL32Hg}
\isIndication{false}
\isCorrection{true}
\chapitre{Polynôme, fraction rationnelle}
\sousChapitre{Fraction rationnelle}
\module{Algèbre}
\niveau{L1}
\difficulte{}

\contenu{
\texte{
On pose $Q_0=(X-1)(X-2)^2$, $Q_1=X(X-2)^2$ et $Q_2=X(X-1)$. 
\`A l'aide de la décomposition en éléments simples de $\frac{1}{X(X-1)(X-2)^2}$, 
trouver des polynômes $A_0,\ A_1,\ A_2$ tels que $A_0Q_0+A_1Q_1+A_2Q_2=1$. 
Que peut-on en déduire sur $Q_1$, $Q_2$ et $Q_3$?
}
\reponse{
La décomposition en élément simple s'écrit :
$$\frac{1}{X(X-1)(X-2)^2}=\frac{-\frac14}{X}+\frac{1}{X-1}+\frac{\frac12}{(X-2)^2}+\frac{-\frac34}{X-2}.$$ 
En multipliant cette identité par le dénominateur $X(X-1)(X-2)^2$, il vient :
$$1 =-\tfrac{1}{4}Q_0+Q_1+\left(\tfrac{1}{2}-\tfrac{3}{4}(X-2)\right)Q_2.$$
Ainsi $A_0=-\frac{1}{4}$, $A_1=1$ et $A_1=(2-\frac{3}{4}X)$ conviennent. 
On a obtenu une relation de Bézout entre $Q_1$, $Q_2$ et $Q_3$ qui prouve que 
ces trois polynômes sont premiers dans leur ensemble : $\pgcd(Q_1,Q_2,Q_3)=1$.
}
}

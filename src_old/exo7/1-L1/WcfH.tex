\uuid{WcfH}
\exo7id{5627}
\titre{exo7 5627}
\auteur{rouget}
\organisation{exo7}
\datecreate{2010-10-16}
\isIndication{false}
\isCorrection{true}
\chapitre{Matrice}
\sousChapitre{Changement de base, matrice de passage}
\module{Algèbre}
\niveau{L1}
\difficulte{}

\contenu{
\texte{
Soient $A$ et $B$ deux éléments de $\mathcal{M}_n(\Rr)$. Montrer que si $A$ et $B$ sont semblables dans $\mathcal{M}_n(\Cc)$, elles le sont dans $\mathcal{M}_n(\Rr)$.
}
\reponse{
Soient $A$ et $B$ deux matrices carrées réelles de format $n$ semblables dans $\mathcal{M}_n(\Cc)$.

Il existe $P$ élément de $\mathcal{GL}_n(\Cc)$ telle que $PB =AP$ (bien plus manipulable que $B=P^{-1}AP$).

Posons $P=Q+iR$ où $Q$ et $R$ sont des matrices réelles. 
Par identification des parties réelles et imaginaires, on a $QB=AQ$ et $RB=AR$ mais cet exercice n'en est pas pour autant achevé car $Q$ ou $R$ n'ont aucune raison d'être inversibles.

On a $QB=AQ$ et $RB= AR$ et donc plus généralement pour tout réel $x$, $(Q+xR)B=A(Q+xR)$.

Maintenant, $\text{det}(Q+xR)$ est un polynôme à coefficients réels en $x$ mais n'est pas le polynôme nul car sa valeur en $i$ (tel que $i^2 = -1$) est $\text{det}P$ qui est non nul. Donc il n'existe qu'un nombre fini de réels $x$, éventuellement nul, tels que $\text{det}(Q+xR) = 0$. En particulier, il existe au moins un réel $x_0$ tel que la matrice $P_0=Q+x_0R$ soit inversible. $P_0$ est une matrice réelle inversible telle que $P_0A=BP_0$ et $A$ et $B$ sont bien semblables dans $\mathcal{M}_n(\Rr)$.
}
}

\uuid{NFAE}
\exo7id{5266}
\titre{exo7 5266}
\auteur{rouget}
\organisation{exo7}
\datecreate{2010-07-04}
\isIndication{false}
\isCorrection{true}
\chapitre{Matrice}
\sousChapitre{Autre}
\module{Algèbre}
\niveau{L1}
\difficulte{}

\contenu{
\texte{
Soit $A\in\mathcal{M}_{3,2}(\Rr)$ et $B\in\mathcal{M}_{2,3}(\Rr)$ telles que~:

$$AB=
\left(
\begin{array}{ccc}
0&-1&-1\\
-1&0&-1\\
1&1&2
\end{array}
\right)
.$$

Montrer l'existence d'au moins un couple $(A,B)$ vérifiant les conditions de l'énoncé puis calculer $BA$. (Indication. Calculer $(AB)^2$ et utiliser le rang.)
}
\reponse{
Soit $(i,j)$ la base canonique de $\Rr^2$ et $(e_1,e_2,e_3)$ la base canonique de $\Rr^3$. On cherche $f\in\mathcal{L}(\Rr^2,\Rr^3)$ et $g\in\mathcal{L}(\Rr^3,\Rr^2)$ tels que

$$f\circ g(e_1)=-e_2+e_3,\;f\circ g(e_2)=-e_1+e_3\;\mbox{et}\;f\circ g(e_3)=-e_1-e_2+2e_3(=f\circ g(e_1+e_2)).$$

On pose $g(e_1)=i$, $g(e_2)=j$ et $g(e_3)=i+j$, puis $f(i)=-e_2+e_3$ et $f(j)=-e_1+e_3$. Les applications linéaires $f$ et $g$ conviennent, ou encore si on pose

$$A=\left(
\begin{array}{cc}
0&-1\\
-1&0\\
1&1
\end{array}
\right)\;\mbox{et}\;B=\left(
\begin{array}{ccc}
1&0&1\\
0&1&1
\end{array}
\right),$$

alors $AB=\left(
\begin{array}{cc}
0&-1\\
-1&0\\
1&1
\end{array}
\right)\left(
\begin{array}{ccc}
1&0&1\\
0&1&1
\end{array}
\right)=\left(
\begin{array}{ccc}
0&-1&-1\\
-1&0&-1\\
1&1&2
\end{array}
\right)$.

$A$ et $B$ désignent maintenant deux matrices quelconques, éléments de $\mathcal{M}_{3,2}(\Rr)$ et $\mathcal{M}_{2,3}(\Rr)$ respectivement, telles que $AB=\left(
\begin{array}{ccc}
0&-1&-1\\
-1&0&-1\\
1&1&2
\end{array}
\right)$. Calculons $(AB)^2$. On obtient

$$(AB)^2=\left(
\begin{array}{ccc}
0&-1&-1\\
-1&0&-1\\
1&1&2
\end{array}
\right)\left(
\begin{array}{ccc}
0&-1&-1\\
-1&0&-1\\
1&1&2
\end{array}
\right)=\left(
\begin{array}{ccc}
0&-1&-1\\
-1&0&-1\\
1&1&2
\end{array}
\right)=AB.$$

Mais alors, en multipliant les deux membres de cette égalité par $B$ à gauche et $A$ à droite, on obtient

$$(BA)^3=(BA)^2\;(*).$$

Notons alors que 

$$\mbox{rg}(BA)\geq\mbox{rg}(ABAB)=\mbox{rg}((AB)^2)=\mbox{rg}(AB)=2,$$

et donc, $BA$ étant une matrice carrée de format $2$, $\mbox{rg}(BA)=2$. $BA$ est donc une matrice inversible. Par suite, on peut simplifier les deux membres de l'égalité $(*)$ par $(BA)^2$ et on obtient $BA=I_2$.
}
}

\uuid{Cgx5}
\exo7id{956}
\titre{exo7 956}
\auteur{legall}
\organisation{exo7}
\datecreate{1998-09-01}
\video{DY3GrL-j6C4}
\isIndication{false}
\isCorrection{true}
\chapitre{Application linéaire}
\sousChapitre{Image et noyau, théorème du rang}
\module{Algèbre}
\niveau{L1}
\difficulte{}

\contenu{
\texte{
Pour les applications lin\'eaires suivantes,  d\'eterminer $\ker f_i$ et 
$\Im f_i$. En d\'eduire si $f_i$ est injective, surjective, bijective.
$$\begin{array}{rl}
f_1 : \Rr^2 \to \Rr^2 & f_1(x,y)=(2x+y,x-y)  \\
f_2 : \Rr^3 \to \Rr^3 & f_2(x,y,z)=(2x+y+z,y-z,x+y) \\
f_3 : \Rr^2 \to \Rr^4 & f_3(x,y)=(y,0,x-7y,x+y) \\
f_4 : \Rr_3[X] \to \Rr^3 & f_4(P) = \big( P(-1), P(0), P(1) \big) \\
\end{array}
$$
}
\reponse{
$f_1$ est injective, surjective (et donc bijective).

  \begin{enumerate}
Faisons tout à la main. Calculons le noyau :
\begin{align*}
(x,y) \in \ker f_1
  & \iff f_1(x,y) = (0,0) 
  \iff (2x+y,x-y) = (0,0) \\
  &\iff \begin{cases}
         2x+y=0 \\
        x-y=0 \\   
        \end{cases}
  \iff (x,y)=(0,0) \\
\end{align*}
Ainsi $\ker f_1 = \{ (0,0) \}$ et donc $f_1$ est injective.
Calculons l'image. Quels éléments $(X,Y)$ peuvent s'écrire $f_1(x,y)$ ?
\begin{align*}
f_1(x,y) = (X,Y) 
  & \iff (2x+y,x-y) = (X,Y) \\
  &\iff \begin{cases}
         2x+y=X \\
         x-y=Y \\   
        \end{cases} 
  \iff \begin{cases}
         x=\frac{X+Y}{3} \\
         y=\frac{X-2Y}{3} \\   
        \end{cases} \\
  & \iff (x,y)=\left(\frac{X+Y}{3},\frac{X-2Y}{3}\right) \\
\end{align*}
Donc pour n'importe quel $(X,Y)\in\Rr^2$ on trouve un antécédent $(x,y)=(\frac{X+Y}{3},\frac{X-2Y}{3})$
qui vérifie donc $f_1(x,y)=(X,Y)$. Donc $\Im f_1 = \Rr^2$. Ainsi $f_1$ est surjective.
Conclusion : $f_1$ est injective et surjective donc bijective.
}
}

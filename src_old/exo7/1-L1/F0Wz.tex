\uuid{F0Wz}
\exo7id{2781}
\titre{exo7 2781}
\auteur{tumpach}
\organisation{exo7}
\datecreate{2009-10-25}
\isIndication{false}
\isCorrection{false}
\chapitre{Espace vectoriel}
\sousChapitre{Système de vecteurs}
\module{Algèbre}
\niveau{L1}
\difficulte{}

\contenu{
\texte{
Les familles suivantes sont-elles libres ?
}
\begin{enumerate}
    \item \question{$\vec{v_1} = (1,0,1)$, $\vec{v_2} = (0,2,2)$ et $\vec{v_3} = (3,7,1)$ dans $\mathbb{R}^3$.}
    \item \question{$\vec{v_1}= (1,0,0)$, $\vec{v_2} = (0,1,1)$ et $\vec{v_3}= (1,1,1)$ dans $\mathbb{R}^3$.}
    \item \question{$\vec{v_1} = (1,2,1,2,1)$, $\vec{v_2} = (2,1,2,1,2)$, $\vec{v_3} = (1,0,1,1,0)$ 
et $\vec{v_4} = (0,1,0,0,1)$ dans $\mathbb{R}^5$.}
    \item \question{$\vec{v_1} = (2,4,3,-1,-2,1)$, $\vec{v_2} = (1,1,2,1,3,1)$ et 
$\vec{v_3} = (0,-1,0,3,6,2)$ dans $\mathbb{R}^6$.}
    \item \question{$\vec{v_1} = (2,1,3,-1,4,-1)$, $\vec{v_2} = (-1,1,-2,2,-3,3)$ et 
$\vec{v_3} = (1,5,0,4,-1,7)$ dans $\mathbb{R}^6$.\\}
\end{enumerate}
}

\uuid{pD4D}
\exo7id{5193}
\titre{exo7 5193}
\auteur{rouget}
\organisation{exo7}
\datecreate{2010-06-30}
\isIndication{false}
\isCorrection{true}
\chapitre{Application linéaire}
\sousChapitre{Morphismes particuliers}
\module{Algèbre}
\niveau{L1}
\difficulte{}

\contenu{
\texte{
Soient $\Kk$ un sous-corps de $\Cc$ et $E$ un $\Kk$-espace vectoriel de dimension finie notée $n$. Soit $u$ un endomorphisme de $E$.
On dit que $u$ est nilpotent si et seulement si $\exists k\in\Nn^*/\;u^k=0$ et on appelle alors indice de
nilpotence de $u$ le plus petit de ces entiers $k$ (par exemple, le seul endomorphisme $u$, nilpotent d'indice $1$ est
$0$).
}
\begin{enumerate}
    \item \question{Soit $u$ un endomorphisme nilpotent d'indice $p$. Montrer qu'il existe un vecteur $x$ de $E$ tel que la
famille

$(x,\;u(x),...,\;u^{p-1}(x))$ soit libre.}
\reponse{Soit $p(\in\Nn^*)$ l'indice de nilpotence de $u$.

Par définition, $u^{p-1}\neq0$ et plus généralement, pour $1\leq k\leq p-1$, $u^k\neq0$ car si $u^k= 0$ alors $u^{p-1}=u^k\circ u^{p-1-k}=0$ ce qui n'est pas.

Puisque $u^{p-1}\neq0$, il existe au moins un vecteur $x$ non nul tel que $u^{p-1}(x)\neq0$.

Montrons que la famille $(u^k(x))_{0\leq k\leq p-1}$ est libre.

Soit $(\lambda_k)_{0\leq k\leq p-1}\in\Kk^p$ tel que $\sum_{k=0}^{p-1}\lambda_ku^k(x)=0$. Supposons qu'au moins un des coefficients $\lambda_k$ ne soit pas nul. Soit $i=\mbox{Min }\{k\in\{0,...,p-1\}/\;\lambda_k\neq0\}$.

\begin{align*}
\sum_{k=0}^{p-1}\lambda_ku^k(x)=0&\Rightarrow\sum_{k=i}^{p-1}\lambda_ku^k(x)=0
\Rightarrow u^{p-1-i}(\sum_{k=i}^{p-1}\lambda_ku^k(x))=0\Rightarrow\sum_{k=i}^{p-1}\lambda_ku^{p-1-i+k}(x)=0\\
 &\Rightarrow\lambda_iu^{p-1}(x)=0\quad(\mbox{car pour}\;k\geq i+1,\;p-1-i+k\geq p\;\mbox{et donc}\;u^{p-1-i+k}=0)\\
 &\Rightarrow\lambda_i=0\quad(\mbox{car}\;u^{p-1}(x)\neq0)
\end{align*} 

ce qui contredit la définition de $i$.

Donc tous les coefficients $\lambda_k$ sont nuls et on a montré que la famille $(u^k(x))_{0\leq k\leq p-1}$ est libre.}
    \item \question{Soit $u$ un endomorphisme nilpotent. Montrer que $u^n=0$.}
\reponse{Le cardinal d'une famille libre est inférieur ou égal à la dimension de l'espace et donc $p\leq n$. 
Par suite, $u^n=u^p\circ u^{n-p}=0$.}
    \item \question{On suppose dans cette question que $u$ est nilpotent d'indice $n$. Déterminer $\mbox{rg}u$.}
\reponse{On applique l'exerice \ref{exo:suprou10}.

Puisque $u^{n-1}\neq0$, on a $N_{n-1}\underset{\neq}{\subset}N_n$.
Par suite (d'après l'exercice \ref{exo:suprou12}, 2), c)), les inclusions $N_0\subset N_1\subset...\subset N_n=E$ sont toutes strictes et donc 

$$0<\mbox{dim }N_1<\mbox{dim }N_2 ...<\mbox{dim }N_n=n.$$

Pour $k\in\{0,...,n\}$, notons $d_k$ est la dimension de $N_k$. Pour $k\in\{0,...,n-1\}$, on a $d_{k+1}\geq d_k$ et une récurrence facile montre que, pour $k\in\{0,...,n\}$, on a $d_k\geq k$.

Mais si de plus, pour un certain indice $i$ élément de $\{1,...,n-1\}$, on a $d_i=\mbox{dim }N_i>i$, alors, par une récurrence facile, pour $i\leq k\leq n$, on a $d_k>k$ et en particulier $d_n>n$ ce qui n'est pas. Donc,

$$\forall k\in\{0,...,n\},\;\mbox{dim }(N_k)=k,$$

ou encore, d'après le théorème du rang,

$\forall k\in\{0,...,n\},\;\mbox{rg }(u^k)=n-k$, et en particulier $\mbox{rg }(u)=n-1$.}
\end{enumerate}
}

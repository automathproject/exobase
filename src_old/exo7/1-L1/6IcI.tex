\uuid{6IcI}
\exo7id{5194}
\titre{exo7 5194}
\auteur{rouget}
\organisation{exo7}
\datecreate{2010-06-30}
\isIndication{false}
\isCorrection{true}
\chapitre{Application linéaire}
\sousChapitre{Image et noyau, théorème du rang}
\module{Algèbre}
\niveau{L1}
\difficulte{}

\contenu{
\texte{
\label{exo:suprou12}
Soient $\Kk$ un sous-corps de $\Cc$, $E$ un $\Kk$-espace vectoriel de dimension quelconque sur $\Kk$ et $f$ un endomorphisme de $E$ vérifiant
$f^2-5f+6Id_E=0$. Montrer que $E=\mbox{Ker}(f-2Id)\oplus\mbox{Ker}(f-3Id)$.
}
\reponse{
Soit $x\in E$.

$$x\in\mbox{Ker }(f-2Id)\cap\mbox{Ker }(f-3Id)\Rightarrow f(x)=2x\;\mbox{et}\;f(x)=3x\Rightarrow 3x-2x=f(x)-f(x)= 0\Rightarrow x=0.$$

Donc, $\mbox{Ker }(f-2Id)\cap\mbox{Ker }(f-3Id)=\{0\}$ (même si $f^2-5f+6Id\neq0$).

Soit $x\in E$. On cherche $y$ et $z$ tels que $y\in\mbox{Ker }(f-2Id)$, $z\in\mbox{Ker }(f-3Id)$ et $x=y+z$.

Si $y$ et $z$ existent, $y$ et $z$ sont solution du système $\left\{
\begin{array}{l}
y+z=x\\
2y+3z=f(x)
\end{array}
\right.$ et donc $\left\{
\begin{array}{l}
y=3x-f(x)\\
z=f(x)-2x
\end{array}
\right.$.

Réciproquement . Soient $x\in E$ puis $y=3x-f(x)$ et $z=f(x)-2x$.

On a bien $y+z=x$ puis

\begin{align*}
f(y)&=3f(x)-f^2(x)=3f(x)-(5f(x)-6x)\quad(\mbox{car}\;f^2=5f-6Id)\\
 &= 6x-2f(x)=2(3x-f(x))=2y
\end{align*}  
  
et $y\in\mbox{Ker }(f-2Id)$. De même,

$$f(z)=f^2(x)-2f(x)=(5f(x)-6x)-2f(x)=3(f(x)-2x)=3z,$$
 
et $z\in\mbox{Ker }(f-3Id)$. On a montré que $E=\mbox{Ker }(f-2Id)+\mbox{Ker }(f-3d)$, et finalement que 

$$E=\mbox{Ker }(f-2Id)\oplus\mbox{Ker }(f-3d).$$
}
}

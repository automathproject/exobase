\uuid{inSn}
\exo7id{5181}
\titre{exo7 5181}
\auteur{rouget}
\organisation{exo7}
\datecreate{2010-06-30}
\isIndication{false}
\isCorrection{true}
\chapitre{Application linéaire}
\sousChapitre{Image et noyau, théorème du rang}
\module{Algèbre}
\niveau{L1}
\difficulte{}

\contenu{
\texte{
Soit $E$ un $\Kk$-espace vectoriel et soit $(u,v)\in(\mathcal{L}(E))^2$.
}
\begin{enumerate}
    \item \question{Montrer que $[\mbox{Ker}v\subset\mbox{Ker}u\Leftrightarrow\exists w\in\mathcal{L}(E)/\;u=w\circ v]$.}
    \item \question{En déduire que $[v\;\mbox{injectif}\Leftrightarrow\exists w\in\mathcal{L}(E)/\;w\circ v=Id_E]$.}
\reponse{
\begin{itemize}
[$\Leftarrow$] Soit $(u,v)((\mathcal{L}(E))^2$. On suppose qu'il existe $w\in\mathcal{L}(E)$ tel que $u=w\circ
v$. Soit $x$ un élément de $\mbox{Ker}v$. Alors $v(x)=0$ et donc $u(x)=w(v(x))=w(0)=0$. Mais alors, $x$ est dans
$\mbox{Ker}u$. Donc $\mbox{Ker}v\subset{Ker}u$.
[$\Rightarrow$] Supposons que $\mbox{Ker}v\subset\mbox{Ker}u$. On cherche à définir $w$, élément de
$\mathcal{L}(E)$ tel que $w\circ v=u$. Il faut définir précisément $w$ sur $\mbox{Im}v$ car sur $E\setminus\mbox{Im}v$,
on a aucune autre contrainte que la linéarité.

Soit $y$ un élément de $\mbox{Im}v$. (Il existe $x$ élément de $E$ tel que $y=v(x)$. On
a alors envie de poser $w(y)=u(x)$ mais le problème est que $y$, élément de $\mbox{Im}v$ donné peut avoir plusieurs
antécédents $x$, $x'$... et on peut avoir $u(x)\neq u(x')$ de sorte que l'on n'aurait même pas défini une application
$w$.)
\end{itemize}

Soient $x$ et $x'$ deux éléments de $E$ tels que $v(x)=v(x')=y$ alors $v(x-x')=0$ et donc
$x-x'\in\mbox{Ker}v\subset\mbox{Ker}u$. Par suite, $u(x-x')=0$ ou encore $u(x)=u(x')$. En résumé, pour $y$ élément donné
de $\mbox{Im}v$, il existe $x$ élément de $E$ tel que $v(x)=y$. On pose alors $w(y)=u(x)$ en notant que $w(y)$ est bien
uniquement défini, car ne dépend pas du choix de l'antécédent $x$ de $y$ par $v$. $w$ n'est pas encore défini sur $E$
tout entier. Notons $F$ un supplémentaire quelconque de $\mbox{Im}v$ dans $E$ (l'existence de $F$ est admise).

Soit $X$ un élément de $E$. Il existe deux vecteurs $y$ et $z$, de $\mbox{Im}v$ et $F$ respectivement, tels que $X=y+z$.
On pose alors $w(X)=u(x)$ où $x$ est un antécédent quelconque de $y$ par $v$ (on a pris pour restriction de $w$ à $F$
l'application nulle). $w$ ainsi définie est une application de $E$ dans $E$ car, pour $X$ donné $y$ est uniquement
défini puis $u(x)$ est uniquement défini (mais pas nécessairement $x$).

Soit $x$ un élément de $E$ et $y=v(x)$. $w(v(x))=w(y)=w(y+0)=u(x)$ (car 1)$y$ est dans $\mbox{Im}v$ 2)$0$ est dans $F$
3) $x$ est un antécédent de $y$ par $v$) et donc $w\circ v=u$.

Montrons que $w$ est linéaire. Soient, avec les notations précédentes, $X_1=y_1+z_1$ et $X_2=y_2+z_2$ ...

\begin{align*}
w(X_1+X_2)&=w((y_1+y_2)+(z_1+z_2))=u(x_1+x_2)\quad(\mbox{car}\;y_1+y_2=v(x_1)+v(x_2)=v(x_1+x2))\\
 &=u(x_1)+u(x_2)=w(X_1)+w(X_2)
\end{align*}

et

$$w(\lambda X)=w(\lambda y+\lambda z)=u(\lambda x)=\lambda u(x)=\lambda w(X).$$
On applique 1) à $u=Id$.

$$v\;\mbox{injective}\Leftrightarrow\mbox{Ker}v=\{0\}\Leftrightarrow\mbox{Ker}v=\mbox{Ker}Id\Leftrightarrow\exists w\in\mathcal{L}(E)/\;w\circ v=Id.$$
}
\end{enumerate}
}

\uuid{Ugfe}
\exo7id{5607}
\titre{exo7 5607}
\auteur{rouget}
\organisation{exo7}
\datecreate{2010-10-16}
\isIndication{false}
\isCorrection{true}
\chapitre{Matrice}
\sousChapitre{Noyau, image}
\module{Algèbre}
\niveau{L1}
\difficulte{}

\contenu{
\texte{
Rang de la matrice $(i+j+ij)_{1\leqslant i,j\leqslant n}$.
}
\reponse{
Pour $j\in\llbracket1,n\rrbracket$, notons $C_j$ la $j$-ème colonne de la matrice $A$. Posons encore $U=\left(
\begin{array}{c}
1\\
2\\
\vdots\\
n
\end{array}
\right)$  et $V=\left(
\begin{array}{c}
2\\
3\\
\vdots\\
n+1
\end{array}
\right)$. Pour $j\in\llbracket1,n\rrbracket$, on a

\begin{center} 
$Cj = (i+j(i+1))_{1\leqslant i\leqslant n}= (i)_{1\leqslant i\leqslant n}+j(i+1)_{1\leqslant i\leqslant n}=U+jV$.
\end{center}

Donc $\text{Vect}(C_1,...,C_n)\subset\text{Vect}(U,V)$ et en particulier, $\text{rg}A\leqslant 2$.
Maintenant, si $n\geqslant 2$, les deux premières colonnes de $A$ ne sont pas colinéaires car $\left|
\begin{array}{cc}
3&5\\
5&8
\end{array}
\right|=-1\neq0$. Donc, si $n\geqslant2$, $\text{rg}A=2$ et si $n=1$, $\text{rg}A =1$.

\begin{center}
\shadowbox{
Si $n\geqslant 2$, $\text{rg}(i+j+ij)_{1\leqslant i,j\leqslant n}=2$ et si $n=1$, $\text{rg}(i+j+ij)_{1\leqslant i,j\leqslant n}=1$.
}
\end{center}
}
}

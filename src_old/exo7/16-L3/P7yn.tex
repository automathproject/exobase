\uuid{P7yn}
\exo7id{6726}
\titre{exo7 6726}
\auteur{queffelec}
\organisation{exo7}
\datecreate{2011-10-16}
\isIndication{false}
\isCorrection{false}
\chapitre{Singularité}
\sousChapitre{Singularité}
\module{Analyse complexe}
\niveau{L3}
\difficulte{}

\contenu{
\texte{
Déterminer les points singuliers des fonctions suivantes, puis
donner la nature de ces points singuliers (singularité effaçable, pôle
d'ordre $n$, singularité essentielle isolée, accumulation de points
singuliers).
}
\begin{enumerate}
    \item \question{$$ z\mapsto {1\over z(z^2+4)^2} $$}
    \item \question{$$ z\mapsto {1\over \exp{(z)}-1}-{1\over z} $$}
    \item \question{$$ z\mapsto \sin{1\over 1-z}$$}
    \item \question{$$ z\mapsto \exp{z\over 1-z}$$}
    \item \question{$$ z\mapsto \mathrm{cotan} z-{1\over z}$$}
    \item \question{$$ z\mapsto \mathrm{cotan}{1\over z}$$}
    \item \question{$$ z\mapsto {1\over \sin z-\sin a}$$}
    \item \question{$$ z\mapsto \sin{\left({1\over \sin{1\over z}}\right) }$$}
    \item \question{$$ z\mapsto \exp{\left(\tan{1\over z}\right)} $$}
\end{enumerate}
}

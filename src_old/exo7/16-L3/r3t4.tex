\uuid{r3t4}
\exo7id{2825}
\titre{exo7 2825}
\auteur{burnol}
\organisation{exo7}
\datecreate{2009-12-15}
\isIndication{false}
\isCorrection{true}
\chapitre{Théorème des résidus}
\sousChapitre{Théorème des résidus}
\module{Analyse complexe}
\niveau{L3}
\difficulte{}

\contenu{
\texte{

}
\begin{enumerate}
    \item \question{Soit $f$ une fonction continue sur $\overline{D(0,1)}$,
holomorphe sur $D(0,1)$, nulle sur le cercle de rayon
$1$. Montrer que $f$ est identiquement nulle.}
\reponse{C'est le principe du maximum.}
    \item \question{Plus fort: on
ne suppose plus que $f(e^{i\theta})$ est nulle pour tout
$\theta$ mais seulement pour $0\leq\theta\leq\pi$. Montrer
que $f$ est identiquement nulle. 
\emph{Indication :} $f(z)f(-z)$.}
\reponse{La fonction $g(z) = f(z) f(-z)$ est nulle sur le cercle de rayon $1$.}
\end{enumerate}
}

\uuid{NYVg}
\exo7id{7610}
\titre{exo7 7610}
\auteur{mourougane}
\organisation{exo7}
\datecreate{2021-08-10}
\isIndication{false}
\isCorrection{true}
\chapitre{Autre}
\sousChapitre{Autre}
\module{Analyse complexe}
\niveau{L3}
\difficulte{}

\contenu{
\texte{
On rappelle la formule de Gutzmer : soit $f$ la somme de la série entière $\sum_{n\geq 0} a_n z^n$ de rayon de convergence $R$.
Alors, pour tout $r<R$,
$$\sum_{n\geq 0} |a_n|^2r^{2n}=\frac{1}{2\pi}\int_0^{2\pi}|f(re^{i\theta})|^2d\theta.$$
}
\begin{enumerate}
    \item \question{Démontrer à l'aide de la formule de Gutzmer que toute application holomorphe $f:\Cc\to\Cc$ bornée est constante..}
\reponse{Soit $f:\Cc\to\Cc$ une application holomorphe bornée en module par $M$.
On sait qu'elle est développable en séries entières sur $\Cc$ avec une série $\sum_{n\geq 0} a_n z^n$ de rayon de convergence infini.
On peut donc appliquer la formule de Gutzmer : pour tout $r\in\Rr^+$,
$$\sum_{n\geq 0} |a_n|^2r^{2n}=\frac{1}{2\pi}\int_0^{2\pi}|f(re^{i\theta})|^2d\theta\leq M^2 \frac{1}{2\pi}\int_0^{2\pi}d\theta=M^2$$
Donc, pour tout $n$ in $\Nn$ et tout $r\in\Rr^+$,
$$|a_n|^2r^{2n}\leq M^2.$$
En particulier, tous les $a_n$ avec $n\not =0$ sont nuls et $f$ est constante}
    \item \question{Soit $f:\Cc\to\Cc$ application holomorphe. On suppose que 
$$\forall r\in ]0,+\infty[, M(r):=\sup_{|z|<r} |f(z)|\leq r.$$
Montrer que $f$ est une application affine.}
\reponse{Soit $f:\Cc\to\Cc$ une application holomorphe telle que $ M(r)\leq r$.
$$\sum_{n\geq 0} |a_n|^2r^{2n}=\frac{1}{2\pi}\int_0^{2\pi}|f(re^{i\theta})|^2d\theta\leq M(r)^2 \frac{1}{2\pi}\int_0^{2\pi}d\theta=M(r)^2\leq r^2$$
Donc, pour tout $n$ in $\Nn$ et tout $r\in\Rr^+$, $|a_n|^2r^{2n}\leq r^2$.
En particulier, tous les $a_n$ avec $n\not =0,1$ sont nuls et $f$ est affine.}
\end{enumerate}
}

\uuid{dwop}
\exo7id{6745}
\titre{exo7 6745}
\auteur{queffelec}
\organisation{exo7}
\datecreate{2011-10-16}
\isIndication{false}
\isCorrection{false}
\chapitre{Théorème des résidus}
\sousChapitre{Théorème des résidus}
\module{Analyse complexe}
\niveau{L3}
\difficulte{}

\contenu{
\texte{

}
\begin{enumerate}
    \item \question{Calculer  $\displaystyle\int_0^{+\infty}{dx\over1+x^{3}}$, en intégrant
$\displaystyle{\log z\over1+z^3}$ sur un cercle privé de $\Rr^+$, $\hbox{log}$
dési\-gnant ici la détermination du logarithme avec coupure sur $\Rr^+$.}
    \item \question{En choisissant la même détermination du logarithme et le même contour,
calculer simutanément les intégrales $I=\displaystyle\int_{\Rr}
{dx\over1+x^4}$ et $J=\displaystyle\int_{\Rr}{\hbox{Ln} x\over1+x^4}\ dx$.
(Intégrer cette fois $\displaystyle{(\log z)^2\over1+z^4}$.)}
    \item \question{Calculer $\displaystyle\int_0^{+\infty}{dx\over1+x^{n}}$ et
$\displaystyle\int_0^{+\infty}{\hbox{Ln} x\over1+x^{n}}\ dx$, $n\geq2$, en
intégrant
$\displaystyle{\log z\over1+z^{n}}$ sur un secteur épointé bien choisi.}
\end{enumerate}
}

\uuid{xhWb}
\exo7id{7214}
\titre{exo7 7214}
\auteur{megy}
\organisation{exo7}
\datecreate{2021-02-22}
\isIndication{false}
\isCorrection{true}
\chapitre{Fonction holomorphe}
\sousChapitre{Fonction holomorphe}
\module{Analyse complexe}
\niveau{L3}
\difficulte{}

\contenu{
\texte{
On considère la fonction $f : \C\to \C, z\mapsto \sqrt{|xy|}$, où l'on a noté $z=x+iy$ la forme algébrique de $z$. Montrer qu'elle vérifie les conditions de Cauchy-Riemann en l'origine, mais qu'elle n'est pas $\C$-dérivable en l'origine.
}
\reponse{
On a \(\frac{\partial v}{\partial x}\equiv 0\) et \(\frac{\partial v}{\partial y}\equiv 0\) sur \(\R^2\).
Au point \((0,0)\in \R^2\) on a 
\[ \lim_{x\to 0}\frac{u(x,0)-u(0,0)}{x}\lim_{x\to 0}\frac{0-0}{x}=0 \]
et
\[ \lim_{y\to 0}\frac{u(0,y)-u(0,0)}{x}\lim_{x\to 0}\frac{0-0}{x}=0. \]
Donc \(u\) est dérivable par rapport à la première et à la seconde variable en \((0,0)\) et de plus \(\frac{\partial u}{\partial x}(0,0)=0\) et \(\frac{\partial u}{\partial y}(0,0)=0\).
En particulier \(f\) vérifie les équations de Cauchy-Riemann.
La fonction \(f\) n'est pas \(\C\)-dérivable en \(0\) car pour \(h\in \R_+^*\) on a 
\[ \frac{f(h)-f(0)}{h}=\frac{0-0}{h}=0 \]
et
\[ \frac{f(h+ih)-f(0)}{h+ih}=\frac{\sqrt{|h^2|}}{h+ih}=\frac{h}{h+ih}=\frac{1}{1+i}.\]
Les deux quantités n'ont pas la même limite lorsque $h\to 0$.
Ce n'est pas une contradiction avec le théorème de Cauchy-Riemann car la fonction \(u\) (et donc la fonction \(f\)) n'est pas différentiable en \((0,0)\). En effet si \(u\) était différentiable en \((0,0)\) on aurait 
\begin{align*}
u(s,t)
& = u(0,0)+du_{(0,0)}(s,t)+o(\|(s,t)\|)\\
& = \frac{\partial u}{\partial x}(0,0)+t\frac{\partial u}{\partial x}(0,0)+o(\|(s,t)\|) \\
& = o(\|(s,t)\|).
\end{align*}
Or \(u(s,s)=|s|\) qui n'est pas un \(o(\|(s,s)\|)\) d'où une contradiction.
}
}

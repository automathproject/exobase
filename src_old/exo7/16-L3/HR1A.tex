\uuid{HR1A}
\exo7id{6824}
\titre{exo7 6824}
\auteur{gijs}
\organisation{exo7}
\datecreate{2011-10-16}
\isIndication{false}
\isCorrection{false}
\chapitre{Fonction logarithme et fonction puissance}
\sousChapitre{Fonction logarithme et fonction puissance}
\module{Analyse complexe}
\niveau{L3}
\difficulte{}

\contenu{
\texte{
On définit les fonctions $f_1$, $f_2$ et $f_3$ par les
formules
\begin{align*}
f_1(z) &= \exp(\tfrac13[ \mathrm{Log}(z+1) + \mathrm{Log}(z) + \mathrm{Log}(z-1)
+ \mathrm{Log}(z - \sqrt3)])
\\
f_2(z) &= \exp(\tfrac13[ \mathrm{Log}(z+1) + \mathrm{Log}(-z) + \mathrm{Log}(1-z)
+ \mathrm{Log}(\sqrt3 - z) +i\pi])
\\
f_3(z) &= \exp(\tfrac13[\mathrm{Log}(-1-z) + \mathrm{Log}(-z) + \mathrm{Log}(1-z)
+ \mathrm{Log}(z-\sqrt3) + i\pi])
\ ,
\end{align*}
où $\mathrm{Log}$ désigne le logarithme principal.
}
\begin{enumerate}
    \item \question{Calculer $f_1(\pm i)$, $f_2(\pm i)$ et
$f_3(\pm i)$.}
    \item \question{Déterminer les domaines de définition de
$f_1$, $f_2$ et $f_3$.}
    \item \question{Démontrer que $f_1$, $f_2$ et $f_3$ sont des
déterminations continues de $\root3 \of {z^4
-\sqrt3 z^3 -z^2 + \sqrt3 z}$.}
    \item \question{Peut-on prolonger $f_1$ sur un ouvert plus
grand~? Si oui, lequel~?}
    \item \question{Peut-on prolonger $f_2$  sur un ouvert plus
grand~?  Si oui, lequel~?}
    \item \question{Peut-on prolonger $f_3$  sur un ouvert plus
grand~?  Si oui, lequel~?}
    \item \question{Y-a-t-il un lien entre $f_1$, $f_2$ et $f_3$~?}
\end{enumerate}
}

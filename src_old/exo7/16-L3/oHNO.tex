\uuid{oHNO}
\exo7id{2830}
\titre{exo7 2830}
\auteur{burnol}
\organisation{exo7}
\datecreate{2009-12-15}
\isIndication{false}
\isCorrection{true}
\chapitre{Théorème des résidus}
\sousChapitre{Théorème des résidus}
\module{Analyse complexe}
\niveau{L3}
\difficulte{}

\contenu{
\texte{
Soit $f$ une fonction entière telle que $|f(z)|\leq
M\,(1+ |z|)^n$ pour un certain $M$ et un certain
$n\in\Nn$. Donner plusieurs démonstrations que $f$ est un
polynôme de degré au plus $n$:
\begin{itemize}
\item en utilisant une formule intégrale de Cauchy pour
  $f^{(n+1)}(z)$, avec comme contour les cercles de rayon
  $R$ centrés en l'origine, ou en $z$ si l'on veut,
\item en utilisant les formules de Cauchy pour
  $f^{(m)}(0)$, avec $m\geq n+1$,
\item en appliquant le théorème de Liouville à $(f(z) -
  P(z))/z^{n+1}$ avec $P$ le polynôme de McLaurin-Taylor à
  l'origine à l'ordre $n$.
\end{itemize}
}
\reponse{
La formule de Cauchy pour $f^{(n+1)}(z)$ est
$$\frac{f^{(n+1)}(z)}{(n+1)!} =\frac{1}{2\pi i} \int_{C_R} \frac{f(\xi )}{(\xi -z)^{n+2}}\, d\xi $$
o\`u $C_R= \{|\xi|=R\}$. Pour les estimations suivantes, prenons $R> \min(2|z|,1)$. Comme $|\xi -z| \geq |\xi| -|z| =R-|z| \geq R/2$,
$$ \frac{|f(\xi )|}{|(\xi -z)^{n+2}|}\leq M \frac{(1+R)^n}{(R/2)^{n+2}} \leq M\frac{(2R)^n}{(R/2)^{n+2}}=2^{2n+2}M\frac{1}{R^2}.$$
Ensemble avec la formule de Cauchy on a donc
$$\frac{|f^{(n+1)}(z)|}{(n+1)!}\leq \frac{1}{2\pi} \int _{C_R} 2^{2n+2}M \frac{1}{R^2} |d\xi| =2^{2n+2} M\frac{1}{R}$$
pour n'importe quel $R>2|z|$. On vient de montrer que $f^{n+1}(z) =0$ pour tout $z\in \C$.
Utilisons maintenant $g(z) = \frac{f(z) -P(z)}{z^{n+1}}$ o\`u $P$ est le polyn\^ome de Taylor de $f$ \`a l'origine \`a l'ordre $n$. On remarque d'abord que l'origine est z\'ero d'ordre $n+1$ de $f(z) -P(z)$ ce qui explique que $g$ se prolonge holomorphiquement \`a l'origine. C'est donc une fonction enti\`ere pour laquelle on a
$$|g(z) | \leq \frac{C|z|^n}{|z|^{n+1}}=C\frac{1}{|z|}$$
pour un certain $C>0$ et pour $z$ de module suffisamment grand. De nouveau, $g$ est une fonction enti\`ere born\'ee, elle est donc constante (notre estimation donne m\^eme $g\equiv 0$ et donc $f=P$).
}
}

\uuid{m4wc}
\exo7id{6655}
\titre{exo7 6655}
\auteur{queffelec}
\organisation{exo7}
\datecreate{2011-10-16}
\isIndication{false}
\isCorrection{false}
\chapitre{Fonction logarithme et fonction puissance}
\sousChapitre{Fonction logarithme et fonction puissance}
\module{Analyse complexe}
\niveau{L3}
\difficulte{}

\contenu{
\texte{
On se propose de calculer les sommes de séries convergentes pour $0<t<2\pi$ 

$$\sum_1^\infty {\cos nt\over n},\quad  \sum_1^\infty {\sin nt\over n}.$$
}
\begin{enumerate}
    \item \question{Rappeler pourquoi $S(z)=-\sum_{n\geq 1}{z^n\over n}$ coincide
sur $D$ avec la détermi\-nation principale $\hbox{Log}(1-z)$.}
    \item \question{Soit $r<1$; calculer $\sum_{n\geq 1} {r^n\cos nt\over n}$ et
$\sum_{n\geq 1} {r^n\sin nt\over n}$.}
    \item \question{En déduire la valeur de ces sommes (on pourra utiliser le théorème d'Abel).}
\end{enumerate}
}

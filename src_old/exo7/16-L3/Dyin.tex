\uuid{Dyin}
\exo7id{7602}
\titre{exo7 7602}
\auteur{mourougane}
\organisation{exo7}
\datecreate{2021-08-10}
\isIndication{false}
\isCorrection{true}
\chapitre{Autre}
\sousChapitre{Autre}
\module{Analyse complexe}
\niveau{L3}
\difficulte{}

\contenu{
\texte{

}
\begin{enumerate}
    \item \question{Soit $f :\Cc\to\Cc$ une application holomorphe non constante. Montrer que l'image du plan complexe par $f$ 
rencontre tous les disques ouverts non vides $\Delta_r(a)$ de $\mathbb{C}$.}
\reponse{Soit $\Delta_r(a)$ le disque de centre $a\in\Cc$ et de rayon $r>0$. Supposons que $Im(f)$ ne rencontre pas $\Delta_r(a)$.
Alors,
$$\forall z\in\Cc,\ \ \ |f(z)-a|\geq r.$$
L'application $\Cc\to\Cc, z\mapsto\frac{1}{f(z)-a}$ est alors bien définie, 
holomorphe sur $\Cc$ et majorée en module par $\frac{1}{r}$.
Par le théorème de Liouville, elle est donc constante.
Puisque $w\mapsto \frac{1}{w-a}$ est injective sur $\Cc-\{a\}$, en déduit que $f$ est constante.}
    \item \question{En déduire que toute application holomorphe de $\Cc$ dans $\mathbb{H}$ est constante.}
\reponse{Une telle application entière ne rencontre par le disque $\Delta_1(-2i)$ : elle est donc constante,
d'après la question précédente.}
    \item \question{On considère $$\begin{array}{cccc}
 h :& \Cc-\{-i\}&\longrightarrow&\Cc \\&z&\longmapsto& \frac{z-i}{z+i}.
\end{array}$$
Montrer que 
$
 \forall z\in\Cc-\{-i\}, \ \ \ 1-|h(z)|^2=\frac{4 Im(z)}{|z+i|^2}.
$}
\reponse{Soit $z\in\Cc-\{-i\}$,
\begin{eqnarray*}
 1-|h(z)|^2&=& 1-\frac{z-i}{z+i}\frac{\overline{z}+i}{\overline{z+i}}
 = 1-\frac{|z|^2+i(z-\overline{z})+1}{|z+i|^2}\\&=&1-\frac{|z|^2-i(z-\overline{z})+1+2i(z-\overline{z})}{|z+i|^2}
= \frac{4 Im(z)}{|z+i|^2}.
\end{eqnarray*}}
    \item \question{En déduire que l'image du demi-plan de Poincaré $\mathbb{H}$ par $h$ est une partie bornée de $\Cc$.}
\reponse{Sur $\mathbb{H}$, $Im(z)>0$ et donc $|h(z)|<1$. L'image du demi-plan $\mathbb{H}$ par l'application $h$ est donc
incluse dans le disque unité $\Delta$ ; c'est donc une partie bornée de~$\Cc$.}
    \item \question{En déduire par une nouvelle démonstration que toute application holomorphe de $\Cc$ dans $\mathbb{H}$ est constante.}
\reponse{Soit $f :\Cc\to\mathbb{H}$ une application holomorphe.
Par composition, $h\circ f :\Cc\to \Cc$ est une application holomorphe dont l'image est bornée.
Par le théorème de Liouville, elle est donc constante. Comme $h$ est injective, on en déduit que $f$ est constante.}
\end{enumerate}
}

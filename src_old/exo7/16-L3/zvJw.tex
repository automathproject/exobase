\uuid{zvJw}
\exo7id{6831}
\titre{exo7 6831}
\auteur{gijs}
\organisation{exo7}
\datecreate{2011-10-16}
\isIndication{false}
\isCorrection{false}
\chapitre{Autre}
\sousChapitre{Autre}
\module{Analyse complexe}
\niveau{L3}
\difficulte{}

\contenu{
\texte{
Soit $a$ un réel, $0<a<2$.
}
\begin{enumerate}
    \item \question{Démontrer que l'intégrale
$ \int_0^\infty \dfrac{x^a}{x(1+x^2)}\,dx$
converge.}
    \item \question{Soit $f(z) = \dfrac{e^{a\log(z)}}{z(1+z^2)}$
avec $\log(z) = \mathrm{Log}(-iz) + i\pi/2$, c'est-à-dire que
$\log$ est le logarithme défini sur $\Omega
= \Cc\setminus i\,]-\infty,0]$ et tel que $\log(1) = 0$.
Déterminer les points singuliers isolés de $f$ dans
$\Omega$ et pour chaque point singulier isolé
déterminer son résidu.}
    \item \question{Soit $D = \{\,z\in \Cc \mid \epsilon< |z|
< R\ \&\ \Im z >0\,\}$. \`A l'aide de $\int_{\partial D}
f(z)\,dz$, déterminer la valeur de 
$ \int_0^\infty \dfrac{x^a}{x(1+x^2)}\,dx$.
N'oubliez pas de justifier les passages à la limite que
vous effectuez.}
\end{enumerate}
}

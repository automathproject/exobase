\uuid{ElRu}
\exo7id{7613}
\titre{exo7 7613}
\auteur{mourougane}
\organisation{exo7}
\datecreate{2021-08-10}
\isIndication{false}
\isCorrection{true}
\chapitre{Autre}
\sousChapitre{Autre}
\module{Analyse complexe}
\niveau{L3}
\difficulte{}

\contenu{
\texte{
\textit{On rappelle qu'à toute matrice $A=\begin{pmatrix} a&b\\c&d\end{pmatrix}$ de $SL(2,\Rr)$,
on associe l'application linéaire fractionnaire
$$\begin{array}{cccc}
 h_A :& \Cc-\{-\frac{d}{c}\}&\longrightarrow&\Cc-\{\frac{a}{c}\}\\&z&\longmapsto& \frac{az+b}{cz+d}
\end{array}$$}
}
\begin{enumerate}
    \item \question{Montrer que $h_A$ envoie $\mathbb{H}$ sur $\mathbb{H}$.}
\reponse{\begin{eqnarray*}
 Im\left(\frac{az+b}{cz+d}\right)&=&\frac{1}{2i}(\frac{az+b}{cz+d}-\overline{\frac{az+b}{cz+d}})\\
 &=&\frac{1}{2i}(\frac{az+b}{cz+d}-\frac{a\overline{z}+b}{c\overline{z}+d})=\frac{1}{2i}\frac{(ad-bc)(z-\overline{z})}{|cz+d|^2}\\
 &=&\frac{Im(z)}{|cz+d|^2}
\end{eqnarray*}
Par conséquent, si $Im(z)>0$, alors $Im\left(h_A(z)\right)>0$ et $h_A$ envoie $\mathbb{H}$ sur $\mathbb{H}$.}
    \item \question{Montrer que pour tout élément $z$ de $\mathbb{H}$, il existe $A\in SL(2,\Rr)$ 
tel que $h_A(i)=z$.}
\reponse{Soit $z=x+iy$. On a $\frac{xi-y}{1\times i+0}=z$. La matrice $\begin{pmatrix} x&-y\\1&0\end{pmatrix}$
convient ainsi donc que la matrice $\frac{1}{y}\begin{pmatrix} x&-y\\1&0\end{pmatrix}$ de déterminant $1$.}
\end{enumerate}
}

\uuid{suEo}
\exo7id{2836}
\titre{exo7 2836}
\auteur{burnol}
\organisation{exo7}
\datecreate{2009-12-15}
\isIndication{false}
\isCorrection{true}
\chapitre{Théorème des résidus}
\sousChapitre{Théorème des résidus}
\module{Analyse complexe}
\niveau{L3}
\difficulte{}

\contenu{
\texte{
Montrer que tout lacet est homotopiquement trivial dans $\Cc$.
}
\reponse{
Soit $\gamma:I\to \C$ un lacet quelconque. Posons
$$H(t,u)=u\gamma (t) \quad \text{pour} \quad t\in I \quad \text{et} \quad u\in [0,1].$$
 C'est clairement une homotopie de lacets
(voir la d\'efinition du cours!) telle que $H(t,1)=\gamma (t)$  et $H(t,0)=0$ pour tout $t\in I$.
}
}

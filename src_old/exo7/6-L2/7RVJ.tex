\uuid{7RVJ}
\exo7id{4988}
\titre{exo7 4988}
\auteur{quercia}
\organisation{exo7}
\datecreate{2010-03-17}
\isIndication{false}
\isCorrection{false}
\chapitre{Courbes planes}
\sousChapitre{Courbes paramétrées}
\module{Géométrie}
\niveau{L2}
\difficulte{}

\contenu{
\texte{
Construire les courbes d'équation polaire :
}
\begin{enumerate}
    \item \question{$\rho = \frac{\cos(\theta/2)}{1+\sin\theta}$.}
    \item \question{$\rho = \frac{\cos2\theta}{\cos\theta}$.
    (Strophoïde, calculer l'aire limitée par la boucle)}
    \item \question{$\rho = \frac{\sin\theta}{2\cos\theta-1}$.
    Vérifier que la courbe traverse ses asymptotes au point double.}
    \item \question{$\rho = \frac1{\cos\theta+\sin2\theta}$.}
    \item \question{$\rho = \cos\theta+ \frac1{\cos\theta}$.}
    \item \question{$\rho = \frac{\cos2\theta}{2\cos\theta-1}$.}
    \item \question{$\rho = \cos\frac\theta3$.}
    \item \question{$\rho = 1 + \sin3\theta$.}
    \item \question{$\rho = \frac1{\sqrt\theta}$.}
    \item \question{$\rho = \ln\theta$.}
\end{enumerate}
}

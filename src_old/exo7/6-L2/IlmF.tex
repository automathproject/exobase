\uuid{IlmF}
\exo7id{4948}
\titre{exo7 4948}
\auteur{quercia}
\organisation{exo7}
\datecreate{2010-03-17}
\isIndication{false}
\isCorrection{false}
\chapitre{Analyse vectorielle}
\sousChapitre{Torseurs}
\module{Géométrie}
\niveau{L2}
\difficulte{}

\contenu{
\texte{
Soient ${\cal T}_1$, ${\cal T}_2$ deux torseurs de sommes
$\vec{R}_1$, $\vec{R}_2$.
On définit le champ $\cal T$ par :
$${\cal T}(M) = \vec{R}_1 \wedge {\cal T}_2(M) + {\cal T}_1(M)\wedge \vec{R}_2.$$
}
\begin{enumerate}
    \item \question{Montrer que $\cal T$ est un torseur de somme $\vec R_1 \wedge \vec R_2$
    (produit vectoriel de ${\cal T}_1$ et ${\cal T}_2$).}
    \item \question{Si $\vec R_1 \wedge \vec R_2 \ne \vec 0$, montrer que l'axe central de
    $\cal T$ est la perpendiculaire commune des axes centraux de ${\cal T}_1$ et
    ${\cal T}_2$.}
\end{enumerate}
}

\uuid{x0DY}
\exo7id{5826}
\titre{exo7 5826}
\auteur{rouget}
\organisation{exo7}
\datecreate{2010-10-16}
\isIndication{false}
\isCorrection{true}
\chapitre{Conique}
\sousChapitre{Quadrique}
\module{Géométrie}
\niveau{L2}
\difficulte{}

\contenu{
\texte{
Déterminer la quadrique contenant le point $A(2,3,2)$ et les deux paraboles $(\mathcal{P})$ d'équations $\left\{
\begin{array}{l}
z=0\\
y^2=2x
\end{array}
\right.$ et $(\mathcal{P}')$ d'équations  $\left\{
\begin{array}{l}
x=0\\
y^2=2z
\end{array}
\right.$.
}
\reponse{
On cherche $(a,b,c,d,e,f,g,h,i,j)\neq (0,...,0)$ tel que la surface $(\mathcal{S})$ d'équation $ax^2+by^2+cz^2+2dxy+2eyz+2fzx+2gx+2hy+2iz+j = 0$ contienne la parabole $(\mathcal{P})$ de représentation paramétrique $\left\{
\begin{array}{l}
x=\frac{t^2}{2}\\
y=t\\
z=0
\end{array}
\right.$, $t\in\Rr$, la parabole $(\mathcal{P}')$ de représentation paramétrique $\left\{
\begin{array}{l}
x=0\\
y=t\\
z=\frac{t^2}{2}
\end{array}
\right.$, $t\in\Rr$, et le point $A(2,3,2)$.

\begin{align*}
(\mathcal{P})\subset(\mathcal{S})&\Leftrightarrow \forall t\in\Rr,\;\frac{a}{4}t^4+bt^2+dt^3+gt^2+2ht+j=0\Leftrightarrow \forall t\in\Rr,\;\frac{a}{4}t^4+dt^3+(b+g)t^2+2ht+j=0\\
 &\Leftrightarrow a=d=h=j=0\;\text{et}\;g = -b.
\end{align*}

Donc $(\mathcal{P})$ est contenue dans $(\mathcal{S})$ si et seulement si $(\mathcal{S})$ a une équation de la forme $by^2+cz^2+2eyz+2fzx-2bx+2iz = 0$ avec $(b,c,e,f,i)\neq(0,0,0,0,0)$. 

\begin{align*}
(\mathcal{P'})\subset(\mathcal{S})&\Leftrightarrow \forall t\in\Rr,\;bt^2+\frac{c}{4}t^4+et^3+it^2=0\Leftrightarrow \forall t\in\Rr,\;\frac{c}{4}t^4+et^3+(b+i)t^2=0\\
 &\Leftrightarrow c=e=0\;\text{et}\;i=-b.
\end{align*}

Donc $(\mathcal{P})$ et $(\mathcal{P}')$ sont contenues dans $(\mathcal{S})$ si et seulement si $(\mathcal{S})$ a une équation de la forme $by^2+2fzx-2bx-2bz = 0$ avec $(b,f)\neq(0,0)$. 

Enfin, $A\in(\mathcal{S})\Leftrightarrow 9b+8f-4b-4b = 0\Leftrightarrow b = -8f$ et $f\neq0$. On trouve donc une et une seule quadrique à savoir la surface $(\mathcal{S})$ d'équation $-4y2+zx+8x+8z=0$.

En posant $X=\frac{1}{\sqrt{2}}(x+z)$, $Y=y$ et $Z=\frac{1}{\sqrt{2}}(x-z)$, on obtient

\begin{align*}\ensuremath
-4y2+zx+8x+8z&=-4Y^2+\frac{1}{2}(X+Z)(X-Z)+8\sqrt{2}X\\
 &=\frac{1}{2}\left(X+8\sqrt{2}\right)^2-4Y^2+\frac{1}{2}Z^2 - 64.
\end{align*}

Dans le nouveau repère ainsi défini, une équation cartésienne de la surface $(\mathcal{S})$ est $\frac{1}{128}\left(X+8\sqrt{2}\right)^2-\frac{1}{16}Y^2+\frac{1}{128}Z^2=1$ et $(\mathcal{S})$ est  un hyperboloïde à deux nappes.
}
}

\uuid{ueSI}
\exo7id{7448}
\titre{exo7 7448}
\auteur{mourougane}
\organisation{exo7}
\datecreate{2021-08-10}
\isIndication{false}
\isCorrection{false}
\chapitre{Géométrie affine dans le plan et dans l'espace}
\sousChapitre{Géométrie affine dans le plan et dans l'espace}
\module{Géométrie}
\niveau{L2}
\difficulte{}

\contenu{
\texte{
Soit $\mathcal{E}$ un espace affine euclidien de dimension $3$. Pour
tout couple de droites $(D_1,D_2)$ on appelle $$d(D_1,D_2)=\inf\{ \|
y_1-y_2 \| , y_1 \in D_1, y_2\in D_2\}.$$
}
\begin{enumerate}
    \item \question{Calculer $d(D_1,D_2)$ quand $D_1$ et $D_2$ sont concourantes.}
    \item \question{Soit $a_1\in D_1$ et $a_2\in D_2$. En décomposant $a_1-a_2$ dans
 $\vec{D_1} +\vec{D_2}+(\vec{D_1} +\vec{D_2})^\perp$ montrer qu'il
 existe $x_1\in D_1$ et $x_2\in D_2$ tels que
 $d(D_1,D_2)=d(x_1,x_2)$.}
    \item \question{Montrer que pour $z_1\in D_1$ et $ z_2\in D_2$, 
$$d(D_1,D_2)=d(z_1,z_2) _iff z_1-z_2\in \vec{D_1}^\perp \cap
\vec{D_2}^\perp.$$}
    \item \question{Montrer que si $e_i$ est un vecteur directeur de $D_i$
 $$d(D_1,D_2)^2=\frac{Gram(a_1-a_2,e_1,e_2)}{Gram(e_1,e_2)}$$ 
o{ù} $Gram(u_1,u_2,\cdots ,u_r):=det(\langle u_i,u_j\rangle
)_{1\leq i,j\leq r}$.}
    \item \question{Calculer la distance entre les deux droites données par les équations
cartésiennes dans un repère orthonormé de $\mathcal{E}$~:
\begin{eqnarray*}
M\left( \begin{array}{c}x\\y\\z\end{array}\right)\in D_1
\iff\left\{\begin{array}{c}x+y=1\\x+y+2z=1\end{array}\right.\ \ \ 
M\left( \begin{array}{c}x\\y\\z\end{array}\right)\in D_2
\iff\left\{\begin{array}{c}y+z=1\\x+y-2z=3\end{array}\right.
\end{eqnarray*}}
\end{enumerate}
}

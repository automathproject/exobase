\uuid{GVNz}
\exo7id{5033}
\titre{exo7 5033}
\auteur{quercia}
\organisation{exo7}
\datecreate{2010-03-17}
\isIndication{false}
\isCorrection{true}
\chapitre{Courbes planes}
\sousChapitre{Propriétés métriques : longueur, courbure,...}
\module{Géométrie}
\niveau{L2}
\difficulte{}

\contenu{
\texte{
On considère la courbe $\mathcal{C}$ d'équation polaire $\rho = 1 + \cos\theta$
(cardioïde).
}
\begin{enumerate}
    \item \question{Dessiner $\mathcal{C}$.}
    \item \question{Une droite $D$ passant par $O$ coupe $\mathcal{C}$ en deux points $M_1$ et $M_2$.
    Soient $\Delta_1$, $\Delta_2$ les normales à $\mathcal{C}$ en ces points et $P$
    le point d'intersection de $\Delta_1$ et $\Delta_2$. Quelle est la courbe
    décrite par $P$ lorsque $D$ tourne autour de $O$~?}
\reponse{
Cercle de centre $(\frac12,0)$ et de rayon $\frac12$.
}
\end{enumerate}
}

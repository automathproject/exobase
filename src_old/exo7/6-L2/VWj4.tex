\uuid{VWj4}
\exo7id{4975}
\titre{exo7 4975}
\auteur{quercia}
\organisation{exo7}
\datecreate{2010-03-17}
\isIndication{false}
\isCorrection{true}
\chapitre{Géométrie affine euclidienne}
\sousChapitre{Géométrie affine euclidienne de l'espace}
\module{Géométrie}
\niveau{L2}
\difficulte{}

\contenu{
\texte{
Soit $S$ une partie de l'espace contenant au moins deux points et telle
que pour tout plan $P$, $P\cap S$ est un cercle, un singleton ou vide.
Montrer que $S$ est une sphère.
}
\reponse{
Soient $A,B$ deux points de $S$ distincts. Intersection de $S$ avec
un plan passant par $A$ et $B$ $ \Rightarrow $ $S$ est réunion de cercles passant par
$A$ et $B$.
On considère le plan médiateur de $[A,B]$, $P$ qui coupe $S$ suivant un cercle
$\mathcal{C}$ de centre $O$.
Le plan $Q = (OAB)$ coupe $S$ suivant un cercle $\mathcal{C}'$. $\mathcal{C}$ et $\mathcal{C}'$ ont en
commun les points $C$, $D$.
$(CD)$ est médiatrice de $[A,B]$ dans $Q$ donc est un diamètre de $\mathcal{C}'$ et
passe par $O$, donc est aussi diamètre de $\mathcal{C}$.
Ainsi $\mathcal{C}$ et $\mathcal{C}'$ sont deux cercles de même centre et même rayon dans des
plans perpendiculaires.
En considérant les plans coupant $P$ et $Q$ à angle droit, on obtient que $S$
est une sphère de centre $O$.
}
}

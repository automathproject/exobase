\uuid{cnc2}
\exo7id{7507}
\titre{exo7 7507}
\auteur{mourougane}
\organisation{exo7}
\datecreate{2021-08-10}
\isIndication{false}
\isCorrection{true}
\chapitre{Géométrie affine euclidienne}
\sousChapitre{Géométrie affine euclidienne du plan}
\module{Géométrie}
\niveau{L2}
\difficulte{}

\contenu{
\texte{

}
\begin{enumerate}
    \item \question{Démontrer que si $A, B, C$ sont trois points distincts d'un plan affine euclidien $\mathcal{P}$
    la somme des angles de vecteurs 
    $$((\vec{AB}, \vec{AC}))+((\vec{BC},\vec{BA}))+((\vec{CA},\vec{CB}))$$
    est un angle plat.}
\reponse{Soit $A, B, C$ sont trois points distincts d'un plan affine euclidien $\mathcal{P}$.
    La somme des angles de vecteurs se réécrit par symétrie centrale par rapport à $C$ comme
    \begin{eqnarray*}
        \lefteqn{((\vec{AB}, \vec{AC}))+((\vec{BC},\vec{BA}))+((\vec{CA},\vec{CB}))}
        &&\\
        &=&((\vec{AB}, \vec{AC}))+((\vec{BC},\vec{BA}))+((-\vec{CA},-\vec{CB}))\\
        &=&((\vec{AB}, \vec{AC}))+((\vec{BC},\vec{BA}))+((\vec{AC},\vec{BC}))\\
        &=&((\vec{AB}, \vec{AC}))+((\vec{AC},\vec{BC}))+((\vec{BC},\vec{BA}))\\
        &=&((\vec{AB},\vec{BA}))\\
    \end{eqnarray*}
    c'est à dire l'angle plat.}
    \item \question{Soit $\mathcal{C}$ un cercle de $\mathcal{P}$ et $A$ un point de $\mathcal{C}$.
    Soit $\mathcal{C}'$ l'image par une rotation $r$ de centre $A$ du cercle $\mathcal{C}$.
    Soit $B$ l'autre point d'intersection de $\mathcal{C}$ et $\mathcal{C}'$.
    Soit $D$ le point de $\mathcal{C}$ diamétralement opposé à $A$ sur $\mathcal{C}$.
    Soit $D'=r(D)$ son image par $r$. Montrer que $D$, $D'$ et $B$ sont alignés.}
\reponse{Soit $\mathcal{C}$ un cercle de $\mathcal{P}$ et $A$ un point de $\mathcal{C}$.
    Soit $\mathcal{C}'$ l'image par une rotation $r$ de centre $A$ du cercle $\mathcal{C}$.
    Soit $B$ l'autre point d'intersection de $\mathcal{C}$ et $\mathcal{C}'$.
    Soit $D$ le point de $\mathcal{C}$ diamétralement opposé à $A$ sur $\mathcal{C}$.
    Soit $D'=r(D)$ son image par $r$. 
    Soit $O$ le centre de $\mathcal{C}$ et $O'$ le centre de $\mathcal{C}'$.
    
    Comme $D$ appartient à $(AO)$, le point $D'=r(D)$ appartient à $r((AO))=(A'O')$.
    Donc $[A'D']$ est un diamètre de $\mathcal{C}'$.
    Comme $B$ appartient à $\mathcal{C}$ et comme $[AD]$ est un diamètre de $\mathcal{C}$, la droite $(BD)$ est orthogonale à $(AB)$.
    De même, comme $B$ appartient à $\mathcal{C}'$ et comme $[AD']$ est un diamètre de $\mathcal{C}$, la droite $(BD')$ est orthogonale à $(AB)$.
    Par conséquent, les droites $(BD)$ et $(BD')$ sont confondues et $D$, $D'$ et $B$ sont alignés.}
    \item \question{Soit $M$ un point quelconque de $\mathcal{C}$.
    Montrer que $M$, $M'=r(M)$ et $B$ sont alignés.}
\reponse{Soit $M$ un point quelconque de $\mathcal{C}$.
    Dans les triangles $BAM'$ et $BAM$
    $$((\vec{BM'}, \vec{BA}))+((\vec{AB},\vec{AM'}))+((\vec{M'A},\vec{M'B}))$$
    et $$(( \vec{BA},\vec{BM}))+((\vec{AM},\vec{AB}))+((\vec{MB},\vec{MA}))$$
    sont deux angles plats.
    Par somme,$$((\vec{BM'},\vec{BM}))+((\vec{AM},\vec{AM'}))
    +((\vec{M'A},\vec{M'B}))+((\vec{MB},\vec{MA}))$$ est un angle nul.
    
    Par propriété des angles inscrits, $((\vec{M'A},\vec{M'B}))=((\vec{D'A},\vec{D'B}))$ et $((\vec{MB},\vec{MA}))=((\vec{DB},\vec{DA}))$.
    Dans le triangle $ADD'$, on trouve que
    $$((\vec{D'A},\vec{D'B}))+((\vec{DB},\vec{DA}))$$
    et la différence d'un angle plat et de $((\vec{AD},\vec{AD'}))=((\vec{AM},\vec{AM'}))$ l'angle de la rotation.
    En conséquence,
    $((\vec{BM'},\vec{BM}))$ est un angle plat et $M$, $M'=r(M)$ et $B$ sont alignés.}
\end{enumerate}
}

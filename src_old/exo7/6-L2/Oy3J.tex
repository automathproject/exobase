\uuid{Oy3J}
\exo7id{7467}
\titre{exo7 7467}
\auteur{mourougane}
\organisation{exo7}
\datecreate{2021-08-10}
\isIndication{false}
\isCorrection{false}
\chapitre{Géométrie affine dans le plan et dans l'espace}
\sousChapitre{Géométrie affine dans le plan et dans l'espace}
\module{Géométrie}
\niveau{L2}
\difficulte{}

\contenu{
\texte{
On considére le plan muni d'un un repère orthonormé ($O, \overrightarrow {\imath},\overrightarrow{\jmath}$) et la courbe $(C)$ d'équation 

\begin{center}$x^{2} - 3xy +2y^{2} + 2x - 3y + 1=0 $ \end{center}
}
\begin{enumerate}
    \item \question{Montrer que cette courbe possède un centre de symétrie $\Omega$ et donner son équation dans le repère ($\Omega, \overrightarrow {\imath},\overrightarrow{\jmath}$)}
    \item \question{En déduire que $(C)$ est la réunion de deux droites dont on donnera les équations dans le rèpere ($O, \overrightarrow {\imath},\overrightarrow{\jmath}$)}
\end{enumerate}
}

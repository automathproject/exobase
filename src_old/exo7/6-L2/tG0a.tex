\uuid{tG0a}
\exo7id{4893}
\titre{exo7 4893}
\auteur{quercia}
\organisation{exo7}
\datecreate{2010-03-17}
\isIndication{false}
\isCorrection{true}
\chapitre{Géométrie affine dans le plan et dans l'espace}
\sousChapitre{Propriétés des triangles}
\module{Géométrie}
\niveau{L2}
\difficulte{}

\contenu{
\texte{
Soit un triangle $ABC$, $A',B',C'$, les milieux des côtés, et $M$ un point
du plan $(ABC)$ de coordonnées barycentriques $(\alpha,\beta,\gamma)$.
}
\begin{enumerate}
    \item \question{Chercher les coordonnées barycentriques de $P,Q,R$ symétriques de $M$
     par rapport aux points $A',B',C'$.}
    \item \question{Montrer que les droites $(AP)$, $(BQ)$, $(CR)$ sont concourantes en un
     point $N$.}
    \item \question{Montrer que $N$ est le milieu de $[A,P]$, $[B,Q]$, $[C,R]$.}
    \item \question{Reconnaître l'application $M \mapsto N$.}
\reponse{
$N = \text{Bar}(A:1-\alpha, B:1-\beta, C:1-\gamma)$.
homothétie de centre $G$, de rapport $-\frac 12$.
}
\end{enumerate}
}

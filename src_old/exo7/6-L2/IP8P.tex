\uuid{IP8P}
\exo7id{5207}
\titre{exo7 5207}
\auteur{rouget}
\organisation{exo7}
\datecreate{2010-06-30}
\isIndication{false}
\isCorrection{true}
\chapitre{Isométrie vectorielle}
\sousChapitre{Isométrie vectorielle}
\module{Géométrie}
\niveau{L2}
\difficulte{}

\contenu{
\texte{
Nature et éléments caractéristiques de la transformation d'expression complexe~:
}
\begin{enumerate}
    \item \question{$z'=z+3-i$}
\reponse{$f$ est la translation de vecteur $\vec{u}(3,-1)$.}
    \item \question{$z'=2z+3$}
\reponse{$\omega=2\omega+3\Leftrightarrow\omega=-3$. $f$ est l'homothétie de rapport $2$ et de centre $\Omega(-3,0)$.}
    \item \question{$z'=iz+1$}
\reponse{$\omega=i\omega+1\Leftrightarrow\omega=\frac{1}{2}(1+i)$. Comme $i=e^{i\pi/2}$, $f$ est la rotation d'angle
$\frac{\pi}{2}$ et de centre $\Omega(\frac{1}{2},\frac{1}{2})$.}
    \item \question{$z'=(1-i)z+2+i$}
\reponse{$\omega=(1-i)\omega+2+i\Leftrightarrow\omega=1-2i$. Comme $1-i=\sqrt{2}e^{-i\pi/4}$, $f$ est la similitude de centre
$\Omega(1,-2)$, de rapport $\sqrt{2}$ et d'angle $-\frac{\pi}{4}$.}
\end{enumerate}
}

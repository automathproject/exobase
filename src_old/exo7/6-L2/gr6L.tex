\uuid{gr6L}
\exo7id{5037}
\titre{exo7 5037}
\auteur{quercia}
\organisation{exo7}
\datecreate{2010-03-17}
\isIndication{false}
\isCorrection{false}
\chapitre{Courbes planes}
\sousChapitre{Propriétés métriques : longueur, courbure,...}
\module{Géométrie}
\niveau{L2}
\difficulte{}

\contenu{
\texte{
Soit $\mathcal{C} : t \mapsto M_t$ une courbe plane paramétrée sans point stationnaire.
Les courbes parallèles à $\mathcal{C}$ sont les courbes de la forme :
$t \mapsto M_t + \lambda \vec N$,
ou $\vec N$ est le vecteur normal en $M_t$ et $\lambda$ est constant.
}
\begin{enumerate}
    \item \question{Montrer que le parallélisme est une relation d'équivalence entre arcs
    sans points stationnaires.}
    \item \question{Construire les parallèles à la parabole d'équation $y = x^2$ pour
    $\lambda = \pm 2$.}
\end{enumerate}
}

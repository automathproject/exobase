\uuid{YFfJ}
\exo7id{4996}
\titre{exo7 4996}
\auteur{quercia}
\organisation{exo7}
\datecreate{2010-03-17}
\isIndication{false}
\isCorrection{true}
\chapitre{Courbes planes}
\sousChapitre{Courbes définies par une condition}
\module{Géométrie}
\niveau{L2}
\difficulte{}

\contenu{
\texte{
Soit $D$ une droite du plan et $\mathcal{C}$ une courbe paramétrée.
Pour $M \in \mathcal{C}$ on note $T$ et $N$ les points d'intersection de $D$ avec la
tangente et la normale à $\mathcal{C}$ en $M$.
Déterminer $\mathcal{C}$ telle que le milieu de $[T,N]$ reste fixe.

$\Bigl($On paramètrera $\mathcal{C}$ par $t = \frac{y'}{x'}\Bigr)$
}
\reponse{
$D = Ox  \Rightarrow  x_T = x - \frac{x'y}{y'}$, $x_N = x + \frac{yy'}{x'}
 \Rightarrow  2x + y\left(t-\frac1t\right) = a$ (cste).

On dérive : $2x' + y'\left(t-\frac1t\right) + y\left(1+\frac1{t^2}\right) = 0
 \Rightarrow  y'\left(t+\frac1t\right) + y\left(1+\frac1{t^2}\right) = 0$.
 
$ \Rightarrow  y = \frac\lambda t$, $x = b + \frac\lambda{2t^2}$ (Parabole)
}
}

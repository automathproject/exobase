\uuid{3JIX}
\exo7id{5051}
\titre{exo7 5051}
\auteur{quercia}
\organisation{exo7}
\datecreate{2010-03-17}
\isIndication{false}
\isCorrection{true}
\chapitre{Surfaces}
\sousChapitre{Surfaces paramétrées}
\module{Géométrie}
\niveau{L2}
\difficulte{}

\contenu{
\texte{
\label{revolution}
Soit ${\cal S}$ une surface d'équation $z = f(x,y)$.
Montrer que ${\cal S}$ est de révolution si et seulement si en tout point $M$,
la normale à ${\cal S}$ en $M$ est parallèle ou sécante à $Oz$.
}
\reponse{
La normale en $M$ est parallèle ou sécante à $Oz
         \Leftrightarrow y\frac{\partial f}{\partial x} - x\frac{\partial f}{\partial y} = 0
	 \Leftrightarrow \frac{\partial f}{\partial \theta} = 0 \Leftrightarrow f = f(\rho)$.
}
}

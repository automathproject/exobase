\uuid{drCj}
\exo7id{2040}
\titre{exo7 2040}
\auteur{liousse}
\organisation{exo7}
\datecreate{2003-10-01}
\isIndication{false}
\isCorrection{false}
\chapitre{Géométrie affine euclidienne}
\sousChapitre{Géométrie affine euclidienne du plan}
\module{Géométrie}
\niveau{L2}
\difficulte{}

\contenu{
\texte{
Dans le plan muni d'un rep\`ere orthonorm\'e direct $(O,\vec{OI},\vec{OJ})$.
}
\begin{enumerate}
    \item \question{Soit $f$ la transformation 
du plan d\'efinie analytiquement par
$$\left\{\begin{array}{ll}x'=&{1\over \sqrt{5}}(x+2y-1)\\y'=&{1\over \sqrt{5}}
(-2x+y+2)\end{array}\right.$$
\begin{enumerate}}
    \item \question{Calculer les coordonn\'ees de $O'$, $I'$, $J'$ les images par $f$ des points $O$, $I$, $J$.}
    \item \question{Montrer que le rep\`ere  $(O',\vec{O'I'},\vec{O'J'})$ est orthonorm\'e, est-il direct ?}
    \item \question{En d\'eduire  que $f$ est une isom\'etrie, est-elle directe ?}
    \item \question{D\'eterminer l'ensemble  des points invariants par $f$ et reconnaitre $f$.}
    \item \question{Donner l'expression analytique de la transformation inverse de $f$.}
    \item \question{Calculer l'image par $f$ la droite d'\'equation $2x-y-1=0$.}
\end{enumerate}
}

\uuid{Txas}
\exo7id{4904}
\titre{exo7 4904}
\auteur{quercia}
\organisation{exo7}
\datecreate{2010-03-17}
\isIndication{false}
\isCorrection{true}
\chapitre{Conique}
\sousChapitre{Parabole}
\module{Géométrie}
\niveau{L2}
\difficulte{}

\contenu{
\texte{
Soit ${\cal C}$ un cercle de centre $O$, et $A,B$ deux points distincts
de ${\cal C}$.
Soit $\Delta$ le diamètre parallèle à $(AB)$.

Pour $M \in {\cal C}$, on note $P,Q$ les intersections de $(MA)$ et $(MB)$ avec
$\Delta$.
Chercher le lieu du centre du cercle circonscrit à $MPQ$.
}
\reponse{
Soit $O'$ ce centre.
         Les triangles $MPQ$ et $MAB$ sont semblables, donc $O$' est l'image de
         $O$ par l'homothétie de centre $M$ qui transforme $A$ en $P$.
         \par
         Soit $(A'B')$ la symétrique de $(AB)$ par rapport à $O$.
         D'après l'homothétie,
         $$\frac {O'M}{d(O',\Delta)} = \frac {OM}{d(O,(AB))} = (cste)
           = \frac {OM-O'M}{d(O,(AB))-d(O',\Delta)}
           = \frac {OO'}{d(O',(A'B'))}.$$
         Donc $O'$ décrit une partie d'une conique de foyer $O$ et de directrice
         $(A'B')$.
}
}

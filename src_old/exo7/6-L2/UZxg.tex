\uuid{UZxg}
\exo7id{7135}
\titre{exo7 7135}
\auteur{megy}
\organisation{exo7}
\datecreate{2017-02-08}
\isIndication{false}
\isCorrection{false}
\chapitre{Géométrie affine euclidienne}
\sousChapitre{Géométrie affine euclidienne du plan}
\module{Géométrie}
\niveau{L2}
\difficulte{}

\contenu{
\texte{
Soit $\mathcal D$ une droite et $A$ et $B$ deux points situés d'un seul côté de $\mathcal D$. L'objectif est de construire un cercle passant par les deux points et tangent à la droite.
}
\begin{enumerate}
    \item \question{Construire un tel cercle si les droites $(AB)$ et $\mathcal D$ sont parallèles. Dans la suite, on supposera qu'elles sont sécantes.}
    \item \question{(Analyse) Soit $\mathcal C$ un tel cercle et $T$ son point de tangence avec $\mathcal D$. Montrer que $(AB,AT) = (TB,\mathcal D)$. % angle inscrit avec le cas limite.}
    \item \question{(Synthèse) Soit $I$ le point d'intersection de $(AB)$ avec $\mathcal D$, $B'$ le symétrique de $B$ par rapport à $I$, et $B''$ le symétrique de $B$ par rapport à $\mathcal D$. Montrer que le cercle circonscrit à $AB'B''$ (de diamètre $[AB']$ dans le cas où $B'=B''$) coupe $\mathcal D$ en deux points qui conviennent pour le choix de $T$.}
\end{enumerate}
}

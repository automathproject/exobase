\uuid{1DZN}
\exo7id{4947}
\titre{exo7 4947}
\auteur{quercia}
\organisation{exo7}
\datecreate{2010-03-17}
\isIndication{false}
\isCorrection{true}
\chapitre{Analyse vectorielle}
\sousChapitre{Torseurs}
\module{Géométrie}
\niveau{L2}
\difficulte{}

\contenu{
\texte{
Soit $ABCD$ un tétraèdre non aplati de l'espace.
Pour $X,Y \in \{A,B,C,D\}$ distincts, on note
${\cal G}_{XY}$ le glisseur d'axe la droite $(XY)$ et de vecteur $\vec{XY}$.

Montrer que $( {\cal G}_{AB},
               {\cal G}_{AC},
               {\cal G}_{AD},
               {\cal G}_{BC},
               {\cal G}_{BD},
               {\cal G}_{CD} )$ est une base de l'espace des torseurs.
}
\reponse{
Soit $\cal T$ un torseur : on décompose ${\cal T}(A)$ en
         $\alpha\vec{AB}\wedge\vec{BC} +
          \beta \vec{AB}\wedge\vec{BD} +
          \gamma\vec{AC}\wedge\vec{CD}$, et $\vec R$ en
         $\alpha'\vec{AB} + \beta'\vec{AC} + \gamma'\vec{AD}$.
         \par   $ \Rightarrow $ famille génératrice.
}
}

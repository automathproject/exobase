\uuid{RddA}
\exo7id{5061}
\titre{exo7 5061}
\auteur{quercia}
\organisation{exo7}
\datecreate{2010-03-17}
\isIndication{false}
\isCorrection{true}
\chapitre{Surfaces}
\sousChapitre{Surfaces paramétrées}
\module{Géométrie}
\niveau{L2}
\difficulte{}

\contenu{
\texte{
On considère la droite $\Delta$ d'équations : $x=a$, $z=0$.
    $P$ est un point décrivant $\Delta$ et $\mathcal{C}_P$ le cercle tangent à $Oz$ en
    $O$ et passant par $P$.
    Faire un schéma et paramétrer la surface engendrée par les cercles $\mathcal{C}_P$
    quand $P$ décrit $\Delta$.
}
\reponse{
$x=\frac a2(1+\cos u)$,
	     $y=\frac v2(1+\cos u)$,
	     $z=\frac{\sqrt{a^2+v^2}}2\sin u$.
}
}

\uuid{OD0l}
\exo7id{7478}
\titre{exo7 7478}
\auteur{mourougane}
\organisation{exo7}
\datecreate{2021-08-10}
\isIndication{false}
\isCorrection{false}
\chapitre{Géométrie affine euclidienne}
\sousChapitre{Géométrie affine euclidienne du plan}
\module{Géométrie}
\niveau{L2}
\difficulte{}

\contenu{
\texte{
Dans un plan affine euclidien orienté on considère deux points 
distincts $O$ et $A$. On note $r$ la rotation de centre $O$ et d'angle $2\pi/3$
et $\rho$ la rotation de centre $A$ et d'angle $2\pi/3$. On pose $B=r(A)$ et
$C=r(B)$. Enfin on note $G$ le groupe d'isométries engendré par $r$
et $\rho$.
}
\begin{enumerate}
    \item \question{Montrer que $G$ ne contient que des translations et des
 rotations d'angle $2\pi/3$ et $-2\pi/3$.}
    \item \question{Expliciter une relation de dépendance entre les vecteurs $\vec{OA},
 \vec{OB}$  et $\vec{OC}$
 (on pourra remarquer que la somme de ces vecteurs est
 invariante par $r$).}
    \item \question{Montrer que $r\circ\rho ^{-1}$  et $r^{-1} \circ \rho$
 sont des translations dont on
 précisera le vecteur (on pourra étudier l'image de $A$).}
    \item \question{Montrer que $G$ contient toutes les translations de vecteur
 $p \vec{OA} + q \vec{OB}$ avec $p\in \Zz$, $q\in\Zz$ et $p+q\in 3\Zz$.}
\end{enumerate}
}

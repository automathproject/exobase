\uuid{lXmb}
\exo7id{5030}
\titre{exo7 5030}
\auteur{quercia}
\organisation{exo7}
\datecreate{2010-03-17}
\isIndication{false}
\isCorrection{true}
\chapitre{Courbes planes}
\sousChapitre{Propriétés métriques : longueur, courbure,...}
\module{Géométrie}
\niveau{L2}
\difficulte{}

\contenu{
\texte{
Soit la courbe $\Gamma$ définie par : $xy = a^2$, $(a>0)$.
    Pour chaque point $M$ on définit le point $\Omega$ par :
    $2\vec{\Omega M} = \vec{MN}$,
    où $N$ est le point où $\Gamma$ recoupe sa normale en $M$.
    Montrer que $\Omega$ est le centre de courbure de $\Gamma$ en $M$.
}
\reponse{
Calcul.
}
}

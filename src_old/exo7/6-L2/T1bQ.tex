\uuid{T1bQ}
\exo7id{7288}
\titre{exo7 7288}
\auteur{mourougane}
\organisation{exo7}
\datecreate{2021-08-10}
\isIndication{false}
\isCorrection{false}
\chapitre{Géométrie affine euclidienne}
\sousChapitre{Géométrie affine euclidienne du plan}
\module{Géométrie}
\niveau{L2}
\difficulte{}

\contenu{
\texte{
La notion de «construction au compas» est un peu ambiguë. On distingue:
\begin{itemize}
\item le \emph{compas traçant}, qui permet, à partir de deux points 
\(A\) et \(B\), de construire le cercle de centre \(A\) passant 
par \(B\);
\item le \emph{compas transporteur}, qui permet, à partir de trois 
points \(A\), \(B\) et \(C\), de construire le cercle de centre \(A\) 
et de rayon \(BC\) (il permet de «transporter» la distance \(BC\), 
d'où son nom).
\end{itemize}
Le compas transporteur permet évidemment toutes les constructions 
possibles au compas traçant. Le but de cet exercice est de montrer 
que toute construction au compas transporteur peut être transformée 
en une construction au compas traçant.

On considère trois points \(A\), \(B\) et \(C\), deux à deux 
distincts.
}
\begin{enumerate}
    \item \question{Construire, au compas traçant, deux points de la médiatrice 
du segment \([AB]\).}
    \item \question{Construire, au compas traçant, le symétrique \(C'\) de \(C\) 
par rapport à la médiatrice du segment \([AB]\).}
    \item \question{Montrer que \(AC' = BC\).}
    \item \question{Construire, au compas traçant, le cercle de centre \(A\) et 
de rayon \(BC\).}
\end{enumerate}
}

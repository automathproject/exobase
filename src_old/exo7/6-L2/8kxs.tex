\uuid{8kxs}
\exo7id{5514}
\titre{exo7 5514}
\auteur{rouget}
\organisation{exo7}
\datecreate{2010-07-15}
\isIndication{false}
\isCorrection{true}
\chapitre{Géométrie affine dans le plan et dans l'espace}
\sousChapitre{Géométrie affine dans le plan et dans l'espace}
\module{Géométrie}
\niveau{L2}
\difficulte{}

\contenu{
\texte{
Soit $M(x,y,z)$ un point de $\Rr^3$ rapporté à un repère orthonormé. Déterminer la distance de $M$ à la droite 
$(D)$ $\left\{
\begin{array}{l}
x+y+z+1=0\\
2x+y+5z=2
\end{array}
\right.$. En déduire une équation du cylindre de révolution d'axe $(D)$ et de rayon $2$.
}
\reponse{
\textbullet~Déterminons un repère de $(D)$.
\begin{center}
$\left\{
\begin{array}{l}
x+y+z+1=0\\
2x+y+5z=2
\end{array}
\right.\Leftrightarrow\left\{
\begin{array}{l}
x+y=-1-z\\
2x+y=2-5z
\end{array}
\right.\Leftrightarrow\Leftrightarrow\left\{
\begin{array}{l}
x=3-4z\\
y=-4+3z
\end{array}
\right.$. 
\end{center}
Un repère de $(D)$ est $\left(A,\overrightarrow{u}\right)$ où $A(3,-4,0)$ et $\overrightarrow{u}(-4,3,1)$.
\textbullet~Soit $M(x,y,z)$ un point du plan. On sait que

\begin{center}
$d(A,(D))=\frac{\|\overrightarrow{AM}\wedge\overrightarrow{u}\|}{\|\overrightarrow{u}\|}=\frac{\sqrt{(y-3z+4)^2+(x+4z-3)^2+(3x+4y+7)^2}}{\sqrt{26}}$
\end{center}
\textbullet~Notons $\mathcal{C}$ le cylindre de révolution d'axe $(D)$ et de rayon $2$.
\begin{center}
$M(x,y,z)\in\mathcal{C}\Leftrightarrow d(A,(D))=2\Leftrightarrow(y-3z+4)^2+(x+4z-3)^2+(3x+4y+7)^2=104$
\end{center}

\begin{center}
\shadowbox{
\begin{tabular}{c}
Une équation cartésienne  du cylindre de révolution d'axe $(D)$ et de rayon $2$ est\\
$(y-3z+4)^2+(x+4z-3)^2+(3x+4y+7)^2=104$.
\end{tabular}
}
\end{center}
}
}

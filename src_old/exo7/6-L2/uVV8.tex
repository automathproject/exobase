\uuid{uVV8}
\exo7id{2695}
\titre{exo7 2695}
\auteur{matexo1}
\organisation{exo7}
\datecreate{2002-02-01}
\isIndication{false}
\isCorrection{false}
\chapitre{Courbes planes}
\sousChapitre{Autre}
\module{Géométrie}
\niveau{L2}
\difficulte{}

\contenu{
\texte{
Soit $\cal P$ la parabole d'{\'e}quation $y^2=2px, p>0$.
\begin{itemize}
\item Montrer que la tangente {\`a} $\cal P$ au point ${\rm M}_{0}= (x_{0},y_{0})$
a pour {\'e}quation $yy_{0}=p(x+x_{0})$. 
\item Un rayon lumineux, port{\'e} par la
droite d'{\'e}quation $y=y_{0}$ et se propageant en sens inverse de l'axe des $x$,
se r{\'e}fl{\'e}chit au point ${\rm M}_{0}$ sur la tangente {\`a} $\cal P$ selon la loi
de Descartes. D{\'e}terminer l'{\'e}quation du rayon r{\'e}fl{\'e}chi. 
\item V{\'e}rifier
que les rayons r{\'e}fl{\'e}chis correspondant aux diverses valeurs de $y_{0}$
passent tous par un m{\^e}me point F situ{\'e} sur l'axe des $x$\,(foyer de la
parabole).

Citer des applications pratiques de cette propri{\'e}t{\'e}. 
\end{itemize}
}
}

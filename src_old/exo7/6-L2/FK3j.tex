\uuid{FK3j}
\exo7id{7469}
\titre{exo7 7469}
\auteur{mourougane}
\organisation{exo7}
\datecreate{2021-08-10}
\isIndication{false}
\isCorrection{false}
\chapitre{Géométrie affine dans le plan et dans l'espace}
\sousChapitre{Géométrie affine dans le plan et dans l'espace}
\module{Géométrie}
\niveau{L2}
\difficulte{}

\contenu{
\texte{
On considère le plan euclidien muni d'un un repère orthonormé ($O, \overrightarrow {\imath},\overrightarrow{\jmath}$) et la courbe $(C)$ d'équation 

\begin{center}$4x^{2} - 4xy +y^{2} -3x -y - 1= 0$ \end{center}
}
\begin{enumerate}
    \item \question{Montrer que $(C)$ est une parabole.}
    \item \question{Trouver un repère orthonormé ($S, \overrightarrow {u_{1}},\overrightarrow{u_{2}}$) tel que $(C)$ ait une équation de la forme $ x^{2} = 2py$ dans ce repère.}
\end{enumerate}
}

\uuid{Quyo}
\exo7id{6953}
\titre{exo7 6953}
\auteur{ruette}
\organisation{exo7}
\datecreate{2013-01-24}
\isIndication{false}
\isCorrection{true}
\chapitre{Variance, covariance, fonction génératrice}
\sousChapitre{Variance, covariance, fonction génératrice}
\module{Probabilité et statistique}
\niveau{L3}
\difficulte{}

\contenu{
\texte{
Soit $X$ une variable aléatoire réelle dans $L^2$.
Montrer que $\displaystyle \text{Var}(X)=\min_{t\in\Rr}E((X-t)^2)$.
}
\reponse{
Soit $g(t)=E((X-t)^2)=E(X^2)-2tE(X)+t^2$. C'est une parabole dont le
minimum est en $t=E(X)$ (on peut aussi dériver $g$ et trouver le minimum
de cette manière).
}
}

\uuid{3P5M}
\exo7id{4197}
\titre{exo7 4197}
\auteur{quercia}
\organisation{exo7}
\datecreate{2010-03-11}
\isIndication{false}
\isCorrection{true}
\chapitre{Fonction de plusieurs variables}
\sousChapitre{Extremums locaux}
\module{Analyse}
\niveau{L2}
\difficulte{}

\contenu{
\texte{
$D_1$, $D_2$, $D_3$ sont trois droites d'un plan portant les côtés
d'un triangle équilatéral de côté~$a$.
On pose $$\varphi : {D_1\times D_2\times D_3} \to \R, {(M,N,P)} \mapsto {MN+NP+PM.}$$
Déterminer $\min\varphi$ et les triplets $(M,N,P)$ où ce minimum est atteint.
}
\reponse{
Le minimum demandé existe car $\varphi(M,N,P)\to+\infty$ quand l'un au
moins des points $M,N,P$ tend vers l'infini sur sa droite personnelle.
Soient $D_1\cap D_2 = \{A\}$, $D_2\cap D_3 = \{B\}$, $D_3\cap D_1 = \{C\}$
et $A'$, $B'$, $C'$ les milieux de $[B,C]$, $[C,A]$ et $[A,B]$.
Déjà on a $\varphi(B',C',A') = \frac32a$ et $\varphi(A,N,P) \ge 2AP \ge a\sqrt3$
donc le minimum n'est pas atteint lorsque l'un des points $M,N,P$ est confondu
avec l'un des points $A,B,C$, ni non plus si l'un des points $M,N,P$
est hors du triangle $ABC$.
Pour $N,P$ fixés hors de $D_1$, on fait varier $M$ sur $D_1$~: $M = A + t\vec{AB}$ avec $t\in\R$
et on considère $f(t) = \varphi(M,N,P)$.
Alors $f'(t) = \Bigl(\vec{AB} \mid \frac{\vec{MN}}{MN} + \frac{\vec{MP}}{MP}\Bigr)$
donc $f(t)$ est minimal lorsque $D_1$ est la bissectrice extérieure des
demi-droites $[MN)$ et $[MP)$.

Soit $(M,N,P)$ un triplet réalisant le minimum de~$\varphi$ et $\alpha,\beta,\gamma$
les angles du triangle $MNP$ en $P,M$ et $N$. Les angles du triangle
$AMN$ sont $\pi/3$, $(\pi-\beta)/2$ et $(\pi-\gamma)/2$ d'où
$2\pi/3 = \beta+\gamma = \pi-\alpha$ et donc $\alpha = \pi/3 = \beta = \gamma$.
On en déduit que $(MP)$ est paralèle à $(AB)$, $(MN)$ à $(BC)$ et $(NP)$ à $(AC)$
puis que $(M,N,P) = (B',C',A')$.
}
}

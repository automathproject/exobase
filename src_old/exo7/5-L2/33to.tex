\uuid{33to}
\exo7id{4592}
\titre{exo7 4592}
\auteur{quercia}
\organisation{exo7}
\datecreate{2010-03-14}
\isIndication{false}
\isCorrection{true}
\chapitre{Série entière}
\sousChapitre{Développement en série entière}
\module{Analyse}
\niveau{L2}
\difficulte{}

\contenu{
\texte{
Soit $(a_n)$ une suite complexe donnée, on construit dans cet exercice
une fonction $f:\R \to \R$ de classe $\mathcal{C}^\infty$ telle que pour tout
entier~$n$ on ait $f^{(n)}(0) = n!\,a_n$.

Soit $\varphi : \R \to \R$ une fonction de classe $\mathcal{C}^\infty$
vérifiant~: $\forall\ x\in{[-1,1]},\ \varphi(x)=1$ et
$\forall\ x\notin{[-2,2]},\ \varphi(x)=0$
(l'existence de $\varphi$ fait l'objet de la question {2.}).
On pose $\varphi_n(x) = x^n\varphi(x)$,
$M_n = \max(\|\varphi_n'\|_\infty,\dots,\|\varphi_n^{(n)}\|_\infty)$
et $f(x) = \sum_{n=0}^\infty a_nx^n\varphi(\lambda_nx)$
où $(\lambda_n)$ est une suite de réels strictement
positifs, tendant vers $+\infty$ et telle que $\sum |a_n|M_n/\lambda_n$
converge.
}
\begin{enumerate}
    \item \question{Montrer que $f$ est bien définie, est
    de classe $\mathcal{C}^\infty$ sur $\R$ et vérifie $f^{(n)}(0) = n!\,a_n$.}
\reponse{Pour $x\ne 0$ la série comporte un nombre fini de termes non nuls
    au voisinage de~$x$, donc est $\mathcal{C}^\infty$ au voisinage de~$x$.
    On a $|f^{(k)}(x)| = \Bigl|\sum_{n=0}^\infty a_n\lambda_n^{k-n}\varphi_n^{(k)}(\lambda_nx)\Bigr|
                    \le\sum_{n=0}^\infty |a_n|\lambda_n^{k-n}M_n
                    \le \text{cste}(k) + \sum_{n=k+1}^\infty|a_n|M_n/\lambda_n$
    en supposant $\lambda_n\ge 1$ pour $n\ge k$, donc $f^{(k)}$ est bornée
    sur~$\R$. Ceci implique que $f$ est $\mathcal{C}^\infty$ en $0$ et on a le
    développemment limité~: $f(x) = \sum_{n=0}^k a_nx^n +  o(x^k)$ car
    $\phi \equiv 1$ au voisinage de~$0$ donc $f^{(k)}(0) = k!\,a_k$.}
    \item \question{Construction de~$\varphi$~: à l'aide de fonctions
    du type $x \mapsto\exp(-1/x)$ construire une fonction~$\psi$
    de classe $\mathcal{C}^\infty$ sur~$[0,+\infty[$ nulle sur
    $[0,1]\cup[2,+\infty[$ et strictement positive sur~$]1,2[$.

    Vérifier alors que $\varphi(x) =  \int_{t=|x|}^{+\infty}\psi(t)\,d t\Bigm/ \int_{t=0}^{+\infty}\psi(t)\,d t$
    convient.}
\reponse{$\psi(x) = \exp\Bigl(\frac1{(1-x)(x-2)}\Bigr)$ sur $]1,2[$, $\psi(x) = 0$ ailleurs.}
\end{enumerate}
}

\uuid{5DYb}
\exo7id{5750}
\titre{exo7 5750}
\auteur{rouget}
\organisation{exo7}
\datecreate{2010-10-16}
\isIndication{false}
\isCorrection{true}
\chapitre{Série entière}
\sousChapitre{Autre}
\module{Analyse}
\niveau{L2}
\difficulte{}

\contenu{
\texte{
Soit  $\sum_{n=0}^{+\infty}a_nz^n$ une série entière de rayon $R > 0$ et telle que $a_0 = 1$ (ou plus généralement $a_0\neq 0$).
}
\begin{enumerate}
    \item \question{Montrer qu'il existe une et une seule suite $(b_n)_{n\in\Nn}$ telle que $\forall n\in\Nn$,  $\sum_{k=0}^{n}a_kb_{n-k}=\delta_{0,n}$.}
\reponse{Montrons par récurrence que $\forall n\in\Nn$, $b_n$ existe et est unique.

\textbullet~Puisque $a_0 = 1$, $a_0b_0=1\Leftrightarrow b_0=1$. Ceci montre l'existence et l'unicité de $b_0$.

\textbullet~Soit $n\in\Nn$. Supposons avoir démontré l'existence et l'unicité de $b_0$, $b_1$, \ldots, $b_n$. 

Alors $a_0b_{n+1}+a_1b_n+\ldots+a_nb_1+a_{n+1}b_0=0\Leftrightarrow b_{n+1}=-a_1b_n-\ldots-a_nb_1-a_{n+1}b_0$. Ceci montre l'existence

et l'unicité de $b_{n+1}$.

On a montré par récurrence que la suite $(b_n)$ existe et est unique.}
    \item \question{Montrer que la série entière $\sum_{n=0}^{+\infty}b_nz^n$ a un rayon strictement positif.}
\reponse{Il faut alors vérifier que la série entière associée à la suite $(b_n)_{n\in\Nn}$ a un rayon de convergence strictement positif .

Soit $R > 0$ le rayon de la série associée à la suite $(a_n)_{n\in\Nn}$ et soit $r$ un réel tel que $0 < r < R$.

On sait que la suite $(a_nr^n)_{n\in\Nn}$ est bornée et il existe $M > 0$ tel que pour tout entier naturel $n$, $|a_n|\leqslant\frac{M}{r^n}$.

$b_0 = 1$ puis $|b_1| = |-a_1b_0|\leqslant\frac{M}{r}$ puis $|b_2|=|-a_2b_0-a_1b_1|\leqslant\frac{M}{r^2}+\frac{M}{r}\times\frac{M}{r}=\frac{M(M+1)}{r^2}$ puis

\begin{center}
$|b_3| =|-a_3b_0-a_2b_1-a_1b_2|\leqslant\frac{M}{r^3}+\frac{M}{r^2}\times+\frac{M}{r}+\frac{M}{r}\times\frac{M(M+1)}{r^2}=\frac{M(M^2+2M+1)}{r^3}=\frac{M(M+1)^2}{r^3}$.
\end{center}

Montrons alors par récurrence que $\forall n\in\Nn^*$, $b_n|\leqslant\frac{M(M+1)^{n-1}}{r^n}$.

\textbullet~C'est vrai pour $n=1$.

\textbullet~Soit $n\geqslant1$, supposons que $\forall k\in\llbracket1,n\rrbracket$, $|b_k|\leqslant\frac{M(M+1)^{k-1}}{r^k}$. Alors 

\begin{align*}\ensuremath
\left|b_{n+1}\right|&\leqslant|-a_{n+1}b_0| +|-a_nb_1| + ...+ |-a_1b_n|\leqslant\frac{M}{r^{n+1}}+\sum_{k=1}^{n}\frac{M(M+1)^{k-1}}{r^k}\times\frac{M}{r^{n+1-k}}\\
 &=\frac{M}{r^{n+1}}\left(1+ M\sum_{k=1}^{n}(M+1)^{k-1}\right)=\frac{M}{r^{n+1}}\left(1+ M\frac{(M+1)^n-1}{(M+1)-1}\right)=\frac{M(M+1)^n}{r^{n+1}}.
\end{align*}

On a montré par récurrence que pour tout entier naturel non nul $n$, $|b_n|\leqslant\frac{M(M+1)^n}{r^{n+1}}$. En particulier , le rayon $R'$ de la série entière associée à la suite $(b_n)_{n\in\Nn}$ vérifie $R'\geqslant\frac{r}{M+1}> 0$. Ceci valide les calculs initiaux sur $]-\rho,\rho[$ où $\rho=\text{Min}(R,R')>0$ et donc l'inverse d'une fonction $f$ développable en série entière à l'origine et telle que $f(0)\neq0$ est développable en série entière à l'origine.}
\end{enumerate}
}

\uuid{OfYK}
\exo7id{4173}
\titre{exo7 4173}
\auteur{quercia}
\organisation{exo7}
\datecreate{2010-03-11}
\isIndication{false}
\isCorrection{true}
\chapitre{Fonction de plusieurs variables}
\sousChapitre{Dérivée partielle}
\module{Analyse}
\niveau{L2}
\difficulte{}

\contenu{
\texte{
Soit $U$ un ouvert convexe de~$\R^2$ contenant~$(0,0)$
et $f : U \to \R$ une fonction de classe $\mathcal{C}^\infty$ telle que $f(0,0) = 0$.
}
\begin{enumerate}
    \item \question{Montrer qu'il existe ${g,h} : U \to \R$ de classe $\mathcal{C}^\infty$
    telles que~:
    $$\forall\ (x,y)\in U,\ f(x,y) = xg(x,y) + yh(x,y).$$}
    \item \question{Y a-t-il unicité de $g$ et $h$~?}
    \item \question{Généraliser au cas où $U$ n'est pas convexe.}
\reponse{
$f(x,y) =  \int_{t=0}^1 (x\frac{\partial f}{\partial x}(tx,ty) + y\frac{\partial f}{\partial y}(tx,ty))\,d t$.
}
\end{enumerate}
}

\uuid{ajuP}
\exo7id{4854}
\titre{exo7 4854}
\auteur{quercia}
\organisation{exo7}
\datecreate{2010-03-16}
\isIndication{false}
\isCorrection{true}
\chapitre{Topologie}
\sousChapitre{Fonctions vectorielles}
\module{Analyse}
\niveau{L2}
\difficulte{}

\contenu{
\texte{
Soient ${\vec e_1,\vec e_2,\vec e_3} : {I\subset \R} \to {\R^3}$
de classe $\mathcal{C}^1$ telles que pour tout $t\in I$,
${\cal B}_t = (\vec e_1(t),\vec e_2(t), \vec e_3(t))$ est une base orthonorm{\'e}e
de $\R^3$ (base orthonorm{\'e}e mobile).
}
\begin{enumerate}
    \item \question{Soit $M_t$ la matrice dans ${\cal B}_t$ des vecteurs d{\'e}riv{\'e}s
    $\vec e_1\,'(t), \vec e_2\,'(t), \vec e_3\,'(t)$.
    Montrer que $M_t$ est antisym{\'e}trique.}
    \item \question{En d{\'e}duire qu'il existe un vecteur $\vec \Omega(t)$ tel que
    $\vec e_i\,'(t) = \vec \Omega(t)\wedge\vec e_i(t)$, $i=1,2,3$.}
    \item \question{Si $\vec e_1,\vec e_2,\vec e_3$ sont de classe $\mathcal{C}^2$, montrer que $\vec \Omega$
    est de classe $\mathcal{C}^1$ et calculer $\vec e_i\,''$ en fonction de
    $\vec \Omega$, $\vec \Omega\,'$ et $\vec e_i$.}
\reponse{
$\vec e_i\,'' = \vec\Omega\,'\wedge\vec e_i
             + (\vec\Omega|\vec e_i)\vec\Omega - \|\vec\Omega\|^2\vec e_i$.
}
\end{enumerate}
}

\uuid{w4A4}
\exo7id{5731}
\titre{exo7 5731}
\auteur{rouget}
\organisation{exo7}
\datecreate{2010-10-16}
\isIndication{false}
\isCorrection{true}
\chapitre{Suite et série de fonctions}
\sousChapitre{Continuité, dérivabilité}
\module{Analyse}
\niveau{L2}
\difficulte{}

\contenu{
\texte{
Soit $f(x) =\sum_{n=1}^{+\infty}\frac{(-1)^{n-1}}{\ln(nx)}$.
}
\begin{enumerate}
    \item \question{Domaine de définition de $f$. On étudie ensuite $f$ sur $]1,+\infty[$.}
\reponse{Pour $n$ entier naturel non nul, on note $f_n$ la fonction $x\mapsto\frac{(-1)^n}{\ln(nx)}$. Pour tout réel $x$, $f(x)$ existe si et seulement si chaque $f_n(x)$, $n\in\Nn^*$, existe et la série numérique de terme général $f_n(x)$, $n\in\Nn^*$, converge.

Pour $n\in\Nn^*$ et $x\in\Rr$, $f_n(x)$ existe si et seulement si $x>0$ et $x\neq \frac{1}{n}$.

Soit donc $x\in D =]0,+\infty[\setminus\left\{\frac{1}{p},\;p\in\Nn^*\right\}$.

Pour $n>\frac{1}{x}$, on a $\ln(nx)>0$. On en déduit que la suite $\left(\frac{1}{\ln(nx)}\right)_{n\in\Nn^*}$ est positive et décroissante à partir d'un certain et tend vers $0$ quand $n$ tend vers $+\infty$. Ainsi, la série numérique de terme général $f_n(x)$ converge en vertu du critère spécial aux séries alternées et donc $f(x)$ existe. 

\begin{center}
\shadowbox{
Le domaine de définition de $f$ est $D=]0,+\infty[\setminus\left\{\frac{1}{n},\;n\in\Nn^*\right\}$.
}
\end{center}}
    \item \question{Continuité de $f$ et limites de $f$ en $1$ et $+\infty$.}
\reponse{\textbf{Limite de $f$ en $+\infty$.}
Soit $x > 1$. Donc $f(x)$ existe. Pour tout entier naturel non nul $n$, $\ln(nx) > 0$. On en déduit que la suite $\left(\frac{1}{\ln(nx)}\right)_{n\in\Nn^*}$ est décroissante. On sait alors que la valeur absolue de $f(x)$ est majorée par la valeur absolue du premier terme de la série. Ainsi

\begin{center}
$\forall x>1$, $|f(x)|\leqslant\left|\frac{(-1)^0}{\ln(x)}\right|=\frac{1}{\ln x}$,
\end{center}

et en particulier 

\begin{center}
\shadowbox{
$\lim_{x \rightarrow +\infty}f(x) = 0$.
}
\end{center}

On peut noter de plus que pour $x>1$, $f(x)$ est du signe du premier terme de la série à savoir $\frac{1}{\ln(x)}$ et donc $\forall x\in]1,+\infty[$, $f(x) > 0$.

\textbf{Convergence uniforme sur$]1,+\infty[$.} D'après une majoration classique du reste à l'ordre $n$ alternée d'une série alternée, pour $x > 1$ et $n$ naturel non nul,

\begin{center}
$|R_n(x)| =\left|\sum_{k=n+1}^{+\infty}\frac{(-1)^{k-1}}{\ln(kx)}\right|\leqslant\left|\frac{(-1)^{n}}{\ln((n+1)x)}\right|=\frac{1}{\ln((n+1)x}\leqslant\frac{1}{n+1}$.
\end{center}

Donc, pour tout entier naturel non nul, $\underset{x\in]1,+\infty[}{\text{sup}}|R_n(x)|\leqslant\frac{1}{\ln(n+1)}$ et donc $\lim_{n \rightarrow +\infty}\underset{x\in]1,+\infty[}{\text{sup}}|R_n(x)|=0$. La série de fonctions de terme général $f_n$ converge uniformément vers sa somme sur $]1,+\infty[$.

\textbf{Continuité sur $]1,+\infty[$.} Chaque fonction $f_n$, $n\in\Nn^*$ est continue sur $]1,+\infty[$ et donc $f$ est donc continue sur $]1,+\infty[$ en tant que limite uniforme sur $]1,+\infty[$ d'une suite de fonctions continues sur $]1,+\infty[$.

\begin{center}
\shadowbox{
$f$ est continue sur $]1,+\infty[$.
}
\end{center}

\textbf{Limite en $1$ à droite.} Soit $n\geqslant2$. Quand $x$ tend vers $1$ par valeurs supérieures, $f_n(x)$ tend vers $\ell_n=\frac{(-1)^{n-1}}{\ln(n)}$. Puisque la série de fonctions de terme général $f_n$, $n\geqslant2$, converge uniformément vers sa somme sur $]1,+\infty[$, le théorème d'interversion des limites permet d'affirmer que la série numérique de terme général $\ell_n$, $n\geqslant2$ converge et que la fonction $x\mapsto f(x) -\frac{1}{\ln(x)}=\sum_{n=2}^{+\infty}f_n(x)$ tend vers le réel $\sum_{n=2}^{+\infty}\frac{(-1)^{n-1}}{\ln(n)}$ quand $x$ tend vers $1$ par valeurs supérieures ou encore 

\begin{center}
\shadowbox{
$f(x)\underset{x\rightarrow1^+}{=}\frac{1}{\ln x}+O(1)$ et en particulier, $\displaystyle\lim_{\substack{x\rightarrow1\\
x>1}}f(x) = +\infty$.
}
\end{center}}
    \item \question{Montrer que $f$ est de classe $C^1$ sur $]1,+\infty[$ et dresser son tableau de variation.}
\reponse{La série de fonctions de terme général $f_n$, $n\geqslant1$, converge simplement vers la fonction $f$ sur $]1,+\infty[$. De plus chaque fonction $f_n$ est de classe $C^1$ sur $]1,+\infty[$ et pour $n\in\Nn^*$ et $x>1$, 

\begin{center}
$f_n'(x) =\frac{(-1)^n}{x\ln^2(nx)}$.
\end{center}

Il reste à vérifier la convergence uniforme de la série de fonctions de terme général $f_n'$ sur $]1,+\infty[$.

Soit $x > 1$. La série de terme général $f_n'(x)$ est alternée car son terme général est alterné en signe et sa valeur absolue à savoir $\frac{1}{x\ln^2(nx)}$ tend vers zéro quand $n$ tend vers $+\infty$ en décroissant. Donc, d'après une majoration classique du reste à l'ordre $n$ d'une série alternée,

\begin{center}
$|R_n(x)|=\left|\sum_{k=n+1}^{+\infty}\frac{(-1)^k}{x\ln^2(kx)}\right|\leqslant\left|\frac{(-1)^{n+1}}{x\ln^2((n+1)x)}\right|=\frac{1}{x\ln^2((n+1)x)}\leqslant\frac{1}{\ln^2(n+1)}$.
\end{center}

Par suite, $\underset{x\in]1,+\infty[}{\text{sup}}|R_n(x)|\leqslant\frac{1}{\ln^2(n+1)}$ et donc $\lim_{n \rightarrow +\infty}\underset{x\in]1,+\infty[}{\text{sup}}|R_n(x)|=0$. Ainsi, la série de fonctions de terme général $f_n'$, $n\geqslant1$, converge uniformément sur $]1,+\infty[$.

En résumé,

\textbullet~la série de fonctions de terme général $f_n$, $n\geqslant1$, converge simplement vers $f$ sur $]1,+\infty[$,

\textbullet~chaque fonction $f_n$, $n\geqslant1$, est de classe $C^1$ sur $]1,+\infty[$,

\textbullet~la série de fonctions de terme général $f_n'$ converge uniformément sur $]1,+\infty[$.

D'après un corollaire du théorème de dérivation terme à terme, $f$ est de classe $C^1$ sur $]1,+\infty[$ et sa dérivée s'obtient par dérivation terme à terme. 

\begin{center}
\shadowbox{
$f$ est de classe $C^1$ sur $]1,+\infty[$ et $\forall x>1$, $f'(x) =\sum_{n=1}^{+\infty}\frac{(-1)^n}{x\ln^2(nx)}$.
}
\end{center}

Pour $x >1$, puisque la série de somme $f'(x)$ est alternée, $f'(x)$ est du signe du premier terme de la somme à savoir $-\frac{1}{x\ln^2x}$. Par suite, $\forall x\in]-1,1[$, $f'(x)\leqslant0$ et $f$ est donc strictement décroissante sur $]1,+\infty[$.

\begin{center}
\shadowbox{
La fonction $f$ est décroissante sur $]1,+\infty[$.
}
\end{center}}
\end{enumerate}
}

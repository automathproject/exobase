\uuid{QX44}
\exo7id{4119}
\titre{exo7 4119}
\auteur{quercia}
\organisation{exo7}
\datecreate{2010-03-11}
\isIndication{false}
\isCorrection{true}
\chapitre{Equation différentielle}
\sousChapitre{Equations différentielles non linéaires}
\module{Analyse}
\niveau{L2}
\difficulte{}

\contenu{
\texte{
Soit $y$ la solution maximale de l'équation $y' =x^3+y^3$ telle que
$y(0) = a \ge 0$, et $I = {]\alpha,\beta[}$ son intervalle de définition.
Montrer que $y$ est strictement croissante sur $[0,\beta[$, que $\beta$ est borné,
et que $y \to +\infty$ lorsque $x\to\beta^-$.
}
\reponse{
Si $a > 0$, $y'(0) = a^3$,
si $a=0$, $y''(0) = 0$, $y'''(0) = 0$, $y^{IV}(0) = 6$.
Donc $y$ est croissante au voisinage de $0$.

Si $y'>0$ sur $]0,\gamma[$, alors $y(\gamma) > 0$ donc $y'(\gamma)>0$ et
$y'>0$ sur $[\gamma,\gamma+\varepsilon[$ donc, par connexité, $y'>0$ sur
$]0,\beta[$.

$y'\ge y^3  \Rightarrow  1 \le \frac{y'}{y^3}  \Rightarrow  x \le \frac1{2a^2}-\frac1{2y^2} \le \frac1{2a^2}$.
}
}

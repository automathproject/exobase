\uuid{oMoe}
\exo7id{5893}
\titre{exo7 5893}
\auteur{rouget}
\organisation{exo7}
\datecreate{2010-10-16}
\isIndication{false}
\isCorrection{true}
\chapitre{Fonction de plusieurs variables}
\sousChapitre{Dérivée partielle}
\module{Analyse}
\niveau{L2}
\difficulte{}

\contenu{
\texte{
Soit $f$ une application de $\Rr^n$ dans $\Rr$ de classe $C^1$. On dit que $f$ est positivement homogène de degré $r$ ($r$ réel donné) si et seulement si $\forall\lambda\in]0,+\infty[$, $\forall x\in\Rr^n$, $f(\lambda x) =\lambda^rf(x)$.

Montrer pour une telle fonction l'identité d'\textsc{Euler} :

\begin{center}
$\forall x = (x_1,...,x_n)\in \Rr^n$ $\sum_{i=1}^{n}x_i \frac{\partial f}{\partial x_i}(x) = rf(x)$.
\end{center}
}
\reponse{
On dérive par rapport à $\lambda$ les deux membres de l'égalité $f(\lambda x) =\lambda^rf(x)$ et on obtient

\begin{center}
$\forall x=(x_1,...,x_n)\in\Rr^n$, $\forall\lambda>0$, $\sum_{i=1}^{n}x_i \frac{\partial f}{\partial x_i}(\lambda x)=r\lambda^{r-1}f(x)$,
\end{center}

et pour $\lambda=1$, on obtient

\begin{center}
$\forall x = (x_1,...,x_n)\in \Rr^n$ $\sum_{i=1}^{n}x_i \frac{\partial f}{\partial x_i}(x) = rf(x)$.
\end{center}
}
}

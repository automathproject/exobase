\uuid{VAjd}
\exo7id{4568}
\titre{exo7 4568}
\auteur{quercia}
\organisation{exo7}
\datecreate{2010-03-14}
\isIndication{false}
\isCorrection{true}
\chapitre{Série entière}
\sousChapitre{Rayon de convergence}
\module{Analyse}
\niveau{L2}
\difficulte{}

\contenu{
\texte{
Rayon de convergence $R$ de la série entière $\sum_{n=1}^\infty \frac{x^n}{\sum_{k=1}^n k^{-\alpha}}$
et étude pour $x=\pm R$.
}
\reponse{
$\sum_{k=1}^n k^{-\alpha}\sim
\begin{cases}
\frac{n^{1-\alpha}}{1-\alpha} & \text{ si } \alpha<1,\cr
\ln(n) & \text{ si } \alpha=1,\cr
\zeta(\alpha) & \text{ si } \alpha>1.\cr
\end{cases}$
Dans les trois cas, on obtient $R=1$.

Il y convergence en~$x=1$ si et seulement si $\alpha<0$ et il y a divergence grossière en~$x=-1$
lorsque $\alpha>1$ vu les équivalents. Pour $\alpha\le 1$ et $x=-1$ il y a convergence (CSA).
}
}

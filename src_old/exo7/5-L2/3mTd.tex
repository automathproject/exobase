\uuid{3mTd}
\exo7id{4556}
\titre{exo7 4556}
\auteur{quercia}
\organisation{exo7}
\datecreate{2010-03-14}
\isIndication{false}
\isCorrection{true}
\chapitre{Suite et série de fonctions}
\sousChapitre{Autre}
\module{Analyse}
\niveau{L2}
\difficulte{}

\contenu{
\texte{
Soit $f : \R \to \R$ continue et $2\pi$-périodique. Pour $n\in\N^*$, on pose
$F_n(x)=\frac{1}{n} \int_{t=0}^nf(x+t)f(t)\,d t$.
}
\begin{enumerate}
    \item \question{Montrer que la suite $(F_n)$ converge vers une fonction $F$ que l'on précisera.}
\reponse{Soit $k=\lfloor n/2\pi\rfloor$.
    On a $F_n(x) = \frac{2k\pi}n \int_{t=0}^{2\pi}f(x+t)f(t)\,d t + \frac1n \int_{t=2k\pi}^nf(x+t)f(t)\,d t
    \to \int_{t=0}^{2\pi}f(x+t)f(t)\,d t$  lorsque $n\to\infty$.}
    \item \question{Nature de la convergence~?}
\reponse{Uniforme.}
    \item \question{Prouver $\|F\|_{\infty}= |F(0)|$.}
\reponse{Cauchy-Schwarz.}
\end{enumerate}
}

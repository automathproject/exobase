\uuid{cX9D}
\exo7id{4331}
\titre{exo7 4331}
\auteur{quercia}
\organisation{exo7}
\datecreate{2010-03-12}
\isIndication{false}
\isCorrection{true}
\chapitre{Intégration}
\sousChapitre{Intégrale de Riemann dépendant d'un paramètre}
\module{Analyse}
\niveau{L2}
\difficulte{}

\contenu{
\texte{
Soit $x\in{[0,n]}$. Montrer que $(1-x/n)^n \le e^{-x}$.
En déduire $\lim_{n\to\infty}  \int_{x=0}^n(1-x/n)^n\,d x$.
}
\reponse{
Soit $f_n(x) = (1-x/n)^n$ si $0\le x \le n$ et $f_n(x) = 0$ si
$x>n$. Alors $f_n(x)$ converge simplement vers $e^{-x}$ et il y a
convergence dominée.
}
}

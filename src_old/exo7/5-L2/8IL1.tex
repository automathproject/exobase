\uuid{8IL1}
\exo7id{5890}
\titre{exo7 5890}
\auteur{rouget}
\organisation{exo7}
\datecreate{2010-10-16}
\isIndication{false}
\isCorrection{true}
\chapitre{Fonction de plusieurs variables}
\sousChapitre{Fonctions implicites}
\module{Analyse}
\niveau{L2}
\difficulte{}

\contenu{
\texte{
Montrer que $\begin{array}[t]{cccc}
\varphi~:&\Rr^2&\rightarrow&\Rr^2\\
 &(x,y)&\mapsto&(e^x-e^y , x+y)
 \end{array}$ est un $C^1$-difféomorphisme de $\Rr^2$ sur lui-même.
}
\reponse{
Soit $(x,y,z,t)\in\Rr^4$.

\begin{align*}\ensuremath
\varphi(x,y)=(z,t)&\Leftrightarrow\left\{
\begin{array}{l}
e^x-e^y=z\\
x+y=t
\end{array}
\right.\Leftrightarrow\left\{
\begin{array}{l}
y=t-x\\
e^x-e^{t-x}=z
\end{array}
\right.\Leftrightarrow\left\{
\begin{array}{l}
y=t-x\\
(e^x)^2-ze^x-e^t=0
\end{array}
\right.\\
 &\Leftrightarrow\left\{
\begin{array}{l}
y=t-x\\
e^x=z-\sqrt{z^2+4e^t}\;\text{ou}\;e^x=z+\sqrt{z^2+4e^t}
\end{array}
\right.\\
&\Leftrightarrow\left\{
\begin{array}{l}
e^x=z+\sqrt{z^2+4e^t}\\
y=t-x
\end{array}
\right.\;(\text{car}\;z-\sqrt{z^2+4e^t}<z-\sqrt{z^2}=z-|z|\leqslant0)\\
&\Leftrightarrow\left\{
\begin{array}{l}
x=\ln(z+\sqrt{z^2+4e^t})\\
y=t-\ln(z+\sqrt{z^2+4e^t})
\end{array}
\right.\;(\text{car}\;z+\sqrt{z^2+4e^t}>z+\sqrt{z^2}=z+|z|\geqslant0).
\end{align*}

Ainsi, tout élément $(z,t)\in\Rr^2$ a un antécédent et un seul dans $\Rr^2$ par $\varphi$ et donc $\varphi$ est une bijection de $\Rr^2$ sur lui-même.

La fonction $\varphi$ est de classe $C^1$ sur $\Rr^2$ de jacobien $J_\varphi(x,y)=\left|
\begin{array}{cc}
e^x&-e^y\\
1&1
\end{array}
\right|=e^x+e^y$. Le jacobien de $\varphi$ ne s'annule pas sur $\Rr^2$. En résumé, $\varphi$ est une bijection de $\Rr^2$ sur lui-même, de classe $C^1$ sur $\Rr^2$ et le jacobien de $\varphi$ ne s'annule pas sur $\Rr^2$. On sait alors que

\begin{center}
\shadowbox{
$\varphi$ est un $C^1$-difféomorphisme de $\Rr^2$ sur lui-même.
}
\end{center}
}
}

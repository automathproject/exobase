\uuid{8DRy}
\exo7id{4555}
\titre{exo7 4555}
\auteur{quercia}
\organisation{exo7}
\datecreate{2010-03-14}
\isIndication{false}
\isCorrection{true}
\chapitre{Suite et série de fonctions}
\sousChapitre{Autre}
\module{Analyse}
\niveau{L2}
\difficulte{}

\contenu{
\texte{
Soin $(a_n)_{n\ge 1}$ une suite complexe telle que la série $\sum a_n$ converge. On pose~:
$f(h) = \sum_{n=1}^\infty a_n\frac{\sin^2(nh)}{(nh)^2}$ si $h\ne 0$ et
$f(0) = \sum_{n=1}^\infty a_n$. Étudier le domaine de définition et la
continuité de~$f$.
}
\reponse{
On suppose $h$ réel.
La série converge localement normalement sur $\R^*$ donc $f$ est définie sur~$\R$
et continue sur $\R^*$. Continuité en~$0$~: on pose $A_n = \sum_{k=n}^\infty a_k$
et $\varphi(t) = \frac{\sin^2(t)}{t^2}$ si $t\ne 0$, $\varphi(0) = 1$ ($\varphi$ est $\mathcal{C}^\infty$
sur~$\R$ comme somme d'une série entière de rayon infini). Pour $h\ne 0$ on a~:
$$f(h) = \sum_{n=1}^\infty(A_n-A_{n+1})\varphi(nh)
       = A_1\varphi(h) + \sum_{n=2}^\infty A_n(\varphi(nh)-\varphi((n-1)h))
       = A_1\varphi(h) + \sum_{n=2}^\infty A_n \int_{t=(n-1)h}^{nh}\varphi'(t)\,d t.$$
Cette dernière série est uniformément convergente sur~$\R$ car $A_n \to 0$ (lorsque $n\to\infty$) et
$ \int_{t=0}^{+\infty}|\varphi'(t)|\,d t$ est convergente.
}
}

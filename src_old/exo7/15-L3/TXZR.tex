\uuid{TXZR}
\exo7id{1878}
\titre{exo7 1878}
\auteur{maillot}
\organisation{exo7}
\datecreate{2001-09-01}
\isIndication{false}
\isCorrection{false}
\chapitre{Espace vectoriel normé}
\sousChapitre{Espace vectoriel normé}
\module{Topologie}
\niveau{L3}
\difficulte{}

\contenu{
\texte{
Soit $E=\R^d$ muni d'une norme $\|\cdot\|$. On rappelle qu'une application
continue $g$ de $E$ dans $E$ est dite \emph{contractante} s'il existe
$K\in ]0,1[$ tel que
$$ \|g(x)-g(y)\|\le K \|x-y\| \qquad \forall x,y\in E.$$ On rappelle
aussi que toute application contractante admet un unique point fixe.

Soit $f$ une application continue de $E$ dans $E$ telle qu'il existe un
entier $n$ tel que $f^n$ soit contractante. On note $x_0$ le point
fixe de $f^n$.
}
\begin{enumerate}
    \item \question{Montrer que tout point fixe de $f$ est un point fixe de $f^n$.}
    \item \question{Montrer que si $x$ est un point fixe de $f^n$, il en est de même
pour $f(x)$.}
    \item \question{En déduire que $x_0$ est l'unique point fixe de $f$.}
\end{enumerate}
}

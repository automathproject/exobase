\uuid{1Wjh}
\exo7id{6240}
\titre{exo7 6240}
\auteur{queffelec}
\organisation{exo7}
\datecreate{2011-10-16}
\isIndication{false}
\isCorrection{false}
\chapitre{Théorème du point fixe}
\sousChapitre{Théorème du point fixe}
\module{Topologie}
\niveau{L3}
\difficulte{}

\contenu{
\texte{
On considère $T:C([0,1])\to C([0,1])$ qui à $f$ associe $F$ définie par 

$$F(t)= \left \{\begin{array}{ccc}
    & 3/4\  f(3t) & {\rm si}\ \ 0\leq t\leq 1/3\\
    & 1/4 + 1/2\  f(2-3t)   &{\rm si}\ \ 1/3\leq t\leq 2/3\\
&1/4 +3/4\ f(3t-2)&{\rm si}\ \ 2/3\leq t\leq 1 

\end{array}\right.$$
}
\begin{enumerate}
    \item \question{Vérifier que $F$ est bien continue et que $T$ est $3/4$-contractante.}
    \item \question{On note $h$ le point fixe de $T$. Montrer par récurrence $\vert
h({{k-1}\over3^n})-h({k\over3^n})\vert\geq 2^{-n}$. 

Soit $a\in[0,1]$; montrer qu'il existe une suite $(t_n)$ telle que $\lim t_n
=a$ et $\lim \vert {{h(t_n)-h(a)}\over{t_n-a}}\vert = +\infty$.}
    \item \question{En déduire l'existence d'une fonction
continue nulle part dérivable.}
\end{enumerate}
}

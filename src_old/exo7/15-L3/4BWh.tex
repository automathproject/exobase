\uuid{4BWh}
\exo7id{6071}
\titre{exo7 6071}
\auteur{queffelec}
\organisation{exo7}
\datecreate{2011-10-16}
\isIndication{false}
\isCorrection{false}
\chapitre{Continuité, uniforme continuité}
\sousChapitre{Continuité, uniforme continuité}
\module{Topologie}
\niveau{L3}
\difficulte{}

\contenu{
\texte{
On note pour tout $x\in \Rr, \ \varphi(x)=$ dist$(x,\Zz)$.
}
\begin{enumerate}
    \item \question{Montrer que la fonction $\varphi$ est continue, $1$-périodique, et étudier
la fonction  $f$ telle que
$$f(x)=\sum_n {\varphi(2^n x)\over{2^n}}.$$}
    \item \question{On fixe $x_0\in \Rr$, et on considère les deux suites de terme 
$$z_k={1\over 2^k}E(2^kx_0),\ \ \ y_k=z_k+{1\over 2^k}.$$
Montrer que la suite $(z_k)$ cro\^\i t vers $x_0$ et que la suite $(y_k)$
décro\^\i t vers $x_0$.
Calculer ${f(z_k)-f(y_k)\over z_k-y_k}$ et en déduire que $f$ n'est pas dérivable
en $x_0$. 

 On a ainsi construit une fonction continue, nulle part dérivable.}
\end{enumerate}
}

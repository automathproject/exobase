\uuid{DZH9}
\exo7id{6513}
\titre{exo7 6513}
\auteur{drutu}
\organisation{exo7}
\datecreate{2011-10-16}
\isIndication{false}
\isCorrection{false}
\chapitre{Anneau, corps}
\sousChapitre{Anneau, corps}
\module{Algèbre et théorie des nombres}
\niveau{L3}
\difficulte{}

\contenu{
\texte{
Soit $A$ un anneau intègre et $a,b,c \in A\setminus \{0 \}$.
 Montrer que, chaque fois que les pgcd suivants existent, on a les égalités :
}
\begin{enumerate}
    \item \question{$\pgcd (ca,cb)\sim c \pgcd (a,b)$}
    \item \question{$\pgcd ( \pgcd (a,b),c)\sim \pgcd (a, \pgcd (b,c))$.
  
\medskip

 Si $A$ est en plus factoriel, montrer que}
    \item \question{$\pgcd (a,b)\sim 1$ et $\pgcd(a,c)\sim 1$ implique $\pgcd(a,bc)\sim 1$.}
    \item \question{Si $a|bc$ et $\pgcd (a,b)\sim 1$ alors $a|c$.}
    \item \question{Si $b|a$ et $c|a$ et $\pgcd (b,c)\sim 1$ alors $(bc)|a$.}
\end{enumerate}
}

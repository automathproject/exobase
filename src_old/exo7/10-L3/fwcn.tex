\uuid{fwcn}
\exo7id{2297}
\titre{exo7 2297}
\auteur{barraud}
\organisation{exo7}
\datecreate{2008-04-24}
\isIndication{false}
\isCorrection{true}
\chapitre{Polynôme}
\sousChapitre{Polynôme}
\module{Algèbre et théorie des nombres}
\niveau{L3}
\difficulte{}

\contenu{
\texte{
Soit $A$ un anneau factoriel.
}
\begin{enumerate}
    \item \question{Pour $a,b\ne 0$ on a $(a)\cdot (b)= (a)\cap(b)\ \ $ ssi $\ \ \pgcd(a,b)\thicksim 1$.}
    \item \question{Si $(a,b)$ est principal, alors $(a,b)=(\pgcd(a,b))$.}
\reponse{
Rappelons que $(a)\cdot(b)=\{\sum_{i=1}^{n}a_{i}b_{i}, n\in\Nn,
  a_{i}\in(a), b_{i}\in(b)\}=(ab)$. De plus $(ab)\subset(a)\cap(b)$ donc 
  \begin{align*}
    (ab)=(a)\cap(b)
    &\Leftrightarrow (a)\cap(b)\subset(ab)\\
    &\Leftrightarrow \forall m\in A,\ (a|m\text{ et }b|m\Rightarrow ab|m)\\
    &\Leftrightarrow \mathrm{ppcm}(a,b)\sim ab\\
    &\Leftrightarrow \mathrm{ppcm}(a,b)\sim \pgcd(a,b)\mathrm{ppcm}(a,b)\\
    &\Leftrightarrow \pgcd(a,b)\sim 1
  \end{align*}

  Si $A$ est principal, alors $\exists d\in A,\ (a,b)=(d)$. Alors
  $a\in(d)$ et $b\in(d)$ donc $d$ est un diviseur commun à $a$ et $b$. Si
  de plus $d'$ est un autre diviseur commun à $a$ et $b$, alors
  $a\in(d')$ et $b\in(d')$ et comme $(a,b)$ est le plus petit idéal
  contenant $a$ et $b$, on en déduit que $(a,b)=(d)\subset(d')$, et donc
  que $d'|d$~: finalement, $\pgcd(a,b)=d$.
}
\end{enumerate}
}

\uuid{qw2r}
\exo7id{2285}
\titre{exo7 2285}
\auteur{barraud}
\organisation{exo7}
\datecreate{2008-04-24}
\isIndication{false}
\isCorrection{true}
\chapitre{Anneau, corps}
\sousChapitre{Anneau, corps}
\module{Algèbre et théorie des nombres}
\niveau{L3}
\difficulte{}

\contenu{
\texte{
Soit $A$ un anneau. Trouver les anneaux quotients
$$
A[x]/(x),\quad A[x,y]/(x),\quad A[x,y]/(x,y),\quad
A[x_1,x_2,\dots,x_n]/(x_1,x_2,\dots,x_n)
$$
o\`u $(x)$, $(x,y)$, $(x_1,x_2,\dots,x_n)$ sont les id\'eaux engendr\'es
r\'espectivement par $x$, $x$ et $y$, $x_1$, $x_2$, ... ,$x_n$.
Sous quelle condition sur l'anneau $A$ ces id\'eaux sont-ils premiers
(maximaux) ?
}
\reponse{
\begin{itemize}
  \item   $A[X]/(X)$~: 
    $X$ est unitaire donc on dispose de la division euclidienne par $X$.
    On vérifie (comme dans le cours) que chaque classe a un et un seul
    représentant de degré $0$. On en déduit que $A[X]/(X)$ est en
    bijection avec $A$. Il reste alors à remarquer que cette bijection
    est un morphisme d'anneaux. 

    Une autre façon de dire la même chose est de remarquer que
    l'application $\phi:A[X]\to A$, $P\mapsto P(0)$ est un morphisme
    d'anneaux. $\ker\phi=(X)$ et $\mathrm{Im}\,\phi=A$. Comme $A/\ker\phi\sim
    \mathrm{Im}\,\phi$, on a bien $A[X]/(X)\sim A$.

  \item
    On peut considérer $\phi:A[X,Y]\to A[Y]$, $P\mapsto P(0,Y)$. C'est un
    morphisme d'anneaux. En séparant les termes ne dépendant que de $Y$
    des autres, on peut mettre tout polynôme $P$ de $A[X,Y]$ sous la
    forme $P=P_{1}(Y)+XP_{2}(X,Y)$ où $P_{1}\in A[Y]$ et $P_{2}\in
    A[X,Y]$. Alors $\phi(P)=0$ ssi $P_{1}=0$, ssi $P=XP_{2}$, c'est à
    dire $P\in(X)$. Ainsi $\ker\phi=(X)$. Par ailleurs, tout polynôme $P$
    de $A[Y]$ peut être vu comme un polynôme $\tilde{P}$ de $A[X,Y]$.
    Alors $P=\phi(\tilde{P})$, donc $\mathrm{Im}\,\phi=A[Y]$. Finalement~:
     $A[X,Y]/(X)\sim A[Y]$.
    
   \item $A[X,Y]/(X,Y)$~:
     Soit $\phi:A[X,Y]\to A$, $P\mapsto P(0,0)$. $\phi$ est un morphisme
     d'anneaux, et avec les notations précédentes, pour
     $P=P_{1}(Y)+XP_{2}(X,Y)$, avec $\phi(P)=0$, on a $P_{1}(0)=0$, donc
     $Y|P_{1}(Y)$. Ainsi, $P$ est la somme de deux polynômes, l'un
     multiple de $X$, l'autre multiple de $Y$ donc $P\in(X,Y)$.
     Réciproquement, si $P\in(X,Y)$, alors $P(0,0)=0$. Donc
     $\ker\phi=(X,Y)$. $\forall a\in A \phi(a)=a$ donc $\phi$ est
     surjective. Finalement $A[X,Y]/(X,Y)\sim A$.
     
   \item $A[X_{1},\dots,X_{n}]/(X_{1},\dots,X_{n})$~:
     Soit $\phi~:A[X_{1},\dots,X_{n}]\to A$, $P\mapsto P(0)$. $\phi$ est
     un morphisme d'anneaux. En regroupant tous les termes dépendant de
     $X_{n}$, puis tous les termes restant dépendant de $X_{n-1}$, et
     ainsi de suite jusqu'aux termes dépendant seulement de $X_{1}$, et
     enfin le terme constant, tout polynôme $P\in A[X_{1},\dots,X_{n}]$
     peut se mettre sous la forme
     $P=X_{n}P_{n}+X_{n-1}P_{n-1}+\dots+X_{1}P_{1}+p_{0}$, avec $P_{i}\in
     A[X_{1},\dots,X_{i}]$ (et $p_{0}\in A$). On en déduit que
     $\ker\phi=(X_{1},\dots,X_{n})$. Par ailleurs $\forall a\in A,
     \phi(a)=a$, donc $A[X_{1},\dots,X_{n}]/(X_{1},\dots,X_{n})\sim A$.
  \end{itemize}
  
  Comme un idéal est premier (resp. maximal) ssi le quotient est intègre
  (resp. un corps), on en déduit que
  \begin{itemize}
  \item dans $A[X]$, $(X)$ est premier ssi $A$ est intègre, maximal ssi $A$ est un corps,
  \item dans $A[X,Y]$, $(X)$ est premier ssi $A$ est intègre, et n'est jamais maximal,
  \item dans $A[X_{1},\dots,X_{n}]$, $(X_{1},\dots,X_{n})$ est premier
    ssi $A$ est intègre, maximal ssi $A$ est un corps.
  \end{itemize}
}
}

\uuid{Br7Z}
\exo7id{2265}
\titre{exo7 2265}
\auteur{barraud}
\organisation{exo7}
\datecreate{2008-04-24}
\isIndication{false}
\isCorrection{true}
\chapitre{Polynôme}
\sousChapitre{Polynôme}
\module{Algèbre et théorie des nombres}
\niveau{L3}
\difficulte{}

\contenu{
\texte{
\label{exoprec}
}
\begin{enumerate}
    \item \question{Si $K$ est un corps, montrer qu'un polyn\^ome $P$ de degr\'e 2 ou 3 
dans $K [x]$ est irr\'eductible si et seulement si il n'a pas 
de z\'ero dans $K$.}
\reponse{Soit $P$ un polynôme de degré $d=2$ ou $3$ de $K[X]$.

    Si $P$ a une racine $a\in K$, alors $(X-a)|P$, et $P$ n'est pas
    irréductible.

    Réciproquement, si $P=AB$ avec $A,B\in K[X]$ et $A,B\notin
    K[X]^{\times}=K\setminus\{0\}$, alors $\deg(A)\geq1$, $\deg(B)\geq1$,
    et $\deg(A)+\deg(B)=d=2$ ou $3$, donc l'un au moins des deux
    polynômes $A$ et $B$ est de degré $1$. On peut supposer que c'est
    $A$. Notons $A=aX+b$. Alors $(X+a^{-1}b)|P$, et $-a^{-1}b$ est racine
    de $P$.

    Finalement $P$ a une racine ssi $P$ n'est pas irréductible.}
    \item \question{Trouver tous les polyn\^omes irr\'eductibles
de degr\'e $2$, $3$ \`a coefficients dans  $\Zz/2\Zz$.}
\reponse{Irréductibles de degré $2$ de $\Zz/2\Zz$: Soit $P=aX^{2}+bX+c$ un
      polynôme de degré $2$. $a\neq0$ donc $a=1$.
      \begin{align*}
        P\text{ irréductible} &\Leftrightarrow P\text{ n'a pas de racine}\\
        &\Leftrightarrow
          \begin{cases}
            P(0)&\neq 0\\
            P(1)&\neq 0
          \end{cases}\\
        &\Leftrightarrow
          \begin{cases}
            P(0)&=1\\
            P(1)&=1
          \end{cases}\\
        &\Leftrightarrow
          \begin{cases}
            c&=1\\
            1+b+1&=1
          \end{cases}\\
        &\Leftrightarrow
          P=X^{2}+X+1
      \end{align*}
      Ainsi, il y a un seul irréductible de degré $2$, c'est $I_{2}=X^{2}+X+1$.

      Irréductibles de degré $3$ de $\Zz/2\Zz$: Soit $P=aX^{3}+bX^{2}+cX+d$ un
      polynôme de degré $2$. $a\neq0$ donc $a=1$.
      \begin{align*}
        P\text{ irréductible} &\Leftrightarrow P\text{ n'a pas de racine}\\
        &\Leftrightarrow
          \begin{cases}
            d&=1\\
            1+b+c+1&=1
          \end{cases}\\
        &\Leftrightarrow
          \begin{cases}
            d&=1\\
            (b,c)&=(1,0)\text{ ou }(b,c)=(0,1)
          \end{cases}\\
        &\Leftrightarrow
          P=X^{3}+X+1 \text{ ou } P=X^{3}+X^{2}+1
      \end{align*}
      Ainsi, il y a deux irréductibles de degré $3$ dans $\Zz/3\Zz[X]$:
      $I_{3}=X^{3}+X+1$ et $I'_{3}=X^{3}+X^{2}+1$.}
    \item \question{En utilisant la partie pr\'ec\'edente, montrer que 
les polyn\^omes $5x^3+8x^2+ 3x+ 15$ et  $x^5+2x^3+3x^2-6x-5$ sont
irr\'eductibles dans $\Zz[x]$.}
\reponse{Soit $P=5X^{3}+8X^{2}+3X+15\in\Zz[X]$. Soient $A$ et $B$ deux
      polynômes tels que $P=AB$. L'application $\Zz\to\Zz/2\Zz,
      n\mapsto\bar n$ induit une application $\Zz[X]\to\Zz/2\Zz[X],
      P=\sum a_{i}X^{i}\mapsto \bar P=\sum \bar{a_{i}}X^{i}$. Cette
      application est compatible avec les opérations: en particulier
      $\overline{AB}=\bar{A}\,\bar{B}$ (pourquoi?). Ainsi on a:
      $\bar{P}=\bar{A}\bar{B}$. Or $\bar{P}=X^{3}+X+1$ est irréductible,
      donc (quitte à échanger les rôles de $A$ et $B$ on peut supposer
      que) $\bar{A}=1$ et $\bar{B}=X^{3}+X+1$. On en déduit que $B$ est
      au moins de degré $3$, d'où $\deg(A)=0$. $A\in\Zz$ et $A|P$, donc
      $A|5$, $A|8$, $A|3$, et $A|15$. On en déduit que $A=\pm1$.
      Finalement, $A=\pm1$ et $B\sim P$. $P$ est donc irréductible dans
      $\Zz[X]$.

      \bigskip
      Soit $P=X^{5}+2X^{3}+3X^{2}-6x-5\in\Zz[X]$. Soient $A$ et $B$ deux
      polynômes tels que $P=AB$. On a comme précédemment:
      $\bar{P}=\bar{A}\bar{B}$ où $\bar{P}=X^{5}+X^{2}+1$. $\bar{P}$ n'a
      pas de racine dans $\Zz/2\Zz$, donc si $\bar{P}$ est réductible, il
      doit être le produit d'un irréductible de degré $2$ et d'un
      irréductible de degré $3$. Or $\bar{P}\neq I_{2}I_{3}$ et
      $\bar{P}\neq I_{2}I'_{3}$ (faire le calcul!), donc $\bar{P}$ est
      irréductible. Le même raisonnement montre alors que $P$ est
      irréductible dans $\Zz[X]$.}
    \item \question{D\'ecrire tous les polyn\^omes irr\'eductibles de degr\'e $4$
et $5$ sur $\Zz/2\Zz$.}
\reponse{Un polynôme de degré $4$ est réductible ssi il a une racine ou est
      le produit de deux irréductibles de degré $2$. Soit
      $P=\sum_{i=0}^{4}a_{i}X^{i}\in\Zz/2\Zz[X]$, avec $a_{4}=1$.
      \begin{align*}
        P\text{ irréductible}&\Leftrightarrow 
          \begin{cases}
            P(0)\neq 0\\
            P(1)\neq 0\\
            P\neq I_{2}^{2}\\
          \end{cases}\\
        &\Leftrightarrow 
          \begin{cases}
            a_{0}= 1\\
            1+a_{3}+a_{2}+a_{1}+1=1\\
            P\neq I_{2}^{2}\\
          \end{cases}\\
        &\Leftrightarrow 
          P\in\{X^{4}+X^{3}+1,X^{4}+X+1,X^{4}+X^{3}+X^{2}+X+1\}
      \end{align*}

      \bigskip
      Un polynôme de degré $5$ est irréductible ssi il n'a pas de racine
      et l'est pas le produit d'un irréductible de degré $2$ et d'un
      irréductible de degré $3$. Tous calculs fait, on obtient la liste
      suivante: $\{%
      X^{5}            +X^{2}  +1,%
      X^{5}      +X^{3}        +1,%
      X^{5}+X^{4}+X^{3}+X^{2}  +1,%
      X^{5}+X^{4}+X^{3}      +X+1,%
      X^{5}+X^{4}      +X^{2}+X+1,%
      X^{5}      +X^{3}+X^{2}+X+1,%
      \}$.}
\end{enumerate}
}

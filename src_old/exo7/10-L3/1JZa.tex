\uuid{1JZa}
\exo7id{7900}
\titre{exo7 7900}
\auteur{mourougane}
\organisation{exo7}
\datecreate{2021-08-11}
\isIndication{false}
\isCorrection{false}
\chapitre{Forme bilinéaire}
\sousChapitre{Forme bilinéaire}
\module{Algèbre et théorie des nombres}
\niveau{L3}
\difficulte{}

\contenu{
\texte{

}
\begin{enumerate}
    \item \question{Montrer le \emph{théorème de Burnside} : soit $G$ un groupe fini agissant sur un ensemble fini $E$.
Alors le nombre~$N$ d'orbites est la moyenne des cardinaux des points fixes des éléments de $G$
et aussi $$N=\frac{1}{|G|}\sum_{g\in G}\text{Card} Fix(\phi(g))=\frac{1}{|G|}\sum_{x\in E}\text{Card} stabl(x).$$
On pourra considérer $\{(x,g)\in E\times G / g\cdot x=x\}$.}
    \item \question{Soit $G$ un sous-groupe fini de $SO(3)$. On considère son action sur la sphère unité.
Soit $X$ l'ensemble des points fixés par un des éléments de $G$ différents de l'identité.
Montrer que $X$ est stable par l'action de $G$.
Montrer que le stabilisateur d'un élément de $X$ est un groupe cyclique.
On notera $N$ le nombre d'orbites de l'action de $G$ sur $X$
et $n_j$ le cardinal du stabilisateur d'un élément de l'orbite $O_j$.}
    \item \question{Montrer que 
$$N|G|=2(|G|-1)+\text{Card} X.$$}
    \item \question{Montrer que 
$$2-\frac{2}{|G|}=\sum_{j=1}^N(1-\frac{1}{n_j}).$$}
    \item \question{En déduire que $N=2$ ou $N=3$.}
    \item \question{Montrer que si $N=2$, $G$ est un sous-groupe cyclique de rotations.}
    \item \question{Si $N=3$, déterminer les possibilités pour les $n_j$.}
\end{enumerate}
}

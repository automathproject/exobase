\uuid{hr6Z}
\exo7id{6432}
\titre{exo7 6432}
\auteur{potyag}
\organisation{exo7}
\datecreate{2011-10-16}
\isIndication{false}
\isCorrection{false}
\chapitre{Géométrie et trigonométrie hyperbolique}
\sousChapitre{Géométrie et trigonométrie hyperbolique}
\module{Algèbre et géométrie}
\niveau{L3}
\difficulte{}

\contenu{
\texte{
Soit $\triangle abc$ un triangle dans $\mathbb{H}^2$ (c-à-d le
sous-ensemble de $\mathbb{H}^2$ bordé par trois géodésiques dont les
points de l'intersection sont $a, b, c$) d'angles intérieurs
$\alpha, \beta, \gamma$. On suppose que $\gamma=\frac{\pi}{2}$ ; en utilisant
les notations indiquées sur le Figure \ref{fig:pot2} démontrer les
identités suivantes :
}
\begin{enumerate}
    \item \question{$ \cosh c = \cosh a \cdot \cosh b$}
    \item \question{$\tanh b = \sinh a\cdot \tan\beta$}
    \item \question{$\cosh a\cdot\sin\beta = \cos\alpha$}
\end{enumerate}
}

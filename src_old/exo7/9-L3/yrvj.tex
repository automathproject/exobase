\uuid{yrvj}
\exo7id{6392}
\titre{exo7 6392}
\auteur{potyag}
\organisation{exo7}
\datecreate{2011-10-16}
\isIndication{false}
\isCorrection{false}
\chapitre{Isométrie euclidienne}
\sousChapitre{Isométrie euclidienne}
\module{Algèbre et géométrie}
\niveau{L3}
\difficulte{}

\contenu{
\texte{
Notons $l \subset \Rr^2$ une droite affine de $\Rr^2$.
}
\begin{enumerate}
    \item \question{Montrer que l'ensemble $I_l$  des $g \in Iso (\Rr^2)$ telles que $g(l)=l$
 est un sous-groupe de $Iso (\Rr)$.}
    \item \question{Déterminer les translations qui appartiennent à $I_l$.}
    \item \question{Montrer que si $g\in I_l$ possède un point fixe alors $g$ a
un point fixe sur $l$.}
    \item \question{Soit $g\in I_l$ , montrer qu'il existe une translation
$t$ de $I_l$ telle que $g .t$  possède  un point fixe.}
    \item \question{Décrire $I_l$.}
\end{enumerate}
}

\uuid{gZ5r}
\exo7id{2217}
\titre{exo7 2217}
\auteur{matos}
\organisation{exo7}
\datecreate{2008-04-23}
\isIndication{false}
\isCorrection{true}
\chapitre{Autre}
\sousChapitre{Autre}
\module{Analyse numérique}
\niveau{L3}
\difficulte{}

\contenu{
\texte{
\emph{D\'efinition}: Soit $\Sigma$ une matrice diagonale de type $(m\times n)$:
$$\Sigma =\left( \begin{array}{cccccc}
\mu_1&&&&&\\
&\ddots&&&&\\
&&\mu_r&&&\\
&&&0&&\\
&&&&\ddots &\\
&&&&&0\\
&&\bigcirc&&&
\end{array}\right)$$
On appelle pseudo--inverse de $\Sigma$ la matrice $\Sigma^\dagger$ de type $(n\times m)$ d\'efinie par
$$\Sigma^\dagger=\left(\begin{array}{cccc}
\mu_1^{-1}&&0&\\
&\ddots &&\bigcirc\\
0&&\mu_r^{-1}&
\end{array}\right)$$
Soit $A$ une matrice de type $(m\times n)$ dont la d\'ecomposition en valeurs singuli\`eres est $A=U\Sigma V^*$.

On appelle {\it pseudo-inverse } de la matrice $A$ la matrice $A^\dagger$ de type $(n\times m)$ d\'efinie par
$$A^\dagger = V\Sigma^\dagger U^* .$$
}
\begin{enumerate}
    \item \question{Quelle application repr\'esente la restriction de $\Sigma^\dagger \Sigma$ au sous-espace span$\{e_1, \cdots , e_r\}$ ?}
    \item \question{Montrer que si $A$ est carr\'ee r\'eguli\`ere alors $A^\dagger =A^{-1}$.}
    \item \question{Montrer que
$$A^\dagger =\sum_{i=1}^r \frac{1}{\mu_i}v_iu_i^*.$$}
    \item \question{Montrer que
\begin{itemize}}
    \item \question{$AA^\dagger $ est la matrice de la projection orthogonale sur Im$(A)$;}
    \item \question{$A^\dagger A $ est la matrice de la projection orthogonale sur Im$(A^*)$
\end{itemize}}
    \item \question{Montrer que la restriction \`a Im$(A^*)=$Ker$(A)^\bot$ de $A^*A$ est une matrice inversible et
$$(A^*A)^{-1} = \sum_{i=1}^r \mu_i^{-2} v_iv_i^* .$$}
\reponse{
$\Sigma^{\dagger}\Sigma e_i=e_i, i=1,\cdots, r$ c'est l'application identit\'e
$AA^{\dagger} =U\Sigma V^*V\Sigma^{\dagger} U =U\Sigma \Sigma^{\dagger}U^*=I$

On a donc obtenu une g\'en\'eralisation de l'inverse.
$U^*\sum_{i=1}^m \epsilon_iu_i^*$ avec$ \{\epsilon_1, \cdots , \epsilon_m\}$ base canonique de $\Rr^m$. Comme $\Sigma^{\dagger} \epsilon_i=0$ pour $r+1\leq i\leq m$ on a
$$\Sigma^{\dagger}U^*=\sum_{i=1}^r \mu_i^{-1}e_iu_i^* \Rightarrow A^{\dagger} =V\Sigma^{\dagger}U^* =\sum_{i=1}^r \mu_i^{-1}(Ve_i)u_i^* =\sum_{i=1}^r \mu_i^{-1}v_iu_i^*$$
On a
 $$AA^{\dagger} =\sum_{i=1}^r \mu_iu_iv_i^* \sum_{j=1}^r\mu_j^{-1}v_ju_j^*=\sum_{j=1}^r\mu_j\mu_j^{-1}u_ju_j^*$$
Comme Im($A$) =span$\{u_1,\cdots , u_r\}$ le r\'esultat suit.
soit $y\in$ Im$A^*$ $\Leftrightarrow u=\sum_{i=1}^r x_iv_i$. Alors
$$A^*A=V\Sigma^* U^*U\Sigma V^* =V\Sigma^*\Sigma V^*=\sum_{i=1}^r \mu_i^2 v_iv_i^*\Rightarrow A^*Ay=\sum_{i=1}^r \mu_i^2x_iv_i$$
et finalement
$$(\sum_{i=1}^r\mu_i^{-2}v_iv_i^*) (A^*Ay) =\sum_{i=1}^rx_iv_i$$
}
\end{enumerate}
}

\uuid{YqYZ}
\exo7id{2230}
\titre{exo7 2230}
\auteur{matos}
\organisation{exo7}
\datecreate{2008-04-23}
\isIndication{false}
\isCorrection{true}
\chapitre{Autre}
\sousChapitre{Autre}
\module{Analyse numérique}
\niveau{L3}
\difficulte{}

\contenu{
\texte{
Etant donn\'es $n$ vecteurs lin\'eairement ind\'ependants de $\Rr^m$, $\{a_1, \cdots , a_n\}$, on veut calculer une base orthonormale pour span$\{a_1, \cdots , a_n\}$.

 On pose $A=[ a_1, a_2, \cdots , a_n] \in \Rr^{m\times n}$ et on consid\`ere la factorisation QR de $A$,
$$ A=QR,\ \ \ Q=[q_1, \cdots , q_n] ,\ \ r_i^T, i=1, \cdots , n \mbox{ les lignes de }R$$
}
\begin{enumerate}
    \item \question{Montrer que 
$$\mbox{Im} A= \mbox{span} \{q_1, \cdots , q_n\} .$$}
\reponse{Pour $k=1,\cdots ,n\quad a_k=\sum_{i=1}^kr_{ik}q_i$ avec $r_{ik}=q_i^Ta_k$ par orthonormalit\'e des $q_i$.}
    \item \question{Montrer que
$$q_k=\frac{1}{r_{kk}} \left( a_k -\sum_{i=1}^{k-1}r_{ik}q_i\right) \ \ \ k=1, \cdots , n$$}
\reponse{D\'ecoule imm\'ediatement de la question pr\'ec\'edente.}
    \item \question{En d\'eduire un algorithme pour le calcul r\'ecursif des $q_i$ (algorithme de Gram--Schmidt).}
\reponse{Algorithme de Gram-Schmidt:

Pour $k=1, \cdots , n$ faire

\hspace{1cm} $r_{ik}=q_i^Ta_k\quad $ pour $i=1, \cdots , k-1$

\hspace{1cm} $z_k=a_k-\sum_{i=1}^{k-1} r_{ik}q_i$

\hspace{1cm} $r_{kk}=(z_k^Tz_k)^{1/2}$

\hspace{1cm}$q_k=z_k/r_{kk}$}
    \item \question{Algorithme de Gram--Schmidt modifi\'e

L'algorithme pr\'ec\'edent est instable num\'eriquement d\^u \`a la perte d'orthogonalit\'e dans le calcul des $q_i$. On va reformuler l'algorithme pour le rendre stable.

Pour $k=1, \cdots , n-1$, on d\'efinit $A^{(k)}\in\Rr^{m\times (n-k+1)}$ de la fa\c con suivante:
$$[0 , A^{(k)}] =A-\sum_{i=1}^{k-1} q_ir_i^T = \sum_{i=k}^n q_ir_i^T$$
et on va d\'ecrire l'\'etape $k$ de l'algorithme.
\begin{enumerate}}
\reponse{\begin{enumerate}}
    \item \question{Montrer que si on pose
$$ A^{(k)} =[z,B] ,\ \ \ z\in\Rr^m , \ \ \ B\in\Rr^{m\times (n-k)}$$
alors
$$r_{kk}=\|z\|_2, \ \ \ q_k=z/r_{kk}.$$}
\reponse{$$\sum_{i=k}^nq_ir_i^T =[q_k \cdots q_n]\left(\begin{array}{c}r_k^T\\ \vdots \\ r_n^T\end{array}\right) = q_k \cdots q_n]\left(\begin{array}{ccccccc} 
0&\cdots &0 &r_{kk} & \cdots &\cdots & r_{kn}\\
0&\cdots &0 & 0 & r_{k+1,k+1}&\cdots & r_{k+1,n}\\
\vdots & \vdots &\vdots &\vdots &\vdots &\vdots &\vdots \\
0&\cdots & \cdots&&&0&r_{nn}\end{array}\right)$$
$$A^{(k)}e_k = z= [q_k \cdots q_n]\left(\begin{array}{c}r_{kk}\\ 0\\ \vdots \\ 0\end{array}\right)=r_{kk}q_k \Rightarrow r_{kk}=\|z\|_2, q_k=z/r_{kk}$$}
    \item \question{Comment peut--on calculer la ligne $k$ de $R$  \`a partir de $A^{(k)}$?}
\reponse{$$q_k^TA^{(k)} =[q_k^Tz, q_k^TB]=[1 , 0, \cdots , 0]\left(\begin{array}{c}r_k^T\\ \vdots \\r_n^T\end{array}\right)=r_k^T$$
et donc
$$[r_{k,k+1}, \cdots , r_{kn}]=q_k^TB$$}
    \item \question{Calculer $A^{(k+1)}$.}
\reponse{$$[0,\cdots , 0, A^{(k+1)}] = \sum_{i=k+1}^n q_ir_i^T= [0,\cdots , 0, A^{(k)}] - q_kr_k^T= [0,\cdots , 0,A^{(k)} - q_k(r_{kk}, \cdots , r_{kn})]$$
$$[ 0,\cdots , 0, z-q_kr_{kk}, B-q_k(r_{k,k+1}, \cdots , r_{kn})] \Rightarrow A^{(k+1)}=B-q_k (r_{k,k+1}, \cdots , r_{kn})$$}
    \item \question{A partir des questions pr\'ec\'edentes, d\'ecrire l'algorithme qui permet le calcul de la factorisation $A=Q_1R_1$, $Q_1\in\Rr^{m \times n}$ orthonormale, $R_1\in\Rr^{n\times n}$ triangulaire sup\'erieure (Gram--Schmidt modifi\'e). Le calcul de $Q_1$ doit se faire sur place.}
\reponse{Donn\'ees: $A\in\Rr^{m\times n}, $ rank$(A)=n$

On calcule la factorisation $A=Q_1R_1$, $Q_1\in\Rr^{m\times n}$ orthonormale, $R_1\in\Rr^{n\times n}$ triangulaire sup\'erieure. Le calcul de $Q_1$ se fait sur place.

Pour $k=1, \cdots , n$

\hspace{1cm} $r_{kk}=\left(\sum_{i=1}^m a_{ik}^2\right)^{1/2} $

\hspace{1cm}pour $i=1, \cdots  ,m$

\hspace{1.5cm}$a_{ik}\leftarrow a_{ik}/r_{kk}$

\hspace{1cm} pour $j=k+1, \cdots , n$

\hspace{1.5cm}$r_{kj}\leftarrow\sum_{i=1}^m a_{ik}a_{ij}$

\hspace{1.5cm} pour $i=1, \cdots , m$

\hspace{2cm}  $a_{ij}\leftarrow a_{ij}-a_{ik}r_{kj}$}
    \item \question{Quelle est la complexit\'e de l'algorithme pr\'ec\'edent?}
\reponse{complexit\'e: $mn^2$ flops.}
\end{enumerate}
}

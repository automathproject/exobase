\uuid{aqd6}
\exo7id{2222}
\titre{exo7 2222}
\auteur{matos}
\organisation{exo7}
\datecreate{2008-04-23}
\isIndication{false}
\isCorrection{true}
\chapitre{Autre}
\sousChapitre{Autre}
\module{Analyse numérique}
\niveau{L3}
\difficulte{}

\contenu{
\texte{
Notons $\tilde{A}_k$ la matrice carr\'ee d'ordre $(n-k+1)$ form\'ee des \'el\'ements $a_{ij}^k , k\leq i,j \leq n$ de la matrice $A_k=(a_{ij}^k)$ obtenue come r\'esultat de la $(k-1)$--\`eme \'etape de l'\'elimination de Gauss. On suppose $A=A_1$ sym\'etrique d\'efinie positive.
}
\begin{enumerate}
    \item \question{Notant $(. , .)$ le produit scalaire euclidien et $v'\in\Rr^{n-k}$ le vecteur form\'e par les $(n-k)$ derni\`eres composantes d'un vecteur $v=(v_i)_{i=k}^n \in \Rr^{n-k+1}$ quelconque, \'etablir l'identit\'e 
$$ (\tilde{A}_k v,v)=(\tilde{A}_{k+1} v', v') + \frac{1}{a_{kk}^k} \left|a_{kk}^k v_k +\sum_{i=k+1}^n a_{ik}^k v_i\right|^2 .$$}
    \item \question{Montrer que chaque matrice $\tilde{A_k}$ est sym\'etrique d\'efinie positive.}
    \item \question{Etablir les in\'egalit\'es suivantes: 
$$ 0< a_{ii}^{k+1} \leq a_{ii}^k ,\ \ \ k+1\leq i \leq n$$ 
$$\max_{k+1\leq i \leq n} a_{ii}^{k+1} = \max_{k+1\leq i,j \leq n}\left|a_{ij}^{k+1}\right| \leq \max_{k \leq i,j \leq n}\left|a_{ij}^k\right| =\max_{k\leq i\leq n}a_{ii}^k$$}
\reponse{
A la k-\`eme \'etape de l'\'elimination de Gauss, l'\'el\'ement $a_{ij}^{k+1}$ est donn\'e par
$$a_{ij}^{k+1}=a_{ij}^k-\frac{a_{kj}^ka_{ik}^k}{a_{kk}^k} \quad k+1\leq i,j\leq n$$
et on remarque imm\'ediatement par r\'ecurrence que toutes les matrices $\tilde{A_k}$ sont sym\'etriques. On a 

$(\tilde{A}_{k+1} v',v')=\sum_{i=k+1}^n v_i (\sum_{j=k+1}^n a_{ij}^{(k)}v_j) - \frac{1}{a_{kk}^k}(\sum_{i=k+1}^n a_{ik}^k v_i)^2$

$(\tilde{A}_k v,v)= \sum_{i=k+1}^nv_i(\sum_{j=k+1}^n a_{ij}^kv_j) +\sum_{i=k+1}^n (a_{ik}^k +a_{ki}^k )v_iv_k +a_{kk}^kv_k^2$

Par sym\'etrie $a_{ik}^k=a_{ki}^k$ et donc

$(\tilde{A}_kv,v)=(\tilde{A}_{k+1}v',v') +\frac{1}{a_{kk}^k} [(\sum_{i=k+1}^n a_{ik}^k v_i)^2 + 2v_k\sum_{i=k+1}^n a_{ik}^kv_i a_{kk}^k +(a_{kk}^k)^2v_k^2]=$

$$(\tilde{A}_{k+1}v',v') +\frac{1}{a_{kk}^k} [a_kk^kv_k +\sum_{i=k+1}^na_{ik}^kv_i]^2$$
Faisons un raisonnement par r\'ecurrence
\begin{itemize}
$\tilde{A}_1$ est sym\'etrique d\'efinie positive;
Par hypoth\`ese supposons que $\tilde{A}_k$ est d\'efinie positive;
Supposons par absurde que $\tilde{A}_{k+1}$ ne soit pas d\'efinie positive: alors $\exists v'\neq 0:\quad (\tilde{A}_{k+1}v',v')\leq 0$. On d\'efinit le vecteur $v\in\Rr^{n-k+1}$ par:
\begin{itemize}
$v_i=v'_i,\quad k+1\leq i\leq n$
$v_k$ est solution de $a_{kk}^k +\sum_{i=k+1}^n a_{ik}^k v_i=0$
\end{itemize}
Alors $(\tilde{A}_kv,v)=0$ et $v\neq 0$; donc $\tilde{A}_k$ n'est pas d\'efinie positive, ce qui contredit l'hypoth\`ese de r\'ecurrence.
\end{itemize}
Premi\`ere in\'egalit\'e: en utilisant la relation d'\'elimination on obtient: $a_{ii}^{k+1}=a_{ii}^k -\frac{\left|a_{ki}^k\right|^2}{a_{kk}^2}$
\begin{itemize}
une matrice d\'efinie positive a tous ses \'el\'ements diagonaux strictement positifs, donc $a_{ii}^{k+1}>0$
$\left|a_{ki}^k\right|^2/\left|a_{kk}^k\right|^2 \geq 0, \quad k+1\leq i\leq n$
\end{itemize}
donc $a_{ii}^{k+1}\leq a_{ii}^k, k+1\geq i$

Deuxi\`eme in\'egalit\'e: supposons qu'il existe un \'el\'ement $a_{ij}^k, i<J$ tel que $\left|a_{ij}^k\right| \geq \max_{k\leq l \leq n} a_{ll}^k$. On consid\`ere le vecteur $v\neq 0$ d\'efini par
$$v_i=1, v_j=-\mbox{sign} (a_{ij}^k), v_l=0\quad l\neq i,j$$
Alors
$$(\tilde{A}_k v,v) =( a_{ii}^k -\left|a_{ij}^k\right|)-(\left|a_{ij}^k\right| -a_{jj}^k) \leq 0$$
ce qui est impossible. Donc
$$\max_{1\leq i,j\leq n}\left|a_{ij}^{k}\right|=\max_{1\leq i \leq n}\left|a_{ii}^k\right|$$
}
\end{enumerate}
}

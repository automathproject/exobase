\uuid{GVZp}
\exo7id{2530}
\titre{exo7 2530}
\auteur{tahani}
\organisation{exo7}
\datecreate{2009-04-01}
\isIndication{false}
\isCorrection{true}
\chapitre{Difféomorphisme, théorème d'inversion locale et des fonctions implicites}
\sousChapitre{Difféomorphisme, théorème d'inversion locale et des fonctions implicites}
\module{Calcul différentiel}
\niveau{L3}
\difficulte{}

\contenu{
\texte{
Soit $f:
\mathbb{R}^n \rightarrow \mathbb{R}^n$ une application $C^1$. On
suppose qu'il existe $\alpha >0$ tel que pour tout $h,x \in
\mathbb{R}^n$,
\[
\langle Df(x)(h),h \rangle \geq \alpha\langle h,h\rangle.
\]
}
\begin{enumerate}
    \item \question{En consid\'erant la fonction 
$t \rightarrow \varphi(t)= \langle f(a+t(b-a)),b_a) \rangle$, montrez que
$$ \langle f(b)-f(a),b-a \rangle \geq \alpha \langle b-a,b-a \rangle 
\text{ pour tout } a,b \in \mathbb{R}^n.$$ 
En déduire que $f$ est une application ferm\'ee.}
    \item \question{D\'emontrer que, pour tout $x \in E, Df(x)$ est un
isomorphisme de $\mathbb{R}^n$. En déduire que $f$ est une
application ouverte.}
    \item \question{Conclure que $f$ est un diff\'eomorphisme de classe $C^1$ de $\mathbb{R}^n$ sur lui même.}
\reponse{
$f$ et $\varphi$ sont de classe $C^1$ car compos\'ees d'applications de classe $C^1$. On a
$$
D\varphi(t)= \varphi'(t)=(\Psi \circ f \circ \theta)(t)'(t)
= D\Psi(f(\theta(t)))\circ Df(\theta(t))\circ D\theta(t)
= \langle Df(a+t(b-a))(b-a),b-a \rangle.
$$ 
Par cons\'equent:
$$
\varphi'(t) \geq \alpha \langle b-a,b-a \rangle.
$$
Or, $\varphi(1)-\varphi(0)=\langle f(b)-f(a),(b-a)\rangle$ et il existe $t\in
]0,1[$ tel que $\varphi(1)-\varphi(0)=\varphi'(t)$ d'o\`u
$$
\langle f(b)-f(a),b-a\rangle \geq \alpha \langle b-a,b-a\rangle.
$$
Indications pour mq $f$ est ferm\'ee: Posons $\|x\|=\sqrt{\langle x,x\rangle}$
alors
$$
\alpha \|b-a\|^2 \leq \langle f(b)-f(a),b-a\rangle 
\leq \|f(b)-f(a)\|.\|b-a\|.
$$ 
D'o\`u $$\|b-a\| \leq 1/\alpha \|f(b)-f(a)\|.
$$ 
Soit $F$ un ferm\'e et $y_n$ une suite de points
de $f(F)$ convergeant vers un point limite $y_\infty$. Il faut
montrer que $y_\infty \in f(F)$. Soit $x_n$ une suite de points de
$\mathbb{R}^n$ tels que $f(x_n)=y_n$. Il reste \`a montrer que
cette suite admet est de Cauchy, qu'elle converge donc et que sa
limite $x_\infty$ v\'erifie $f(x_\infty)=y_\infty$.
}
\end{enumerate}
}

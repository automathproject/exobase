\uuid{Shkr}
\exo7id{2546}
\titre{exo7 2546}
\auteur{tahani}
\organisation{exo7}
\datecreate{2009-04-01}
\isIndication{false}
\isCorrection{false}
\chapitre{Difféomorphisme, théorème d'inversion locale et des fonctions implicites}
\sousChapitre{Difféomorphisme, théorème d'inversion locale et des fonctions implicites}
\module{Calcul différentiel}
\niveau{L3}
\difficulte{}

\contenu{
\texte{
Montrer que l'\'equation
$e^x+e^y+x+y-2=0$ d\'efinit, au voisinage de l'origine, une
fonction implicite $\varphi$ de $x$ dont on calculera le
d\'eveloppement limit\'e d'ordre trois en $0$.
}
}

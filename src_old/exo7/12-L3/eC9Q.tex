\uuid{eC9Q}
\exo7id{1825}
\titre{exo7 1825}
\auteur{drutu}
\organisation{exo7}
\datecreate{2003-10-01}
\isIndication{false}
\isCorrection{false}
\chapitre{Extremum, extremum lié}
\sousChapitre{Extremum, extremum lié}
\module{Calcul différentiel}
\niveau{L3}
\difficulte{}

\contenu{
\texte{
Pour chacune des fonctions suivantes etudiez la nature du point
critique donn\'e :
}
\begin{enumerate}
    \item \question{$f(x,y)=x^2-xy+y^2$ au point critique $(0,0)$ ;}
    \item \question{$f(x,y)=x^2+2xy+y^2+6$ au point critique $(0,0)$ ;}
    \item \question{$f(x,y,z)=x^2+y^2+2z^2+xyz$ au point critique $(0,0,0)$ ;}
    \item \question{$f(x,y)=x^3+2xy^2-y^4+x^2+3xy+y^2+10$ au point critique
$(0,0)$.}
\end{enumerate}
}

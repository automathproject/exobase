\uuid{UQ72}
\exo7id{5978}
\titre{exo7 5978}
\auteur{tumpach}
\organisation{exo7}
\datecreate{2010-11-11}
\isIndication{false}
\isCorrection{true}
\chapitre{Transformée de Fourier}
\sousChapitre{Transformée de Fourier}
\module{Théorie de la mesure, intégrale de Lebesgue}
\niveau{L3}
\difficulte{}

\contenu{
\texte{
Soit $f~:\mathbb{R}^{3}\rightarrow \mathbb{R}$ une fonction
radiale, i.e. telle que $f(x)= h({r})$ o\`u $x=(x_1,x_2,x_3)$ et $r = |x|$ et
$h~:\mathbb{R}^{+}\rightarrow \mathbb{R}$. Montrer que la
transform\'ee de Fourier $\hat{f}$ de $f$ s'\'ecrit~: $$
\hat{f}(k) = \frac{2}{|k|}\int_{0}^{+\infty} h(r) r
\sin(2\pi|k|r)\,dr.
$$
}
\reponse{
A l'aide des coordonnées sphériques, on a
$$
\begin{array}{lcl}
\hat{f}(k) &= & \int_{\mathbb{R}^{3}} f(x) e^{-2\pi i(x, k)}\,dx\\
& = & \\ & = & \int_{r =0}^{+\infty}\int_{\theta
=0}^{\pi}\int_{\varphi = 0}^{2\pi} h(r) e^{-2\pi i r
|k|\cos\theta} r^2 \sin\theta\,d\theta
dr d\varphi\\& = & \\
& = & 2\pi \int_{0}^{+\infty}\int_{0}^{\pi} h(r)
\frac{d}{d\theta}\left(\frac{1}{2\pi i r |k|} e^{-2\pi i r |k|\cos
\theta} \right) r^2 \,d\theta dr\\& = & \\
& = &\frac{1}{|k|} \int_{0}^{\infty} h(r) r
\frac{1}{i}\left[e^{+2\pi i r|k|} - e^{-2\pi i r|k|}\right]\, dr\\& = & \\
 & = &\frac{2}{|k|}\int_{0}^{+\infty} h(r) r \sin(2\pi|k|r)\,dr.
\end{array}
$$
}
}

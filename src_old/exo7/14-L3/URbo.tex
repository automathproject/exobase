\uuid{URbo}
\exo7id{5974}
\titre{exo7 5974}
\auteur{tumpach}
\organisation{exo7}
\datecreate{2010-11-11}
\isIndication{false}
\isCorrection{true}
\chapitre{Autre}
\sousChapitre{Autre}
\module{Théorie de la mesure, intégrale de Lebesgue}
\niveau{L3}
\difficulte{}

\contenu{
\texte{
Soit $g$ une fonction sur $\mathbb{R}^+$ et
$f~:\mathbb{R}^3\rightarrow \mathbb{R}$ telle que $f(x) = g(|x|)$,
o\`u $|x|$ d\'esigne la norme de $x$ dans $\mathbb{R}^3$.
}
\begin{enumerate}
    \item \question{Montrer que pour $r = |x|$, on a
$$
\int_{\mathbb{R}^3} \frac{f(y)}{|x - y|}\,dy = 4\pi
\frac{1}{r}\int_{0}^{r} g(s) s^2\,ds + 4\pi\int_{r}^{+\infty} g(s)
s\,ds.
$$}
\reponse{Posons
$$
I = \int_{\mathbb{R}^3} \frac{g(|y|)}{|x - y|}\,dy,
$$
et $r = |x|$, $s = |y|$. Alors $|x - y| = \sqrt{r^2 + s^2 -2 rs
\cos\theta}$ o\`u $\theta$ est l'angle entre l'axe $(Ox)$ et l'axe
$(Oy)$. On consid\'ere les coordonn\'ees sph\'eriques de centre
$O$ et d'axe $(Ox)$ suivantes~:
$$
\begin{array}{lcl}
y_1 &=& s \cos\theta\\
y_2 & = & s \sin\theta \cos\varphi\\
y_3 & = & s\sin\theta \sin\varphi
\end{array}
$$
On a~
$$
I =
\int_{s=0}^{+\infty}\int_{\theta=0}^{\pi}\int_{\varphi=0}^{2\pi}
\frac{g(s)}{\sqrt{r^2 + s^2 - 2rs\cos\theta}} s^2 \sin\theta\,ds
d\theta d\varphi.
$$
On note que
$$
\frac{\sin\theta}{\sqrt{r^2 + s^2 - 2rs\cos\theta}} =
\frac{d}{d\theta}\frac{1}{rs} \sqrt{r^2 + s^2 - 2rs\cos\theta}.
$$
Ainsi~:
$$
\begin{array}{lcl}
I &=& 2\pi \int_{s=0}^{+\infty}\left[\frac{1}{rs} \sqrt{r^2 + s^2
-
2rs\cos\theta} \right]_{\theta = 0}^{\pi} g(s)s^2 \,ds\\
& = & 2\pi \int_{s=0}^{+\infty} \frac{1}{rs} \left(\sqrt{(r+s)^2}
- \sqrt{(r-s)^2} \right) g(s) s^2 \,ds.
\end{array}
$$
Lorsque $s\leq r$, on a $$\sqrt{(r+s)^2} - \sqrt{(r-s)^2} =
(r+s)-(r-s) = 2s,$$ et lorsque $s>r$, il vient $$\sqrt{(r+s)^2} -
\sqrt{(r-s)^2} = (r+s)-(s-r) = 2r.$$ On en d\'eduit alors~:
$$
I = \frac{4\pi}{r}\int_{0}^{r} g(s) s^2\,ds +
4\pi\int_{r}^{+\infty} g(s) s\,ds.
$$}
    \item \question{Que peut-on en d\'eduire pour une distribution de masse
$f(x) = g(|x|)$ lorsque $g$ est \`a support dans $[0, R]$ ?}
\reponse{Lorsque $g$ est \`a
support dans $[0, R]$, le potentiel newtonien cr\'e\'e  par la
distribution de masse $f(y) = g(|y|)$ en un point
$x\in\mathbb{R}^3$ tel que $|x|>R$, est identique au potentiel
cr\'e\'e par une masse totale \'egale concentr\'ee \`a l'origine.}
\end{enumerate}
}

\uuid{R8dG}
\exo7id{5934}
\titre{exo7 5934}
\auteur{tumpach}
\organisation{exo7}
\datecreate{2010-11-11}
\isIndication{false}
\isCorrection{true}
\chapitre{Tribu, fonction mesurable}
\sousChapitre{Tribu, fonction mesurable}
\module{Théorie de la mesure, intégrale de Lebesgue}
\niveau{L3}
\difficulte{}

\contenu{
\texte{
Soit $(\Omega, \Sigma, \mu)$ un espace mesur\'e et $f~:\Omega
\rightarrow \mathbb{R}$ une fonction
($\Sigma$-$\mathcal{B}(\mathbb{R})$)-mesurable. Montrer que la
troncature $f_A$ de $f$ d\'efinie par~:
$$
f_A(x) = \left\{\begin{array}{ll}-A & \text{si}\quad f(x) < -A\\
                                  f(x) & \text{si}\quad |f(x)| \leq A\\
                                  A & \text{si}\quad f(x)> A
\end{array} \right.
$$
est ($\Sigma$-$\mathcal{B}(\mathbb{R})$)-mesurable.
}
\reponse{
Soit $(\Omega, \Sigma, \mu)$ un espace mesur\'e et $f~:\Omega
\rightarrow \mathbb{R}$ une fonction
($\Sigma$-$\mathcal{B}(\mathbb{R})$)-mesurable. Montrons que la
troncature $f_A$ de $f$ d\'efinie par~:
$$
f_A(x) = \left\{\begin{array}{ll}-A & \text{si}\quad f(x) < -A\\
                                  f(x) & \text{si}\quad |f(x)| \leq A\\
                                  A & \text{si}\quad f(x)> A
\end{array} \right.
$$
est mesurable. Notons
$$
\begin{array}{l}
E_{1} := \{x\in\Omega~|~f(x)< -A\}= f^{-1}\left(]-\infty,-A[\right),\\
E_{2}:= \{x\in\Omega~|~|f(x)| \leq A\} = f^{-1}\left([-A, A]\right),\\
E_{3}:=\{x\in\Omega~|~f(x)>A\} = f^{-1}\left(]A, +\infty[\right).
\end{array}
$$
Comme $]-\infty, -A[$, $[-A, A]$, $]A, +\infty[$ appartiennent \`a
la tribu bor\'elienne et $f$ est
($\Sigma$-$\mathcal{B}(\mathbb{R})$)-mesurable, les ensembles
$E_{1}$, $E_2$, et $E_3$ appartiennent \`a $\Sigma$. Alors $f_{A}
= f\cdot \mathbf{1}_{E_2} -A \cdot \mathbf{1}_{E_1} + A \cdot\mathbf{1}_{E_{3}}$ est
mesurable comme somme de produits de fonctions mesurables.
}
}

\uuid{3F9s}
\exo7id{2153}
\titre{exo7 2153}
\auteur{debes}
\organisation{exo7}
\datecreate{2008-02-12}
\isIndication{false}
\isCorrection{true}
\chapitre{Sous-groupe distingué}
\sousChapitre{Sous-groupe distingué}
\module{Théorie des groupes}
\niveau{L3}
\difficulte{}

\contenu{
\texte{
\label{ex:le18}
Soit $G$ un groupe et $H$ un sous groupe distingu\'e de $G$ d'indice
$n$. Montrer que pour tout $a\in G$, $a^n \in H$. Donner un exemple de sous-groupe $H$
non distingu\'e de $G$ pour lequel la conclusion pr\'ec\'edente est fausse.
}
\reponse{
On a $n=|G/H|$. Pour toute classe $aH\in G/H$, on a donc $(aH)^n=H$ c'est-\`a-dire,
$a^nH=H$ ou encore $a^n\in H$. Cela devient faux si $H$ n'est pas distingu\'e dans $G$. Par
exemple le sous-groupe $H$ de $S_3$ engendr\'e par la transposition $(1\hskip 2pt 2)$ est
d'indice $3$ dans $S_3$ et, pour $a= (2\hskip 2pt 3)$, on a $a^3=a\notin H$.
}
}

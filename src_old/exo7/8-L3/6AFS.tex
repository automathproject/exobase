\uuid{6AFS}
\exo7id{2116}
\titre{exo7 2116}
\auteur{debes}
\organisation{exo7}
\datecreate{2008-02-12}
\isIndication{true}
\isCorrection{false}
\chapitre{Ordre d'un élément}
\sousChapitre{Ordre d'un élément}
\module{Théorie des groupes}
\niveau{L3}
\difficulte{}

\contenu{
\texte{
Soit $G$ un groupe d'ordre impair. Montrer que l'application $f$ de $G$ sur lui-m\^eme
donn\'ee par $f(x)=x^{2}$ est une bijection. En d\'eduire que l'\'equation $x^2=e$ a une
unique solution, \`a savoir $x=e$.
}
\indication{On commence par montrer que $f$ est surjective, en notant que si $|G|=2m+1$, alors pour
tout $y\in G$ on a $y=(y^{m+1})^2$.}
}

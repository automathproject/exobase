\uuid{iwjU}
\exo7id{2172}
\titre{exo7 2172}
\auteur{debes}
\organisation{exo7}
\datecreate{2008-02-12}
\isIndication{false}
\isCorrection{true}
\chapitre{Action de groupe}
\sousChapitre{Action de groupe}
\module{Théorie des groupes}
\niveau{L3}
\difficulte{}

\contenu{
\texte{
Dans le groupe sym\'etrique $S_4$ on consid\`ere les
sous-ensembles suivants :
$$H = \{ \sigma \in S_4 \hskip 2pt |\hskip 2pt  \sigma ( \{ 1,2\} ) = \{ 1,2\} \} $$
$$K= \{ \sigma \in S_4 \hskip 2pt |\hskip 2pt  \forall a,b \quad a\equiv b\ [\hbox{\rm mod}\
2] \Rightarrow
\sigma (a) \equiv \sigma (b)\ [\hbox{\rm mod}\ 2] \} $$
Montrer que $H$ et $K$ sont des sous-groupes de $S_4$. Les d\'ecrire.
}
\reponse{
L'ensemble $H$ est le sous-groupe de $S_4$ fixant la paire $\{1,2\}$. Tout \'el\'ement de $H$
fixe aussi la paire $\{3,4\}$. Cela fournit un morphisme $H\rightarrow S_2 \times S_2$ qui est
clairement bijectif. D'o\`u $H\simeq S_2 \times S_2 \simeq \Z/2\Z \times \Z/2\Z$.
\medskip

\hskip 5mm On a $\sigma\in K$ si et seulement si $\sigma(1) \equiv \sigma(3)\ [\hbox{\rm mod}\ 2]$ et
$\sigma(2) \equiv \sigma(4)\ [\hbox{\rm mod}\ 2]$, c'est-\`a-dire si et seulement si
$\sigma(\{1,3\})$ est soit la paire $\{1,3\}$ soit la paire $\{2,4\}$ (auquel cas
$\sigma(\{2,4\})$ est la paire $\{2,4\}$ ou la paire $\{1,3\}$ respectivement).
Gr\^ace \`a l'identit\'e $\sigma \hskip 2pt(1\hskip 2pt 3)\hskip 2pt(2\hskip 2pt 4)\hskip
2pt \sigma^{-1} = (\sigma(1)\hskip 2pt \sigma(3))\hskip 2pt(\sigma(2)\hskip 2pt
\sigma(4))$, on voit que la condition est \'egalement \'equivalente au fait que la
conjugaison par $\sigma$ stabilise la permutation $(1\hskip 2pt 3)\hskip 2pt(2\hskip 2pt 4)$.
Autrement dit $K$ est le sous-groupe des \'el\'ements de $S_4$ commutant avec $(1\hskip 2pt
3)\hskip 2pt(2\hskip 2pt 4)$. La classe de conjugaison 2-2 ayant $3$ \'el\'ements, le
groupe $H$ est d'ordre $4!/3=8$. On peut dresser la liste de ses \'el\'ements: si $\omega =
(1\hskip 2pt 2\hskip 2pt 3\hskip 2pt 4)$ et $\tau = (1\hskip 2pt 2)\hskip 2pt (3\hskip 2pt
4)$, alors
$K=\{1,\omega,\omega^2,\omega^3,\tau,\omega \tau, \omega^2 \tau, \omega^3\tau\}$. On
v\'erifie les relations $\sigma^4=1$, $\tau^2=1$ et $\tau \sigma \tau^{-1}= \sigma^{-1}$. Le
groupe $K$ est \'egal au produit semi-direct de son sous-groupe distingu\'e
$<\omega>$ par son sous-groupe $<\tau>$ et est donc isomorphe au groupe di\'edral 
d'ordre $8$.
}
}

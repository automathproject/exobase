\uuid{qSlN}
\exo7id{2178}
\titre{exo7 2178}
\auteur{debes}
\organisation{exo7}
\datecreate{2008-02-12}
\isIndication{true}
\isCorrection{true}
\chapitre{Action de groupe}
\sousChapitre{Action de groupe}
\module{Théorie des groupes}
\niveau{L3}
\difficulte{}

\contenu{
\texte{
Montrer que pour $m\geq 3$, un groupe simple d'ordre $\geq m!$ ne peut avoir de sous-groupe d'indice $m$.
}
\indication{Etudier l'action du groupe par translation sur l'ensemble quotient des classes modulo le sous-groupe.}
\reponse{
Soit $H$ un sous-groupe d'indice $m$ d'un groupe $G$.  L'action de $G$ par translation \`a gauche sur l'ensemble quotient $G/H$ des classes \`a gauche modulo $H$ induit un morphisme $G\rightarrow \textrm{Per}(G/H)$ qui est non-trivial et donc est injectif puisque le noyau, distingu\'e dans $G$, ne peut \^etre trivial si $G$ est simple. L'ordre de $G$ doit donc diviser l'ordre du groupe $\textrm{Per}(G/H)$ qui vaut $m!\ $. Il faut n\'ecessairement que $|G|=m!\ $. Mais alors le morphisme pr\'ec\'edent est un isomorphisme et $G$ est isomorphe au groupe sym\'etrique $S_m$, ce qui contredit la simplicit\'e de $G$.
}
}

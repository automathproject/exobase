\uuid{WAod}
\exo7id{2192}
\titre{exo7 2192}
\auteur{debes}
\organisation{exo7}
\datecreate{2008-02-12}
\isIndication{true}
\isCorrection{true}
\chapitre{Théorème de Sylow}
\sousChapitre{Théorème de Sylow}
\module{Théorie des groupes}
\niveau{L3}
\difficulte{}

\contenu{
\texte{
Soit $G$ un $p$-groupe d'ordre $p^r$. 
\smallskip

(a) Montrer que pour tout entier $k\leq r$, $G$ poss\`ede un sous-groupe
distingu\'e d'ordre $p^k$. 
\smallskip

(b) Montrer qu'il existe une suite $G_0=\{1\} \subset G_1 \subset \ldots \subset
G_r=G$ de sous-groupes $G_i$ distingu\'es d'ordre $p^i$ ($i=1,\ldots,r$).
\smallskip

(c) Montrer que pour tout sous-groupe $H$ de $G$ d'ordre $p^s$ avec $s<r$, il
existe un sous-groupe d'ordre $p^{s+1}$ de $G$ qui contient $H$.
}
\indication{Pour les trois \'enonc\'es (a), (b) et (c), raisonner par r\'ecurrence sur $r$    
en utilisant le fait que le centre d'un $p$-groupe n'est pas trivial.}
\reponse{
(a) Soit $G$ un $p$-groupe d'ordre $p^r$. Son centre $Z(G)$ est un $p$-groupe    
non trivial. Soit $x\in Z(G)\setminus\{1\}$. Si $p^\nu>0$ est son ordre, alors
$x^{p^{\nu-1}}$ est d'ordre $p$ et dans $Z(G)$; on peut donc supposer que $x$
lui-m\^eme est d'ordre $p$. Le groupe $<x>$ est distingu\'e dans
$G$ et le groupe quotient $G/<x>$ est d'ordre $p^{r-1}$. Par hypoth\`ese
de r\'ecurrence, pour tout $k\leq r$, le groupe $G/<x>$ poss\`ede un sous-groupe
distingu\'e ${\cal H}$ d'ordre $p^{k-1}$. Soit $H$ le sous-groupe image r\'eciproque
de ${\cal H}$ par la surjection canonique $G\rightarrow G/<x>$. Le sous-groupe $H$,
image r\'eciproque par un morphisme d'un sous-groupe distingu\'e, est
distingu\'e dans $G$ et ${\cal H}=H/<x>$, ce qui donne $|H| = |{\cal H}| \hskip 2pt
|<x>| = p^k$.
}
}

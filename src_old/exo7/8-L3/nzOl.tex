\uuid{nzOl}
\exo7id{7873}
\titre{exo7 7873}
\auteur{mourougane}
\organisation{exo7}
\datecreate{2021-08-11}
\isIndication{false}
\isCorrection{false}
\chapitre{Groupe symétrique, décomposition en cycles disjoints, signature}
\sousChapitre{Groupe symétrique, décomposition en cycles disjoints, signature}
\module{Théorie des groupes}
\niveau{L3}
\difficulte{}

\contenu{
\texte{
Soint $n$ un entier naturel supérieur à $3$.
}
\begin{enumerate}
    \item \question{Montrer que les permutations $(i,j)(j,k)$ et $(i,j)(k,l)$ s'écrivent comme produit de $3$-cycles.}
    \item \question{En déduire que le groupe alterné $\mathcal{A}_n$ est engendré par les $3$-cycles.}
    \item \question{Montrer que si $n\geq 5$, tous les $3$-cycles sont conjugués dans $\mathcal{A}_n$.}
\end{enumerate}
}

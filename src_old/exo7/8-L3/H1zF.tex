\uuid{H1zF}
\exo7id{7786}
\titre{exo7 7786}
\auteur{mourougane}
\organisation{exo7}
\datecreate{2021-08-11}
\isIndication{false}
\isCorrection{false}
\chapitre{Sous-groupe distingué}
\sousChapitre{Sous-groupe distingué}
\module{Théorie des groupes}
\niveau{L3}
\difficulte{}

\contenu{
\texte{

}
\begin{enumerate}
    \item \question{On rappelle que le centre d'un $p$-groupe n'est jamais réduit à $\{e\}$. Montrer par récurrence qu'un $p$-groupe est toujours résoluble.}
    \item \question{Soient $p,q$ deux nombres premiers distincts. Montrer qu'un groupe de cardinal $pq$ est toujours résoluble. (Supposer $p>q$ et considérer un $p$-Sylow)}
    \item \question{Soit $G$ un groupe d'ordre $12$. Montrez que $G$ est résoluble (En supposant les $3$-Sylow non distingués, comptez le nombre d'éléments d'ordre $3$).}
    \item \question{Soient $p,q$ deux nombres premiers distincts. Montrez qu'un groupe de cardinal $p^2q$ est toujours résoluble.}
\end{enumerate}
}

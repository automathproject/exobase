\uuid{Ex9R}
\exo7id{2199}
\titre{exo7 2199}
\auteur{debes}
\organisation{exo7}
\datecreate{2008-02-12}
\isIndication{false}
\isCorrection{true}
\chapitre{Théorème de Sylow}
\sousChapitre{Théorème de Sylow}
\module{Théorie des groupes}
\niveau{L3}
\difficulte{}

\contenu{
\texte{
Montrer qu'un groupe d'ordre $200$ n'est pas simple.
}
\reponse{
D'apr\`es les th\'eor\`emes de Sylow, le nombre de $5$-Sylow d'un groupe d'ordre
$200=5^2.2^3$ est $\equiv 1\ [\hbox{\rm mod}\ 5]$ et divise $8$. Ce ne peut \^etre
que $1$. L'unique $5$-Sylow est n\'ecessairement distingu\'e puisque ses 
conjugu\'es sont des $5$-Sylow et coincident donc avec lui. Le groupe ne peut pas
\^etre simple.
}
}

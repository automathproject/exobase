\uuid{D0xl}
\exo7id{7694}
\titre{exo7 7694}
\auteur{mourougane}
\organisation{exo7}
\datecreate{2021-08-11}
\isIndication{false}
\isCorrection{false}
\chapitre{Sous-variété}
\sousChapitre{Sous-variété}
\module{Géométrie différentielle}
\niveau{L3}
\difficulte{}

\contenu{
\texte{
On considère dans $\Rr^3$ euclidien le cylindre $\mathcal{C}$ d'équation $x^2+y^2=1$
muni de la métrique riemannienne restriction du produit scalaire de $\Rr^3$.
}
\begin{enumerate}
    \item \question{Montrer que $\mathcal{C}$ est une surface régulière.}
    \item \question{Montrer que l'application $F: ]-\pi,\pi[\times\Rr\to \Rr^3$, $(\theta,h)\mapsto (\cos\theta,\sin\theta,h)$
donne un paramétrage de $\mathcal{C}$ au voisinage du point $p$ de coordonnées $(1,0,0)$.}
    \item \question{Calculer la matrice $G(u)$ de la première forme fondamentale $I$
dans la base $$(X_\theta(\theta,h):=\frac{\partial F}{\partial \theta}(\theta,h),
X_h:=\frac{\partial F}{\partial h}(\theta,h))$$ de $T_{F(\theta,h)}\mathcal{C}$
correspondant à ce paramétrage $F$.}
    \item \question{Déterminer un champs de vecteurs normaux unitaires $N(\theta,h)$ au point $F(\theta,h)$.}
    \item \question{Calculer les symboles de Christoffel de la base 
$(X_\theta(\theta,h),X_h(\theta,h))$ de $T_{F(\theta,h)}\mathcal{C}$.}
    \item \question{Soit $a$ un nombre réel fixé et $c :\Rr\to S$, $t\mapsto (\cos t,\sin t,at)$
la courbe paramétrée tracée sur le cylindre $\mathcal{C}$.
Exprimer le vecteur vitesse au point de paramètre $t$ dans la base 
$(X_\theta(u), X_h)$ de $T_{F(\theta,h)}\mathcal{C})$.}
    \item \question{Les courbes paramétrées précédentes sont-elles des géodésiques ?}
    \item \question{Les sections planes du cylindre paramétrées par la longueur d'arc
sont-elles des géodésiques ?}
\end{enumerate}
}

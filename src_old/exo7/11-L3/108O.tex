\uuid{108O}
\exo7id{7649}
\titre{exo7 7649}
\auteur{mourougane}
\organisation{exo7}
\datecreate{2021-08-11}
\isIndication{false}
\isCorrection{false}
\chapitre{Sous-variété}
\sousChapitre{Sous-variété}
\module{Géométrie différentielle}
\niveau{L3}
\difficulte{}

\contenu{
\texte{

}
\begin{enumerate}
    \item \question{On rappelle qu'une ligne polygonale $P$ de $\Rr^n$ est la donnée d'un uplet $P=(a_0,a_1,\cdots,a_k)$
de points de $\Rr^n$. On supposera aussi que deux points consécutifs sont distincts.
Rappeler la formule pour la longueur d'une ligne polygonale $P$.}
    \item \question{On cherche à montrer le théorème suivant.
 
\textbf{Théorème.}
\emph{
Soit $c~:~[a,b]\to \Rr^n$ une courbe paramétrée.
Alors, pour tout $\epsilon>0$, il existe $\delta>$ tel que 
pour toute partition $(t_0=a<t_1<t_2<\cdots<t_m=b)$ de $[a,b]$ de pas inférieur à $\delta$,
$$L[P]\leq L[c]\leq L[P]+\epsilon$$
où $P=(c(t_0),c(t_1),\cdots,c(t_m))$ est la ligne polygonale simplement inscrite dans la courbe $c$
associée à la partition $(t_0=a<t_1<t_2<\cdots<t_m=b)$ de $[a,b]$.
}

\begin{enumerate}}
    \item \question{Écrire en termes de quantificateurs, en partant d'un $\epsilon_1>0$,
le théorème des sommes de Riemann pour la fonction $[a,b]\to\Rr$, 
$t\mapsto \| \dot{c} (t)\|$.}
    \item \question{\'Ecrire en termes de quantificateurs en partant d'un $\epsilon_2>0$
la propriété de continuité uniforme des fonctions composantes $[a,b]\to\Rr$, 
$t\mapsto \dot{c}_j (t)$.}
    \item \question{Soit une partition $(t_0=a<t_1<t_2<\cdots<t_m=b)$ de $[a,b]$.
\'Ecrire en termes de quantificateurs le théorème des accroissements finis
pour la fonction $[t_i,t_{i+1}]\to\Rr$, $t\mapsto {c}_j (t)$.}
\end{enumerate}
}

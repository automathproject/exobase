\uuid{NIq6}
\exo7id{6923}
\titre{exo7 6923}
\auteur{ruette}
\organisation{exo7}
\datecreate{2013-01-24}
\isIndication{false}
\isCorrection{true}
\chapitre{Probabilité continue}
\sousChapitre{Loi normale}
\module{Probabilité et statistique}
\niveau{L2}
\difficulte{}

\contenu{
\texte{
Une machine est conçue pour confectionner 
des paquets d'un poids de 500g, mais ils n'ont pas exactement tous le même poids. 
On a constaté que la distribution des poids autour de la valeur moyenne de 500g avait un écart-type de 25g.
}
\begin{enumerate}
    \item \question{Par quelle loi est-il raisonnable de modéliser le poids des paquets ?}
\reponse{L'énoncé suggère que le poids en grammes des paquets est une variable aléatoire 
qui suit une loi normale d'espérance 500 et d'écart-type 25. Soit $X$ la variable aléatoire correspondante, et $Y=(X-500)/25$.}
    \item \question{Sur 1000 paquets, quel est le nombre moyen de paquets pesant entre 480g et 520g ? 
(\textit{utiliser la table  de $\mathcal{N}(0,1)$})}
\reponse{$P(480\leq X\leq 520)=P(|Y|\leq  0,8)=0,576.$
On s'attend donc à ce que, sur 1000 paquets, il y en ait 576 dont le poids est compris entre 480g et 520g.}
    \item \question{Combien de paquets pèsent entre 480g et 490g ?}
\reponse{\begin{eqnarray*}
P(480\leq X\leq 490)&=&P(-0,8\leq Y\leq -0,4)\\
&=&P(0,4\leq Y\leq 0,8)\\
&=&p(Y\leq 0,8)-p(Y\leq 0,4)=0,1327.
\end{eqnarray*}
On s'attend donc à ce que, sur 1000 paquets, il y en ait 132 dont le poids est compris entre 480g et 490g.}
    \item \question{Sur 1000 paquets, quel est le nombre moyen de paquets pesant plus de 450g ?}
\reponse{$P(450\leq X)=0,5+P(0\le Y\leq 2)=0,5+\frac 12 P(-2\le Y\leq 2)=0,5+0,4772=0,9772.$
On s'attend donc à ce que, sur 1000 paquets, il y en ait 977 dont le poids est supérieur à 450g.}
    \item \question{Trouver $a$ tel que les 9/10 de cette production aient un poids compris
entre $500-a$ et $500+a$.}
\reponse{Il faut trouver $t$ tel que $p(|Y|<t)=0,9$. La table donne $t=1,645$, 
puis $a=25t=41$. Par conséquent, environ 90\% de la production a un poids 
compris entre $500-41=459$g et $500+41=541$g.}
\end{enumerate}
}

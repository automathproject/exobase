\uuid{UKeU}
\exo7id{5992}
\titre{exo7 5992}
\auteur{quinio}
\organisation{exo7}
\datecreate{2011-05-11}
\isIndication{false}
\isCorrection{true}
\chapitre{Probabilité discrète}
\sousChapitre{Probabilité conditionnelle}
\module{Probabilité et statistique}
\niveau{L2}
\difficulte{}

\contenu{
\texte{
Dans la salle des profs $60$\% sont des femmes ; une femme sur trois
porte des lunettes et un homme sur deux porte des lunettes : quelle est la
probabilité pour qu'un porteur de lunettes pris au hasard soit une femme?
}
\reponse{
Notons les différents événements :
$Fe$ : <<être femme>>, $Lu$ : <<porter des lunettes>>, $H$ : <<être homme>>

Alors on a $P(Fe)=0.6,$ $P(Lu/Fe)=\frac{1}{3};$ il s'agit de la 
probabilité conditionnelle probabilité de 
<<porter des lunettes>> sachant que la personne est une femme.
De même, on a $P(Lu/H)=0.5$. On cherche la probabilité
conditionnelle $P(Fe/Lu)$.
D'après la formule des probabilités totales on a :
$P(Fe/Lu)P(Lu)=P(Lu/Fe)P(Fe)$ avec $P(Lu)=P(Lu/Fe)P(Fe)+P(Lu/H)P(H)$.

Application numérique : $P(Lu)=0.4$, donc  $P(Fe/Lu)=\frac{P(Lu/Fe)P(Fe)}{P(Lu)}=0.5$.
Remarque : on peut trouver les mêmes réponses par des raisonnements 
élémentaires.
}
}

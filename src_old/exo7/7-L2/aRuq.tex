\uuid{aRuq}
\exo7id{6906}
\titre{exo7 6906}
\auteur{ruette}
\organisation{exo7}
\datecreate{2013-01-24}
\isIndication{false}
\isCorrection{true}
\chapitre{Probabilité discrète}
\sousChapitre{Variable aléatoire discrète}
\module{Probabilité et statistique}
\niveau{L2}
\difficulte{}

\contenu{
\texte{
On lance $10$ fois une pièce supposée bien équilibrée. 
 On désigne par $X$ la fréquence du nombre de fois où pile a 
 été obtenu (c'est-à-dire le nombre de pile divisé
par 10).
}
\begin{enumerate}
    \item \question{Quelle est  la loi de $X$ ?}
\reponse{$Y\sim B(10,1/2)$, $X=Y/10$.}
    \item \question{Avec quelle probabilité $X$  est-elle strictement au dessus de 0,5 ?}
\reponse{$P(X>0,5)=P(Y>5)=P(Y=6,7,8,9,10)\simeq 0,377$.}
    \item \question{Avec quelle probabilité $X$ est-elle comprise entre 0,4 et 0,6 (bornes
incluses) ?}
\reponse{$P(0,4\le X\le 0,6)=P(4\le Y\le 6)=P(Y=4,5,6)\simeq 0,656$}
    \item \question{Déterminer le plus petit entier $a>0$ telle que la probabilité que $X$ 
soit dans l'intervalle $[0,5-\frac{a}{10},0,5+\frac{a}{10}]$ soit supérieure à $95\%$.}
\reponse{$P(3\le Y\le 7)\simeq 0,891$. $P(2\le Y\le 8)\simeq 0,978$. Donc $a=3$.}
    \item \question{On lance la pièce $10$ fois. Elle tombe $3$ fois sur pile et
$7$ fois sur face. D'après vous la pièce est-elle bien
équilibrée (on justifiera sa réponse  en utilisant la question
3~? Même question si on obtient 1 fois pile et 9 fois face.}
\reponse{Oui. Non.}
\end{enumerate}
}

\uuid{1PPl}
\titre{Dérivées partielles et règle des chaînes}
\chapitre{Fonction de plusieurs variables}
\niveau{L2}
\module{Analyse}
\sousChapitre{Dérivées partielles}
\auteur{Jean-François Culus}
\organisation{AMSCC}
\difficulte{2}
\contenu{

\texte{On considère la fonction $f$ définie sur $\mathbb{R}^2$ par 
$$f(s,t)= s^2t+e^{st}$$
On considère  les fonctions $u$ et $v$, définies sur $\mathbb{R}^2$ par:
$$u(x,y)=x^2+y \text{ et } v(x,y)=x-y^2$$
Enfin, on pose $z(x,y)=f(u(x,y),v(x,y))$. 
}

\begin{enumerate}
\item 
\question{ Calculer $\frac{\partial f}{\partial s}(s,t)$ et $\frac{\partial f}{\partial t}(s,t)$}
\reponse{
$$\frac{\partial f}{\partial s}(s,t)=2st+t\,e^{st},
\qquad
\frac{\partial f}{\partial t}(s,t)=s^{2}+s\,e^{st}.
$$
}

\item \question{Calculer les dérivées partielles $\frac{\partial u}{\partial x}(x,y)$, $\frac{\partial u}{\partial y}(x,y)$,
$\frac{\partial v}{\partial x}(x,y)$ et $\frac{\partial v}{\partial y}(x,y)$.}

\reponse{$$\frac{\partial u}{\partial x}=2x,\quad \frac{\partial u}{\partial y}=1,\quad
\frac{\partial v}{\partial x}=1,\quad \frac{\partial v}{\partial y}=-2y$$}

\item \question{En appliquant la règle des chaînes, calculer $\frac{\partial z}{\partial x}(x,y)$, $\frac{\partial z}{\partial y}(x,y)$.}

\reponse{Rappelons la règle des chaînes:
$$\frac{\partial z}{\partial x}
=\frac{\partial f}{\partial s}(u,v)\,\frac{\partial u}{\partial x}
+\frac{\partial f}{\partial t}(u,v)\,\frac{\partial v}{\partial x},
\qquad
\frac{\partial z}{\partial y}
=\frac{\partial f}{\partial s}(u,v)\,\frac{\partial u}{\partial y}
+\frac{\partial f}{\partial t}(u,v)\,\frac{\partial v}{\partial y}$$
Nous avions:
$$\frac{\partial f}{\partial s}(s,t)=2st+t\,e^{st},
\qquad
\frac{\partial f}{\partial t}(s,t)=s^{2}+s\,e^{st}.$$
En appliquant en $(u,v)=(x^{2}+y,\;x-y^{2})$ nous obtenons:
$$\frac{\partial f}{\partial s}(u,v)=2u v+v\,e^{uv},\qquad
\frac{\partial f}{\partial t}(u,v)=u^{2}+u\,e^{uv}.$$
Donc, par la règle des chaînes nous avons: 
$$
\frac{\partial z}{\partial x}
= (2(x^2+y)(x-y^2)+(x-y^2)\,e^{(x^2+y)(x-y^2)})\,(2x)\;+\;((x^2+y)^{2}+(x^2+y)\,e^{(x^2+y)(x-y^2)})
$$

$$
\frac{\partial z}{\partial y}
= (2(x^2+y)(x-y^2)+(x-y^2)\,e^{(x^2+y)(x-y^2)})\,(1)\;+\;((x^2+y)^{2}+(x^2+y)\,e^{(x^2+y)(x-y^2)})\,(-2y)
$$
}
\end{enumerate}
}
\uuid{Cefu}
\chapitre{Série numérique}
\niveau{L1}
\module{Analyse}
\sousChapitre{Autre}
\titre{Vrai ou faux}
\theme{séries}
\auteur{}
\datecreate{2023-06-14}
\organisation{AMSCC}
\difficulte{}
\contenu{

\texte{ Pour chacune des assertions suivantes, dire si elle est vraie ou fausse. Une justification, le cas échéant avec un contre exemple, est attendue.  }

\begin{enumerate}
	\item \question{ Soit $(u_n)$ une suite de réels positifs tels que $\lim\limits_{n \to +\infty} u_n = +\infty$. \\ Alors $\lim\limits_{n \to +\infty} \frac{1}{u_n^2} = 0$ et la série $\sum\limits_{n \geq 0} \frac{1}{u_n^2}$ converge. }
	\reponse{ Faux. Par exemple, considérer $u_n = \sqrt{n}$ pour tout $n \in \N$. }
	\item \question{ Soit $(a_n)$ une suite de réels tels que la série entière $\sum a_n x^n$ ait un rayon de convergence $R=2$. \\ Alors la série $\sum a_n (-3)^n$ diverge. }
	\reponse{ Vrai. Par définition du rayon de convergence, si $R=2$ alors la série $\sum a_n x^n$ diverge  pour tout $x \in ]-\infty;-2[$, donc en particulier pour $x=-3$. }
\end{enumerate}


}

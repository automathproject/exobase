\uuid{Ajfy}
\chapitre{Probabilité continue}
\niveau{L2}
\module{Probabilité et statistique}
\sousChapitre{Loi normale}
\titre{Deux approximations}
\theme{loi normale, approximation de loi, loi de Poisson}
\auteur{Maxime Nguyen}
\datecreate{2023-09-18}
\organisation{AMSCC}

\difficulte{}
\contenu{

\texte{ Une machine produit des rondelles métalliques en grande série. Une rondelle est acceptée si son diamètre extérieur est compris entre $21.9$ et $22.1$ mm. On suppose que sur l'ensemble de la production, le diamètre extérieur des rondelles est une variable aléatoire $X$ qui suit une loi normale de moyenne $22$ mm et d'écart-type $0.05$ mm. }
\begin{enumerate}
	\item \question{ Quelle est la probabilité $p$ qu'une pièce soit refusée ? }
	\reponse{ On calcule la probabilité qu'une pièce soit acceptée
		\begin{align*}
		\prob(21.9\leq X\leq 22.1)
		&= \prob(-2\leq \frac{X-22}{0.05}\leq 2)
		=2\prob(\frac{X-22}{0.05}\leq 2)-1
		=2\times 0.9772-1
		=0.9544.
		\end{align*}
		Ainsi, la probabilité qu'une pièce soit refusée est $p=1-0.9544=0.0456$.
	}
	
	\item \question{ On prélève $100$ pièces. En utilisant une approximation par la loi de Poisson, donner une approximation de la probabilité qu'il y ait $k$ rondelles refusées, pour $k\in\{0,1,2,3,4\}$. }
	\reponse{ 
		Soit $Y$ le nombre de rondelles refusées sur les $100$ pièces. On a $Y\sim \mathcal{B}(100,p)$ et $\E(Y)=100\times p = 4.56$ donc $Y$ peut être approchée par la variable aléatoire $Z$ de loi $\mathcal{P}(4.56)$. Ainsi,
		\[ \forall k \in\{0,\cdots , 4\},\quad \prob(Y=k)\simeq \prob(Z=k)=\frac{4.56^k}{k!}e^{-4.56}.\]
		Les résultats demandés sont dans le tableau suivant:
		\begin{center}
			\begin{tabular}{|c|c|c|c|c|c|}
				\hline
				$k$ & 0 & 1 & 2 & 3 & 4 \\
				\hline
				$\prob(Y=k)$ & $0.0105$ & $0.0477$ & $0.1088$ & $0.1653$ & $0.1885$ \\
				\hline
			\end{tabular}
		\end{center}
	}
	
	\item \question{ On prélève $\nombre{1000}$ pièces. Proposer une approximation de la probabilité qu'il y ait au moins $50$ pièces refusées. }
	\reponse{ 
		Soit $R$ le nombre de pièces refusées parmi les $\nombre{1000}$ pièces. Alors $Z\sim \mathcal{B}(\nombre{1000},p)$ qui peut être approchée par une loi Normale:
		\begin{align*}
		\prob(Y\geq 50) 
		&\simeq \prob(Z\geq 49.5) \quad \text{ où } Z \sim \mathcal{N}(45.6,\sigma^2=43.52) \\
		&\simeq \p\left(\frac{Z-45.6}{\sqrt{43.52}}\geq 0.59\right) \\
		& \simeq 1- \p\left(\frac{Z-45.6}{\sqrt{43.52}}\leq 0.59\right) \\
		&\simeq 1-0.7224 \\
		&\simeq 0.2776
		\end{align*}
		Il y a donc environ $27.76$\% de chances d'avoir au moins $50$ pièces refusées dans le lot de $\nombre{1000}$ pièces.
	}
	
\end{enumerate}
}
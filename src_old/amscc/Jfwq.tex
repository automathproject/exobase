\uuid{Jfwq}
\chapitre{Série numérique}
\niveau{L1}
\module{Analyse}
\sousChapitre{Autre}
\titre{Série télescopique}
\theme{séries}
\auteur{}
\datecreate{2023-05-17}
\organisation{AMSCC}
\difficulte{}
\contenu{



\texte{ 	Soit $n$ un entier naturel tel que $n\geq 2$. }
	\begin{enumerate}
		\item \question{ Décomposer $\displaystyle\frac{1}{n^2-1}$ en éléments simples, c'est-à-dire chercher des réels $a$ et $b$ tels que :
		$$\frac{1}{n^2-1}=\frac{a}{n-1}+\frac{b}{n+1}.$$ }
		\reponse{On remarque que $$\frac1{n^2-1}=\frac1{2(n-1)}-\frac{1}{2(n+1)}$$.}
		\item \question{ En déduire que la série $\displaystyle\sum_{n\geq 2} \frac{1}{n^2-1}$ converge et calculer sa somme. }
		\indication{Écrire la somme partielle et observer les simplifications. }
		\reponse{On écrit la somme partielle : 
			\begin{align*}
			\sum_{n=2}^N \frac1{n^2-1} &= 	\sum_{n=2}^N \frac1{2(n-1)}-\frac{1}{2(n+1)} \\
			&= \frac12 \sum_{n=2}^N \frac{1}{n-1} - \frac12 \sum_{n=2}^N \frac{1}{n+1} \\
			&= \frac12 \sum_{n=1}^{N-1} \frac{1}{n} - \frac12 \sum_{n=3}^{N+1} \frac{1}{n} \\
			&= \frac12 \left(1 + \frac{1}{2} + \sum_{n=3}^{N-1} \frac{1}{n} - \left(\sum_{n=3}^{N-1} \frac{1}{n} + \frac{1}{N} + \frac{1}{N+1}\right)   \right) \\
			&= \frac12 \left(1 + \frac{1}{2} - \frac{1}{N} - \frac{1}{N+1} \right)\\
			&\xrightarrow[N \to +\infty]{} \frac12 \left(1 + \frac{1}{2}\right) = \frac{3}{4}
			\end{align*}
			Donc la série converge et sa somme vaut $\frac34$.
		}
	\end{enumerate}
}

\uuid{8SbE}
\chapitre{Série de Fourier}
\niveau{L2}
\module{Analyse}
\sousChapitre{Calcul de coefficients}
\titre{Fonction périodique et série de Fourier}
\theme{séries de Fourier}
\auteur{}
\datecreate{2024-06-13}
\organisation{AMSCC}
\difficulte{}
\contenu{

Soit 
$f:\R\rightarrow \R$, $2\pi$-p\'eriodique, d\'efinie par $f(x)=\pi-|x|$ pour tout $x\in]-\pi,\pi[ $.
\begin{enumerate}
	\item \question{ Calculer la série de Fourier trigonométrique de $f$. }
	\reponse{
		La fonction $f$ étant paire, les coefficients de Fourier $b_n(f)$ sont nuls et pour tout $n \geq 1$ : 
		\begin{align*}
			a_n(f)  &= \frac{2}{\pi} \int_0^{\pi} (\pi-t)\cos(nt)\mathrm{d}t \\
			&= 0 - \frac{2}{\pi} \int_0^{\pi} t\cos(nt)\mathrm{d}t \\
			&= -\frac{2}{\pi} \left[\frac{t}{n}\sin(nt)\right]_0^{\pi} + \frac{2}{\pi} \int_0^\pi \frac{1}{n}\sin(nt)\mathrm{d}t \\
			&= 0 + \frac{2}{n\pi}\left[\frac{-1}{n}\cos(nt)\right]_0^{\pi} \\
			&= \frac{2}{n^2\pi}(1-(-1)^n) \\
			&= \begin{cases}
				0 & \text{si $n$ pair} \\
				\frac{4}{n^2\pi} & \text{si $n$ impair} 
			\end{cases} 
		\end{align*}
		De plus, $a_0(f) = \frac{2}{\pi}\int_0^{\pi} (\pi-t) \mathrm{d}t = 	\pi$. 
		Donc la série de Fourier est  $\displaystyle S_n(f) = \frac{\pi}{2} + \sum_{n \geq 0} \frac{4}{(2n+1)^2\pi} \cos((2n+1)x)$. 
	}
	\item \question{ En déduire la valeur de la somme $\displaystyle \sum_{n=0}^{+\infty} \frac{1}{(2n+1)^2}$. }
	\reponse{La fonction $f$ est continue sur $\R$ donc d'après le théorème de Dirichlet, pour tout $x \in \R$, $$f(x) = \frac{\pi}{2} + \sum_{n = 0}^{+\infty} \frac{4}{(2n+1)^2\pi} \cos((2n+1)x)$$
		Donc en particulier pour $x = 0$, on a 
		$$f(0) = \sum_{n = 0}^{+\infty} \frac{4}{(2n+1)^2\pi}$$
		Or $f(0) = \pi$ donc 
		$$\pi = \frac{\pi}{2} + \sum_{n = 0}^{+\infty} \frac{4}{(2n+1)^2\pi}$$
		donc $$\sum_{n=0}^{+\infty} \frac{1}{(2n+1)^2} = \frac{\pi^2}{8}$$ }
\end{enumerate}
}
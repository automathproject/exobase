\uuid{j7Pn}
\chapitre{Probabilité continue}
\niveau{L2}
\module{Probabilité et statistique}
\sousChapitre{Densité de probabilité}
\titre{Paramètres de la loi uniforme}
\theme{variables aléatoires à densité}
\auteur{Maxime NGUYEN}
\datecreate{2022-10-17}
\organisation{AMSCC}
\difficulte{}
\contenu{


 \question{ Soit $X$ une variable aléatoire suivant une loi uniforme sur $[a;b]$. Calculer $\EX$ et $\mathbb{V}(X)$. }
 
 \reponse{ Soit $f$ une fonction densité de $X$. Il suffit de calculer 
 	\begin{align*}
 		\mathbb{E}(X) &= \int_{-\infty}^{+\infty} xf(x)dx \\
 		&= \int_a^b x \times \frac{1}{b-a} dx \\
 		&=  \frac{1}{b-a} \times \left( \frac{b^2}{2} - \frac{a^2}{2} \right) \\
 		&=  \frac{(b+a)(b-a)}{2(b-a)}\\
 		&= \frac{a+b}{2}
 	\end{align*}
 	
 	Pour calculer la variance $\mathbb{V}(X)$, il reste à calculer :
 	\begin{align*}
 		\mathbb{E}(X^2) &= \int_{-\infty}^{+\infty} x^2f(x)dx \\
 		&= \int_a^b x^2 \times \frac{1}{b-a} dx \\
 		&=  \frac{1}{b-a} \times \left( \frac{b^3}{3} - \frac{a^3}{3} \right) \\
 		&= \frac{a^2+2ab+b^2}{3}
 	\end{align*}
 	puis on applique la formule de Huygens :
 	\begin{align*}
 		\mathbb{V}(X)&= \mathbb{E}(X^2) - \mathbb{E}(X)^2 \\
 		&= \frac{a^2+2ab+b^2}{3} - \left(\frac{a+b}{2}\right)^2 \\
 		&= \frac{(b-a)^2}{12}
 \end{align*} }}

\uuid{V44u}
\chapitre{Probabilité continue}
\niveau{L2}
\module{Probabilité et statistique}
\sousChapitre{Loi normale}
\titre{Fréquence d’arrivées logistiques}
\theme{loi de Poisson, loi normale, approximation de loi}
\auteur{Maxime Nguyen}
\datecreate{2025-10-07}
\organisation{AMSCC}

\difficulte{3}

\contenu{
	
	\texte{ 
		L’observation a permis d’affirmer que le nombre de camions de ravitaillement arrivant à un poste logistique entre 9h00 et 10h30 suit une loi de Poisson de paramètre $\lambda = 16$.  
		On donne ci-dessous un extrait de la table de répartition :
		\begin{center}
			\begin{tabular}{|c|c|c|c|c|c|}
				\hline
				$k$ & 14 & 15 & 16 & 17 & 18 \\
				\hline
				$\prob(X\leq k)$ & $0.3675$ & $0.4667$ & $0.5660$ & $0.6593$ & $0.7423$ \\
				\hline
			\end{tabular}
		\end{center}
	}
	
	\begin{enumerate}
		\item 
		\begin{enumerate}
			\item \question{ En utilisant le tableau ci-dessus, déterminer $\prob(X=15)$. }
			\reponse{ 
				\[
				\prob(X=15) = \prob(X\leq 15) - \prob(X\leq 14)
				= 0.4667 - 0.3675 \simeq 0.0992.
				\]
				Il y a donc environ $9.9\%$ de chance que le poste reçoive exactement $15$ camions durant cette période.
			}
			
			\item \question{ Déterminer les paramètres de la loi normale que suit la variable aléatoire $Y$ qui approche $X$. }
			\reponse{ 
				La variable $X$ peut être approchée par $Y$ qui suit la loi normale :
				\[
				Y \sim \mathcal{N}(\mu=16, \sigma^2=16).
				\]
			}
			
			\item \question{ Calculer $\alpha = \prob(14.5 \leq Y \leq 15.5)$. Quel est le lien entre $\alpha$ et $\prob(X=15)$ ? }
			\reponse{ 
				\begin{align*}
					\alpha &= \prob(14.5 \leq Y \leq 15.5)
					= \prob\left( \frac{-1.5}{4} \leq \frac{Y-16}{4} \leq \frac{-0.5}{4} \right) \\
					&= \prob(-0.375 \leq Z \leq -0.125) \quad \text{où } Z \sim \mathcal{N}(0,1) \\
					&= \prob(0.125 \leq Z \leq 0.375) \\
					&= \prob(Z\leq 0.375) - \prob(Z\leq 0.125) \\
					&= 0.64615 - 0.54975 = 0.0964.
				\end{align*}
				$\alpha$ est une approximation (avec correction de continuité) de $\prob(X=15)$, soit environ $9.6\%$.
			}
		\end{enumerate}
		
		\item \question{ Déterminer une approximation de $\prob(15 \leq X \leq 20)$. Interpréter le résultat dans le contexte. }
		\reponse{ 
			\begin{align*}
				\prob(15\leq X\leq 20)
				&\simeq \prob(14.5\leq Y\leq 20.5) \\
				&= \prob\left(-0.375 \leq \frac{Y-16}{4} \leq 1.125\right) \\
				&= \prob(Z\leq 1.125) - \prob(Z\leq -0.375) \\
				&= 0.8697 - (1 - 0.64615) \\
				&= 0.8697 - 0.35385 = 0.51585.
			\end{align*}
			Il y a donc environ $51.6\%$ de chance que le nombre de camions arrivant entre 9h00 et 10h30 soit compris entre $15$ et $20$.  
			Cela permet au responsable logistique de dimensionner la zone de déchargement pour un trafic médian.
		}
	\end{enumerate}
}

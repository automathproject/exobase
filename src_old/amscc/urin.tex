\uuid{urin}
\chapitre{Série entière}
\niveau{L2}
\module{Analyse}
\sousChapitre{Calcul de la somme d'une série entière}
\titre{ Calcul d'une somme de série entière}
\theme {séries entières}
\auteur{ }
\datecreate{2023-06-01}
\organisation{ AMSCC }

\difficulte{}
\contenu{
	\begin{enumerate}
	\item \question{ En utilisant le résultat de la somme de la série entière $\sum_{k=0}^{+\infty} x^k = \frac{1}{1-x}$, déterminer la somme de la série entière à valeurs réelles $\displaystyle S_1(x)=\sum_{k=0}^{+\infty} k x^k$. On précisera son domaine de convergence. }
	\reponse{
		On dérive $\frac{1}{1-x}$ : $\displaystyle \frac{1}{(1-x)^2}=\sum_{k=0}^{+\infty} k x^{k-1}$. En multipliant par $x$, on obtient :
		\[\forall x \in ]-1;1[, \quad S_1(x)=\frac{x}{(1-x)^2}.\]
	}
	\item \question{ Déterminer la somme de la série entière à valeurs réelles $\displaystyle S_2(x)= \sum_{k=0}^{+\infty} (k+1)^2x^k$. On précisera son domaine de convergence. }
	\reponse{ 
		On dérive deux fois $\frac{1}{1-x}$ : $\displaystyle \frac{2}{(1-x)^3}=\sum_{k=0}^{+\infty} k(k-1)x^{k-2} = \sum_{k=2}^{+\infty} k(k-1)x^{k-2}$ car les deux premiers termes sont nuls. Par changement d'indice, on obtient : 
		\[\forall x \in ]-1;1[, \quad \frac{2}{(1-x)^3} = \sum_{k=0}^{+\infty} (k+1)k x^{k-1}.\] d'où : 
		\[\forall x \in ]-1;1[, \quad \frac{2x}{(1-x)^3} = \sum_{k=0}^{+\infty} (k+1)k x^k.\] 
		Or $(k+1)^2 = k(k+1) + (k+1)$ donc :
		\[\forall x \in ]-1;1[,\quad S_2(x) = \sum_{k=0}^{+\infty} k(k+1)x^k + \sum_{k=0}^{+\infty} (k+1)x^k = \frac{2x}{(1-x)^3} + \frac{1}{(1-x)^2} = \frac{x+1}{(1-x)^3}.\]
	}
	
	\item \question{ Après avoir décomposé la fraction rationnelle $\displaystyle \frac{1}{n(n+2)}$ en éléments simples,  déterminer la somme de la série entière à valeurs réelles $\displaystyle S_3(x)=\sum_{n= 1}^{+\infty} \frac{x^n}{n(n+2)}$. On précisera son rayon de convergence. }
	\reponse{ Ici, $R=1$ :
		\[\forall x \in ]-1;1[, \quad S_3(x)=\frac{1}{2}\left( -\ln(1-x)+\frac{1}{x^2}\left[\ln(1-x)+x+\frac{x^2}{2}\right]\right).\]
	}
\end{enumerate}
}







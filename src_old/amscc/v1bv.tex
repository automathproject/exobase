\uuid{v1bv}
\titre{Estimation et Tests pour une Loi Uniforme}
\chapitre{Statistique}
\niveau{L2}
\module{Probabilité et statistique}
\sousChapitre{Estimation}
\theme{}
\auteur{}
\datecreate{2025-03-20}
\organisation{}

\difficulte{}
\contenu{

\texte{
Soit $X_1, \ldots, X_n$ un échantillon de loi uniforme sur $[0, q]$.
}

\begin{enumerate}
 %   \item \question{Le modèle est-il régulier ?}
%    \indication{}
 %_{[x,+\infty[}(q)$ est discontinue en $x$ et le modèle n’est donc pas régulier.}

    \item \question{Par la méthode des moments, proposer un estimateur de $q$. Montrer sa consistance et sa normalité asymptotique. En déduire un intervalle de confiance asymptotique de niveau $1 - \alpha$.}
    \indication{}
    \reponse{On a $E[X] = \frac{q}{2}$ et $V[X] = \frac{q^2}{12}$, et on a donc l’estimateur $\hat{q}_n = 2\bar{X}_n$. On a la consistance via la LFGN et le théorème de continuité, et comme
    \[
    \sqrt{n} (\bar{X}_n - \frac{q}{2}) \xrightarrow{L} \mathcal{N}(0, \frac{q^2}{12}).
    \]
    En multipliant par 2, on obtient
    \[
    \sqrt{n} (\hat{q}_n - q) \xrightarrow{L} \mathcal{N}(0, \frac{q^2}{3}).
    \]
    Intervalle de confiance, version 1 : De plus, par Slutsky, on a
    \[
    \sqrt{3n} \frac{\hat{q}_n - q}{\hat{q}_n} \xrightarrow{L} \mathcal{N}(0, 1)
    \]
    et on obtient donc l’intervalle de confiance
    \[
    \left[ \hat{q}_n - q_{1-\alpha} \frac{\hat{q}_n}{\sqrt{3n}}, \hat{q}_n + q_{1-\alpha} \frac{\hat{q}_n}{\sqrt{3n}} \right]
    \]
    où $q_{1-\alpha}$ est le quantile d’ordre $1 - \alpha$ de la loi normale centrée réduite.

    Intervalle de confiance version 2 : En divisant par $\frac{q}{\sqrt{3}}$ le TLC, on obtient
    \[
    \sqrt{3n} \left( \frac{\hat{q}_n}{q} - 1 \right) \xrightarrow{L} \mathcal{N}(0, 1)
    \]
    et donc
    \[
    P \left( \frac{\hat{q}_n}{q} - 1 \leq q_{1-\alpha} \frac{1}{\sqrt{3n}} \right) \approx 1 - \alpha.
    \]
    On obtient donc l’intervalle (pour $n > q^2_{1-\alpha}$)
    \[
    \left[ \hat{q}_n \left( 1 - q_{1-\alpha} \frac{1}{\sqrt{3n}} \right), \hat{q}_n \left( 1 + q_{1-\alpha} \frac{1}{\sqrt{3n}} \right) \right].
    \]
    }

    \item \question{Soit $q_0 > 0$, proposer un test de niveau asymptotique $\alpha$ pour tester $H_0 : q = q_0$ contre $H_1 : q \neq q_0$.}
    \indication{}
    \reponse{On a la zone de rejet
    \[
    ZR = \left( \sqrt{3n} \frac{\hat{q}_n - q_0}{\hat{q}_n} > q_{1-\alpha} \right) = \left( \hat{q}_n > q_0 + q_{1-\alpha} \frac{\hat{q}_n}{\sqrt{3n}} \right).
    \]
    }

    \item \question{Calculer l’estimateur du maximum de vraisemblance et calculer son risque quadratique.}
    \indication{}
    \reponse{On a pour tout $q > 0$,
    \[
    L_X(q) = \prod_{i=1}^n \frac{1}{q} 1_{[0,q]}(X_i) = \frac{1}{q^n} 1_{[X_{(n)},+\infty[}(q).
    \]
    et le maximum est donc atteint en $X_{(n)}$.
    }

    \item \question{Calculer la loi limite de $n(q_{MV} - q)$.}
    \indication{}
    \reponse{Pour tout $x \in [0, nq]$, on a par indépendance des $X_i$
    \[
    P(n(q - X_{(n)}) \geq x) = P(X_{(n)} \leq q - \frac{x}{n}) = \left( P(X_1 \leq q - \frac{x}{n}) \right)^n = \left( 1 - \frac{x}{qn} \right)^n \xrightarrow{n \to \infty} \exp \left( -\frac{x}{q} \right).
    \]
    et donc $n(q - X_{(n)})$ converge en loi vers une loi exponentielle de paramètre $q^{-1}$.
    }

    \item \question{Déterminer $c_{\alpha,n}$ tel que $[X_{(n)}, c_{\alpha,n}X_{(n)}]$ soit un intervalle de confiance de niveau $1 - \alpha$.}
    \indication{}
    \reponse{On a
    \[
    P(q \leq c_{\alpha,n}X_{(n)}) = 1 - P(X_{(n)} \leq \frac{q}{c_{\alpha,n}}) = 1 - \left( P(X_1 \leq \frac{q}{c_{\alpha,n}}) \right)^n = 1 - \frac{1}{c_{\alpha,n}^n}.
    \]
    De plus,
    \[
    \frac{1}{c_{\alpha,n}^n} = \alpha \implies c_{\alpha,n} = \exp \left( -\frac{1}{n} \ln(\alpha) \right).
    \]
    }

    \item \question{Calculer la médiane de $X_1$ et en déduire un nouvel estimateur de $q$. Donner sa normalité asymptotique ainsi qu’un nouvel intervalle de confiance asymptotique.}
    \indication{}
    \reponse{On a $X_{([n/2])} = \frac{q}{2}$ et on propose donc l’estimateur $2X_{([n/2])}$. On a la convergence de $X_{([n/2])}$ via le théorème du cours et donc la consistance de $2X_{([n/2])}$. On a de plus
    \[
    \sqrt{n} \left( X_{([n/2])} - \frac{q}{2} \right) \xrightarrow{L} \mathcal{N}(0, \frac{q^2}{4})
    \]
    et on obtient en multipliant par 2 la normalité asymptotique. On obtient les intervalles de confiances comme pour l’estimateur des moments.
    }

    \item \question{Quel intervalle choisiriez-vous ?}
    \indication{}
    \reponse{Non seulement celui obtenu avec le maximum de vraisemblance est plus précis quand $n$ est grand (faire un DL pour le vérifier), mais en plus il est non asymptotique.}
\end{enumerate}

}

\uuid{gAMY}
\chapitre{Matrice}
\niveau{L1}
\module{Algèbre}
\sousChapitre{Autre}
\titre{ Puissance d'une matrice à paramètres }
\theme{calcul matriciel}
\auteur{}
\datecreate{2024-01-31}
\organisation{AMSCC}

\difficulte{}
\contenu{
    \texte{
Soit la matrice de  $M_{\alpha,\beta} = \begin{pmatrix} \alpha & 0 & -1  \\
    2 & \alpha & \beta\\
    -3-\beta & -1 & \alpha \\
    \end{pmatrix} \in \mathcal{M}_{3}(\R)$.
    }
    \begin{enumerate}
    \item \question{ Montrer que pour tout $\alpha,\beta$ on a $\det(M_{\alpha,\beta})=P(\alpha)$ où $P$ est un polynôme de degré 3 à préciser, ne dépendant pas de $\beta$.  }
    \item \question{ Vérifier que $1$ est racine de $P$.  }
    \item \question{ Déterminer l'ensemble des valeurs de $\alpha,\beta$ pour lesquelles la matrice $M_{\alpha,\beta}$ n'est pas inversible.  }
    \item \question{ Démontrer que $M_{1,\beta}=M_{\alpha,\beta}-(\alpha-1)I_3$ o\`u $I_{3}$ est la matrice identité de $\mathcal{M}_{3}(\R)$. }
    \item \question{ Démontrer que pour tout $\beta \in \R,\,(M_{1,\beta})^{3}=3(M_{1,\beta})^{2}.$  }
    \end{enumerate}
    }
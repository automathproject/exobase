\uuid{QqHC}
\chapitre{Probabilité discrète}
\niveau{L2}
\module{Probabilité et statistique}
\sousChapitre{Variable aléatoire discrète}
\titre{Jeu de boules}
\theme{variables aléatoires discrètes}
\auteur{}
\datecreate{2023-09-01}
\organisation{AMSCC}

\difficulte{4}
\contenu{
	

\texte{ Une urne contient $n$ boules blanches numérotées de $1$ à $n$, où $n\in\N^*$ et deux boules noires numérotées $1$ et $2$. On effectue le tirage de toutes les boules de l'urne, une à une et sans remise. On appelle $X$ le rang d'apparition de la première boule blanche et $Y$ celui du premier numéro $1$. }
\begin{enumerate}
 \item \question{ Déterminer la loi de $X$. }
 \reponse{
 $X$ étant le rang d'apparition de la première boule blanche et l'urne contenant $n$ boules blanches et $2$ boules noires, les valeurs prises par $X$ sont les suivantes
 \[ X(\Omega)=\{1,2,3\}.\]
 Déterminons les probabilités de chaque issue :
 \begin{align*}
  \prob(X=1)&=\prob(\text{''On obtient une boule blanche au premier tirage``})\\
  &=\frac{n}{n+2} \\
  \prob(X=2)&=\prob(\text{''On obtient une boule noire puis une boule blanche``})\\
  &=\frac{2}{n+2}\times\frac{n}{n+1} \\
  \prob(X=3)&=\prob(\text{''On obtient deux boules noires puis une boule blanche``}) \\ &=\frac{2}{n+2}\times \frac{1}{n+1}\times \frac{n}{n}\\
  &=\frac{2}{(n+2)(n+2)}.
 \end{align*}
 On peut vérifier que $\prob(X=1)+\prob(X=2)+\prob(X=3)=1$. On a ainsi déterminer la loi de $X$, que l'on peut résumer dans le tableau ci-dessous:
 \begin{center}
  \begin{tabular}{|c|c|c|c|}
   \hline
   $k$ & $1$ & $2$ & $3$ \\
   \hline
   $\prob(X=k)$ & $\frac{n}{n+2}$ & $\frac{2n}{(n+2)(n+1)}$ & $\frac{2}{(n+2)(n+1)}$ \\
   \hline
  \end{tabular}
 \end{center}
 }
 
 \item \question{ Montrer que les événements $\{X=1\}$ et $\{Y=1\}$ sont indépendants si et seulement si $n=2$. }
 \reponse{ On a:
 \begin{itemize}
  \item $\prob(X=1,Y=1)=\prob(\text{''On obtient la boule blanche numérotée $1$ au premier tirage``})=\frac{1}{n+2}$.
  \item $\prob(X=1)\prob(Y=1)=\frac{n}{n+2}\times \frac{2}{n+2}$
  \item $\{X=1\}$ et $\{Y=1\}$ sont indépendants si et seulement si
  \begin{align*}
   \prob(X=1,Y=1)=\prob(X=1)\prob(Y=1) \quad
   & \Leftrightarrow \quad \frac{1}{n+2}=\frac{2n}{(n+2)^2} \\
   & \Leftrightarrow \quad 2n=n+2 \\
   & \Leftrightarrow \quad n=2.
  \end{align*}
 \end{itemize}
 }
 
 \item \question{ Montrer que les variables aléatoires $X$ et $Y$ ne sont pas indépendantes. }
 \reponse{ 
 Pour $n\neq 2$, on a montré, par la question précédente, que les événements $\{X=1\}$ et $\{Y=1\}$ n'étaient pas indépendants, ce qui montre que les variables $X$ et $Y$ ne sont pas indépendantes.
 
 Pour $n=2$, on a alors $2$ boules blanches et $2$ boules noires dans l'urne. Ainsi,
 \[ \prob(X=2,Y=2)=\frac{1}{4}\times \frac{1}{3}=\frac{1}{12}\]
 et $\prob(X=2)=\frac{1}{3}$ et $\prob(Y=2)=\frac{1}{3}$. Par conséquent, $\prob(X=2,Y=2)\neq \prob(X=2)\prob(Y=2)$, ce qui implique que les variables aléatoires $X$ et $Y$ ne sont pas indépendantes.
 }
 
 \item \texte{ On suppose maintenant que $n=2$. }
 \begin{enumerate}
  \item \question{ Montrer que $X$ et $Y$ ont même loi. }
  \reponse{
  La loi de $X$ a été déterminée à la question $1$. Pour $Y$, on a $Y(\Omega)=\{1,2,3\}$ et
  \begin{align*}
   &\prob(Y=1)=\frac{2}{4}=\frac{1}{2}=\prob(X=1) \\
   &\prob(Y=2)=\frac{1}{3}=\prob(X=2) \\
   &\prob(Y=3)=\frac{2}{4}\times \frac{1}{3}\times 1 = \frac{1}{6}=\prob(X=3)
  \end{align*}
donc $X$ et $Y$ ont même loi.
  }
  
  \item \question{ Déterminer la loi du couple $(X,Y)$. }
  \reponse{
    \begin{center}
\begin{tabular}{|c|c|c|c||c|}
\hline
 $Y \backslash X$ & $1$ & $2$ & $3$ & $\prob_{Y}$ (loi de $Y$)  \\
 \hline
 $1$ & $\frac{1}{4}$ & $\frac{1}{6}$ & $\frac{1}{12}$ & $\frac{1}{2}$  \\
 \hline
 $2$ & $\frac{1}{6}$ & $\frac{1}{12}$ & $\frac{1}{12}$ & $\frac{1}{3}$  \\
 \hline
 $3$ & $\frac{1}{12}$ & $\frac{1}{12}$ & $0$ & $\frac{1}{6}$ \\
 \hline
 \hline
 $\prob_{X}$ (loi de $X$) & $\frac{1}{2}$ & $\frac{1}{3}$ & $\frac{1}{6}$ & $1$  \\
 \hline
\end{tabular}
\end{center}
  }  
 \end{enumerate}
\end{enumerate}
}
\uuid{IwCg}
\titre{Une fonction "créneau" simple et sa parité}
\chapitre{Série de Fourier}
\niveau{L2}
\module{Analyse}
\sousChapitre{Calcul de coefficients}
\theme{Séries de Fourier, Coefficients de Fourier}
\auteur{Maxime Nguyen}
\datecreate{2025-05-09}
\organisation{AMSCC}

\difficulte{}
\contenu{
	
	\texte{
		Soit la fonction $g(x)$ définie sur $\mathbb{R}$, $2\pi$-périodique, telle que pour $x \in [-\pi, \pi]$ :
		$$ g(x) = \begin{cases} 1 & \text{si } x \in [-\pi/2, \pi/2] \\ 0 & \text{si } x \in ]-\pi, -\pi/2[ \cup ]\pi/2, \pi[ \end{cases} $$
		L'objectif de cet exercice est de calculer les coefficients de Fourier pour cette fonction paire simple et d'interpréter le coefficient $a_0$.
	}
	
	\begin{enumerate}
		\item \question{Tracer le graphe de $g(x)$ sur l'intervalle $[-2\pi, 2\pi]$.}
		\indication{Commencez par tracer la fonction sur $[-\pi, \pi]$ puis utilisez la $2\pi$-périodicité pour l'étendre.}
		\reponse{
			La fonction $g(x)$ vaut $1$ sur l'intervalle $[-\pi/2, \pi/2]$. Elle vaut $0$ sur les intervalles $]-\pi, -\pi/2[$ et $]\pi/2, \pi[$.
			Par $2\pi$-périodicité ($g(x+2k\pi)=g(x)$ pour $k \in \mathbb{Z}$):
			\begin{itemize}
				\item Sur $[-2\pi, -3\pi/2]$: $x \in [-2\pi, -3\pi/2] \implies x+2\pi \in [0, \pi/2]$. Sur cet intervalle, $g(x+2\pi)=1$, donc $g(x)=1$.
				\item Sur $]-3\pi/2, -\pi/2[$: $x \in ]-3\pi/2, -\pi] \implies x+2\pi \in ]\pi/2, \pi]$. Sur cet intervalle, $g(x+2\pi)=0$. Et pour $x \in ]-\pi, -\pi/2[$, $g(x)=0$ par définition. Donc $g(x)=0$ sur $]-3\pi/2, -\pi/2[$.
				\item Sur $[-\pi/2, \pi/2]$: $g(x)=1$ par définition.
				\item Sur $]\pi/2, 3\pi/2[$: $x \in ]\pi/2, \pi] \implies g(x)=0$ par définition. Et pour $x \in ]\pi, 3\pi/2[ \implies x-2\pi \in ]-\pi, -\pi/2[$. Sur cet intervalle, $g(x-2\pi)=0$. Donc $g(x)=0$ sur $]\pi/2, 3\pi/2[$.
				\item Sur $[3\pi/2, 2\pi]$: $x \in [3\pi/2, 2\pi] \implies x-2\pi \in [-\pi/2, 0]$. Sur cet intervalle, $g(x-2\pi)=1$, donc $g(x)=1$.
			\end{itemize}
			Le graphe est le suivant :
			\begin{center}
				\begin{tikzpicture}[scale=0.75, every node/.style={scale=0.75}]
					% Axes
					\draw[->] (-2.3*pi,0) -- (2.3*pi,0) node[right] {$x$};
					\draw[->] (0,-0.3) -- (0,1.3) node[above] {$g(x)$};
					
					% Graduations x
					\foreach \x/\xtext in {-2*pi/$-2\pi$, -1.5*pi/$\frac{-3\pi}{2}$, -pi/$-\pi$, -0.5*pi/$\frac{-\pi}{2}$, 0.5*pi/$\frac{\pi}{2}$, pi/$\pi$, 1.5*pi/$\frac{3\pi}{2}$, 2*pi/$2\pi$} {
						\ifdim\x pt=0pt \else \draw (\x,0.05) -- (\x,-0.05) node[below] {\xtext}; \fi
					}
					\node[below left] at (0,0) {$0$};
					
					% Graduations y
					\draw (-0.05,1) -- (0.05,1) node[left] {$1$};
					
					% Graphe de g(x)
					\def\linewidth{1pt}
					\color{blue}
					% Segments où g(x) = 1
					\draw[line width=\linewidth] (-2*pi,1) -- (-1.5*pi,1);
					\draw[line width=\linewidth] (-0.5*pi,1) -- (0.5*pi,1);
					\draw[line width=\linewidth] (1.5*pi,1) -- (2*pi,1);
					
					% Segments où g(x) = 0
					\draw[line width=\linewidth] (-1.5*pi,0) -- (-0.5*pi,0);
					\draw[line width=\linewidth] (0.5*pi,0) -- (1.5*pi,0);
					
					% Lignes verticales pour les discontinuités
					\draw[densely dashed, blue] (-1.5*pi,0) -- (-1.5*pi,1);
					\draw[densely dashed, blue] (-0.5*pi,0) -- (-0.5*pi,1);
					\draw[densely dashed, blue] (0.5*pi,0) -- (0.5*pi,1);
					\draw[densely dashed, blue] (1.5*pi,0) -- (1.5*pi,1);
					
					% Points fermés sur les segments g(x)=1 (valeurs effectives aux points de saut)
					\foreach \xp in {-2*pi, -1.5*pi, -0.5*pi, 0.5*pi, 1.5*pi, 2*pi} {
						\node[circle, fill=blue, draw=blue, inner sep=0pt, minimum size=1.5pt] at (\xp,1) {};
					}
					% Points ouverts sur les segments g(x)=0 (limites aux points de saut)
					\foreach \xp in {-1.5*pi, -0.5*pi, 0.5*pi, 1.5*pi} {
						\node[circle, fill=white, draw=blue, inner sep=0pt, minimum size=1.5pt] at (\xp,0) {};
					}
				\end{tikzpicture}
			\end{center}
		}
		
		\item \question{La fonction $g(x)$ est-elle paire ? Impaire ? Ni l'une ni l'autre ? Justifier.}

		\reponse{
			Considérons $x \in [-\pi, \pi]$.
			\begin{itemize}
				\item Si $x \in [-\pi/2, \pi/2]$, alors $g(x)=1$. Dans ce cas, $-x \in [-\pi/2, \pi/2]$ également. Donc $g(-x)=1$. Ainsi $g(-x)=g(x)$.
				\item Si $x \in ]-\pi, -\pi/2[$, alors $g(x)=0$. Dans ce cas, $-x \in ]\pi/2, \pi[$. Donc $g(-x)=0$. Ainsi $g(-x)=g(x)$.
				\item Si $x \in ]\pi/2, \pi[$, alors $g(x)=0$. Dans ce cas, $-x \in ]-\pi, -\pi/2[$. Donc $g(-x)=0$. Ainsi $g(-x)=g(x)$.
			\end{itemize}
			Pour les points frontières $\pi$ et $-\pi$: $g(\pi) = g(-\pi)$ par périodicité ($g(\pi)=g(-\pi + 2\pi)=g(-\pi)$). De plus $g(\pi) = 0$ (car $\pi \in ]\pi/2, \pi]$ en considérant la définition sur $[-\pi, \pi[$ ou en prenant la limite pour $x \to \pi^-$). Donc $g(-\pi)=0$. Vérifions $g(-(-\pi))=g(\pi)=0$.
			La fonction $g(x)$ est paire car $g(-x)=g(x)$ pour tout $x \in \mathbb{R}$.
		}
		
		\item \question{Quelle conséquence la parité de $g(x)$ a-t-elle sur ses coefficients de Fourier $b_n$ ?}

		\reponse{
			Si une fonction $f$ est $2\pi$-périodique et paire, alors ses coefficients de Fourier $b_n$ sont tous nuls. En effet, $b_n = \frac{1}{\pi} \int_{-\pi}^{\pi} g(x)\sin(nx) \, dx$. L'intégrande $g(x)\sin(nx)$ est le produit d'une fonction paire ($g(x)$) et d'une fonction impaire ($\sin(nx)$), ce qui donne une fonction impaire. L'intégrale d'une fonction impaire sur un intervalle symétrique par rapport à l'origine ($[-\pi, \pi]$) est nulle.
			Donc, $b_n = 0$ pour tout $n \ge 1$.
		}
		
		\item \question{Calculer le coefficient $a_0$. Quelle est son interprétation graphique ?
			On rappelle que $a_0 = \frac{1}{\pi} \int_{-\pi}^{\pi} g(x) \, dx$.}

		\reponse{
			$a_0 = \frac{1}{\pi} \int_{-\pi}^{\pi} g(x) \, dx$.
			Comme $g(x)$ est paire, $\int_{-\pi}^{\pi} g(x) \, dx = 2 \int_{0}^{\pi} g(x) \, dx$.
			$a_0 = \frac{2}{\pi} \int_{0}^{\pi} g(x) \, dx$.
			Sur $[0, \pi]$, $g(x)=1$ pour $x \in [0, \pi/2]$ et $g(x)=0$ pour $x \in ]\pi/2, \pi]$.
			Donc, $a_0 = \frac{2}{\pi} \left( \int_{0}^{\pi/2} 1 \, dx + \int_{\pi/2}^{\pi} 0 \, dx \right) = \frac{2}{\pi} \left( [x]_{0}^{\pi/2} + 0 \right) = \frac{2}{\pi} \left( \frac{\pi}{2} - 0 \right) = 1$.
			Donc $a_0 = 1$.
			Interprétation graphique : $a_0/2$ est la valeur moyenne de la fonction $g(x)$ sur une période. Ici, $a_0/2 = 1/2$.
			L'aire sous la courbe de $g(x)$ sur une période, par exemple $[-\pi, \pi]$, est $\int_{-\pi/2}^{\pi/2} 1 \, dx = \pi/2 - (-\pi/2) = \pi$. La longueur de la période est $2\pi$. La valeur moyenne est donc $\frac{\text{Aire}}{\text{Longueur période}} = \frac{\pi}{2\pi} = \frac{1}{2}$.
		}
		
		\item \question{Calculer les coefficients $a_n$ pour $n \ge 1$.
			On rappelle que $a_n = \frac{1}{\pi} \int_{-\pi}^{\pi} g(x)\cos(nx) \, dx$. Discuter la valeur de $a_n$ selon la parité de $n$ (si $n$ est pair ou impair, en distinguant le cas $n=0$ déjà traité).}

		\reponse{
			Pour $n \ge 1$, $a_n = \frac{1}{\pi} \int_{-\pi}^{\pi} g(x)\cos(nx) \, dx$.
			Comme $g(x)$ est paire et $\cos(nx)$ est paire, leur produit $g(x)\cos(nx)$ est une fonction paire.
			Donc, $a_n = \frac{2}{\pi} \int_{0}^{\pi} g(x)\cos(nx) \, dx$.
			Sur $[0, \pi]$, $g(x)=1$ pour $x \in [0, \pi/2]$ et $g(x)=0$ pour $x \in ]\pi/2, \pi]$.
			$a_n = \frac{2}{\pi} \left( \int_{0}^{\pi/2} 1 \cdot \cos(nx) \, dx + \int_{\pi/2}^{\pi} 0 \cdot \cos(nx) \, dx \right) = \frac{2}{\pi} \int_{0}^{\pi/2} \cos(nx) \, dx$.
			$a_n = \frac{2}{\pi} \left[ \frac{\sin(nx)}{n} \right]_{0}^{\pi/2} = \frac{2}{\pi n} \left( \sin\left(\frac{n\pi}{2}\right) - \sin(0) \right) = \frac{2}{\pi n} \sin\left(\frac{n\pi}{2}\right)$.
			
			Discussion de la valeur de $a_n$ :
			\begin{itemize}
				\item Si $n$ est pair, $n=2k$ pour $k \in \mathbb{N}^*$.
				Alors $\sin\left(\frac{2k\pi}{2}\right) = \sin(k\pi) = 0$.
				Donc $a_{2k} = 0$ pour $k \ge 1$.
				\item Si $n$ est impair, $n=2k+1$ pour $k \in \mathbb{N}$.
				Alors $\sin\left(\frac{(2k+1)\pi}{2}\right)$ vaut $1$ si $k$ est pair (e.g., $n=1, 5, 9, \dots$) et $-1$ si $k$ est impair (e.g., $n=3, 7, 11, \dots$).
				Ceci peut s'écrire $\sin\left(\frac{(2k+1)\pi}{2}\right) = (-1)^k$.
				Donc $a_{2k+1} = \frac{2(-1)^k}{\pi (2k+1)}$ pour $k \ge 0$.
			\end{itemize}
			En résumé, pour $n \ge 1$:
			$a_n = \begin{cases} 0 & \text{si } n \text{ est pair} \\ \frac{2(-1)^{(n-1)/2}}{\pi n} & \text{si } n \text{ est impair} \end{cases}$
			Par exemple : $a_1 = \frac{2}{\pi}$, $a_2=0$, $a_3 = -\frac{2}{3\pi}$, $a_4=0$, $a_5 = \frac{2}{5\pi}$.
		}
		
		\item \question{Écrire les premiers termes non nuls de la série de Fourier de $g(x)$.}

		\reponse{
			La série de Fourier de $g(x)$ est donnée par $S_g(x) = \frac{a_0}{2} + \sum_{n=1}^{\infty} (a_n \cos(nx) + b_n \sin(nx))$.
			Avec $a_0=1$, $b_n=0$ pour tout $n$, et les $a_n$ calculés :
			$S_g(x) = \frac{1}{2} + a_1 \cos(x) + a_3 \cos(3x) + a_5 \cos(5x) + \dots$
			$S_g(x) = \frac{1}{2} + \frac{2}{\pi}\cos(x) - \frac{2}{3\pi}\cos(3x) + \frac{2}{5\pi}\cos(5x) - \dots$
			On peut écrire la série sous forme compacte :
			$S_g(x) = \frac{1}{2} + \sum_{k=0}^{\infty} a_{2k+1} \cos((2k+1)x) = \frac{1}{2} + \sum_{k=0}^{\infty} \frac{2(-1)^k}{\pi(2k+1)} \cos((2k+1)x)$.
			$S_g(x) = \frac{1}{2} + \frac{2}{\pi} \left( \cos(x) - \frac{1}{3}\cos(3x) + \frac{1}{5}\cos(5x) - \frac{1}{7}\cos(7x) + \dots \right)$.
		}
	\end{enumerate}
	
}
\uuid{5sS2}
\chapitre{Polynôme, fraction rationnelle}
\niveau{L1}
\module{Algèbre}
\sousChapitre{Autre}
\titre{Déterminant de Van Der Monde}
\theme{polynômes}
\auteur{}
\datecreate{2023-01-23}
\organisation{AMSCC}
\difficulte{}
\contenu{

\texte{ On se propose de calculer le déterminant de VAN DER MONDE :
$$
\Delta=\left|\begin{array}{cccc}
1 & 1 & 1 & 1 \\
a & b & c & x \\
a^2 & b^2 & c^2 & x^2 \\
a^3 & b^3 & c^3 & x^3
\end{array}\right|
$$ }

\begin{enumerate}
\item \question{  Quel est le degré du polynôme $P(x)=\Delta$ ? }
\indication{Exprimer, sans le calculer, le déterminant en développant par rapport à la dernière colonne. }
\reponse{ Si l'on développe le déterminant $\Delta$ par rapport à sa 4 ème colonne, on voit que $P(x)$ est un polynôme de degré au plus 3. }
\item \question{ Sans calculer $\Delta$, donner trois racines évidentes du polynôme $P(x)$. En déduire $\Delta$. }
\reponse{ Pour $x=a, \Delta=0$ car deux colonnes sont identiques. De même, pour $x=b$ et pour $x=c$. Ainsi, $a, b$ et $c$ sont trois racines évidentes de $P(x)$.
Ainsi :
$$
P(x)=k(x-a)(x-b)(x-c)
$$
où $k$ est le coefficient de $x^3$. Ainsi :
$$
\begin{aligned}
& k=\left|\begin{array}{ccc}
1 & 1 & 1 \\
a & b & c \\
a^2 & b^2 & c^2
\end{array}\right|=\left|\begin{array}{ccc}
1 & 0 & 0 \\
a & b-a & c-a \\
a^2 & b^2-a^2 & c^2-a^2
\end{array}\right| \\
& =(b-a)(c-a)\left|\begin{array}{ccc}
\mathcal{c}_1 & c_2 & c_3-c_2 \\
1 & 0 & 0 \\
a & 1 & 0 \\
a^2 & b+a & c-b
\end{array}\right| \\
& =(b-a)(c-a)(c-b) \\
&
\end{aligned}
$$
Conclusion :
$$
P(x)=(b-a)(c-a)(c-b)(x-a)(x-b)(x-c)
$$ }

\item \question{ Calculer $\left|\begin{array}{cccc}1 & 1 & 1 & 1 \\ 1 & i & -1 & -i \\ 1 & -1 & 1 & -1 \\ 1 & -i & -1 & i\end{array}\right|$. }
\indication{Utiliser les questions précédentes en remplaçant $a$, $b$ et $c$ par des valeurs judicieuses. }

\reponse{ $\left|\begin{array}{cccc}1 & 1 & 1 & 1 \\ 1 & i & -1 & -i \\ 1 & -1 & 1 & -1 \\ 1 & -i & -1 & i\end{array}\right|=P(-i)$ pour $a=1, b=i$ et $c=-1$.
Aussi :
$$
\begin{aligned}
\left|\begin{array}{cccc}
1 & 1 & 1 & 1 \\
1 & i & -1 & -i \\
1 & -1 & 1 & -1 \\
1 & -i & -1 & i
\end{array}\right| & =(i-1) \underbrace{(-1-1)}_{-2}(-1-i)(-i-1) \underbrace{(-i-i)}_{-2 i}(-i+1) \\
& =4 i \underbrace{(i-1)(-1-i)}_2 \underbrace{(-i-1)(-i+1)}_{-2} \\
& =-16 i
\end{aligned}
$$ }
\end{enumerate}}

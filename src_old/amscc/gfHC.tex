\uuid{gfHC}
\chapitre{Probabilité discrète}
\niveau{L2}
\module{Probabilité et statistique}
\sousChapitre{Probabilité et dénombrement}
\titre{Définition d'une probabilité}
\theme{probabilités}
\auteur{}
\datecreate{2023-01-24}
\organisation{AMSCC}
\difficulte{}
\contenu{

\question{ 	Déterminer une probabilité sur $\Omega  = \left\{ {1,2, \ldots ,n} \right\}$ telle que la probabilité de l'événement $\left\{ {1,2, \ldots ,k} \right\}$ soit proportionnelle à $k^2 $. }

\reponse{ Pour $k = n$, on a $\PP(\Omega) = \alpha n^2 = 1$. Donc $\alpha = \frac{1}{n^2}$. Ainsi, $\PP(\left\{ {1,2, \ldots ,k} \right\}) = \frac{k^2}{n^2}$. }
}

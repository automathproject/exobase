\uuid{bPo3}
\chapitre{Probabilité continue}
\niveau{L2}
\module{Probabilité et statistique}
\sousChapitre{Densité de probabilité}
\titre{Comparaison de probabilités}
\theme{variables aléatoires à densité}
\auteur{Maxime NGUYEN}
\datecreate{2023-09-13}
\organisation{AMSCC}

\difficulte{2}
\contenu{

\question{ Si l'on tire un réel au hasard dans $[0,1]$, est-il plus probable d'avoir sa racine carrée inférieure à $0.4$ ou d'avoir son carré plus grand que $0.7$ ? }
\indication{ Modéliser le nombre tiré au hasard par une variable aléatoire $X$ suivant une loi uniforme sur $[0;1]$. }
\reponse{
	Soit $X\sim\mathcal{U}([0,1])$ modélisant le réel tiré au hasard dans $[0,1]$. Alors
	\[\forall t\in\R, \quad f_X(t)=\begin{cases} 1 & \text{ si } t\in[0,1] \\
		0 & \text{ sinon.}
	\end{cases}\]
	On souhaite comparer les deux probabilités suivantes:
	\begin{itemize}
		\item $\prob(\sqrt{X}\leq 0.4)=\prob(X\leq 0.4^2)=\int_0^{0.16} \dx t=0.16$
		\item $\prob(X^2\geq 0.7)=\prob(X\geq \sqrt{0.7})=\int_{\sqrt{0.7}}^1  \dx t=1-\sqrt{0.7}\simeq 0.163$
	\end{itemize}
	On en conclut qu'il est plus probable d'avoir son carré plus grand que $0.7$ (par rapport à avoir sa racine carrée inférieure à $0.4$).
}
}
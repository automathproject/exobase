\uuid{7UCt}
\chapitre{Probabilité continue}
\sousChapitre{Lois des grands nombres, théorème central limite}

\titre{Surbooking militaire} %PROBA 277
\theme{variables aléatoires, théorème central limite}
\auteur{}
\datecreate{2022-09-27}
\organisation{AMSCC}
\contenu{




\texte{
Lors d'une opération de transport aérien stratégique, un avion militaire peut transporter au maximum 300 soldats. Pour optimiser l'utilisation des ressources, l'état-major décide de pratiquer du surbooking en acceptant plus de $300$ réservations. Chaque soldat a une probabilité
de $10 \%$ de ne pas se présenter à l'embarquement (pour diverses raisons opérationnelles). }
\question{Combien de réservations ($n>300$) faut-il accepter au maximum pour que la probabilité qu'il y ait plus de 300 soldats à l'embarquement soit inférieure à $10 \%$?}
\reponse{
	Soit $n$ le nombre de réservations et soit $X$ le nombre de soldats à se présenter. Alors $X\sim \mathcal{B}(n,0.9)$.
	On souhaite déterminer la valeur maximale de $n$ pour laquelle $\mathbb{P}(X>300)\leq 0.01$. \\
	Comme $n$ est grand, on peut approcher la loi de $X$ par la loi Normale $\mathcal{N}(0.9n,\sigma^2=\frac{9}{100}n)$ par le théorème de Moivre-Laplace, ce qui donne:
	\begin{align*}
	\mathbb{P}(X>300)\leq 0.01 \quad
	& \Leftrightarrow \quad 1- \mathbb{P}(X\leq 300)\leq 0.01 \\
	& \Leftrightarrow \quad \mathbb{P}(X\leq 300) \geq 0.99 \\
	& \Leftrightarrow \quad \mathbb{P}\left( \frac{X-0.9n}{\frac{3}{10}\sqrt{n}} \leq \frac{300-0.9n}{\frac{3}{10}\sqrt{n}} \right) \geq 0.99 \\
	& \Leftrightarrow \quad \mathbb{P}\left (Z \leq \frac{300-0.9n}{\frac{3}{10}\sqrt{n}}\right) \geq 0.99 \quad \text{où } Z\sim \mathcal{N}(0,1) \\
	& \Leftrightarrow \quad  \frac{300-0.9n}{\frac{3}{10}\sqrt{n}}\geq 2.33 \quad \text{ par lecture du tableau de loi.}
	\end{align*}
	On résout donc l'équation $300-\frac{9}{10}n=2.33\times \frac{3}{10}\sqrt{n}$, c'est-à-dire en posant $x^2=n$:
	\[ 9x^2+7x-3000=0.\]
	Le discriminant associé est $\Delta=7^2-4\times 9\times (-3000)=108049$ donc $\sqrt{\Delta}\simeq 328.71$ et les racines de cette équation sont les réels
	\[ x_1=\frac{-7-328.71}{2\times 9}<0 \quad \text{et} \quad x_2=\frac{-7+328.71}{2\times 9}=17.87\]
	donc $n=x_2^2 \simeq 319.43$. On conclut qu'il ne faut pas dépasser $319$ passagers pour avoir un surbooking dans moins de $1$\% des cas.
}


}













\uuid{gsn9}
\chapitre{Probabilité continue}
\niveau{L2}
\module{Probabilité et statistique}
\sousChapitre{Densité de probabilité}
\titre{Fonction de répartition}
\theme{variables aléatoires à densité}
\auteur{}
\datecreate{2023-09-13}
\organisation{AMSCC}
\difficulte{2}
\contenu{

\texte{ Soit $\lambda>0$ et $X$ une variable aléatoire admettant pour densité $f(x)=\lambda e^{-\lambda x}1_{[0;+\infty[}(x)$.  }

\begin{enumerate}
    \item \question{ Vérifier que $f$ définit bien une fonction densité, puis déterminer la fonction de répartition $F_X$ de $X$. }
    \indication{On revient à la définition : on pose $t \in \R$ quelconque et on calcule $F_X(t) = \prob(X \leq t) = \int_{-\infty}^{t} f(x) \, \mathrm{d}x$. }
    \reponse{Il suffit de vérifier que $f(x) \geq 0$ pour tout $x \in \R$ puis de calculer :
        \begin{align*}
        \int_{-\infty}^{+\infty} f(x)dx &= \int_0^{+\infty} \lambda e^{-\lambda x} dx \\
                                       &= \left[-e^{-\lambda x}\right]_0^{+\infty} \\
                                       &= 1
        \end{align*}
   On détermine maintenant la fonction de répartition : soit $t \in \R$ ;
   \begin{itemize}
       \item si $t<0$, alors $F_X(t) = \int_{-\infty}^t f(x)dx = \int_{-\infty}^t 0 dx = 0$ ;
       \item si $t \geq 0$, alors $F_X(t) = \int_{-\infty}^t f(x)dx = \int_{-\infty}^0 0 dx + \int_0^t \lambda e^{-\lambda x} dx = 0 + \left[-e^{-\lambda x}\right]_0^t = 1 - e^{-\lambda t}$.
   \end{itemize} 
   }

    \item \question{ Exprimer $\prob(-1 \leq X \leq 1)$ en fonction de $F_X$ et en déduire une valeur numérique. }
    \indication{Le résultat dépend de $\lambda$ qui est le paramètre de cette loi. }
    \reponse{
        \begin{align*}
            \prob(-1 \leq X \leq 1) &= F_X(1) - F_X(-1) \\
                                    &= (1-e^{-\lambda}) - 0 \\
                                    &= 1-e^{-\lambda}
        \end{align*}
    }
\end{enumerate}
}

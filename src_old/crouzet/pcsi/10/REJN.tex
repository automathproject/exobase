\uuid{REJN}
\niveau{PCSI}
\module{Analyse}
\chapitre{Convergence d'une suite}
\sousChapitre{Suites et limites}
\duree{20}
\difficulte{2}
\auteur{Antoine Crouzet}
\datecreate{01/12/2024}
\titre{Suites adjacentes II}
\contenu{
\texte{On définit la suite $(h_n)_{n\geq 1}$, pour $n\geq1$, par :
$ h_n=\ds{\sum_{k=1}^n \frac{1}{k}}$.}
\question{Montrer que $(h_n)$ est croissante. On note $\ell$ la limite, finie ou infinie, de $(h_n)$. Justifier son existence.}
\reponse{\begin{align*}
Pour tout $n\geq 1$, on a \[ h_{n+1}-h_n = \sum_{k=1}^{n+1} \frac1k - \sum_{k=1}^n \frac1k = \frac{1}{n+1}. \]
  La suite $(h_n)_{n\geq 1}$ est donc croissante, et admet donc une limite (finie ou infinie).
\end{align*}}
\question{Montrer que $h_{2n}-h_n\geq \frac{1}{2}$, pour tout $n\in\N*$.}
\reponse{Pour tout $n\geq 1$ :
  \begin{align*}
     h_{2n} - h_n &= \sum_{k=1}^{2n} \frac1k - \sum_{k=1}^n \frac1k \\
     &= \sum_{k=n+1}^{2n} \frac1k
    \end{align*}
    Remarquons que pour tout $k\in \interent{n+1 2n}$, on a, par décroissance de la fonction inverse sur $\R>$ :
    \[ n+1 \leq k \leq 2n \implies \frac{1}{2n} \leq \frac1k \leq \frac{1}{n+1}. \]
    Ainsi,
  \begin{align*}
h_{2n}-h_n &= \sum_{k=n+1}^{2n} \frac1k \\
&\geq \sum_{k=n+1}^{2n} \frac{1}{2n} = \frac{1}{2n}\left(2n-(n+1)+1\right) = \frac12.
  \end{align*}}
\question{En raisonnant par l'absurde, montrer que $\ell=+\infty$.}
\reponse{\begin{align*}
Supposons par l'absurde que la limite de $(h_n)$ soit finie. On la note $\ell$. Mais alors \[ h_n \tendversen{n\to +\infty} \ell \qeq h_{2n}\tendversen{n\to +\infty} \ell \]
  $(h_{2n})$ étant une suite extraite de la suite $h$, elle a la même limite. Mais alors \[ h_{2n}-h_n \tendversen{n\to +\infty} 0. \]
  Or, pour tout $n\geq 1$, $h_{2n}-h_n \geq \frac12$ : c'est absurde.

  Ainsi \[ \boxed{\lim_{n\to +\infty} h_n = +\infty.}\]
\end{align*}}
}

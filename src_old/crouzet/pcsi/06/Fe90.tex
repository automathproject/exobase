\uuid{Fe90}
\niveau{PCSI}
\module{Analyse}
\chapitre{Généralités sur les fonctions}
\sousChapitre{Études de fonctions}
%!TeX root=../../../encours.nouveau.tex
%%% Début exercice %%%

\duree{20}
\difficulte{2}
\auteur{Antoine Crouzet}
\datecreate{01/12/2024}
\titre{Des sinus et des $x$}
\contenu{
\texte{Soit $f$ la fonction définie sur $\R$ par $f(x)=2\sin(x)+x$.}
\question{Etudier $f$ sur $[-\pi;\pi]$.}
\reponse{\begin{align*}
$[-\pi;\pi]$ est symétrique par rapport à $0$. Pour tout réel $x\in [-\pi;\pi]$, on a \[f(-x)=2\sin(-x)-x=-2\sin(x)-x=-(2\sin(x)+x)=-f(x)\]
		$f$ est donc impaire. Etudions $f$ sur $[0;\pi]$. $f$ est dérivable comme somme de fonctions dérivables, et on a pour tout $x\in [0;\pi]$
		\[ f'(x) = 2\cos(x)+1 \]
		On a alors, sur $[0;\pi]$ :
		\begin{eqnarray*}
			f'(x)>0 &\Leftrightarrow&  2\cos(x)+1>0 \\
					&\Leftrightarrow& \cos(x)> -\frac{1}{2} \\
					&\Leftrightarrow& x\in \left[0;\frac{2\pi}{3}\right]
		\end{eqnarray*}
		On obtient le tableau de variations suivant :
		\begin{center}
			\input{tab.ex4}
		\end{center}
		et par imparité :
		\begin{center}
			\input{tab.ex42}
		\end{center}
\end{align*}}
\question{Montrer que la courbe de $f$ est invariante par translation de vecteur $\vv{u}\matrice{2\pi\\ 2\pi}$ (on montrera pour cela que pour tout réel $x$, $f(x+2\pi)=f(x)+2\pi$).}
\reponse{\begin{align*}
Pour tout réel $x$, on a \[ f(x+2\pi)=2\sin(x+2\pi)+x+2\pi=2\sin(x)+x+2\pi=f(x)+2\pi \]
	 Ainsi, la courbe de $f$ est invariante par translation de vecteur $\vv{u}\matrice{2\pi\\2\pi}$.
\end{align*}}
\question{Représenter $f$ sur $[-2\pi; 4\pi]$.}
\reponse{\begin{align*}
En utilisant le tableau de variations sur $[-\pi;\pi]$, puis par translation, on obtient la courbe suivante sur $[-2\pi;4\pi]$. 
	 		\begin{center}
				\input{fig.ex4}
			\end{center}
\end{align*}}
\question{Déterminer les limites de $f$ en $+\infty$ et en $-\infty$.}
\reponse{\begin{align*}
Pour tout réel $x$, on a
	\[ -2 \leq 2\sin(x) \leq 2 \quad \text{et donc} \quad -2+x \leq 2\sin(x) \leq 2+x \]
	Puisque $\ds{\lim_{x\rightarrow +\infty} -2+x = +\infty}$, par comparaison, \[ \lim_{x\rightarrow +\infty} f(x)=+\infty \]
	De même, $\ds{\lim_{x\rightarrow -\infty} 2+x = -\infty}$ et donc 
	\[ \lim_{x\rightarrow -\infty} f(x)=-\infty \]
\end{align*}}
}

%%% Fin exercice %%%

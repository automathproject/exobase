\uuid{FmPI}
\niveau{PCSI}
\module{Analyse}
\chapitre{Généralités sur les fonctions}
\sousChapitre{Généralités}
%!TeX root=../../../encours.nouveau.tex
%%% Début exercice %%%

\duree{5}
\difficulte{1}
\auteur{Antoine Crouzet}
\datecreate{01/12/2024}
\titre{Des puissances de $-1$}
\contenu{
\question{Soit $f:\Z\dans \R$ définie pour tout entier $x$ par $f(x)=(-1)^x$. Déterminer l'image de la fonction $f$, $f(\Z)$. Déterminer l'image réciproque par $f$ de $\{1\}$.}
\reponse{\begin{align*}
Par définition, $f$ prend deux valeurs : $1$ et $-1$. Donc 
\[f(\Z)=\{-1;1\}\]
Enfin, l'image réciproque de $1$ est composé de tous les éléments de $\Z$ ayant pour image $1$, c'est-à-dire les nombres pairs. Ainsi, $\ds{f^{-1}(\{1\}) = 2\Z}$.
\end{align*}}
}

%%% Fin exercice %%%

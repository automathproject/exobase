\uuid{QoU0}
\niveau{PCSI}
\module{Analyse}
\chapitre{Généralités sur les nombres réels}
%!TeX root=../../../encours.nouveau.tex
\duree{30}
\difficulte{2}
\auteur{Antoine Crouzet}
\datecreate{01/12/2024}
\titre{Caractérisation des intervalles}
\contenu{
\texte{Soit $I$ une partie non vide de $\R$ vérifiant la propriété suivante :
\[ \forall~x\in I,\,\forall~y\in I,\, x\leq y \implies \interff{x y} \subset I\]
L'objectif est de démontrer que $I$ est un intervalle.}
\question{On suppose que $I$ est majorée et non minorée.
	\begin{enumerate}
		\item Montrer qu'il existe un réel $a$ tel que $I \subset \interof{-\infty{} a}$.
		\item Soit $x\in \interoo{-\infty{} a}$. Montrer qu'il existe $y$ et $z$ dans $I$ tels que $y\leq x\leq z$.
		\item En déduire que $\interoo{-\infty{} a}\subset I$ puis que $I$ est un intervalle.
	\end{enumerate}}
\reponse{\begin{align*}
\begin{enumerate}
	\item $I$ est non vide (par hypothèse), et majorée. Elle admet une borne supérieure, que l'on note $a$. Ainsi, $I\subset \interof{-\infty{} a}$.
	\item Soit $x\in \interoo{-\infty{} a}$. Puisque $I$ n'est pas minorée, il existe $y\in I$ tel que $y\leq x$ (sinon, $I$ est minorée par $y$).

	De plus, puisque $x<a$, par définition de la borne supérieure, il existe $z\in I$ tel que $x<z\leq a$.

	Ainsi, il existe bien deux éléments $y$ et $z$ de $I$ tels que $y\leq x\leq z$.
	\item Finalement, pour tout $x\in \interoo{-\infty{} a}$, il existe $y$ et $z$ dans $I$ tels que $y\leq x\leq z$, c'est-à-dire $x\in \interff{y z}$. Or, par propriété de $I$, $\interff{y z}\subset I$, et donc $x\in I$.

	On a ainsi démontré que \[ \forall x\in \interoo{-\infty{} a},\quad x\in I \iff \interoo{-\infty{} a}\subset I. \]
	Avec les deux inclusions démontrées en a) et c), on peut en déduire que $I=\interoo{-\infty{} a}$ ou $I=\interof{-\infty{} a}$ : $I$ est bien un intervalle.
	\end{enumerate}
\end{align*}}
\question{Traiter de même les trois autres cas.}
\reponse{Les trois autres cas (majorée, minorée; minorée non majorée; non minorée non majorée) ce traite par le même principe.}
}

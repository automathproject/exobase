\uuid{SS1T}
\niveau{PCSI}
\module{Analyse}
\chapitre{Généralités sur les nombres réels}
\sousChapitre{Calculs}
%!TeX root=../../../encours.nouveau.tex
%%% Début exercice %%%

\duree{5}
\difficulte{1}
\auteur{Antoine Crouzet}
\datecreate{01/12/2024}
\titre{Fractions}
\contenu{
\question{Calculer les expressions suivantes. On donnera le résultat sous la forme d'une fraction irréductible.

\begin{align*}
A=\frac{4}{3}\times\left(\frac{13}{4}-\frac{12}{6}\right)  &\quad\quad\quad\quad B = \frac{4}{3}-1\\
C=\frac{\frac{1}{3}+2}{\frac{5}{6}-1}&\quad\quad\quad\quad D=\frac{\frac{1}{3}-\frac{1}{2}}{\frac{2}{5}+\frac{3}{8}}
\end{align*}}
\reponse{On utilise les propriétés des fractions. On obtient alors :
	\begin{align*}
		A &= \frac{4}{3}\left(\frac{39 - 24}{12}\right) \\
		   &= \frac{4}{3} \frac{15}{12}  = \frac{5}{3}
	\end{align*}

De même $\ds{B=\frac{4}{3}-\frac{3}{3}=\frac{1}{3}}$,

	\begin{align*}
		C &= \frac{ \frac{1}{3}+\frac{6}{3}}{\frac{5}{6}-\frac{6}{6}} \\
		   &= \frac{\frac{7}{3}}{-\frac{1}{6}}\\
		   &= \frac{7}{3}\times (-6) = -14
	\end{align*}
et
	\begin{align*}
		D &= \frac{\frac{2}{6}-\frac{3}{6}}{\frac{16}{40}+\frac{15}{40}}\\
		   &= \frac{-1}{6}\times \frac{40}{31} = -\frac{20}{93}
	\end{align*}}
}

%%% Début exercice %%%

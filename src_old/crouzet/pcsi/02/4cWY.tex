\uuid{4cWY}
\niveau{PCSI}
\module{Analyse}
\chapitre{Généralités sur les nombres réels}
\sousChapitre{Parties entières}
%!TeX root=../../../encours.nouveau.tex

\duree{10}
\difficulte{1}
\auteur{Antoine Crouzet}
\datecreate{01/12/2024}
\titre{Partie entière}
\contenu{
\texte{Soient $x$ et $y$ deux réels.}
\question{Calculer $\lfloor x\rfloor + \lfloor -x \rfloor$.}
\reponse{\begin{align*}
Remarquons que, si $x\in \Z$, alors $\lfloor x\rfloor + \lfloor -x\rfloor = x-x=0$.

\textbf{Première méthode} :  soit $x\notin \Z$. Par définition $\lfloor x \rfloor < x < \lfloor x\rfloor +1$, et donc \[ -\lfloor x \rfloor -1 < -x < -\lfloor x \rfloor. \]
	c'est-à-dire $\lfloor -x \rfloor = -\lfloor x \rfloor -1$, par définition de la partie entière; dans ce cas $\lfloor x \rfloor + \lfloor -x \rfloor = -1$.

\textbf{Deuxième méthode} : soit $x\notin \Z$. Par propriété de la partie entière :
	\[ x-1 < \lfloor x \rfloor < x \qeq -x-1 < \lfloor -x\rfloor < -x. \]
	En ajoutant ces inégalités
	\[ -2 < \lfloor x \rfloor + \lfloor-x \rfloor < 0. \]
	Or $\lfloor x\rfloor+\lfloor -x\rfloor \in \Z$. Le seul entier strictement compris entre $-2$ et $0$ étant $1$, on conclut que $\lfloor x\rfloor+\lfloor -x\rfloor=-1$.

	\textbf{Bilan} : \[ \forall x\in \R, \lfloor x \rfloor +\lfloor -x\rfloor = \left \{ \begin{array}{lll} 0 & \text{si} & x\in \Z\\ -1 & \text{si}& x\notin \Z\end{array}\right.. \]
\end{align*}}
\question{Montrer que $\lfloor x+y \rfloor - \lfloor x \rfloor - \lfloor y \rfloor \in \{0, 1\}$.}
\reponse{\begin{align*}
On applique le même raisonnement :
	\[ x+y-1 < \lfloor x+y \rfloor \leq x+y,\quad x-1< \lfloor x\rfloor \leq x \qeq y-1 < \lfloor y\rfloor \leq y \]
	soit
		\[ x+y-1 < \lfloor x+y \rfloor \leq x+y,\quad -x\leq  -\lfloor x\rfloor <-x+1 \qeq -y \leq  -\lfloor y\rfloor<-y+1. \]
		En ajoutant les trois inégalités :
		\[ -1 < \lfloor x+y \rfloor - \lfloor x \rfloor - \lfloor y \rfloor <  2. \]
		Puisque $\lfloor x+y \rfloor - \lfloor x \rfloor - \lfloor y \rfloor\in \Z$, on peut en déduire que \[ \lfloor x+y \rfloor - \lfloor x \rfloor - \lfloor y \rfloor \in \left \{ 0, 1\right \}. \]
\end{align*}}
\question{Montrer que pour tout entier $n\in \Z$, $\lfloor x+n \rfloor = \lfloor x \rfloor + n$.}
\reponse{\begin{align*}
Soit $n\in \Z$ et $x\in \R$. Puisque $\lfloor x\rfloor \leq x < \lfloor x\rfloor +1$, on a \[ \lfloor x\rfloor + n \leq x+n < \left(\lfloor x\rfloor +n \right)+1.\]
		Puisque $\lfloor x\rfloor + n\in \Z$, par définition de la partie entière, \[ \lfloor x+n\rfloor = \lfloor x\rfloor + n.\]
\end{align*}}
\question{Soit $n$ un entier naturel non nul. \`A quelle condition sur $x$ a-t-on $\lfloor nx \rfloor = n \lfloor x \rfloor$ ?}
\reponse{\begin{align*}
Soit $x\in \R$ tel que $\lfloor nx\rfloor = n \lfloor x\rfloor$. Cela équivaut à écrire, puisque $n\lfloor x\rfloor \in \Z$,
		\[ n\lfloor x\rfloor \leq  nx < n \lfloor x \rfloor +1 \]
    ou encore
		\[ \lfloor x \rfloor \leq x < \lfloor x \rfloor + \frac{1}{n}.\]
		soit finalement \[ 0 \leq x-\lfloor x \rfloor < \frac{1}{n}. \]
		Ainsi, $\lfloor nx\rfloor = n \lfloor x \rfloor$ si et seulement si $x\in \left [ p,\, {p+\frac{1}{n}}\right[$, avec $p\in \Z$.

		\textbf{Bilan} : \[ \boxed{\lfloor nx\rfloor = n\lfloor x\rfloor \iff x \in \bigcup_{p\in \Z} \left [ p,\, {p+\frac{1}{n}}\right[.} \]
\end{align*}}
}

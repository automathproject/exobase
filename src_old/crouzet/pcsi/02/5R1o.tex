\uuid{5R1o}
\niveau{PCSI}
\module{Analyse}
\chapitre{Généralités sur les nombres réels}
%!TeX root=../../../encours.nouveau.tex
\duree{15}
\difficulte{3}
\auteur{Antoine Crouzet}
\datecreate{01/12/2024}
\titre{Sur la somme de deux ensembles}
\contenu{
\texte{Soient $A$ et $B$ deux parties non vides de $\R$. On pose \[ A+B = \left \{a+b,\, a\in A,\, b\in B \right \}\] l'ensemble formé des réels qui s'écrivent comme la somme d'un réel de $A$ et d'un réel de $B$.

On suppose de plus que $A$ et $B$ sont majorées.}
\question{Montrer que $A+B$ admet une borne supérieure, et que $\sup(A+B)\leq \sup(A)+\sup(B)$.}
\reponse{On suppose que $A$ et $B$ sont des parties non vides et majorées. Elles admettent donc toutes les deux des bornes supérieures. On a donc, pour tout $a\in A$, $a\leq \sup(A)$ et pour tout $b\in B$, $b\leq \sup(B)$. Mais alors :
	\begin{align*}
		\forall (a,b)\in A\times B,\quad a+b &\leq \sup(A)+\sup(B)
	\end{align*}
	Donc $A+B$ est une partie non vide (car $A$ et $B$ sont non vides), majorée par $\sup(A)+\sup(B)$ : elle admet une borne supérieure. De plus, par définition de la borne supérieure, \[ \sup(A+B) \leq \sup(A)+\sup(B). \]}
\question{Montrer que $\sup(A+B)=\sup(A)+\sup(B)$.}
\reponse{\begin{align*}
Soit $\eps > 0$. Par définition de la borne supérieure :
	\begin{itemize}
		\item il existe $a_0\in A$ tel que $a_0 \geq \sup(A)-\frac{\eps}{2}$;
		\item il existe $b_0\in B$ tel que $b_0\geq \sup(B)-\frac{\eps}{2}$.
	\end{itemize}
	Mais alors $a_0+b_0 \geq \sup(A)+\sup(B) - \eps$.

	On a donc démontré que \[ \forall \eps>0,\quad \sup(A+B)\geq \sup(A)+\sup(B)-\eps \]
	c'est-à-dire $\sup(A+B)\geq \sup(A)+\sup(B)$.

	Finalement, $\sup(A+B)=\sup(A)+\sup(B)$.
\end{align*}}
}

\uuid{cjka}
\niveau{PCSI}
\module{Analyse}
\chapitre{Généralités sur les nombres réels}
\sousChapitre{Ensembles de nombres}
%!TeX root=../../../encours.nouveau.tex

\duree{5}
\difficulte{2}
\auteur{Antoine Crouzet}
\datecreate{01/12/2024}
\contenu{
\question{Montrer que $\sqrt{2}$ n'est pas rationnel.}
\reponse{\begin{align*}
On raisonne par l'absurde. On écrit $\sqrt{2}=\frac{p}{q}$, avec $p$ et $q$ deux entiers non nuls, premiers entre eux (c'est-à-dire que la fraction est irréductible). On a alors $q\sqrt{2}=p$, soit $2q^2=p^2$. $p^2$ est pair, et donc $p$ aussi d'après un résultat vu dans le cours. On peut alors écrire $p=2k$, ce qui donne \[ 2q^2 = (2k)^2 \Leftrightarrow 2q^2 = 4k^2 \Leftrightarrow q^2=2k^2 \]
Ainsi, $q^2$ est pair, et donc $q$ aussi. Mais alors, $p$ et $q$ sont pairs, et donc ne sont pas premiers entre eux : c'est absurde.

On peut conclure que $\sqrt{2}$ est irrationnel.
\end{align*}}
}

\uuid{xcLe}
\niveau{PCSI}
\module{Analyse}
\chapitre{Généralités sur les suites}
\sousChapitre{Suites et récurrence}
\duree{20}
\difficulte{1}
\auteur{Antoine Crouzet}
\datecreate{01/12/2024}
\titre{Suite récurrente II}
\contenu{
\texte{Soit $(u_n)$ la suite définie par $u_0=1$ et $u_{n+1}=u_n+2n+3$.}
\question{Etudier la monotonie de la suite $(u_n)$.}
\reponse{\begin{align*}
Constatons que pour tout entier $n$, on a
		$$u_{n+1}-u_n=(u_n+2n+3)-u_n=2n+3$$
		et $2n+3>0$ pour tout entier $n$.\\\textbf{Bilan} : la suite $(u_n)$ est strictement croissante.
\end{align*}}
\question{Démontrer par récurrence que $$\forall~n,~u_n> n^2$$}
\reponse{\begin{align*}
Soit $P$ la proposition définie pour tout entier $n$ par $P_n:$ ~``$u_n>n^2$''.
		\begin{itemize}
			\item[$\circ$] Pour $n=0$, $u_0=1$ et $1>0^2$. La proposition $P_0$ est donc vraie.
			\item[$\circ$] Supposons la proposition $P_n$ vraie pour un certain entier $n$. Montrons que $P_{n+1}$ est vraie.\\
				Par hypothèse de récurrence, $u_n>n^2$. Mais alors
				$$u_n+2n+3 > n^2+2n+3$$
				soit,
				$$u_{n+1} > (n+1)^2+2 > (n+1)^2$$
				La proposition $P_{n+1}$ est donc vraie
		\end{itemize}
		D'après le principe de récurrence, la proposition $P_n$ est vraie pour tout entier $n$.
		\\\textbf{Bilan} : $\forall~n, ~u_n>n^2$.
\end{align*}}
\question{Conjecturer, puis démontrer, une expression de $u_n$ en fonction de $n$.}
\reponse{\begin{align*}
Après calcul des premières valeurs, il semblerait que $u_n=(n+1)^2$. Montrons-le par récurrence.
	\\Soit $P$ la proposition définie pour tout entier $n$ par $P_n:$ ~``$u_n=(n+1)^2$''.
		\begin{itemize}
			\item[$\circ$] Pour $n=0$, $u_0=1$ et $1=(0+1)^2$. La proposition $P_0$ est donc vraie.
			\item[$\circ$] Supposons la proposition $P_n$ vraie pour un certain entier $n$. Montrons que $P_{n+1}$ est vraie.\\
				Par hypothèse de récurrence, $u_n=(n+1)^2$. Mais alors
				$$u_{n+1}=u_n+2n+3=\underbrace{(n+1)^2}_{\textrm{H.R.}}+2n+3=n^2+2n+1+2n+3=n^2+4n+4=(n+2)^2$$
				La proposition $P_{n+1}$ est donc vraie
		\end{itemize}
		D'après le principe de récurrence, la proposition $P_n$ est vraie pour tout entier $n$.
		\\\textbf{Bilan} : $\forall~n,~u_n=(n+1)^2$.
\end{align*}}
}

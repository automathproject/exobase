\uuid{mGNr}
\niveau{PCSI}
\module{Analyse}
\chapitre{Généralités sur les suites}
\sousChapitre{Suites et récurrence}
\duree{15}
\difficulte{1}
\auteur{Antoine Crouzet}
\datecreate{01/12/2024}
\titre{Suite récurrente I}
\contenu{
\texte{Soit $(u_n)$ la suite définie par $u_0=1$ et pour tout $n$, $$u_{n+1}=\sqrt{2+u_n}$$}
\question{Démontrer par récurrence que pour tout $n$, $0 \leq u_n \leq 2$.}
\reponse{\begin{align*}
Soit $P$ la proposition définie pour tout entier $n$ par $P_n:$ ~``$0\leq u_n\leq 2$''.
		\begin{itemize}
			\item[$\circ$] Pour $n=0$, $u_0=1$ et $0\leq 1 \leq 2$. La proposition $P_0$ est donc vraie.
			\item[$\circ$] Supposons la proposition $P_n$ vraie pour un certain entier $n$. Montrons que $P_{n+1}$ est vraie.\\
				Par hypothèse de récurrence, $0\leq u_n \leq 2$. Mais alors
				$$2 \leq 2+u_n \leq 4$$
				soit, en appliquant la fonction racine qui est croissante sur $\R^+$,
				$$\sqrt{2}\leq \sqrt{2+u_n} \leq \sqrt{4}$$
				c'est-à-dire
				$$0 \leq \sqrt{2} \leq u_{n+1} \leq 2$$
				La proposition $P_{n+1}$ est donc vraie
		\end{itemize}
		D'après le principe de récurrence, la proposition $P_n$ est vraie pour tout entier $n$.
		\\\textbf{Bilan} : $\forall~n,~0\leq u_n\leq 2$.
\end{align*}}
\question{Démontrer par récurrence que $(u_n)$ est croissante.}
\reponse{\begin{align*}
Soit $Q$ la proposition définie pour tout entier $n$ par $Q_n:$ ~``$u_n\leq u_{n+1}$''.
		\begin{itemize}
			\item[$\circ$] Pour $n=0$, $u_0=1$ et $u_1=\sqrt{3}$. On a donc $1<\sqrt{3}$ : la proposition $Q_0$ est donc vraie.
			\item[$\circ$] Supposons la proposition $Q_n$ vraie pour un certain entier $n$. Montrons que $Q_{n+1}$ est vraie.\\
				Par hypothèse de récurrence, $u_n\leq u_{n+1}$. Mais alors
				$$2+u_n\leq 2+u_{n+1}$$
				soit, en appliquant la fonction racine qui est croissante sur $\R^+$,
				$$\sqrt{2+u_n}\leq \sqrt{2+u_{n+1}}$$
				c'est-à-dire
				$$u_{n+1}\leq u_{n+2}$$
				La proposition $Q_{n+1}$ est donc vraie
		\end{itemize}
		D'après le principe de récurrence, la proposition $Q_n$ est vraie pour tout entier $n$.
		\\\textbf{Bilan} : la suite $(u_n)$ est croissante.
\end{align*}}
}

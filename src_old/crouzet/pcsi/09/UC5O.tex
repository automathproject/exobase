\uuid{UC5O}
\niveau{PCSI}
\module{Analyse}
\chapitre{Généralités sur les suites}
\sousChapitre{Suites et récurrence}
\duree{15}
\difficulte{1}
\auteur{Antoine Crouzet}
\datecreate{01/12/2024}
\titre{Suite récurrente III}
\contenu{
\texte{On considère la suite $(u_n)$ définie par
$$\left\{\begin{array}{rcl}u_0&=&2\\\forall~n,~u_{n+1}&=&\sqrt{4u_n-3}\end{array}\right.$$}
\question{Montrer que pour tout $n$, $1\leq u_n \leq 3$.}
\reponse{\begin{align*}
Soit $P$ la proposition définie pour tout entier $n$ par $P_n:$ ~``$1\leq u_n\leq 3$''.
		\begin{itemize}
			\item[$\circ$] Pour $n=0$, $u_0=2$ et $1\leq 2 \leq 3$. La proposition $P_0$ est donc vraie.
			\item[$\circ$] Supposons la proposition $P_n$ vraie pour un certain entier $n$. Montrons que $P_{n+1}$ est vraie.\\
				Par hypothèse de récurrence, $1\leq u_n \leq 3$. Mais alors
			 $4 \leq 4u_n \leq 12$ puis $1\leq 4u_n-3\leq 9$
				soit, en appliquant la fonction racine qui est croissante sur $\R^+$,
				$$\sqrt{1}\leq \sqrt{4u_n-3} \leq \sqrt{9}$$
				c'est-à-dire
				$$1 \leq u_{n+1} \leq 3$$
				La proposition $P_{n+1}$ est donc vraie
		\end{itemize}
		D'après le principe de récurrence, la proposition $P_n$ est vraie pour tout entier $n$.
		\\\textbf{Bilan} : $\forall~n,~1\leq u_n\leq 3$.
\end{align*}}
\question{Démontrer que la suite $(u_n)$ est croissante.}
\reponse{\begin{align*}
Soit $Q$ la proposition définie pour tout entier $n$ par $Q_n:$ ~``$u_n\leq u_{n+1}$''.
		\begin{itemize}
			\item[$\circ$] Pour $n=0$, $u_0=2$ et $u_1=\sqrt{7}$. On a donc $2<\sqrt{7}$ : la proposition $Q_0$ est donc vraie.
			\item[$\circ$] Supposons la proposition $Q_n$ vraie pour un certain entier $n$. Montrons que $Q_{n+1}$ est vraie.\\
				Par hypothèse de récurrence, $u_n\leq u_{n+1}$. Mais alors
				$$4u_n-3\leq 4u_{n+1}-3$$
				soit, en appliquant la fonction racine qui est croissante sur $\R^+$,
				$$\sqrt{4u_n-3}\leq \sqrt{4u_{n+1}-3}$$
				c'est-à-dire
				$$u_{n+1}\leq u_{n+2}$$
				La proposition $Q_{n+1}$ est donc vraie
		\end{itemize}
		D'après le principe de récurrence, la proposition $Q_n$ est vraie pour tout entier $n$.
		\\\textbf{Bilan} : la suite $(u_n)$ est croissante.
\end{align*}}
}

\uuid{vuIE}
\niveau{PCSI}
\module{Analyse}
\chapitre{Sommes et produits de réels}
%!TeX root=../../../encours.nouveau.tex
%%% Début exercice %%%

\duree{10}
\difficulte{1}
\auteur{Antoine Crouzet}
\datecreate{01/12/2024}
\titre{Une somme combinatoire}
\contenu{
\question{Soient $n$ et $p$ deux entiers naturels tels que $p\leq n$. Montrer que \[ \sum_{i=p}^n \binom{i}{p} = \binom{n+1}{p+1} \]}
\reponse{On fixe $p$. On démontre le résultat par récurrence sur $n$, pour $n\geq p$.
\begin{itemize}
  \item \textbf{Initialisation} : pour $n=p$, la somme vaut $\binom{p}{p}=1$ et $\binom{p+1}{p+1}=1$ : la proposition est donc vérifiée pour $n=p$.
  \item \textbf{Hérédité} : supposons que l'égalité est vraie pour un certain entier $n\geq p$ fixé. Alors :
  \begin{align*}
    \sum_{i=p}^{n+1} \binom{i}{p} &= \sum_{i=p}^n \binom{i}{p}+\binom{n+1}{p} \\
    &= \binom{n+1}{p+1}+\binom{n+1}{p} \text{ par hypothèse de récurrence}\\
    &= \binom{n+2}{p+1} \text{ par la formule de Pascal}
  \end{align*}
  Ainsi, la proposition est vérifiée au rang $n+1$.
\end{itemize}
Conclusion : on a bien montré par récurrence que pour tout entier $n\geq p$, on a \[ \sum_{i=p}^n \binom{i}{p} = \binom{n+1}{p+1} \]}
}

%%% Fin exercice %%%

\uuid{fRkl}
\niveau{PCSI}
\module{Analyse}
\chapitre{Sommes et produits de réels}
\sousChapitre{Sommes doubles}
%!TeX root=../../../encours.nouveau.tex
%%% Début exercice %%%

\duree{20}
\difficulte{3}
\auteur{Antoine Crouzet}
\datecreate{01/12/2024}
\titre{Des sommes plus compliquées}
\contenu{
\question{Soit $n$ un entier naturel non nul. Calculer les sommes suivantes :
\[ \sum_{1\leq i\leq j\leq n} \frac{i}{1+j} \quad \quad \sum_{1\leq i\leq j\leq n} \max(i,j)\quad\quad \sum_{1\leq i,j\leq n} \max(i,j)\quad\quad \sum_{1\leq i\leq j\leq n} 3^{-|i-j|} \]}
\reponse{On va appliquer le théorème de Fubini, en essayant de simplifier :
\begin{align*}
 \boxed{\sum_{1\leq i \leq j \leq n} \frac{i}{1+j}} &= \sum_{j=1}^n \sum_{i=1}^j \frac{i}{1+j} \\
 &= \sum_{j=1}^n \frac{1}{1+j} \sum_{i=1}^j i \\
 &= \sum_{j=1}^n \frac{1}{1+j} \frac{j(j+1)}{2} \\
 &= \sum_{j=1}^n \frac{j}{2} = \boxed{\frac{n(n+1)}{4}}
\end{align*}
Pour la deuxième, on utilise le fait que $\max(i,j)= j$ si $j\geq i$ :
\begin{align*}
 \boxed{\sum_{1\leq i\leq j\leq n} \max(i,j)} & \sum_{j=1}^n \sum_{i=1}^j \max(i,j)\\
 &= \sum_{j=1}^n \sum_{i=1}^j j = \sum_{j=1}^n j\times j \\
 &=\sum_{j=1}^n j^2 = \boxed{\frac{n(n+1)(2n+1)}{6}}
\end{align*}
Pour la troisième, on écrit par sommation par parquet puisque $1\leq i,j\leq n$ si et seulement si $1\leq i\leq j\leq n$ ou $1\leq i < j \leq n$ :
\begin{align*}
	\boxed{\sum_{1\leq i,j\leq n} \max(i,j)}&= \sum_{1\leq i\leq j\leq n} \max(i,j) + \sum_{1\leq j < i \leq n} \max(i,j)	\\
	&= \frac{n(n+1)(2n+1)}{6} + \sum_{i=2}^n \sum_{j=1}^{i-1} i \\
	&= \frac{n(n+1)(2n+1)}{6} + \sum_{i=2}^n i(i-1) \\
	&= \frac{n(n+1)(2n+1)}{6} + \sum_{i=1}^n i(i-1) \text{ car le terme est nul pour } i=1\\
	&= \frac{n(n+1)(2n+1)}{6} + \left( \frac{n(n+1)(2n+1)}{6} - \frac{n(n+1)}{2}\right) \\
	&= n(n+1)\left( \frac{2n+1}{3}- \frac{1}{2}\right) = \boxed{\frac{n(n+1)(4n-1)}{6}} 
\end{align*}

De même, $|i-j|=j-i$ si $j\geq i$ :
\begin{align*}
 \boxed{\sum_{1\leq i\leq j\leq n} 3^{-|i-j|}} &= \sum_{j=1}^n \sum_{i=1}^j 3^{-|i-j|} = \sum_{j=1}^n \sum_{i=1}^j 3^{-(j-i)} \\
 &= \sum_{j=1}^n 3^{-j}\sum_{i=1}^j 3^i = \sum_{j=1}^n 3^{-j} \frac{3-3^{j+1}}{1-3} \\
 &= \frac12 \sum_{j=1}^n \left(3-3^{-j+1}\right)\\
 &= \frac12 \left( \sum_{j=1}^n 3 - 3\sum_{j=1}^n \left(\frac13\right)^j \right)\\
 &= \frac12 \left( 3n - 3\frac{\frac{1}{3}-\left(\frac13\right)^{n+1}}{1-\frac13} \right) = \boxed{\frac{1}{2}\left(3n-\frac{3}{2}\left(1-\left(\frac13\right)^n\right)\right)}
\end{align*}}
}

%%% Fin exercice %%%

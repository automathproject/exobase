\uuid{iz4b}
\niveau{PCSI}
\module{Analyse}
\chapitre{Sommes et produits de réels}
\sousChapitre{Factorielle et coefficients binomiaux}
%!TeX root=../../../encours.nouveau.tex
%%% Début exercice %%%

\duree{10}
\difficulte{1}
\auteur{Antoine Crouzet}
\datecreate{01/12/2024}
\titre{Retour de l'inégalité de Bernoulli}
\contenu{
\question{Montrer, sans récurrence, l'inégalité de Bernoulli dans le cas où $q\ge 0$ :
\[ \forall~n\in \N*,\, \forall~x\in \R+,\quad (1+x)^n\geq 1+nx \]}
\reponse{D'après la formule du binôme de Newton :
\begin{align*}
  (1+x)^n &= \sum_{k=0}^n \binom{n}{k}x^k 1^{n-k} \\
  &= \sum_{k=0}^n \binom{n}{k} x^k \\
  &= \binom{n}{0}x^0 +\binom{n}{1} x^1 +\sum_{k=2}^n \binom{n}{k} x^k \\
  &= 1+nx + \sum_{k=2}^n \binom{n}{k} x^k  \geq 1+nx
\end{align*}
cette dernière inégalité étant vraie car $x\geq 0$ et les coefficients binomiaux sont positifs, donc la somme elle-même est positive.}
}

%%% Fin exercice %%%

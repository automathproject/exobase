\uuid{uFXe}
\niveau{PCSI}
\module{Analyse}
\chapitre{Sommes et produits de réels}
\sousChapitre{Manipulations et calculs}
%!TeX root=../../../encours.nouveau.tex
%%% Début exercice %%%

\duree{15}
\difficulte{2}
\auteur{Antoine Crouzet}
\datecreate{01/12/2024}
\titre{Une somme plus difficile}
\contenu{
\texte{Soit $n$ un entier naturel non nul.}
\question{Exprimer, en fonction de $n$, le plus grand entier naturel $m$ tel que $2m\leq n$ et le plus grand entier naturel $p$ tel que $2p+1\leq n$.}
\reponse{\begin{align*}
Soit $n\in \N$. Par définition, $m=\left [ \dfrac{n}{2}\right ]$ est le plus grand entier tel que $m\leq \dfrac{n}{2}$, soit $2m\leq n$, et $p=\left [ \dfrac{n-1}{2}\right ]$ est le plus grand entier tel que $p\leq  \dfrac{n-1}{2}$, soit $2p+1\leq n$.
\end{align*}}
\question{En déduire la valeur de $\ds{\sum_{k=1}^n (-1)^k k}$.}
\reponse{On sépare, dans la somme, termes pairs et termes impairs. On fixe $n$ et on note $m$ et $p$ tels que définis dans la question précédente. Remarquons que si $n$ est pair, $p = m-1$ et si $n$ est impair, $m=p$.
	
	Alors : 
	\begin{align*}
		\sum_{k=1}^n (-1)^k k &= \sum_{\substack{k\in \ll 1, n\rr\\k \text{ pair}}} (-1)^k k + \sum_{\substack{k\in \ll 1, n\rr\\k \text{ impair}}} (-1)^k k \\
		&= \sum_{i=1}^{m} 2i - \sum_{i=0}^{p} \left(2i+1\right)\\
		&= \left \{ {\def\arraystretch{2.2}\begin{array}{rcl} \sum\limits_{i=1}^m 2i - \sum\limits_{i=0}^{m-1} (2i+1)& \text{ si }& n\text{ est pair}\\\sum\limits_{i=1}^p 2i - \sum\limits_{i=0}^{p} (2i+1)& \text{ si }& n\text{ est impair}\end{array} } \right.\\
		&= \left \{ {\def\arraystretch{2.2}\begin{array}{lcl} \sum\limits_{i=1}^{m-1} 2i + 2m - \left(1 + \sum\limits_{i=1}^{m-1} (2i+1)\right)& \text{ si }& n\text{ est pair}\\\sum\limits_{i=1}^p 2i - \left(1+\sum\limits_{i=1}^{p} (2i+1)\right)& \text{ si }& n\text{ est impair}\end{array} } \right.\\
		&= \left \{  \begin{array}{lcl} 2m- 1-\sum\limits_{i=1}^{m-1} 1 = m &\text{ si }& n\text{ est pair} \\ -1-\sum\limits_{i=1}^p 1 = -p-1 &\text{ si }& n\text{ est pair}\end{array}\right.
	\end{align*}
	En remarquant que si $n$ est pair, $m=\left [ \dfrac{n}{2}\right]$ et si $n$ est impair, $p=\left[ \dfrac{n-1}{2}\right]=\left[ \dfrac{n}{2} \right]$, on en déduit le résultat :
	\[ \boxed{\sum_{k=1}^n (-1)^k k = \left[\frac{n}{2}\right] \text{ si $n$ est pair, } -\left[\frac{n}{2}\right]-1 \text{ si $n$ est impair.}}\]}
}

%%% Fin exercice %%%

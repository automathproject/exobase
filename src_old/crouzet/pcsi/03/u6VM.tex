\uuid{u6VM}
\niveau{PCSI}
\module{Analyse}
\chapitre{Sommes et produits de réels}
%!TeX root=../../../encours.nouveau.tex
%%% Début exercice %%%

\duree{15}
\difficulte{2}
\auteur{Antoine Crouzet}
\datecreate{01/12/2024}
\titre{Formule de Koenig-Huygens}
\contenu{
\question{Soient $x_1, \hdots, x_n$ des réels. On note $\ds{m=\frac{x_1+\hdots+x_n}{n}}$ la moyenne des $(x_i)$.

Montrer la formule de Koenig-Huygens :
\[ \frac{1}{n}\sum_{i=1}^n \left(x_i-m\right)^2 = \left(\frac{1}{n}\sum_{i=1}^n x_i^2\right) - m^2 \]}
\reponse{On va développer, en constatant que $\sum\limits_{i=1}^n x_i = n\times m$.
\begin{align*}
 \frac{1}{n}\sum_{i=1}^n \left(x_i-m\right)^2 &= \frac{1}{n}\sum_{i=1}^n \left(x_i^2 - 2x_im+m^2\right) \\
 &= \left( \frac{1}{n}\sum_{i=1}^n x_i^2\right) - \frac{2m\sum\limits_{i=1}^n x_i}{n} + \frac{1}{n}\sum_{i=1}^n m^2 \\
 &= \left( \frac{1}{n}\sum_{i=1}^n x_i^2\right) - \frac{2m (nm)}{n} + \frac{1}{n}\left(n m^2\right) \\
 &=  \left( \frac{1}{n}\sum_{i=1}^n x_i^2\right) - m^2.
\end{align*}}
}

%%% Fin exercice %%%

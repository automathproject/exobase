\uuid{8quA}
\niveau{PCSI}
\module{Analyse}
\chapitre{Sommes et produits de réels}
%!TeX root=../../../encours.nouveau.tex
%%% Début exercice %%%

\duree{20}
\difficulte{2}
\auteur{Antoine Crouzet}
\datecreate{01/12/2024}
\titre{Inégalité de Cauchy-Schwarz}
\contenu{
\texte{Soit $n$ un entier naturel non nul. Soient $x_1, \hdots, x_n, y_1, \hdots y_n$ des réels.}
\question{Justifier que, pour tout $t\in \R$, \[ \sum_{i=1}^n \left(|x_i|+t|y_i|\right)^2 = at^2+bt+c \]
	où $a, b$ et $c$ sont trois réels à exprimer en fonction des $(x_i)$ et des $(y_i)$.}
\reponse{Soit $t\in \R$. On développe : \[ \left(|x_i|+t|y_i|\right)^2 = |x_i|^2+2|x_i|\cdot|y_i|t + t^2 |y_i|^2 = x_i^2+2|x_iy_i|t + t^2 y_i^2 \]
	soit, en sommant
	\begin{align*}
		\sum_{i=1}^n \left(|x_i|+t|y_i|\right)^2 &= \sum_{i=1}^n x_i^2+2|x_iy_i|t + t^2 y_i^2 \\
		&= \sum_{i=1}^n x_i^2 + 2t \sum_{i=1}^n |x_iy_i| + t^2\sum_{i=1}^n y_i^2\\
		&= at^2+bt+c
	\end{align*}
	avec $\ds{a=\sum_{i=1}^n y_i^2,\, b=2\sum_{i=1}^n |x_iy_i|,\,\text{et}\,x=\sum_{i=1}^n x_i^2}$.}
\question{En déterminant le signe du trinôme du second degré précédent de deux manières différentes, montrer que \[ \sum_{i=1}^n \left|x_iy_i\right| \leq \sqrt{\sum_{i=1}^n x_i^2} \sqrt{\sum_{i=1}^n y_i^2} \]
	 Cette inégalité est appelée \textbf{inégalité de Cauchy-Schwarz}.}
\reponse{\begin{align*}
Tout d'abord, puisque $|x_i+ty_i|^2\geq 0$, par somme, \[ \sum_{i=1}^n \left( |x_i|+t|y_i|\right)^2 \geq 0 \]
	Ainsi, le trinôme du second degré est de signe constant. Cela signifie que son discriminant $\Delta$ est négatif. Or
	\[ \Delta = b^2-4ac = \left(2\sum_{i=1}^n |x_iy_i|\right)^2-4\sum_{i=1}^n x_i^2 \sum_{i=1}^n y_i^2 \]
	et cela donne
	\[ \left(\sum_{i=1}^n |x_iy_i|\right)^2 \leq \sum_{i=1}^n x_i^2 \sum_{i=1}^n y_i^2 \]
	puis, en appliquant la fonction racine croissante sur $\R+$, et par positivité des termes

	\[ \sum_{i=1}^n |x_iy_i| \leq \sqrt{\sum_{i=1}^n x_i^2}\sqrt{\sum_{i=1}^n y_i^2}. \]
\end{align*}}
\question{En appliquant la précédente inégalité à des réels bien choisis, montrer que \[ \frac{6n}{(n+1)(2n+1)} \leq \sum_{k=1}^n \frac{1}{k^2} \]}
\reponse{Appliquons l'inégalité de Cauchy-Schwarz à $x_i=\frac{1}{i}$ et $y_i=i$, pour tout $i\in \interent{1 n}$. Cela donne :
	\[ \sum_{i=1}^n \left|\frac{1}{i}i\right| \leq \sqrt{\sum_{i=1}^n \frac{1}{i^2}}\sqrt{\sum_{i=1}^n i^2}. \]
	 Or :
	 \begin{align*}
   \sum_{i=1}^n 1 &= n \\
	 \text{et }\sum_{i=1}^n i^2 &= \frac{n(n+1)(2n+1)}{6}
	 \end{align*}
et l'inégalité devient
\[ n \leq \sqrt{\sum_{i=1}^n \frac{1}{i^2}} \sqrt{\frac{n(n+1)(2n+1)}{6}} \]
soit, en appliquant la fonction carré, croissante sur $\R+$
\[ n^2 \leq  \frac{n(n+1)(2n+1)}{6}\sum_{i=1}^n \frac{1}{i^2} \]
et finalement
\[ \frac{6n}{(n+1)(2n+1)} \leq \sum_{i=1}^n \frac{1}{i^2}. \]}
}

%%% Fin exercice %%%

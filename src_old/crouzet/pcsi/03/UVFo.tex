\uuid{UVFo}
\niveau{PCSI}
\module{Analyse}
\chapitre{Sommes et produits de réels}
\sousChapitre{Factorielle et coefficients binomiaux}
%!TeX root=../../../encours.nouveau.tex
%%% Début exercice %%%

\duree{10}
\difficulte{1}
\auteur{Antoine Crouzet}
\datecreate{01/12/2024}
\titre{Formule du binôme}
\contenu{
\question{Calculer \[\sum_{k=0}^n \binom{n}{k}~~~~~\sum_{k=0}^n \binom{n}{k} x^k(1-x)^{n-k}~~~~~\sum_{k=0}^n (-1)^k \binom{n}{k}~~~~~\sum_{k=0}^n \binom{n}{k} 2^k\]}
\reponse{Dans tous les cas, on utilise la formule du binôme de Newton avec des réels bien choisis :
\begin{align*}
2^n = (1+1)^n &= \sum_{k=0}^n \binom{n}{k} 1^{n-k} 1^k = \sum_{k=0}^n \binom{n}{k}\\
1 = 1^n = ((1-x)+x)^n &= \sum_{k=0}^n \binom{n}{k} (1-x)^{n-k} x^k\\
\text{si } n\geq 1,\quad 0=(1-1)^n &= \sum_{k=0}^n \binom{n}{k} 1^{n-k} (-1)^k= \sum_{k=0}^n \binom{n}{k} (-1)^k\\
3^n = (1+2)^n &= \sum_{k=0}^n \binom{n}{k} 1^{n-k} 2^k = \sum_{k=0}^n \binom{n}{k} 2^k
\end{align*}}
}

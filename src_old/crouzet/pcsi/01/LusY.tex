\uuid{LusY}
\niveau{PCSI}
\module{Analyse}
\chapitre{Éléments de logique et raisonnement}
\sousChapitre{Logiques}
%!TeX root=../../../encours.nouveau.tex

\duree{5}
\difficulte{1}
\auteur{Antoine Crouzet}
\datecreate{01/12/2024}
\titre{Sur la conjonction}
\contenu{
\question{Démontrer l'idempotence, la commutativité et l'associativité de la conjonction, à l'aide de tables de vérité.}
\reponse{\begin{align*}
En écrivant les tables de vérités :
\begin{center}\[ \begin{array}{|c|c|}\hline
 \AA & \AA \et \AA  \\\hline
 V & V  \\
 F & F  \\\hline
 \end{array}\]

   \[\begin{array}{|c|c|c|c|}\hline
   \AA & \BB & \AA \et \BB & \BB \et \AA  \\\hline
   V & V & V & V \\
   V & F & F & F \\
   F & V & F & F \\
   F & F & F & F \\\hline
   \end{array}
  	\]\end{center} et
    \[ \begin{array}{|c|c|c|c|>{\columncolor{gray!20}}c|c|>{\columncolor{gray!20}}c|}\hline
     \AA & \BB & \CC & \AA \et \BB & (\AA \et \BB) \et \CC & \BB \et \CC & \AA \et (\BB \et \CC)   \\\hline
     V & V & V & V & V & V & V\\
     V & V & F & V & F & F & F\\
     V & F & V & F & F & F & F\\
     V & F & F & F & F & F & F\\
     F & V & V & F & F & V & F\\
     F & V & F & F & F & F & F\\
     F & F & V & F & F & F & F\\
     F & F & F & F & F & F & F\\\hline
     \end{array}
    	\]
\end{align*}}
}

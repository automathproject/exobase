\uuid{2LMS}
\niveau{PCSI}
\module{Analyse}
\chapitre{Ensembles et applications}
\sousChapitre{Ensemble}
%!TeX root=../../../encours.nouveau.tex
%%% Début exercice %%%
\duree{5}
\difficulte{1}
\auteur{Antoine Crouzet}
\datecreate{01/12/2024}
\titre{Ensemble des parties}
\contenu{
\question{Déterminer les éléments de $\partie\left(\{0, 1, 2\}\right)$ et de $\partie \left( \{A, C, G, T\}\right )$.}
\reponse{On doit déterminer tous les sous-ensembles de $\{0,1,2\}$ :
\begin{itemize}[label=\textbullet]
    \item à $0$ élément, il n'y a que $\vide$.
    \item à $1$ éléments, il y a $\{0\}, \{1\}$ et $\{2\}$.
    \item à $2$ éléments, il y a $\{0,1\}, \{0,2\}$ et $\{1,2\}$.
    \item à $3$ éléments, il n'y a que $\{0,1,2\}$.
\end{itemize}
Ainsi,
\[\partie\left(\{0,1,2\}\right)= \left \{
\vide,
\{0\}, \{1\}, \{2\},
\{0,1\}, \{0,2\}, \{1,2\},
\{0,1,2\}
\right \}
\]
On obtient bien $2^3=8$ sous ensembles.

De la même manière :
\begin{align*}
 \partie\left( \{A, C, G, T\}\right ) =&\left \{ \vide, \{A\}, \{C\},\{G\}, \{T\}, \{A, C\},\{A,G\},\{A,T\},\{C,G\}, \{C,T\}, \right.\\&\left. \{G,T\}, \{A,C,G\}, \{A,C,T\}, \{A,G,T\}, \{C,G,T\}, \{A,C,G,T\} \right \}.
\end{align*}}
}

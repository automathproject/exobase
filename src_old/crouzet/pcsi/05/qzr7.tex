\uuid{qzr7}
\niveau{PCSI}
\module{Analyse}
\chapitre{Ensembles et applications}
\sousChapitre{Fonctions}
%!TeX root=../../../encours.nouveau.tex
%%% Début exercice %%%

\duree{10}
\difficulte{2}
\auteur{Antoine Crouzet}
\datecreate{01/12/2024}
\titre{Des ensembles en bijection}
\contenu{
\texte{Montrer que les ensembles suivants sont en bijection, en explicitant une telle bijection :}
\question{$\interff{0 1}$ et $\interff{a b}$.}
\reponse{Pour chacun des cas, il faut d'une part déterminer une fonction puis montrer qu'elle est bijective.
\begin{enumerate}
	\item Soit $a<b$. Prenons $f:\interff{0 1} \to \interff{a b}$ définie par $f:x\mapsto (b-a)x+a$. Alors $f$ est strictement croissante (car $b-a>0$) donc injective. De plus, soit $y\in \interff{a b}$ :
	\begin{align*}
		(b-a)x+a = y &\Leftrightarrow (b-a)x = y-a \\
		&\Leftrightarrow x =\frac{y-a}{b-a} \in \interff{0 1} \text{ car } y\in \interff{a b}
	\end{align*}
	\item Soit $f:}
\question{$\interof{0 1}$ et $\interfo{0 +\infty}$.}
\reponse{\interof{0 1} \to \interfo{0 +\infty}$ définie par $f:x\mapsto \frac{1}{x}-1$. $f$ est bijective; en effet, soit $y\in \interfo{0 +\infty}$ :
	\begin{align*}
		f(x)=y &\Leftrightarrow \frac{1}{x}-1=y\\
		&\Leftrightarrow \frac{1}{x}=y+1\\
		&\Leftrightarrow x=\frac{1}{y+1} \text{ car $y\geq 0$}
	\end{align*}
	On remarque que, si $y\in \interfo{0 +\infty}$, $y+1\in \interfo{1 +\infty}$ et donc $\frac{1}{y+1} \in \interof{0 1}$. $f$ est bien bijective.
	\item Prenons $f:}
\question{$\N$ et l'ensemble des entiers naturels pairs.}
\reponse{\begin{align*}
\N \to \{2p,\,p\in
\end{align*}}
\question{$\N$ et l'ensemble des entiers naturels impairs.}
\reponse{\begin{align*}
\N\}$ définie par $f(n)=2n$. $f$ est bijective. Soit $y\in \{2p,\,p\in \N\}$; on écrit $y=2p$ pour $p\in \N$. Alors \[ f(n)=y \Leftrightarrow 2n=2p \Leftrightarrow n=p \in \N \]
	Ainsi, $f$ est bijective.
	\item De la même manière, on introduit $f:\N \to \{2p+1,\,p\in \N\}$ définie par $f(n)=2n+1$. 
\end{enumerate}
\end{align*}}
}

%%% Fin exercice %%%

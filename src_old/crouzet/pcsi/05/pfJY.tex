\uuid{pfJY}
\niveau{PCSI}
\module{Analyse}
\chapitre{Ensembles et applications}
\sousChapitre{Ensemble}
%!TeX root=../../../encours.nouveau.tex
%%% Début exercice %%%

\duree{5}
\difficulte{1}
\auteur{Antoine Crouzet}
\datecreate{01/12/2024}
\titre{Système hexadécimal}
\contenu{
\question{Soit $E=\left \{ 0, 1,2,3,4,5,6,7,8,9,A,B,C,D,E,F\right\}$ l'ensemble des chiffres du système hexadécimal. On considère les trois parties suivantes :
\[ X=\{A,B,C,D\},\quad\quad Y=\left \{ 0,2,4,6,8,A,C,E\right \}\quad\qeq Z=\left \{ 3,5,7,9,B \right\} \]
Donner en extension les parties suivantes :
\[ \overline{X},\quad \overline{Y},\quad \overline{Z},\quad X\cap Y,\quad Y\cup \overline{X},\quad X\setminus Y,\quad \overline{ \left(\overline{Y}\cap X\right)\cup Z}\setminus Y. \]}
\reponse{\begin{align*}
Sans difficulté :
\[ \overline{X} = \left \{ 0,1,2,3,4,5,6,7,8,9,E,F \right \} \quad \overline{Y} = \left \{ 1,3,5,7,9,B,D,F \right \} \qeq \overline{Z} = \left \{ 0,1,2,4,6,8,A,C,D,E,F \right \} \]
\[ X\cap Y = \left \{ A, C \right \} \quad Y\cup \overline{X} = \left \{ 0, 1, 2, 3, 4, 5, 6, 7, 8 , 9, A, C, E, F \right \} \quad X \setminus Y = \left \{ B, D \right \} \]
et enfin
\[ \overline{ \left(\overline{Y}\cap X\right)\cup Z}\setminus Y  = \left \{ 1, F \right \}\]
\end{align*}}
}

%%% Fin exercice %%%

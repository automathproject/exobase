\uuid{WCeg}
\niveau{PCSI}
\module{Analyse}
\chapitre{Ensembles et applications}
\sousChapitre{Fonctions}
%!TeX root=../../../encours.nouveau.tex
%%% Début exercice %%%

\duree{15}
\difficulte{1}
\auteur{Antoine Crouzet}
\datecreate{01/12/2024}
\titre{Encore des injections, des surjections, des bijections}
\contenu{
\texte{Pour chacun des applications suivantes, justifier si elles sont injectives, surjectives, bijectives. Si elles sont bijectives, déterminer la bijection réciproque.}
\question{$f:k\in \N \donne 3k+1 \in \N*$.}
\reponse{\begin{align*}
L'application $f$ est injective mais pas surjective. En effet, soient $(k,k')\in \N^2$. Alors \[ f(k)=f(k) \implies 3k+1=3k'+1 \implies k=k' \]
	et la fonction est injective. En revanche, $f$ n'est pas surjective car (par exemple) $2$ n'est pas atteint. En efft :
	\[ f(k)=2 \implies k = \frac{1}{3} \notin \N \]
\end{align*}}
\question{$g:\R\setminus \{5\} \to \R \setminus \{2\}$ définie par $g(x)=\dfrac{3+2x}{x-5}$.}
\reponse{$g$ est bijective. On peut démontrer l'injective puis la surjectivité, mais montrons directement la bijectivité. Soit $y\in \R\setminus\{2\}$. On cherche $x\in \R\setminus \{5\}$ tel que $g(x)=y$. Alors :
	\begin{align*}
		g(x) = y &\Leftrightarrow \frac{3+2x}{x-5}= y \\
		&\Leftrightarrow 3+2x = y(x-5) \text{ avec } x\neq 5 \\
		&\Leftrightarrow x(2-y)=-5y-3 \\
		&\Leftrightarrow x= \frac{-5y-3}{2-y} \text{ car } y\neq 2
	\end{align*}
	Ainsi, $g$ est bijective de $\R\setminus \{5\}$ dans $\R \setminus\{2\}$, de bijection réciproque \[ g^{-1} : x\mapsto \frac{-5x-3}{2-x} \]}
\question{$h:\R \to \R+$ définie par $h(y)=\sqrt{y^2+y+1}$.}
\reponse{$h$ n'est ni injective, ni surjective. Constatons tout d'abord que $y\mapsto y^2+y+1$ est strictement positif sur $\R$, donc $h$ est bien définie. De plus, pour tout $y$, $y^2+y+1\neq 0$ donc $h$ ne s'annule jamais : $h$ n'est pas surjective car $0$ n'est pas atteint. Pour l'injectivité, soient $(a,b)\in \R^2$ tels que $h(a)=h(b)$. Alors :
	\begin{align*}
		\sqrt{a^2+a+1} = \sqrt{b^2+b+1} &\implies a^2+a+1 = b^2+b+1 \\
		&\implies a^2-b^2+a-b = 0 \\
		&\implies (a-b)(a+b) + a-b = 0 \\
		&\implies (a-b)(a+b+1) = 0
	\end{align*}
	On a alors deux possibilités : $a=b$ ou $a=-1-b$, ce qui semble indiquer qu'elle n'est pas injective. On prend alors un contre exemple : $b=0$ et $a=-1$. Alors $h(0)=1=h(-1)$ : $h$ n'est pas injective.}
\question{$i:\R\to\R$ définie par $i(t)=\dfrac{\eu{t}-1}{\eu{t}+1}$.}
\reponse{\begin{align*}
Remarquons que $i$ est dérivable sur $\R$ (quotient de fonctions exponentielles dont le dénominateur ne s'annule pas) et on a \[ \forall~t\in \R,\, i'(t) = \frac{\eu{t}(\eu{t}+1)-(\eu{t}-1)\eu{t}}{(\eu{t}+1)^2}= \frac{2\eu{t}}{(\eu{t}+1)^2}> 0\]
	La fonction $i$ est strictement croissante sur $\R$, et donc injective. De plus : \[ \lim_{t\to -\infty} i(t) = -1 \text{ par quotient } \qeq \lim_{t\to +\infty} i(t)=\lim_{t\to+\infty} \frac{\eu{t}(1-\eu{-t})}{\eu{t}(1+\eu{-t})}=1. \]
	Ainsi, par stricte monotonie, on a $\forall~t\in \R,\, -1\leq i(t) \leq 1$ et donc $i$ n'est pas surjective ($2$, par exemple, n'est pas atteint).
\end{align*}}
\question{$j:\R^2\to \R^2$ définie par $j(x,y)=(x+2y, 5y-3x)$.}
\reponse{Soient $(x,y)$ et $(x',y')$ deux couples de $\R^2$ tels que $j(x,y)=j(x',y')$. On a alors $(x+2y,5y-3x)=(x'+2y',5y'-3x')$. On résout le système :
	\begin{align*}
		\systeme[xy]{x+2y=x'+2y',5y-3x=5y'-3x'} &\Leftrightarrow \systeme[xy]{11y=11y',5y-3x=5y'-3x'}\\
		&\Leftrightarrow \systeme*{x=x', y=y'}.
	\end{align*}
	Ainsi, $(x,y)=(x',y')$ : la fonction $j$ est injective. \\
	Pour la surjectivité, soit $(a,b)\in \R^2$. On cherche $(x,y)\in \R^2$ tel que $j(x,y)=(a,b)$. Alors :
	\begin{align*}
		\systeme{x+2y=a,5y-3x=b} &\Leftrightarrow \systeme{11y=3a+b, -3x+5y=b}\\
		&= \systeme*{y=\frac{3}{11}a+\frac{1}{11}b,x=\frac{5}{11}a-\frac{2}{11}b }.
	\end{align*}
	Ainsi, $j$ est surjective, et même bijective, d'application réciproque \[ (a,b) \mapsto \left( \frac{5}{11}a-\frac{2}{11}b, \frac{3}{11}a+\frac{1}{11}b\right). \]}
}

%%% Fin exercice %%%

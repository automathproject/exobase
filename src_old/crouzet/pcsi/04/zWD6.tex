\uuid{zWD6}
\niveau{PCSI}
\module{Analyse}
\chapitre{Systèmes linéaires}
\sousChapitre{Applications}
%!TeX root=../../../encours.nouveau.tex
%%% Début exercice %%%

\duree{20}
\difficulte{1}
\auteur{Antoine Crouzet}
\datecreate{01/12/2024}
\titre{Décomposition en éléments simples}
\contenu{
\question{Montrer qu'il existe des réels $a, b$ et $c$ tels que,  pour tout $x\in \R \setminus \{-2, 0, 1\}$ :
\[ \frac{3x+2}{x(x-1)(x+2)} = \frac{a}{x}+\frac{b}{x-1}+\frac{c}{x+2}. \]
On a effectué la \textbf{décomposition en éléments simples} de $\dfrac{3x+2}{x(x-1)(x+2)}$.
De même, décomposer en éléments simples \[ \frac{2}{x(x+1)(x+2)} \qeq \frac{8}{x^3 +3x^2 -x -3}. \]}
\reponse{On met au même dénominateur, et on identifie les numérateurs qui sont des polynômes. Ainsi :
\begin{align*}
 \frac{a}{x}+\frac{b}{x-1}+\frac{c}{x+2} &= \frac{a(x-1)(x+2)+bx(x+2)+cx(x-1)}{x(x-1)(x+2)}\\
 &= \frac{ (a+b+c)x^2+(a+2b-c)x-2a}{x(x-1)(x+2)}.
\end{align*}
Par identification des numérateurs, on doit avoir
\[ \systeme{a+b+c=0, a+2b-c=3, -2a=2} \iff \systeme*{a=-1, b=\frac53,c=-\frac23}.\]
Ainsi, pour tout réel $x\in \R\setminus \{-2,0,1\}$ :
\[ \frac{3x+2}{x(x-1)(x+2)} = \frac{-1}{x} + \frac{\frac53}{x-1}- \frac{\frac23}{x+2}. \]

Par le même raisonnement, on trouve, pour tout $x\in \R\setminus\{-2, -1, 0\}$ :
\[ \frac{2}{x(x+1)(x+2)} = \frac{1}{x}-\frac{2}{x+1}+\frac{1}{x+2}. \]
Enfin, après factorisation
\[ x^3+3x^2-x-3=(x-1)(x+1)(x+3)\]
et finalement, pour tout $x\in \R \setminus \{ -3, -1, 1\}$ :
\[ \frac{8}{x^3+3x^2-x-3} = \frac{1}{x+3}+\frac{1}{x-1}-\frac{2}{x+1}. \]}
}

%%% Fin exercice %%%
